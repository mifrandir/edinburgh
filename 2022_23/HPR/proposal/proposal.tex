\documentclass{article}
\usepackage{homework-preamble}
\usepackage{csquotes}
\begin{document}

\title{Models of Synthetic Measure Theory}
\author{Franz Miltz}
\maketitle

Anders Kock formalised synthetic measure theory (SMT) using commutative monads. \cite{kock2011commutative}
While it turns out that SMT is not actually a theory for the standard measure-theoretic formailsation of probability theory,
other models have been introduced.

\enquote{[Quasi-Borel spaces] form a new formalization of probability theory replacing measurable spaces; form a cartesian
closed category and so support higher-order functions; form a well-pointed category and so support good proof principles 
for equational reasoning; and support continuous probability distributions.} \cite{DBLP:journals/corr/HeunenKSY17} They have been used to validate 
higher-order Bayesian inference. \cite{DBLP:journals/corr/abs-1711-03219}

I shall investigate quasi-Borel spaces as a model of SMT. I will then explore the extent to which
the category of categories and functors may be formalised as a model of SMT. This may require carefully restricting 
my attention to a well-behaved subcategory to avoid size issue and generalising SMT to pseudomonads as in \cite{Marmolejo_no-iterationpseudomonads}
by re-stating and re-proving the relevant results. I hope to formally relate categorical constructions like products, co-products 
and co-ends to their intuitive counterparts in probability theory. 

A possible extension is further generalising SMT to use relative pseudomonads \cite{Fiore_relative-pseudomonads2017}.

\bibliography{proposal}{}
\bibliographystyle{plain}

\end{document}