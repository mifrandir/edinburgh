\documentclass{article}
\usepackage{homework-preamble}
\usepackage{csquotes}
\begin{document}

\title{Categorical Models of Synthetic Measure Theory}
\author{Franz Miltz}
\maketitle

Kock formalised synthetic measure theory (SMT) using commutative monads. \cite{kock2011commutative}
SMT is not a theory for the standard measure-theoretic formailsation of probability theory since it 
postulates a function space.

My goal for this project is to try and construct non-standard and categorical models for SMT. 
Since category theory has not been covered in any of my courses, a significant part of the project 
includes learning more of it and familiarising myself with its methods.

Quasi-Borel spaces present a new formalisation of probability theory while forming a cartesian 
closed category. \cite{DBLP:journals/corr/HeunenKSY17} 
They have been used to validate higher-order Bayesian inference. \cite{DBLP:journals/corr/abs-1711-03219}
As a basic goal, I shall investigate quasi-Borel spaces as a model of SMT.

To extend this, I will then explore the extent to which the category of categories and functors may
be formalised as a model of SMT. 
This will require carefully restricting my attention to a well-behaved subcategory to avoid size 
issues and generalising SMT to pseudomonads as in \cite{Marmolejo_no-iterationpseudomonads} by 
generalising and re-proving the relevant results. 
I hope to formally relate categorical concepts like coends and the Grothendieck construction to 
their intuitive counterparts in probability theory. 

Beyond that, a possible next step would be to further generalise SMT to use relative pseudomonads,
side-stepping the aforementioned size issues \cite{Fiore_relative-pseudomonads2017}.

\bibliography{proposal}{}
\bibliographystyle{plain}

\end{document}