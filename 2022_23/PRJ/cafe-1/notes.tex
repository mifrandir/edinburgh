\documentclass{article}


\usepackage{notes-preamble}
\usepackage{enumitem}
\usepackage{extarrows}
\usepackage{babel}
\usepackage{tikz-cd}
\tikzset{
   commutative diagrams/.cd,
   arrow style=tikz,
   diagrams={>=latex}}
\usetikzlibrary{babel}


\begin{document}
\mkawodeythms
\title{Cartesian Closed Categories}
\author{Franz Miltz}
\maketitle
\section{Introduction}

In this little excursion we shall have a quick glance towards the beautiful mathematics that 
one may find in the vast world of category theory. The point of all this is neither to be useful 
nor to be particularly fleshed out, but instead to provide some food for thought to those who 
are not content with the mundane lessons taught by university.

\section{Categories}

Unfortunately, even in this little fairy-tale, some house-keeping is required. One may not talk 
about the wonders of category theory while the reader is yet to find out what a category actually is.
So, let's get this over with. 

\begin{definition}[Category]
    A \emph{category} $\cat C$ consists of the following:
    \begin{itemize}
        \item $\cat C_0$, the class of \emph{objects}, and
        \item $\cat C_1$, the class of \emph{arrows} or \emph{morphisms},
    \end{itemize} 
    such that 
    \begin{itemize}
        \item for each arrow $f\in\cat C_1$ there exist distinguished objects $\dom f,\cod f\in\cat C_0$
            called the \emph{domain} and \emph{codomain} of $f$. We may write $f:A\to B$ to indicate
            $\dom f=A$ and $\cod f=B$. Further,
        \item for each pair of arrows $f:A\to B$ and $g:B\to C$, i.e. whenever $\cod f = \dom g$, there 
            is given an arrow $g\circ f: A\to C$ called the \emph{composite} of $f$ and $g$ such that 
            \begin{align*}
                h \circ (g\circ f) = (h\circ g)\circ f
            \end{align*}
            for all $f:A\to B$, $g:B\to C$ and $h:C\to D$. Finally,
        \item for each object $A\in\cat C$ there is given an arrow $1_A:A\to A$, the \emph{identity} of $A$,
            such that 
            \begin{align*}
                f \circ 1_A = f = 1_B\circ f
            \end{align*}
            for all arrows $f:A\to B$.
    \end{itemize}
\end{definition}
\paragraph{Remarks} we make in passing.
\begin{itemize}
    \item Neither objects nor arrows are restricted to sets. This is not a coincidence.
    \item You may ask: `We have $\cat C_0$ and $\cat C_1$, why stop there? What about $\cat C_2$?' I like 
        your thinking and so do the inventors of higher category theory.
    \item You may want to think of categories as directed graphs where the vertices are objects and 
        the edges are arrows.
\end{itemize}

\paragraph{Examples} are everywhere:
\begin{itemize}
    \item sets and functions,
    \item groups and group homomorphisms,
    \item vector spaces and linear mappings,
    \item open subsets $U\subseteq\R$ and continuous functions defined on them.
\end{itemize}
The list goes on but we shall stop here for now. Though an interesting example is coming up.

\section{Functors}

\begin{definition}[Functor]
    A \emph{functor} $F:\cat C\to\cat D$ between categories $\cat C$ and $\cat D$ is a mapping of 
    objects to objects and arrows to arrows such that 
    \begin{enumerate}
        \item $F(f:A\to B)=F(f):F(A)\to F(B)$, 
        \item $F(1_A)=1_{F(A)}$,
        \item $F(g\circ f)=F(g)\circ F(f)$.
    \end{enumerate}
    In other words, a functor preserves domains, codomains, identities and composites.
\end{definition}

One may observe that functors are structure preserving maps that are very similar to homomorphism
of other algebras (e.g. groups, rings, vector spaces, ...). In light of the examples given above, 
this leads us to a very powerful idea: the category of categories and functors.

\newcommand{\Cat}{\mathbf{Cat}}
\begin{definition}
    The category $\Cat$ has as objects small categories and as arrows functors between them.
\end{definition}

What is a small category? One where the classes of objects and arrows are in fact sets. We restrict 
our attention to such small categories for now to avoid any possible complications.

\end{document}