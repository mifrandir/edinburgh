\documentclass{article}


\usepackage{notes-preamble}
\usepackage{enumitem}
\usepackage{extarrows}
\usepackage{babel}
\usepackage{tikz-cd}
\tikzset{
   commutative diagrams/.cd,
   arrow style=tikz,
   diagrams={>=latex}}
\usetikzlibrary{babel}


\begin{document}
\mkawodeythms
\title{Category Theory}
\author{Franz Miltz}
\maketitle
\noindent Textbook: Steve Awodey, \emph{Category Theory}
\tableofcontents
\pagebreak

\section{Categories}

\subsection{Definition}

\begin{definition}[Category; Awodey 1.1]
	A \emph{category} consists of the following data:
	\begin{itemize}
		\item Objects: $A$, $B$, $C$, \dots
		\item Arrows: $f$, $g$, $h$, \dots
		\item For each arrow $f$, there are given objects \begin{align*}
			      \dom(f),\hs \cod(f)
		      \end{align*}
		      called the \emph{domain} and \emph{codomain} of $f$. We write
		      \begin{align*}
			      f:A\to B
		      \end{align*}
		      to indicate that $A=\dom(f)$ and $B=\cod(f)$.
		\item Given arrows $f:A\to B$ and $g:B\to C$, that is, with \begin{align*}
			      \cod(f) = \dom(g)
		      \end{align*}
		      there is given an arrow
		      \begin{align*}
			      g\circ f: A\to C
		      \end{align*}
		      called the \emph{composite} of $f$ and $g$.
		\item For each object $A$ there is given an arrow \begin{align*}
			      1_A : A\to A
		      \end{align*}
		      called the \emph{identity arrow} of $A$.
	\end{itemize}
	These data are required to satisfy the following laws \begin{enumerate}[label=C\arabic*.]
		\item \emph{Associativity}: \begin{align*}
			      f \circ (g \circ h) = (f\circ g) \circ h
		      \end{align*}
		      for all arrows $f:A\to B$, $g:B\to C$, $h:C\to D$.
		\item \emph{Unit}: \begin{align*}
			      f \circ 1_A = f = 1_B \circ f
		      \end{align*}
		      for all $f:A\to B$.
	\end{enumerate}
\end{definition}

\subsection{Examples}

\begin{definition}[Posets and monotone functions]
	A partially ordered set or \emph{poset} is a set $A$ equipped
	with a binary relation $a\leq_A b$ such that the following conditions
	hold for all $a,b,c\in A$:
	\begin{enumerate}[label=P\arabic*.]
		\item \emph{reflexivity}: $a\leq_A a$.
		\item \emph{transitivity}: If $a\leq_A b$ and $b\leq_A c$, then $a\leq_A c$.
		\item \emph{antisymmetry}: If $a\leq_A b$ and $b\leq_A a$, then $a=b$.
	\end{enumerate}
	A function $f:A\to B$ between two posets is \emph{monotone} if
	\begin{align*}
		a \leq_A a' \text{ implies } f(a) \leq_A f(a')\hs \text{for all }a,a'\in A.
	\end{align*}
\end{definition}

\begin{definition}[Functors; Awodey 1.2]
	A \emph{functor}
	\begin{align*}
		F:\cat{C}\to\cat{D}
	\end{align*}
	between categories $\cat{C}$ and $\cat{D}$ is a mapping
	of objects to objects and arrows to arrows, in such a way that
	\begin{enumerate}[label=F\arabic*.]
		\item $F(f:A\to B)=F(f):F(A)\to F(B)$
		\item $F(1_A)=1_{F(A)}$
		\item $F(f\circ g)=F(f)\circ F(g)$
	\end{enumerate}
\end{definition}

\begin{definition}[Monoids and homomorphisms]
	A \emph{monoid} is a set $M$ equipped with a binary operation
	$\cdot : M\times M \to M$ and a unit element $u\in M$ such that
	for all $x,y,z\in M$
	\begin{align*}
		x\cdot (y\cdot z) = (x\cdot y) \cdot z
	\end{align*}
	and
	\begin{align*}
		u \cdot x = x = x \cdot u.
	\end{align*}
	For monoids $M,N$ a \emph{monoid homomorphism} is a function
	$h:M\to N$ such that for all $a,b\in M$,
	\begin{align*}
		h(a\cdot_M b) = h(a)\cdot_N h(b)
	\end{align*}
	and
	\begin{align*}
		h(u_M) = u_N.
	\end{align*}
\end{definition}

\paragraph{Examples of categories}
\begin{itemize}
	\item \textbf{Sets}: of sets and functions
	\item $\textbf{Sets}_\text{fin}$: of finite sets and functions
	\item \textbf{Pos}: of posets and monotone functions
	\item \textbf{Rel}: of sets and binary relations
	\item \textbf{Cat}: of categories and functors
	\item \textbf{Dis$(X)$}: of elements in $X$ and all identity arrows
	\item \textbf{Mon}: of monoids and monoid homomorphisms
\end{itemize}

\begin{definition}
	For a category $\cat C$ the collection of morphisms between two objects
	$X$ and $Y$ is denoted as
	\begin{align*}
		\Hom_{\cat C}(X,Y).
	\end{align*}
\end{definition}

\subsection{Isomorphisms}

\begin{definition}[Isomorphism; Awodey 1.3]
	In any category $\cat{C}$, an arrow $f:A\to B$ is called an
	\emph{isomorphism} if there exists an arrow $g:B\to A$ in
	$\cat{C}$ such that
	\begin{align*}
		g\circ f = 1_A\hs\text{and}\hs f\circ g = 1_B.
	\end{align*}
	We say $A$ is isomorphic to $B$ if there exists an isomorphism
	between them and write $A\cong B$.
\end{definition}

\begin{definition}
	A \emph{group} $G$ is a monoid with an inverse $\inv g$ for every element $g$.
	Thus, $G$ is a category with one object, in which every arrow is an isomorphism.
\end{definition}

\begin{theorem}[Cayley]
	Every group $G$ is isomorphic to a group of permutations.
\end{theorem}



\subsection{Construction of categories}

\begin{definition}
	The product of two categories $\cat{C}$ and $\cat{D}$, written
	as
	\begin{align*}
		\cat{C}\times\cat{D}
	\end{align*}
	has objects of the form $(C,D)$ for $C\in\cat{C}$ and $D\in\cat{D}$,
	and arrows of the form
	\begin{align*}
		(f,g):(C,D)\to (C',D')
	\end{align*}
	for $f:C\to C'\in\cat{C}$ and $g:D\to D'\in\cat{D}$.
\end{definition}

\begin{definition}
	The \emph{opposite} category $\catop C$ of a category $\cat C$ has
	the same objects as $\cat C$ and an arrow $f:C\to D$ in $\catop C$
	is an arrow $f:D\to C$ in $\cat C$. We write
	\begin{align*}
		f^* : D^* \to C^*
	\end{align*}
	in $\catop C$ for $f:C\to D$ in $\cat C$. We further define
	\begin{align*}
		(1_{C^*})    & = (1_C)^*,      \\
		f^*\circ g^* & = (g\circ f)^*.
	\end{align*}
	There are two obvious projection functors
	\begin{center}
		\begin{tikzcd}
			\cat C &&
			\cat C \times \cat D \arrow[ll, swap, "\pi_1"] \arrow[rr, "\pi_2"] &&
			\cat D
		\end{tikzcd}
	\end{center}
\end{definition}

\begin{definition}
	The \emph{arrow category} $\catar C$ of a category $\cat C$
	has the arrows of $\cat C$ as objects and an arrow $g$ from
	$f:A\to B$ to $f':A'\to B'$ in $\catar C$ is a pair of arrows
	$g=(g_1, g_2)$ in $\cat C$ such that
	\begin{align*}
		g_2\circ f = f'\circ g_1.
	\end{align*}
	Observe that there are two functors
	\begin{center}
		\begin{tikzcd}
			\cat C &&
			\catar C \arrow[ll, swap, "\textbf{dom}"] \arrow[rr, "\textbf{cod}"] &&
			\cat C
		\end{tikzcd}
	\end{center}
\end{definition}

\begin{definition}
	The \emph{slice category} $\cat C/C$ of a category $\cat C$ over an
	object $C\in\cat C$ has as objects all arrows $f\in\cat C$ such that
	$\dom(f)=C$ and an arrow from $f:X\to C$ to $f':X'\to C$ is an arrow
	$a:X\to X'$ in $\cat C$ such that $f'\circ a = f$.\\
	Note that there is a functor $U:\cat C/C \to \cat C$ defined by
	\begin{align*}
		U(A:X\to C)                   & = X
		                              &             & \text{for all objects }A\in\cat C/C, \\
		U(f:(A:X\to C)\to(B:X'\to C)) & = a:X\to X'
		                              &             & \text{for all arrows }f\in\cat C/C.
	\end{align*}
	Further, if $g:C\to D\in\cat C$ is any arrow, then there is a composition
	functor,
	\begin{align*}
		g_*:\cat C/C\to\cat C/D
	\end{align*}
	defined by $g_*(f)= g\circ f$.
\end{definition}

\subsection{Free categories}

\begin{definition}[UMP of $M(A)$]
	Every monoid $N$ has an underlying set $\abs N$, and every monoid
	homomorphism $f:N\to M$ has an underlying function $\abs f: \abs N
		\to \abs M$.
	The free monoid $M(A)$ on a set $A$ is defined to be
	the monoid with the \emph{universal mapping property}.
	I.e. there is a function $i:A\to \abs{M(A)}$, and given any
	monoid $N$ and any function $f:A\to\abs N$, there is a
	\emph{unique} monoid homomorphism $\bar f: M(A)\to N$
	such that $\abs f \circ i = f$.
\end{definition}

\begin{proposition}[Awodey 1.9]
	Given a set $A$, the \emph{Kleene colosure} $A^*$ containing
	all the finite sequences of elements of $A$ has the UMP of the
	free monoid on $A$.
\end{proposition}

\begin{proposition}
	Given monoids $M$ and $N$ with functions $i:A\to\abs M$ and
	$j:A\to\abs N$, each with the UMP of the free monoid on $A$,
	there is a (unique) monoid isomorphism $h:M\cong N$ such that
	$\abs h i=j$ and $\abs{\inv h}j = i$.
\end{proposition}

\begin{definition}
	Let $G$ be a graph. Then the free category $\cat C(G)$ has the vertices
	of $G$ as objects and the paths in $G$ as arrows, where a path is a finite
	sequence of edges $e_1,...,e_n$ where $t(e_i)=s(e_{i+1})$.
\end{definition}

\begin{definition}[UMP of $\cat C(G)$]
	The free category on a graph has the UMP. I.e. there is a graph
	homomorphism $i:G\to\abs{\cat C(G)}$, and given any category $\cat D$
	and any graph homomorphism $h:G\to\abs D$, there is a unique functor
	$\bar h:\cat C(G)\to \cat D$ with $\abs{\bar h}\circ i = h$.
\end{definition}

\subsection{Foundations}

\begin{definition}[Awodey 1.11]
	A category $\cat C$ is called \emph{small} if both the collection $C_0$
	of objects and the collection $C_1$ of arrows are sets. Otherwise $\cat C$
	is \emph{large}.
\end{definition}

\begin{theorem}[Awodey 1.6]
	Every small category $\cat{C}$ is isomorphic
	to one in which the objects are sets and the arrows are functions.
\end{theorem}

\begin{definition}[Awodey 1.12]
	A category $\cat C$ is called \emph{locally small} if for all objects
	$X,Y\in\cat C$, the collection $\Hom_{\cat C}(X,Y)$ is a set.
\end{definition}

\section{Abstract structures}


\subsection{Epis and monos}

\begin{definition}[Awodey 2.1]
	In any category $\cat C$, an arrow $f:A\to B$ is called a
	\begin{itemize}
		\item \emph{monomorphism} and we write $f:A\mono B$, if given any $g,h:C\to A$, $fg=fh$ implies $g=h$,
		\item \emph{epimorphism} and we write $f:A\epi B$, if given any $i,j:B\to D$, $if=jf$ implies $i=j$.
	\end{itemize}
\end{definition}

\begin{proposition}[Awodey 2.2]
	A function $f:A\to B$ between sets is monic (epic) iff it is injective (surjective).
\end{proposition}

\begin{proposition}[Awodey 2.6]
	Every isomorphism is both monic and epic.
\end{proposition}

\begin{definition}[Awodey 2.7]
	A \emph{split} mono (epi) is an arrow with a left (right) inverse. Given
	arrows $e:X\to A$ and $s:A\to X$ such that $es=1_A$, the arrow $s$ is called
	a \emph{section} or \emph{splitting} of $e$, and the arrow $e$ is called a
	\emph{retraction} of $s$. The object $A$ is called a \emph{retract} of $X$.
\end{definition}

\begin{definition}
	An object $P$ is said to be \emph{projective} if for any epi $e:E\epi X$
	and arrow $f:P\to X$ there is some arrow $\bar f:P\to E$ such that
	$e\circ \bar f = f$.
\end{definition}

\subsection{Initial and terminal objects}

\begin{definition}[Awodey 2.9]
	In any category $\cat C$, an object
	\begin{itemize}
		\item $0$ is \emph{initial} if for any object $C$ there is a unique morphism
		      \[0\to C,\]
		\item $1$ is \emph{terminal} if for any object $C$ there is a unique morphism
		      \[C\to 1.\]
	\end{itemize}
\end{definition}

\begin{proposition}[Awodey 2.10]
	Initial (terminal) objects are unique up to isomorphism.
\end{proposition}

\subsection{Generalised elements}

\begin{definition}
	For any object $A\in\cat C$ an arrow $x\in\cat C$ of the form
	\begin{align*}
		x: 1 \to A
	\end{align*}
	where $1\in\cat C$ is terminal, is called a \emph{global element}
	of $A$.\\
	Further, for any object $A\in\cat C$ an arrow $x\in\cat C$ of the
	form
	\begin{align*}
		x: X\to A
	\end{align*}
	with arbitrary domain $X\in\cat C$ is called a \emph{generalised element}.
\end{definition}

\begin{lemma}
	Let $x,x': X\to A$ be generalised elements of $A\in\cat C$. Then an arrow
	$f:A\to B$ is monic iff $x\not=x'$ implies $fx\not=f'x$.
\end{lemma}

\begin{lemma}[Commuting square test]
	Let $f,g,\alpha,\beta$ be arrows in any category $\cat C$ such that
	\begin{align*}
		f:A\to B,\hs g:A\to D,\hs \alpha:B\to C,\hs \beta:D\to C.
	\end{align*}
	Then we have $\alpha f=\beta g$ just if $\alpha f x = \beta g x$ for all generalised
	elements $x:X\to A$.
\end{lemma}

\subsection{Products}

\begin{definition}[Product; Awodey 2.15]
	In any category $\cat C$, a \emph{product diagram} for the objects $A$
	and $B$ consists of an object $P$ and arrows $p_1:P\to A$ and $p_2:P\to B$
	satisfying the following UMP: Given any arrows $x_1:X\to A$ and $x_2:X\to B$
	where $X$ is an object of $\cat C$, there exists a unique arrow $u:X\to P$
	such that $x_1=p_1u$ and $x_2=p_2u$.
	\begin{center}
		\begin{tikzcd}
			& X \ar{dl}[swap]{x_1} \ar[d, dotted, "u"] \ar{dr}{x_2} \\
			A & P \ar{l}{p_1} \ar{r}[swap]{p_2} & B
		\end{tikzcd}
	\end{center}
	We write $P=A\times B$.
\end{definition}

\begin{theorem}[Awodey 2.17]
	Products are unique up to isomorphism.
\end{theorem}

\begin{definition}[Awodey 2.19]
	A categroy $\cat C$ is said to \emph{have all finite products}, if it has a
	terminal object and all binary products (and therewith all products of any
	finite cardinality). The category $\cat C$ \emph{has all (small) products}
	if every set of objects in $\cat C$ has a product.
\end{definition}

\subsection{Hom-sets}

\begin{theorem}
	Let $g:B\to B'$ be an arrow in $\cat C$. Then the function
	defined by
	\begin{align*}
		\Hom(A, g):\Hom(A,B)\to\Hom(A,B'), \\
		(f:A\to B) \mapsto (g\circ f:A\to B\to B'),
	\end{align*}
	i.e. $\Hom(A,g)(f)=g\circ f$, determines a functor
	\begin{align*}
		\Hom(A,-):\cat C \to \cat{Sets}
	\end{align*}
	called the \emph{representable functor}.
\end{theorem}

\begin{proposition}[Awodey 2.20]
	A diagram of the form
	\begin{center}
		\begin{tikzcd}
			A &&
			P \arrow[ll, swap, "p_1"] \arrow[rr, "p_2"] &&
			B
		\end{tikzcd}
	\end{center}
	is a product for A and B iff for every object $X$, the canonical
	function
	\begin{gather*}
		\vartheta_X = (\Hom(X, p_1), \Hom(X, p_2)) : \Hom(X, P)\to \Hom(X,A)\times \Hom(X,B)\\
		x \mapsto (x_1, x_2)
	\end{gather*}
	is an isomorphism,
	\begin{align*}
		\vartheta_X:\Hom(X,P)\cong \Hom(X,A)\times\Hom(X,B).
	\end{align*}
\end{proposition}

\begin{definition}[Awodey 2.21]
	Let $\cat C$, $\cat D$ be categories with binary products. A functor
	$F:\cat C\to \cat D$ is said to \emph{preserve binary products} if it takes
	every product diagram
	\begin{center}
		\begin{tikzcd}
			A &&
			A\times B \arrow[ll, swap, "p_1"] \arrow[rr, "p_2"] &&
			B
		\end{tikzcd}
	\end{center}
	to a product diagram
	\begin{center}
		\begin{tikzcd}
			FA &&
			F(A\times B) \arrow[ll, swap, "Fp_1"] \arrow[rr, "Fp_2"] &&
			FB
		\end{tikzcd}
	\end{center}
\end{definition}

\begin{corollary}[Awodey 2.22]
	For any object $X$ in a category $\cat C$ with products, the representable
	functor
	\begin{align*}
		\Hom_{\cat C}(X, -) : \cat C \to \cat{Sets}
	\end{align*}
	preserves products.
\end{corollary}

\section{Duality}

\subsection{The duality principle}

\begin{proposition}[Formal duality; Awodey 3.1]
	For any sentence $\Sigma$ in the language of category theory,
	if $\Sigma$ follows from the axioms for categories, then so does
	its dual $\Sigma^*$
	\begin{align*}
		\text{CT}\Rightarrow \Sigma \text{\hs implies\hs} \text{CT}\Rightarrow \Sigma^*.
	\end{align*}
\end{proposition}

\begin{proposition}[Conceptual duality; Awodey 3.2]
	For any statement $\Sigma$ about categories, if $\Sigma$ holds for all categories,
	then so does the dual statement $\Sigma^*$.
\end{proposition}

\subsection{Coproducts}

\begin{definition}[Coproduct; Awodey 3.3]
	In any category $\cat C$, a \emph{coproduct diagram} for the objects $A$ and $B$
	consists of an object $A$ and arrows $q_1:A\to Q$ and $q_2:B\to Q$ satisfying
	the following UMP: Given any arrows $x_1:A\to X$ and $x_2:B\to X$ where $X$ is an
	object of $\cat C$, there exists a unique arrow $u:X\to P$ such that $x_1=up_1$
	and $x_2=up_2$.
	\begin{center}
		\begin{tikzcd}
			& X & \\
			A \ar{ur}{x_1} \ar{r}[swap]{q_1} & Q \ar[u, dotted, "u"] & B \ar[l, "q_2"] \ar{ul}[swap]{x_2}
		\end{tikzcd}
	\end{center}
	We write $Q=A+B$.
\end{definition}

\begin{proposition}[Awodey 3.11]
	In the category $\cat{Ab}$ of abelian groups, there is a canonical isomorphism between the binary
	coproduct and product
	\begin{align*}
		A + B \cong A \times B.
	\end{align*}
\end{proposition}

\begin{proposition}[Awodey 3.12]
	Coproducts are unique up to isomorphism.
\end{proposition}

\subsection{Equalisers}

\begin{definition}[Equaliser; Awodey 3.13]
	In any category $\cat C$, given parallel arrows
	\begin{align*}
		f,g: A\to B
	\end{align*}
	an \emph{equaliser} of $f$ and $g$ consists of an object $E$ and an arrow $e:E\to A$,
	universal such that
	\begin{align*}
		f \circ e = g \circ e.
	\end{align*}
	That is, given any $z:Z\to A$ with $f\circ z=g\circ z$, there is a \emph{unique}
	$u:Z\to E$ with $e\circ u = z$.
\end{definition}

\begin{proposition}[Awodey 3.16]
	In any category, if $e:E\to A$ is an equaliser of some pair of arrows, $e$ is monic.
\end{proposition}

\subsection{Coequalisers}

\begin{definition}[Coequaliser; Awodey 3.18]
	For any parallel arrows $f,g:A\to B$ in any category $\cat C$, a \emph{coequaliser}
	consists of $Q$ and $q:B\to Q$, universal with the property $qf=qg$. That is given
	any $Z$ and $z:B\to Z$, if $zf=zg$, then there exists a unique $u:Q\to Z$ such that
	$uq=z$.
\end{definition}

\begin{proposition}[Awodey 3.19]
	If $q:B\to Q$ is a coequaliser of some pair of arrows, then $q$ is epic.
\end{proposition}

\begin{proposition}[Awodey 3.24]
	For every monoid $M$ there are sets $R$ and $G$ and a coequaliser diagram,
	\begin{center}
		\begin{tikzcd}
			F(R) \arrow[r, shift left=.3ex, "r_1"]
			\arrow[r, shift right=.3ex, swap, "r_2"] & F(G)
			\arrow[r] & M
		\end{tikzcd}
	\end{center}
	with $F(R)$ and $F(G)$ free; thus, $M\cong F(G)/(r_1=r_2)$.
\end{proposition}

\section{Groups and categories}

\subsection{Groups in a category}

\begin{definition}[Group; Awodey p. 75]
	A \emph{group} in $\cat C$ consists of objects and arrows as so:

	\begin{center}
		\begin{tikzcd}
			G\times G\arrow[r, "m"] & G & G \arrow[l, swap, "i"]\\
			& 1 \arrow[u, "u"]
		\end{tikzcd}
	\end{center}

	satisfying the following conditions, for all generalised elements
	\begin{align*}
		x,y,z:Z\to G,
	\end{align*}

	\begin{enumerate}
		\item $m$ is associative, that is, the following commutes: \begin{center}
			      \begin{tikzcd}
				      {(G\times G)\times G} \arrow[d, swap, "m\times 1"] \arrow[rr, "\cong"] & & {G\times(G\times G)} \arrow[d, "1\times m"]\\
				      {G\times G}\arrow[dr, swap, "m"] & & G\times G\arrow[dl, "m"]\\
				      & G &
			      \end{tikzcd}
		      \end{center}
		      where $\cong$ is the canonical associativity isomorphism for products. I.e. \begin{align*}
			      m\circ \lra{m\circ \lra{x,y}, z} = m\circ\lra{x,m\circ\lra{y,z}}.
		      \end{align*}
		\item $u$ is a unit for $m$, that is, both triangles in the following commute: \begin{center}
			      \begin{tikzcd}
				      G \arrow[d, swap, "\lra{1_G, u}"] \arrow[dr, "1_G"] \arrow[r, "\lra{u, 1_G}"]
				      & G\times G\arrow[d, "m"]\\
				      G\times G \arrow[r, swap, "m"]
				      & G
			      \end{tikzcd}
		      \end{center}
		      I.e. \begin{align*}
			      m\circ\lra{x,u} = x = m\circ\lra{u, x}.
		      \end{align*}
		\item $i$ is an inverse with respect to $m$, that is, both sides of the following commute: \begin{center}
			      \begin{tikzcd}
				      G\times G \arrow[d, swap, "1_G\times i"] &
				      G \arrow[l, swap, "\Delta"] \arrow[d, "u"] \arrow[r, "\Delta"] &
				      G\times G \arrow[d, "i\times 1_G"] \\
				      G\times G \arrow[r, swap, "m"] &
				      G &
				      G\times G \arrow[l, "m"]
			      \end{tikzcd}
		      \end{center}
		      I.e. \begin{align*}
			      m\circ\lra{x,i\circ x} = u = m\circ\lra{i\circ x, x}
		      \end{align*}
	\end{enumerate}
\end{definition}

\begin{definition}[Group homomorphism; Awodey p. 77]
	A \emph{homorphism $h:G\to H$} of groups in $\cat C$ consists of an arrow in $\cat C$
	such that
	\begin{enumerate}
		\item $h$ preserves $m$: \begin{center}
			      \begin{tikzcd}
				      G\times G \arrow[d, swap, "m"] \arrow[r, "h\times h"] &
				      H\times H \arrow[d, "m"] \\
				      G \arrow[r, swap, "h"] &
				      H
			      \end{tikzcd}
		      \end{center}
		\item $h$ preserves $u$: \begin{center}
			      \begin{tikzcd}
				      G \arrow[r, swap, "h"] & H \\
				      1 \arrow[u, "u"] \arrow[ur, swap, "u"]
			      \end{tikzcd}
		      \end{center}
		\item $h$ perserves $i$: \begin{center}
			      \begin{tikzcd}
				      G \arrow[r, "h"] \arrow[d, swap, "i"] &
				      H \arrow[d, "i"] \\
				      G \arrow[r, swap, "h"] & H
			      \end{tikzcd}
		      \end{center}
	\end{enumerate}
\end{definition}

\begin{proposition}[Awodey p. 78]
	Given any set $G$ equipped with two binary operations $\circ,\star$:
	$G\times G\to G$ with units $1^\circ$ and $1^\star$, respectively and
	satisfying
	\begin{align*}
		(g_1 \circ h_1)\star (g_2\circ h_2) = (g_1\star g_2) \circ (h_1\star h_2)
	\end{align*}
	for any $g_1, g_2, h_1, h_2\in G$, the following hold.
	\begin{enumerate}
		\item $1^\circ = 1^\star$.
		\item $\circ = \star$.
		\item The operation $\circ = \star$ is commutative.
	\end{enumerate}
\end{proposition}

\begin{corollary}[Awodey p. 79]
	The groups in the category of groups are exactly the abelian groups.
\end{corollary}

\begin{definition}[Awodey p. 79]
	A \emph{strict monoidal category} is a category $\cat C$ equipped with a
	binary operation $\otimes:\cat C\times \cat C\to \cat C$ which is functorial
	and associative,
	\begin{align*}
		A\otimes(B\otimes C) = (A\otimes B)\otimes C,
	\end{align*}
	together with a distinguished object $I$ that acts as a unit,
	\begin{align*}
		I\otimes C = C = C \otimes I.
	\end{align*}
\end{definition}

\subsection{The category of groups}

\begin{theorem}[Awodey p. 83]
	Every group homomorphism $h:G\to H$ has a kernel $\ker(h)=\inv h(u)$
	which is a normal subgroup of $G$ with the property that, for any normal
	subgroup $N\subseteq G$
	\begin{align*}
		N\subseteq \ker(h)
	\end{align*}
	iff there is a (necessarily unique) homomorphism
	$\bar h:G/N\to H$ with $\bar h \circ \pi = h$ where
	$\pi:G\to G/N$ takes every element of $G$ to its
	coset: $g\mapsto [g]$.
\end{theorem}

\subsection{Groups as categories}

\begin{definition}[Awodey p. 83]
	A \emph{congruence} on a category $\cat C$ is an equivalence relation
	$f\sim g$ on arrows such that
	\begin{enumerate}
		\item $f\sim g$ implies $\dom(f) = \dom(g)$ and $\cod(f) = \cod(g)$, \begin{center}
			      \begin{tikzcd}
				      \bullet \arrow[r, shift left=.3ex, "f"] \arrow[r, swap, shift right=.3ex, "g"] & \bullet
			      \end{tikzcd}
		      \end{center}
		\item $f\sim g$ implies $bfa\sim bga$ for all arrows $a:A\to X$ and $b:Y\to B$, where
		      $\dom(f) = X = \dom(g)$ and $\cod(f) = Y = \cod(g)$, \begin{center}
			      \begin{tikzcd}
				      \bullet \arrow[r, "a"] &
				      \bullet \arrow[r, shift left=.3ex, "f"] \arrow[r, swap, shift right=.3ex, "g"] &
				      \bullet \arrow[r, "b"] &
				      \bullet
			      \end{tikzcd}
		      \end{center}
	\end{enumerate}
\end{definition}

\begin{corollary}
	Every functor $F:\cat C \to \cat D$ factors as $F=\tilde F \circ \pi$, \begin{center}
		\begin{tikzcd}
			\cat C \arrow[d, swap, "\pi"] \arrow[r, "F"] &
			\cat D \\
			\cat C / \ker(F) \arrow[ur, swap, "\tilde F"]
		\end{tikzcd}
	\end{center}
	where $\pi$ is bijective on objects and surjective on Hom-sets, and $\tilde F$ is
	injective on Hom-sets (i.e., "faithful"):
	\begin{align*}
		\tilde F_{A,B}:\Hom(A,B)\mono \Hom(FA, FB)\hs\text{for all }A,B\in \cat C/\ker(F).
	\end{align*}
\end{corollary}

\section{Limits and colimits}

\subsection{Subobjects}

\begin{definition}[Awodey p. 89]
	A \emph{subobject} of an object $X$ in a category $\cat C$ is
	a monomorphism $m:M\mono X$.
\end{definition}

\begin{corollary}
	An equaliser of parallel arrows $f,g:A\to B$ is a subobject
	of $A$ with the property
	\begin{align*}
		z\in_A E \hs\text{iff}\hs f(z) = g(z).
	\end{align*}
\end{corollary}

\subsection{Pullbacks}

\begin{definition}[Awodey p. 92]
	If any category $\cat C$, given arrows $f,g$ with $\cod(f)=\cod(g)$,
	the \emph{pullback} of $f$ and $g$ consists of arrows such that $fp_1=gp_2$
	and universal with this property. That is, given any $z_1:Z\to A$ and
	$z_2:Z\to B$ with $fz_1=gz_2$, there exists a unique $u:Z\to P$ with
	$z_1=p_1u$ and $z_2=p_2u$.
	\begin{center}
		\begin{tikzcd}
			Z \arrow[ddr, swap, "z_1"] \arrow[dr, dotted, "u" description] \arrow[rrd, "z_2"]\\
			& P \arrow[d, "p_1"] \arrow[r, swap, "p_2"]
			& B \arrow[d, "g"]\\
			& A \arrow[r, swap, "f"] & C
		\end{tikzcd}
	\end{center}
\end{definition}

\begin{proposition}[Awodey p. 93]
	In a category with products and equalisers, given a corner of arrows
	\begin{center}
		\begin{tikzcd}
			& B \arrow[d, "g"]\\
			A \arrow[r, swap, "f"] & C
		\end{tikzcd}
	\end{center}
	Consider the diagram
	\begin{center}
		\begin{tikzcd}
			E \arrow[ddr, swap, "p_1"] \arrow[dr, "e" description] \arrow[rrd, "p_2"]\\
			& A\times B \arrow[d, "\pi_1"] \arrow[r, swap, "\pi_2"]
			& B \arrow[d, "g"]\\
			& A \arrow[r, swap, "f"] & C
		\end{tikzcd}
	\end{center}
	in which $e$ is an equaliser of $f\pi_1$ and $g\pi_2$ and $p_1=\pi_1 e$, $p2=\pi_2 e$. Then,
	$E$, $p_1$, $p_2$ is a pullback of $f$ and $g$. Conversely, if $E$, $p_1$, $p_2$ are given as
	such a pullback, then the arrow
	\begin{align*}
		e=\lra{p_1,p_2}:E\to A\times B
	\end{align*}
	is an equaliser of $f\pi_1$ and $g\pi_2$.
\end{proposition}

\begin{corollary}[Awodey p. 94]
	If a category $\cat C$ has binary products and equalisers, then it has pullbacks.
\end{corollary}

\begin{lemma}[Awodey p. 95]
	Consider the commutative diagram below in a category with pullbacks:
	\begin{center}
		\begin{tikzcd}
			F \arrow[d, swap, "h''"] \arrow[r, "f'"] &
			E \arrow[d, "h'"] \arrow[r, "g'"] &
			D \arrow[d, "h"] \\
			A \arrow[r, swap, "f"] &
			B \arrow[r, swap, "g"] &
			C
		\end{tikzcd}
	\end{center}
	\begin{enumerate}
		\item If the two squares are pullbacks, so is the outer rectangle. Thus, \begin{align*}
			      A\times_B (B\times_C D) \cong A\times_C D.
		      \end{align*}
		\item If the right square and the outer rectangle are pullbacks, so is the left square.
	\end{enumerate}
\end{lemma}

\begin{corollary}[Awodey p. 96]
	The pullback of a commutative triangle is a commutative triangle.

	\begin{center}
		\begin{tikzcd}
			A'\arrow[dd, "\alpha'"] \arrow[dr, dotted, "\gamma'"] \arrow[rrr, "h_\alpha"] &&&
			A \arrow[dd, "\alpha"] \arrow[dr, "\gamma"] \\ &
			B' \arrow[rrr, "h_\beta"] \arrow[dl, "\beta'"] &&&
			B \arrow[dl, "\beta"]\\
			C' \arrow[rrr, "h"] &&&
			C
		\end{tikzcd}
	\end{center}
\end{corollary}

\begin{proposition}
	Pullback is a functor. That is, for fixed $h:C'\to C$ in a category $\cat C$ with pullbacks,
	there is a functor
	\begin{align*}
		h^*:\cat C/C \to \cat C/C'
	\end{align*}
	defined by
	\begin{align*}
		(A\to_\alpha C)\mapsto (C'\times_C A\to_{\alpha'}C')
	\end{align*}
	where $\alpha'$ is the pullback of $\alpha$ along $h$.
\end{proposition}

\begin{corollary}[Awodey p. 97]
	Let $\cat C$ be a category with pullbacks. For any arrow $f:A\to B$
	in $\cat C$, we have the following diagram of categories and functors:
	\begin{center}
		\begin{tikzcd}
			\Sub(A) \arrow[d] &
			\Sub(B) \arrow[l, swap, "\inv f"] \arrow[d] \\
			\cat C / A &
			\cat C / B \arrow[l, "f^*"]
		\end{tikzcd}
	\end{center}
	where for any $m$ and its pullback $m'$ along $f$ we define
	the map
	\begin{align*}
		\inv f: \Sub(B) & \to\Sub(A), \\
		m               & \mapsto m'.
	\end{align*}
\end{corollary}

\subsection{Limits}

\begin{definition}[Cones; Awodey p. 101]
	Let $\cat J$ and $\cat C$ be categories. A \emph{diagram of type} $\cat J$
	in $\cat C$ is a functor.
	\begin{align*}
		D:\cat J \to \cat C.
	\end{align*}
	For all objects $j\in\cat J$ we write $D_j\coloneqq D(j)$ and for all arrows
	$\alpha:i\to j\in\cat J$ we write
	\begin{align*}
		D_\alpha = D(\alpha):D_i\to D_j.
	\end{align*}
	A \emph{cone} to a diagram $D$ consists of an object $C$ in $\cat C$ and
	a family of arrows in $\cat C$,
	\begin{align*}
		\left(c_j:C\to D_j\right)_{j\in\cat J}
	\end{align*}
	such that for each arrow $\alpha:i\to j$ in $\cat J$,
	the following triangle commutes
	\begin{center}
		\begin{tikzcd}
			C \arrow[d, swap, "c_i"] \arrow[r, "c_j"] &
			D_j \\
			D_i \arrow[ur, swap, "D_\alpha"]
		\end{tikzcd}
	\end{center}
	A \emph{morphism of cones} with objects $C,C'$ and arrows
	$\left(c_j\right)_{j\in\cat J}$, $\left(c'_j\right)_{j\in\cat J}$
	is an arrow $\vartheta\in\cat C$ making each triangle
	\begin{center}
		\begin{tikzcd}
			C  \arrow[dr, swap, "c_j"] \arrow[r, "\vartheta"] &
			C' \arrow[d, "c'_j"] \\
			& D_j
		\end{tikzcd}
	\end{center}
	commute. That is, such that $c_j = c'_j \circ \vartheta$ for all $j\in\cat J$.
	We have a category $\cat{Cone}(D)$ of cones to $D$.
\end{definition}

\begin{definition}[Awodey p. 102]
	A \emph{limit} of a diagram $D:\cat J \to \cat C$ is a terminal object in the
	category $\cat{Cone}(D)$. A \emph{finite limit} is a limit of a diagram on
	a finite index category $\cat J$.
\end{definition}

\begin{proposition}[Awodey p. 105]
	Let $\cat C$ be a category. Then the following are equivalent:
	\begin{enumerate}
		\item $\cat C$ has finite products and equalisers of $\kappa$ many objects.
		\item $\cat C$ has pullbacks and a terminal object of cardinality $\kappa$.
		\item $\cat C$ has limits for all diagrams $D:\cat J\to\cat C$ where $\text{card}(\cat J_1)\leq \kappa$.
	\end{enumerate}
	Dually for coproducts, coequalisers, pushouts, inital objects and colimits.
\end{proposition}

\begin{definition}[Awodey p. 106]
	A functor $F:\cat C\to\cat D$ is said to \emph{preserve limits of type $\cat J$} if,
	whenever $p_j:L\to D_j$ is a limit for a diagram $D:\cat J\to\cat C$; the cone
	$Fp_j:FL\to FD_j$ is then a limit for the diagram $FD:\cat J\to\cat D$. Briefly,
	\begin{align*}
		F(\catlim D_j)\cong \catlim F(D_j).
	\end{align*}
	A functor that preserves all limits is said to be \emph{continuous}.
\end{definition}

\begin{proposition}
	The representable functors $\Hom(C,-)$ preserve all limits.
\end{proposition}

\begin{definition}[Awodey p. 107]
	A functor of the form $F:\catop C\to \cat D$ is called a \emph{contravariant
		functor} on $\cat C$. Explicitly, such a functor takes $f:A\to B$ to $F(f):
		F(B)\to F(A)$ and $F(g\circ f)=F(f)\circ F(g)$.
\end{definition}

\begin{corollary}
	Contravariant representable functors map all colimits to limits.
\end{corollary}

\subsection{Colimits}

\begin{definition}[Awodey p. 110]
	A functor $F:\cat C\to\cat D$ is said to \emph{create limits of type $\cat J$}
	if for every diagram $C:\cat J\to\cat C$ and limit $p_j:L\to FC_j$ in $\cat D$
	there is a unique cone $\bar{p_j}:\bar L \to C_j$ with $F(\bar L) = L$ and
	$F(\bar{p_j})=p_j$, which furthermore, is a limit for $C$. Briefly, every limit
	in $\cat D$ is the image of a unique cone in $\cat C$, which is a limit there.
	The notion of \emph{creating colimits} is dfined analogously.
\end{definition}

\begin{proposition}
	The forgetful functor $U:\cat{Groups}\to\cat{Sets}$ creates $\omega$-colimits.
	It also creates al limits.
\end{proposition}

\section{Exponentials}

\subsection{Exponential in a category}

\begin{definition}[Awodey p. 121]
	Let the category $\cat C$ have binary products. An \emph{exponential}
	of objects $B$ and $C$ consists of an object
	\begin{align*}
		C^B
	\end{align*}
	and an arrow
	\begin{align*}
		\epsilon:C^B\times B \to C
	\end{align*}
	such that, for any object $A$ and an arrow
	\begin{align*}
		f:A\times B\to C
	\end{align*}
	there is a unique arrow
	\begin{align*}
		\tilde f:A\to C^B
	\end{align*}
	such that
	\begin{align*}
		\epsilon \circ (\tilde f\times 1_B) = f
	\end{align*}
	as in the diagram
	\begin{center}
		\begin{tikzcd}
			C^B &
			C^B\times B \arrow[r, "\epsilon"] &
			C \\
			A \arrow[u, "\tilde f"] &
			A\times B \arrow[u, "\tilde f\times 1_B"] \arrow[ru, swap, "f"]
		\end{tikzcd}
	\end{center}
\end{definition}

\subsection{Cartesian closed categories}

\begin{definition}[Awodey p. 122]
	A category is \emph{cartesian closed}, if it has all finite
	products and exponentials.
\end{definition}

\begin{proposition}[Awodey p. 126]
	In any cartesian closed category $\cat C$, exponentiation by a
	fixed object $A$ is a functor
	\begin{align*}
		(-)^A:\cat C\to\cat C.
	\end{align*}
\end{proposition}

\begin{proposition}[Awodey p. 134]
	A category $\cat C$ is a CCC iff it has the following structure:

	\begin{itemize}
		\item A distinguished object $1$, and for each object $C$ there
		      is given an arrow \begin{align*}
			      !_C:C\to 1
		      \end{align*}
		      such that for each arrow $f:C\to 1$, \begin{align*}
			      f=!_C.
		      \end{align*}
		\item For each pair of objects $A$, $B$, there is given an object
		      $A\times B$ and arrows, \begin{align*}
			      p_1:A\times B \to A\hs\text{and}\hs p_2:A\times B\to B
		      \end{align*}
		      and for each pair of arrows $f:Z\to A$ and $g:Z\to B$, there is given
		      an arrow, \begin{align*}
			      \lra{f,g}:Z\to A\times B
		      \end{align*}
		      such that \begin{align*}
			      p_1\lra{f,g}    & =f                                    \\
			      p_2\lra{f,g}    & =g                                    \\
			      \lra{p_1h,p_2h} & =h & \text{for all }h:Z\to A\times B.
		      \end{align*}
		\item For each pair of objects $A$, $B$, there is given an object
		      $B^A$ and an arrow, \begin{align*}
			      \epsilon : B^A\times B \to B
		      \end{align*}
		      and for each arrow $f:Z\times A \to B$, there is given an arrow \begin{align*}
			      \tilde{f}:Z\to B^A
		      \end{align*}
		      such that \begin{align*}
			      \epsilon\circ(\tilde f \times 1_A)=f
		      \end{align*}
		      and \begin{align*}
			      \widetilde{(\epsilon\circ(g\times 1_A))}=g
		      \end{align*}
		      for all $g:Z\to B^A$. Here, and generally, for any $a:X\to A$ and
		      $b:Y\to B$, we write
		      \begin{align*}
			      a\times b = \lra{a\circ p_1, b\circ p_2}:X\times Y\to A\times B.
		      \end{align*}
	\end{itemize}
\end{proposition}

\subsection{Heyting algebras}

\begin{definition}[Awodey p. 129]
	A \emph{Heyting algebra} is a poset with
	\begin{enumerate}
		\item Finite meets: $1$ and $p\wedge q$,
		\item Finite joins: $0$ and $p\vee q$,
		\item Exponentials: for each $a$, $b$, an element $a\Rightarrow b$
		      such that \begin{align*}
			      a\wedge b \leq c \hs\text{iff}\hs a\leq b\Rightarrow c.
		      \end{align*}
	\end{enumerate}
\end{definition}

\begin{definition}[Awodey p. 130]
	A poset is \emph{(co)complete} if it is so as a category. THis if it has all
	set-indexed meets $\bigwedge_{i\in I}a_i$ (resp. join $\bigvee_{i\in I}a_i$).
	A lattice, Heyting algebra, Boolean algebra, etc. is called \emph{(co)complete}
	if it is so as a poset.
\end{definition}

\begin{proposition}[Awodey p. 130]
	A complete lattice is a Heyting algebra iff it satisfies the
	infinite distributive law
	\begin{align*}
		a\wedge\left(\bigvee_i b_i\right)=\bigvee_i (a\wedge b_i).
	\end{align*}
\end{proposition}

\subsection{\lambda-calculus}

\begin{definition}
	A \emph{theory} $\mathcal{L}$ in the $\lambda$-calculus is a set
	of basic types and terms, together with a set of equations between
	terms.
\end{definition}

\begin{definition}[Awodey p. 138]
	A \emph{model} of a theory $\mathcal{L}$ in a category $\cat C$
	is an assignment of the types and terms of $\mathcal{L}$ to objects
	and arrows of $\cat C$:
	\begin{align*}
		X\text{ basic type} \hs
		 & \mapsto\hs\eqc X \text{ object}                   \\
		b:A\to B\text{ basic term} \hs
		 & \mapsto\hs \eqc b: \eqc A \to \eqc B\text{ arrow}
	\end{align*}
	such that
	\begin{align*}
		\eqc{A\times B} & = \eqc A \times \eqc B \\
		\eqc{\lra{f,g}} & = \lra{\eqc f, \eqc g} \\
		\text{etc.}
	\end{align*}
	and
	\begin{align*}
		\mathcal{L}\vdash a = b : A \to B \hs\text{implies}\hs
		\eqc a = \eqc b : \eqc A \to \eqc B.
	\end{align*}
\end{definition}

\begin{proposition}[Awodey p. 139]
	For any theory $\mathcal{L}$ in the $\lambda$-calculus, one has the
	following:
	\begin{enumerate}
		\item For any terms $a,b,\mathcal{L}\vdash a = b$ iff
		      for all models $M$ in CCCs, $[\![a]\!]_M=[\![b]\!]_M$.
		\item Moreover, for any type $A$, there is a closed $t:A$ iff for all models $M$ in CCCs, there is an arrow $1\to\eqc{A}_M$.
	\end{enumerate}
\end{proposition}

\subsection{Variable sets}

\begin{definition}
	Let $I$ be a poset. Then for each $i\in I$ we define
	the upper set above $i$ as
	\begin{align*}
		\upset(i) \coloneqq \{j\in I : i\leq j\}
	\end{align*}
	and the lower set below $i$ as
	\begin{align*}
		\loset(i) \coloneqq \{j\in I : j\leq i\}.
	\end{align*}
\end{definition}

\begin{definition}[Kripke model]
	A \emph{Kripke model} of a language $\mathcal{L}$ involving $\top$,
	$p\wedge q$, $p\Rightarrow q$, and variables consists of a poset $I$
	of possible worlds, which we write $i\leq j$, together with a relation
	between worlds $i$ and propositions $p$,
	\begin{align*}
		i\Vdash p,
	\end{align*}
	read "$p$ holds at $i$". This relation is assumed to satisfy the
	following conditions:
	\begin{enumerate}[label=(\arabic*)]
		\item $i\Vdash p$ and $i\leq j$ implies $j\Vdash p$
		\item $i\Vdash\top$
		\item $i\Vdash p\wedge q$ iff $i\Vdash p$ and $j\Vdash q$
		\item $i\Vdash p\Rightarrow q$ iff $j\Vdash p$ implies $j\Vdash p$ for all $j\geq i$.
	\end{enumerate}
	One then sets
	\begin{align*}
		I\Vdash p\hs\text{iff}\hs i\Vdash p\hs\text{for all $i\in I$}.
	\end{align*}
\end{definition}

\begin{theorem}[Kripke completeness for IPC; Awodey p. 140]
	A propositional formula $p$ is provable from the rules for IPC
	iff it holds in all Kripke models, that is, iff $I\Vdash p$ for all
	reations $\Vdash$ over all posets $I$,
	\begin{align*}
		\text{IPC}\vdash p \hs\text{iff}\hs I\Vdash p\hs\text{for all $I$}.
	\end{align*}
\end{theorem}

\begin{proposition}
	For every poset CCC $\cat A$, there is a poset $I$ and an injective,
	monotone map,
	\begin{align*}
		y : \cat A \mono \cat 2^I,
	\end{align*}
	preserving CCC structure.
\end{proposition}

\begin{definition}
	Given a poset $I$, an \emph{$I$-indexed set} is a family of sets
	$(A_i)_{i\in I}$ together with transition functions $\alpha_{ij}:
		A_i\to A_j$ for each $i\leq j$, satisfying the compatibility
	conditions:
	\begin{itemize}
		\item $\alpha_{ik}=\alpha_{jk}\circ\alpha_{ij}$ whenever $i\leq j\leq k$,
		\item $\alpha_{ii}=1_{A_i}$ for all $i$.
	\end{itemize}
\end{definition}

\begin{proposition}[Awodey p. 143]
	For any poset $I$, the category $\cat{Sets}^I$ of $I$-indexed sets
	and functions is cartesian closed.
\end{proposition}

\begin{definition}[Awodey p. 144]
	A \emph{Kripke model} of a theory $\mathcal{L}$ in the $\lambda$-calculus
	is a model in a cartesian closed category of the form $\cat{Sets}^I$
	for a poset $I$.
\end{definition}

\begin{proposition}[Awodey p. 144]
	For every CCC $\cat C$, there is a poset $I$ and a functor,
	\begin{align*}
		y : \cat C \mono \cat{Sets}^I,
	\end{align*}
	that is injective on both objects and arrows and preserves CCC structure.
	Moreover, every map between objects in the image of $y$ is itself in the
	image of $y$.
\end{proposition}

\section{Naturality}

\subsection{Category of categories}

\begin{definition}[Awodey p. 148]
	A functor $F:\cat C \to \cat C$ is said to be
	\begin{itemize}
		\item \emph{injective (resp. surjective) on objects} if the object part
		      $F_0:\cat C_0 \to \cat D_0$ is injective.
		\item \emph{injective (resp. surjective) on arrows} if the arrow part
		      $F_1:\cat C_1 \to \cat D_1$ is injective.
		\item \emph{faithful} if for all $A,B\in \cat C_0$ the map \begin{align*}
			      F_{A,B}:\Hom_{\cat C}(A,B)\to\Hom_{\cat D}(FA,FB)
		      \end{align*}
		      defined by $f\mapsto F(f)$ is injective.
		\item \emph{full} if $F_{A,B}$ is always surjective.
	\end{itemize}
\end{definition}

\subsection{Representable structure}

\begin{definition}[Awodey p. 150]
	An object $C\in\cat C$ is called a \emph{generator of $\cat C$} if
	for all objects $X,Y\in\cat C$ and parallel arrows $f,g:X\to Y$
	where $f\not=g$, there exists an arrow $x:C\to X$ such that $fx\not=gx$.
\end{definition}

\begin{proposition}[Special case of Stone Duality; Awodey p. 155]
	Every Boolean algebra $B$ is isomorphic to one consisting of subsets
	of some set $X$, equipped with the set-theoretical Boolean operations.
\end{proposition}

\subsection{Natural transformations}

\begin{definition}[Awodey p. 156]
	For categories $\cat C$, $\cat D$ and functors
	\begin{align*}
		F, G: \cat C \to \cat D
	\end{align*}
	a \emph{natural transformation} $\vartheta:F\to G$ is a family of
	arrows in $\cat D$
	\begin{align*}
		(\vartheta_C: FC\to GC)_{C\in\cat C_0}
	\end{align*}
	such that, for any $f:C\to C'$ in $\cat C$, one has
	\begin{align*}
		\vartheta_{C'}\circ F(f) = G(f) \circ \vartheta_C,
	\end{align*}
	that is, the following commutes:
	\begin{center}
		\begin{tikzcd}
			FC \arrow[r, "\vartheta_C"] \arrow[d, "Ff"] &
			GC \arrow[d, "Gf"] \\
			FC' \arrow[r, "\vartheta_{C'}"] &
			GC'
		\end{tikzcd}
	\end{center}
	Given such a natural transformation $\vartheta: F\to G$, the
	$\cat D$-arrow $\vartheta_C:FC\to GC$ is called the \emph{component
		of $\vartheta$ at $C$}.
\end{definition}

\begin{definition}[Awodey p. 158]
	The \emph{functor category} $\cat{Fun}(\cat C, \cat D)$ has
	\begin{itemize}
		\item objects: functors $F:\cat C \to \cat D$,
		\item arrows: natural transformations $\vartheta:F\to G$.
	\end{itemize}
	For each object $F$, the natural transformation $1_F$ has components
	\begin{align*}
		(1_F)_C = 1_{FC}:FC\to FC
	\end{align*}
	and the composite natural transformation of $\vartheta:F\to G$ and
	$\varphi:G\to H$ has components
	\begin{align*}
		(\varphi\circ\vartheta)_C=\varphi_C\circ\vartheta_C.
	\end{align*}
\end{definition}

\begin{definition}[Awodey p. 158]
	A \emph{natural isomorphism} is a natural transformation
	\begin{align*}
		\vartheta: F\to G
	\end{align*}
	which is an isomorphism in the functor category $\cat{Fun}(\cat C, \cat D)$.
\end{definition}

\begin{lemma}[Awodey p. 159]
	A natural transformation $\vartheta:F\to G$ is a natural isomorphism
	iff each component $\vartheta_C:FC\to GC$ is an isomorphism.
\end{lemma}

\subsection{Exponentials of categories}

\begin{proposition}[Awodey p. 161]
	$\cat{Cat}$ is cartesian closed, with the exponentials
	\begin{align*}
		\cat D^{\cat C} = \text{Fun}(\cat C, \cat D).
	\end{align*}
\end{proposition}

\begin{lemma}[Bifunctor lemma; Awodey p. 161]
	Given categories $\cat A, \cat B$, and $\cat C$, a map of arrows and objects,
	\begin{align*}
		F_0 : \cat A_0 \times \cat B_0 \to \cat C_0 \\
		F_1 : \cat A_1 \times \cat B_1 \to \cat C_1
	\end{align*}
	is a functor $F:\cat A \times \cat B \to \cat C$ iff
	\begin{enumerate}
		\item $F$ is functorial in each argument: $F(A, -):\cat B \to \cat C$ and
		      $F(-, B):\cat A \to \cat C$ are functors for all $A\in\cat A_0$ and $B\in \cat B_0$
		\item $F$ satisfies the following "interchange law." Given $\alpha: A\to A'\in \cat A$ and
		      $\beta: B\to B'\in \cat B$, the following commutes:
		      \begin{center}
			      \begin{tikzcd}
				      F(A,B) \arrow[r, "{F(A,\beta)}"] \arrow[d, "{F(\alpha, B)}"] &
				      F(A,B') \arrow[d, "{F(\alpha, B')}"] \\
				      F(A',B) \arrow[r, "{F(A',\beta)}"] &
				      F(A',B')
			      \end{tikzcd}
		      \end{center}
	\end{enumerate}
\end{lemma}

\end{document}