\documentclass{article}
\usepackage{notes-preamble}
\usepackage{enumitem}
\usepackage{tikz-cd}
\begin{document}
\mkthmstwounified
\title{Group Theory (SEM7)}
\author{Franz Miltz}
\maketitle
\tableofcontents
\pagebreak

\section{Isomorphism theorems}
\label{sec:isomorphism-theorems}

\begin{theorem}[Lagrange]
  \label{thm}
  Let $H$ be a subgroup of a finite group $G$. Then
  \begin{align*}
    \abs{G}=\abs{G/H}\abs{H}.
  \end{align*}
\end{theorem}

\begin{definition}
  \label{def:normal-subgroup}
  A subgroup $H\leq G$ is normal iff, for all $g\in G$, $gH=Hg$. We write $H\trianglelefteq G$.
\end{definition}

\begin{theorem}
  Let $G$ be a group and let $N\leq G$. Then $N\trianglelefteq G$ iff $N$ is the kernel
  of a group homorphism $G\to H$ for some group $H$.
\end{theorem}

\begin{theorem}[First Isomorphism Theorem]
  \label{thm:first-iso-theorem}
  Let $\phi:G\to H$ be a group homorphism. Then $N=\ker\phi$ is a normal subgroup of
  $G$, $\im\phi$ is a subgroup of $H$ and there is an isomorphism
  \begin{align*}
    \psi : G/\ker\phi \to \im\phi
  \end{align*}
  defined by $\psi\rr{gN}=\phi\rr{g}$.
\end{theorem}

\begin{theorem}
  \label{thm:up-factor-groups}
  Let $G$ be a group and let $N\trianglelefteq G$. Then for any homorphism $\psi:G\to H$
  with $N\subseteq\ker\psi$, there is a unique homorphism $\overline\psi:G/N\to H$ such
  that $\overline\psi\circ\text{can}=\psi$. I.e. the following commutes
  \begin{equation}
    \begin{tikzcd}[row sep=huge, column sep=huge]
      G \arrow{r}{\text{can}} \arrow{dr}{\psi} & G/N \arrow{d}{\overline\psi} \\
                                               & H
    \end{tikzcd}
  \end{equation}
\end{theorem}

\begin{proposition}
  Let $G$be a group and let $N\trianglelefteq G$. Let $\text{can}:G\to G/N$ be the canonical
  map. Let $K\leq G/N$. Then
  \begin{enumerate}
    \item $\inv{\text{can}}\rr{K}\leq G$, with $N\subseteq\inv{\text{can}}\rr{K}$;
    \item $\inv{\text{can}}\rr{K}\trianglelefteq G$ iff $K\trianglelefteq G/N$.
  \end{enumerate}
\end{proposition}

\begin{proposition}
  Let $N\trianglelefteq G$ and let $\text{can}:G\to G/N$ be the canonical map. If $N\leq H\leq G$,
  then $H=\inv{\text{can}\rr{\text{can}\rr{H}}}$.
\end{proposition}

\begin{theorem}[Correspondence]
  \label{thm:correspondence}
  Let $G$ be a group, $N\trianglelefteq G$, and let $\text{can}:G\to G/N$ be the canonical map.
  The map $H\mapsto\text{can}\rr{H}$ is a bijection between subgroups of $G$ containing $N$ and
  subgroups of $G/N$. Under this bijection, normal subgroups match with normal subgroups; further,
  if $N\subseteq A,B$ are subgroups of $G$, then $\text{can}\rr{A}\subseteq\text{can}\rr{B}$
  iff $A\subseteq B$.
\end{theorem}

\begin{theorem}[Second Isomorphism Theorem for Groups]
  \label{thm:second-iso-theorem}
  Let $N\trianglelefteq G$ be a normal subgroup. Let $H$ be a subgroup of $G$. Then
  \begin{enumerate}
    \item $HN\leq G$,
    \item $N\trianglelefteq HN$,
    \item $H\cap N\trianglelefteq H$, and
    \item there is an isomorphism \begin{align*}
        HN/N \cong H/\rr{H\cap N}.
      \end{align*}
  \end{enumerate}
\end{theorem}

\begin{theorem}[Third Isomorphism Theorem]
  \label{thm:third-iso-theorem}
  If $N\leq H\leq G$ with $N,H\trianglelefteq G$, then \begin{align*}
    \rr{G/N}/\rr{H/N}\cong G/H.
  \end{align*}
\end{theorem}

\section{Representation}
\label{sec:representation}

\begin{definition}
  \label{def:free-group}
  The free group on generators $x_1,...,x_m$ is the group whose elements are words in the
  symbols $x_1,...,x_m,\inv x_1, ...,\inv x_m$, subject to the group axioms and all logical
  consequences. The group operation is concatenation.

  This group is written $\aa{x_1,...,x_m}$.
\end{definition}

\begin{definition}
  Let $r_1,...,r_n\in\aa{x_1,...,x_m}$. The group generated by $x_1,...,x_m$ subject
  to the relations $r_1,...,r_n$ is the group with generators $x_1,...,x_m$ subject to the
  group axioms, the rules that $r_1=r_2=\cdots=r_n=e$, and all logical consequences.

  We write
  \begin{align*}
    \aa{x_1,...,x_m : r_1,...,r_m}.
  \end{align*}
\end{definition}

\begin{theorem}[Novikov]
  There is no algorithm for deciding whether or not
  \begin{align*}
    \ava{x_1,...,x_m}{r_1,...,r_m}=\cc{e}.
  \end{align*}
\end{theorem}

\begin{proposition}
  Let $G$ be a group generated by a set $\cc{s_1,...,s_n}$. Let $F=\aa{S_1,...,S_n}$
  be the free group generated on the letters $\cc{S_1,...,S_n}$. Then there is a unique
  surjective homorphism $\pi:F\to G$ so that $\pi\rr{S_i}=s_i$ for all $i$.
\end{proposition}

\section{Sylow theorems}
\label{sec:sylow-theorems}

\begin{theorem}[Cauchy]
  If $p$ is a prime that divides $\abs{G}$ then $G$ has a subgroup of order $p$.
\end{theorem}

\begin{definition}
  \label{def:p-subgroup}
  Let $G$ be a finite group and let $p$ be a prime.
  A $p$-subgroup of $G$ is a subgroup $H\leq G$ such that $\abs{H}=p^n$ for some $n$.
  A Sylow $p$-subgroup of $G$ is a $p$-subgroup $H$ such that $\abs{H}$ is the highest
  power of $p$ that divides $\abs{G}$.
\end{definition}

\begin{theorem}[Sylow I]
  Let $\abs{G}=n$ and suppose that $p$ is a prime that divides $n$. Write $n=p^mr$ with
  $p$ not dividing $r$.

  Then there exists at least one subgroup $H\leq G$ of order $p^m$; that is, there is at
  least one Sylow $p$-subgroup.
\end{theorem}

\begin{theorem}[Sylow II]
  Let $\abs{G}=n$ and suppose that $p$ is a prime that divides $n$. Write $n=p^mr$ with
  $p$ not dividing $r$.

  Suppose that $p$ is a Sylow $p$-subgroup and that $H\leq G$ is any $p$-subgroup of $G$.
  Then there exists $x\in G$ with $H\subseteq x\inv Px$. In particular, any two Sylow
  $p$-subgroups are conjugate in $G$.
\end{theorem}

\begin{theorem}[Sylow III]
  Let $\abs{G}=n$ and suppose that $p$ is a prime that divides $n$. Write $n=p^mr$ with
  $p$ not dividing $r$.

  Let $n_p$ be the number of distinct Sylow $p$-subgroups of $G$. Then $n_p\vert r$ and
  $n_p\equiv 1\mod p$.
\end{theorem}

\begin{definition} \label{def:simple-group}
  A simple group is a group $G$ that has no nontrivial normal subgroups.
\end{definition}

\begin{lemma}
  If a group $G$ has a unique Sylow $p$-subgroup $P$ then $P\trianglelefteq G$.
\end{lemma}

\section{Group actions}
\label{sec:group-actions}

\begin{lemma}
  Let $G$ act on $X$.
  \begin{enumerate}
    \item The action induces an equivalence relation $\sim$ on $X$ defined by: $x\sim y$ iff
      there exists $g\in G$ with $g\cdot x= y$.
    \item The equivalence classes of this equivalence are the orbits.
    \item The distinct orbits in $X$ form a partition of $X$.
  \end{enumerate}
\end{lemma}

\begin{lemma}
  Let $G$ be a group that acts on a set $X$. For all $x\in X$, $\stab_G\rr{x}\leq G$.
\end{lemma}

\begin{theorem}[Orbit-Stabiliser]
  Let $G$ be a finite group acting on a set $X$, and let $x\in X$. Then
  \begin{align*}
    \abs{G}=\abs{\stab_G\rr{x}}\abs{G\cdot x}.
  \end{align*}
\end{theorem}

\begin{lemma}
  Let $G$ be a finite group. For any $g\in G$ we have
  \begin{align*}
    \abs{G}=\abs{c_G\rr{g}}\abs{\cl\rr{g}}
  \end{align*}
\end{lemma}

\begin{definition}
  \label{def:p-group}
  Let $p$ be prime. A $p$-group is a group $G$ such that each element has order a power of $p$.
  If $\abs{G}$ is finite, then $G$ is a $p$-group iff $\abs{G}$ is a power of $p$.
\end{definition}

\begin{theorem}[Class equation]
  Let $G$ be a finite group then there are elements $g_1,...,g_n\in G$ such that
  \begin{align*}
    G = \cl\rr{g_1}\sqcup\cdots\sqcup\cl\rr{g_n}.
  \end{align*}
\end{theorem}

\begin{theorem}
  Let $G$ be a nontrivial finite $p$-group. Then the centre $Z\rr{G}\neq \cc{e}$.
\end{theorem}

\begin{lemma}
  Let $p$ be a prime and let $G$ be a finite $p$-gropu acting on a finite set $X$. Then the
  number of fixed points in $X$ is congruent to $\abs{X}$ modulo $p$.
\end{lemma}

\begin{definition}
  \label{def:normaliser}
  Let $G$ be a group and $H\leq G$. The normaliser of $H$ is
  \begin{align*}
    n_G\rr{H}=\cc{g\in G : gH\inv g = H}.
  \end{align*}
\end{definition}

\begin{lemma}
  Let $G$ be a finite group.
  \begin{enumerate}
    \item For any subgroup $H\leq G$, we have
      \begin{align*}
        \abs{G/n_G\rr{H}}=\text{the number of distinct conjugates of $H$}.
      \end{align*}
    \item Let $p\vert \abs{G}$ and let $P$ be a Sylow $p$-subgroup of $G$. Then $n_p=\abs{G/n_G\rr{P}}$.
  \end{enumerate}
\end{lemma}

\end{document}
