\documentclass{article}
\usepackage{notes-preamble}
\usepackage{enumitem}
\usepackage{tikz-cd}
\begin{document}
\mkthmstwounified
\title{Group Theory (SEM7)}
\author{Franz Miltz}
\maketitle
\tableofcontents
\pagebreak

\section{Isomorphism theorems}
\label{sec:isomorphism-theorems}

\begin{theorem}[Lagrange]
  \label{thm}
  Let $H$ be a subgroup of a finite group $G$. Then
  \begin{align*}
    \abs{G}=\abs{G/H}\abs{H}.
  \end{align*}
\end{theorem}

\begin{definition}
  \label{def:normal-subgroup}
  A subgroup $H\leq G$ is normal iff, for all $g\in G$, $gH=Hg$. We write $H\trianglelefteq G$.
\end{definition}

\begin{theorem}
  Let $G$ be a group and let $N\leq G$. Then $N\trianglelefteq G$ iff $N$ is the kernel
  of a group homorphism $G\to H$ for some group $H$.
\end{theorem}

\begin{theorem}[First Isomorphism Theorem]
  \label{thm:first-iso-theorem}
  Let $\phi:G\to H$ be a group homorphism. Then $N=\ker\phi$ is a normal subgroup of
  $G$, $\im\phi$ is a subgroup of $H$ and there is an isomorphism
  \begin{align*}
    \psi : G/\ker\phi \to \im\phi
  \end{align*}
  defined by $\psi\rr{gN}=\phi\rr{g}$.
\end{theorem}

\begin{theorem}
  \label{thm:up-factor-groups}
  Let $G$ be a group and let $N\trianglelefteq G$. Then for any homorphism $\psi:G\to H$
  with $N\subseteq\ker\psi$, there is a unique homorphism $\overline\psi:G/N\to H$ such
  that $\overline\psi\circ\text{can}=\psi$. I.e. the following commutes
  \begin{equation}
    \begin{tikzcd}[row sep=huge, column sep=huge]
      G \arrow{r}{\text{can}} \arrow{dr}{\psi} & G/N \arrow{d}{\overline\psi} \\
                                               & H
    \end{tikzcd}
  \end{equation}
\end{theorem}

\begin{proposition}
  Let $G$be a group and let $N\trianglelefteq G$. Let $\text{can}:G\to G/N$ be the canonical
  map. Let $K\leq G/N$. Then
  \begin{enumerate}
    \item $\inv{\text{can}}\rr{K}\leq G$, with $N\subseteq\inv{\text{can}}\rr{K}$;
    \item $\inv{\text{can}}\rr{K}\trianglelefteq G$ iff $K\trianglelefteq G/N$.
  \end{enumerate}
\end{proposition}

\begin{proposition}
  Let $N\trianglelefteq G$ and let $\text{can}:G\to G/N$ be the canonical map. If $N\leq H\leq G$,
  then $H=\inv{\text{can}\rr{\text{can}\rr{H}}}$.
\end{proposition}

\begin{theorem}[Correspondence]
  \label{thm:correspondence}
  Let $G$ be a group, $N\trianglelefteq G$, and let $\text{can}:G\to G/N$ be the canonical map.
  The map $H\mapsto\text{can}\rr{H}$ is a bijection between subgroups of $G$ containing $N$ and
  subgroups of $G/N$. Under this bijection, normal subgroups match with normal subgroups; further,
  if $N\subseteq A,B$ are subgroups of $G$, then $\text{can}\rr{A}\subseteq\text{can}\rr{B}$
  iff $A\subseteq B$.
\end{theorem}

\begin{theorem}[Second Isomorphism Theorem for Groups]
  \label{thm:second-iso-theorem}
  Let $N\trianglelefteq G$ be a normal subgroup. Let $H$ be a subgroup of $G$. Then
  \begin{enumerate}
    \item $HN\leq G$,
    \item $N\trianglelefteq HN$,
    \item $H\cap N\trianglelefteq H$, and
    \item there is an isomorphism \begin{align*}
        HN/N \cong H/\rr{H\cap N}.
      \end{align*}
  \end{enumerate}
\end{theorem}

\begin{theorem}[Third Isomorphism Theorem]
  \label{thm:third-iso-theorem}
  If $N\leq H\leq G$ with $N,H\trianglelefteq G$, then \begin{align*}
    \rr{G/N}/\rr{H/N}\cong G/H.
  \end{align*}
\end{theorem}

\section{Representation}
\label{sec:representation}

\begin{definition}
  \label{def:free-group}
  The free group on generators $x_1,...,x_m$ is the group whose elements are words in the
  symbols $x_1,...,x_m,\inv x_1, ...,\inv x_m$, subject to the group axioms and all logical
  consequences. The group operation is concatenation.

  This group is written $\aa{x_1,...,x_m}$.
\end{definition}

\begin{definition}
  Let $r_1,...,r_n\in\aa{x_1,...,x_m}$. The group generated by $x_1,...,x_m$ subject
  to the relations $r_1,...,r_n$ is the group with generators $x_1,...,x_m$ subject to the
  group axioms, the rules that $r_1=r_2=\cdots=r_n=e$, and all logical consequences.

  We write
  \begin{align*}
    \aa{x_1,...,x_m : r_1,...,r_m}.
  \end{align*}
\end{definition}

\begin{theorem}[Novikov]
  There is no algorithm for deciding whether or not
  \begin{align*}
    \ava{x_1,...,x_m}{r_1,...,r_m}=\cc{e}.
  \end{align*}
\end{theorem}

\begin{proposition}
  Let $G$ be a group generated by a set $\cc{s_1,...,s_n}$. Let $F=\aa{S_1,...,S_n}$
  be the free group generated on the letters $\cc{S_1,...,S_n}$. Then there is a unique
  surjective homorphism $\pi:F\to G$ so that $\pi\rr{S_i}=s_i$ for all $i$.
\end{proposition}

\section{Sylow theorems}
\label{sec:sylow-theorems}

\begin{theorem}[Cauchy]
  If $p$ is a prime that divides $\abs{G}$ then $G$ has a subgroup of order $p$.
\end{theorem}

\begin{definition}
  \label{def:p-subgroup}
  Let $G$ be a finite group and let $p$ be a prime.
  A $p$-subgroup of $G$ is a subgroup $H\leq G$ such that $\abs{H}=p^n$ for some $n$.
  A Sylow $p$-subgroup of $G$ is a $p$-subgroup $H$ such that $\abs{H}$ is the highest
  power of $p$ that divides $\abs{G}$.
\end{definition}

\begin{theorem}[Sylow I]
  Let $\abs{G}=n$ and suppose that $p$ is a prime that divides $n$. Write $n=p^mr$ with
  $p$ not dividing $r$.

  Then there exists at least one subgroup $H\leq G$ of order $p^m$; that is, there is at
  least one Sylow $p$-subgroup.
\end{theorem}

\begin{theorem}[Sylow II]
  Let $\abs{G}=n$ and suppose that $p$ is a prime that divides $n$. Write $n=p^mr$ with
  $p$ not dividing $r$.

  Suppose that $p$ is a Sylow $p$-subgroup and that $H\leq G$ is any $p$-subgroup of $G$.
  Then there exists $x\in G$ with $H\subseteq x\inv Px$. In particular, any two Sylow
  $p$-subgroups are conjugate in $G$.
\end{theorem}

\begin{theorem}[Sylow III]
  Let $\abs{G}=n$ and suppose that $p$ is a prime that divides $n$. Write $n=p^mr$ with
  $p$ not dividing $r$.

  Let $n_p$ be the number of distinct Sylow $p$-subgroups of $G$. Then $n_p\vert r$ and
  $n_p\equiv 1\mod p$.
\end{theorem}

\begin{definition} \label{def:simple-group}
  A simple group is a group $G$ that has no nontrivial normal subgroups.
\end{definition}

\begin{lemma}
  If a group $G$ has a unique Sylow $p$-subgroup $P$ then $P\trianglelefteq G$.
\end{lemma}

\section{Group actions}
\label{sec:group-actions}

\begin{lemma}
  Let $G$ act on $X$.
  \begin{enumerate}
    \item The action induces an equivalence relation $\sim$ on $X$ defined by: $x\sim y$ iff
      there exists $g\in G$ with $g\cdot x= y$.
    \item The equivalence classes of this equivalence are the orbits.
    \item The distinct orbits in $X$ form a partition of $X$.
  \end{enumerate}
\end{lemma}

\begin{lemma}
  Let $G$ be a group that acts on a set $X$. For all $x\in X$, $\stab_G\rr{x}\leq G$.
\end{lemma}

\begin{theorem}[Orbit-Stabiliser]
  Let $G$ be a finite group acting on a set $X$, and let $x\in X$. Then
  \begin{align*}
    \abs{G}=\abs{\stab_G\rr{x}}\abs{G\cdot x}.
  \end{align*}
\end{theorem}

\begin{lemma}
  Let $G$ be a finite group. For any $g\in G$ we have
  \begin{align*}
    \abs{G}=\abs{c_G\rr{g}}\abs{\cl\rr{g}}
  \end{align*}
\end{lemma}

\begin{definition}
  \label{def:p-group}
  Let $p$ be prime. A $p$-group is a group $G$ such that each element has order a power of $p$.
  If $\abs{G}$ is finite, then $G$ is a $p$-group iff $\abs{G}$ is a power of $p$.
\end{definition}

\begin{theorem}[Class equation]
  Let $G$ be a finite group then there are elements $g_1,...,g_n\in G$ such that
  \begin{align*}
    G = \cl\rr{g_1}\sqcup\cdots\sqcup\cl\rr{g_n}.
  \end{align*}
\end{theorem}

\begin{theorem}
  Let $G$ be a nontrivial finite $p$-group. Then the centre $Z\rr{G}\neq \cc{e}$.
\end{theorem}

\begin{lemma}
  Let $p$ be a prime and let $G$ be a finite $p$-gropu acting on a finite set $X$. Then the
  number of fixed points in $X$ is congruent to $\abs{X}$ modulo $p$.
\end{lemma}

\begin{definition}
  \label{def:normaliser}
  Let $G$ be a group and $H\leq G$. The normaliser of $H$ is
  \begin{align*}
    n_G\rr{H}=\cc{g\in G : gH\inv g = H}.
  \end{align*}
\end{definition}

\begin{lemma}
  Let $G$ be a finite group.
  \begin{enumerate}
    \item For any subgroup $H\leq G$, we have
      \begin{align*}
        \abs{G/n_G\rr{H}}=\text{the number of distinct conjugates of $H$}.
      \end{align*}
    \item Let $p\vert \abs{G}$ and let $P$ be a Sylow $p$-subgroup of $G$. Then $n_p=\abs{G/n_G\rr{P}}$.
  \end{enumerate}
\end{lemma}

\section{Finitely generated abelian groups}\label{sec:finitely-generated-abelian-groups}

\subsection{Finite abelian groups}\label{sec:finite-abelian-groups}

\begin{theorem}
  Suppose that $A$ is a finite abelian group of order $n$ and that $n=p_1^{s_1}\cdots p_t^{s_t}$.
  Let $A_{p_i}$ be the unique Sylow $p_i$-subgroup of $A$. Then
  \begin{align*}
    A\cong A_{p_1}\times \cdots\times A_{p_t};
  \end{align*}
  that is, $A$ is isomorphic to the direct product of its Sylow subgroups.
\end{theorem}

\begin{theorem}
  Let $A$ be an abelian group with $\abs{A}=p^n$ for some prime $p$. Then $A$ is isomorphic
  to the direct product of cyclic subgroups of order $p^{e_1},...,p^{e_s}$ where
  $e_1\geq e_2\geq \cdots \geq 1$ and $e_1+\cdots+e_s=n$. This product is unique up to
  reordering the factors.
\end{theorem}

\begin{theorem}[Fundamental Theorem of Finite Abelian Groups, I]
  Let $A$ be a finite abelian group. Then $A$ is a direct product of cyclic groups of prime
  power order. This product is unique up to reordering the factors.
\end{theorem}

\begin{theorem}[Chinese Remainder Theorem]
  Let $m,n$ be nonzero coprime integers. Then $C_{mn}\cong C_m\times C_n$.
\end{theorem}

\begin{theorem}[Fundamental Theorem of Finite Abelian Groups, II]
  Any finite abelian group of order $n$ can be written as a direct product of cyclic groups
  \begin{align*}
    C_{n_1}\times\cdots\times C_{n_s}
  \end{align*}
  where $n_i$ divides $n_{i+1}$ for each $i=1,...,s-1$ and $n_1\cdots n_s=n$. This product is
  unique up to reordering of the factors.
\end{theorem}

\begin{definition}
  The exponent $e\rr{G}$ of a finite group is the least common multiple of the orders of the
  elements of $G$.
\end{definition}

\begin{lemma}
  Let $A$ be a finite abelian group. Then $A$ contains an element of order $e\rr{A}$.
\end{lemma}

\begin{corollary}
  If $A$ is a finite abelian group with $e\rr{A}=\abs{A}$ then $A$ is cyclic.
\end{corollary}

\begin{theorem}
  Let $A$ be a finite subgroup of the multiplicative group $K^\times = K\setminus\cc{0}$
  of a field $K$. Then $A$ is cyclic.
\end{theorem}

\begin{corollary}
  The multiplicative group of nonzero elements of a finite field is cyclic.
\end{corollary}

\subsection{Linear algebra over $\Z$}\label{sec:linear-algebra-over-z}

\begin{definition}\label{def:module}
  Let $R$ be a ring. An $R$-module is an abelian group $\rr{M,+}$ together with a mapping
  \begin{align*}
    R\times M&\to M
    \rr{r,a} &\mapsto ra
  \end{align*}
  that is distributive, associative, and unital.
\end{definition}

\begin{definition}\label{def:free-module}
  Let $R$ be a ring and let $n\in\N$. The free $R$-module of rank $n$ is the $n$-fold
  cartesian product $R^n$. It is given a module structure by
  \begin{align*}
    r\rr{a_1,\ldots,a_k}=\rr{ra_1,\ldots,ra_k}.
  \end{align*}
\end{definition}

\begin{theorem}[Fundamental Theorem of Finitely Generated Abelian Groups]
  Let $A$ be a finitely generated abelian group. Then
  \begin{align*}
    A\cong \Z/r_1\Z\times\cdots\times \Z/r_k\Z\times \Z^\ell,
  \end{align*}
  for some $k,\ell\in\N$ and $r_1,...,r_k$ nonzero elements of $\Z$ with $r_1\vert r_2\vert\ldots\vert r_k$.
\end{theorem}

\section{Alternating groups}

\subsection{Symmetric groups}

\begin{lemma}
  $S_n$ is generated by transpositions.
\end{lemma}

\begin{definition}
  Suppose that $\sigma=c_1\ldots c_k$ is a product of $k$ disjoint cycles of lengths
  $l_1,...,l_k$ with $l_1\geq\ldots\geq l_k$. Then the $k$-tuple $\rr{l_1,...,l_k}$
  is called the cycle type of $\sigma$.
\end{definition}

\begin{lemma}
  Let $\sigma=\rr{a_1\cdots a_k}\in S_n$ and let $\tau\in S_n$. Then
  \begin{align*}
    \tau\sigma\inv\tau = \rr{\tau\rr{a_1}\cdots\tau\rr{a_k}}.
  \end{align*}
\end{lemma}

\begin{theorem}
  Two permutations in $S_n$ are conjugate iff they have the same cycle type.
\end{theorem}

\subsection{Alternating groups}

\begin{lemma}
  The product of two even permutations is even. The product of two odd permutations
  is even. The product of an odd and an even permutation (in either order) is odd.

  A cycle of length $n$ is even iff $n$ is odd.
\end{lemma}

\begin{theorem}
  Let $n\geq 2$. The set of even permutations $A_n$ is a normal subgroup of $S_n$
  of index $2$ so that $\abs{A_n}=\abs{S_n}/2=n!/2$ for $n\geq 2$.
\end{theorem}

\begin{proposition}
  The alternating group $A_4$ has order $12$. It has a unique subgroup $N$ of order $4$.
  The subgroup $N$ is normal in $S_4$ and $A_4/N\cong C_3$ while $S_4/N\cong S_3$.
\end{proposition}

\begin{lemma}
  Let $G$ be a finite group and suppose that $H\trianglelefteq G$. Then there are
  $h_1,\ldots,h_k\in H$ so that $H=\bigsqcup\cl_G\rr{h_i}$.
\end{lemma}

\begin{definition}
  A group is simple if it only has trivial normal subgroups.
\end{definition}

\begin{theorem}
  Let $n\geq 5$. Then $A_n$ is simple.
\end{theorem}

\section{Jordan-H\"older Theorem}

\begin{definition}
  Let $G$ be a group. A composition series for $G$ is a chain of subgroups
  \begin{align*}
    \cc{e}=G_0\trianglelefteq G_1\trianglelefteq\ldots\trianglelefteq G_s = G
  \end{align*}
  where $G_i\neq G_{i+1}$ and $G_{i+1}/G_i$ is simple for all $i$.
\end{definition}

\begin{theorem}[Jordan, H\"older]
  Let $G$ be a finite group. Then $G$ has a composition series. Moreover, any two
  composition series have the same composition length, and they have the same composition
  factors up to isomorphism of groups and order of the factors.
\end{theorem}

\section{Solvable groups}

\begin{definition}
  Let $G$ be a group. A subnormal series for $G$ is a series of subgroups
  \begin{align*}
    \cc{e}=G_0\trianglelefteq G_1\trianglelefteq\ldots\trianglelefteq G_s = G.
  \end{align*}
\end{definition}

\begin{definition}
  A group $G$ is solvable provided that it has a subnormal series
  \begin{align*}
    \cc{e}=G_0\trianglelefteq G_1\trianglelefteq\ldots\trianglelefteq G_s = G
  \end{align*}
  such that each factor $G_{i+1}/G_i$ is abelian.
\end{definition}

\begin{theorem}
  A finite group $G$ is solvable iff all the composition factors of $G$ are cyclic.
\end{theorem}

\begin{lemma}
  If $A$ is a finite abelian group of order $p_1^{n_1}...p_k^{n_k}$ then the composition
  factors of $A$ are
  \begin{align*}
    C_{p_1},...,C_{p_1},\ldots,C_{p_k}\ldots C_{p_k}
  \end{align*}
  in some order.
\end{lemma}

\begin{theorem}
  Let $G$ be a group and let $N\trianglelefteq G$. Then $G$ is solvable iff both $N$ and
  $G/N$ are solvable.
\end{theorem}

\begin{theorem}
  If $G$ is solvable and $H\leq G$ is solvable then $H$ is solvable.
\end{theorem}

\begin{theorem}
  A general degree $n$ polynomial $f\rr{x}$ with rational coefficients is not solvable
  by radicals if $n\geq 5$.
\end{theorem}

\section{Derived subgroups}

\begin{definition}
  Let $G$ be a group. The commutator of two elements $a,b\in G$ is the element $ab\inv a\inv b$
  and is often denoted by $\bb{a,b}$. The derived subgroup $G'$ of a group $G$ is the
  subgroup generated by all possible commutators in $G$, i.e.
  \begin{align*}
    G':=\aa{ab\inv a\inv b : a,b\in G}.
  \end{align*}
\end{definition}

\begin{theorem}
  Let $G$ be a group and let $N$ be a normal subgroup of $G$. Then $G/N$ is abelian iff
  $G'\subseteq N$. In particular, $G/G'$ is abelian.
\end{theorem}

\begin{definition}
  Let $G$ be a group. Set $G^0=G$ and for each $i\geq 0$ set $G^{\rr{i+1}}=\rr{G^{\rr{i}}}'$.
  The sequence
  \begin{align*}
    G=G^{\rr{0}}\trianglerighteq G^{\rr{1}}\trianglerighteq \cdots
  \end{align*}
  is called the derived series of $G$.
\end{definition}

\begin{theorem}
  A group $G$ is solvable iff there is an $n$ with $G^{\rr{n}}=\cc{e}$.
\end{theorem}

\begin{definition}
  Let $G$ be a solvable group. Then $G^{\rr{n}}=\cc{e}$ for some $n$. The least such
  $n$ is the derived length of $G$.
\end{definition}

\begin{theorem}[Feit-Thompson]
  Let $G$ be a finite group with odd order. Then $G$ is solvable.
\end{theorem}

\end{document}
