\documentclass{article}
\usepackage{notes-preamble}
\usepackage{enumitem}
\usepackage{tikz-cd}
\usepackage{quiver}

\begin{document}
\mkthmstwounified
\title{Categories and Quantum Informatics (SEM8)}
\author{Franz Miltz}

\maketitle

\tableofcontents

\pagebreak

\section{Linear structure}

Let $(\cat{C}, \otimes, I)$ be a monoidal category.

\subsection{Scalars}

\begin{definition}
  A \emph{scalar} in $\cat{C}$ is a morphism $I\to I$.
\end{definition}

\begin{lemma}
  Scalars in $\cat{C}$ commute. That is, for scalars $s,t:I\to I$, $s\circ t = t\circ s$.
\end{lemma}

\begin{definition}
  Let $s:I\to I$ be a scalar and let $f:X\to Y$ be a morphism. Then the \emph{left scalar
  multiplication of $f$ by $s$} is the composite
  \begin{equation}
    % https://q.uiver.app/?q=WzAsNCxbMCwwLCJYIl0sWzIsMCwiWSJdLFswLDEsIklcXG90aW1lcyBYIl0sWzIsMSwiSVxcb3RpbWVzIFkiXSxbMiwzLCJzXFxvdGltZXMgZiIsMl0sWzAsMiwiXFxsYW1iZGFeey0xfSIsMl0sWzMsMSwiXFxyaG9eey0xfSIsMl0sWzAsMSwic1xcYnVsbGV0IGYiXV0=
    \begin{tikzcd}
      X && Y \\
      {I\otimes X} && {I\otimes Y}
      \arrow["{s\otimes f}"', from=2-1, to=2-3]
      \arrow["{\lambda^{-1}}"', from=1-1, to=2-1]
      \arrow["{\rho^{-1}}"', from=2-3, to=1-3]
      \arrow["{s\bullet f}", from=1-1, to=1-3]
    \end{tikzcd}
  \end{equation}
\end{definition}

\begin{lemma}
  Let $s,t:I\to I$ be scalars and $f:X\to Y$ and $g:Y\to Z$ be morphisms. Then
  \begin{enumerate}
    \item $\identity_Y\bullet f = f$;
    \item $s\bullet t = s\circ t$;
    \item $(s\bullet t)\bullet f = s\bullet (t\bullet f)$;
    \item $(t\bullet g)\circ(s\bullet f) = (t\circ s)\bullet (g\circ f)$.
  \end{enumerate}
\end{lemma}

\subsection{Superposition}

\begin{definition}
  A \emph{superposition rule} for $\cat{C}$ consists of
  \begin{enumerate}
    \item for all $f,g:X\to Y$, an arrow $f+g:X\to Y$;
    \item for all $X,Y\in\cat{C}$, an arrow $u_{X,Y}:X\to Y$
  \end{enumerate}
  such that, for suitable morphisms,
  \begin{enumerate}
    \item $f+g=g+f$;
    \item $f+(g+h)=(f+g)+h$;
    \item $f+u = f$;
    \item $(g'\circ g)+f=(g'\circ f)+(g\circ f)$;
    \item $g+(f'\circ f) = (g\circ f')+(g\circ f)$;
    \item $u\circ f = u$ and $f\circ u = u$.
  \end{enumerate}
\end{definition}

\begin{lemma}
  In a category with a zero object and a superposition rule, $u_{X,Y}=0_{X,Y}$ for
  all $X,Y\in\cat{C}$.
\end{lemma}

\begin{lemma}
  If a monoidal category has a zero object and a superposition rule then its scalars
  form a commutative semiring with absorbing zero.
\end{lemma}

\subsection{Biproducts}

\begin{definition}
  In a category with a superposition rule and a zero bject, the biproduct
  of objects $X_1,X_2$ is an object $X_1\oplus X_2$ together with projections
  $\pi_j : X_1\oplus X_2\to X_j$ and injections $\iota_j :X_j\to X_1\oplus X_2$
  such that
  \begin{enumerate}
    \item $\pi_j\circ\iota_j = \identity_{X_j}$;
    \item $\pi_j\circ\iota_{k} = 0_{A_k,A_j}$ for $j\neq k$;
    \item $\pi_1\circ\iota_1 + \pi_2\circ\iota_1 = \identity_{X_1\oplus X_2}$.
  \end{enumerate}
\end{definition}

\begin{lemma}
  Biproducts are products and coproducts.
\end{lemma}

\begin{lemma}
  If a category has biproducts then it has a unique superposition rule.
\end{lemma}

\begin{lemma}
  Let $\cat{C}$ be a category with biproducts and a zero object. If $F:\cat{C}\to\cat{D}$
  preserves zero objects then it preserves biproducts iff it is linear.
\end{lemma}

\subsection{Matrices}

\begin{definition}
  Let $X_1,\ldots,X_M$ and $Y_1,\ldots,Y_N$ be objects and $f_{m,n}:X_m\to Y_n$
  be arrows. Then we have the \emph{matrix}
  \begin{align*}
    (f_{j,k}) =
    \begin{pmatrix}
      f_{1,1} &\cdots & f_{M,1} \\
      \vdots &\ddots & \vdots \\
      f_{1,N} & \cdots & f_{M,N}
    \end{pmatrix}
    = \sum_{m,n} \iota_n \circ f_{m,n} \circ \pi_m.
  \end{align*}
\end{definition}

\begin{lemma}
  In a category with biproducts, every morphism $\oplus_{m=1}^M X_m \to \oplus_{n=1}^N Y_n$
  has a matrix representation.
\end{lemma}

\begin{corollary}
  Morphisms between biproducts are equal iff their matrix entries are equal.
\end{corollary}

\subsection{Daggers}

\begin{definition}
  A \emph{dagger} on a category $\cat{C}$ is an involutive contravariant functor
  $\dagger : \catop{C}\to\cat{C}$ that is the identity on objects. A \emph{dagger category}
  is a category equipped with a dagger.
\end{definition}

\begin{example}
  We have the following dagger structures:
  \begin{itemize}
    \item In $\Hilb$, the dagger is given by taking adjoints.
    \item In $\Rel$, the dagger is given by $t R^\dagger s$ iff $s R t$.
  \end{itemize}
\end{example}

\begin{definition}
  A \emph{monoidal dagger category} is a monoidal category $(C,\otimes)$ equipped with
  a dagger such that $(f\otimes g)^\dagger = f^\dagger \otimes g^\dagger$.
\end{definition}

\begin{definition}
  In a dagger category, a \emph{dagger biproduct} of objects $X_1$ and $X_2$ is a biproduct
  such that $\iota_j^\dagger = \pi_j$.
\end{definition}

\begin{lemma}
  In a dagger category with dagger biproducts, the dagger of a matrix is its conjugate transpose.
\end{lemma}

\begin{definition}
  A morphism $f:X\to Y$ in a dagger category is
  \begin{itemize}
    \item \emph{idempotent} when $f=f^\dagger$;
    \item a \emph{projection} when it is idempotent and self-adjoint;
    \item \emph{unitary} when $f^\dagger \circ f = \identity$ and $f\circ f^\dagger=\identity$;
    \item an \emph{isometry} when $f^\dagger \circ f=\identity$;
    \item a \emph{partial isometry} when $f^\dagger\circ f$ is a projection;
    \item \emph{positive} when $f=g^\dagger\circ g$ for some $g$.
  \end{itemize}
\end{definition}

\subsection{Measurement}

\begin{definition}
  Consider a state $a : I\to X$ and an effect $x : X\to I$ in a monoidal dagger category.
  Define
  \begin{align*}
    \Pr(x,a) = a^\dagger \circ x^\dagger \circ a\circ x.
  \end{align*}
\end{definition}

\begin{definition}
  In a dagger category with a zero object, an isometry $k:K\to X$ is \emph{a dagger kernel}
  of $f:X\to Y$ when $f\circ k=0$ and $k$ is universal with this property.
\end{definition}

\begin{lemma}
  In a dagger category with a zero object, isometries always have a dagger kernel given
  by the zero morphism.
\end{lemma}

\begin{lemma}
  In a dagger category with a zero object and dagger kernels of arbitrary morphisms,
  for all $f:X\to Y$, $f^\dagger \circ f = 0$ implies $f=0$.
\end{lemma}

\begin{definition}
  A set of effects $\cc{x_j}$ with $x_j : X\to I$ is
  \begin{enumerate}
    \item \emph{complete} if, for all non-zero $f:W\to X$, there is some $j$
      such that $x_j \circ f \neq 0$;
    \item \emph{disjoint} if, for all $i\neq j$, $x_i\circ x_i^\dagger = \identity$
      and $x_i\circ x_j^\dagger = 0$.
  \end{enumerate}
\end{definition}

\begin{example}
  We have the following complete disjoint sets of effects:
  \begin{itemize}
    \item In \Hilb, complete disjoint sets of effects correspond to unitaries $H\simeq\C^n$,
      i.e. choices of orthonormal bases.
    \item In \Rel, a complete disjoint set of effects for a set corresponds to a partition.
  \end{itemize}
\end{example}

\begin{lemma}
  Finite complete disjoint sets of effects $x_j:X\to I$ correspond exactly to morphisms
  $x:X\to\bigoplus_j I$ for which $x^\dagger$ is an isometry and that have zero kernel.
\end{lemma}

\begin{proposition}[Born rule]
  Let $x_j:X\to I$ be a finite complete set of effects in a monoidal dagger category with
  dagger biproducts, and let $a:I\to X$ be an isometric state. Then
  $\sum_j \Pr{x_j,a}=1$.
\end{proposition}

\section{Duals}\label{sec:duals}

\subsection{Dual objects}\label{sec:dual_objects}

\begin{definition}
  In a monoidal category, an object is \emph{left-dual} to an object $R$, written $L\dashv R$,
  when there exists a unit $\eta:I\to R\otimes L$ and a counit $\varepsilon : L\otimes R\to I$
  such that
  \begin{equation}
    % https://q.uiver.app/?q=WzAsNixbMCwwLCJMIl0sWzIsMSwiSVxcb3RpbWVzIEwiXSxbMiwwLCJMXFxvdGltZXMgSSJdLFs0LDEsIihMXFxvdGltZXMgUilcXG90aW1lcyBMIl0sWzAsMSwiTCJdLFs0LDAsIkxcXG90aW1lcyhSXFxvdGltZXMgTCkiXSxbMCwyLCJcXGNvbmciLDFdLFs1LDMsIlxcY29uZyIsMV0sWzEsNCwiXFxjb25nIiwxXSxbMiw1LCJMXFxvdGltZXNcXGV0YSIsMV0sWzMsMSwiXFx2YXJlcHNpbG9uXFxvdGltZXMgTCIsMV0sWzAsNCwiIiwxLHsibGV2ZWwiOjIsInN0eWxlIjp7ImhlYWQiOnsibmFtZSI6Im5vbmUifX19XV0=
    \begin{tikzcd}
      L && {L\otimes I} && {L\otimes(R\otimes L)} \\
      L && {I\otimes L} && {(L\otimes R)\otimes L}
      \arrow["\cong"{description}, from=1-1, to=1-3]
      \arrow["\cong"{description}, from=1-5, to=2-5]
      \arrow["\cong"{description}, from=2-3, to=2-1]
      \arrow["L\otimes\eta"{description}, from=1-3, to=1-5]
      \arrow["{\varepsilon\otimes L}"{description}, from=2-5, to=2-3]
      \arrow[Rightarrow, no head, from=1-1, to=2-1]
    \end{tikzcd}
  \end{equation}
\end{definition}

\begin{example}
  We have the following duals:
  \begin{itemize}
    \item In \FHilb, duals are dual spaces and the corresponding maps are
      \begin{align*}
        \eta : 1 \mapsto \sum_i \av{i}\otimes\va{i},\hs
        \e : \va{\phi}\otimes\va{\psi}\mapsto\ava{\phi}{\psi}
      \end{align*}
      where $\va{i}$ is an orthonormal basis.
    \item In \Hilb, infinite dimensional spaces do not have duals.
    \item In \Rel, every object is its own dual and the corresponding relations
      are
      \begin{align*}
        \eta : * \sim (x,x),\hs
        \e : (x,x)\sim *.
      \end{align*}
    \item In $\Mat_{\C}$, every object $n$ is its own dual with linear maps
      \begin{align*}
        \eta : 1 \mapsto \sum_i \va{i}\otimes\va{i}, \hs
        \e : \va{i}\otimes\va{j}\to \delta_{ij}.
      \end{align*}
  \end{itemize}
\end{example}

\begin{definition}
  In a monoidal category with dualities $A\dashv X^*$ and $B\dashv Y^*$,
  the name and coname of a morphism $f:X\to Y$ are
  the morphisms $\overline f : I \to A^* \otimes B$ and
  $\underline f : A\otimes B^* \to I$ given by
  \begin{align*}
    \overline f = (\identity\otimes f)\circ\eta,\hs
    \underline f = \e \circ (f\otimes\identity).
  \end{align*}
\end{definition}

\begin{lemma}
  Duals are unique up to canonical isomorphism.
\end{lemma}

\begin{lemma}
  For a duality $X\dashv X^*$, choosing a unit fixes the counit and vice versa.
\end{lemma}

\begin{lemma}
  $I\dashv I$.
\end{lemma}

\subsection{Right duals functor}

\begin{definition}
  In a monoidal category with dualitities $X\dashv X^*$ and $Y\dashv Y^*$ and a morphism
  $f:X\to Y$, the \emph{transpose} is the morphism
  \begin{equation}
    % https://q.uiver.app/?q=WzAsNixbMCwwLCJCXioiXSxbMiwwLCJBXioiXSxbMCwxLCJJXFxvdGltZXMgQl4qIl0sWzAsMiwiKEFeKlxcb3RpbWVzIEIpXFxvdGltZXMgQl4qIl0sWzIsMiwiQV4qXFxvdGltZXMgKEJcXG90aW1lcyBCXiopIl0sWzIsMSwiQV4qXFxvdGltZXMgSSJdLFswLDEsImZeKiIsMV0sWzUsMSwiXFxjb25nIiwxXSxbMyw0LCJcXGNvbmciLDFdLFswLDIsIlxcY29uZyIsMV0sWzIsMywiXFxldGFcXG90aW1lcyBCXioiLDJdLFs0LDUsIkFeKlxcb3RpbWVzXFx2YXJlcHNpbG9uIiwyXV0=
    \begin{tikzcd}
      {B^*} && {A^*} \\
      {I\otimes B^*} && {A^*\otimes I} \\
      {(A^*\otimes B)\otimes B^*} && {A^*\otimes (B\otimes B^*)}
      \arrow["{f^*}"{description}, from=1-1, to=1-3]
      \arrow["\cong"{description}, from=2-3, to=1-3]
      \arrow["\cong"{description}, from=3-1, to=3-3]
      \arrow["\cong"{description}, from=1-1, to=2-1]
      \arrow["{\eta\otimes B^*}"', from=2-1, to=3-1]
      \arrow["{A^*\otimes\varepsilon}"', from=3-3, to=2-3]
    \end{tikzcd}
  \end{equation}
\end{definition}

\begin{definition}
  In a monoidal category $\cat{C}$ with a choice of duals, the right duals functor
  $(-)^* : \cat{C}\to\catop{C}$ takes objects to right duals and morphisms to transposes.
\end{definition}

\section{Compact categories}

\begin{definition}
  A monoidal category with right duals is pivotal when it is equipped with a monoidal
  natural transformation $\pi_X : X\to X^{**}$.
\end{definition}

\begin{definition}
  A braided monoidal category is \emph{balanced} when it is equipped with a natural
  isomorphism $\theta_X:X\to X$, such that $\theta_I = \identity$ and
  \begin{equation}
    % https://q.uiver.app/?q=WzAsNCxbMCwwLCJYXFxvdGltZXMgWSJdLFsyLDAsIlhcXG90aW1lcyBZIl0sWzAsMSwiWVxcb3RpbWVzIFgiXSxbMiwxLCJZXFxvdGltZXMgWCJdLFswLDIsIlxcZ2FtbWEiLDJdLFszLDEsIlxcZ2FtbWEiLDJdLFsyLDMsIlxcdGhldGFcXG90aW1lc1xcdGhldGEiLDJdLFswLDEsIlxcdGhldGEiXV0=
    \begin{tikzcd}
      {X\otimes Y} && {X\otimes Y} \\
      {Y\otimes X} && {Y\otimes X}
      \arrow["\gamma"', from=1-1, to=2-1]
      \arrow["\gamma"', from=2-3, to=1-3]
      \arrow["\theta\otimes\theta"', from=2-1, to=2-3]
      \arrow["\theta", from=1-1, to=1-3]
    \end{tikzcd}
  \end{equation}
\end{definition}

\begin{theorem}
  In a braided monoidal category with right duals, a pivotal structure induces a twist and
  vice versa.
\end{theorem}

The structure induce each other as follows:
\begin{itemize}
  \item Given a pivotal structure $\pi_X : X\to X^{**}$, define the induced twist by
    \begin{equation}\label{eq:canonical_twist}
      % https://q.uiver.app/?q=WzAsNyxbMCwwLCJYIl0sWzQsMCwiWCJdLFs0LDEsIlheeyoqfSJdLFswLDEsIlhcXG90aW1lcyBJIl0sWzAsMiwiWFxcb3RpbWVzIChYXnsqKn1cXG90aW1lcyBYXiopIl0sWzIsMiwiWF57Kip9XFxvdGltZXMgKFhcXG90aW1lcyBYXiopIl0sWzQsMiwiWF57Kip9XFxvdGltZXMgSSJdLFswLDEsIlxcdGhldGEiXSxbMiwxLCJcXHBpXnstMX0iLDJdLFszLDQsIlhcXG90aW1lcyBcXGV0YSIsMl0sWzAsMywiXFxjb25nIiwyXSxbNiwyLCJcXGNvbmciLDJdLFs0LDUsIlxcY29uZyIsMl0sWzUsNiwiWF57Kip9XFxvdGltZXNcXHZhcmVwc2lsb24iLDJdXQ==
      \begin{tikzcd}
        X &&&& X \\
        {X\otimes I} &&&& {X^{**}} \\
        {X\otimes (X^{**}\otimes X^*)} && {X^{**}\otimes (X\otimes X^*)} && {X^{**}\otimes I}
        \arrow["\theta", from=1-1, to=1-5]
        \arrow["{\pi^{-1}}"', from=2-5, to=1-5]
        \arrow["{X\otimes \eta}"', from=2-1, to=3-1]
        \arrow["\cong"', from=1-1, to=2-1]
        \arrow["\cong"', from=3-5, to=2-5]
        \arrow["\cong"', from=3-1, to=3-3]
        \arrow["{X^{**}\otimes\varepsilon}"', from=3-3, to=3-5]
      \end{tikzcd}
    \end{equation}
  \item Given a twist $\theta_X : X\to X$, define the induced pivotal structure by
    \begin{equation}
      % https://q.uiver.app/?q=WzAsNyxbMCwwLCJYIl0sWzQsMCwiWF57Kip9Il0sWzAsMSwiWCJdLFswLDIsIlhcXG90aW1lcyBJIl0sWzIsMiwiWFxcb3RpbWVzKFheeyoqfVxcb3RpbWVzIFheKikiXSxbNCwyLCJYXnsqKn1cXG90aW1lcyhYXFxvdGltZXMgWF4qKSJdLFs0LDEsIlheeyoqfVxcb3RpbWVzIEkiXSxbMCwxLCJcXHBpIl0sWzAsMiwiXFx0aGV0YV57LTF9IiwyXSxbNiwxLCJcXGNvbmciLDJdLFs1LDYsIlheeyoqfVxcb3RpbWVzXFx2YXJlcHNpbG9uIiwyXSxbNCw1LCJcXGNvbmciLDJdLFszLDQsIlhcXG90aW1lcyBcXGV0YSIsMl0sWzIsMywiXFxjb25nIiwyXV0=
      \begin{tikzcd}
        X &&&& {X^{**}} \\
        X &&&& {X^{**}\otimes I} \\
        {X\otimes I} && {X\otimes(X^{**}\otimes X^*)} && {X^{**}\otimes(X\otimes X^*)}
        \arrow["\pi", from=1-1, to=1-5]
        \arrow["{\theta^{-1}}"', from=1-1, to=2-1]
        \arrow["\cong"', from=2-5, to=1-5]
        \arrow["{X^{**}\otimes\varepsilon}"', from=3-5, to=2-5]
        \arrow["\cong"', from=3-3, to=3-5]
        \arrow["{X\otimes \eta}"', from=3-1, to=3-3]
        \arrow["\cong"', from=2-1, to=3-1]
      \end{tikzcd}
    \end{equation}
\end{itemize}

\begin{definition}
  A \emph{compact category} is a pivotal symmetric monoidal category where the
  canonical twist (\ref{eq:canonical_twist}) is the identity.
\end{definition}

\end{document}
