\documentclass{article}
\usepackage{homework-preamble}

\title{Categories for Quantum Informatics: Coursework}
\author{Franz Miltz}

\begin{document}
\maketitle

\section*{Question 1}

\begin{claim*}[a]
  An arrow in $\textbf{Set}$ is an isomorphism if and only if it is a bijection.
  \begin{proof}
    Let $f:X\to Y$ be a bijection. By surjectivity, for all $y\in Y$, there is an $x\in X$
    such that $f(x)=y$. Moreover, this $x$ is unique by injectivity. We thus have a function
    $\inv f : Y\to X$. We verify
    \begin{itemize}
      \item $f\circ f^{-1}=\text{id}_Y$ as $f(\inv f(y)) = y$ by definition of $\inv f$;
      \item $\inv f\circ f=\text{id}_X$ as $\inv f(f(x)) = x$ as $x=x'$ is the unique
        solution to $f(x)=f(x')$.
    \end{itemize}
    Thus $f$ is an isomorphism in $\textbf{Set}$.

    Conversely, assume $f:X\to Y$ is an isomorphism in \textbf{Set}. Then there is an
    inverse $\inv f:Y\to X$. Since $f\circ \inv f=\text{id}_Y$, $f$ must be surjective.
    Similarly, it follows from $\inv f\circ f=\text{id}_X$ that $f$ must be injective, too.
  \end{proof}
\end{claim*}

\begin{claim*}[b]
  An arrow in \textbf{Mon} is an isomorphism if and only if it is a bijection.
  \begin{proof}
    Note that we have a forgetful functor $U:\textbf{Mon}\to\textbf{Set}$. For any
    isomorphism $f$ in \textbf{Mon} we have an isomorphism $Uf$ in \textbf{Set}. As
    shown above, $Uf$ must be a bijection so $f$ is, too. Conversely, if $f$ is a
    bijective monoid homomorphism then it must be an isomorphism due to reasoning
    analogous to the above.
  \end{proof}
\end{claim*}

\begin{claim*}[c]
  There exists a monotone bijection $f:X\to Y$ of posets that is not an isomorphism in the corresponding
  category.
  \begin{proof}
    Consider the two element posets $X=\cc{x_1,x_2}$ with $x_i\leq x_j$ if and only if $i=j$
    and $Y=\cc{y_1,y_2}$ with $y_1\leq y_2$. Now there is a monotone bijection
    $f:X\to Y$ given by $x_i\mapsto y_i$. If $f$ is to be an isomorphism then it must have
    an inverse which, moreover, agrees with its underlying function in \textbf{Set}.
    However, its inverse function is not monotone as $y_1\leq y_2$
    but $x_1\not\leq x_2$. Thus $f$ is not an isomorphism in the category of posets and
    monotone functions.
  \end{proof}
\end{claim*}

\section*{Question 2}

Consider the state space $\mathbb C^2\otimes\mathbb C^2$ in \textbf{Hilb}.

\begin{claim*}
  The states
  \begin{align*}
    \va{\phi}=\frac{1}{2}(\va{00}+\va{01}+\va{10}+\va{11}),\hs
    \va{\psi}=\frac{1}{2}(\va{00}-\va{01}-\va{10}+\va{11})
  \end{align*}
  are not entangled.
  \begin{proof}
    We note $\va{\phi}=(\va{0}+\va{1})\otimes(\va{0}+\va{1})$ and likewise
    $\va{\psi}=(\va{0}-\va{1})\otimes(\va{0}-\va{1})$.
  \end{proof}
\end{claim*}

\begin{claim*}
  The state $\va{\phi}=\frac{1}{2}(\va{00}+\va{01}+\va{10}-\va{11})$ is entangled
  \begin{proof}
    Assume there exist $\alpha,\beta\in\mathbb C^2$ such that
    $\va{\alpha}\otimes\va{\beta}=\va{\phi}$. Then, in particular, there are
    $a,b,c,d\in\mathbb C$ such that $(a\va{0}+b\va{1})\otimes(c\va{0}+d\va{1})=\va{\phi}$.
    We must have $ac>0$, $ad>0$, and $bc>0$. Thus either $a,b,c,d>0$ or $a,b,c,d<0$.
    This contradicts $bd<0$, which is also required.
  \end{proof}
\end{claim*}

\begin{claim*}
  The state $\va{\phi}=\frac{1}{2}(\va{00}+\va{01}-\va{10}+\va{11})$ is entangled.
  \begin{proof}
    Similar to the above assume $(a\va{0}+b\va{1})\otimes(c\va{0}+d\va{1})=\va{\phi}$,
    we obtain $ac>0$, $ad>0$, and $bd>0$. Thus either $a,b,c,d>0$ or $a,b,c,d<0$,
    contradicting $bc<0$.
  \end{proof}
\end{claim*}

Now consider the state space $\cc{0,1}\otimes\cc{0,1}$ in \textbf{Rel}.

\begin{claim*}
  The states $S=\cc{(0,0),(0,1)}$ and $R=\cc{(0,0),(0,1),(1,0),(1,1)}$ are not entangled.
  \begin{proof}
    We note $S=\cc{0}\times\cc{0,1}$ and $R=\cc{0,1}\times\cc{0,1}$.
  \end{proof}
\end{claim*}

\begin{claim*}
  The state $S=\cc{\rr{0,0},\rr{0,1},\rr{1,0}}$ is entangled.
  \begin{proof}
    Assume there are states $A,B$ of $\cc{0,1}$ such that $A\otimes B=S$.
    Then we have $(0,0)\in S$ and $(1,0)\in S$ so $0,1\in A$. Further,
    $(0,1)\in S$ so $1\in B$. However, $(1,1)\not\in S$. Contradiction.
  \end{proof}
\end{claim*}

\begin{claim*}
  The state $S=\cc{\rr{0,1},\rr{1,0}}$ is entangled.
  \begin{proof}
    Assume $S=A\otimes B$ for some $A$ and $B$. We must have $(0,1)\in S$ so
    $0\in A$. Similarly, $(1,0)\in S$ so $0\in B$. However, $(0,0)\not\in S$.
    Contradiction.
  \end{proof}
\end{claim*}

\section*{Question 3}

We say that two joint states $a,b:I\to A\otimes B$ are locally equivalent and write $a\sim b$
if and only if there exist invertible arrows $f:A\to A$ and $g:B\to B$ such that
$(f\otimes g)a = b$.

\begin{claim*}[a]
  The relation $\sim$ is an equivalence relation.
  \begin{proof}
    We verify the axioms:
    \begin{itemize}
      \item reflexivity: $(\text{id}\otimes\text{id})a=a$;
      \item symmetry: if $(f\otimes g)a=b$ then $\rr{\inv f\otimes\inv g}b=\inv{\rr{f\otimes g}}b=a$;
      \item transitivity: if $(f\otimes g)a=b$ and $(f'\otimes g')b=c$ then $\rr{f'f\otimes g'g}a=\rr{f'\otimes g'}\rr{f\otimes g}a=c$.
    \end{itemize}
  \end{proof}
\end{claim*}

\begin{answer*}[b]
  There are two automorphisms of $\cc{0,1}$ in \textbf{Rel}:
  \begin{enumerate}
    \item $\cc{\rr{0,0},\rr{1,1}}$;
    \item $\cc{\rr{0,1},\rr{1,0}}$.
  \end{enumerate}
  Both are their own inverse. We note that all isomorphisms in \textbf{Rel} must correspond to
  bijective functions so so these are in fact the only automorphisms.
\end{answer*}

\begin{answer*}[c]
  In \textbf{Rel} we identify states with subsets. Thus the states of $\cc{0,1}\times\cc{0,1}$
  are
  \begin{enumerate}
    \item $\emptyset$;
    \item $\cc{(0,0)}$;
    \item $\cc{(0,1)}$;
    \item $\cc{\rr{1,0}}$;
    \item $\cc{\rr{1,1}}$;
    \item $\cc{\rr{0,0},\rr{0,1}}$;
    \item $\cc{\rr{0,0},\rr{1,0}}$;
    \item $\cc{\rr{0,0},\rr{1,1}}$;
    \item $\cc{\rr{0,1},\rr{1,1}}$;
    \item $\cc{\rr{0,1},\rr{1,0}}$;
    \item $\cc{\rr{1,0},\rr{1,1}}$;
    \item $\cc{\rr{0,0},\rr{0,1},\rr{1,0}}$;
    \item $\cc{\rr{0,0},\rr{0,1},\rr{1,1}}$;
    \item $\cc{\rr{0,0},\rr{1,0},\rr{1,1}}$;
    \item $\cc{\rr{0,1},\rr{1,0},\rr{1,1}}$;
    \item $\cc{\rr{0,0},\rr{0,1},\rr{1,0},\rr{1,1}}$.
  \end{enumerate}
\end{answer*}

\begin{answer*}[d]
  We have the following equivalence classes:
  \begin{enumerate}
    \item $\cc{\emptyset}$;
    \item $\cc{\cc{\rr{0,0}},\cc{\rr{0,1}},\cc{\rr{1,0}},\cc{\rr{1,1}}}$;
    \item $\cc{\cc{\rr{0,0},\rr{0,1}},\cc{\rr{1,0},\rr{1,1}}}$;
    \item $\cc{\cc{\rr{0,0},\rr{1,0}},\cc{\rr{0,1},\rr{1,1}}}$;
    \item $\cc{\cc{\rr{0,0},\rr{1,1}},\cc{\rr{0,1},\rr{1,0}}}$;
    \item $\cc{\cc{\rr{0,0},\rr{0,1},\rr{1,0}},
        \cc{\rr{0,0},\rr{0,1},\rr{1,1}},
        \cc{\rr{0,0},\rr{1,0},\rr{1,1}},
      \cc{\rr{0,1},\rr{1,0},\rr{1,1}}}$;
    \item $\cc{\cc{\rr{0,0},\rr{0,1},\rr{1,0},\rr{1,1}}}$.
  \end{enumerate}
  Of these, only 5 and 6 are entangled. This can be seen as follows: if
  we have locally equivalent states $p\sim q$ such that $p=a\otimes b$
  then $fa\otimes gb$ for some $f,g$. Thus $\sim$ preserves entanglement.
  We now note $\emptyset=\emptyset\otimes\emptyset$,
  $\cc{(0,0)}=\cc{0}\otimes\cc{0}$, $\cc{(0,0),(0,1)}=\cc{0}\otimes\cc{0,1}$,
  $\cc{(0,0),(1,0)}=\cc{0,1}\otimes\cc{0}$, and
  $\cc{(0,0),(0,1),(1,0),(1,1)}=\cc{0,1}\otimes\cc{0,1}$.
  The first listed states in 5 and 6 were previously proven to be entangled.
\end{answer*}

\section*{Question 4}

Consider the $\mathcal H=\mathbb C^2\otimes\mathbb C^2$ in \textbf{Hilb}.

\begin{claim*}[a]
  A state $\phi : \mathbb C\to \mathcal H$ is entangled if and only if its corresponding
  matrix $M_\phi$ is invertible.
  \begin{proof}
    Fix $\phi$. Assume $M_\phi$ is not invertible. Then there exists $\lambda\in\mathbb C$
    such that $a=\lambda c$ and $b=\lambda d$. In particular,
    \begin{align*}
      \phi
      = a\va{00}+b\va{10}+\lambda a\va{01}+\lambda b\va{11}
      = (a\va{0}+b\va{1})\otimes(\va{0}+\lambda\va{1})
    \end{align*}
    Conversely, if $\phi$ is not entangled then
    $\phi = (a\va{0}+b\va{1})\otimes(c\va{0}+d\va{1})$. Thus
    \begin{align*}
      M_\phi =
      \begin{pmatrix}
        ac & bc \\
        ad & bd
      \end{pmatrix}
    \end{align*}
    Clearly, $ac\va{0}+bc\va{1}=(c/d)(ad\va{0}+bd\va{1})$. Thus
    $\text{rank}(M_\phi) < 2$, i.e. $M_\phi$ is not invertible.
  \end{proof}
\end{claim*}

\begin{claim*}[b]
  Let $\phi$ be a state of $\mathbb C^2\otimes\mathbb C^2$ and let
  $f:\mathbb C^2\to\mathbb C^2$ be a linear map.
  Then $M_{(\mathbb C^2\otimes f)\circ\phi} =M_\phi\circ f^\top$.
  \begin{proof}
    Denote by $M_f$ the matrix corresponding to $f$ and let
    \begin{align*}
      M_\phi =
      \begin{pmatrix}
        a & b \\
        c & d
      \end{pmatrix},\hs
      M_f =
      \begin{pmatrix}
        e & f \\
        g & h
      \end{pmatrix}.
    \end{align*}
    Then we compute
    \begin{align*}
      (\text{id}\otimes f)\circ\phi
      &= (\text{id}\otimes f)(a\va{00}+b\va{01}+c\va{10}+d\va{11})\\
      &= \va{0}\otimes M_f(a\va{0}+b\va{1})+\va{1}\otimes M_f(c\va{0}+d\va{1})\\
      &= \va{0}\otimes((ae+bf)\va{0}+(ag+bh)\va{1})+
      \va{1}\otimes((ce+df)\va{0}+(cg+dh)\va{1}).
    \end{align*}
    Equivalently
    \begin{align*}
      M_{(\text{id} \otimes f)\circ\phi} =
      \begin{pmatrix}
        ae+bf & ag+bh \\
        ce+df & cg+dh
      \end{pmatrix}.
    \end{align*}
    We then notice
    \begin{align*}
      M_\phi M_f^\top =
      \begin{pmatrix}
        a & b \\
        c & d
      \end{pmatrix}
      \begin{pmatrix}
        e & g \\
        f & h
      \end{pmatrix}
      =
      \begin{pmatrix}
        ae+bf & ag+bh \\
        ce+df & cg+dh
      \end{pmatrix}
    \end{align*}
  \end{proof}
\end{claim*}

\begin{claim*}[c]
  There are three families of locally equivalent states of $\mathcal H$.
  \begin{proof}
    The first family contains only the zero state $0\in\mathbb C^2\otimes\mathbb C^2$.
    We note that, for any isomorphism $f:\mathcal H\to\mathcal H$,
    $f(0)=0$. Thus $0$ cannot be locally equivalent to any state but itself.

    The second family is given by all the non-zero product states. Any two
    non-zero product states are
    clearly isomorphic as, for all non-zero states $\va{\phi},\va{\psi}\in\mathbb C^2$, there is
    an isomorphism $f:\mathbb C^2\to\mathbb C^2$ such that $f(\phi)=\psi$. Thus,
    for all product states $\va{\phi}\otimes\va{\phi'}$ and $\va{\psi}\otimes\va{\psi'}$,
    there are isomorphisms $f,f':\mathbb C^2\to\mathbb C^2$ such that
    $(f\otimes f')(\va{\phi}\otimes\va{\phi'}) = \va{\psi}\otimes\va{\psi'}$.

    Thirdly, all entangled states are locally equivalent: Let $\va{\phi},\va{\psi}\in\mathcal H$
    be entangled. Then $M_\phi$ and $M_\psi$ are invertible so
    $M_\phi (\inv M_\phi M_\psi) = M_\psi$. In particular, let $f:\mathbb C^2\to\mathbb C^2$
    be the isomorphism corresponding to the invertible matrix $(\inv M_\phi M_\psi)^\top$. Then
    $(\text{id}\otimes f)\circ \phi = \psi$, i.e. $\phi\sim\psi$.
  \end{proof}
\end{claim*}

\end{document}
