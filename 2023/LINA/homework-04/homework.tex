\documentclass{article}
\usepackage{homework-preamble}

\begin{document}
\title{Linear Analysis: Homework 4}
\author{Franz Miltz (UUN: S1971811)}
\date{8 November 2022}
\maketitle

For every $n\geq 1$, define
\begin{align*}
  T_n:\ell^2 &\to\ell^2 \\
  \rr{x_1,x_2,\ldots,x_n,\ldots} &\mapsto \rr{x_1,x_2,\ldots,x_n,0,0,\ldots}.
\end{align*}

\begin{claim*}[3a]
  For all $n\geq 1$, $\vabs{T_n}=1$.
  \begin{proof}
    Fix $n\geq 1$. We have
    \begin{align*}
      \vabs{T_n} = \sup_{\vabs{x}=1} \vabs{T_n{x}}_{\ell^2} = \sup_{\vabs{x}_{\ell^2}=1} \rr{\sum_{i=1}^{n} \abs{x_i}^2 }^{1/2} \leq \sup_{\vabs{x}_{\ell^2}=1} \vabs{x}_{\ell^2} = 1.
    \end{align*}
    This bound is achieved by $x=\rr{1,0,0,\ldots,0,\ldots}\in\ell^2$ for all $n\geq 1$.
  \end{proof}
\end{claim*}

\begin{claim*}[3b]
  For every $x\in\ell^2$, $T_nx\to x$ in $\ell^2$.
  \begin{proof}
    Fix $x=\rr{x_i}_{i\in\N}\in\ell^2$. We note that
    \begin{align*}
      \vabs{x}=\rr{\sum_{i=1}^{\infty} \abs{x_i}}^{1/2} < \infty
    \end{align*}
    and thus
    \begin{align*}
      \sum_{i=1}^{\infty} \abs{x_i} < \infty.
    \end{align*}
    Thus, for all $\e > 0$, there exists $N\in\N$ such that, for all $n>N$,
    \begin{align*}
      \sum_{i=n}^{\infty} \abs{x_i} < \e.
    \end{align*}
    Now let $\e > 0$. Choose $N\in\N$ such that, for all $n>N$,
    \begin{align*}
      \sum_{i=n}^{\infty} \abs{x_i} < \e^2.
    \end{align*}
    Then, for all $n>N$,
    \begin{align*}
      \vabs{T_n x - x}_{\ell^2}= \rr{\sum_{i=n}^{\infty} \abs{x_i}^2}^{1/2} < e
    \end{align*}
    as required. Thus $T_nx\to x$ in $\ell^2$.
  \end{proof}
\end{claim*}

\begin{claim*}[3c]
  $\rr{T_n}_{n\in\N}$ does not converge in $\mathcal L\rr{\ell^2}$.
  \begin{proof}
    Assume $\rr{T_n}_{n\in\N}$ converges in $\mathcal L\rr{\ell^2}$. Then $\rr{T_n}_{n\in\N}$
    is Cauchy. Let $\e=1/2$ and choose $N\in\N$ such that whenever $m,n>N$,
    \begin{align*}
      \vabs{T_m-T_n}<\e.
    \end{align*}
    In particular, note, for all $m,n>N$,
    \begin{align}
      \label{eq:sup-leq-e}
      \vabs{T_m-T_n} = \sup_{\vabs{x}_{\ell^2}=1} \vabs{T_m x-T_n x}_{\ell^2} < \e.
    \end{align}
    Let $m=N+1$ and $n=N+2$. Then let $x=\rr{x_i}_{i\in\N}\in\ell^2$ such that $x_i=1$ if
    $i=n$ and $x_i=0$ otherwise. Then note $\vabs{x}_{\ell^2}=1$ but
    \begin{align*}
      \vabs{T_m x-T_n x}_{\ell^2} = \abs{x_n} = 1 > \e.
    \end{align*}
    This contradicts (\ref{eq:sup-leq-e}) so $\rr{T_n}_{n\in\N}$ must not converge in $\mathcal L\rr{\ell^2}$.
  \end{proof}
\end{claim*}

\begin{claim*}[6]
  Let $X$ and $Y$ be normed linear spaces. Let $T:X\to Y$ be a bounded linear
  operator. Then
  \begin{align*}
    \vabs{T}=\inf\cc{M\geq 0 : \forall x\in X.\: \vabs{Tx}\leq M\vabs{x}}.
  \end{align*}
  \begin{proof}
    Define
    \begin{align*}
      M_0=\inf\cc{M\geq 0 : \forall x\in X.\: \vabs{Tx}\leq M\vabs{x}}.
    \end{align*}
    Now, note that, for all $x\in X$,
    \begin{align*}
      \vabs{Tx}\leq \vabs{T}\vabs{x}
    \end{align*}
    so
    \begin{align*}
      \vabs{T}\in \cc{M\geq 0 : \forall x\in X.\: \vabs{Tx}\leq M\vabs{x}}.
    \end{align*}
    Thus
    \begin{align*}
      \vabs{T}\geq M_0.
    \end{align*}
    Further, assume
    \begin{align*}
      \vabs{T}>\inf\cc{M\geq 0 : \forall x\in X.\: \vabs{Tx}\leq M\vabs{x}}.
    \end{align*}
    Then there exists $M\geq 0$ such that $M<\vabs{T}$ and, for all $x\in X$,
    \begin{align*}
      \vabs{Tx}\leq M\vabs{x}.
    \end{align*}
    In particular, for all $x\in X$ with $\vabs{x}=1$,
    \begin{align*}
      \vabs{Tx}\leq M < \vabs{T}.
    \end{align*}
    This contradicts the fact that
    \begin{align*}
      \vabs{T}=\sup_{\vabs{x}=1} \vabs{Tx}.
    \end{align*}
    Thus
    \begin{align*}
      \vabs{T}=\inf\cc{M\geq 0 : \forall x\in X.\: \vabs{Tx}\leq M\vabs{x}}.
    \end{align*}
  \end{proof}
\end{claim*}

\end{document}
