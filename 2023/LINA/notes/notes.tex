\documentclass{article}
\usepackage{notes-preamble}
\usepackage{enumitem}
\begin{document}
\mkthmstwounified
\title{Linear Analysis (SEM7)}
\author{Franz Miltz}
\maketitle
\tableofcontents
\pagebreak

\section{Normed linear spaces, Banach spaces}
\label{sec:definitions}

Let $\F\in \cc{\R,\C}$.

\begin{definition}[Normed linear space]
  \label{def:nls}
  Let $X$ be an $\F$-vector space.
  A function $ \vabs{-} : X \to \R$ is defined to be a norm on $X$ iff the following hold
  \begin{enumerate}
    \item $\forall x\in X.\: \vabs{x}\geq 0$ and $ \vabs{x} = 0$ iff $x = 0$.
    \item $\forall \lambda\in\F.\:\forall x \in X.\: \vabs{\lambda x} = \abs{\lambda} \vabs{x}$.
    \item $\forall x,y\in X.\: \vabs{x + y} \leq \vabs{x} + \vabs{y}$
  \end{enumerate}
  A pair $ \rr{X, \vabs{-}}$ is a normed linear space iff $X$ is a vector space and $ \vabs{-} $
  is a norm on $X$.
\end{definition}


\begin{theorem}
  \label{thm:symmetry}
  Let $X$ be a normed linear space. For $x,y\in X$, $ \vabs{x - y} = \vabs{y - x} $.
\end{theorem}


\begin{definition}
  \label{def:convergence}
  Let $X$ be a normed linear space. A sequence is defined to be a function $\N\to X$. A
  sequence $ \rr{x_n}_{n\in\N}$ is defined to converge to $x\in X$ iff, for all $\e > 0$,
  there exists $N\in\N$ such that, for all $n>N$, $ \vabs{x_n - x} < \e$.
\end{definition}


\begin{theorem}
  \label{thm:uniquenes-of-the-limit}
  Let $X$ be a normed linear space. A sequence $ \rr{x_n}_{n\in\N}$ converges to at most one point.
\end{theorem}


\begin{definition}
  \label{def:cauchy}
  Let $X$ be a normed linear space. A sequence $ \rr{x_n}_{n\in\N}$ is defined to be Cauchy iff,
  for all $\e>0$, there exists $N\in\N$ such that, for all $m,n>\N$, $ \vabs{x_m - x_n} < \e$.
\end{definition}


\begin{theorem}
  \label{thm:convergent-implies-cauchy}
  Let $X$ be a normed linear space. If a sequence is convergent then it is Cauchy.
\end{theorem}


\begin{definition}
  \label{def:complete}
  Let $X$ be a normed linear space. A set $E\subseteq X$ is defined to be complete iff every Cauchy
  sequence in $E$ is convergent.
\end{definition}


\begin{definition}
  \label{def:banach}
  A pair $ \rr{X, \vabs{-}} $ is defined to be a Banach space iff it is a complete normed linear space.
\end{definition}


\section{Inequalities}
\label{sec:ineqaulities}


\begin{theorem}
  \label{thm:gagm}
  If $A,B\geq 0$ and $0\leq\theta\leq 1$ then
  \begin{align*}
    A^\theta B^{1-\theta} \leq \theta A + \rr{1-\theta}B.
  \end{align*}
\end{theorem}

\begin{definition}
  \label{def:holder-conjugate-exponent}
  For $p\in \rr{1,\infty}$ the H\"older conjuate exponent of $p$ is defined to be $p/ \rr{p-1}$.
  For $p=1$, the H\"older conjugate exponent is defined to be $\infty$. For $p=\infty$, the
  H\"older conjugate exponent is defined to be $1$.
\end{definition}

\begin{theorem}
  \label{thm:hce-equals-one}
  If $p\in \rr{1,\infty}$, and $q$ is the H\"older conjugate exponent of $p$ then
  \begin{align*}
    1 = \frac{1}{p} + \frac{1}{q}.
  \end{align*}
\end{theorem}


\begin{theorem}
  \label{thm:holder-inequality-finite}
  Let $p\in \br{0,\infty}$, and let $q$ be the H\"older conjugate exponent of $p$.

  For any two finite $\F$-values sequences $ \rr{a_1, ..., a_n}$ and $ \rr{b_1, ..., b_n}$,
  \begin{align*}
    \abs{\sum_{i=1}^n a_i b_i}
    \leq \rr{\sum_{i=1}^n \abs{a_j}^p}^{1/p} \rr{\sum_{i=1}^n \abs{b_j}^p}^{1/p}
  \end{align*}
\end{theorem}

\begin{theorem}
  \label{thm:minkowski-for-p-infinite}
  For any two $\F$-values sequences $ \rr{a_n}_{n\in\N}$ and $ \rr{b_n}_{n\in\N}$,
  \begin{align*}
    \max_i \abs{a_i + b_i} \leq \max_i \abs{a_i} + \max_i \abs{b_i}.
  \end{align*}
\end{theorem}

\begin{theorem}
  \label{thm:dl-norms}
  Let $p\in \bb{1,\infty}$.
  \begin{enumerate}
    \item Let $ \vabs{-}_{\ell^p} : \ell^p \to \R$ be defined by \begin{align*}
        \vabs{x}_{\ell^p} = \rr{\sum_{n}^{} \abs{x_n}^p}^{1/p}.
      \end{align*}
      Then $ \rr{\ell^p, \vabs{-}_{\ell^p}}$ is a normed linear space.
    \item Let $ \vabs{-}_{d_p} : \F^n \to \R$ be defined by \begin{align*}
        \vabs{x}_{d_p} = \rr{\sum_{i=1}^{n} \abs{x_i}^p}^{1/p}.
      \end{align*}
      Then $ \rr{\F^n, \vabs{-}_{d_p}} $ is a normed linear space.
  \end{enumerate}
\end{theorem}

\section{Topology} \label{sec:topology}

\begin{definition}
  \label{def:open}
  Let $X$ be a normed linear space. Given $x\in X$ and $r>0$, the open ball with radius
  $r$ and centre $x$ is defined to be
  \begin{align*}
    B\rr{x,r}=\cc{y\in X:\vabs{y-x}<r}.
  \end{align*}
  The nunit ball is $B\rr{0,1}$.

  A subset $G\subseteq X$ is open iff for all $x\in X$ there is an $r>0$ such that
  $B\rr{x,r}\subseteq G$. A subset $F\subseteq X$ is closed iff $F^C$ is open.
\end{definition}

\begin{theorem}
  Let $X$ be a normed linear space.
  \begin{enumerate}
    \item $\emptyset$ and $X$ are open.
    \item The union of any collection of open sets is open.
    \item The intersection of a finite number of open sets is open.
  \end{enumerate}
\end{theorem}

\begin{theorem}
  Let $X$ be a normed linear space. Let $x\in X$ and $r>0$.
  \begin{enumerate}
    \item $B\rr{x,r}$ is open.
    \item $x\in B\rr{x,r}$.
    \item $\cc{y\in X : \vabs{y-x}\leq r}$ is closed.
  \end{enumerate}
\end{theorem}

\begin{theorem}
  Let $X$ be a normed linear space. Let $F\subseteq X$. $F$ is closed iff every sequence of
  elements in $F$, convergent in $X$, has its limit in $F$.
\end{theorem}

\begin{theorem}
  Let $X$ be a Banach space. Let $F\subseteq X$. $F$ is clossed iff every Cauchy sequence in
  $F$ has a limit in $F$.
\end{theorem}

\begin{theorem}
  Let $X$ be a Banach space. Let $W$ be a subspace of $X$. $W$ is a Banach space iff it is
  closed.
\end{theorem}

\begin{definition}
  \label{def:closure}
  Let $X$ be a normed linear space. Let $W\subseteq X$. The closure of $W$ is
  \begin{align*}
    \overline W = \bigcap \cc{F\subseteq X : W\subseteq F, \text{$F$ closed}}
  \end{align*}
\end{definition}

\begin{definition}
  Let $\rr{X, \vabs{-}_{X}}$ and $\rr{Y, \vabs{-}_{Y}}$ be normed linear spaces. Let $E,F\subseteq X$.

  A set $E\subseteq F$ is defined to be dense in $F$ iff, for all $f\in F$ and for all
  $\e > 0$, there exists an $e\in E$ such that $\vabs{e-f}_{X}<\e$. A set $E$ is defined
  to be dense if it is dense in $X$.

  A linear map $T:X\to Y$ is defined to be an isometry iff, for all $x\in X$,
  $\vabs{Tx}_{Y}=\vabs{x}_{X}$. $X$ and $Y$ are defined to be isometrically isomorphic iff
  there exists an onto isometry $T:X\to Y$.

  A Banach space completion of $\rr{X,\vabs{-}_{X}}$ is a pair $\rr{\rr{Y,\vabs{-}_{Y}}}$
  such that $\rr{Y,\vabs{-}_{Y}}$ is a Banach space, $T:X\to Y$ is an isometry, and $T\rr{X}$
  is dense in $Y$.
\end{definition}

\begin{theorem}[Existence of unique completion]
  Let $X$ be a normed linear space.

  There exists a Banach space completion of $X$.

  If $\rr{\rr{Y,\vabs{-}_{Y}}, T_Y}$ and $\rr{\rr{Z,\vabs{-}_{Z}}, T_Z}$ are Banach space
  completion of $X$, then $\rr{Y,\vabs{-}_{Y}}$ and $\rr{Z,\vabs{-}_{Z}}$ are isometrically
  isomorphic.
\end{theorem}

\section{Compactness}
\label{sec:compactness}

\begin{definition}
  Let $X$ be a normed linear space. Let $E\subseteq X$. A cover $E$ is defined to be a
  collection of sets $\cc{U_\alpha}_{\alpha\in A}$ such that
  $E\subseteq\bigcup_{\alpha\in A}U_\alpha$. Given a cover $\cc{U_\alpha}_{\alpha\in A}$
  is defined to be a subset $A'\subseteq A$ such that $\cc{U_\alpha}_{\alpha\in A'}$ is also
  a cover of $E$. An open cover is a cover $\cc{U_\alpha}_{\alpha\in A}$ such that each
  $U_\alpha$ is open.

  The set $E$ is compact iff every open cover of $E$ has a finite subcover.
\end{definition}

\begin{theorem}[Heine-Borel in $\R$]
  Every closed and bounded interval is compact in $\R$.
\end{theorem}

\begin{theorem}
  Let $\rr{X,\vabs{-}_{X}}$ and $\rr{Y,\vabs{-}_{Y}}$ be normed linear spaces. A function
  $f:X\to Y$ is continuous at $x_0\in X$ iff, for lal $\e > 0$, there exists a $\delta > 0$,
  such that, for all $x\in X$, $\vabs{x-x_0}_{X}<\delta$ implies
  $\vabs{f\rr{x}-f\rr{x_0}}_{Y}<\e$.
\end{theorem}

\begin{theorem}[Extreme value theorem]
  Let $X$ be a normed linear space and $f:X\to\R$ be a function. If $K\subseteq X$ is compact
  and $f$ is continuous, then $f$ has a maximum and a minimum on $K$.
\end{theorem}

\begin{theorem}
  Let $X$ be a normed linear space. Any closed subset of a compact set is compact.
\end{theorem}

\begin{definition}
  \label{def:bounded-set}
  Let $X$ be a normed linear space. Let $E\subseteq X$. $E$ is a bounded set iff there is an
  $r>0$ such that $E\subseteq B\rr{0,r}$.
\end{definition}

\begin{lemma}
  Let $X$ be a normed linear space.
  \begin{enumerate}
    \item If a Cauchy sequence has a subsequential limit, then it converges.
    \item Any subsequence of a convergent sequence converges to the same point.
  \end{enumerate}
\end{lemma}

\begin{theorem}
  Let $X$ be a normed linear space. Let $E\subseteq X$. If $E$ is compact then it is closed and bounded.
\end{theorem}

\begin{definition}
  Let $X$ be a finite-dimensional vector space. Let $\mathcal B=\rr{e_1,...,e_n}$ be
  an ordered basis for $X$.

  The $\vabs{-}_{d_\infty,\mathcal B}$ norm is given by
  \begin{align*}
    \vabs{x}_{d_\infty,\mathcal B}=\max_i \abs{x_i},
  \end{align*}
  where the $\rr{x_i}_{i=1}^n$ are the coefficients such that
  \begin{align*}
    x = \sum_{i=1}^{n} x_i e_i.
  \end{align*}
\end{definition}

\begin{lemma}[Heine-Borel in $\rr{X,\vabs{-}_{d_\infty,\mathcal B}}$]
  Let $X$ be a finite-dimensional vector space with an ordered basis $\mathcal B = \rr{e_1, ..., e_n}$.

  In $\rr{X,\vabs{-}_{d_\infty,\mathcal B}}$ there is the following equivalence: $E\subseteq X$
  is compact iff $E$ is closed and bounded.
\end{lemma}

\section{Equivalence}
\label{sec:equivalence}

\begin{definition}
  \label{def:equivalence}
  Let $X$ be an $F$-vector space. Let $\vabs{-}_{1}$ andd $\vabs{-}_{2}$ be norms on $X$.

  Then $\vabs{-}_{1}$ and $\vabs{-}_{2}$ are equivalent iff there is a constant $A>0$ such
  that, for all $x\in X$,
  \begin{align*}
    \vabs{x}_{1}\leq A\vabs{x}_{2}, \hs
    \vabs{x}_{2}\leq A\vabs{x}_{1}.
  \end{align*}
  A is called an equivalence constant.
\end{definition}

\begin{lemma}
  Let $X$ be an $F$-vector space. Let $\vabs{-}_{1}$ and $\vabs{-}_{2}$ be equivalent norms
  and $A>0$ an equivalence constant.

  For all $x\in X$ and $r>0$,
  \begin{align*}
    B^1\rr{x,r}\subseteq B^2\rr{x,Ar}, \hs
    B^2\rr{x,r}\subseteq B^1\rr{x,Ar}.
  \end{align*}
\end{lemma}

\begin{lemma}
  Let $X$ be an $F$-vector space and let $\vabs{-}_{1},\vabs{-}_{2}$ be equivalent norms
  on $X$. Then the each of the following statements is true with respect to $\vabs{-}_{1}$ iff it is true with respect to $\vabs{-}_{2}$.
  \begin{enumerate}
    \item $\rr{n}_{n\in\N}$ converges to $x$.
    \item $\rr{n}_{n\in\N}$ is Cauchy.
    \item $X$ is Banach.
    \item $G\subseteq X$ is open.
    \item $F\subseteq X$ is closed.
    \item $K\subseteq X$ is compact.
  \end{enumerate}
\end{lemma}

\begin{lemma}
  Let $X$ be a vector space. Equivalence of norms is an equivalence relation on the set of
  norms on $X$.
\end{lemma}

\begin{theorem}
  Let $X$ be a finite-dimensional $F$-vector space. All norms over $X$ are equivalent.
\end{theorem}

\begin{theorem}[Heine-Borel in finite-dimensional spaces]
  Let $\rr{X, \vabs{-}}$ be a normed linear space.

  If $X$ is finite dimensional then there is the following equivalence: $K\subseteq X$
  is compact iff it is closed and bounded.
\end{theorem}

\begin{theorem}
  Let $X$ be a normed linear space. The unit sphere is compact iff $X$ is finite dimensional.
\end{theorem}

\section{Hilbert spaces}
\label{sec:hilbert-spaces}

\begin{definition}
  \label{def:inner-product}
  Let $X$ be an $F$-vector space. A map $\aa{-,-}:X\times X\to F$ is an inner product iff
  the following hold:
  \begin{enumerate}
    \item For all $x\in X$, $\aa{x,x}\geq 0$ and $\aa{x,x}=0$ iff $x=0$.
    \item For all $x,y\in X$, $\aa{x,y}=\overline{\aa{y,x}}$.
    \item For all $\lambda,\mu\in F$ and for all $x,y,z\in X$, $\aa{\lambda x+\mu y,z}=\lambda\aa{x,z}+\mu\aa{y,z}$.
  \end{enumerate}
  An inner-product space is a pair $\rr{X,\aa{-,-}}$ such that $X$ is a vector space and
  $\aa{-,-}$ is an inner product on $X$.
\end{definition}

\begin{definition}
  \label{def:linear}
  Let $X$ be a vector space. A function $f:X\to F$ is
  \begin{enumerate}
    \item linear iff, for all $\lambda,\mu\in F$ and for all $x,y\in X$, $f\rr{\lambda x+\mu y}=\lambda f\rr{x}+\mu f\rr{y}$;
    \item antilinear iff, for all $\lambda,\mu\in F$ and for all $x,y\in X$, $f\rr{\lambda x+\mu y}=\overline\lambda f\rr{x}+\overline\mu f\rr{y}$;
  \end{enumerate}
  A function $\aa{-,-}:X\times X\to F$ is
  \begin{enumerate}
    \item bilinear if it is linear in each argument;
    \item sesquilinear if it is linear in its first argument and antilinear in its second argument.
  \end{enumerate}
\end{definition}


\begin{theorem}
  Let $\rr{X,\aa{-,-}}$ be an inner product space. The map $\vabs{-}:x\mapsto \aa{x,x}^{1/2}$ defines a norm.
\end{theorem}

\begin{theorem}[Cauchy-Schwarz]
  Let $X$ be an inner product space. For all $x,y\in X$,
  \begin{align*}
    \abs{\aa{x,y}}\leq \vabs{x}\vabs{y}
  \end{align*}
\end{theorem}

\begin{theorem}
  Let $X$ be an inner product space. Let $x,y\in X$.
  \begin{enumerate}
    \item $\vabs{x+y}^2+\vabs{x-y}^2= 2\rr{\vabs{x}^2+\vabs{y}^2}$.
    \item If $F=\R$ then $4\aa{x,y}=\vabs{x+y}^2 - \vabs{x-y}^2$.
    \item If $F=\C$ then $r\aa{x,y}=\sum_{k=1}^4 i^k \vabs{x+i^ky}^2$.
  \end{enumerate}
\end{theorem}

\section{Orthogonality}\label{sec:orthogonality}

\begin{lemma}
  Let $X,Y,Z$ be normed linear spaces.
  \begin{enumerate}
    \item If $f:X\to Y$ is continuous and $\rr{n}_{n\in\N}$ converges to $x\in X$ then $\rr{f\rr{x_n}}_{n\in\N}$
      converges to $f\rr{x}\in Y$.
    \item If $f:X\to Y$ and $g:Y\to Z$ are continuous then $g\circ f:X\to Z$ is continuous.
    \item The set of continuous functions from $X$ to $Y$ is a vector space with pointwise
      addition and scalar multiplication.
  \end{enumerate}
\end{lemma}

\begin{lemma}
  Let $X$ be an inner-product space. Let $z\in X$. The maps $x\mapsto\aa{x,z}$ and
  $x\mapsto\aa{z,x}$ are continuous.
\end{lemma}

\begin{definition}\label{def:convex-set}
  Let $X$ be a normed linear space. A set $S\subseteq X$ is defined to be convex if, for all
  $s,t\in S$ and all $\lambda\in\bb{0,1}$, $\lambda s + \rr{1-\lambda}t\in S$.
\end{definition}

\begin{definition}\label{def:orthogonality}
  Let $X$ be an inner product space. Let $x\in X$ and let $S\subseteq X$. A vector $y\in X$
  is orthogonal to $x$ iff $\aa{x,y}=0$.

  The orthogonal complement to $S$ is
  \begin{align*}
    S^\bot = \cc{x\in X : \forall s\in S. \aa{s,x}=0}.
  \end{align*}
\end{definition}

\begin{lemma}
  Let $X$ be an inner-product space.
  \begin{enumerate}
    \item $x$ is orthogonal to $y$ iff $y$ is orthognal to $x$.
    \item If $S\subseteq X$ then $S^\bot=\cc{x\in X : \forall s\in S. \aa{x,s}=0 }$.
  \end{enumerate}
\end{lemma}

\begin{lemma}
  Let $X$ be an inner-product space. Let $M\subseteq X$.
  \begin{enumerate}
    \item $M^\bot$ is a closed subspace.
    \item $M\subseteq M^{\bot\bot}$.
    \item $\rr{\overline{M}}^\bot=M^\bot$.
    \item If $M$ is a subspace, $M\cap M^\bot = \cc{0}$.
  \end{enumerate}
\end{lemma}

\begin{theorem}
  Let $X$ be an inner-product space. Let $W$ be a closed convex subset of $X$. Let
  $x\in X$. If $W$ is nonempty then there exists $w\in W$ such that
  \begin{align*}
    \vabs{w-x}=\inf_{w'\in W}\vabs{w'-x}.
  \end{align*}
\end{theorem}

\begin{definition}\label{def:orthogonal-projection}
  Let $X$ be an inner-product space. Let $W$ be a closed subspace. The orthognal projection
  $P_W$ is defined to be the map $P_W:X\to W$ such that, for all $x\in X$, $P_W\rr{x}$
  is the unique closest point in $W$.
\end{definition}

\begin{theorem}
  Let $X$ be an inner-product space. Let $W$ be a closed subspace.
  \begin{enumerate}
    \item For all $x\in X$, there is a unique $y\in W$ and $z\in W^\bot$ such that
      $x=y+z$; furthermore, $y=P_W\rr{x}$ and $z=P_{W^\bot}\rr{x}$.
    \item For all $x\in X$, $P_{W^\bot}\rr{x}=x-P_{W^\bot}\rr{x}$.
    \item $W^{\bot\bot}=W$.
  \end{enumerate}
\end{theorem}

\begin{theorem}
  Let $X$ be an inner-product space and let $W\subseteq X$ be a subspace. $\overline W = W^{\bot\bot}$.
\end{theorem}

\section{Orthonormal sets}\label{sec:orthonormal-sets}

\begin{definition}
  Let $X$ be an inner-product space. A set $S$ is defined to be orthonormal if, for all
  $x,y\in S$, $\aa{x,y}=1$ if $x=y$ and $0$ otherwise.
\end{definition}

\begin{lemma}
  Let $X$ be an inner-product space. If $S$ is an orthonormal set then it is linearly
  independent.
\end{lemma}

\begin{theorem}
  Let $X$ be an inner-product space. Let $N\in\Z_{>0}\cup \cc{\infty}$. If $\rr{v_i}_{i=1}^N$
  is an ordered set of linearly independent vectors in $X$ then there exists an ordered set
  $\rr{w_i}_{i=1}^N$ with the properties that
  \begin{enumerate}
    \item $w_1=\frac{q}{\vabs{v_1}}v_1$.
    \item For $i>1$, \begin{align*}
        w_i = \frac{q}{v_i-P_{\text{span}\cc{v_1,...,v_{i-1}}}\rr{x}}\rr{v_i-P_{\text{span}\cc{v_1,...,v_{i-1}}}\rr{v_i}}.
      \end{align*}
    \item For all $i\leq N$, $\text{span}\cc{w_1,...,w_i}=\text{span}\cc{v_1,...,v_i}$.
    \item $\cc{w_i}_{i=1}^N$ is an orthonormal set.
  \end{enumerate}
\end{theorem}

\begin{definition}
  Let $X$ be a normed linear space. Let $\rr{n}_{n\in\N}$ be a sequence in $X$. The
  series $\sum_{i=1}^{\infty} x_i$ is defined to mean $\lim_{n\to\infty}\sum_{i=1}^{n} x_i$.
\end{definition}

\begin{lemma}
  Let $X$ be an inner-product space. Let $y\in X$. Let $\rr{x_n}_{n\in\N}$ be a sequence
  in $X$. If $\sum_{n=1}^{\infty} x_n$ converges then
  \begin{align*}
    \aa{\sum_{n=1}^{\infty} x_n,y}=\sum_{n=1}^{\infty} \aa{x_n,y}, \hs
    \aa{\sum_{n=1}^{\infty} y,x_n}=\sum_{n=1}^{\infty} \aa{y,x_n}.
  \end{align*}
\end{lemma}

\begin{theorem}
  Let $X$ be an inner-product space. Let $N\in\Z_{>0}\cup\cc{\infty}$. Let $\rr{e_n}_{n=1}^N$
  be an ordered orthonormal set.
  \begin{enumerate}
    \item If $x=\sum_{n=1}^{N} \alpha_n e_n$ then $\aa{x,e_n}=\alpha_n$.
    \item For all $x\in X$, \begin{align*}
        P_{\overline{\text{span}\cc{e_n}_{n=1}^N}}\rr{x}=\sum_{n=1}^{N} \aa{x,e_n}e_n.
      \end{align*}
  \end{enumerate}
\end{theorem}

\section{Orthonormal bases}\label{sec:orthonormal-bases}

\begin{definition}\label{def:orthonormal-basis}
  Let $X$ be an inner-product space. An orthonormal basis is defined to be an orthonormal
  set $S$ such that $X=\overline{\text{span}S}$. The inner-product space $X$ is separable
  iff there is a countable orthonormal basis.
\end{definition}

\begin{theorem}
  Let $X$ be an inner-product space. Let $\rr{e_n}_{n\in\N}$ be an orthonormal set.
  The following are equivalent:
  \begin{enumerate}
    \item $X=\overline{\text{span}\cc{e_n}_{n\in\N}}$.
    \item For all $x\in X$, $x=\sum_{n=1}^{\infty} \aa{x,e_n}e_n$.
    \item For all $x,y\in X$, $\aa{x,y}=\sum_{n=1}^{\infty} \aa{x,e_n}\overline{\aa{y,e_n}}$.
    \item For all $x\in X$, \begin{align*}
        \vabs{x}^2 = \sum_{n=1}^{\infty} \abs{\aa{x,e_n}}^2.
      \end{align*}
  \end{enumerate}
\end{theorem}

\begin{corollary}
  Any infinite-dimensional, separable Hilbert space is isometrically isomorphic to $\ell^2$.
\end{corollary}

\section{Bounded linear operators}\label{sec:bounded-linear-operators}

\begin{definition}
  Let $X$ and $Y$ be normed linear spaces. A linear function $T:X\to Y$ is bounded
  if there is an $M\geq 0$ such that, for all $x\in X$,
  \begin{align*}
    \vabs{Tx}\leq M\vabs{x}.
  \end{align*}
  A bounded linear operator is a bounded linear function.
\end{definition}

\begin{lemma}
  Let $X$ and $Y$ be a normed linear spaces. Let $T:X\to Y$ be a bounded linear operator.
  Then
  \begin{align*}
    \sup_{\vabs{x}\leq 1}\vabs{Tx}
  \end{align*}
  is finite and nonnegative.
\end{lemma}

\begin{definition}
  Let $X$ and $Y$ be normed linear spaces. Let $T:X\to Y$ be a bounded linear operator.
  The norm of $T$ is defined to be
  \begin{align*}
    \vabs{T}=\sup_{\vabs{x}\leq 1}\vabs{T x}.
  \end{align*}
\end{definition}

\begin{lemma}
  Let $X$ and $Y$ be a normed linear spaces. Let $T:X\to Y$ be a bounded linear operator.
  For all $x\in X$,
  \begin{align*}
    \vabs{Tx}\leq \vabs{T}\vabs{x}.
  \end{align*}
  Furthermore, if $X\neq\cc{0}$,
  \begin{align*}
    \vabs{T}=\sup_{\vabs{x}=1}\vabs{Tx}=\sup_{x\neq 0}\frac{\vabs{Tx}}{\vabs{x}}.
  \end{align*}
\end{lemma}

\begin{theorem}
  Let $X$ and $Y$ be normed linear spaces. Let $T:X\to Y$ be a linear function
  \begin{align*}
    \vabs{T}=\inf\cc{M\geq 0 : \forall x \in X. \vabs{Tx}\leq M\vabs{x}}.
  \end{align*}
\end{theorem}

\begin{theorem}
  Let $X$ and $Y$ be normed linear spaces. Let $T:X\to Y$ be a linear function.

  The following are equivalent:
  \begin{enumerate}
    \item $T$ is bounded.
    \item $T$ is continuous at every $x\in X$.
    \item $T$ is continuous at $0$.
  \end{enumerate}
\end{theorem}

\begin{theorem}
  Let $X$ and $Y$ be normed linear spaces. Let $T\in\mathcal L\rr{X,Y}$.
  If $X$ is finite-dimensional then $T$ is continuous.
\end{theorem}

\begin{theorem}
  Let $X$ and $Y$ be normed linear spaces. The operator norm is a norm on
  $\mathcal L\rr{X,Y}$.
\end{theorem}

\begin{lemma}
  Let $X$ be a normed linear space. The norm is continuous.
\end{lemma}

\begin{theorem}
  Let $X$ and $Y$ be normed linear spaces. If $Y$ is a Banach space, then
  $\mathcal L\rr{X,Y}$ with the operator is a Banach space.
\end{theorem}

\begin{theorem}
  Let $X$ be a normed linear space. Let $Y$ be a Banach space. Let $W$ be a dense
  subset of $X$.

  If $T\in\mathcal L\rr{W,Y}$ then there is a unique $\bar T\in\mathcal L\rr{X,Y}$
  such that $\forall w\in W.\: \bar T w = T w$. Furthermore, if $T$ is an isometry then so
  is $\bar T$.
\end{theorem}

\begin{theorem}
  Let $X$ and $Y$ be normed linear spaces. Let $T\in\mathcal L\rr{X,Y}$. The kernel of
  $T$, $\ker T$, is a closed subspace.
\end{theorem}

\end{document}
