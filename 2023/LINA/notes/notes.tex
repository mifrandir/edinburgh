\documentclass{article}
\usepackage{notes-preamble}
\usepackage{enumitem}
\begin{document}
\mkthmstwounified
\title{Linear Analysis (SEM7)}
\author{Franz Miltz}
\maketitle
\tableofcontents
\pagebreak

\section{Norms}

\begin{definition}
  Let $X$ be an $\F$-vector space.
  A function $ \vabs{-} : X \to \R$ is defined to be a norm on $X$ if, and only if, the following hold
  \begin{enumerate}
    \item $\forall x\in X.\: \vabs{x}\geq 0$ and $ \vabs{x} = 0$ if, and only if, $x = 0$.
    \item $\forall \lambda\in\F.\:\forall x \in X.\: \vabs{\lambda x} = \abs{\lambda} \vabs{x}$.
    \item $\forall x,y\in X.\: \vabs{x + y} \leq \vabs{x} + \vabs{y}$
  \end{enumerate}
\end{definition}

\begin{definition}
  Let $p\in\br{1,\infty}$ and $f:X\to\F$ measurable. Then
  \begin{align*}
    \vabs{f}_{p} = \rr{\int_X \vv{f}^p}^{1/p}.
  \end{align*}
\end{definition}

\begin{definition}
  Let $f:X\to\F$ be measurable. Then $\vabs{f}_{\infty}$ is the
  essential supremum of $f$.
\end{definition}

\subsection{Inequalities}

\begin{lemma}
  For all $x,y\in X$, $\vv{\vabs{x}-\vabs{y}}\leq \vabs{x-y}$.
\end{lemma}

\begin{theorem}
  If $A,B\geq 0$ and $0\leq\theta\leq 1$ then
  $A^\theta B^{1-\theta} \leq \theta A + \rr{1-\theta}B$.
\end{theorem}

\begin{theorem}
  Let $p,q\in \br{1,\infty}$ be such that $1/p+1/q=1$.
  For measurable functions functions $f,g:X\to\F$,
  \begin{align*}
    \vabs{fg}_{1} \leq \vabs{f}_{p}\vabs{g}_{q}.
  \end{align*}
\end{theorem}

\begin{theorem}
  \label{thm:minkowski-for-p-infinite}
  For any two $\F$-values sequences $ \rr{a_n}_{n\in\N}$ and $ \rr{b_n}_{n\in\N}$,
  \begin{align*}
    \max_i \abs{a_i + b_i} \leq \max_i \abs{a_i} + \max_i \abs{b_i}.
  \end{align*}
\end{theorem}

\subsection{Isometries}

\begin{definition}
  Consider normed linear spaces $X$ and $Y$.
  \begin{itemize}
    \item A linear map $T:X\to Y$ is an \emph{isometry} iff, for all $x\in X$,
      $\vabs{Tx}_{Y}=\vabs{x}_{X}$.
    \item The spaces $X$ and $Y$ are \emph{isometrically isomorphic} if
      there is an onto isometry $X\to Y$.
  \end{itemize}
\end{definition}

\subsection{Completeness}

\begin{definition}
  Consider normed linear spaces $X$ and $Y$.
  \begin{itemize}
    \item A set $E\subseteq X$ is \emph{complete}
      iff every Cauchy sequence in $E$ is convergent.
    \item The space $X$ is \emph{Banach} iff it is complete.
    \item A set $U\subseteq X$ is \emph{dense} iff, for all $x\in X$ and $\e>0$,
      there is a $u\in U$ such that $\vabs{x-u}_{X}<\e$.
    \item A linear map $T:X\to Y$ is a \emph{Banach space completion}
      if $Y$ is Banach, $T$ is an isometry, and $TX$ is dense in $Y$.
    \item A set $U\subseteq X$ is \emph{compact} if every open cover
      has a finite subcover or, equivalently, every bounded sequence has
      a subsequential limit.
  \end{itemize}
\end{definition}

\begin{theorem}
  Let $X$ be Banach.
  \begin{itemize}
    \item A subset $U\subseteq X$ is closed iff every Cauchy sequence in $F$ has
      a limit in $U$.
    \item A subspace $W\subseteq X$ is Banach iff it is closed.
  \end{itemize}
\end{theorem}

\subsection{Compactness}

\begin{definition}
  Consider normed linear spaces $X$ and $Y$.
  \begin{itemize}
    \item A set $U\subseteq X$ is \emph{compact} if every open cover
      has a finite subcover or, equivalently, every bounded sequence has
      a subsequential limit.
    \item An operator $T:X\to Y$ is \emph{compact} if
      $\overline{T(B(0,1))}$ is compact.
  \end{itemize}
\end{definition}

\begin{theorem}
  Consider a normed linear space $X$.
  \begin{itemize}
    \item Any closed subset of a compact set is compact.
    \item Every compact subset is closed and bounded.
    \item The unit sphere is compact iff $X$ is finite dimensional.
  \end{itemize}
\end{theorem}

\subsection{Equivalence}

\begin{definition}
  Norms $\vabs{-}$ and $\vabs{-}'$ on $X$ are \emph{equivalent} iff
  there is a $\lambda>0$ such that, for all $x\in X$,
  $\vabs{x}\leq \lambda\vabs{x}'$ and $\vabs{x}'\leq \lambda\vabs{x}$.
\end{definition}

\begin{lemma}
  Equivalent norms agree on limit points, Cauchy sequences, completeness of
  the space, open sets, closed sets, and compact sets.
\end{lemma}

\begin{theorem}
  All norms on a finite-dimensional vector space are equivalent.
\end{theorem}

\begin{theorem}
  In a finite-dimensional normed linear space, a set is compact iff it is
  closed and bounded.
\end{theorem}

\section{Inner products}

\begin{definition}
  Let $X$ be an $\F$-vector space. A map $\aa{-,-}:X\times X\to\F$ is an inner
  product iff the following hold:
  \begin{enumerate}
    \item For all $x\in X$, $\aa{x,x}\geq 0$ and $\aa{x,x}=0$ iff $x=0$.
    \item For all $x,y\in X$, $\aa{x,y}=\overline{\aa{y,x}}$.
    \item For all $\lambda,\mu\in F$ and for all $x,y,z\in X$, $\aa{\lambda x+\mu y,z}=\lambda\aa{x,z}+\mu\aa{y,z}$.
  \end{enumerate}
\end{definition}

\begin{theorem}
  Let $X$ be an inner product space and let $x,y\in X$.
  \begin{itemize}
    \item $\vabs{x+y}^2 + \vabs{x-y}^2 =
      2\rr{\vabs{x}^2+\vabs{y}^2}$.
    \item If $\F=\R$, then
      \begin{align*}
        \vabs{x+y}^2 - \vabs{x-y}^2 = 4\aa{x,y}.
      \end{align*}
    \item If $\F=\C$, then
      \begin{align*}
        \sum_{k=1}^4 i^k \vabs{x+i^k y}^2 = 4\aa{x,y}.
      \end{align*}
  \end{itemize}
\end{theorem}

\subsection{Hilbert spaces}

\begin{definition}
  Consider an inner product space $X$.
  \begin{itemize}
    \item The \emph{induced norm} is given by $\vabs{x} = \sqrt{\aa{x,x}}$.
    \item The space $X$ is \emph{Hilbert} iff it is complete with respect
      to the induced norm.
  \end{itemize}
\end{definition}

\begin{theorem}
  Let $X$ be an inner product space. For all $x,y\in X$,
  \begin{align*}
    \abs{\aa{x,y}}\leq \vabs{x}\vabs{y}
  \end{align*}
\end{theorem}

\subsection{Orthogonality}\label{sec:orthogonality}

\begin{definition}\label{def:orthogonality}
  Let $X$ be an inner product space.
  \begin{itemize}
    \item Elements $x,y\in X$ are orthogonal to each other iff $\aa{x,y}=0$.
    \item The \emph{orthogonal complement} to a subset $U\subseteq X$ is
      $U^\bot = \cc{x \in X : \forall u\in U. \aa{u,x}=0}$.
    \item The \emph{orthogonal projection} onto a subspace $W\subseteq X$
      is the map $P_W : X \to W$ taking $x\in X$ to its unique closest
      point $P_W(x)\in W$.
  \end{itemize}
\end{definition}

\begin{lemma}
  Let $X$ be an inner-product space and $U\subseteq X$.
  \begin{enumerate}
    \item $U^\bot$ is a closed subspace.
    \item $U\subseteq U^{\bot\bot}$.
    \item $\rr{\overline{U}}^\bot=U^\bot$.
    \item If $U$ is a subspace, then $U\cap M^\bot = \cc{0}$.
  \end{enumerate}
\end{lemma}

\begin{theorem}
  Let $X$ be an inner-product space and let $W\subseteq X$ be a closed subspace.
  \begin{enumerate}
    \item For all $x\in X$, $x=P_{W}\rr{x}+P_{W^\bot}\rr{x}$, and $P_W(x)\in W$
      and $P_{W^\bot}(x)\in W^\bot$ are unique with this property.
    \item $W^{\bot\bot}=W$.
  \end{enumerate}
\end{theorem}

\begin{theorem}
  Let $X$ be an inner-product space and let $W\subseteq X$ be a subspace. $\overline W = W^{\bot\bot}$.
\end{theorem}

\subsection{Orthonormal sets}

\begin{definition}
  Consider a normed linear space $X$.
  \begin{itemize}
    \item A set $U\subseteq X$ is \emph{orthogonal} if, for all $u,v\in U$ with
      $u\neq v$, $\aa{u,v}=0$.
    \item A set $U\subseteq X$ is \emph{orthonormal} if it is orthogonal and,
      for all $u\in U$, $\vabs{u}=1$. I.e. $\aa{u,v} = \delta_{u,v}$.
    \item A set $U\subseteq X$ is an \emph{orthonormal basis} if it is
      orthonormal and $X=\overline{\text{span}(U)}$.
    \item The space $X$ is \emph{separable} iff there is a countable orthonormal
      basis.
  \end{itemize}
\end{definition}

\begin{lemma}
  Every orthogonal set is linearly independent.
\end{lemma}

\begin{theorem}[Gram-Schmidt]
  Let $X$ be an inner-product space. Let $N\in\Z_{>0}\cup \cc{\infty}$. If $\rr{v_i}_{i=1}^N$
  is an ordered set of linearly independent vectors in $X$ then there exists an ordered set
  $\rr{w_i}_{i=1}^N$ with the properties that
  \begin{enumerate}
    \item $w_1=\frac{q}{\vabs{v_1}}v_1$.
    \item For $i>1$, \begin{align*}
        w_i = \frac{q}{v_i-P_{\text{span}\cc{v_1,...,v_{i-1}}}\rr{x}}\rr{v_i-P_{\text{span}\cc{v_1,...,v_{i-1}}}\rr{v_i}}.
      \end{align*}
    \item For all $i\leq N$, $\text{span}\cc{w_1,...,w_i}=\text{span}\cc{v_1,...,v_i}$.
    \item $\cc{w_i}_{i=1}^N$ is an orthonormal set.
  \end{enumerate}
\end{theorem}

\begin{lemma}
  Let $X$ be an inner-product space. Let $y\in X$. Let $\rr{x_n}_{n\in\N}$ be a sequence
  in $X$. If $\sum_{n=1}^{\infty} x_n$ converges then
  \begin{align*}
    \aa{\sum_{n=1}^{\infty} x_n,y}=\sum_{n=1}^{\infty} \aa{x_n,y}, \hs
    \aa{\sum_{n=1}^{\infty} y,x_n}=\sum_{n=1}^{\infty} \aa{y,x_n}.
  \end{align*}
\end{lemma}

\begin{theorem}
  Let $X$ be an inner-product space and let $\rr{e_n}$ be an orthonormal
  family.
  \begin{enumerate}
    \item If $x=\sum_{n=1}^{N} \alpha_n e_n$ then $\aa{x,e_n}=\alpha_n$.
    \item Consider the subspace $W=\overline{\text{span}\cc{e_n}}$. For all
      $x\in X$, $P_{W}\rr{x}=\sum_{n=1}^{N} \aa{x,e_n}e_n$.
  \end{enumerate}
\end{theorem}

\begin{theorem}
  Let $X$ be an inner-product space. Let $\rr{e_n}_{n\in\N}$ be an orthonormal
  set. Tfae:
  \begin{enumerate}
    \item $X=\overline{\text{span}\cc{e_n}_{n\in\N}}$.
    \item For all $x\in X$, $x=\sum_{n=1}^{\infty} \aa{x,e_n}e_n$.
    \item For all $x,y\in X$, $\aa{x,y}=\sum_{n=1}^{\infty} \aa{x,e_n}\overline{\aa{y,e_n}}$.
    \item For all $x\in X$, $\vabs{x}^2 = \sum_{n=1}^{\infty} \abs{\aa{x,e_n}}^2$.
  \end{enumerate}
\end{theorem}

\begin{corollary}
  Any infinite-dimensional, separable Hilbert space is isometrically isomorphic
  to $\ell^2$.
\end{corollary}

\subsection{Duals}

\begin{definition}
  The \emph{dual space} of a normed $\mathbb F$-linear space $X$ is $X^* =
  \mathcal L\rr{X,\mathbb F}$.
\end{definition}

\begin{theorem}
  Let $p,q\in\rr{1,\infty}$ such that $1/p+1/q = 1$ and let $I$ be a closed
  and bounded interval in $\R$.

  There is an isometric isomorphism $i:L^q\to \rr{L^p}^*$ given by
  \begin{align*}
    \rr{ig}\rr{f} = \int_I f\rr{x}g \rr{x} dx
  \end{align*}
  for all $f\in L^p$ and $g\in L^q$.

  In particular, the dual of $L^p\rr{I}$ is $L^q\rr{I}$.
\end{theorem}

\begin{lemma}
  Let $X$ be an inner-product space. For all $x\in X$, $\vabs{x}=\sup_{x\in X,\vabs{y}\leq 1} \abs{\aa{x,y}}$.
\end{lemma}

\begin{theorem}
  Let $X$ be a normed linear space. The map $i:X\to X^{**}$ given by
  $i(x)(f) = f(x)$ is an isometric embedding. In particular, for all $x\in X$,
  $\vabs{x}_{X}=\sup_{\vabs{f}_{X^*}\leq 1}\abs{f\rr{x}}$.
\end{theorem}

\begin{definition}
  A normed linear space $X$ is \emph{reflexive} if the
  map $i:X\to X^{**}$ is an isometric isomorphism.
\end{definition}

\begin{theorem}
  Let $H$ be a Hilbert space.
  \begin{itemize}
    \item There is a conjugate-linear isometric
      isomorphism $T:H\to H^*$ given by $T(y)(x) = \aa{x,y}$.
    \item The map given by $\aa{f,g}_{H^*}=\aa{\inv Tf,\inv Tg}_H$ defines an
      inner product on $H^*$.
  \end{itemize}
\end{theorem}


\section{Bounded linear operators}

\begin{definition}
  Consider normed linear spaces $X$ and $Y$.
  \begin{itemize}
    \item An operator $T:X\to Y$ is \emph{bounded} if there is an $\lambda\geq 0$ such that, for all $x\in X$, $\vabs{Tx}\leq \lambda\vabs{x}$.
    \item The norm of a bounded operator $T:X\to Y$ is given by
      $\vabs{T}=\sup_{\vabs{x}\leq 1}\vabs{T x}$.
    \item An operator $T:X\to Y$ is \emph{compact} if
      $\overline{T(B(0,1))}$ is compact.
    \item An operator $T:X\to Y$ is \emph{finite-rank} if $\im T$ is
      finite dimensional.
  \end{itemize}
\end{definition}

\begin{theorem}
  Let $X$ and $Y$ be a normed linear spaces and let $T:X\to Y$ be a bounded linear operator.
  \begin{itemize}
    \item For all $x\in X$, $\vabs{Tx}\leq \vabs{T}\vabs{x}$.
    \item If $X\neq\cc{0}$, $\vabs{T}=\sup_{\vabs{x}=1}\vabs{Tx}=\sup_{x\neq 0}\frac{\vabs{Tx}}{\vabs{x}}$.
    \item $\vabs{T}=\inf\cc{M\geq 0 : \forall x \in X. \vabs{Tx}\leq M\vabs{x}}$.
  \end{itemize}
\end{theorem}

\begin{theorem}
  Let $T:X\to Y$ be a operator. The following are equivalent:
  \begin{enumerate}
    \item $T$ is bounded.
    \item $T$ is continuous.
    \item $T$ is continuous at $0$.
  \end{enumerate}
\end{theorem}

\begin{theorem}
  Consider a linear operator $T:X\to Y$ between normed linear spaces.
  \begin{itemize}
    \item $T$ is compact iff, for every bounded sequence $\rr{x_n}_{n\in\N}$,
      the sequence $\rr{Tx_n}_{n\in\N}$ has a convergent subsequence.
    \item If $Y$ is Banach and there is a sequence of compact operators
      $\rr{T_n}_{n\in\N}$ such that $T_n\to T$, then $T$ is compact.
    \item If $Y$ is Banach and $T$ has finite rank, then $T$ is compact.
    \item If $X$ and $Y$ are separable Hilbert spaces and $T$ is
      Hilbert-Schmidt, then $T$ is compact.
  \end{itemize}
\end{theorem}

\subsection{The space of bounded linear operators}

\begin{theorem}
  Let $X$ and $Y$ be normed linear spaces.
  \begin{itemize}
    \item If $X$ is finite-dimensional, then every $T\in\mathcal L\rr{X,Y}$ is
      continuous.
    \item If $T\in\mathcal L\rr{X,Y}$ then $\ker T\subseteq X$ is a
      closed subspace.
  \end{itemize}
\end{theorem}

\begin{theorem}
  Let $X$ be a normed linear space and let $Y$ be Banach.
  \begin{itemize}
    \item $\mathcal L\rr{X,Y}$ is Banach.
    \item If $W\subseteq X$ is dense then there is a bijection
      $\overline{-}:\mathcal L\rr{W,Y}\to\mathcal L\rr{X,Y}$.
    \item If $W\subseteq X$ is dense and $T:W\to Y$ is an isometry
      then $\overline T:X\to Y$ is an isometry.
  \end{itemize}
\end{theorem}

\begin{theorem}[Finite-dimensional norm bounds]
  Let $n\in\N$. Let $\vabs{-}$ be a norm on $\mathbb F^n$ and let $A$ be a matrix. Then
  \begin{enumerate}
    \item If $\lambda$ is an eigenvalue of $A$, then $\vabs{A}\geq \abs{\lambda}$.
    \item $\vabs{A}\geq \rho\rr{A}$.
    \item If $A$ is a self-adjoint matrix then $\vabs{A}_{d_2}=\rho\rr{A}$.
  \end{enumerate}
\end{theorem}

\begin{definition}[Integral kernel]
  Let $K\in C^0\rr{\bb{0,1}\times \bb{0,1}}$ and let $T$ be an operator on $X=C^0\rr{\bb{0,1}}$.
  $K$ is an integral kernel of $T$ if, for all $f\in C^0\rr{\bb{0,1}}$,
  \begin{align*}
    \rr{Tf}\rr{x}=\int_0^1 dy K\rr{x,y}f\rr{y}.
  \end{align*}
\end{definition}

\subsection{Hilbert-Schmidt}

\begin{definition}
  Let $H$ and $K$ be separable Hilbert spaces. Let $\rr{e_n}_{n\in\N}$ be an orthonormal
  basis for $H$.
  \begin{itemize}
    \item A bounded linear operator $T:H\to K$ is \emph{Hilbert-Schmidt} iff
      $\sum_{i=1}^{\infty} \vabs{Te_i}^2<\infty$.
    \item The \emph{Hilbert-Schmidt} norm of a Hilbert-Schmidt operator $T:H\to
      K$ is $\vabs{T}_{HS}=\sqrt{\sum_{i=1}^{\infty} \vabs{Te_i}^2}$.
  \end{itemize}
\end{definition}

\begin{theorem}
  Let $H$ and $K$ be seperable Hilbert spaces.
  \begin{enumerate}
    \item The Hilbert-Schmidt norm is basis-independent.
    \item If $T:H\to K$ is a Hilbert-Schmidt operator then
      $\vabs{T}\leq \vabs{T}_{HS}$.
    \item If $T:H\to K$ is a Hilbert-Schmidt operator, then there is a sequence
      of finite-rank operators $\rr{T_n}_{n\in\N}$ such that $\vabs{T-T_n}\to
      0$.
  \end{enumerate}
\end{theorem}

\begin{theorem}
  Let $H,K$ be Hilbert spaces and let $\rr{e_n}_{n\in\N}$ be an orthonormal
  basis. If $T:\cc{e_n}_{n\in\N}\to K$ is such that
  $\sum_{n\in\N}\vabs{Te_n}^2 < \infty$ then
  \begin{itemize}
    \item $T$ extends uniquely to linear operator $\overline T:H\to K$;
    \item $\overline T$ is Hilbert-Schmidt.
  \end{itemize}
\end{theorem}

\subsection{Classification}

\begin{theorem}
  Let $H$ be an infinite-dimensional, separable Hilbert space. Then
  \begin{align*}
    \cc{T \text{ finite-rank}}
    \subset \cc{T\text{ Hilbert-Schmidt}}
    \subset \cc{T\text{ compact}}
    \subset \cc{T\text{ bounded}}.
  \end{align*}
\end{theorem}

\begin{theorem}
  Let $H$ be an infinite-dimensional Banach space. Then
  \begin{align*}
    \cc{T \text{ finite-rank}}
    \subset \cc{T\text{ compact}}
    \subset \cc{T\text{ bounded}}.
  \end{align*}
\end{theorem}


\subsection{Transposes and adjoints}

\begin{theorem}
  Let $X$ and $Y$ be normed  linear spaces. Let $T:X\to Y$ be a bounded linear operator.
  \begin{enumerate}
    \item There is a unique transpose $T':Y^*\to X^*$ satisfying
      $\rr{T'g}\rr{x}=g\rr{Tx}$.
    \item $T'$ is bounded.
    \item $\vabs{T'}=\vabs{T}$.
  \end{enumerate}
\end{theorem}

\begin{theorem}
  Let $T:H\to K$ be a bounded linear operator between Hilbert spaces.
  \begin{itemize}
    \item There is a unique adjoint $T^*:K\to H$ satisfying
      $\aa{Tx,y} = \aa{x,T^*y}$
    \item $T^*$ is bounded.
    \item $\vabs{T^*}=\vabs{T}$.
  \end{itemize}
\end{theorem}

\begin{theorem}
  Let $H$ be an infinite-dimensional, separable Hilbert space over $\C$.
  If $T:H\to H$ is a self-adjoint operator, then
  \begin{itemize}
    \item there exists an orthonormal basis $\rr{e_n}_{n\in\N}$ of eigenvectors
      of $T$ with corresponding eigenvalues $\rr{\lambda_n}_{n\in\N}$ such that
      \begin{align*}
        x = \sum_{n=1}^{\infty} \aa{x,e_n}e_n,\hs
        Tx = \sum_{n=1}^{\infty} \lambda_n\aa{x,e_n}e_n;
      \end{align*}
    \item the eigenvalues $\lambda_n$ are real;
    \item $\lim_{n\to\infty} \lambda_n=0$;
    \item the eigenspace for each non-zero eigenvalue is finite dimensional.
  \end{itemize}
\end{theorem}

\end{document}
