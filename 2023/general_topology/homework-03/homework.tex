\documentclass{article}
\usepackage{homework-preamble}
\mkanonthms
\begin{document}
\title{General Topology: Assignment 3}
\author{Franz Miltz (UUN: S1971811)}
\date{13 October 2022}
\maketitle

Fix $n\in\N$. Let $\rr{\R^n}^+=\R^n\cup\cc{\infty}$ where $\infty\not\in\R^n$ and define the following map:
\begin{align*}
	\tau:\P\rr{\rr{\R^n}^+} & \to\P\rr{\rr{\R^n}^+}                                                                                                      \\
	S                       & \mapsto \begin{cases}
		                                  \tau_{\R^n}\rr{S\setminus\cc{\infty}}\cup \cc{\infty} & \text{if $\infty\in S$}                          \\
		                                  \tau_{\R^n}\rr{S}\cup\cc{\infty}                      & \text{if $\infty\not\in S$ and $S$ is unbounded} \\
		                                  \tau_{\R^n}\rr{S}                                     & \text{if $\infty\not\in S$ and $S$ is bounded}
	                                  \end{cases}
\end{align*}

\begin{claim*}
	The map $\tau$ defines a topology on $\rr{\R^n}^+$.
	\begin{proof}
		We verify all the axioms.

		Firstly, let $S\subseteq \rr{\R^n}^+$. As $\tau_{\R^n}$ is a topology and $\infty\in S$ implies $\infty\in\tau\rr{S}$,
		we must have $S\subseteq \tau\rr{S}$.

		Secondly, let $\Sigma\subseteq \P\rr{\R^n}^+$. Clearly, if, for all $S\in\Sigma$, $\infty\not\in\tau\rr{S}$ then
		\begin{align*}
			\tau\rr{\bigcup_{S\in\Sigma}S}=\tau_{\R^n}\rr{\bigcup_{S\in\Sigma} S}=\bigcup_{S\in\Sigma}\tau_{\R^n}\rr{S} = \bigcup_{S\in\Sigma}\tau\rr{S}.
		\end{align*}
		Further, if $\infty\in\tau\rr{S}$ for some $S\in\Sigma$, then
		\begin{align*}
			\tau\rr{\bigcup_{S\in\Sigma}} = \tau_{\R^n}\rr{\bigcup_{S\in\Sigma}\rr{S\setminus \cc{\infty}}} \cup \cc{\infty} = \bigcup_{S\in\Sigma}\tau\rr{S}.
		\end{align*}

		Finally, let $S\subseteq\rr{\R^n}^+$. It follows from the above that $\tau\rr{S}\subseteq\tau^2\rr{S}$. Now consider $x\in\tau^2\rr{S}$.
		If $x\in\R^n$ then immediately $x\in\tau\rr{S}$. Consider the case $x=\infty$. Assume $\infty\not\in\tau\rr{S}$. Then
		$\tau_{\R^n}\rr{S}\subseteq\R^n$ must be unbounded while $S\subseteq\R^n$ is bounded. It is straightforward to show analytically that this is
		a contradiction. Thus $\infty\in\tau\rr{S}$, i.e. $\tau\rr{S}\supseteq \tau^2\rr{S}$ and thereby $\tau\rr{S}=\tau^2\rr{S}$.
	\end{proof}
\end{claim*}

Define the following map:
\begin{align*}
	\phi : S^n          & \to \rr{\R^n}^+                                                    \\
	\rr{x_0,\ldots,x_n} & \mapsto \begin{cases}
		                              \frac{1}{1-x_0}\rr{x_1,\ldots,x_n} & \text{if $x_0\neq 1$} \\
		                              \infty                             & \text{if $x_0=1$}
	                              \end{cases}
\end{align*}

\begin{claim*}
	The map $\phi$ is a homeomorphism relative to the topology $\tau$.
	\begin{proof}
		Let $p=\rr{1,0,\ldots,0}$. We construct the inverse:
		\begin{align*}
			\inv\phi : \rr{\R^n}^+ & \to S^n                                                                                            \\
			x                      & \mapsto \begin{cases}
				                                 \rr{1, 0, \ldots, 0}                      & \text{if $x=\infty$}                        \\
				                                 \frac{1}{s+1}\rr{s-1, 2x_1, \ldots, 2x_n} & \text{if $x=\rr{x_1,\ldots,x_n}\neq\infty$}
			                                 \end{cases}
		\end{align*}
		where $s=\sum_{i=1}^n x_i^2$. It is straightforward to verify that $\inv\phi$ is indeed
		the inverse of $\phi$.
		We use the fact that the restrictions
		\begin{align*}
			\phi: S^n\setminus\cc{p}\to\R^n, \\
			\inv\phi:\R^n\to S^n\setminus\cc{p}
		\end{align*}
		are continuous with respect to the usual topologies.

		Let $F\subseteq S^n$ be closed. If $p\not\in F$ then $\inv\phi\rr{F}$ is
		closed as the restriction
		is continuous. Now consider the case where $p\in F$. Then
		$\infty\in \inv\phi\rr{F}$. In particular we note
		\begin{align*}
			\tau\rr{\inv\phi\rr{F}} & = \tau\rr{\inv\phi\rr{F\setminus\cc{p}}\cup\cc{\infty}}          \\
			                        & = \tau\rr{\inv\phi\rr{F\setminus\cc{p}}}\cup\tau\rr{\cc{\infty}} \\
			                        & = \tau_{\R^n}\rr{\inv\phi\rr{F\setminus\cc{p}}}\cup\cc{\infty}
		\end{align*}
		As $F\setminus\cc{p} = F\cap \rr{S^n\setminus\cc{p}}$, its inverse image
		must be closed in $\R^n$. Thus
		\begin{align*}
			\tau\rr{\inv\phi\rr{F}} & = \inv\phi\rr{F\setminus\cc{p}}\cup\cc{\infty} = \inv\phi\rr{F}.
		\end{align*}
		Therefore $\phi$ is continuous.

		Now let $F\subseteq \rr{\R^n}^+$ be closed. Once more we have
		$F\setminus\cc{\infty}=F\cap\R^n$. Thus if $\infty\not\in F$ then
		the inverse image of $F$ under $\inv\phi$, $\phi\rr{F}$, is closed in $S^n$.
		Consider the case $\infty\in F$. Then $p\in\phi\rr{F}$. Thus
		\begin{align*}
			\tau_{S^n}\rr{\phi\rr{F}} & = \tau_{S^n}\rr{\phi\rr{F\setminus\cc{\infty}}\cup\cc{p}}                \\
			                          & = \tau_{S^n}\rr{\phi\rr{F\setminus\cc{\infty}}}\cup\tau_{S^n}\rr{\cc{p}} \\
			                          & = \phi\rr{F\setminus\cc{\infty}}\cup\cc{p}                               \\
			                          & = \phi\rr{F}.
		\end{align*}
		Thus $\inv\phi$ is continuous, making $\phi$ a homeomorphism.
	\end{proof}
\end{claim*}

Define the following map:
\begin{align*}
	f : \R & \to S^1                  \\
	t      & \mapsto \exp\rr{2\pi it}
\end{align*}

Let $\tau:S^1\to S^1$ denote the usual topology on $S^1\subseteq\C$, i.e.
the subspace topology induced by $\tau_\C$.

\begin{claim*}
	$\tau$ is the finest topology such that $f$ is continuous.
	\begin{proof}
		From complex analysis we know that $f$ is continuous with respect to the usual topologies.

		Consider the action of $f$ on open intervals. It is straightforward to verify that,
		for any open interval $(a,b)\in\R$, $f(a,b)$ is open in $S^1$ with respect to the
		usual topology. Since any open set in $\R$ is a union of open intervals, the direct
		image $f\rr{E}$ of any open set $E\subseteq\R$ is open.

		Now suppose, for a contradiction, $\tau':\P S^1 \to \P S^1$ is a topology, finer than
		$\tau$, such that
		$f$ is continuous. Then there exists $E\subseteq S^1$ that is open with respect to
		$\tau'$ but not with repsect to $\tau$. Consider the inverse image $\inv f\rr{E}$.
		We have
		\begin{align*}
			\inv f\rr{E} = \bigcup\mathcal I
		\end{align*}
		for some family of disjoint open intervals $\mathcal I$. Thus
		\begin{align*}
			f\rr{\inv f\rr{E}}=\bigcup_{I\in\mathcal I} f\rr{I}
		\end{align*}
		is open in $S^1$ with respect to $\tau$.

		Consider $y\in f\rr{\inv f\rr{E}}$.
		Then there is an $x\in \inv f\rr{E}$ such that $f\rr{x}=y$. Since $x\in\inv f\rr{E}$
		there is a $y'\in E$ such that $f\rr{x}=y'$. Thus $y=y'\in E$, i.e. $f\rr{\inv f\rr{E}}\subseteq E$.

		Finally, consider $y\in E$. There exists a set $X=\inv f\rr{\cc{y}}\subseteq\R$ such that,
		for all $x\in X$, $f\rr{x}=y$. Immediately $f\rr{X}=\cc{y}$, i.e. $y\in f\rr{\inv f\rr{E}}$.
		Thus $f\rr{\inv f\rr{E}}\supseteq E$.

		Thus $E=f\rr{\inv f\rr{E}}$ is open with respect to $\tau$. Contradiction.
	\end{proof}
\end{claim*}

\begin{claim*}
	$f$ is a local homeomorphism.
	\begin{proof}
		Let $0<r<\pi$ be a constant and let $x\in\R$. Then let $U=B\rr{x,r}$. Clearly
		$U$ is open in $\R$ and thus $f\rr{U}$ is open in $S^1$. As $f$ is continuous, so
		is the restriction $\eval{f}{U}:U\to f\rr{U}$.

		We note that $f\rr{U}\subset S^1$, i.e. a strict subset. Thus we may choose a branch
		$\log:f\rr{U}\to\C$ that is continuous. Dividing by $2\pi i$ and using the original value
		of $x$ it is then possible to construct a continuous inverse $\inv{\rr{\eval{f}{U}}}$, e.g.:
		\begin{align*}
			\inv{\rr{\eval{f}{U}}}:f\rr{U} & \to U                                                  \\
			y                              & \mapsto x+\frac{1}{2\pi i}\rr{\log y-\log\rr{f\rr{x}}}
		\end{align*}
		Thus $\eval{f}{U}$ is a local homeomorphism.
	\end{proof}
\end{claim*}

\end{document}
