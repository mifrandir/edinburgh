\documentclass{article}
\usepackage{homework-preamble}

\begin{document}
\title{Linear Analysis: Homework 1}
\author{Franz Miltz (UUN: S1971811)}
\date{27 September 2022}
\maketitle

Let $C^0=C^0 \rr{ \bb{ 0,1 }, \R }$.

\begin{claim*}[2d]
	The set $S$ of all $\F$-valued sequences with limit $0$ is a vector space.
	\begin{proof}
		Let $(x_n)_{n\in\N},(y_n)_{n\in\N}\in S$. Let $\lambda\in\F$. Using limit laws we have
		\begin{align*}
			\lim_{n\to\infty}(\lambda x_n) = \lambda \lim_{n\to\infty} x_n = 0
		\end{align*}
		and
		\begin{align*}
			\lim_{n\to \infty}(x_n + y_n) = \lim_{n\to\infty} x_n + \lim_{n\to\infty} y_n = 0.
		\end{align*}
		Thus $(\lambda x_n)_{n\in\N},(x_n + y_n)_{n\in\N}\in S$.
	\end{proof}
\end{claim*}

\begin{claim*}[5]
	Let $f,g\in C^0$ be given by
	\begin{align*}
		f(x)=\cos\rr{2\pi x}, \hs g \rr{ x } = \cos \rr{ 4\pi x }.
	\end{align*}
	Then $f$ and $g$ are linearly independent.
	\begin{proof}
		Firstly we note that $f,g$ are indeed continuous on $[0,1]$ so they are elements of $C^0$.
		Consider $\lambda,\mu\in\R$ such that
		\begin{align*}
			\lambda f + \mu g = 0.
		\end{align*}
		I.e., for all $x\in[0,1]$, \begin{align*}
			\lambda f \rr{ x } + \mu g \rr{ x } = 0.
		\end{align*}
		We choose $x=1/4$ to find $\lambda = 0$. Using this and choosing $x=1/8$ we obtain $\mu=0$.
		Thus $f$ and $g$ are linearly independent.
	\end{proof}
\end{claim*}

\begin{claim*}[6]
	Let $\vabs{-}_{C^0}:C^0\to \R$ be given by
	\begin{align*}
		\vabs{f}_{C^0}=\max_{x\in \bb{ 0,1 }}\abs{f \rr{ x }}.
	\end{align*}
	$\vabs{-}_{C^0}$ defines a norm.
	\begin{proof}
		To begin we note that all $f\in C^0$ are continuous on a closed and bounded interval
		and thus attain a maximum and minimum values.

		Let $f\in C^0$ and let $\lambda\in\R$. Then
		\begin{align*}
			\vabs{\lambda f}_{C^0} & = \max_{x\in \bb{ 0,1 }}\abs{\lambda f \rr{ x }}       \\
			                       & = \abs{\lambda} \max_{x\in \bb{ 0,1 }}\abs{f \rr{ x }} \\
			                       & = \abs\lambda \vabs{f}_{C^0}.
		\end{align*}
		Let $f,g\in C^0$. Let $x_0,y_0\in \bb{ 0,1 }$ such that
		\begin{align*}
			\vabs{f}_{C^0} = \abs{f \rr{ x_0 }}, \hs \vabs{g}_{C^0} = \abs{g \rr{ y_0 }}.
		\end{align*}
		Then
		\begin{align*}
			\vabs{f + g}_{C^0} & = \max_{x\in \bb{ 0,1 }} \abs{ f \rr{ x } + g \rr{ x } }                                \\
			                   & \leq \max_{x\in \bb{ 0,1 }} \rr{ \abs{f \rr{ x }} + \abs{g \rr{ x } }}                  \\
			                   & = \max_{x\in \bb{ 0,1 }} \abs{ f \rr{ x } } + \max_{x\in \bb{ 0,1 }} \abs{ g \rr{ x } } \\
			                   & = \vabs{f}_{C^0} + \vabs{g}_{C^0}.
		\end{align*}
		Finally, assume $f\in C^0$ such that $\vabs{f}_{C^0}=0$. Then, for all $x\in \bb{ 0,1 }$,
		\begin{align*}
			\abs{f \rr{ x }} \leq 0.
		\end{align*}
		Immediately, $f \rr{ x }=0$ everywhere, i.e. $f=0$.
	\end{proof}
\end{claim*}

\begin{lemma} \label{uniform-iff-pointwise}
	Let $ \rr{u_n}_{n\in\N}$ a sequence of functions in $C^0$. Then $ \rr{u_n}_{n\in\N}$ converges
	uniformly iff there exists a function $u\in C^0$ such that
	\begin{align*}
		\lim_{n\to\infty} \vabs{u_n-u}_{C^0}=0.
	\end{align*}
	\begin{proof}
		Assume $ \rr{u_n}_{n\in\N}$ converges uniformly to some $u\in C^0$. Then, for all $\e > 0$,
		there exists an $N\in\N$ such that, for all $x\in \bb{ 0,1 }$ and for all $n>N$,
		\begin{align*}
			\abs{u_n \rr{ x } - u \rr{ x }} < \e.
		\end{align*}
		Now fix $\e>0$ and choose $N$ accordingly. Then, for $n>N$,
		\begin{align*}
			\vabs{u_n-u}_{C^0} = \max_{x\in \bb{ 0,1 }} \abs{u_n \rr{ x } - u \rr{ x } } < \max_{x\in \bb{0,1}} \rr{\e} < \e.
		\end{align*}
		Thus \begin{align*}
			\lim_{n\to\infty} \vabs{u_n-u}_{C^0} = 0.
		\end{align*}
		Conversely, assume for some $u\in C^0$,
		\begin{align*}
			\lim_{n\to\infty} \vabs{u_n-u}_{C^0} = 0.
		\end{align*}
		Now fix $\e>0$ and choose $N$ such that, for all $n>N$,
		\begin{align*}
			\vabs{u_n-u}_{C^0}<\e.
		\end{align*}
		Then, for all $x\in \bb{ 0,1 }$ and for all $n>N$,
		\begin{align*}
			\abs{u_n \rr{ x } - u \rr{ x }} \leq \max_{x\in \bb{ 0,1 }} \abs{u_n \rr{ x } - u \rr{ x }} = \vabs{u_n-u}_{C^0} < \e.
		\end{align*}
		Thus $ \rr{u_n}_{n\in\N}$ converges uniformly to $u$.
	\end{proof}
\end{lemma}

\begin{lemma} \label{cauchy-implies-uniform}
	Let $ \rr{u_n}_{n\in\N}$ such that, for all $\e>0$, there exists $N\in\N$ such that, for all $n,m\geq N$ and for all
	$x\in \bb{ 0,1 }$, $\abs{u_m \rr{ x } - u_n \rr{ x }} < \e$. Then $u_n\to u$ uniformly for some $u\in C^0$.
	\begin{proof}
		Assume the condition above holds. In $\R$, every Cauchy sequence converges. Thus $u_n\to u$
		pointwise for some $u: \bb{ 0,1 }\to \R$. We claim that this convergence is uniform.
		To see this, fix $\e > 0$ and choose $N\in\N$ such that, for all $m,n>N$ and all $x\in \bb{ 0,1 }$,
		\begin{align*}
			\abs{u_m \rr{ x } - u_n \rr{ x }} < \e/2.
		\end{align*}
		In particular, fix $n>N$ and calculate
		\begin{align*}
			\lim_{m\to\infty} \abs{ u_m \rr{ x } - u_n \rr{ x }} = \abs{ u \rr{ x } - u_n \rr{ x } }.
		\end{align*}
		Now
		\begin{align*}
			\lim_{m\to\infty} \abs{ u_m \rr{ x } - u_n \rr{ x }} \leq \e / 2
		\end{align*}
		so finally
		\begin{align*}
			\abs{u \rr{ x } - u_n \rr{ x }} < \e.
		\end{align*}

	\end{proof}
\end{lemma}

\begin{claim*}[7]
	$ \rr{ C^0, \vabs{-}_{C^0} }$ is a Banach space.
	\begin{proof}
		We have shown above that $\vabs{-}_{C^0}$ is a norm.

		Let $ \rr{u_n}_{n\in\N}$ be a Cauchy sequence. Now fix $\e>0$ and choose $N\in\N$
		such that, for all $m,n>N$,
		\begin{align*}
			\vabs{u_m - u_n}_{C^0}<\e.
		\end{align*}
		Then, in particular, for all $x\in \bb{ 0,1 }$, $\abs{u_m \rr{ x } - u_n \rr{ x }}<\e$.
		By~\ref{cauchy-implies-uniform}, $ \rr{u_n}_{n\in\N}$ converges uniformly to some $u\in C^0$.
		We can now apply~\ref{uniform-iff-pointwise} to find that
		\begin{align*}
			\lim_{n\to\infty} \vabs{u_n-u}_{C^0} = 0.
		\end{align*}
		Thus $u_n\to u$.
	\end{proof}
\end{claim*}

\end{document}
