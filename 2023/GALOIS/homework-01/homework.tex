\documentclass{article}
\usepackage{homework-preamble}

\begin{document}
\title{Galois Theory: Homework 1}
\author{Franz Miltz (UUN: S1971811)}
\date{2 February 2023}
\maketitle

Let $k\geq 0$. For $\sigma\in S_k$ and $p\in\Q\bb{t_1,\ldots,t_k}$, define
$\sigma p\in\Q\bb{t_1,\ldots,t_k}$ by
\begin{align*}
  \rr{\sigma p}\rr{t_1,\ldots,t_k}=p\rr{t_{\sigma\rr{1}},\ldots,t_{\sigma\rr{k}}}.
\end{align*}

\begin{claim*}[i]
  The map
  \begin{align*}
    S_k \times\Q\bb{t_1,\ldots,t_k}&\to\Q\bb{t_1,\ldots,t_k}\\
    \rr{\sigma,p}&\mapsto\sigma p
  \end{align*}
  defines a left action.
  \begin{proof}
    Firstly, consider the identity perumtation $e\in S^k$. For all $p\in\Q\bb{t_1,\ldots,t_k}$,
    \begin{align*}
      \rr{ep}\rr{t_1,\ldots,t_k}=p\rr{t_1,\ldots,t_k}.
    \end{align*}
    Further, let $\sigma,\tau\in S^k$ and let $p\in\Q\bb{t_1,\ldots,t_k}$. Then
    \begin{align*}
      \rr{\sigma\rr{\tau p}}\rr{t_1,\ldots,t_k}
      &= \sigma p\rr{t_{\tau\rr{1}},\ldots,t_{\tau\rr{k}}}\\
      &= p\rr{t_{\sigma\rr{\tau\rr{1}}},\ldots,t_{\sigma\rr{\tau\rr{k}}}}\\
      &= p\rr{t_{\rr{\sigma\tau}\rr{1}},\ldots,t_{\rr{\sigma\tau}\rr{k}}}\\
      &= \rr{\rr{\sigma\tau}p}\rr{t_1,\ldots,t_k},
    \end{align*}
    i.e. $\sigma\rr{\tau p}=\rr{\sigma \tau}p$.
  \end{proof}
\end{claim*}

\begin{claim*}[ii]
  Let $f\in\Q\bb{t}$ with $k$ distinct roots in $\C$. Then $\Gal{f}$ is a subgroup of $S_k$.
  \begin{proof}
    Let $\alpha_1,\ldots,\alpha_k\in\C$ be the roots. Consider $\sigma,\tau\in\Gal{f}$.
    By definition of conjugacy we have, for all polynomials $p\in\Q\bb{t_1,\ldots,t_k}$,
    \begin{align}
      \label{eq:sigma_conjugacy}
      p\rr{\alpha_1,\ldots,\alpha_k} = 0 \text{ iff }
      p\rr{\alpha_{\sigma\rr{1}},\ldots,\alpha_{\sigma\rr{k}}} = 0
    \end{align}
    and
    \begin{align}
      \label{eq:tau_conjugacy}
      p\rr{\alpha_1,\ldots,\alpha_k} = 0 \text{ iff }
      p\rr{\alpha_{\tau\rr{1}},\ldots,\alpha_{\tau\rr{k}}} = 0.
    \end{align}
    In particular, fix $p\in\Q\bb{t_1,\ldots,t_k}$. Then (\ref{eq:tau_conjugacy}) yields
    \begin{align*}
      \rr{\sigma p}\rr{\alpha_1,\ldots,\alpha_k} = 0 \text{ iff }
      \rr{\sigma p}\rr{\alpha_{\tau\rr{1}},\ldots,\alpha_{\tau\rr{k}}} = 0
    \end{align*}
    and thus, unpacking the definition of $\sigma p$,
    \begin{align*}
      p\rr{\alpha_{\sigma\rr{1}},\ldots,\alpha_{\sigma\rr{k}}} = 0 \text{ iff }
      p\rr{\alpha_{\rr{\sigma\tau}\rr{1}},\ldots,\alpha_{\rr{\sigma\tau}\rr{k}}} = 0
    \end{align*}
    Combining this with (\ref{eq:sigma_conjugacy}), we find $\sigma\tau\in\Gal{f}$.
    Further, (\ref{eq:sigma_conjugacy}) implies
    \begin{align*}
      \rr{\inv\sigma p}\rr{\alpha_1,\ldots,\alpha_k} = 0 \text{ iff }
      \rr{\inv\sigma p}\rr{\alpha_{\sigma\rr{1}},\ldots,\alpha_{\sigma\rr{k}}} = 0
    \end{align*}
    i.e.
    \begin{align*}
      p\rr{\alpha_{\inv\sigma\rr{1}},\ldots,\alpha_{\inv\sigma\rr{k}}} = 0 \text{ iff }
      p\rr{\alpha_1,\ldots,\alpha_k} = 0
    \end{align*}
    This shows $\inv\sigma\in\Gal f$. Now $\Gal f$ is a subset of $S_k$ that is closed
    under the group operation and taking inverses, completing the proof.
  \end{proof}
\end{claim*}

\end{document}
