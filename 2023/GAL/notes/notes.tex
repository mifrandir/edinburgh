\documentclass{article}
\usepackage{notes-preamble}
\usepackage{enumitem}
\begin{document}
\mkthmstwounified
\title{Galois Theory (SEM8)}
\author{Franz Miltz}
\maketitle
\tableofcontents
\pagebreak

\setcounter{section}{-1}

\section{Solvable Groups}\label{sec:solvable_groups}

\begin{definition}
  Let $G$ be a group. A subnormal series for $G$ is a series of subgroups
  \begin{align*}
    \cc{e}=G_0\triangleleft G_1\triangleleft\cdots\triangleleft G_s=G.
  \end{align*}
\end{definition}

\begin{definition}
  A solvable group $G$ is a group that has a subnormal series such that for each
  factor $G_{i+1}/G_i$ is abelian.
\end{definition}

\begin{theorem}
  A finite group is solvable iff all its composition factors are cyclic.
\end{theorem}

\begin{theorem}
  Let $G$ be a group and let $N\triangleleft G$. Then $G$ is solvable iff both $N$ and
  $G/N$ are solvable.
\end{theorem}

\begin{theorem}
  If $G$ is solvable and $H\leq G$ then $H$ is solvable.
\end{theorem}

\begin{definition}
  Let $G$ be a group. The commutator of two elements $a,b\in G$ is the element
  $\bb{a,b}=ab\inv a\inv b$. The derived subgroup $G'$ of a group $G$ is the subgroup
  generated by the commutators in $G$:
  \begin{align*}
    G'=\aa{ab\inv a\inv b : a,b \in G}.
  \end{align*}
\end{definition}

\begin{theorem}
  Let $G$ be aq group and let $N\triangleleft G$. Then $G/N$ is abelian iff $G'\subseteq N$.
  In particular, $G/G'$ is abelian.
\end{theorem}

\begin{definition}
  Let $G$ be a group. Let $G^{\rr{0}}=G$ and for each $i\geq 0$, let $G^{\rr{i+1}}=\rr{G^{\rr{i}}}'$.

  The sequence
  \begin{align*}
    G = G^{\rr{0}}\triangleright G^{\rr{1}}\triangleright \cdots
  \end{align*}
  is called the derived series of $G$.
\end{definition}

\begin{theorem}
  A group is solvable iff there is an $n\geq 0$ with $G^{\rr{n}}=\cc{e}$.
\end{theorem}

\begin{definition}
  Let $G$ be a solvable group. The least $n$ such that $G^{\rr{n}}=\cc{e}$ is called the
  derived length of $G$.
\end{definition}

\section{Introduction}\label{sec:introduction}

\begin{definition}
  Let $k\geq 0$ and let $\rr{z_1,...,z_k}$ and $\rr{z'_1,\ldots,z'_k}$ be $k$-tuples
  of complex numbers. Then $\rr{z_1,\ldots,z_k}$ and $\rr{z'_1,\ldots,z'_k}$ are conjugate
  over $\Q$ iff, for all polynomials $p\in\Q\bb{X_1,\ldots,X_k}$,
  \begin{align*}
    p\rr{z_1,\ldots,z_k}=0 \Leftrightarrow p\rr{z'_1,\ldots,z'_k}=0.
  \end{align*}
\end{definition}

\begin{definition}
  Let $f\in\Q\bb{X}$ and write $\alpha_1,...,\alpha_k$ for its distinct roots in $\C$.
  The Galois group of $f$ is
  \begin{align*}
    \Gal{f}=\cc{\sigma\in S_k:\rr{\alpha_1,\ldots,\alpha_k}\text{ and }\rr{\alpha_{\sigma(1)},\ldots,\alpha_{\sigma\rr{k}}}\text{ are conjugate}}
  \end{align*}
\end{definition}

\begin{theorem}[Galois]
  Let $f\in\Q\bb{f}$. Then the following are equivalent:
  \begin{enumerate}
    \item $f$ is solvable by radicals;
    \item $\Gal{f}$ is a solvable group.
  \end{enumerate}
\end{theorem}

\end{document}
