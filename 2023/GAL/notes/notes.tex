\documentclass{article}
\usepackage{notes-preamble}
\usepackage{enumitem}
\begin{document}
\mkthmstwounified
\title{Galois Theory (SEM8)}
\author{Franz Miltz}
\maketitle
\tableofcontents
\pagebreak

\setcounter{section}{-1}

\section{Solvable Groups}\label{sec:solvable_groups}

\begin{definition}
  Let $G$ be a group. A subnormal series for $G$ is a series of subgroups
  \begin{align*}
    \cc{e}=G_0\triangleleft G_1\triangleleft\cdots\triangleleft G_s=G.
  \end{align*}
\end{definition}

\begin{definition}
  A solvable group $G$ is a group that has a subnormal series such that for each
  factor $G_{i+1}/G_i$ is abelian.
\end{definition}

\begin{theorem}
  A finite group is solvable iff all its composition factors are cyclic.
\end{theorem}

\begin{theorem}
  Let $G$ be a group and let $N\triangleleft G$. Then $G$ is solvable iff both $N$ and
  $G/N$ are solvable.
\end{theorem}

\begin{theorem}
  If $G$ is solvable and $H\leq G$ then $H$ is solvable.
\end{theorem}

\begin{definition}
  Let $G$ be a group. The commutator of two elements $a,b\in G$ is the element
  $\bb{a,b}=ab\inv a\inv b$. The derived subgroup $G'$ of a group $G$ is the subgroup
  generated by the commutators in $G$:
  \begin{align*}
    G'=\aa{ab\inv a\inv b : a,b \in G}.
  \end{align*}
\end{definition}

\begin{theorem}
  Let $G$ be a group and let $N\triangleleft G$. Then $G/N$ is abelian iff $G'\subseteq N$.
  In particular, $G/G'$ is abelian.
\end{theorem}

\begin{definition}
  Let $G$ be a group. Let $G^{\rr{0}}=G$ and for each $i\geq 0$, let $G^{\rr{i+1}}=\rr{G^{\rr{i}}}'$.

  The sequence
  \begin{align*}
    G = G^{\rr{0}}\triangleright G^{\rr{1}}\triangleright \cdots
  \end{align*}
  is called the derived series of $G$.
\end{definition}

\begin{theorem}
  A group is solvable iff there is an $n\geq 0$ with $G^{\rr{n}}=\cc{e}$.
\end{theorem}

\begin{definition}
  Let $G$ be a solvable group. The least $n$ such that $G^{\rr{n}}=\cc{e}$ is called the
  derived length of $G$.
\end{definition}

\section{Introduction}\label{sec:introduction}

\begin{definition}
  Let $k\geq 0$ and let $\rr{z_1,...,z_k}$ and $\rr{z'_1,\ldots,z'_k}$ be $k$-tuples
  of complex numbers. Then $\rr{z_1,\ldots,z_k}$ and $\rr{z'_1,\ldots,z'_k}$ are conjugate
  over $\Q$ iff, for all polynomials $p\in\Q\bb{X_1,\ldots,X_k}$,
  \begin{align*}
    p\rr{z_1,\ldots,z_k}=0 \Leftrightarrow p\rr{z'_1,\ldots,z'_k}=0.
  \end{align*}
\end{definition}

\begin{definition}
  Let $f\in\Q\bb{X}$ and write $\alpha_1,...,\alpha_k$ for its distinct roots in $\C$.
  The Galois group of $f$ is
  \begin{align*}
    \Gal{f}=\cc{\sigma\in S_k:\rr{\alpha_1,\ldots,\alpha_k}\text{ and }\rr{\alpha_{\sigma(1)},\ldots,\alpha_{\sigma\rr{k}}}\text{ are conjugate}}
  \end{align*}
\end{definition}

\begin{theorem}[Galois]
  Let $f\in\Q\bb{f}$. Then the following are equivalent:
  \begin{enumerate}
    \item $f$ is solvable by radicals;
    \item $\Gal{f}$ is a solvable group.
  \end{enumerate}
\end{theorem}

\section{Group actions, rings, and fields}

\subsection{Group actions}\label{sec:group_actions}

\begin{definition}
  Let $G$ be a group and $X$ a set. An action of $G$ on $X$ is a function
  $G\times X\to X$, written as $\rr{g,x}\mapsto gx$, such that
  \begin{align*}
    \rr{gh}x=g\rr{hx}
  \end{align*}
  for all $g,h\in G$ and $x\in X$, and $ex=x$, for all $x\in X$.
\end{definition}

\begin{definition}
  An action of $G$ on $X$ is faithful if, for $g,h\in G$,
  $g=h$ whenever $gx = hx$ for all $x\in X$.
\end{definition}

\begin{lemma}
  For an action of a group $G$ on $X$, t.f.a.e.
  \begin{enumerate}
    \item the action is faithful;
    \item for $g\in G$, if $gx=x$ for all $x\in X$ then $g=e$;
    \item the homorphism $\Sigma:G\to\text{Sym}\rr{X}$ is injective;
    \item $\ker\Sigma$ is trivial.
  \end{enumerate}
\end{lemma}

\begin{lemma}
  Let $G$ act faithfully on $X$. Then $G$ is isomorphic to the subgroup
  \begin{align*}
    \im\Sigma = \cc{\bar g : g\in G}
  \end{align*}
  of $\text{Sym}\rr{X}$.
\end{lemma}

\begin{definition}
  Let $G$ act on $X$. Let $S\subseteq G$. The fixed set of $S$ is
  \begin{align*}
    \text{Fix}\rr{S}=\cc{x\in X : sx = x \text{ for all }s\in S}.
  \end{align*}
\end{definition}

\begin{lemma}
  Let $G$ act on $X$, let $S\subseteq G$, and let $g\in G$. Then
  $\text{Fix}\rr{gS\inv g}=g\text{Fix}\rr{S}$.
\end{lemma}

\subsection{Rings}\label{sec:rings}

\begin{lemma}
  Let $R$ be a ring and let $S$ be a set of subrings of $R$. Then their intersection
  $\bigcap S$ is also a subring of $R$.
\end{lemma}

\begin{definition}
  An integral domain is a ring $R$ such that $0\neq 1$ and $r=0$ or $r'=0$ whenever $rr'=0$.
\end{definition}

\begin{lemma}
  Let $R$ be a ring and let $Y=\cc{r_1,\ldots, r_n}$ be a finite subset. Then
  \begin{align*}
    \aa{Y}=\cc{\sum_{j=1}^{ n } a_jr_j : a_1,\ldots,a_n\in\R }.
  \end{align*}
\end{lemma}

\begin{definition}
  An ideal $I\trianglelefteq R$ is principal if, for some $r\in R$, $I=\aa{r}$.
  A principal ideal domain is a ring where every ideal is principal.
\end{definition}

\begin{definition}
  Let $R$ a ring. We say $r | s$ if, for some $a\in R$, $s=ar$. Equivalently, $s\subseteq\aa{r}$
  and $\aa{s}\subseteq\aa{r}$.

  An element $u\in R$ is a unit if it has a multiplicative inverse, or equivalently $\aa{u}=R$.
\end{definition}

\begin{proposition}
  Let $R$ be a principal ideal domain and $r,s\in R$. Then t.f.a.e.
  \begin{enumerate}
    \item $r$ and $s$ are coprime;
    \item $ar+bs = 1$ for some $a,b\in R$.
  \end{enumerate}
\end{proposition}

\end{document}
