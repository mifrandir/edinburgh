\documentclass{article}
\usepackage{homework-preamble}

\title{Randomised Algorithms: Coursework 2}
\author{Franz Miltz}
\begin{document}
\maketitle

Let $G=\rr{V,E}$ be an undirected finite graph with maximum degree $\Delta$, let $\Omega\subseteq\mathcal P\rr{V}$
be the set of independent sets on $G$, and let $\e:\P\rr{V}\to \cc{0,1}^V$ be the bijection
between subsets of $V$ and corresponding binary encodings, i.e. $\e\rr{x}\rr{v}=1$ iff $v\in x$.
We abuse notation and write $x\rr{v}=\e\rr{x}\rr{v}$.


Let $\rr{X_t}_{t\geq 0}$ be the discrete time stochastic process associated with the Glauber dynamics
update. That is, for all $t\geq 0$, $X_t$ takes values in $\Omega$ and an update is
performed in two steps from a state $X_t=x$:
\begin{enumerate}
  \item An element $v\in V$ is chosen uniformly at random.
  \item A new state $y\in\Omega\rr{x,v}$ is chosen uniformly at random.
\end{enumerate}
Here $\Omega\rr{x,v}=\cc{y\in\Omega : \forall w\neq v.\: x\rr{w}=y\rr{w}}$.

\begin{claim*}[1]
  Let $\rr{Y_t}_{t\geq 0}$ be a discrete time stochastic process over $\Omega$ with the
  following transition rule from a given state $Y_t=x$ to a state $Y_{t+1}=y$:
  \begin{enumerate}
    \item An element $v\in V$ is chosen uniformly at random.
    \item For all $w\neq v$, $y\rr{w}=x\rr{w}$.
    \item If there is an $w\in N\rr{v}$ such that $x\rr{w}=1$ then $y\rr{v}=0$.
    \item Else, assign $y\rr{v}=1$ with probability $\lambda/\rr{1+\lambda}$ and $y\rr{v}=0$
      with probability $1/\rr{1+\lambda}$.
  \end{enumerate}
  There exists a $\lambda>0$ such that
  the processes $\rr{X_t}_{t\geq 0}$ and $\rr{Y_t}_{t\geq 0}$ are equivalent.
  \begin{proof}
    Let $x,y\in\Omega$ and $t\geq 0$. Let $x\ominus y= \rr{x\setminus y}\cup \rr{y\setminus x}$.
    Thus we calculate
    \begin{align*}
      \prc{X_{t+1}=y}{X_{t}=x} =
      \begin{cases}
        \sum_{v\in V} \frac{1}{\abs{V}\abs{\Omega\rr{x,v}}} &\text{if }\abs{x\ominus y}=0 \\
        \frac{1}{2\abs{V}} &\text{if } \abs{x\ominus y}=1 \\
        0 &\text{if } \abs{x\ominus y} > 1 \\
      \end{cases}
    \end{align*}
    The $\abs{x\ominus y}=0$ case follows as, for each $v\in V$, there are $\abs{\Omega\rr{x,v}}$
    possible states to transition to but only one of them leads to the desired case.
    In the $\abs{x\ominus y}=1$ the transition requires precisely one vertex $v\in V$ to be selected,
    such that $y\in\Omega\rr{x,v}$. After this has occurred, we note that there are
    $\abs{\Omega\rr{x,v}}=2$ further choices of which precisely one leads to success.
    Finally, if $\abs{x\ominus y} > 1$ then $y\not\in\Omega\rr{x,v}$ for any $v\in V$
    so the transition is impossible.

    We clearly have
    \begin{align*}
      \prc{Y_{t+1}=y}{Y_t=x, \abs{x\ominus y}>1} = \prc{X_{t+1}=y}{X_t=x, \abs{x\ominus y}>1} = 0
    \end{align*}
    as each update only affects at most one vertex. This is independent of the choice of $\lambda$.

    Further, we break the case $\abs{x\ominus y}=1$
    into $x\subset y$ and $x\supset y$. We then find
    \begin{align*}
      \prc{Y_{t+1}=y}{Y_t=x, x\subset y} = \frac{1}{\abs{V}} \frac{\lambda}{1+\lambda}
    \end{align*}
    and
    \begin{align*}
      \prc{Y_{t+1}=y}{Y_t=x, x\supset y} = \frac{1}{\abs{V}} \frac{1}{1+\lambda}.
    \end{align*}
    For equivalence of $\rr{X_t}_{t\geq 0}$ and $\rr{Y_t}_{t\geq 0}$ we now require
    \begin{align*}
      \prc{Y_{t+1}=y}{Y_t=x, x\subset y} = \prc{Y_{t+1}=y}{Y_t=x, x\supset y},
    \end{align*}
    i.e.
    \begin{align*}
      \frac{1}{\abs{V}} \frac{\lambda}{1+\lambda} = \frac{1}{\abs{V}} \frac{1}{1+\lambda}.
    \end{align*}
    Thus we choose $\lambda = 1$. Note that this leads to
    \begin{align*}
      \prc{Y_{t+1}=y}{Y_t=x, x\subset y} = \prc{Y_{t+1}=y}{Y_t=x, x\supset y}=\prc{X_{t+1}=y}{X_t=x, \abs{x\ominus y}=1}.
    \end{align*}

    It remains to show
    \begin{align}
      \label{eq:equal-case}
      \prc{X_{t+1}=y}{X_t=x, x=y}= \prc{Y_{t+1}=y}{Y_t=x, x=y}.
    \end{align}
    We obtain
    \begin{align*}
      \prc{Y_{t+1}=y}{Y_t=x, x=y}
      &= \frac{1}{\abs{V}} \rr{\sum_{v\in x} \frac{\lambda}{\rr{\lambda + 1}} + \sum_{v\in N\rr{x}} 1 + \sum_{v\in V\setminus\rr{x\cup N\rr{v}}} \frac{1}{\lambda+1}} \\
      &= \frac{1}{\abs{V}} \rr{\frac{\abs{x}\lambda}{\lambda + 1} + \abs{N\rr{x}} + \frac{\abs{V}-\abs{x}-\abs{N\rr{x}}}{\lambda + 1}} \\
      &= \frac{1}{\abs{V}} \rr{\abs{N\rr{x}} + \frac{\abs{V}-\abs{N\rr{x}}}{2}}
    \end{align*}
    where $N\rr{x}=\cc{v \in V : \exists u\in x. \: v\in N\rr{u}}$. This may be derived as follows:
    In the case where $v\in x$ the condition in (3.) fails and we require $y\rr{v}=1$ so we have
    a probability of success $\lambda/\rr{1+\lambda}$. In the case where $v\in N\rr{x}$ (note $x$
    is an independent set so $x\cap N\rr{x}=\emptyset$) the condition in the third step will
    always hold, leading to a guaranteed success. Finally, if $v\not\in x$ and $v\not\in N\rr{x}$
    then we require $y\rr{v}=0$ so the probability is $1/\rr{\lambda + 1}$.

    Now (\ref{eq:equal-case}) is equivalent to
    \begin{align}
      \label{eq:explicit-equal-case}
      \sum_{v\in V} \frac{1}{\abs{V}\abs{\Omega\rr{x,v}}} = \frac{1}{\abs{V}} \rr{\abs{N\rr{x}} + \frac{\abs{V}-\abs{N\rr{x}}}{2}}.
    \end{align}
    We now observe
    \begin{align*}
      \abs{\Omega\rr{x,v}} =
      \begin{cases}
        1 & \text{if } v\in N\rr{x} \\
        2 & \text{otherwise}
      \end{cases}
    \end{align*}
    I.e.
    \begin{align*}
      \sum_{v\in V} \frac{1}{\abs{\Omega\rr{x,v}}} = \sum_{v\in N\rr{x}} 1 + \sum_{v\in V\setminus N\rr{x}} \frac{1}{2} =  \abs{N\rr{x}} + \frac{\abs{V}-\abs{N\rr{x}}}{2}
    \end{align*}
    which proves (\ref{eq:explicit-equal-case}) and thus (\ref{eq:equal-case}).
  \end{proof}
\end{claim*}

Let $\pi:\mathcal P\rr{V}\to\bb{0,1}$ be the probability distribution defined by
\begin{align*}
  \pi\rr{x}=
  \begin{cases}
    \frac{\lambda^{\abs{x}}}{Z\rr{\lambda}} &\text{if }\forall \cc{v,w}\in E.\:x\rr{v}x\rr{w}=0 \\
    0 &\text{otherwise}
  \end{cases}
\end{align*}
where $Z\rr{\lambda}=\sum_{x\in\cc{0,1}^V} \lambda^{\abs{x}}$.



\begin{claim*}[2]
  $\pi$ satisfies the detailed balance condition for $\rr{X_t}$.
  I.e. for all $x,y\in\Omega$,
  \begin{align*}
    \pi\rr{x}P\rr{x,y}=\pi\rr{y}P\rr{y,x}.
  \end{align*}
  \begin{proof}
    For all $t$ and $x,y\in\Omega$, write
    \begin{align*}
      P\rr{x,y}=\prc{X_{t+1}=y}{X_t=x}.
    \end{align*}

    Let $x,y\in\Omega$. If $x=y$ then the claim is immediate. If $\abs{x\ominus y}>1$ then
    \begin{align*}
      P\rr{x,y}=P\rr{y,x}=0.
    \end{align*}
    Finally, consider the case $\abs{x\ominus y}=1$. Note $\lambda=1$. Thus
    \begin{align*}
      \pi\rr{x}P\rr{x,y}=\rr{\frac{1^{\abs{x}}}{\sum_{z\in\Omega} 1^{\abs{z}}}}\rr{\frac{1}{2\abs{V}}}
      =\rr{\frac{1^{\abs{y}}}{\sum_{z\in\Omega} 1^{\abs{z}}}}\rr{\frac{1}{2\abs{V}}}
      =\pi\rr{y}P\rr{y,x}.
    \end{align*}
  \end{proof}
\end{claim*}

Let $\rr{Z_t}=\rr{X_t,Y_t}$ be coupled Markov chains that transition
based on the Glauber dynamics. In particular, define the update rule for $\rr{X_t,Y_t}$
from a state $\rr{x,y}$ to a state $\rr{x',y'}$:
\begin{enumerate}
  \item An element $v\in V$ is chosen uniformly at random.
  \item For all $w\neq v$, $x'\rr{w}=x\rr{w}$ and $y'\rr{w}=y\rr{w}$.
  \item A bit $c\in\cc{0,1}$ is chosen uniformly at random.
  \item If there is a $w\in N\rr{v}$ such that $x\rr{w}=1$ then $x'\rr{v}=0$.
  \item Else, assign $x'\rr{v}=c$.
  \item If there is a $w\in N\rr{v}$ such that $y\rr{w}=1$ then $y'\rr{v}=0$.
  \item Else, assign $y'\rr{v}=c$.
\end{enumerate}
It is straightforward to see that this is a valid coupling, i.e. $\rr{X_t}$ and $\rr{Y_t}$
follow the correct distribution locally. For each $t$, let $W_t$ be the random variable
corresponding to the vertex selected in the transition from $Z_t$ to $Z_{t+1}$ and let
$C_t$ be random variable corresponding to the value of $c$ in that transition.

Moreover, for independent sets $x,y\in\Omega$, define their distance by
\begin{align*}
  \rho\rr{x,y} = \sum_{v\in V} \abs{x\rr{v}-y\rr{v}}.
\end{align*}
We immediately note $\rho\rr{x,y}=\abs{x\ominus y}$.

\begin{claim*}[3a]
  \begin{align*}
    \prc{\rho\rr{X_{t+1},Y_{t+1}}=0}{X_t\ominus Y_t=\cc{W_t}}=1.
  \end{align*}
  \begin{proof}
    Let $v\in V$ and let $x,x',y,y'\in\Omega$ such that $x\ominus y=\cc{v}$ and such that
    \begin{align*}
      \prc{Z_{t+1}=\rr{x',y'}}{Z_t=\rr{x,y},W_t=v}>0.
    \end{align*}
    Note that, for all $w\neq v$, $x\rr{w}=y\rr{w}$. Thus we have $x'\rr{w}=y'\rr{w}$.
    Now consider $x'\rr{v}$ and $y'\rr{v}$. Observe that the checks in steps 4 and 6 will
    both fail as both $x$ and $y$ are independent sets. Thus, for some $c\in\cc{0,1}$,
    $x'\rr{v}=c=y'\rr{v}$. Now, for all $w\in V$, $x'\rr{w}=y'\rr{v}$, i.e. $\rho\rr{x',y'}=0$.
  \end{proof}
\end{claim*}

\begin{claim*}[3b]
  For all $v\in V$,
  \begin{align*}
    \prc{\rho\rr{X_{t+1},Y_{t+1}}=1}{X_t\ominus Y_t=\cc{v},W_t\not\in N\rr{v}\cup\cc{v}}=1.
  \end{align*}
  \begin{proof}
    Let $v,w\in V$ and let $x,x',y,y'\in\Omega$ such that $x\ominus y=\cc{v}$, $w\not\in N\rr{v}\cup\cc{v}$, and
    \begin{align*}
      \prc{Z_{t+1}=\rr{x',y'}}{Z_t=\rr{x,y},W_t=w}>0.
    \end{align*}
    Again, we need only consider $x'\rr{w}$ and $y'\rr{w}$. Clearly $x\rr{w}=y\rr{w}$, so
    we require $x'\rr{w}=y'\rr{w}$. For a contradiction, assume $x'\rr{w}\neq y'\rr{w}$. Without loss of generality
    assume $x'\rr{w}=0$. Then it must be the case that there is some $u\in N\rr{w}$ such that
    $x\rr{u}=1$. Fix this $u$. We note that in order to achieve $x\rr{w}\neq y\rr{w}$ we must
    have sampled $c=1$ and the check in 6 must have failed. I.e., for all $u'\in N\rr{w}$,
    $y\rr{u'}=0$. In particular, $x\rr{u}=1$ and $y\rr{u}=0$. This implies $x\rr{u}\neq y\rr{u}$
    and thus $u=v$. However, $u\in N\rr{w}$, so $v\in N\rr{w}$, and then $w\in N\rr{v}$.
    This contradicts the premise. We conclude $x'\rr{w}=y'\rr{w}$.
  \end{proof}
\end{claim*}

\begin{claim*}[3c]
  For all $v\in V$,
  \begin{align*}
    \prc{\rho\rr{X_{t+1},Y_{t+1}}=2}{X_t\ominus Y_t=\cc{v},W_t\in N\rr{v}}<1/2.
  \end{align*}
  \begin{proof}
    Let $v,w\in V$ and let $x,x',y,y'\in\Omega$ such that $x\ominus y=\cc{v}$,
    $w\in N\rr{v}$, and
    \begin{align*}
      \prc{Z_{t+1}=\rr{x',y'}}{Z_t=\rr{x,y},W_t=w}>0.
    \end{align*}
    Once more we consider $x'\rr{w}$ and $y'\rr{w}$ and note $\rho\rr{x', y'}=2$
    iff $x'\rr{w}\neq y'\rr{w}$.

    Consider the case $v\in x$ and $v\not\in y$. Then the check in step 4 of the update
    will succeed so $x'\rr{w}=0$. The case $y'\rr{w}=1$ occurs only if the check in step
    6 fails and we sample $c=1$. I.e., for all $v\in V$,
    \begin{align*}
      &\prc{Y_{t+1}=1}{X_t\neq Y_t, X_t=Y_t\cup\cc{v},W_t\in N\rr{v}}\\
      &= \prc{C_t=1\cap\rr{\exists w\in N\rr{v}. \: y\rr{w}=1}}{X_t\neq Y_t, X_t=Y_t\cup\cc{v},W_t\in N\rr{v}}
    \end{align*}
    We note independence of the events on the left so, for all $v\in V$,
    \begin{align*}
      &\prc{Y_{t+1}=1}{X_t\neq Y_t, X_t=Y_t\cup\cc{v},W_t\in N\rr{v}}\\
      &=\pr{C_t=1}\prc{\exists w\in N\rr{v}. \: y\rr{w}=1}{X_t\neq Y_t, X_t=Y_t\cup\cc{v},W_t\in N\rr{v}}\\
      &\leq\pr{C_t=1} = 1/2.
    \end{align*}
  \end{proof}
\end{claim*}

\end{document}
