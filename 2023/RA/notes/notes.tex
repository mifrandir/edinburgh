\documentclass{article}
\usepackage{notes-preamble}
\usepackage{enumitem}
\mkthmstwounified

\title{Randomised Algorithms (SEM7)}
\author{Franz Miltz}
\begin{document}
\maketitle
\tableofcontents
\pagebreak

\section{Probability}


\subsection{Moments}

\begin{definition}
  \label{def:expectation}
  Let $X$ be a random variable. Then $X$ has finite expectation iff
  \begin{align*}
    \sum_{x} \abs{x} \pr{X=x} < \infty
  \end{align*}
  and we write
  \begin{align*}
    \ex{X} = \sum_{x} x \pr{X=x}.
  \end{align*}
\end{definition}

\begin{definition}
  \label{def:moment}
  The $k$-th moment of a random variable $X$ is $\ex{X^k}$.
\end{definition}

\begin{definition}
  \label{def:variance}
  Let $X$ be a random variable. Then the variance is given by
  \begin{align*}
    \var{X}=\ex{\rr{X-\ex{X}}^2} = \ex{X^2}-\rr{\ex{X}}^2
  \end{align*}
\end{definition}

\begin{definition}
  \label{def:covariance}
  The covariance of two random variables $X$ and $Y$ is
  \begin{align*}
    \cov{X,Y}=\ex{\rr{X-\ex{X}}\rr{Y-\ex{Y}}}.
  \end{align*}
\end{definition}

\begin{theorem}
  For any two random variables $X,Y$,
  \begin{align*}
    \var{X+Y}=\var{X}+\var{Y}+2\cov{X,Y}
  \end{align*}
\end{theorem}

\begin{theorem}
  If $X,Y$ are independent random variables, then
  \begin{align*}
    \ex{XY}=\ex{X}\ex{Y}.
  \end{align*}
\end{theorem}

\begin{corollary}
  If $X,Y$ are independent random variables, then
  \begin{align*}
    \cov{X,Y}=0
  \end{align*}
  and
  \begin{align*}
    \var{X+Y}=\var{X}+\var{Y}.
  \end{align*}
\end{corollary}

\begin{theorem}
  Let $X_1,...,X_n$ be mutually independent random variables. Then
  \begin{align*}
    \var{\sum_{i=1}^{n} X_i}=\sum_{i=1}^{n} \var{X_i}.
  \end{align*}
\end{theorem}

\begin{theorem}
  Let $X$ be a random variable with finite expectation and finite median $m$. Then
  \begin{enumerate}
    \item the expectation is the value of $c$ that minimises the expression $\ex{\rr{X-c}^2}$;
    \item the median is the value of $c$ that minimises the expression $\ex{\abs{X-c}}$.
  \end{enumerate}
\end{theorem}

\begin{theorem}
  If $X$ is a random variable with finite standard deviation $\sigma$, expectation $\mu$ and
  median $m$, then
  \begin{align*}
    \abs{\mu-m}\leq\sigma.
  \end{align*}
\end{theorem}

\subsection{Moment generating function}\label{sec:moment-generating-function}

\begin{definition}\label{def:moment-generating-function}
  The moment generating function of a random variable $X$ is
  \begin{align*}
    M_X\rr{t}=\ex{\exp\rr{tX}}.
  \end{align*}
\end{definition}

\begin{theorem}
  Let $X$ be a random variable and $n>1$. Then
  \begin{align*}
    \ex{X^n}=\frac{d^n}{dt^n}\rr{M_X\rr{t}}.
  \end{align*}
\end{theorem}

\begin{theorem}
  Let $X,Y$ be random variables. If there exists $\delta>0$ such that, for all $t\in\rr{-\delta,\delta}$,
  $M_X\rr{t}=M_Y\rr{t}$, then $X$ and $Y$ have the same distribution.
\end{theorem}

\begin{theorem}
  Let $X,Y$ be independent random variables. Then
  \begin{align*}
    M_{X+Y}\rr{t}=M_X\rr{t}M_Y\rr{t}.
  \end{align*}
\end{theorem}

\subsection{Special random variables}

\begin{definition}
  \label{def:binomial}
  A random variable $X$ is binomial with parameters $n$ and $p$ iff,
  for all $0\leq j\leq n$,
  \begin{align*}
    \pr{X=j}=\binom{n}{j}p^j \rr{1-p}^{n-j}.
  \end{align*}
\end{definition}

\begin{lemma}
  The variance of a geometric random variable with parameter $p$ is $\rr{1-p}/p^2$.
\end{lemma}

\begin{definition}
  \label{def:geometric}
  A random variable $X$ is geometric with parameter $P$ iff, for all $n\in\N$,
  \begin{align*}
    \pr{X=n}=\rr{1-p}^{n-1} p
  \end{align*}
\end{definition}

\begin{lemma}
  For a geometric random variable $X$ with parameter $p$ and for $n>0$,
  \begin{align*}
    \pr{X=n+k}{X>k} = \pr{X=n}.
  \end{align*}
\end{lemma}

\begin{lemma}
  Let $X$ be a discrete random variable that takes on only nonnegative integer values. Then
  \begin{align*}
    \ex{X} = \sum_{i=1}^{\infty} \pr{X\geq i}
  \end{align*}
\end{lemma}

\begin{lemma}
  The harmonic number $H\rr{n}=\sum_{i=1}^{n} 1/i$ staisfies $H\rr{n}=\ln n+\Theta\rr{1}$.
\end{lemma}

\subsection{Inequalities}

\begin{definition}
  \label{def:convex}
  A function $f:\R\to\R$ is said to be convex if, for any $x_1,x_2\in\R$
  and $\lambda\in\bb{0,1}$,
  \begin{align*}
    f\rr{\lambda x_1 + \rr{1-\lambda}x_2} \leq \lambda f\rr{x_1} + \rr{1-\lambda}f \rr{x_2}.
  \end{align*}
\end{definition}

\begin{theorem}[Jensen's inequality]
  \label{thm:jensens}
  Let $X$ be a random variable and let $f:\R\to\R$ be a convex function. Then
  \begin{align*}
    \ex{f\rr{X}} \geq f\rr{\ex{X}}.
  \end{align*}
\end{theorem}

\begin{theorem}[Chebyshev's inequality]
  Let $X$ be a random variable and $a>0$. Then
  \begin{align*}
    \pr{\abs{X-\ex{X}}\geq a}\leq \frac{\var{X}}{a^2}.
  \end{align*}
\end{theorem}

\begin{corollary}
  For any $t>1$,
  \begin{align*}
    \pr{\abs{X-\ex{X}}\geq t\sqrt{\var{X}}}\leq \frac{1}{t^2}, \hs
    \pr{\abs{X-\ex{X}}\geq t\ex{X}}\leq \frac{\var{X}}{t^2\rr{\ex{X}}^2}.
  \end{align*}
\end{corollary}

\begin{theorem}[Chernoff bounds]
  Let $X_1,...,X_n$ be independent Poisson trials such that $\pr{X_i=1}=p_i$.
  Let $X=\sum_{i=1}^{n} X_i$ and $\mu=\ex{X}$. Then the following hold:
  \begin{enumerate}
    \item For any $\delta>0$, \begin{align*}
        \pr{X\geq \rr{1+\delta}\mu} \leq \rr{\frac{\exp\rr{\delta}}{\rr{1+\delta}^{1+\delta}}}^\mu
      \end{align*}
    \item For any $\delta\in\rr{0,1}$, \begin{align*}
        \pr{X\leq \rr{1-\delta}\mu} \leq \rr{\frac{\exp\rr{-\delta}}{\rr{1-\delta}^{1-\delta}}}^\mu
      \end{align*}
    \item For any $\delta\in\rb{0,1}$, \begin{align*}
        \pr{X\geq \rr{1+\delta}\mu}\leq \exp\rr{-\mu\delta^2/3}.
      \end{align*}
    \item For any $\delta\in\rr{0,1}$, \begin{align*}
        \pr{X\leq \rr{1-\delta}\mu}\leq \exp\rr{-\mu\delta^2/2}.
      \end{align*}
    \item For any $\delta\in\rr{0,1}$, \begin{align*}
        \pr{\abs{X-\mu}\geq \delta\mu}\leq 2\exp\rr{-\mu\delta^2/3}.
      \end{align*}
    \item For any $R\geq 6\mu$, \begin{align*}
        \pr{X\geq R}\leq 2^{-R}.
      \end{align*}
  \end{enumerate}
\end{theorem}

\begin{theorem}[Chernoff bounds]
  Let $X_1,...,X_n$ be independent random variables with
  \begin{align*}
    \pr{X_i=1}=\pr{X_i=-1}=1/2.
  \end{align*}
  Let $X=\sum_{i=1}^{n} X_i$. For any $a>0$,
  \begin{align*}
    \pr{X\geq a}\leq \exp\rr{-a^2/2n},
  \end{align*}
  and
  \begin{align*}
    \pr{\abs X\geq a}\leq \exp\rr{-a^2/2n}
  \end{align*}
\end{theorem}

\begin{corollary}
  Let $Y_1,...,Y_n$ be independent random variables with
  \begin{align*}
    \pr{Y_i=1}=\pr{Y_i=0}=1/2.
  \end{align*}
  Let $Y=\sum_{i=1}^{n} Y_i$ and $\mu=\ex{Y}=n/2$.
  \begin{enumerate}
    \item For any $a>0$, \begin{align*}
        \pr{Y\geq \mu+a}\leq \exp\rr{-2a^2/n}.
      \end{align*}
    \item For any $a\in\rr{0,\mu}$, \begin{align*}
        \pr{Y\leq \mu-a}\leq \exp\rr{-2a^2/n}.
      \end{align*}
    \item For any $\delta>0$, \begin{align*}
        \pr{Y\geq \rr{1+\delta}\mu}\leq \exp\rr{-\delta^2\mu}.
      \end{align*}
    \item For any $\delta\in\rr{0,1}$, \begin{align*}
        \pr{Y\leq \rr{1-\delta}\mu}\leq \exp\rr{-\delta^2\mu}.
      \end{align*}
  \end{enumerate}
\end{corollary}

\begin{theorem}[Hoeffding bound]
  Let $X_1,...,X_n$ be independent random variables such that for all $1\leq i\leq n$,
  $\ex{X_i}=\mu$ and $\pr{a\leq X_i\leq b}=1$. Then
  \begin{align*}
    \pr{\abs{\frac{1}{n}\sum_{i=1}^{n} X_i-\mu}\geq \e}\leq 2\exp\rr{-\frac{2n\e^2}{\rr{b-a}^2}}
  \end{align*}
\end{theorem}

\begin{theorem}
  Let $X_1,...,X_n$ be independent random variables with $\ex{X_i}=\mu_i$ and $\pr{a_i\leq X_i\leq b_i}=1$ for
  constants $a_i,b_i$. Then
  \begin{align*}
    \pr{\abs{\sum_{i=1}^{n} X_i - \sum_{i=1}^{n} \mu_i}\geq\e} \leq 2\exp\rr{-\frac{2\e^2}{\sum_{i=1}^{n} \rr{b_i-a_i}^2}}.
  \end{align*}
\end{theorem}

\subsection{Balls into bins}

\begin{lemma}
  Let $n$ balls be thrown independently and uniformly at random into $n$ bins. Then, for sufficiently
  large $n$, the maximum load is bounded above by
  \begin{align*}
    \frac{3\ln{n}}{\ln^2{n}}
  \end{align*}
  with probability at least $1-1/n$.
\end{lemma}

\begin{lemma}
  Let $n$ balls be thrown sequentially into $n$ bins. For each ball, generate two choices of bins
  and assign it to the bin with the lower load. Then the maximum load is
  \begin{align*}
    \frac{\ln^2 n}{\ln 2}+O\rr{1}
  \end{align*}
  with probability $1-o\rr{1/n}$.
\end{lemma}

\subsection{Probabilistic method}

\begin{theorem}[Ramsay]
  For any positive integer $k$, there is a positive integer $n$ such that any undirected
  graph with at least $n$ vertices has a $k$-clique or a $k$-independent-set.
\end{theorem}

\begin{definition}
  A random graph $G=\rr{V,E}$ sampled from $G_{n,p}$ is obtained as follows:
  \begin{itemize}
    \item $G$ has $n=\abs{V}$ nodes.
    \item For each of the $\binom{n}{2}$ possible pairs, $\cc{u,v}$ with $u,v\in V$ and
      $u\neq v$, assign $\cc{u,v}\in E$ with independent probability $p$.
  \end{itemize}
\end{definition}

\begin{definition}
  A dependency graph for a set of events $E_1,...,E_n$ is a directed graph $G=\rr{V,E}$
  such that $V=\cc{1,...,n}$ and for each $i=1,...,n$ the event $E_i$ is mutually
  independent of all events $E\setminus N\rr{i}$.
\end{definition}

\begin{theorem}[Locasz Local Lemma]
  Let $E_1,..,E_n$ be a set of events. Suppose that for some $p\in \rr{0,1}$ and some
  $d\in\N$ the following hold:
  \begin{enumerate}
    \item For all $1\leq i\leq n$, $\pr{E_i}\leq p$.
    \item A dependency graph on $\cc{E_1,...,E_n}$ has degree $\leq d$.
    \item $4dp\leq 1$.
  \end{enumerate}
  Then
  \begin{align*}
    \pr{\bigcup_{i=1}^N \overline{E_i}} > 0.
  \end{align*}
\end{theorem}

\section{Algorithms}
\label{sec:algorithms}

\subsection{Testing polynomial equalities}

\begin{algorithm}\label{alg:stan-ulman}
  Consider polynomials
  \begin{align*}
    F \rr{ x } = \prod_{i=1}^n \rr{ x - a_i }, \hs G \rr{ x } = \sum_{ i=0 }^{ n } b_i x^i.
  \end{align*}
  We randomly check for equality as follows:
  \begin{enumerate}
    \item Choose an integer $x\in \cc{ 1, 100n }$.
    \item Calculate $F \rr{ x }$. Overall at most $n$ additions and $2n$ multiplications.
    \item Calculate $G \rr{ x }$. Overall at most $n$ additions and $2n$ multiplications.
    \item Compare the result, return $\bot$ or $\top$ accordingly.
  \end{enumerate}
\end{algorithm}


\begin{theorem}
  \label{thm:stan-ulman-probabilities}
  The algorithm~\ref{alg:stan-ulman} has the following probabilities of success:
  \begin{itemize}
    \item $P \rr{ \top | F\equiv G } = 1$
    \item $P \rr{ \bot | F\not\equiv G } \geq 99/100$
  \end{itemize}
  One can improve these by using
  \begin{itemize}
    \item repeated sampling, or
    \item sampling from a larger set.
  \end{itemize}
\end{theorem}

\subsection{Verifying matrix multiplication}

\begin{algorithm}\label{alg:mmverify}
  Let $R$ be a ring, $n,k\in\N$, and $A,B,C\in\Mat \rr{ n; R }$. Then
  $\texttt{MMVerify} \rr{ n, A, B, C, k }$ is the following algorithm:
  \begin{enumerate}
    \item For $j=1,...,k$ \begin{enumerate}
      \item Generate a vector $x\in \cc{ 0,1 }^n$ u.a.r.
      \item Calculate $y_B = Bx$ in $O(n^2)$ time.
      \item Calculate $y_{AB} = Ay^B$ in $O(n^2)$ time.
      \item Calculate $y_C = Cx$ in $O(n^2)$ time.
      \item Compare $y_{AB}$ with $y_C$ in $O(n)$ time and return $\bot$ or $\top$ accordingly.
    \end{enumerate}
\end{enumerate}
\end{algorithm}

\begin{theorem}
  \label{thm:mmverify-proabilities}
  The algorithm~\ref{alg:mmverify} has the following probabilities of success:
  \begin{itemize}
    \item $\prc{ \top}{A = B } = 1$
    \item $\prc{ \top}{A \not= B } = 2^{-k}$
  \end{itemize}
\end{theorem}

\subsection{Minimum cut}

\begin{algorithm}
  \label{alg:krager}
  Let $G= \rr{ V,E }$ a graph and let $n= \abs{ V }$ and $m= \abs{ E }$. Then let
  $\texttt{KragerOneTrial} \rr{ G }$ be the following algorithm: Repeatedly choose an edge
  uniformly at random (from the not-yet contracted edges) and contract its endpoints.
  When there are just two “vertices” left, return that cut.

  Let $ \texttt{Krager} \rr{ G, k }$ be $k$ repetition of the above, taking the minimum
  result.
\end{algorithm}

\begin{theorem}
  \label{thm:krager-probabilities}
  The probability of finding a minimum cut in one iteration of $ \texttt{KragerOneTrial} \rr{ G }$
  is
  \begin{align*}
    p \geq \frac{2}{n \rr{ n-1 }}.
  \end{align*}
  The probability of finding a minimum cut using $ \texttt{Krager} \rr{ G,k }$ is
  \begin{align*}
    1-\rr{ 1-p }^k \geq \frac{1}{n^c} \hs \text{where} \hs c = \frac{2k}{n(n-1)\ln n}.
  \end{align*}
\end{theorem}

\subsection{Cliques and independent sets}

\begin{theorem}
  If $p\rr{n}=o\rr{n^{-2/3}}$ then
  \begin{align*}
    \lim_{n\to\infty} \pr{\text{$G\to G_{n,p\rr{n}}$ has a 4-clique}}=0.
  \end{align*}
  If $p\rr{n}=\omega\rr{n^{-2/3}}$ then
  \begin{align*}
    \lim_{n\to\infty} \pr{\text{$G\to G_{n,p\rr{n}}$ has a 4-clique}}=1.
  \end{align*}
\end{theorem}

\subsection{MAXSAT}\label{sec:maxsat}

\begin{theorem}
  Given a CNF boolean formula with $m$ clauses, where each clause contains at least $k$ literals,
  there exists a truth assignment to the variables that satisfies at least $m(1-2^{-k})$ clauses.
\end{theorem}

\begin{lemma}
  There exists a polynomial-time $7/8$-approximation algorithm for MAX-E-3SAT.
\end{lemma}

\begin{theorem}[Hastad]
  If for any $\e>0$ there exists a polynomial-time $(7/8+\e)$-approximation algorithm for
  MAX-E-3SAT then $P=NP$.
\end{theorem}

\begin{theorem}[Moser]
  If no variable in a $k$-CNF formula $\phi$ appears in more than $2^k/4k$ clauses then
  $\phi$ is satisfiable.

  Moreover, there exists a Las Vegas randomised algorithm that given such a $\phi$ runs in
  expected polynomial time and outputs a satisfying assignment to $\phi$.
\end{theorem}

\begin{algorithm}
  The following algorithm takes as input a list of clauses $C_1,...,C_m$ of a $k$-CNF
  formula $\phi$ over $n$ variables $x_1,\ldots,x_n$ and outputs a satisfying truth
  assignment for $\phi$.

  \begin{enumerate}
    \item Sample a truth assignment $\alpha$ to the variables u.a.r
    \item While some $C_i$ is not satisfied by $\alpha$
      \begin{enumerate}
        \item choose the first unsatisfied $C_i$
        \item call the routine below with input $\alpha$ and $i$
        \item overwrite $\alpha$ with the output of the routine below
      \end{enumerate}
  \end{enumerate}
  The algorithm uses the following helper routine with input $\alpha$, a truth assignment
  to the variables, and an $1\leq i\leq m$.
  \begin{enumerate}
    \item Resample a truth assignment $\alpha'$ which differs from $\alpha$ only in the variables
      in $C_i$ which are sampled u.a.r
    \item While some $C_j$ that shares a variable with $C_i$ is not satisfied by $\alpha'$
      \begin{enumerate}
        \item choose the first such unsatisfied $C_j$
        \item call this routine with input $\alpha'$ and $j$
        \item overwrite $\alpha'$ with the ouptut of the routine
      \end{enumerate}
    \item return $\alpha'$
  \end{enumerate}
\end{algorithm}

\end{document}
