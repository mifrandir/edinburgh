\documentclass{article}
\usepackage{notes-preamble}
\usepackage{enumitem}
\begin{document}
\mkthmstwounified
\title{Fourier Analysis (SEM8)}
\author{Franz Miltz}
\maketitle
\tableofcontents
\pagebreak

\section{Prerequisites}

\begin{definition}
  The complex exponential function $\exp : \C\to\C$ is defined by
  \begin{align*}
    \exp x = \sum_{k=0}^{ \infty } \frac{z^k}{k!}.
  \end{align*}
\end{definition}

\subsection{Function spaces}


\begin{definition}
  A \emph{trigonometric polynomial} is a function $p:\R\to\C$ of the form
  \begin{align*}
    p\rr{x} = \sum_{k=-N}^N c_k e^{-2\pi i k x}
  \end{align*}
  for some $c_{-N},\ldots,c_N\in\C$.

  Let $\mathcal P\rr{\mathbb T}\subseteq\mathcal L^1\rr{\mathbb T}$ denote the vector space
  of all trigonometric polynomials.
\end{definition}

\begin{definition}
  Let $\mathcal L^p\rr{\mathbb T}$ denote the space of functions
  $f:\mathbb T\to\mathbb C$ such that $x\mapsto\vv{f\rr{x}}^p$ is integrable.
\end{definition}

\begin{lemma}
  $\mathcal P\rr{\mathbb T}\subset C\rr{\mathbb T}\subset \mathcal L^2\rr{\mathbb T}\subset\mathcal L^1\rr{\mathbb T}$.
\end{lemma}

\begin{lemma}
  The class of trigonometric polynomials $\mathcal P\rr{\mathbb T}$ is dense in the spaces
  \begin{enumerate}
    \item $C(\mathbb T)$ wrt $\vabs{-}_{\infty}$;
    \item for $p=1,2$, $\mathcal L^p(\mathbb T)$ wrt $\vabs{-}_{p}$.
  \end{enumerate}
\end{lemma}

\begin{definition}
  The normed linear space $L^p$ is defined by
  \begin{align*}
    \vabs{f}_{L^p} = \rr{\int_{\mathbb T} \vv{f\rr{x}}^p dx}^{1/p},\hs
    L^p = \mathcal L^p / \cc{f\in\mathcal L^p : \vabs{f}_{L^p} = 0}.
  \end{align*}
\end{definition}

\begin{theorem}
  Let $f\in\mathcal L^1\rr{\R}$. Then
  \begin{itemize}
    \item $(x\mapsto f(-x))\in\mathcal L^1\rr{\R}$.
    \item $\vv{f}\in\mathcal L^1\rr{\R}$.
    \item For all $g\in\mathcal L^{\infty}\rr{\R}$, $fg\in\mathcal L^1\rr{\R}$.
    \item For all $h\in\R$, $(x\mapsto f(x-h))\in\mathcal L^1\rr{\R}$.
    \item For all $t>0$, $(x\mapsto \inv t f(\inv t x))\in\mathcal L^1\rr{\R}$.
  \end{itemize}
\end{theorem}

\begin{theorem}
  Let $f\in\mathcal L^1\rr{\R}$. Then
  \begin{itemize}
    \item For $g\in\mathcal L^{\infty}\rr{\R}$ and $a,b\in\C$,
      \begin{align*}
        \int_R af(x) + bg(x) = a\int_\R f(x)dx + b\int_\R g(x) dx.
      \end{align*}
    \item For all $h\in\R$,
      \begin{align*}
        \int_\R f(x-h)dx = \int_\R f(x)dx.
      \end{align*}
    \item For all $t>0$,
      \begin{align*}
        \int_\R \inv tf(\inv tx) dx = \int_\R f(x) dx.
      \end{align*}
    \item For all $g\in\mathcal L^{\infty}\rr{\R}$,
      \begin{align*}
        \int_\R \vv{f(x)g(x)} dx \leq \rr{\text{ess sup } g}
        \int_\R \vv{f(x)} dx.
      \end{align*}
    \item \begin{align*}
        \vv{\int_\R f(x) dx}\leq \int_\R \vv{f(x)} dx.
      \end{align*}
    \item \begin{align*}
        \int_\R f(-x) dx = \int_\R f(x) dx.
      \end{align*}
  \end{itemize}
\end{theorem}

\subsection{Convolution}

\begin{definition}
  Consider an interval $I\subseteq\R$ and $f,g\in\mathcal L^1\rr{I}$.
  The \emph{convolution $f*g\in L^1(I)$} is defined by
  \begin{align*}
    f*g(x) = \int_I f(x-t)g(t)dt.
  \end{align*}
\end{definition}

\begin{lemma}
  If $f,g\in\mathcal L^1\rr{\R}$, at least one of them bounded, then
  $f*g$ is continuous.
\end{lemma}

\begin{lemma}
  If $f,g\in\mathcal L^1\rr{\mathbb T}$, at least one of them bounded, then
  $f*g$ is uniformly continuous.
\end{lemma}

\begin{theorem}[Young]
  Let $f\in\mathcal L^1\rr{\R}$ and $g\in\mathcal L^{p}\rr{\R}$
  for some $1\leq p\leq\infty$. Then $f*g\in L^p\rr{\R}$ and
  \begin{align*}
    \vabs{f*g}_{L^p(\R)} \leq \vabs{f}_{L^1\rr{\R}}\vabs{g}_{L^p(\R)}.
  \end{align*}
\end{theorem}

\begin{theorem}
  For all $f,g\in\mathcal L^1\rr{\R}$,
  \begin{align*}
    \int_\R f*g = \rr{\int_\R f}\rr{\int_\R g}.
  \end{align*}
\end{theorem}

\section{Fourier series}

\begin{definition}
  The \emph{Fourier coefficients} of $f\in\mathcal L^1\rr{\mathbb T}$ are given by
  \begin{align}\label{eq:fourier_coefficients}
    \hat f\rr{k} = \int_{\mathbb T} f\rr{x} e^{-2\pi i k x} dx,
  \end{align}
  the \emph{Fourier series of $f$ at $x\in\R$} is given by the formal expression
  \begin{align*}
    \sum_{k=-\infty}^\infty \hat f\rr{k} e^{2\pi i k x},
  \end{align*}
  and the \emph{$n$th partial sum of the Fourier series of $f$ at $x\in\R$} is
  given by
  \begin{align*}
    S_N f\rr{x} = \sum_{k=-N}^N \hat f\rr{k}e^{2\pi i k x}.
  \end{align*}
\end{definition}

\subsection{Properties}

\begin{lemma}
  Let $p\rr{x}=\sum_{k=-N}^{ N } c_k e^{2\pi i k x}$ be a trigonometric polynomial.
  \begin{enumerate}
    \item The Fourier coefficients $\hat p(k)$ are precisely the $c_k$
      for $k\leq\vv{N}$ and otherwise zero;
    \item $\sum_{k\in\Z}\vv{\hat p\rr{k}}^2 = \int_{\mathbb T}\vv{p\rr{x}}^2 dx$.
  \end{enumerate}
\end{lemma}

\begin{lemma}[Riemann-Lebesgue]
  Let $f\in\mathcal L^1\rr{\mathbb T}$. Then $\hat f\rr{k}\to 0$ as $k\to \pm\infty$.
\end{lemma}

\begin{proposition}[Weierstrass M-test]\label{prop:weierstrass_m_test}
  Suppose $A\subseteq\br{0,1}$ and $F_N\rr{x}=\sum_{k=1}^{ N } f_k\rr{x}$
  where the $f_k:A\to\C$ satisfy
  \begin{align*}
    \sup_{x\in A}\vv{f_k\rr{x}} \leq M_k,\hs \sum_{k=1}^{ \infty } M_k < \infty.
  \end{align*}
  Then
  \begin{enumerate}
    \item $F_N$ converges uniformly on $A$ to some $F:A\to\C$;
    \item if each $f_k$ is continuous at some point $x_0\in A$,
      then $F$ is continuous at $x_0\in A$.
  \end{enumerate}
\end{proposition}

\begin{lemma}
  Let $f,g\in\mathcal L^1\rr{\mathbb T}$ be two functions such that, for all $k\in\Z$, $\hat f\rr{k}=\hat g\rr{k}$.
  Then at all points of continuity of $f-g$ we have $f\rr{x}=g\rr{x}$.
\end{lemma}

\begin{theorem}
  Suppose $f\in\mathcal L^1\rr{\mathbb T}$ has an absolutely convergent Fourier
  series. Then the Fourier series of $f$ converges uniformly on $\br{0,1}$ to a
  continuous function $g\in C\rr{\mathbb T}$ such that
  \begin{itemize}
    \item for all $k\in\Z$, $\hat g\rr{k}=\hat f\rr{k}$;
    \item for all $x\in\br{0,1}$ at which $f$ is continuous, $g(x)=f(x)$.
  \end{itemize}
\end{theorem}

\begin{proposition}
  If $f\in C^n\rr{\mathbb T}$ for some $n\geq 0$, then, for all $k\in\Z$,
  \begin{align*}
    \hat{f^{\rr{n}}}\rr{k}=\rr{2\pi ik}^n \hat f\rr{k}
  \end{align*}
\end{proposition}

\begin{corollary}
  If $f\in C^n\rr{\mathbb T}$ for some $n\geq 0$, then, for all $k\in\Z\setminus\cc{0}$,
  \begin{align*}
    \vv{\hat f\rr{k}}\leq \vabs{f^{\rr{n}}}_{L^1\rr{\mathbb T}} \frac{1}{\rr{2\pi\vv{k}}^n}.
  \end{align*}
\end{corollary}

\begin{proposition}
  Suppose $\rr{a_k}_{k\in\Z}$ is a sequence of complex numbers satisfying
  \begin{align*}
    \vv{a_k}\leq C\rr{1+\vv{k}}^{-n}
  \end{align*}
  for some $n\geq 3$. Then the series
  \begin{align*}
    g\rr{x}=\sum_{k\in\Z} a_k e^{2\pi ikx}
  \end{align*}
  defines a 1-periodic function which s of the class $C^{\rr{n-2}}\rr{\mathbb T}$. If,
  in addition, $a_k=\hat f\rr{k}$ for some $f\in C\rr{\mathbb T}$, then
  $f\in C^{\rr{n-2}}\rr{\mathbb T}$.
\end{proposition}

\begin{proposition}
  For any $f\in\mathcal L^1\rr{\mathbb T}$ we have
  \begin{align*}
    \min_{p\in\mathcal P_N\rr{\mathbb T}} \int_{\mathbb T}\vv{f\rr{x}-p\rr{x}}^2 dx
    = \int_{\mathbb T}\vv{f\rr{x}-S_Nf\rr{x}}^2dx
    = \int_{\mathbb T}\vv{f\rr{x}}^2dx - \sum_{k=-N}^{ N } \vv{\hat f\rr{k}}^2.
  \end{align*}
\end{proposition}

\begin{corollary}[Bessel]
  For any $f\in\mathcal L^1\rr{\mathbb T}$, we have
  \begin{align*}
    \sum_{k\in\Z}\vv{\hat f\rr{k}}^2 \leq \int_{\mathbb T}\vv{f\rr{x}}^2dx.
  \end{align*}
\end{corollary}

\begin{proposition}
  If $f\in C^1\rr{\mathbb T}$ then the Fourier series of $f$ converges absolutely and hence
  uniformly to $f$.
\end{proposition}

\begin{proposition}
  Let $f,g\in\mathcal L^1\rr{\mathbb T}$. Then, for all $k\in\Z$,
  \begin{align*}
    \widehat{f*g}(k) = \hat f(k)\hat g(k).
  \end{align*}
  In particular,
  \begin{align*}
    \int_{\mathbb T}f*g = \rr{\int_{\mathbb T} f}\rr{\int_{\mathbb T}g}.
  \end{align*}
\end{proposition}

\begin{theorem}
  Let $f\in\mathcal L^2(\mathbb T)$. Then
  \begin{align*}
    \lim_{N\to\infty}\int_{\mathbb T}\vv{S_N f(x)-f(x)}^2 dx = 0.
  \end{align*}
  Further we have the \emph{Riesz-Fischer identity},
  \begin{align*}
    \sum_{k\in\Z} \vv{\hat f(k)}^2 = \int_{\mathbb T} \vv{f(x)}^2dx.
  \end{align*}
\end{theorem}

\begin{corollary}[Parseval]
  Let $f,g\in\mathcal L^2\rr{\mathbb T}$. Then
  \begin{align*}
    \int_{\mathbb T}f\bar g = \sum_{k\in\Z} \hat f(k)\overline{\hat g(k)}.
  \end{align*}
\end{corollary}

\subsection{Pointwise convergence}

\begin{theorem}[Katznelson]
  For any set of measure zero $E\subseteq\br{0,1}$, there exists an $f\in C\rr{\mathbb T}$
  whose Fourier series diverges at every point in $E$.
\end{theorem}

\begin{theorem}[Carleson]
  For $f\in C\rr{\mathbb T}$, there exists a set $E\subseteq\br{0,1}$ of measure zero
  such that, for all $x\in \br{0,1}\setminus E$,
  \begin{align*}
    \sum_{k=0}^\infty \hat f\rr{k}e^{2\pi i k x} = f\rr{x}.
  \end{align*}
\end{theorem}

\begin{theorem}
  There exists a function $f\in C(\mathbb T)$ such that
  $\hat f$ diverges at $0$.
  \begin{proof}
    Define
    \begin{align*}
      F_N(x) = \sum_{k=-N,k\neq 0}^N \frac{\exp\rr{2\pi ikx}}{k},\hs
      f_n(x) = F_N(x)\exp\rr{2\pi i2Nx}.
    \end{align*}
    Now let $\rr{N_k}_{k\in\N}$ be the rapidly increasing sequence
    of integers given by $N_k = 3^{2^k}$ and choose
    \begin{align*}
      f(x) = \sum_{k=1}^\infty \frac{1}{k^2} f_{N_k}(x).
    \end{align*}
  \end{proof}
\end{theorem}

\begin{definition}
  Let $f:\R\to\C$ and $x\in\R$. Then $f$ is \emph{Lipschitz at $x$} if there is a
  constant $C>0$ such that, for all $t\in\R$,
  \begin{align*}
    \vv{f\rr{x-t}-f\rr{t}}\leq C\vv{t}.
  \end{align*}
\end{definition}

\begin{theorem}
  If $f\in\mathcal L^1\rr{\mathbb T}$ and $f$ is Lipschitz at $x\in\br{0,1}$,
  then the Fourier series of $f$ converges at $x$.
\end{theorem}

\begin{theorem}
  Let $f,g\in\mathcal L^1\rr{\mathbb T}$ and $I\subseteq\R$ an open interval.
  If $f\rr{x}=g\rr{x}$ for all $x\in I$, then
  \begin{align*}
    S_N f\rr{x}-S_N g\rr{x}\to 0 \text{ as } N\to\infty
  \end{align*}
  for all $x\in I$.
\end{theorem}

\subsection{Uniform convergence}

\begin{definition}
  Let $N\geq 1$ and $x\in\R$. Then the \emph{$N$th Dirichlet kernel at $x$} is
  \begin{align*}
    D_N\rr{x}=\sum_{k=-N}^N e^{2\pi ikx}.
  \end{align*}
\end{definition}

\begin{lemma}
  For all $N\geq 1$ and $x\in\R$,
  \begin{align*}
    D_N\rr{x} =
    \begin{cases}
      \frac{\sin\rr{\rr{N+1}\pi x}}{\sin\rr{\pi x}} & \text{if }x\not\in\Z,\\
      2N+1 &\text{if }x\in\Z.
    \end{cases}
  \end{align*}
\end{lemma}

\begin{definition}
  For $f\in\mathcal C\rr{\mathbb T}$, the \emph{$n$th Cesaro mean of $f$ at
  $x$} is given by $\sigma_N f\rr{x} = \frac{1}{N+1} \sum_{n=0}^N S_n f\rr{x}$.
\end{definition}

\begin{theorem}
  For $f\in C\rr{\mathbb T}$, the sequence of Cesaro means $\rr{\sigma_n f}_{n\geq 1}$
  converges uniformly to $f$ on $\br{0,1}$.
\end{theorem}

\begin{definition}
  We say that a sequence $\rr{K_N}_{N\in\N}$ with each $K_N\in\mathcal L^1\rr{\mathbb T}$
  is an \emph{approximate identity} if
  \begin{enumerate}
    \item for all $N\in\N$, $\int_{\mathbb T} K_N = 1$;
    \item there exists $B\geq 1$ such that, for all $N\in\N$, $\int_{\mathbb T}\vv{K_N}\leq B$;
    \item for all $\delta>0$,
      \begin{align*}
        \lim_{N\to\infty}\int_{\delta\leq\vv{x}\leq 1/2}\vv{K_N\rr{x}} dx = 0.
      \end{align*}
  \end{enumerate}
  The sequence $\rr{K_N}_{N\in\N}$ is a \emph{strong approximate identity} if, additionally,
  for every $\delta>0$,
  \begin{align*}
    \lim_{N\to\infty} \sup_{\delta\leq\vv{x}\leq 1/2}\vv{K_N\rr{x}}=0.
  \end{align*}
\end{definition}

\begin{theorem}
  Let $\rr{K_N}_{N\in\N}$ be an approximate identity.
  \begin{enumerate}
    \item If $f\in\mathcal L^\infty\rr{\mathbb T}$ is continuous at $x\in\R$ then
      $K_N * f(x) \to f(x)$ as $N\to\infty$;
    \item If $f\in C(\mathbb T)$ then $K_N*f$ converges uniformly to $f$;
    \item For $1\leq p<\infty$, if $f\in\mathcal L^p\rr{\mathbb T}$ then
      \begin{align*}
        \lim_{N\to\infty}\vabs{K_N*f-f}_{L^p\rr{\mathbb T}}=0.
      \end{align*}
  \end{enumerate}
\end{theorem}

\begin{definition}
  Let $N\geq 1$ and $x\in\R$. Then the \emph{$N$th Fej\'er kernel at $x$} is
  \begin{align*}
    F_N\rr{x}=\frac{1}{N+1}\sum_{n=0}^{N} D_n\rr{x}.
  \end{align*}
\end{definition}

\begin{proposition}
  The sequence of Fej\'er kernels $\rr{F_N}_{N\in\N}$ is a strong approximate identity.
\end{proposition}

\begin{corollary}
  If $f\in\mathcal L^\infty\rr{\mathbb T}$ is continuous at $x\in\R$ then
  $\sigma_N f(x)\to f(x)$. Moreover, if $f\in C(\mathbb T)$ then $\sigma_Nf\to f$
  uniformly.
\end{corollary}


\begin{definition}
  Let $f\in\mathcal L^1\rr{\mathbb T}$ and $0<r<1$. Then the \emph{Abel mean at $x$}
  is
  \begin{align*}
    A_r f(x) = \sum_{k\in\Z} \hat f(k)\exp\rr{2\pi ikx}r^{\vv{k}}.
  \end{align*}
\end{definition}

\begin{definition}
  Let $0<r<1$ and $x\in\R$. Then the \emph{Poisson kernel at $x$} is
  \begin{align*}
    P_r\rr{x}= \sum_{k\in\Z} r^{\vv{k}}\exp\rr{2\pi ikx}.
  \end{align*}
\end{definition}

\begin{proposition}
  For all $0<r<1$, $A_r f = f * P_r$.
\end{proposition}

\begin{theorem}
  If $f\in\mathcal L^1\rr{\mathbb T}$ is continuous at $x\in\br{0,1}$ then $A_r f(x)\to f(x)$
  as $r\to 1^-$. Moreover, if $f\in C\rr{\mathbb T}$ then $A_r f\to f$ uniformly as
  $r\to 1^-$.
\end{theorem}

\subsection{Applications}

\begin{definition}
  The \emph{length} of a simple, closed $C^1$ curve
  given by a parametrisation $\gamma : \bb{0,1}\to\R^2$ is
  defined by
  \begin{align*}
    \text{Length}(\Gamma) = \int_0^1 \vv{\gamma'(t)} dt.
  \end{align*}
\end{definition}

\begin{theorem}
  Let $R_\Gamma\subseteq\R^2$ be the region inclosed by a simple,
  closed, piecewise $C^1$ curve $\Gamma\subseteq\R^2$. Then
  \begin{align*}
    \text{Area}(R_\Gamma) \leq \text{Length}(\Gamma)^2 / 4\pi.
  \end{align*}
  Furthermore, equality holds if and only if $\Gamma$ is a circle.
\end{theorem}

\begin{definition}
  Consider $\rr{x_n}_{n\in\N}$ with each $x_n\in\br{0,1}$. The sequence is \emph{equidistributed
  in $\br{0,1}$} if, for all intervals $I\subseteq \br{0,1}$,
  \begin{align*}
    \lim_{N\to\infty} \frac{1}{N}\sum_{n=1}^N \chi_I\rr{x_n} = \int_{\mathbb T}\chi_I.
  \end{align*}
\end{definition}

\begin{theorem}
  Let $\rr{x_n}_{n\in\N}\subseteq\br{0,1}$. Then tfae
  \begin{enumerate}
    \item $\rr{x_n}_{n\in\N}$ is equidistributed;
    \item for all $f\in\mathcal R\rr{\mathbb T}$,
      \begin{align*}
        \lim_{N\to\infty}\frac{1}{N}\sum_{n=1}^N f(x_n) = \int_{\mathbb T} f;
      \end{align*}
    \item for all $k\in\Z\setminus\cc{0}$,
      \begin{align*}
        \lim_{N\to\infty} \frac{1}{N}\sum_{n=1}^{N} \exp\rr{2\pi ikx_n}=0.
      \end{align*}
  \end{enumerate}
\end{theorem}

\section{Fourier transform}

\begin{definition}
  The \emph{Fourier transform} of a function $f:\R\to\C$ in
  $\mathcal L^1(\R)$ is given by
  \begin{align*}
    \mathcal Ff(\xi) = \int_{\R} f(x) \exp\rr{-2\pi i\xi x} dx.
  \end{align*}
\end{definition}

\subsection{Properties}

\begin{theorem}
  For $f,g\in\mathcal L^1\rr{\R}$,
  \begin{itemize}
    \item For all $a,b\in\C$,
      $\mathcal F(af + bf)(\xi) = a\mathcal F(f)(\xi) + b\mathcal F(g)(\xi)$.
    \item For all $h\in\R$, $\mathcal F(x\mapsto f(x-h))(\xi) = \exp\rr{-2\pi i\xi
      h}\mathcal Ff(\xi)$.
    \item For all $t>\R$,
      $\mathcal F(x\mapsto\inv tf(\inv tx)) = \mathcal Ff(t\xi)$.
    \item $\mathcal F(x\mapsto f(-x))(\xi) = \mathcal Ff(-\xi)$.
    \item For all $h\in\R$, $\mathcal F(x\mapsto\exp\rr{2\pi ihx}f(x))(\xi) = (\mathcal Ff)(\xi - h)$.
  \end{itemize}
\end{theorem}

\begin{example}
  $\mathcal F(\chi_{\bb{-1,1}})\not\in\mathcal L^1\rr{\R}$.
\end{example}

\begin{theorem}
  If $f\in\mathcal L^1\rr{\R}$, then $\mathcal Ff$ is continuous
  and $Ff(\xi)\to 0$.
\end{theorem}

\begin{proposition}
  Let $f\in\mathcal L^1\rr{\R}$.
  \begin{itemize}
    \item If $f'\in\mathcal L^1\rr{\R}$, then $\mathcal F(f')(\xi) = (2\pi i\xi)\mathcal Ff(\xi)$.
    \item If $xf\in\mathcal L^1\rr{\R}$, then $\mathcal Ff$ is
      differentiable and
      \begin{align*}
        (\mathcal Ff)'(\xi) = \mathcal F(x\mapsto-2\pi ixf(x))(\xi).
      \end{align*}
  \end{itemize}
\end{proposition}

\note{capture 9.6}

\begin{lemma}
  Let $f,g\in\mathcal L^1\rr{\R}$. Then
  \begin{itemize}
    \item $(\mathcal Ff*\mathcal Fg)=\mathcal F(f*g)$;
    \item $\int_\R f(\mathcal Fg) = \int_\R (\mathcal F f) g$
  \end{itemize}
\end{lemma}

\subsection{Kernels}

\begin{proposition}
  Let $G(x)=\exp\rr{-\pi x^2}$. Then $\mathcal FG(\xi)=G(\xi)$.
\end{proposition}

\begin{definition}
  A family $\rr{K_t}_{t>0}\subseteq\mathcal L^1(\R)$
  is an \emph{approximate identity} if
  \begin{enumerate}
    \item for all $t>0$, $\int_{\R} K_t = 1$;
    \item there exists $B\geq 1$ such that, for all $t>0$, $\int_{\R}\vv{K_t}\leq B$;
    \item for all $\delta>0$,
      \begin{align*}
        \lim_{t\to 0^+}\int_{\delta\leq\vv{x}}\vv{K_t\rr{x}} dx = 0.
      \end{align*}
  \end{enumerate}
\end{definition}

\begin{example}
  Consider a function $K\in\mathcal L^1(\R)$ with $\int_\R K = 1$.
  For example,
  \begin{itemize}
    \item the \emph{Gaussian} $G(x)=\exp\rr{-\pi x^2}$;
    \item the \emph{Fej\'er kernel} $F(x)=\sin^2(\pi x)/\pi^2x^2$;
    \item the \emph{Poisson kernel} $P(x)=1/\pi(1+x^2)$.
  \end{itemize}
  Then the family $\rr{K_t}_{t>0}$ given by
  $K_t(x) = \inv t K(\inv t x)$ is an approximate identity.
\end{example}

\begin{theorem}
  Let $\rr{K_t}_{t>0}$ be an approximate identity.  \begin{itemize}
    \item  If $f\in\mathcal L^{\infty}\rr{\R}$ is continuous at $x$,
      then $K_t * f(x) \to f(x)$.
    \item  If $f\in\mathcal L^{\infty}\rr{\R}$ is continuous,
      then $\vabs{K_t * f - f}_\infty \to 0$.
    \item  If $f\in\mathcal L^p\rr{\R}$ for $1\leq p<\infty$,
      then $\vabs{K_t * f - f}_p \to 0$.
  \end{itemize}
\end{theorem}

\subsection{Inversion}

\begin{lemma}
  If $f\in\mathcal L^1\rr{\R}$, then, for all $x\in\R$ and $t>0$,
  \begin{align*}
    G_t * f(x) = \int_\R \mathcal F(G_t * f)(\xi)\exp\rr{2\pi ix\xi} d\xi
    = \int_\R \mathcal F(G_t)(\xi)\mathcal Ff(\xi)\exp\rr{2\pi ix\xi} d\xi.
  \end{align*}
\end{lemma}

\begin{theorem}
  Let $f,g\in\mathcal L^1\rr{\R}$ such that $\mathcal Ff=\mathcal Fg$.
  If $f-g$ is continuous at $x\in\R$, then $f(x)=g(x)$.
\end{theorem}

\begin{theorem}
  Let $f\in\mathcal L^1\rr{\R}$ such that $\mathcal Ff\in\mathcal L^1\rr{\R}$. If $f$ is continuous at $x\in\R$, then
  \begin{align*}
    f(x) = \int_\R \mathcal Ff(\xi)\exp\rr{2\pi ix\xi} d\xi.
  \end{align*}
\end{theorem}

\begin{corollary}
  Let $\mathcal B(\R) = \cc{f:\R\to\C : f,\mathcal Ff\in\mathcal L^1\rr{\R}\cap C(\R)}$. Then $\mathcal F$ is a bijection of $\mathcal B(\R)$ and
  its inverse is
  \begin{align*}
    (\inv{\mathcal F}f)(x) = \int_\R g(\xi)\exp\rr{2\pi ix\xi} d\xi.
  \end{align*}
\end{corollary}

\subsection{Plancharel}

\begin{theorem}[Plancherel]
  Suppose that $f\in\mathcal L^1\rr{\R}\cap\mathcal L^{2}\rr{\R}$. Then
  $\mathcal F f\in\mathcal L^{2}\rr{\R}$ and
  \begin{align*}
    \int_\R \vv{\mathcal Ff}^2 = \int_\R \vv{f}^2.
  \end{align*}
\end{theorem}

\begin{theorem}
  Let $f\in\mathcal L^1\rr{\R}\cap\mathcal L^{2}\rr{\R}$ and, for all
  $R>0$, define
  \begin{align*}
    S_R f(x) = \int_{-R}^R \mathcal F f(\xi)\exp\rr{2\pi i x\xi}d\xi.
  \end{align*}
  Then, for all $R>0$,
  \begin{itemize}
    \item $S_R f\in\mathcal L^2\rr{\R}$ and
      \begin{align*}
        \int_\R \vv{S_R f}^2 = \int_{\vv{\xi}\leq R} \vv{\mathcal F f(\xi)}^2
        d\xi \leq \int_\R \vv{f}^2;
      \end{align*}
    \item $S_R f-f\in\mathcal L^{2}\rr{\R}$ and, as $R\to\infty$,
      \begin{align*}
        \int_\R \vv{S_r f - f}^2 = \int_{\vv{\xi}\geq R} \vv{\mathcal F f(\xi)}^2 \to 0.
      \end{align*}
  \end{itemize}
\end{theorem}

\end{document}
