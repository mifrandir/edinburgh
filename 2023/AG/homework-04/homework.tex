\documentclass{article}
\usepackage{homework-preamble}

\begin{document}

\title{Algebraic Geometry: Homework 4}
\author{Franz Miltz (UUN: S1971811)}
\date{22 March 2023}
\maketitle

\section*{Exercise 1}

Consider an irreducible curve $C\subseteq\proj{2}$ given by a homogeneous polynomial
$f\in k[x,y,z]$. Moreover, consider the cone $\widehat C\subseteq\proj{3}$,
given by $f\in k[x,y,z,w]$.

\begin{claim*}[1]
  The cone $\widehat C$ is birational to $C\times\proj{1}$.
  \begin{proof}
    Consider the rational maps
    \begin{align*}
      f: C\times\proj{1} &\dashto \widehat C, &
      g : \widehat C &\dashto C\times\proj{1} \\
      [x : y : z], [p:q] &\mapsto \bb{x:y:z:p/q}, &
      [x:y:z:w] &\mapsto [x:y:z], [w : 1].
    \end{align*}
    We note $\dom(f) = \cc{[x:y:z],[p:q] : q\neq 0}$ and $\dom(g) = \widehat C$.
    Moreover $f(\dom(f)) = \widehat C$ so $f$ is dominant.

    Now let $[x:y:z]\in C$ and $[p:q]\in\proj{1}$ such that $q\neq 0$. Then
    \begin{align*}
      (g\circ f)([x:y:z],[p:q]) &= g([x:y:z:p/q]) \\
                                &= [x:y:z],[p/q:1]\\
                                &= [x:y:z],[p:q]
    \end{align*}
    where the last step follows because $[p:q] = \lambda[p:q]$ for all $\lambda\neq 0$.
    Similarly, let $[x:y:z:w]\in\widehat C$
    \begin{align*}
      (f\circ g)([x:y:z:w]) = f([x:y:z],[w:1]) = [x:y:z:w].
    \end{align*}
    We have shown $f\circ g = \identity$ and $g\circ f=\identity$ as rational maps.
  \end{proof}
\end{claim*}

\begin{claim*}[2]
  Any irreducible singular quadric surface in $\proj{3}_{\C}$ is rational.
  \begin{proof}
    Let $S$ be such a surface. By \emph{Theorem 4.8}, $S$ is projectively equivalent
    to the cone $\widehat C$ of a smooth conic $C\subseteq\proj{2}$. Now $C$ is isomorphic
    to $\proj{1}_{\C}$ so we may use the previous claim to find that $S$ is birational
    to $\proj{1}_{\C}\times\proj{1}_{\C}$.

    Now consider the rational maps
    \begin{align*}
      f : \proj{1}_{\C}\times\proj{1}_{\C} &\dashto \proj{2}_{\C} &
      g : \proj{2}_{\C} &\dashto \proj{1}_{\C}\times\proj{1}_{\C} \\
      [x:y],[z:w] &\mapsto [x:y:z/w] &
      [x:y:z] &\mapsto [x:y:z:1]
    \end{align*}
    We observe $\dom(f)=\cc{([x:y],[z:w])\in\proj{1}_{\C}\times\proj{1}_{\C} : w \neq 0}$
    and $\dom(g) = \proj{2}_{\C}$. Moreover, $f(\dom(f))=\proj{2}_{\C}$ so $f$ is dominant.

    Now let $[x:y],[z:w]\in\proj{1}_{\C}$ with $w\neq 0$. Then
    \begin{align*}
      (g\circ f)([x:y],[z:w]) = [x:y],[z/w : 1] = [x:y],[z:w].
    \end{align*}
    and
    \begin{align*}
      (f\circ g)([x:y:z]) = f([x:y],[z:1]) = [x:y:z].
    \end{align*}
    We have shown $f\circ g = \identity$ and $g\circ f=\identity$ as rational maps,
    proving that $\proj{1}_{\C}\times\proj{1}_{\C}$ is birational to $\proj{2}_{\C}$.

    Birationality is transitive so $S$ is birational to $\proj{2}_{\C}$.
  \end{proof}
\end{claim*}

\section*{Exercise 2}

Consider the polynomial $f(x,y,z,w)=wx^2 + xyz + y^3$ and the cubic surface
$S:f(x,y,z,w) = 0$ in $\proj{3}_{\C}$.

\begin{claim*}[1a]
  $S$ is irreducible.
  \begin{proof}
    Let $g,h\in\C[x,y,z,w]$ be such that $f=gh$. Considering $f$ as a polynomial in $w$,
    we obtain that there are unique polynomials $p,q\in\C[x,y,z]$ such that
    $f=p w + q$. Clearly, $p=x^2$ and $q = xyz + y^3$. Without loss of generality, we
    have $g = (p/h) w + q/h$. However, there is no nonconstant polynomial $h$ such that
    $h | x^2$ and $h | (xyz + y^3)$. Thus $h$ is constant and $f$ is irreducible.
  \end{proof}
\end{claim*}

\begin{claim*}[1b]
  A point $[x:y:z:w]\in\proj{3}_{\C}$ is a singular point on $S$ if, and only if, $x=y=0$.
  \begin{proof}
    We have the partial derivate $\partial f/\partial w = x^2$. Thus, for all singular
    points $[x:y:z:w]\in S$, $x=0$. Now we have the partial derivative
    $\partial f/\partial y=xz+3y^2$. Thus we must have $y=0$.

    We note that any point $[x:y:z:w]\in\proj{3}_{\C}$ with $x=y=0$ satisfies $f(x,y,z,w)=0$ and,
    moreover, the gradient equations
    \begin{align*}
      2 xw + yz = 0,\hs
      xz + 3y^2 = 0,\hs
      xy = 0,\hs
      x^2 = 0.
    \end{align*}
    Thus a point $P\in\proj{3}_{\C}$ is a singular point of $S$ if, and only if,
    it is of the form $[0 : 0 : z : w]$.
  \end{proof}
\end{claim*}

\begin{claim*}[2]
  A line in $\proj{3}_{\C}$ is on $S$ if, and only if, it is given by one of the following
  \begin{align*}
    \lambda,\mu \mapsto [ 0 : 0 : \lambda : \mu ],\hs
    \lambda,\mu \mapsto [ \lambda : 0 : \mu : 0 ],\hs
    \lambda,\mu \mapsto [ \lambda : -\alpha\lambda : \alpha^3 \mu - \alpha^2\lambda :\alpha^4\mu]
  \end{align*}
  where $\alpha\in\C$.
  \begin{proof}
    We follow the algorithm from the notes:
    \begin{enumerate}
      \item Note that the line $L:x=y=0$ is contained in $S$. Moreover, it is given by the
        intersection $\Pi\cap S$ with the plane $\Pi:x = 0$.
      \item Suppose $L'\subseteq S$ is any other line. Then it intersects $\Pi$ in a point
        on $L$.
      \item Since they intersect, $L$ and $L'$ must lie on a plane. The planes containing
        $L$ are $\Pi_{[\alpha:\beta]}:\alpha x + \beta y = 0$ for some $[\alpha:\beta]\in\proj{1}_{\C}$.
      \item $\beta=0$ corresponds to $\Pi$. Suppose without loss of generality $\beta=1$. Then
        $\Pi_{[\alpha:1]}\cap S$ is given by
        \begin{align*}
          \alpha x + y = 0, \hs wx^2 = \alpha x^2 z + \alpha^3 x^3.
        \end{align*}
        If $\alpha\neq 0$, this intersection is $L$ union with the line
        $w = \alpha z + \alpha^3 x$ in $\proj{2}_{\C}$. The corresponding line in $\proj{3}_{\C}$ is explicitly
        given by $\lambda,\mu \mapsto [\lambda : -\alpha\lambda : \alpha^3\mu - \alpha^2\lambda : \alpha^4\mu]$.

        If $\alpha=0$, then this intersection is given by the cubic $wx^2 = y = 0$ in $\proj{2}_{\C}$.
        This is the union of the line $L$ and the line $w=y=0$.
    \end{enumerate}
  \end{proof}
\end{claim*}

\end{document}
