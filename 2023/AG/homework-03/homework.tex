\documentclass{article}
\usepackage{homework-preamble}

\begin{document}
\title{Algebraic Geometry: Homework 3}
\author{Franz Miltz (UUN: S1971811)}
\date{8 March 2023}
\maketitle

\section*{Exercise 1}

Let $X,Y\subseteq\mathbf A^n$ be algebraic subsets and
Let $f_1,\ldots,f_s,g_1,\ldots,g_t\in k\bb{x_1,\ldots,x_n}$ be polynomials such that
\begin{align*}
  X=V(f_1,\ldots,f_s), \hs
  Y=V(g_1,\ldots,g_t).
\end{align*}

\begin{claim*}[1]
  $X\cup Y\subseteq \mathbf A^n$ is an algebraic subset.
  \begin{proof}
    Consider the ideal $I$ generated by the set
    \begin{align*}
      \cc{f_i g_j \in k\bb{x_1,\ldots,x_m} : 1\leq i\leq s, 1\leq j\leq t}.
    \end{align*}
    Let $P\in X\cup Y$. Then $P\in X$ or $P\in Y$. Thus $f_i(P)g_j(P)=0$ for all
    $i$ and $j$. I.e. $P\in V(I)$.

    Conversely, assume $P\in V(I)$. As $k\bb{x_1,\ldots,x_n}$ is an integral domain,
    we have $f_i(P)=0$ or $g_j(P)=0$ for all $i$ and $j$. Assume, for some $i$,
    $f_i(P)\neq 0$. Then $g_j(P)=0$ for all $j$ so $P\in Y$. Similarly, if $g_j(P)\neq 0$
    for some $j$ then $f_i(P)=0$ for all $i$. In either case, $P\in X\cup Y$. We
    have shown $X\cup Y=V(I)$, as required.
  \end{proof}
\end{claim*}

\begin{claim*}[2]
  $X\cap Y\subseteq\mathbf A^n$ is an algebraic subset and
  $I(X\cap Y)=\sqrt{I(X)+I(Y)}$.
  \begin{proof}
    Let $P\in V(I(X)+I(Y))$. Then $f(P)+g(P)=0$ for all $f\in I(X)$ and $g\in I(Y)$.
    In particular, $f(P)=0$ for all $f\in I(X)$ so $P\in X$. Similarly, $g(P)=0$ for
    all $g\in I(Y)$ so $P\in Y$. Thus $P\in X\cap Y$.

    Conversely, let $P\in X\cap Y$. Then $f(P)=0$ for all $f\in I(X)$ and
    $g(P)=0$ for all $g\in I(Y)$. Thus $f(P)+g(P)=0$ for all $f\in I(X)$ and
    $g\in I(Y)$, i.e. $P\in V(I(X)+I(Y))$. We have shown $X\cap Y=V(I(X)+I(Y))$
    as required.

    Moreover, the \emph{Nullstellensatz} immediately yields $I(X\cap Y)=\sqrt{I(X) + I(Y)}$.
  \end{proof}
\end{claim*}

\end{document}
