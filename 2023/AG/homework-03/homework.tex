\documentclass{article}
\usepackage{homework-preamble}

\begin{document}

\title{Algebraic Geometry: Homework 3}
\author{Franz Miltz (UUN: S1971811)}
\date{8 March 2023}
\maketitle

Consider an algebraically closed field $k$.

\section*{Exercise 1}

Let $X,Y\subseteq\mathbf A^n$ be algebraic subsets.

\begin{claim*}[1]
  $X\cup Y\subseteq \mathbf A^n$ is an algebraic subset.
  \begin{proof}
    Let $f_1,\ldots,f_s,g_1,\ldots,g_t\in k\bb{x_1,\ldots,x_n}$ be polynomials such that
    \begin{align*}
      X=V(f_1,\ldots,f_s), \hs
      Y=V(g_1,\ldots,g_t)
    \end{align*}
    and consider the ideal $I$ generated by the set
    \begin{align*}
      \cc{f_i g_j \in k\bb{x_1,\ldots,x_m} : 1\leq i\leq s, 1\leq j\leq t}.
    \end{align*}
    Let $P\in X\cup Y$. Then $P\in X$ or $P\in Y$. Thus $f_i(P)g_j(P)=0$ for all
    $i$ and $j$. I.e. $P\in V(I)$.

    Conversely, assume $P\in V(I)$. As $k\bb{x_1,\ldots,x_n}$ is an integral domain,
    we have $f_i(P)=0$ or $g_j(P)=0$ for all $i$ and $j$. Assume, for some $i$,
    $f_i(P)\neq 0$. Then $g_j(P)=0$ for all $j$ so $P\in Y$. Similarly, if $g_j(P)\neq 0$
    for some $j$ then $f_i(P)=0$ for all $i$. In either case, $P\in X\cup Y$. We
    have shown $X\cup Y=V(I)$, as required.
  \end{proof}
\end{claim*}

\begin{claim*}[2]
  $X\cap Y\subseteq\mathbf A^n$ is an algebraic subset and
  $I(X\cap Y)=\sqrt{I(X)+I(Y)}$.
  \begin{proof}
    Let $P\in V(I(X)+I(Y))$. Then $f(P)+g(P)=0$ for all $f\in I(X)$ and $g\in I(Y)$.
    In particular, $f(P)=0$ for all $f\in I(X)$ so $P\in X$. Similarly, $g(P)=0$ for
    all $g\in I(Y)$ so $P\in Y$. Thus $P\in X\cap Y$.

    Conversely, let $P\in X\cap Y$. Then $f(P)=0$ for all $f\in I(X)$ and
    $g(P)=0$ for all $g\in I(Y)$. Thus $f(P)+g(P)=0$ for all $f\in I(X)$ and
    $g\in I(Y)$, i.e. $P\in V(I(X)+I(Y))$. We have shown $X\cap Y=V(I(X)+I(Y))$
    as required.

    Moreover, the \emph{Nullstellensatz} immediately yields $I(X\cap Y)=\sqrt{I(X) + I(Y)}$.
  \end{proof}
\end{claim*}

\begin{claim*}[3]
  There exist algebraic subsets $X,Y\subseteq \affine{n}$ such that $I(X\cap Y)\neq I(X)+I(Y)$.
  \begin{proof}
    Consider the algebraic subsets $X:x^3=0$ and $Y:y^2-x^3=0$. Then $I(X)=\aa{x^3}$
    and $I(Y)=\aa{y^2-x^3}$. Hence $I(X)+I(Y)=\aa{x^3}+\aa{y^2-x^3}=\aa{x^3,y^2}$.
    Note that $\aa{x^3,y^2}$ is not radical.
    We now observe that $X\cap Y=\cc{(0,0)}$. Thus $x\in I(X\cap Y)$. However,
    $x\not\in\aa{x^3,y^2}$, as required.
  \end{proof}
\end{claim*}

\section*{Exercise 2}

Consider the nodal cubic $C:y^2=x^2(x-1)$ in $\affine{2}$.

\begin{claim*}[1]
  $C$ is not normal.
  \begin{proof}
    Consider the polynomials $f,g,h\in k\bb{x,y}$ given by
    \begin{align*}
      f(x,y)=y^2,\hs g(x,y)=x^2,\hs h(x,y)= 1-x.
    \end{align*}
    We note that $f(1,0)\neq g(1,0)$, $f(0,0)\neq h(0,0)$, and $g(1,0)\neq h(1,0)$.
    Thus $f,g,h$ correspond to distinct elements in the coordinate ring $k\bb{C}$.
    In particular, $f\neq 0$, $g\neq 0$, and $g\neq 1$ as elements of $k\bb{C}$
    so $f/g\in k\rr{C}$ but $f/g\not\in k\bb{C}$.
    Finally, we observe that $f(P)+g(P)h(P) = 0$ for all $P\in C$ and conclude that
    $f/g\in k(C)$ is integral over $k\bb{C}$.
  \end{proof}
\end{claim*}

Consider $\tilde C\subseteq \affine{3}$ given by $x=z^2+1$, $y=z(z^2+1)$
and the map $\alpha:\tilde C\to\affine{2}$ given by $(x,y,z)\mapsto (x,y)$.

\begin{claim*}[2]
  The image of $\alpha$ is $C$.
  \begin{proof}
    Consider $(x,y)\in\im(\alpha)$. Then there is a $z\in k$ such that $x=z^2+1$ and $y=z(z^2+1)$.
    In particular,
    \begin{align*}
      y^2 = z^2(z^2+1)^2=(x-1)x^2.
    \end{align*}
    Thus $(x,y)\in C$, i.e. $\im(\alpha)\subseteq C$.

    Conversely, assume $(x,y)\in C$ and let $z=y/x$. Note
    \begin{align*}
      z^2 + 1 = \frac{y^2}{x^2}+1 = \frac{x^2(x-1)}{x^2} + 1 = x
    \end{align*}
    and
    \begin{align*}
      z(z^2+1) = zx = y.
    \end{align*}
    Thus $(x,y,y/x)\in\tilde C$ and $\alpha(x,y,y/x)=(x,y)$. I.e. $C\subseteq\im(\alpha)$.
  \end{proof}
\end{claim*}

\begin{claim*}[3a]
  There is an isomorphism of affine varieties $\tilde C\cong\affine{1}$.
  \begin{proof}
    Consider the polynomial maps $\phi:\tilde C\to\affine{1}$ and
    $\psi:\affine{1}\to\tilde C$ given by
    \begin{align*}
      \phi(x,y,z)=z, \hs \psi(t)=(t^2+1,t(t^2+1),t).
    \end{align*}
    Consider $(x,y,z)\in\tilde C$. Then
    \begin{align*}
      (\psi\circ\phi)(x,y,z) = \psi(z) = (z^2 + 1,z(z^2+1),z) = (x,y,z).
    \end{align*}
    Similarly, let $t\in\affine{1}$ and note
    \begin{align*}
      (\phi\circ\psi)(t) = \phi(t^2+1,t(t^2+1),t) = t.
    \end{align*}
    Thus we have an isomorphism $\phi : \tilde C \cong \affine{1} : \psi$.
  \end{proof}
\end{claim*}

\begin{claim*}[3b]
  There is an isomorphism of fields $k(\tilde C)\cong k(C)$.
  \begin{proof}
    We have a map $\alpha:\tilde C\to C$ and thus an induced homomorphism
    of $k$-algebras $\alpha^* : k[C]\to k[\tilde C]$ given by
    $f \mapsto f\circ\alpha$. By \emph{Proposition 3.54}, $\alpha^*$ is injective
    because $\alpha$ is surjective. Injective maps of integral domains induce
    homomorphisms of their fields of fractions, so we get a homomorphism
    $\sigma:k(C)\to k(\tilde C)$ given by
    \begin{align}\label{eq:isomorphism_of_fields}
      f/g\mapsto (f\circ\alpha)/(g\circ\alpha).
    \end{align}
    i.e. $f\mapsto f\circ\alpha$.
    Consider the map $\tau : k(\tilde C)\to k(C)$ given by
    \begin{align*}
      (f/g)(x,y,z)\mapsto (f/g)(x,y,y/x)
    \end{align*}
    Consider $f/g\in k(C)$. Now
    \begin{align*}
      \tau(\sigma(f/g))(x,y)
      = \sigma(f/g)(x,y,y/x)
      = \frac{(f\circ\alpha)(x,y,z)}{(g\circ\alpha)(x,y,z)}
      = \frac{f(x,y)}{g(x,y)}
    \end{align*}
    Thus $\tau\circ\sigma = \identity$. Similarly, consider $f/g\in k(\tilde C)$. Then
    \begin{align*}
      \sigma(\tau(f/g))(x,y,z) = \tau(f/g)(x,y) = \frac{f(x,y,y/x)}{g(x,y,y/x)}
    \end{align*}
    We note that for all $(x,y,z)\in\tilde C$, $z=y/x$ so
    \begin{align*}
      \frac{f(x,y,y/x)}{g(x,y,y/x)}  = \frac{f(x,y,z)}{g(x,y,z)}.
    \end{align*}
    In particular, $\sigma(\tau(f/g)) = f/g$ in the function field $k(\tilde C)$, i.e.
    $\sigma\circ\tau = \identity$. Thus (\ref{eq:isomorphism_of_fields}) is a bijective
    field homomorphism. Therefore we have an isomorphism of fields
    $\sigma : k(C)\cong k(\tilde C):\tau$.
  \end{proof}
\end{claim*}

\begin{claim*}[3c]
  $k[\tilde C]$ is the integral closure of $k\bb{C}$.
  \begin{proof}
    We recall $\sigma: k(C)\cong k(\tilde C):\tau$ from previous considerations.
    In particular, $\tau$ is an isomorphism of $k$-algebras and hence an isomorphism of
    subalgebras $k[\tilde C]\cong \tau(k[\tilde C])$.
    We now note that we have an injection $\alpha^*:k[C]\to k[\tilde C]$, i.e.
    $k[C]\subseteq\tau(k[\tilde C])$.

    Observe that the element $y/x\in k(C)$ is integral over $k\bb{C}$ as
    it satisfies the equation
    \begin{align*}
      \rr{\frac{y}{x}}^2 - x + 1 = 0.
    \end{align*}
    Thus $\tau(k[\tilde C])\subseteq \overline{k[C]}$.

    Now let $f/g\in k(C)$ be integral over ${k[C]}$. In particular, $f/g$ satisfies
    a monic polynomial equation with coefficients in $k[C]$.
    As $k[C]\subseteq\tau(k[\tilde C])$, it follows that the coefficients are in
    $\tau(k[\tilde C])$. However $\affine{1}\cong\tilde C$ so $\tau(k[\tilde C])\cong k[t]$
    is a UFD and thus, by \emph{Proposition 3.63}, integrally closed. It follows that
    $f/g\in\tau(k[\tilde C])$, i.e. $\overline{k[C]}\subseteq \tau(k[\tilde C])$.

    Thus we have an isomorphism of $k$-algebras
    $\sigma : \overline{k\bb{C}} \cong k[\tilde C] : \tau$.
  \end{proof}
\end{claim*}

\section*{Exercise 3}

Let $R$ be an integral domain and $f:R\to R$ an automorphism.

\begin{claim*}[1a]
  $f$ extends to an automorphism $\tilde f:K\to K$ of the field of fractions.
  \begin{proof}
    Consider the maps $\tilde f,\tilde{\inv f}: K\to K$ given by
    \begin{align*}
      \tilde f(x/y) = f(x)/f(y), \hs \tilde{\inv f}(x/y)=\inv f(x)/\inv f(y).
    \end{align*}
    Clearly $\tilde f$ is a homomorphism of fields and thus injective. Let $x/y\in K$
    and check
    \begin{align*}
      (\tilde f\circ\tilde{\inv f})(x/y) = \tilde f(\inv f(x)/\inv f(y)) = (f\circ\inv f)(x)/(f\circ\inv f)(y) = x/y.
    \end{align*}
    Thus $\tilde f$ is surjective. A bijective homomorphism of fields is an isomorphism.
  \end{proof}
\end{claim*}

\begin{claim*}[1b]
  $\tilde f(\overline R)=\overline R$.
  \begin{proof}
    Consider $x/y\in \overline R$. Let $a_1,\ldots,a_d\in R$ be such that
    \begin{align*}
      \rr{\frac{x}{y}}^d + a_1\rr{\frac{x}{y}}^{d-1} + \cdots + a_d = 0.
    \end{align*}
    We note that
    \begin{align*}
      \rr{\frac{f(x)}{f(y)}}^d + f(a_1)\rr{\frac{f(x)}{f(y)}}^{d-1} + \cdots + f(a_d) = 0.
    \end{align*}
    Thus $\tilde f(x/y)=f(x)/f(y)$ is integral over $R$, i.e. $\tilde f(\overline R)\subseteq\overline R$.
    By symmetry, $\tilde{\inv f}(\overline R)\subseteq\overline R$, i.e. $\overline R\subseteq \tilde f(\overline R)$.
    Finally, $f(\overline R)=\overline R$ as required.
  \end{proof}
\end{claim*}

\begin{claim*}[1c]
  The homomorphism $\Aut(R)\to\Aut(\overline R)$ given by $f\mapsto \tilde f$ is injective.
  \begin{proof}
    Consider $f,g\in\Aut(R)$ such that $\tilde f = \tilde g$. In particular,
    for all $x\in R$,
    \begin{align*}
      f(x) = f(x)/1 = f(x)/f(1) = \tilde f(x/1) = \tilde g(x/1) = g(x)/g(1) = g(x)/1 = g(x).
    \end{align*}
    Thus $f=g$ as required.
  \end{proof}
\end{claim*}

Now suppose $k$ is of characteristic zero.

\begin{claim*}[2]
  Let $C:y^2=x^3$. Then $\Aut(C)\cong k^*$.
  \begin{proof}
    Fix $f\in\Aut(C)$ and let $f_1,f_2\in k\bb{x,y}$ such that $f(x,y) = (f_1(x,y),f_2(x,y))$.
    We recall $\tilde C\cong \affine{1}$ by \emph{Proposition 3.67}. By previous considerations
    we have an embedding $\mu:\Aut(C)\to\Aut(\affine{1})$ given by
    \begin{align*}
      \mu(f)(t) = \frac{f_2(t^2,t^3)}{f_1(t^2,t^3)}.
    \end{align*}
    Moreover, we know that automorphisms of $\affine{1}$ are exactly affine transformations.
    Let $a,b\in k$ with $a\neq 0$ such that
    \begin{align}\label{eq:linearity2}
      \mu(f)(t) = at+b.
    \end{align}
    Recall, for all $(x,y)\in C$, $(f_1(x,y))^3 = (f_2(x,y))^2$. Thus
    \begin{align*}
      \frac{(f_2(t^2,t^3))^2}{(f_1(t^2,t^3))^2}
      = f_1(t^2,t^3) = (at + b)^2.
    \end{align*}
    We note that both sides must have degree 2. Thus $f_1(0,y)=0$ for all $y$
    and $\deg f_1(x,0)\leq 1$. Moreover, we must have $2ab = 0$ and hence $b=0$.
    Direct computation shows $f_1(x,y) = a^2 x$ and $f_2(x,y)=a^3 y$.
    In particular, each $a\in k^*$ yields a valid isomorphism and $\mu(f)(t) = at$.
    Thus $\im\mu \cong k^*$.
  \end{proof}
\end{claim*}

\begin{claim*}[3]
  Let $C:y^2=x^2(x-1)$. Then $\Aut(C)\cong \Z/2\Z$.
  \begin{proof}
    Fix $f\in\Aut(C)$ and let $f_1,f_2\in k\bb{x,y}$ such that $f(x,y) = (f_1(x,y),f_2(x,y))$.
    Combining the surjective map $\alpha:\tilde C\to C$ and its right inverse, the 
    isomorphism $\phi:\tilde C\to\affine{1}$, and the first part of this exercise,
    we obtain an embedding $\iota:\Aut(C)\to\Aut(\affine{1})$ given by
    \begin{align*}
      \iota(f)(t) = \frac{f_2(t^2+1,t(t^2+1))}{f_1(t^2+1,t(t^2+1))}.
    \end{align*}
    Moreover, we know that automorphisms of $\affine{1}$ are exactly affine transformations.
    Let $a,b\in k$ with $a\neq 0$ such that
    $\iota(f)(t) = a t + b$. Thus
    \begin{align}\label{eq:linearity}
      \frac{f_2(t^2+1,t(t^2+1))}{f_1(t^2+1,t(t^2+1))} = at+b.
    \end{align}
    Recall, for all $(x,y)\in C$, $(f_2(x,y))^2 = (f_1(x,y))^2(f_1(x,y) - 1)$. Hence
    \begin{align*}
      \frac{(f_2(t^2+1,t(t^2+1)))^2}{(f_1(t^2+1,t(t^2+1)))^2}
      = f_1(t^2+1,t(t^2+1))-1 = (at + b)^2.
    \end{align*}
    We note that both sides must have degree 2. Thus $f_1(0,y)=0$ for all $y$
    and $\deg f(x,0)\leq 1$. Write $f_1(x,y) = cx + d$. Then
    \begin{align*}
      c(t^2+1) + d - 1 = (at+b)^2.
    \end{align*}
    This implies $2ab=0$, i.e. $a=0$ or $b=0$. As $a\neq 0$ we must have $b=0$. Next we
    find $d = 1 - c$ and $c=a^2$. Therefore $f_1(x,y)=a^2x - a^2 + 1$.

    We now compute $f_2$ using (\ref{eq:linearity}):
    \begin{align*}
      f_2(t^2+1,t(t^2+1)) &= a t f_1(t^2+1,t(t^2+1))\\
                          &= a t (a^2 (t^2+1) - a^2 + 1)\\
                          &= a^3 t(t^2+1) - a^3t + a t.
    \end{align*}
    Noting that $f_2$ is a polynomial, we must have $a^3 = a$, i.e. $a^2=1$. As $k$
    has characteristic zero, this implies $a=\pm 1$. Thus $f(x,y)=(x, y)$
    or $f(x,y)=(x,-y)$. Both are valid automorphisms of $C$. This shows that $\Aut(C)$
    has exactly two elements, as required.
  \end{proof}
\end{claim*}

\end{document}
