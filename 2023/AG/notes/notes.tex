\documentclass{article}
\usepackage{notes-preamble}
\usepackage{enumitem}

\begin{document}
\mkthmstwounified
\title{Algebraic Geometry (SEM8)}
\author{Franz Miltz}
\maketitle
\tableofcontents
\pagebreak

\section{Plane curves}

Let $k$ be a field.

\subsection{Affine space}\label{sec:affine_space}

\begin{definition}[Affine space]
  The $n$-dimensional affine space is the set
  \begin{align*}
    \affine n = \cc{\rr{x_1,\ldots,x_n}:x_1,\ldots,x_n\in k}.
  \end{align*}
\end{definition}

\begin{definition}[Curves, Lines, Conics]
  A curve in $\affine n$ is a subset $C\subseteq \affine n$ given by the
  set of roots of a nonconstant polynomial $p\in k\bb{x_1,\ldots,x_n}$.
  In particular
  \begin{enumerate}
    \item if $\deg p = 1$ then $C$ is a line;
    \item if $\deg p = 2$ then $C$ is a conic;
    \item if $\deg p = 3$ then $C$ is a cubic;
    \item if $\deg p = 4$ then $C$ is a quartic.
  \end{enumerate}
\end{definition}

\begin{proposition}
  Suppose
  \begin{align*}
    L_1 : a_1 x + b_1 y + c = 0, \hs
    L_2 : a_2 x + b_2 y + c = 0
  \end{align*}
  are two equal lines. Then there is a nonzero $d\in k$ such that
  \begin{align*}
    a_1 = d a_2, \hs b_1 = d b_2, \hs c_1 = d c_2.
  \end{align*}
\end{proposition}

\begin{proposition}
  Suppose $\rr{x_1,y_1}$ and $\rr{x_2,y_2}$ are distinct points in $\affine 2$. Then
  there is a unique line passing through both of them which is moreover given by the
  equation
  \begin{align*}
    \det\begin{pmatrix}
      x   & y   & 1 \\
      x_1 & y_1 & 1 \\
      x_2 & y_2 & 1
    \end{pmatrix}
    = 0.
  \end{align*}
\end{proposition}

\begin{proposition}
  Suppose $L_1,L_2$ are lines in $\affine 2$. Then exactly one of the following is true:
  \begin{enumerate}
    \item $L_1=L_2$;
    \item $\vv{L_1\cap L_2}=1$;
    \item $\vv{L_1\cap L_2}=0$.
  \end{enumerate}
\end{proposition}

\begin{proposition}
  Let $L\subseteq\affine 2$ a line and $C\subseteq\affine n$ a conic. Then
  exactly one of the following is true:
  \begin{enumerate}
    \item there is a line $L'\subseteq\affine 2$ such that $C=L\cup L'$;
    \item $\vv{C\cap L}=2$.
  \end{enumerate}
\end{proposition}

\begin{definition}[Affine transformation]
  An affine transformation is a map $\affine n\to\affine n$ given by a combination of
  translations
  \begin{align*}
    \rr{x_1,\ldots,x_n} \mapsto \rr{x_1 + c_1,\ldots,x_n+c_n}
  \end{align*}
  for some $c_1,\ldots,c_n\in k$ and linear transformations given by matrices $A\in GL_n\rr{k}$.
\end{definition}

\begin{definition}[Semi-direct product]
  Let $G,H$ be groups and $\phi:G\to\Aut\rr{H}$ a homomorphism.
  Then the semidirect product $G\ltimes H$ is given by the set $G\times H$ and
  the multiplication
  \begin{align*}
    \rr{g_1,h_1}\rr{g_2,h_2} = \rr{g_1g_2,\phi\rr{\inv g_2}\rr{h_1}h_2}.
  \end{align*}
\end{definition}

\begin{proposition}
  The group of affine transformations of $\affine 2$ is the semidirect procut
  $GL_2\rr{k}\ltimes k^2$, where $k^2$ acts by translations, $GL_2\rr{k}$ acts by
  linear change of coordinates, and $GL_2\rr{k}$ acts on $k^2$ in the obvious way.
\end{proposition}

\subsection{Projective space}\label{sec:projective_space}

\begin{definition}[Projective space]
  Let $V$ be a $k$-vector space. The projective space is the set
  \begin{align*}
    \mathbf P\rr{V} = \rr{V\setminus 0} / \sim
  \end{align*}
  where, for $x,y\in V$, $x\sim y$ iff $x=\lambda y$ for some $\lambda\in k$.

  We denote $\proj n = \mathbf P\rr{k^{n+1}}$.
  Points in $\proj n$ are defined by homogeneous coordinates
  $\bb{x_1 : \cdots : x_n}$.
\end{definition}

\begin{definition}
  A curve in $\proj n$ is a subset $C\subseteq\proj n$ such that, for some
  nonconstant $p\in k\bb{x_1,\ldots,x_{n+1}}$, each $\bb{x_1:\ldots:x_{n+1}}\in C$
  yields $p\rr{x_1,\ldots,x_{n+1}}=0$.
\end{definition}

\begin{proposition}
  Consider the point $P=\bb{1:0}\in\proj 1$ and its complement
  $U=\proj 1\setminus P$. Then the map $U\to k$ given by $\bb{x:y}\mapsto x/y$
  is a bijection.
\end{proposition}

\begin{theorem}
  The topological space $\proj 1_{\R}$ is homeomorphic to $S^1$. The topological space
  $\proj 1_{\C}$ is homeomorphic to $S^2$.
\end{theorem}

\begin{proposition}
  Consider the subset $U\subseteq\proj 2$ given by the points $\bb{x:y:z}$ with
  $z\neq 0$. Then $\proj 2\setminus U$ is isomorphic to $\proj 1$.
\end{proposition}

\begin{definition}
  The homogenisation of a curve $C\subseteq\affine n:p\rr{x_1,\ldots,x_n}=0$, where
  $p$ is a polynomial of degree $d$, is the curve
  $\bar C\subseteq\proj n:x_{n+1}^d p\rr{x_1/x_{n+1},\ldots,x_n/x_{n+1}}=0$.

  Conversely, the dehomogenisation of a curve $C\subseteq\proj n:p\rr{x_1,\ldots,x_{n+1}}=0$,
  where $p$ is a homogeneous polynomial, is the curve $\bar C:p\rr{\bar x_1,\ldots,\bar x_n,1}=0$
  with respect to the affine coordinates $\rr{\bar x_j = x_j / x_{n+1}}_{1\leq j\leq n}$.
\end{definition}

\begin{definition}[Projective transformation]
  A projective transformation is a map $\phi : \mathbf P\rr{V}\to\mathbf P\rr{V}$
  obtaines as
  \begin{align*}
    \phi\rr{\bb{v}}=\bb{f\rr{v}}
  \end{align*}
  for some linear map $f:V\to V$.
\end{definition}

\begin{proposition}
  The group of projective transformations is the quotient group
  \begin{align*}
    \text{PGL}\rr{V}=\text{GL}\rr{V}/k^*.
  \end{align*}
\end{proposition}

\begin{proposition}
  Suppose $L\subseteq\proj 2$ is a line. Then there is a projective transformation that
  sends it to the line $L':x=0$.
\end{proposition}

\begin{theorem}
  Let $P_1,\ldots,P_4\in\proj 2$ be points such that no three of them are colinear. Then
  there exists a projective transformation $\phi:\proj 2\to\proj 2$ such that
  \begin{align*}
    \phi\rr{P_1}=\bb{1:0:0},\hs \phi\rr{P_2}=\bb{0:1:0}, \hs\phi\rr{P_3}=\bb{0:0:1},\hs
    \phi\rr{P_4}=\bb{1:1:1}.
  \end{align*}
\end{theorem}

\begin{corollary}
  Let $L\subseteq\proj 2$ be a line and $C\subseteq\proj 2$ a conic. Then
  either $C$ is the union of $L$ with another line or $L\cap C$ at most two points.
  If $k$ is algebraically closed then $L\cap C$ is nonempty.
\end{corollary}

\begin{corollary}
  Let $P_1,\ldots,P_5\in\proj 2$ be points such that no three of them are colinear.
  Then there exists a unique conic passing through them.
\end{corollary}

\subsection{Conics}\label{sec:conics}

\begin{theorem}
  Let $C\subseteq\proj 2_{\C}$ be a conic. Then there exists a projective transformation
  $\phi:\proj 2_{\C}\to\proj 2_{\C}$ such that $\phi\rr{C}$ has one of the following
  forms:
  \begin{enumerate}
    \item $x^2+y^2+z^2=0$ (an irreducible conic);
    \item $x^2+y^2=0$ (a union of two lines);
    \item $x^2=0$ (a double line).
  \end{enumerate}
\end{theorem}

\begin{definition}
  The symmetric matrix induced by a conic $C:ax^2 + bxy + cy^2 + dxz + eyz + fz^2 =0$ is the
  matrix
  \begin{align*}
    B =
    \begin{pmatrix}
      a & b/2 & d/2\\
      b/2 & c & e/2\\
      d/2 & e/2 & f
    \end{pmatrix}.
  \end{align*}
\end{definition}

\begin{corollary}
  A conic $C\subseteq\proj 2$ is irreducible iff $\det\rr{B}\neq 0$ where $B$ is its
  induced symmetric matrix.
\end{corollary}

\begin{theorem}
  Suppose $C,C'\subseteq\proj 2$ are two irreducible unequal conics. Then
  $1\leq\vv{C\cap C'}\leq 4$.
\end{theorem}

\subsection{Smooth curves}\label{sec:smooth_curves}

\begin{definition}
  Suppose $C\subseteq\proj 2:f\rr{x,y,z}=0$ is an irreducible curve. A point
  $P=\bb{x:y:z}$ is smooth iff
  \begin{align*}
    \eval{\rr{\frac{\partial f}{\partial x},\frac{\partial f}{\partial y}, \frac{\partial f}{\partial z}}}{P}\neq 0.
  \end{align*}
  We say $C$ is smooth if every point on $C$ is smooth.
\end{definition}

\begin{definition}
  Let $P=\bb{0:0:1}$ and let $C\subseteq\proj 2$ be a projective curve given
  by the homogeneous polynomial
  \begin{align*}
    f\rr{x,y,z}=z^d h_0\rr{x,y} + \cdots + zh_{d-1}\rr{x,y} + h_{d}
  \end{align*}
  where each $h_n\rr{x,y}$ is homogeneous and of degree $n$. The multiplicity
  $\text{mult}_P\rr{C}$ of $P$ on $C$ is the smallest $n\geq 0$ such that $h_n\rr{x,y}\neq 0$.
\end{definition}

\begin{proposition}
  Suppose $C\subseteq\proj 2$ is a curve and $P\in\proj 2$. Then
  \begin{enumerate}
    \item $P$ lies on $C$ iff $\text{mult}_P\rr{C}\geq 1$;
    \item $P$ is a singular point iff $\text{mult}_P\rr{C}\geq 2$.
  \end{enumerate}
\end{proposition}

\begin{lemma}
  A line $L\subseteq\proj 2$ is smooth.
\end{lemma}

\begin{lemma}
  An irreducible conic $C\subseteq\proj 2_{\C}$ is smooth.
\end{lemma}

\begin{definition}
  Let $C:f\rr{x,y,z}=0$ be an irreducible curve in $\proj 2$ and $P=\bb{\alpha:\beta:\gamma}$
  a smooth point. Then
  \begin{align*}
    \frac{\partial f}{\partial x}\rr{\alpha,\beta,\gamma}
    +\frac{\partial f}{\partial y}\rr{\alpha,\beta,\gamma}
    +\frac{\partial f}{\partial z}\rr{\alpha,\beta,\gamma}
    =0
  \end{align*}
  is the tangent line at $P$.
\end{definition}

\begin{definition}
  Suppose $C_1$ and $C_2$ are two curves in $\proj 2$ intersecting at $P$. We say this
  intersection is transverse if
  \begin{enumerate}
    \item $P$ is a smooth point of $C_1$ and $C_2$;
    \item the tangent lines at $P$ to $C_1$ and $C_2$ are different.
  \end{enumerate}
\end{definition}

\begin{proposition}
  Let $k$ be an algebraically closed field. Suppose $L\subseteq\proj 2_k$ is a line and
  $C\subseteq\proj 2_k$ an irreducible conic. Then either $L$ is tangent to $C$ at some point
  or $L$ intersects $C$ transversely at two points.
\end{proposition}

\begin{proposition}
  Suppose $C_1,C_2\subseteq\proj 2_{\mathbb C}$ are two distinct irreducible concis. Then they
  intersect transversally at all points iff there are four intersection points.
\end{proposition}

\subsection{Bezout's theorem}\label{sec:bezouts_theorem}

Let $k$ be algebraically closed.

\begin{theorem}
  Suppose $f\rr{x,y}$ is a nonzero homogeneous polyonomial of degree $d$. Then the equation
  $f\rr{x,y}=0$ in $\proj 1$ consists of $d$ points counted with multiplicity.
\end{theorem}

\begin{definition}
  Let $f,g\in k\bb{x,y}$ be nonconstant without common factors. Let
  $I=\rr{f,g}\subseteq k\bb{x,y}$ be the ideal generated by $f$ and $g$. The intersection
  multiplicity of $f$ and $g$ is $\dim_k\rr{R/I}$.

  More generally, for nonzero homogeneus $f,g\in k\bb{x,y,z}$ with no common factors
  defining curves $C_1,C_2\subseteq\proj 2$ and $P\in C_1\cap C_2$, we define the
  intersection multiplicity $I_P\rr{C_1,C_2}$ to be the intersection multiplicity of
  $\rr{f\circ\phi}\rr{x,y,1}$ and $\rr{g\circ\phi}\rr{x,y,1}$ where $\phi$ is a projective
  transformation with $\phi[0:0:1]=P$.
\end{definition}

\begin{proposition}
  Suppose $C_1$ and $C_2$ are distinct irreducible curves in $\proj 2$ and $P\in C_1\cap C_2$.
  Then
  \begin{align*}
    I_P\rr{C_1,C_2}\geq\mult_P\rr{C_1}\mult_P\rr{C_2}.
  \end{align*}
\end{proposition}

\begin{proposition}
  Suppose $f,g\in k\bb{x,y,z}$ are nonzero, homogeneous, and without common factors such that
  $f\rr{P}=g\rr{P}=0$ for some $P\in\proj 2$. Then
  \begin{enumerate}
    \item $I_P\rr{f,g}\geq 1$;
    \item Suppose $h\in k\bb{x,y,z}$ is homogeneous with $h\rr{P}=0$ then
      \begin{align*}
        I_P\rr{f,gh}=I_P\rr{f,g}+I_P\rr{f,h};
      \end{align*}
    \item Suppose $h\in k\bb{x,y,z}$ is homogeneous with $h\rr{P}\neq 0$ then
      \begin{align*}
        I_P\rr{f,gh}=I_P\rr{f,g}.
      \end{align*}
    \item The intersection of $f\rr{x,y,z}=0$ and $g\rr{x,y,z}=0$ is transverse at $P$ iff
      $I_P\rr{f,g}=1$.
  \end{enumerate}
\end{proposition}

\begin{theorem}[Bezout]
  Suppose $f,g\in k\bb{x,y,z}$ are homogeneous and without common factors. Then the solutions
  of
  \begin{align*}
    f\rr{x,y,z}&=0\\
    g\rr{x,y,z}&=0
  \end{align*}
  in $\proj 2$ are given by $\rr{\deg f}\rr{\deg g}$ points counted with multiplicity.
\end{theorem}

\begin{proposition}
  Let $f\in k\bb{x,y,z}$ be homogeneous. If the system
  \begin{align*}
    \frac{\partial f}{\partial x}=\frac{\partial f}{\partial y}=\frac{\partial f}{\partial z}=0
  \end{align*}
  as no solutions in $\proj 2$ then $f$ is irreducible.
\end{proposition}

\begin{proposition}
  Suppose $C$ is an irreducible curve of degree $d\geq 2$ and $P,Q\in C$ are distinct points.
  Then
  \begin{align*}
    \mult_P\rr{C}+\mult_Q\rr{C} \leq d.
  \end{align*}
\end{proposition}

\begin{proposition}
  Let $C\subseteq\proj 2$ be an irreducible curve of degree 4. Then it has at most three
  singular points.
\end{proposition}

\begin{definition}
  Let $\Sigma\subset\proj 2$ be a finite set of points. Define the space
  \begin{align*}
    S_d(\Sigma) = \cc{f \in S_d : \forall P\in\Sigma.\: f(P)=0}
  \end{align*}
  where $S_d$ is the space of all homogeneous degree $d$ polynomials. Then $\Sigma$
  \emph{imposes independent conditions on $S_d$} if
  $\dim S_d\rr{\Sigma} = \dim S_d - \vv{\Sigma}$.
\end{definition}

\todo{missing content}

\section{Cubic curves}

\begin{definition}
  Let $C\subseteq\proj 2$ be a projective curve. A point $P\in C$ is an
  \emph{inflcetion point} if it is smooth and $I_P(L,C)\geq 3$ where $L$ is
  the tangent line at $P$.
\end{definition}

\begin{definition}
  Let $f\in k\bb{x,y,z}$ be homogeneous. Its \emph{Hessian} is
  \begin{align*}
    \Hess f = \det
    \begin{pmatrix}
      \frac{\partial^2 f}{\partial x^2} &
      \frac{\partial^2 f}{\partial x\partial y} &
      \frac{\partial^2 f}{\partial x\partial z} \\
      \frac{\partial^2 f}{\partial x\partial y} &
      \frac{\partial^2 f}{\partial y^2} &
      \frac{\partial^2 f}{\partial y\partial z} \\
      \frac{\partial^2 f}{\partial x\partial z} &
      \frac{\partial^2 f}{\partial y\partial z} &
      \frac{\partial^2 f}{\partial z^2}
    \end{pmatrix}
  \end{align*}
\end{definition}

\begin{theorem}
  Let $f$ be a homogeneous polynomial of degree $d\geq 3$ with coefficients in a field
  $k$ of characteristic zero or greater than $d$
\end{theorem}

\section{Commutative algebra}

\subsection{Noetherian rings}

\begin{definition}
  An ideal $I$ of a commutative ring $R$ is \emph{finitely generated} if, for some
  $f_1,\ldots,f_n\in R$, $I=\aa{f_1,\ldots,f_n}$.
\end{definition}

\begin{definition}
  A commutative ring $R$ is \emph{Noetherian} if all its ideals are finitely generated.
\end{definition}

\begin{proposition}
  Let $R$ be a Noetherian ring and $I\subseteq R$ an ideal. Then $R/I$ is Noetherian.
\end{proposition}

\begin{theorem}
  Let $R$ be a commutative ring. Then tfae
  \begin{enumerate}
    \item $R$ is Noetherian.
    \item Every ascending chain of ideals stabilises.
    \item Every nonempty set of ideals in $R$ has a maximal element.
  \end{enumerate}
\end{theorem}

\begin{theorem}[Hilbert]
  If $R$ is a Noetherian ring then $R\bb{x}$ is Noetherian.
\end{theorem}

\begin{corollary}
  Let $R$ be Noetherian. For all $n$ and all ideals $I\subseteq R\bb{x_1,\ldots,x_n}$,
  the quotient $R\bb{x_1,\ldots,x_n}/I$ is Noetherian.
\end{corollary}

\begin{proposition}
  Suppose $k$ is a field. Then the commutative ring $R=k\bb{x_1,x_2,\ldots}$ of polynomials
  in countably many variables is not Noetherian.
\end{proposition}

\subsection{Nullstellensatz}

Let $k$ be an algebraically closed field.

\begin{definition}
  Let $\Sigma\subseteq\affine n$ be a subset. The \emph{vanishing ideal}
  $I(\Sigma)\subseteq k\bb{x_1,\ldots,x_n}$ consists of polynomials
  $f\in k\bb{x_1,\ldots,x_n}$ such that $f\rr{P}=0$ for every $P\in\Sigma$.
\end{definition}

\begin{definition}
  Let $I\subseteq k\bb{x_1,\ldots,x_n}$ be an ideal. Then $V(I)\subseteq\affine n$
  consists of points $P\in\affine n$ such that $f(P)=0$ for every $f\in I$. Any such

  A subset $X\subseteq\affine n$ is \emph{algebraic} if there is an ideal $I$ such that
  $X=V(I)$.
\end{definition}

\begin{lemma}
  Let $\Sigma\subseteq\affine n$ be a subset. Then $\Sigma\subseteq V(I(\Sigma))$.
\end{lemma}

\begin{lemma}
  Let $I\subseteq k\bb{x_1,\ldots,x_n}$ be an ideal. Then $I\subseteq I(V(I))$.
\end{lemma}

\begin{definition}
  Let $R$ be a commutative ring and $I\subseteq R$ an ideal. Its \emph{radical}
  $\sqrt{I}\subseteq R$ is the set of $f\in R$ such that $f^m \in I$ for some
  $m$. The ideal $I$ is \emph{radical} if $\sqrt{I}=I$.
\end{definition}

\begin{theorem}[Nullstellensatz]
  Let $I\subseteq k\bb{x_1,\ldots,x_n}$ be an ideal. Then $\sqrt{I}=I(V(I))$.
\end{theorem}

\begin{corollary}
  There is a $1:1$ correspondence
  \begin{align*}
    V:\cc{\text{radical ideals of $k\bb{x_1,\ldots,x_n}$}}
    \cong \cc{\text{algebraic subsets of $\affine n$}}:I.
  \end{align*}
\end{corollary}

\begin{definition}
  An algebraic subset $\affine n$ is \emph{irreducible} if it is not a union of
  two distinct algebraic subsets.
\end{definition}

\begin{corollary}
  There is a $1:1$ correspondence
  \begin{align*}
    V:\cc{\text{prime ideals of $k\bb{x_1,\ldots,x_n}$}}
    \cong \cc{\text{irreducible algebraic subsets of $\affine n$}}:I.
  \end{align*}
\end{corollary}

\subsection{Polynomial maps}

\begin{definition}
  A \emph{polynomial function} $\affine n\to \affine 1$ is a function of the form
  $P\mapsto f(P)$ for some $f\in k\bb{x_1,\ldots,x_n}$.
\end{definition}

\begin{definition}
  Let $X\subseteq\affine{n}$ be an algebraic subset. The \emph{coordinate ring} is
  $k\bb{X}=k\bb{x_1,\ldots,x_n}/I(X)$.
\end{definition}

\begin{definition}
  Let $X\subseteq\affine{n}$ and $Y\subseteq\affine{m}$ be algebraic subsets. A
  \emph{polynomial map} $f:X\to Y$ is a map given by $P\mapsto (f_1(P),\ldots,f_m(P))$
  for some $f_1,\ldots,f_m\in k\bb{x_1,\ldots,x_n}$.

  The \emph{induced homomorphism of $k$-algebras} of a polynomial map $f:X\to Y$
  is
  \begin{align*}
    f^* = - \circ f : k\bb{Y}\to k\bb{X}.
  \end{align*}
\end{definition}

\begin{lemma}
  Let $X,Y$ be algebraic subsets. Suppose $\Phi : k\bb{Y}\to k\bb{X}$ is a homomorphism
  of $k$-algebras. Then $\Phi=f^*$ for a unique polynomial map $f:X\to Y$.
\end{lemma}

\subsection{Affine algebraic sets}

\begin{definition}
  An \emph{affine algebraic set} is an algebraic subset $X\subseteq\affine{n}$ considered
  up to isomorphism.
\end{definition}

\begin{definition}
  A commutative ring $R$ is \emph{reduced} if $f^N=0$ for $f\in R$ implies that $f=0$.
\end{definition}

\begin{lemma}
  The commutative ring $k\bb{x_1,\ldots,x_n}$ is reduced iff $I$ is radical.
\end{lemma}

\begin{theorem}
  Let $k$ be an algebraically closed field. There is a $1:1$ correspondence
  between
  \begin{enumerate}
    \item affine algebraic subsets and
    \item reduced finitely generated $k$-algebras (up to isomorphism)
  \end{enumerate}
  given by $X\mapsto k\bb{X}$.
\end{theorem}

\begin{proposition}
  Suppose $f:X\to Y$ is a morphism of affine algebraic sets which is surjective. Then
  the induced map of coordinate rings $f^*$ is injective.
\end{proposition}

\subsection{Affine varieties}

\begin{definition}
  An \emph{affine variety} is an irreducible affine algebraic set.
\end{definition}

\begin{theorem}
  Let $k$ be an algebraically closed field. There is a $1:1$ correspondence
  between
  \begin{enumerate}
    \item affine varieties and
    \item finitely generated $k$-algebras that are integral domains (up to isomorphism)
  \end{enumerate}
  given by $X\mapsto k\bb{X}$.
\end{theorem}

\begin{proposition}
  Let $k$ be algebraically closed. A prime ideal $I\subseteq k\bb{x,y}$ is one of the
  following:
  \begin{itemize}
    \item $I=0$;
    \item $I=\aa{f}$ fo ran irreducible $f\in k\bb{x,y}$;
    \item $I=\aa{x-a,y-b}$ for $a,b\in k$.
  \end{itemize}
\end{proposition}

\begin{definition}
  Let $R$ be an integral domain and $K$ its field of fractions.
  \begin{itemize}
    \item An element $\alpha\in K$ is \emph{integral over $R$} if there are elements
      $a_1,\ldots,a_d\in R$ such that $\alpha^d + a_1\alpha^{d-1} +\cdots + a_d = 0$.
    \item The \emph{integral closure} $\overline R\subseteq K$ is the set of elements integral over $R$.
    \item $R$ is \emph{integrally closed} if $\overline R=R$.
  \end{itemize}
\end{definition}

\begin{definition}
  An affine variety $X$ is \emph{normal} if $k\bb{X}$ is integrally closed.
\end{definition}

\begin{definition}
  Let $X$ be an affine variety. The \emph{normalisation $\overline X$} is the unique
  affine variety corresponding to the integral closure $\overline{k\bb{X}}$.
\end{definition}

\begin{definition}
  Let $X$ be an affine variety. The \emph{function field $k(X)$} is the field of fractions of $k\bb{X}$.
\end{definition}

\begin{proposition}
  Every unique factorisation domain is integrally closed.
\end{proposition}

\begin{theorem}[Zariski]
  A curve in $\affine{2}$ is smooth iff it is normal.
\end{theorem}

\subsection{Zarisiki topology}

\begin{definition}
  The \emph{Zariski topology} on $\affine{n}$ has as closed subsets exactly the algebraic
  subsets.

  Using the subset topology, we obtain topologies on all algebraic subsets $X\subseteq \affine{n}$.
\end{definition}

\begin{proposition}
  If $f:X\to Y$ is a polynomial function then it is continuous in the Zariski topology.
\end{proposition}

\subsection{Rational maps between affine varieties}

\begin{definition}
  Let $X$ be an affine variety and let $P\in X$. A rational function $\phi\in k(X)$ is
  \emph{regular at $P$} if there are $f,g\in k\bb{X}$ such that $\phi = f/g$ and $g\rr{P}\neq 0$.
  The \emph{domain of definition} $\dom\phi\subseteq X$ is the subset where the function
  is regular.
\end{definition}

\begin{definition}
  Let $X$ be an affine variety. A \emph{rational map $f:X\dashto\affine{m}$} is a collection of
  rational functions $f_1,\ldots,f_m\in k(X)$. Its \emph{domain} is
  $\dom(f) = \bigcap_{i = 1}^m \dom(f_i)$.
\end{definition}

\begin{definition}
  Let $X$ be an affine variety and let $Y\subseteq\affine{m}$. A \emph{rational map}
  $f:X\dashto Y$ is a rational map $f:X\dashto\affine{m}$ with $f(\dom(f))\subseteq Y$.
\end{definition}

\begin{theorem}
  Let $\phi:X\dashto Y$ be rational map of affine varieties.
  \begin{itemize}
    \item $\dom(\phi)\subseteq X$ is open and dense in the Zariski topology.
    \item $\dom(\phi)=X$ iff $\phi$ is a polynomial map.
  \end{itemize}
\end{theorem}

\begin{definition}
  A map $\phi : X\dashto Y$ is \emph{dominant} if $\phi(\dom(\phi))$ is dense in $Y$.
\end{definition}

\begin{proposition}
  If $\phi:X\dashto Y$ is dominant and $\psi:Y\dashto Z$ is arbitrary, then $\psi\circ\phi$
  is well-defined.
\end{proposition}

\begin{lemma}
  If $\Phi:k(Y)\to k(X)$ is a homomorphism of $k$-algebras then there is a unique dominant
  map $\phi:X\dashto Y$ such that $\Phi = \phi^*$.
\end{lemma}

\begin{definition}
  A dominant rational map $\phi:X\dashto Y$ is \emph{birational} if it is an isomorphism
  in the category of affine varieties and rational maps. In this case $X$ and $Y$
  are \emph{birational}.
\end{definition}

\begin{corollary}
  $X$ and $Y$ are birational \iff $k(X)\cong k(Y)$ as $k$-algebras.
\end{corollary}

\begin{definition}
  Let $X$ be an affine variety. THe group of birational maps $X\dashto X$ is $\Bir(X)$.
  $\Bir\rr{\affine{n}}$ is the \emph{Cremona group}.
\end{definition}

\begin{theorem}[Noether]
  $\Bir\rr{\affine{2}}$ is generated by $(x,y)\mapsto (1/x,1/y)$ and projective transformations
  \begin{align*}
    (x,y)\mapsto \rr{\frac{a_{11}x + a_{12}y + a_{13}}{a_{31}x + a_{32}y + a_{33}},
    \frac{a_{21}x + a_{22}y + a_{23}}{a_{31}x + a_{32}y + a_{33}}}
  \end{align*}
  where the matrix $(a_{ij})$ is invertible.
\end{theorem}

\subsection{Dimension}

\begin{definition}
  Let $L,K$ be fields. A \emph{field extension} $L/K$ is an embedding of fields $K\to L$.
  Given a field extension $L/K$ and a collection of elements $\alpha_1,\ldots,\alpha_n\in L$,
  $K(\alpha_1,\ldots,\alpha_n)$ is the smallest subfield of $L$ that contains $K$ and the
  elements $\alpha_1,\ldots,\alpha_n$.
\end{definition}

\begin{definition}
  Consider an extension $L/K$ of fields.
  \begin{itemize}
    \item An element $\alpha\in L$ is \emph{algebraic} over $K$ if it satisfies a monic
      polynomial equation of degree $d\geq 1$ with coefficients in $K$. Otherwise, $\alpha$
      is \emph{transcendental}.
    \item An extension $L/K$ is \emph{algebraic} if every element of $L$ is algebraic over $K$.
      Otherwise $L/K$ is \emph{transcendental}.
  \end{itemize}
\end{definition}

\begin{proposition}
  Suppose $L/K$ is a transcendental extension. Consider $L$ as a vector space over the field
  $K$. Then its dimension is infinite.
\end{proposition}

\begin{definition}
  Consider an extension $L/K$ of fields.
  \begin{itemize}
    \item A set $S\subseteq L$ is \emph{algebraically independent} over $K$ if elements in
      $S$ do not satisfy a nontrivial polynomial equation with coefficients in $K$.
    \item The \emph{transcendence degree} of the extension $L/K$ is the largest cardinality
      of an algebraically independent subset of $L$ over $K$.
  \end{itemize}
\end{definition}

\begin{definition}
  Let $X$ be an affine variety. Its \emph{dimension} $\dim(X)$ is the transcendence
  degree of the field extension $k(X)/k$.
\end{definition}

\begin{proposition}
  Consider field extensions $F/L/K$. The transcendence degree of $F/K$ is the sum of
  transcendence degrees of $L/K$ and $F/L$.
\end{proposition}

\begin{proposition}
  Consider the affine variety $X\subseteq\affine{n}$ given by solutions of a nontrivial
  polynomial equation $f(x_1,\ldots,x_n)=0$. Then $k(X)/k$ has transcendence degree $n-1$,
  i.e. $\dim(X)=n-1$.
\end{proposition}

\subsection{Projective varieties}

\begin{definition}
  The \emph{homogeneous components} of a polynomial $f\in k[x_0,\ldots,x_n]$ are
  homogeneous polynomials $f_0,\ldots,f_d$ of a fixed degree such that $f = \sum_i f_i$.

  An ideal $I\subseteq k\bb{x_0,\ldots,x_n}$ is \emph{homogeneous} if, for every $f\in I$,
  its homogeneous components lie in $I$.
\end{definition}

\begin{lemma}
  An ideal $I\subseteq k[x_0,\ldots,x_n]$ is homogeneous \iff{} it is generated by homogeneous
  polynomials.
\end{lemma}

\begin{definition}
  Let $I\subseteq k[x_0,\ldots,x_n]$ be a homogeneous ideal. The \emph{vanishing set}
  is $V(I)\subseteq\affine{n}$ consisting of points $P=[x_0 : \cdots : x_n]$ such that
  $h(P)=0$ for all $h\in I$. Every such subset is \emph{algebraic}.
\end{definition}

\begin{definition}
  Let $X\subseteq\proj{n}$ be a subset. The \emph{ideal of vanishing $I(X)$} consists of
  homogeneous $k\in k[x_0,\ldots,x_n]$ such that $h(P)=0$.
\end{definition}

\begin{theorem}
  Let $I\subseteq k[x_0,\ldots,x_n]$ be a homogeneous ideal.
  \begin{itemize}
    \item $V(I)=\emptyset$ \iff{} $(x_0,\ldots,x_n)\subseteq\sqrt I$.
    \item If $V(I)\neq\emptyset$ then $\sqrt I = I(V(I))$.
  \end{itemize}
\end{theorem}

\begin{corollary}
  There is a 1:1 correspondence between the following:
  \begin{itemize}
    \item homogeneous radical ideals of $k[x_0,\ldots,x_n]$ not containing $(x_0,\ldots,x_n)$;
    \item algebraic subsets of $\proj{n}$.
  \end{itemize}
\end{corollary}

\begin{definition}
  The \emph{Zariski topology} on $\proj{n}$ has as closed subsets exactly the algebraic
  subsets.

  Using the subset topology, we obtain topologies on all algebraic subsets $X\subseteq \proj{n}$.
\end{definition}

\end{document}
