\documentclass{article}
\usepackage{notes-preamble}
\usepackage{enumitem}
\begin{document}
\mkthmstwounified
\title{Algebraic Geometry (SEM8)}
\author{Franz Miltz}
\maketitle
\tableofcontents
\pagebreak

\section{Plane curves}

Let $k$ be a field.

\subsection{Affine space}\label{sec:affine_space}

\begin{definition}[Affine space]
  The $n$-dimensional affine space is the set
  \begin{align*}
    \mathbf A^n = \cc{\rr{x_1,\ldots,x_n}:x_1,\ldots,x_n\in k}.
  \end{align*}
\end{definition}

\begin{definition}[Curves, Lines, Conics]
  A curve in $\mathbf A^n$ is a subset $C\subseteq \mathbf A^n$ given by the
  set of roots of a nonconstant polynomial $p\in k\bb{x_1,\ldots,x_n}$.
  In particular
  \begin{enumerate}
    \item if $\deg p = 1$ then $C$ is a line;
    \item if $\deg p = 2$ then $C$ is a conic;
    \item if $\deg p = 3$ then $C$ is a cubic;
    \item if $\deg p = 4$ then $C$ is a quartic.
  \end{enumerate}
\end{definition}

\begin{proposition}
  Suppose
  \begin{align*}
    L_1 : a_1 x + b_1 y + c = 0, \hs
    L_2 : a_2 x + b_2 y + c = 0
  \end{align*}
  are two equal lines. Then there is a nonzero $d\in k$ such that
  \begin{align*}
    a_1 = d a_2, \hs b_1 = d b_2, \hs c_1 = d c_2.
  \end{align*}
\end{proposition}

\begin{proposition}
  Suppose $\rr{x_1,y_1}$ and $\rr{x_2,y_2}$ are distinct points in $\mathbf A^2$. Then
  there is a unique line passing through both of them which is moreover given by the
  equation
  \begin{align*}
    \det\begin{pmatrix}
      x   & y   & 1 \\
      x_1 & y_1 & 1 \\
      x_2 & y_2 & 1
    \end{pmatrix}
    = 0.
  \end{align*}
\end{proposition}

\begin{proposition}
  Suppose $L_1,L_2$ are lines in $\mathbf A^2$. Then exactly one of the following is true:
  \begin{enumerate}
    \item $L_1=L_2$;
    \item $\vv{L_1\cap L_2}=1$;
    \item $\vv{L_1\cap L_2}=0$.
  \end{enumerate}
\end{proposition}

\begin{proposition}
  Let $L\subseteq\mathbf A^2$ a line and $C\subseteq\mathbf A^n$ a conic. Then
  exactly one of the following is true:
  \begin{enumerate}
    \item there is a line $L'\subseteq\mathbf A^2$ such that $C=L\cup L'$;
    \item $\vv{C\cap L}=2$.
  \end{enumerate}
\end{proposition}

\begin{definition}[Affine transformation]
  An affine transformation is a map $\mathbf A^n\to\mathbf A^n$ given by a combination of
  translations
  \begin{align*}
    \rr{x_1,\ldots,x_n} \mapsto \rr{x_1 + c_1,\ldots,x_n+c_n}
  \end{align*}
  for some $c_1,\ldots,c_n\in k$ and linear transformations given by matrices $A\in GL_n\rr{k}$.
\end{definition}

\begin{definition}[Semi-direct product]
  Let $G,H$ be groups and $\phi:G\to\Aut\rr{H}$ a homomorphism.
  Then the semidirect product $G\ltimes H$ is given by the set $G\times H$ and
  the multiplication
  \begin{align*}
    \rr{g_1,h_1}\rr{g_2,h_2} = \rr{g_1g_2,\phi\rr{\inv g_2}\rr{h_1}h_2}.
  \end{align*}
\end{definition}

\begin{proposition}
  The group of affine transformations of $\mathbf A^2$ is the semidirect procut
  $GL_2\rr{k}\ltimes k^2$, where $k^2$ acts by translations, $GL_2\rr{k}$ acts by
  linear change of coordinates, and $GL_2\rr{k}$ acts on $k^2$ in the obvious way.
\end{proposition}

\subsection{Projective space}\label{sec:projective_space}

\begin{definition}[Projective space]
  Let $V$ be a $k$-vector space. The projective space is the set
  \begin{align*}
    \mathbf P\rr{V} = \rr{V\setminus 0} / \sim
  \end{align*}
  where, for $x,y\in V$, $x\sim y$ iff $x=\lambda y$ for some $\lambda\in k$.

  We denote $\mathbf P^n = \mathbf P\rr{k^{n+1}}$.
  Points in $\mathbf P^n$ are defined by homogeneous coordinates
  $\bb{x_1 : \cdots : x_n}$.
\end{definition}

\begin{definition}
  A curve in $\mathbf P^n$ is a subset $C\subseteq\mathbf P^n$ such that, for some
  nonconstant $p\in k\bb{x_1,\ldots,x_{n+1}}$, each $\bb{x_1:\ldots:x_{n+1}}\in C$
  yields $p\rr{x_1,\ldots,x_{n+1}}=0$.
\end{definition}

\begin{proposition}
  Consider the point $P=\bb{1:0}\in\mathbf P^1$ and its complement
  $U=\mathbf P^1\setminus P$. Then the map $U\to k$ given by $\bb{x:y}\mapsto x/y$
  is a bijection.
\end{proposition}

\begin{theorem}
  The topological space $\mathbf P^1_{\R}$ is homeomorphic to $S^1$. The topological space
  $\mathbf P^1_{\C}$ is homeomorphic to $S^2$.
\end{theorem}

\begin{proposition}
  Consider the subset $U\subseteq\mathbf P^2$ given by the points $\bb{x:y:z}$ with
  $z\neq 0$. Then $\mathbf P^2\setminus U$ is isomorphic to $\mathbf P^1$.
\end{proposition}

\begin{definition}
  The homogenisation of a curve $C\subseteq\mathbf A^n:p\rr{x_1,\ldots,x_n}=0$, where
  $p$ is a polynomial of degree $d$, is the curve
  $\bar C\subseteq\mathbf P^n:x_{n+1}^d p\rr{x_1/x_{n+1},\ldots,x_n/x_{n+1}}=0$.

  Conversely, the dehomogenisation of a curve $C\subseteq\mathbf P^n:p\rr{x_1,\ldots,x_{n+1}}=0$,
  where $p$ is a homogeneous polynomial, is the curve $\bar C:p\rr{\bar x_1,\ldots,\bar x_n,1}=0$
  with respect to the affine coordinates $\rr{\bar x_j = x_j / x_{n+1}}_{1\leq j\leq n}$.
\end{definition}

\begin{definition}[Projective transformation]
  A projective transformation is a map $\phi : \mathbf P\rr{V}\to\mathbf P\rr{V}$
  obtaines as
  \begin{align*}
    \phi\rr{\bb{v}}=\bb{f\rr{v}}
  \end{align*}
  for some linear map $f:V\to V$.
\end{definition}

\begin{proposition}
  The group of projective transformations is the quotient group
  \begin{align*}
    \text{PGL}\rr{V}=\text{GL}\rr{V}/k^*.
  \end{align*}
\end{proposition}

\begin{proposition}
  Suppose $L\subseteq\mathbf P^2$ is a line. Then there is a projective transformation that
  sends it to the line $L':x=0$.
\end{proposition}

\begin{theorem}
  Let $P_1,\ldots,P_4\in\mathbf P^2$ be points such that no three of them are colinear. Then
  there exists a projective transformation $\phi:\mathbf P^2\to\mathbf P^2$ such that
  \begin{align*}
    \phi\rr{P_1}=\bb{1:0:0},\hs \phi\rr{P_2}=\bb{0:1:0}, \hs\phi\rr{P_3}=\bb{0:0:1},\hs
    \phi\rr{P_4}=\bb{1:1:1}.
  \end{align*}
\end{theorem}

\begin{corollary}
  Let $L\subseteq\mathbf P^2$ be a line and $C\subseteq\mathbf P^2$ a conic. Then
  either $C$ is the union of $L$ with another line or $L\cap C$ at most two points.
  If $k$ is algebraically closed then $L\cap C$ is nonempty.
\end{corollary}

\begin{corollary}
  Let $P_1,\ldots,P_5\in\mathbf P^2$ be points such that no three of them are colinear.
  Then there exists a unique conic passing through them.
\end{corollary}

\subsection{Conics}\label{sec:conics}

\begin{theorem}
  Let $C\subseteq\mathbf P^2_{\C}$ be a conic. Then there exists a projective transformation
  $\phi:\mathbf P^2_{\C}\to\mathbf P^2_{\C}$ such that $\phi\rr{C}$ has one of the following
  forms:
  \begin{enumerate}
    \item $x^2+y^2+z^2=0$ (an irreducible conic);
    \item $x^2+y^2=0$ (a union of two lines);
    \item $x^2=0$ (a double line).
  \end{enumerate}
\end{theorem}

\begin{definition}
  The symmetric matrix induced by a conic $C:ax^2 + bxy + cy^2 + dxz + eyz + fz^2 =0$ is the
  matrix
  \begin{align*}
    B =
    \begin{pmatrix}
      a & b/2 & d/2\\
      b/2 & c & e/2\\
      d/2 & e/2 & f
    \end{pmatrix}.
  \end{align*}
\end{definition}

\begin{corollary}
  A conic $C\subseteq\mathbf P^2$ is irreducible iff $\det\rr{B}\neq 0$ where $B$ is its
  induced symmetric matrix.
\end{corollary}

\begin{theorem}
  Suppose $C,C'\subseteq\mathbf P^2$ are two irreducible unequal conics. Then
  $1\leq\vv{C\cap C'}\leq 4$.
\end{theorem}

\subsection{Smooth curves}\label{sec:smooth_curves}

\begin{definition}
  Suppose $C\subseteq\mathbf P^2:f\rr{x,y,z}=0$ is an irreducible curve. A point
  $P=\bb{x:y:z}$ is smooth iff
  \begin{align*}
    \eval{\rr{\frac{\partial f}{\partial x},\frac{\partial f}{\partial y}, \frac{\partial f}{\partial z}}}{P}\neq 0.
  \end{align*}
  We say $C$ is smooth if every point on $C$ is smooth.
\end{definition}

\begin{definition}
  Let $P=\bb{0:0:1}$ and let $C\subseteq\mathbf P^2$ be a projective curve given
  by the homogeneous polynomial
  \begin{align*}
    f\rr{x,y,z}=z^d h_0\rr{x,y} + \cdots + zh_{d-1}\rr{x,y} + h_{d}
  \end{align*}
  where each $h_n\rr{x,y}$ is homogeneous and of degree $n$. The multiplicity
  $\text{mult}_P\rr{C}$ of $P$ on $C$ is the smallest $n\geq 0$ such that $h_n\rr{x,y}\neq 0$.
\end{definition}

\begin{proposition}
  Suppose $C\subseteq\mathbf P^2$ is a curve and $P\in\mathbf P^2$. Then
  \begin{enumerate}
    \item $P$ lies on $C$ iff $\text{mult}_P\rr{C}\geq 1$;
    \item $P$ is a singular point iff $\text{mult}_P\rr{C}\geq 2$.
  \end{enumerate}
\end{proposition}

\begin{lemma}
  A line $L\subseteq\mathbf P^2$ is smooth.
\end{lemma}

\begin{lemma}
  An irreducible conic $C\subseteq\mathbf P^2_{\C}$ is smooth.
\end{lemma}

\begin{definition}
  Let $C:f\rr{x,y,z}=0$ be an irreducible curve in $\mathbf P^2$ and $P=\bb{\alpha:\beta:\gamma}$
  a smooth point. Then
  \begin{align*}
    \frac{\partial f}{\partial x}\rr{\alpha,\beta,\gamma}
    +\frac{\partial f}{\partial y}\rr{\alpha,\beta,\gamma}
    +\frac{\partial f}{\partial z}\rr{\alpha,\beta,\gamma}
    =0
  \end{align*}
  is the tangent line at $P$.
\end{definition}

\begin{definition}
  Suppose $C_1$ and $C_2$ are two curves in $\mathbf P^2$ intersecting at $P$. We say this
  intersection is transverse if
  \begin{enumerate}
    \item $P$ is a smooth point of $C_1$ and $C_2$;
    \item the tangent lines at $P$ to $C_1$ and $C_2$ are different.
  \end{enumerate}
\end{definition}

\begin{proposition}
  Let $k$ be an algebraically closed field. Suppose $L\subseteq\mathbf P^2_k$ is a line and
  $C\subseteq\mathbf P^2_k$ an irreducible conic. Then either $L$ is tangent to $C$ at some point
  or $L$ intersects $C$ transversely at two points.
\end{proposition}

\begin{proposition}
  Suppose $C_1,C_2\subseteq\mathbf P^2_{\mathbb C}$ are two distinct irreducible concis. Then they
  intersect transversally at all points iff there are four intersection points.
\end{proposition}

\subsection{Bezout's theorem}\label{sec:bezouts_theorem}

Let $k$ be algebraically closed.

\begin{theorem}
  Suppose $f\rr{x,y}$ is a nonzero homogeneous polyonomial of degree $d$. Then the equation
  $f\rr{x,y}=0$ in $\mathbf P^1$ consists of $d$ points counted with multiplicity.
\end{theorem}

\begin{definition}
  Let $f,g\in k\bb{x,y}$ be nonconstant without common factors. Let
  $I=\rr{f,g}\subseteq k\bb{x,y}$ be the ideal generated by $f$ and $g$. The intersection
  multiplicity of $f$ and $g$ is $\dim_k\rr{R/I}$.

  More generally, for nonzero homogeneus $f,g\in k\bb{x,y,z}$ with no common factors
  defining curves $C_1,C_2\subseteq\mathbf P^2$ and $P\in C_1\cap C_2$, we define the
  intersection multiplicity $I_P\rr{C_1,C_2}$ to be the intersection multiplicity of
  $\rr{f\circ\phi}\rr{x,y,1}$ and $\rr{g\circ\phi}\rr{x,y,1}$ where $\phi$ is a projective
  transformation with $\phi[0:0:1]=P$.
\end{definition}

\begin{proposition}
  Suppose $C_1$ and $C_2$ are distinct irreducible curves in $\mathbf P^2$ and $P\in C_1\cap C_2$.
  Then
  \begin{align*}
    I_P\rr{C_1,C_2}\geq\mult_P\rr{C_1}\mult_P\rr{C_2}.
  \end{align*}
\end{proposition}

\begin{proposition}
  Suppose $f,g\in k\bb{x,y,z}$ are nonzero, homogeneous, and without common factors such that
  $f\rr{P}=g\rr{P}=0$ for some $P\in\mathbf P^2$. Then
  \begin{enumerate}
    \item $I_P\rr{f,g}\geq 1$;
    \item Suppose $h\in k\bb{x,y,z}$ is homogeneous with $h\rr{P}=0$ then
      \begin{align*}
        I_P\rr{f,gh}=I_P\rr{f,g}+I_P\rr{f,h};
      \end{align*}
    \item Suppose $h\in k\bb{x,y,z}$ is homogeneous with $h\rr{P}\neq 0$ then
      \begin{align*}
        I_P\rr{f,gh}=I_P\rr{f,g}.
      \end{align*}
    \item The intersection of $f\rr{x,y,z}=0$ and $g\rr{x,y,z}=0$ is transverse at $P$ iff
      $I_P\rr{f,g}=1$.
  \end{enumerate}
\end{proposition}

\begin{theorem}[Bezout]
  Suppose $f,g\in k\bb{x,y,z}$ are homogeneous and without common factors. Then the solutions
  of
  \begin{align*}
    f\rr{x,y,z}&=0\\
    g\rr{x,y,z}&=0
  \end{align*}
  in $\mathbf P^2$ are given by $\rr{\deg f}\rr{\deg g}$ points counted with multiplicity.
\end{theorem}

\begin{proposition}
  Let $f\in k\bb{x,y,z}$ be homogeneous. If the system
  \begin{align*}
    \frac{\partial f}{\partial x}=\frac{\partial f}{\partial y}=\frac{\partial f}{\partial z}=0
  \end{align*}
  as no solutions in $\mathbf P^2$ then $f$ is irreducible.
\end{proposition}

\begin{proposition}
  Suppose $C$ is an irreducible curve of degree $d\geq 2$ and $P,Q\in C$ are distinct points.
  Then
  \begin{align*}
    \mult_P\rr{C}+\mult_Q\rr{C} \leq d.
  \end{align*}
\end{proposition}

\begin{proposition}
  Let $C\subseteq\mathbf P^2$ be an irreducible curve of degree 4. Then it has at most three
  singular points.
\end{proposition}

\end{document}
