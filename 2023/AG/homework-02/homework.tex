\documentclass{article}
\usepackage{homework-preamble}

\begin{document}
\title{Algebraic Geometry: Homework 2}
\author{Franz Miltz (UUN: S1971811)}
\date{15 February 2023}
\maketitle

\section*{Exercise 1}

Consider the irreducible polynomial $f(x,y,z)=x^3-y^2z-yz^2$ over $\F_2$ and
the cubic $C:f(x,y,z)=0$.

\begin{claim*}[1]
  There are exactly three points on the cubic: $\bb{0:0:1}$, $\bb{0:1:0}$, and $\bb{0:1:1}$.
  \begin{proof}
    We check all the seven elements of $\mathbb P^2$ over $\F_2$:
    \begin{enumerate}
      \item $\bb{0:0:1}$: $0-0\times 1-0\times 1\equiv 0 \mod 2$;
      \item $\bb{0:1:0}$: $0-1\times 0-1\times 0\equiv 0 \mod 2$;
      \item $\bb{0:1:1}$: $0-1\times 1-1\times 1\equiv 0 \mod 2$;
      \item $\bb{1:0:0}$: $1-0\times 0-0\times 0\not\equiv 0 \mod 2$;
      \item $\bb{1:1:0}$: $1-1\times 0-1\times 0\not\equiv 0 \mod 2$;
      \item $\bb{1:0:1}$: $1-0\times 1-0\times 1\not\equiv 0 \mod 2$;
      \item $\bb{1:1:1}$: $1-1\times 1-1\times 1\not\equiv 0 \mod 2$.
    \end{enumerate}
  \end{proof}
\end{claim*}

We denote the points on the cubic as follows:
\begin{align*}
  O=\bb{0:1:0}, \hs P=\bb{0:0:1}, \hs Q=\bb{0:1:1}.
\end{align*}

\begin{claim*}[2]
  The cubic $C$ is smooth.
  \begin{proof}
    We have the gradient
    \begin{align*}
      \nabla C = (3x^2, -2yz - z^2, -y^2 -2yz).
    \end{align*}
    We calculate
    \begin{align*}
      \nabla C\rr{O} = (0,0,1), \hs
      \nabla C\rr{P} = (0,1,0), \hs
      \nabla C\rr{Q} = (0,1,1).
    \end{align*}
    Thus the gradient does not vanish at any point on the cubic.
  \end{proof}
\end{claim*}

\begin{claim*}[3]
  The group operation acts as follows on non-identity elements:
  \begin{align}\label{eq:group_law}
    P+P=Q, \hs P+Q=O, \hs Q+Q=P.
  \end{align}
  \begin{proof}
    We note that the group of $C$ must be isomorphic to $C_3=\Z/3\Z$. In particular,
    we have $1+1=2$, $2+2=1$, and $1+2=0$ in $C_3$. Thus there are two possible
    isomorphisms the group law could be determined by, but both of them agree.
    We conclude that the group law must behave as in (\ref{eq:group_law}).
  \end{proof}
\end{claim*}

\section*{Exercise 2}

Consider the polynomial $f(x,y,z)=x^3-y^2z$ and the cubic $C:f(x,y,z)=0$ over the
complex numbers $\mathbb C$. Denote the set of smooth points on $C$ by $C^{sm}$.

\begin{claim*}[1]
  The map $\phi:\mathbb C\to C$ given by $t\mapsto\bb{t:1:t^3}$ is injective and its image is
  $C^{sm}$.
  \begin{proof}
    Let $s,t\in\mathbb C$ and assume $\phi(s)=\phi(t)$. Then there exists a $\lambda\in\mathbb C$
    such that $(t,1,t^3)=(\lambda s,\lambda,\lambda s^3)$. In particular, this implies
    $\lambda=1$, i.e. $s=t$. We have thus proven injectivity.

    Further, we calculate
    \begin{align*}
      \nabla C(x,y,z)=(3x^2, -2yz, -y^2).
    \end{align*}
    We note that, on $C$, $y=0$ implies $x=0$ and thus $\nabla C=(0,0,0)$. Thus every smooth
    point has $y\neq 0$. Further, every point with $y\neq 0$ is smooth as the gradient
    does not vanish. This shows $\im\phi\subseteq C^{sm}$.

    Finally, consider a point $\bb{x:y:z}\in C^{sm}$ and assume without loss of generality
    $y=1$. Now $f(x,1,z)=x^3-z$ so $z=x^3$. Therefore $\phi(x)=\bb{x:y:z}$. We have shown
    $C^{sm}\subseteq\im\phi$, as required.
  \end{proof}
\end{claim*}

For $P,Q\in C^{sm}$, define $L(P,Q)$ to be the tangent at $P$ if $P=Q$ or the unique line
passing through $P$ and $Q$ otherwise.

\begin{claim*}[2]
  For all $P,Q\in C^{sm}$, $L(P,Q)\cap C \subseteq C^{sm}$.
  \begin{proof}
    Let $P=\phi(s)$ and $Q=\phi(t)$, for some $s,t\in\mathbb C$.

    Assume $s\neq t$. By \emph{Proposition 1.33}, the line $L(P,Q)$ is given by the equation
    \begin{align*}
      (t^3-s^3)x + (s^3t-st^3)y + (s-t)z = 0.
    \end{align*}
    It is straightforward to verify that $L(P,Q)$ intersects $C$ at $[-(s+t) : 1 : -\rr{s+t}^3]$,
    which is clearly smooth.

    Now assme $s=t$. Then the tangent line $L(P,Q)$ is given by
    \begin{align*}
      3t^2 x - 2t^3 y - z.
    \end{align*}
    Once again, we have the intersection point $[-2t:1:\rr{-2t}^3]$, which is also smooth.
  \end{proof}
\end{claim*}

Let $O=[0:1:0]$ and define the group law on $C^{sm}$ as follows: For $P,Q\in C^{sm}$,
\begin{itemize}
  \item denote the intersection points of $L(P,Q)$ with $C^{sm}$ by $P,Q,R$, counted with multiplicity;
  \item denote the intersection points of $L(R,O)$ with $C^{sm}$ by $R,O,P+Q$, counted with multiplicity.
\end{itemize}

\begin{claim*}[3]
  The map $\phi:\mathbb C\to C^{sm}$ is an isomorphism of groups.
  \begin{proof}
    Let $P=\phi(s)$ and $Q=\phi(t)$, for some $s,t\in\C$. Once again, we differentiate
    between cases.

    Assume $s\neq t$. Then we previously determined $R=[-(s+t) : 1 : -\rr{s+t}^3]$. Further,
    $C$ is a Weierstrass cubic and it was shown in a lecture that the intersection of $L(R,O)$
    with $C$ amounts to mirroring along the y-axis. Thus $P+Q=[s+t : 1 : \rr{s+t}^3]$,
    as required.

    Now assume $s=t$. Then $R=[-2t:1:\rr{-2t}^3]$ so once again,
    \begin{align*}
      P+Q=[2t:1:\rr{2t}^3]=[s+t : 1 : \rr{s+t}^3].
    \end{align*}
  \end{proof}
\end{claim*}

\section*{Exercise 3}

Let $f(x,y,z)=x^3 + y^3 + z^3 - 5xyz$ and the cubic $C:f(x,y,z)=0$ over $\mathbb C$.

\begin{claim*}[1]
  The cubic $C$ is smooth.
  \begin{proof}
    We have the gradient
    \begin{align*}
      \nabla C(x,y,z) = (3x^2-5yz,3y^2-5xz,3z^2-5xy).
    \end{align*}
    Note that if any one of $x,y,z$ is zero and the gradient vanishes, then
    $x=y=z=0$. We may thus assume that all of $x,y,z$ are non-zero. Moreover,
    we assume witout loss of generality that $z=1$. From $\nabla C(x,y,1)=0$ we
    obtain the equations
    \begin{align}
      \label{eq:1}
      3x^2 - 5y &= 0\\
      3y^2 - 5x &= 0\\
      \label{eq:3}
      3 - 5xy &= 0
    \end{align}
    From (\ref{eq:3}) we have $y=3/(5x)$. Subsituting into (\ref{eq:1}) we find
    $3x^2-3x=0$. This has solutions $x=0$ and $x=\frac{-1\pm i\sqrt{3}}{2}$. It is
    straightforward to verify that none of these yield a solution to the whole system.
    Thus the gradient never vanishes and $C$ is smooth.
  \end{proof}
\end{claim*}

Let $O=[0:1:-1]$, $P=[1:2:1]$, and $Q=[-1:0:1]$.

\begin{claim*}[2]
  By the group law on $C$ with identity $O$, we have $-P=[1:1:2]$, $-Q=[1:-1:0]$,
  and $P+Q=[1:1:2]$.
  \begin{proof}
    We begin by showing that $O$ is an inflection point. This follows because
    \begin{align*}
      \text{Hess}(f) = -150x^3 - 150y^3 - 150z^3 - 34xyz.
    \end{align*}
    In particular, $\text{Hess}(f)(O)=0$. Noting that $C$ is smooth, we conclude that $O$
    is an inflection point by the \emph{Hessian criterion}. Thus $L(O,O)\cap C={O}$.
    We use this to compute $-P$ and $-Q$ as it implies $R=O$ in the computation of the
    group law.

    Consider $-P$. By definition, $P+(-P)=O$. In particular, $L(P,-P)\cap C=\cc{P,-P, O}$.
    We have the line $L(P,O):3x-y-z=0$ which intersects $C$ in $-P=[1:1:2]$.

    Consider $-Q$. Once again, we must have $L(Q,-Q)=\cc{Q,-Q,O}$. We have the line
    $L(Q,O):x+y+z=0$ which intersects $C$ in $-Q=[1:-1:0]$.

    Finally, consider $P+Q$. We have the line $L(P,Q):x-y+z=0$ which intersects
    $C$ only in $P$ and $Q$. However, calculating the tangent $L(P,P):x-y+z=0$
    we find that $P$ has intersection multiplicity 2 and thus $R=P$.
    We have considered $L(P,O)$ before and conclude that $P+Q=-P$.
  \end{proof}
\end{claim*}

\begin{claim*}[3]
  Let $n_P$ and $n_Q$ be the order of $P$ and $Q$. Then $n_P=6$ and $n_Q=3$.
  \begin{proof}
    We calculate $Q+Q$. We have the line $L(Q,Q):3x+5y+3z=0$ which intersects $C$
    only in $Q$ itself. Thus $Q+Q=-Q$. Immediately $3Q=O$ and $n_Q=3$.

    Now we consider $P+P$. We have the line $L(P,P):x-y+z=0$ which intersects $C$ at
    $Q$. By previous considerations, we must have $P+P=-Q$. By basic group theory,
    $Q$ and $-Q$ must have the same order. As $2P\neq O$ and $6P=3(-Q)=O$ we must have $n_P=6$.
  \end{proof}
\end{claim*}

\end{document}
