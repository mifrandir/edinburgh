\documentclass{article}
\usepackage{homework-preamble}

\begin{document}
\title{Algebraic Geometry: Homework 2}
\author{Franz Miltz (UUN: S1971811)}
\date{15 February 2023}
\maketitle

\section*{Exercise 1}

Consider the irreducible polynomial $f(x,y,z)=x^3-y^2z-yz^2$ over $\F_2$ and
the cubic $C:f(x,y,z)=0$.

\begin{claim*}[1]
  There are exactly three points on the cubic: $\bb{0:0:1}$, $\bb{0:1:0}$, and $\bb{0:1:1}$.
  \begin{proof}
    We check all the seven elements of $\mathbb P^2$ over $\F_2$:
    \begin{enumerate}
      \item $\bb{0:0:1}$: $0-0\times 1-0\times 1\equiv 0 \mod 2$;
      \item $\bb{0:1:0}$: $0-1\times 0-1\times 0\equiv 0 \mod 2$;
      \item $\bb{0:1:1}$: $0-1\times 1-1\times 1\equiv 0 \mod 2$;
      \item $\bb{1:0:0}$: $1-0\times 0-0\times 0\not\equiv 0 \mod 2$;
      \item $\bb{1:1:0}$: $1-1\times 0-1\times 0\not\equiv 0 \mod 2$;
      \item $\bb{1:0:1}$: $1-0\times 1-0\times 1\not\equiv 0 \mod 2$;
      \item $\bb{1:1:1}$: $1-1\times 1-1\times 1\not\equiv 0 \mod 2$.
    \end{enumerate}
  \end{proof}
\end{claim*}

We denote the points on the cubic as follows:
\begin{align*}
  O=\bb{0:1:0}, \hs P=\bb{0:0:1}, \hs Q=\bb{0:1:1}.
\end{align*}

\begin{claim*}[2]
  The cubic $C$ is smooth.
  \begin{proof}
    We have the gradient
    \begin{align*}
      \nabla C = (3x^2, -2yz - z^2, -y^2 -2yz).
    \end{align*}
    We calculate
    \begin{align*}
      \nabla C\rr{O} = (0,0,1), \hs
      \nabla C\rr{P} = (0,1,0), \hs
      \nabla C\rr{Q} = (0,1,1).
    \end{align*}
    Thus the gradient does not vanish at any point on the cubic.
  \end{proof}
\end{claim*}

\begin{claim*}[3]
  The group operation acts as follows on non-identity elements:
  \begin{align}\label{eq:group_law}
    P+P=Q, \hs P+Q=O, \hs Q+Q=P.
  \end{align}
  \begin{proof}
    We note that the group of $C$ must be isomorphic to $C_3=\Z/3\Z$. In particular,
    we have $1+1=2$, $2+2=1$, and $1+2=0$ in $C_3$. Thus there are two possible
    isomorphisms the group law could be determined by, but both of them agree.
    We conclude that the group law must behave as in (\ref{eq:group_law}).
  \end{proof}
\end{claim*}

\section*{Exercise 2}

\section*{Exercise 3}


\end{document}
