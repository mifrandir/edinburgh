\documentclass{article}
\usepackage{notes-preamble}
\usepackage{enumitem}
\begin{document}
\mkthmstwounified
\title{Fourier Analysis (SEM8)}
\author{Franz Miltz}
\maketitle
\tableofcontents
\pagebreak

\section{Fourier series}

\begin{definition}[Fourier coefficients]
  The Fourier coefficients of $f\in\mathcal L^1\rr{\mathbb T}$ are given by
  \begin{align*}
    a_k = 2\int_{\mathbb T} f\rr{x}\cos\rr{2\pi kx}dx,\hs
    b_k = 2\int_{\mathbb T} f\rr{x}\sin\rr{2\pi kx}dx.
  \end{align*}
\end{definition}

\begin{definition}[Fourier series]
  For $f\in\mathcal L^1\rr{\mathbb T}$, the Fourier series of $f$ at $x\in\R$ is
  given by the formal series
  \begin{align*}
    \sum_{k=0}^\infty \rr{a_k\cos\rr{2\pi kx}+b_k\sin\rr{2\pi kx}}.
  \end{align*}
\end{definition}

\begin{definition}[Partial sums]
  For $f\in\mathcal L^1\rr{\mathbb T}$, the $n$th partial sum of the Fourier series
  of $f$ at $x\in\R$ is given by
  \begin{align*}
    S_N f\rr{x} = \sum_{k=0}^N \rr{a_k\cos\rr{2\pi k x} + b_k\sin\rr{2\pi k x}}.
  \end{align*}
\end{definition}

\begin{theorem}[Katznelson]
  For any set of measure zero $E\subseteq\br{0,1}$, there exists an $f\in C\rr{\mathbb T}$
  whose Fourier series diverges at every point in $E$.
\end{theorem}

\begin{theorem}[Carleson]
  For $f\in C\rr{\mathbb T}$, there exists a set $E\subseteq\br{0,1}$ of measure zero
  such that, for all $x\in \br{0,1}\setminus E$,
  \begin{align*}
    \sum_{k=0}^\infty \rr{a_k\cos\rr{2\pi k x} + b_k\sin\rr{2\pi k x}} = f\rr{x}.
  \end{align*}
\end{theorem}

\begin{definition}[Cesaro means]
  For $f\in\mathcal C\rr{\mathbb T}$, the $n$th Cesaro mean of $f$ at $x$ is
  given by
  \begin{align*}
    \sigma_n f\rr{x} = \frac{1}{n+1} \sum_{k=0}^n S_n f\rr{x}.
  \end{align*}
\end{definition}

\begin{theorem}
  For $f\in C\rr{\mathbb T}$, the sequence of Cesaro means $\rr{\sigma_n f}_{n\geq 1}$
  converges uniformly to $f$ on $\br{0,1}$.
\end{theorem}

\end{document}
