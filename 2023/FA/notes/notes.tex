\documentclass{article}
\usepackage{notes-preamble}
\usepackage{enumitem}
\begin{document}
\mkthmstwounified
\title{Fourier Analysis (SEM8)}
\author{Franz Miltz}
\maketitle
\tableofcontents
\pagebreak

\section{Fourier series}

\begin{definition}
  The complex exponential function $\exp : \C\to\C$ is defined by
  \begin{align*}
    \exp x = \sum_{k=0}^{ \infty } \frac{z^k}{k!}.
  \end{align*}
\end{definition}

\begin{definition}[Fourier coefficients]
  The Fourier coefficients of $f\in\mathcal L^1\rr{\mathbb T}$ are given by
  \begin{align}\label{eq:fourier_coefficients}
    \hat f\rr{k} = \int_{\mathbb T} f\rr{x} e^{-2\pi i k x} dx.
  \end{align}
\end{definition}

\begin{definition}[Fourier series]
  For $f\in\mathcal L^1\rr{\mathbb T}$, the Fourier series of $f$ at $x\in\R$ is
  given by the formal expression
  \begin{align*}
    \sum_{k=0}^\infty \hat f\rr{k} e^{2\pi i k x}.
  \end{align*}
\end{definition}

\begin{definition}[Partial sums]
  For $f\in\mathcal L^1\rr{\mathbb T}$, the $n$th partial sum of the Fourier series
  of $f$ at $x\in\R$ is given by
  \begin{align*}
    S_N f\rr{x} = \sum_{k=0}^N \hat f\rr{k}e^{2\pi i k x}.
  \end{align*}
\end{definition}

\begin{theorem}[Katznelson]
  For any set of measure zero $E\subseteq\br{0,1}$, there exists an $f\in C\rr{\mathbb T}$
  whose Fourier series diverges at every point in $E$.
\end{theorem}

\begin{theorem}[Carleson]
  For $f\in C\rr{\mathbb T}$, there exists a set $E\subseteq\br{0,1}$ of measure zero
  such that, for all $x\in \br{0,1}\setminus E$,
  \begin{align*}
    \sum_{k=0}^\infty \hat f\rr{k}e^{2\pi i k x} = f\rr{x}.
  \end{align*}
\end{theorem}

\begin{definition}[Cesaro means]
  For $f\in\mathcal C\rr{\mathbb T}$, the $n$th Cesaro mean of $f$ at $x$ is
  given by
  \begin{align*}
    \sigma_n f\rr{x} = \frac{1}{n+1} \sum_{k=0}^n S_n f\rr{x}.
  \end{align*}
\end{definition}

\begin{theorem}
  For $f\in C\rr{\mathbb T}$, the sequence of Cesaro means $\rr{\sigma_n f}_{n\geq 1}$
  converges uniformly to $f$ on $\br{0,1}$.
\end{theorem}

\section{$L^p$ space}\label{sec:lp_space}

\begin{definition}[Trigonometric polynomials]
  A trigonometric polynomial is a function $p:\R\to\C$ of the form
  \begin{align*}
    p\rr{x} = \sum_{k=-N}^N c_k e^{-2\pi i k x}
  \end{align*}
  for some $c_{-N},\ldots,c_N\in\C$.

  Let $\mathcal P\rr{\mathbb T}\subseteq\mathcal L^1\rr{\mathbb T}$ denote the vector space
  of all trigonometric polynomials.
\end{definition}

\begin{lemma}
  Let $p\rr{x}=\sum_{k=-N}^{ N } c_k e^{2\pi i k x}$ be a trigonometric polynomial.
  Then
  \begin{enumerate}
    \item the Fourier coeffiicents of $p$ satisfy
      \begin{align*}
        \hat p\rr{j}=\begin{cases}
          c_j &\text{if }\vv{j}\leq N,\\
          0 & \text{otherwise;}
        \end{cases}
      \end{align*}
    \item \begin{align*}
        \sum_{k\in\Z}\vv{\hat p\rr{k}}^2 = \int_{\mathbb T}\vv{p\rr{x}}^2 dx.
      \end{align*}
  \end{enumerate}
\end{lemma}

\begin{definition}
  Let $\mathcal L^p\rr{\mathbb T}$ denote the space of functions
  $f:\mathbb T\to\mathbb C$ such that $x\mapsto\vv{f\rr{x}}^p$ is integrable.
\end{definition}

\begin{lemma}
  $\mathcal P\rr{\mathbb T}\subset C\rr{\mathbb T}\subset \mathcal L^2\rr{\mathbb T}\subset\mathcal L^1\rr{\mathbb T}$.
\end{lemma}

\begin{definition}
  The $L^p$-norm $\vabs{-}_{\mathcal L^p}:\mathcal L^p\to\R$ is given by
  \begin{align*}
    f \mapsto \rr{\int_{\mathbb T} \vv{f\rr{x}}^p dx}^{1/p}.
  \end{align*}
\end{definition}

\begin{theorem}
  The $L^2$-norm is a norm for $\mathcal P\rr{\mathbb T}$.
\end{theorem}

\begin{theorem}
  The $L^2$-norm is not a norm for $\mathcal L^2\rr{\mathbb T}$.
\end{theorem}

\begin{definition}
  Define the space
  \begin{align*}
    L^p = \mathcal L^p / \cc{f\in\mathcal L^p : \vabs{f}_{\mathcal L^p} = 0}.
  \end{align*}
\end{definition}

\begin{theorem}
  The $L^p$-norm defines a norm for $L^p\rr{\mathbb T}$.
\end{theorem}

\begin{definition}
  A function $f:\mathbb T\to\mathbb C$ is essentially bounded if there exists a constant
  $M>0$ and a set of measure zero $E\subseteq\mathbb T$ such that, for all
  $x\in\mathbb T\setminus E$, $\vv{f\rr{x}}\leq M$.
\end{definition}

\end{document}
