\documentclass{article}
\usepackage{notes-preamble}
\usepackage{enumitem}
\mkthms
\title{$\F_3$-Caf\'e: The category of measurable spaces is not cartesian closed}
\author{Franz Miltz}

\begin{document}
\maketitle

\section{Measurable spaces}

\begin{definition}[Measurable space]
	A subset $\Sigma\subseteq\mathcal{P}(X)$ is called a \sigma-algebra over a non-empty set $X$
	iff
	\begin{enumerate}
		\item $\emptyset\in\Sigma$,
		\item for every sequence $(A_n)_{n\in\N}$ of sets $A\in\Sigma$, $\bigcup_{n\in\N}A_n\in\Sigma$, and
		\item if $A\in\Sigma$ then $A^c\in\Sigma$.
	\end{enumerate}
	The pair $(X,\Sigma)$ is called a measurable space. A subset $A\subseteq X$ is called measurable
	in $(X,\Sigma)$ iff $A\in\Sigma$.
\end{definition}

\begin{definition}[Measurable functions]
	Let $(X, \Sigma_X),(Y,\Sigma_Y)$ be measurable spaces. Then a function $f:X\to Y$ is called measurable
	iff
	\begin{align}
		\label{measurable_cond}
		\inv f(A)\in \Sigma_X \text{ for all }A\in\Sigma_Y.
	\end{align}
\end{definition}

\begin{proposition}[Category of measurable spaces]
	Let $(X,\Sigma_X),(Y,\Sigma_Y),(Z,\Sigma_Z)$ be measurable spaces. Then
	\begin{enumerate}
		\item The identity $1_X:X\to X$ is measurable, and
		\item if $f:X\to Y$, $g:Y\to Z$ are measurable, then $f\circ g:X\to Z$ is measurable.
	\end{enumerate}
	In particular, measurable spaces and measurable functions form a categroy, denoted $\cat{Meas}$.
	\begin{proof}
		Exercise.
	\end{proof}
\end{proposition}

\begin{definition}[Generated \sigma-algebras]
	Let $S\subseteq\mathcal{P}(X)$ be a set of subsets of a non-empty set $X$. Let $(\Sigma_i)_{i\in I}$
	be the family of \sigma-algebras such that $S\subseteq\Sigma_i\subseteq\mathcal{P}(X)$.
	We define the
	\sigma-algebra generated by $S$ to be
	\begin{align*}
		\sigma(S):=\bigcap_{i\in I}\Sigma_i.
	\end{align*}
\end{definition}

\begin{definition}[Product of \sigma-algebras]
	Let $\Sigma_X,\Sigma_Y$ be \sigma-algebras. Then their product $\Sigma_X\times\Sigma_Y$ is given by
	\begin{align}
		\label{prodalgebra}
		\Sigma_X\times\Sigma_Y=\sigma\left(A\times B\subseteq X\times Y : A\in\Sigma_X,B\in\Sigma_Y\right),
	\end{align}
	i.e. \sigma-algebra generated by all rectangles in $X\times Y$.
\end{definition}

\begin{proposition}[Product of measurable spaces]
	$\cat{Meas}$ has all finite products. In particular, for objects $(X,\Sigma_X),(Y,\Sigma_Y)$,
	their product is
	\begin{align*}
		(X,\Sigma_X)\times(Y,\Sigma_Y)=(X\times Y,\Sigma_X\times\Sigma_Y)
	\end{align*}
	where $X\times Y$ is the cartesian product and $\Sigma_X\times\Sigma_Y$ is the product \sigma-algebra
	as in (\ref{prodalgebra}).
	\begin{proof}
		Exercise.
	\end{proof}
\end{proposition}


\section{The Borel hiearchy}

\begin{definition}[Von Neumann ordinals]
	A set $S$ is an ordinal iff $S$ is strictly well-ordered with respect to membership and every
	element of $S$ is also a subset of $S$.

	Every natural number is an ordinal, e.g. $4=\{0,1,2,3\}$. We denote by $\omega=\{0,1,2,...\}$ the first countably
	infinite ordinal and by $\omega_1=\{0,1,...,\omega,\omega+1,...\}$ the first uncountably infinite ordinal.
\end{definition}

\begin{definition}[The Borel hiearchy]
	The (boldface) Borel hiearchy on a space $X$ consists of classes
	\begin{align*}
		\Sigma_\alpha^0,\Pi_\alpha^0,\Delta_\alpha^0\subseteq\mathcal{P}(X)
	\end{align*}
	for every ordinal $1\leq\alpha<\omega_1$ such that
	\begin{enumerate}
		\item $S\in\Sigma_1^0$ iff $S$ is open in $X$,
		\item $S\in\Pi_\alpha^1$ iff $S^c\in\Sigma_\alpha^1$,
		\item $S\in\Sigma_\alpha^0$ for $\alpha>1$ iff there exists a sequence $(S_i)_{i\in I}$ such that
		      $A_i\in\Pi_{\alpha_i}^0$ for some $\alpha_i<\alpha$, for all $i\in I$, and $A=\bigcup_{i\in I} A_i$, and
		\item $\Delta_\alpha^0=\Sigma_\alpha^0\cap\Pi^0_\alpha$.
	\end{enumerate}
\end{definition}

\begin{definition}
	The Borel algebra $\Sigma_\R$ is the \sigma-algebra generated by open subsets of $\R$. I.e.
	\begin{align*}
		\Sigma_\R = \bigcup_{1\leq\alpha<\omega_1}\Delta_\alpha^0.
	\end{align*}
\end{definition}

\begin{definition}[Rank]
	Let $A\in\Sigma_\R$. Then the rank of $A$ is defined to be the following ordinal:
	\begin{align*}
		\rank A = \min\{1\leq\alpha<\omega_1:A\in\Delta_\alpha^0\}.
	\end{align*}
\end{definition}

\begin{proposition}
	\label{borelcap}
	In the space $\R$, for all $1\leq\alpha<\omega_1$, there exists a set $A\in\Sigma_\R$
	such that $\rank A=\alpha$.
	\begin{proof}
		Result from descriptive set theory. Ohad tells me this is not too difficult.
	\end{proof}
\end{proposition}

\section{Aumann's theorem}

\begin{theorem}[Aumann, 1961]
	The category $\cat{Meas}$ is not cartesian closed: there is no measurable space of functions $\R\to\R$.
	Specifically, the evaluation function
	\begin{align*}
		\e : \cat{Meas}(\R,\R) \times \R \to \R
	\end{align*}
	is never measurable in
	\begin{align*}
		(\cat{Meas}(\R,\R)\times\R,\Sigma_{\R^\R}\times \Sigma_\R) \to (\R, \Sigma_\R).
	\end{align*}
	\begin{proof}
		We may assume $\Sigma_{\R^\R}=\mathcal{P}(\cat{Meas}(\R,\R))$ as any restriction will make
		(\ref{measurable_cond}) more strict.
		Now assume $\e$ is measurable. Then define
		\begin{align*}
			\alpha := \sup\{\rank(\inv\e([p,q])):p,q\in\Q\}
		\end{align*}
		and observer $\alpha<\omega_1$.
		Using \ref{borelcap}, we fix $A\in \Sigma_\R$ such that $\rank A>\alpha$.
		Consider the characteristic function $\chi_A:\R\to\R$. Now we note
		$\e(\chi_A,A)=\{1\}$, so $\lra{f,A}\in\inv\e(\{1\})$. Further, let
		\begin{align*}
			\lra{f,-}:\R & \to\cat{Meas}(\R,\R)\times\R, \\
			x            & \mapsto \lra{f,x}.
		\end{align*}
		Then $\inv{\lra{f,-}}(\inv\e(\{1\}))=A$. So, in particular,
		\begin{align*}
			\rank A = \rank(\inv{\lra{f,-}}(\inv\e(\{1\}))).
		\end{align*}
		Now we note $f\in\cat{Meas}(\R,\R)$ and $A\in \Sigma_\R$, so $\{f\}\times A\in \Sigma_{\R^\R\times\R}$.
		Specifically, $\{f\}\times A\subseteq\inv\e(\{1\})\in \Sigma_{\R^\R\times\R}$. Therefore
		we obtain the contradiction
		\begin{align*}
			\alpha<\rank A \leq \rank(\inv\e(\{1\}))=\rank(\inv\e([1,1]))\leq \alpha.
		\end{align*}

	\end{proof}
\end{theorem}

\end{document}