\documentclass{article}
\usepackage{notes-preamble}
\usepackage{enumitem}
\begin{document}
\mkthmstwounified
\title{Fourier Analysis (SEM8)}
\author{Franz Miltz}
\maketitle
\tableofcontents
\pagebreak

\section{Fourier series}

\begin{definition}
  The complex exponential function $\exp : \C\to\C$ is defined by
  \begin{align*}
    \exp x = \sum_{k=0}^{ \infty } \frac{z^k}{k!}.
  \end{align*}
\end{definition}

\begin{definition}[Fourier coefficients]
  The Fourier coefficients of $f\in\mathcal L^1\rr{\mathbb T}$ are given by
  \begin{align}\label{eq:fourier_coefficients}
    \hat f\rr{k} = \int_{\mathbb T} f\rr{x} e^{-2\pi i k x} dx.
  \end{align}
\end{definition}

\begin{definition}[Fourier series]
  For $f\in\mathcal L^1\rr{\mathbb T}$, the Fourier series of $f$ at $x\in\R$ is
  given by the formal expression
  \begin{align*}
    \sum_{k=0}^\infty \hat f\rr{k} e^{2\pi i k x}.
  \end{align*}
\end{definition}

\begin{definition}[Partial sums]
  For $f\in\mathcal L^1\rr{\mathbb T}$, the $n$th partial sum of the Fourier series
  of $f$ at $x\in\R$ is given by
  \begin{align*}
    S_N f\rr{x} = \sum_{k=0}^N \hat f\rr{k}e^{2\pi i k x}.
  \end{align*}
\end{definition}

\begin{theorem}[Katznelson]
  For any set of measure zero $E\subseteq\br{0,1}$, there exists an $f\in C\rr{\mathbb T}$
  whose Fourier series diverges at every point in $E$.
\end{theorem}

\begin{theorem}[Carleson]
  For $f\in C\rr{\mathbb T}$, there exists a set $E\subseteq\br{0,1}$ of measure zero
  such that, for all $x\in \br{0,1}\setminus E$,
  \begin{align*}
    \sum_{k=0}^\infty \hat f\rr{k}e^{2\pi i k x} = f\rr{x}.
  \end{align*}
\end{theorem}

\begin{definition}[Cesaro means]
  For $f\in\mathcal C\rr{\mathbb T}$, the $n$th Cesaro mean of $f$ at $x$ is
  given by
  \begin{align*}
    \sigma_n f\rr{x} = \frac{1}{n+1} \sum_{k=0}^n S_n f\rr{x}.
  \end{align*}
\end{definition}

\begin{theorem}
  For $f\in C\rr{\mathbb T}$, the sequence of Cesaro means $\rr{\sigma_n f}_{n\geq 1}$
  converges uniformly to $f$ on $\br{0,1}$.
\end{theorem}

\section{$L^p$ space}\label{sec:lp_space}

\begin{definition}[Trigonometric polynomials]
  A trigonometric polynomial is a function $p:\R\to\C$ of the form
  \begin{align*}
    p\rr{x} = \sum_{k=-N}^N c_k e^{-2\pi i k x}
  \end{align*}
  for some $c_{-N},\ldots,c_N\in\C$.

  Let $\mathcal P\rr{\mathbb T}\subseteq\mathcal L^1\rr{\mathbb T}$ denote the vector space
  of all trigonometric polynomials.
\end{definition}

\begin{lemma}
  Let $p\rr{x}=\sum_{k=-N}^{ N } c_k e^{2\pi i k x}$ be a trigonometric polynomial.
  Then
  \begin{enumerate}
    \item the Fourier coeffiicents of $p$ satisfy
      \begin{align*}
        \hat p\rr{j}=\begin{cases}
          c_j &\text{if }\vv{j}\leq N,\\
          0 & \text{otherwise;}
        \end{cases}
      \end{align*}
    \item \begin{align*}
        \sum_{k\in\Z}\vv{\hat p\rr{k}}^2 = \int_{\mathbb T}\vv{p\rr{x}}^2 dx.
      \end{align*}
  \end{enumerate}
\end{lemma}

\begin{definition}
  Let $\mathcal L^p\rr{\mathbb T}$ denote the space of functions
  $f:\mathbb T\to\mathbb C$ such that $x\mapsto\vv{f\rr{x}}^p$ is integrable.
\end{definition}

\begin{lemma}
  $\mathcal P\rr{\mathbb T}\subset C\rr{\mathbb T}\subset \mathcal L^2\rr{\mathbb T}\subset\mathcal L^1\rr{\mathbb T}$.
\end{lemma}

\begin{definition}
  The $L^p$-norm $\vabs{-}_{\mathcal L^p}:\mathcal L^p\to\R$ is given by
  \begin{align*}
    f \mapsto \rr{\int_{\mathbb T} \vv{f\rr{x}}^p dx}^{1/p}.
  \end{align*}
\end{definition}

\begin{theorem}
  The $L^2$-norm is a norm for $\mathcal P\rr{\mathbb T}$.
\end{theorem}

\begin{theorem}
  The $L^2$-norm is not a norm for $\mathcal L^2\rr{\mathbb T}$.
\end{theorem}

\begin{definition}
  Define the space
  \begin{align*}
    L^p = \mathcal L^p / \cc{f\in\mathcal L^p : \vabs{f}_{L^p} = 0}.
  \end{align*}
\end{definition}

\begin{theorem}
  The $L^p$-norm defines a norm for $L^p\rr{\mathbb T}$.
\end{theorem}

\begin{definition}
  A function $f:\mathbb T\to\mathbb C$ is essentially bounded if there exists a constant
  $M>0$ and a set of measure zero $E\subseteq\mathbb T$ such that, for all
  $x\in\mathbb T\setminus E$, $\vv{f\rr{x}}\leq M$.
\end{definition}

\section{Basic properties of Fourier coefficients}

\begin{lemma}[Riemann-Lebesgue]
  Let $f\in\mathcal L^1\rr{\mathbb T}$. Then $\hat f\rr{k}\to 0$ as $k\to \pm\infty$.
\end{lemma}

\begin{proposition}[Weierstrass M-test]\label{prop:weierstrass_m_test}
  Suppose $A\subseteq\br{0,1}$ and $F_N\rr{x}=\sum_{k=1}^{ N } f_k\rr{x}$
  where the $f_k:A\to\C$ satisfy
  \begin{align*}
    \sup_{x\in A}\vv{f_k\rr{x}} \leq M_k,\hs \sum_{k=1}^{ \infty } M_k < \infty.
  \end{align*}
  Then
  \begin{enumerate}
    \item $F_N$ converges uniformly on $A$ to some $F:A\to\C$;
    \item if each $f_k$ is continuous at some point $x_0\in A$,
      then $F$ is continuous at $x_0\in A$.
  \end{enumerate}
\end{proposition}

\begin{theorem}
  Suppose $f\in\mathcal L^1\rr{\mathbb T}$ has an absolutely convergent Fourier series. Then the
  Fourier series $\sum_{k\in\Z} \hat f\rr{k}e^{2\pi i k x}$ of $f$ converges uniformly on
  $\br{0,1}$ to a continuous function $g\in C\rr{\mathbb T}$ such that $\hat g\rr{k}=\hat f\rr{k}$,
  for all $k\in\Z$.
  \begin{proof}
    We apply \ref{prop:weierstrass_m_test} to the sequence of functions
    $f_k\rr{x}=\hat f\rr{k}e^{2\pi ikx}$, noting that $\vv{f_k\rr{x}}=\vv{\hat f\rr{k}}$
    for all $x\in\mathbb T$, and hence
    \begin{align*}
      \sup_{x\in\br{0,1}}\vv{f_k\rr{x}}=\vv{\hat f\rr{k}}.
    \end{align*}
    Thus the Fourier series of $f$ converges to some continuous function $g$.

    We now consider the Fourier coefficients $\hat g\rr{k}$ of $g$, given by
    \begin{align*}
      \hat g\rr{k}=\int_{\mathbb T} g\rr{x}e^{-2\pi ikx} dx
      = \int_{\mathbb T} \rr{\sum_{n\in\Z} \hat f\rr{n}e^{2\pi inx}}e^{-2\pi ikx}dx.
    \end{align*}
    Since the series $g(x)$ converges uniformly, it follows that
    \begin{align*}
      \hat g\rr{k}=\sum_{n\in\Z}\hat f\rr{n}\int_{\mathbb T}e^{2\pi i\rr{n-k}x}dx.
    \end{align*}
    Applying the orthogonality relations, we obtain $\hat g\rr{k}=\hat f\rr{k}$.
  \end{proof}
\end{theorem}

\begin{theorem}
  Let $f,g\in\mathcal L^1\rr{\mathbb T}$ be two functions such that, for all $k\in\Z$,
  \begin{align*}
    \hat f\rr{k}=\hat g\rr{k}.
  \end{align*}
  Then at all points of continuity of $f-g$ we have $f\rr{x}=g\rr{x}$.
\end{theorem}

\begin{theorem}
  Suppose $f\in\mathcal L^1\rr{\mathbb T}$ has an absolutely convergent Fourier series.
  Then the Fourier series $\sum_{k\in\Z} \hat f\rr{k}e^{2\pi ikx}$ of $f$ converges uniformly
  on $\br{0,1}$ to a continuous function $g\in C\rr{\mathbb T}$. Furthermore, $g\rr{x}=f\rr{x}$
  whenever $f$ is continuous at $x\in\br{0,1}$.
\end{theorem}

\begin{proposition}
  If $f\in C^n\rr{\mathbb T}$ for some $n\geq 0$, then, for all $k\in\Z$,
  \begin{align*}
    \hat{f^{\rr{n}}}\rr{k}=\rr{2\pi ik}^n \hat f\rr{k}
  \end{align*}
  \begin{proof}
    By induction. Note the base case $n=0$ is vacuous. Assume the result holds for some
    $n\geq 0$ and $f\in C^{n+1}\rr{\mathbb T}$. Letting $g=f^{\rr{n}}$, it follows that
    $g\in C\rr{\mathbb T}$ and, using integration by parts,
    \begin{align*}
      \hat{\rr{g'}} = \int_0^1 g'\rr{x}e^{-2\pi ikx}dx
      = \left. g\rr{x}e^{-2\pi ikx}\right\rvert_0^1 - \int_0^1 g\rr{x}\frac{d}{dx}\rr{e^{-2\pi ikx}}dx.
    \end{align*}
    Due to periodicity,
    \begin{align*}
      \hat{\rr{g'}}\rr{k}= g\rr{1}-g\rr{0}+\rr{2\pi ikx}\int_0^1 g\rr{x}e^{-2\pi ikx}dx=\rr{2\pi ikx}\hat g\rr{k}.
    \end{align*}
    Thus
    \begin{align*}
      \hat{\rr{f^{\rr{n}}}}\rr{k}=\rr{2\pi ik}^n\hat f\rr{k}.
    \end{align*}
  \end{proof}
\end{proposition}

\begin{corollary}
  If $f\in C^n\rr{\mathbb T}$ for some $n\geq 0$, then, for all $k\in\Z\setminus\cc{0}$,
  \begin{align*}
    \vv{\hat f\rr{k}}\leq \vabs{f^{\rr{n}}}_{L^1\rr{\mathbb T}} \frac{1}{\rr{2\pi\vv{k}}^n}.
  \end{align*}
\end{corollary}

\begin{proposition}
  Suppose $\rr{a_k}_{k\in\Z}$ is a sequence of complex numbers satisfying
  \begin{align*}
    \vv{a_k}\leq C\rr{1+\vv{k}}^{-n}
  \end{align*}
  for some $n\geq 3$. Then the series
  \begin{align*}
    g\rr{x}=\sum_{k\in\Z} a_k e^{2\pi ikx}
  \end{align*}
  defines a 1-periodic function which s of the class $C^{\rr{n-2}}\rr{\mathbb T}$. If,
  in addition, $a_k=\hat f\rr{k}$ for some $f\in C\rr{\mathbb T}$, then
  $f\in C^{\rr{n-2}}\rr{\mathbb T}$.
\end{proposition}

\begin{proposition}
  For any $f\in\mathcal L^1\rr{\mathbb T}$ we have
  \begin{align*}
    \min_{p\in\mathcal P_N\rr{\mathbb T}} \int_{\mathbb T}\vv{f\rr{x}-p\rr{x}}^2 dx
    = \int_{\mathbb T}\vv{f\rr{x}-S_Nf\rr{x}}^2dx
    = \int_{\mathbb T}\vv{f\rr{x}}^2dx - \sum_{k=-N}^{ N } \vv{\hat f\rr{k}}^2.
  \end{align*}
\end{proposition}

\begin{corollary}[Bessel]
  For any $f\in\mathcal L^1\rr{\mathbb T}$, we have
  \begin{align*}
    \sum_{k\in\Z}\vv{\hat f\rr{k}}^2 \leq \int_{\mathbb T}\vv{f\rr{x}}^2dx.
  \end{align*}
\end{corollary}

\begin{proposition}
  If $f\in C^1\rr{\mathbb T}$ then the Fourier series of $f$ converges absolutely and hence
  uniformly to $f$.
\end{proposition}

\end{document}
