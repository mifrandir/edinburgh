 %! TEX program = lualatex
\documentclass{article}
\usepackage{notes-preamble}
\mkthmstwounified

\title{Introduction to Quantum Computing (SEM7)}
\author{Franz Miltz}

\begin{document}
\maketitle
\tableofcontents
\pagebreak


\section{Postulates}
\label{sec:postulates}

\subsection{Quantum States}
\label{sec:quantum-states}

\begin{definition}
  \label{def:hilbert-space}
  A Hilbert space is a complex inner product space which is complete with respect to the induced metric.
\end{definition}

\begin{definition}
  \label{def:quantum-state}
  A quantum state is a vector
  \begin{align*}
    \va{ \psi } = \begin{bmatrix}
      \psi_1 \\
      \vdots \\
      \psi_d
    \end{bmatrix} \in \C^d
  \end{align*}
  where $\psi_1,...,\psi_d$ are the probability amplitudes of the $d$ outcomes.
  In particular, the probability of the outcome $1\leq i\leq d$, is $P \rr{ i } = \abs{ \psi_i }^2$.
  For $n$ bits there are $d=2^n$ outcomes.
\end{definition}

\begin{definition}
  \label{def:qubit}
  A qubit is a quantum state $ \av{ \psi } \in \C^2$. We define
  \begin{align*}
    \va{ 0 } = \rr{ 1, 0 }^\top\\
    \va{ 1 } = \rr{ 0, 1 }^\top
  \end{align*}
\end{definition}

\begin{definition}
  \label{def:superposition}
  \begin{align*}
    \va{ + } = \frac{1}{\sqrt{2}} \rr{ \va{ 0 } + \va{ 1 } } \\
    \va{ - } = \frac{1}{\sqrt{2}} \rr{ \va{ 0 } - \va{ 1 } }
  \end{align*}
\end{definition}

\subsection{Quantum operations}
\label{sec:quantum-operations}

\begin{definition}
  \label{def:quantum-operation}
  An operation on a quantum state $ \va{ \psi } \in \mathcal H = \C^d$ is given by
  a unitary transformation $U:\mathcal H \to \mathcal H$, i.e.
  \begin{align*}
    UU^\dagger = U^\dagger U = I_d.
  \end{align*}
\end{definition}


\begin{definition}
  \label{def:phase-flip}
  The phase flip gate $Z$ is given by
  \begin{align*}
    Z = \begin{pmatrix} 1 & 0 \\ 0 & -1 \end{pmatrix}.
  \end{align*}
  It transforms as follows:
  \begin{align*}
    Z \rr{ \alpha \va{ 0 } + \beta \va{ 1 } } = \alpha \va{ 0 } - \beta \va{ 1 }.
  \end{align*}
\end{definition}


\begin{definition}
  \label{def:hadamard}
  The Hadamard gate $H$ is given by
  \begin{align*}
    H = \frac{1}{\sqrt{2}} \begin{pmatrix} 1 & 1 \\ 1 & -1 \end{pmatrix}.
  \end{align*}
  It acts as a change of basis from $ \cc{ \va{ 0 }, \va{ 1 } } $ to $ \cc{ \va{ + }, \va{ - } }$.
\end{definition}


\subsection{Measurement}
\label{sec:measurement}

\begin{definition}
  \label{def:probability-of-outcome}
  Let $\cc{\va{v_i}}$ be a basis of a Hilbert space $\mathcal{H}$ and an associated measurement $\psi$.
  Then the probability of outcome $i$ is
  \begin{align*}
    P\rr{i} = \abs{\ava{v_i}{\psi}}^2.
  \end{align*}
  The quantum state is updated to $\va{v_i}$.
\end{definition}

\end{document}
