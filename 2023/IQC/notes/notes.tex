%! TEX program = lualatex
\documentclass{article}
\usepackage{notes-preamble}
\mkthmstwounified

\title{Introduction to Quantum Computing (SEM7)}
\author{Franz Miltz}

\begin{document}
\maketitle
\tableofcontents
\pagebreak

\section{Mathematical prerequisites}
\label{sec:mathetmatical-prerequisites}

\begin{definition}
  \label{def:hilbert-space}
  A Hilbert space is a complex inner product space which is complete with respect to the induced metric.
\end{definition}

\begin{definition}
  \label{def:unitary-matrix}
  Let $U\in\Mat\rr{n; \C}$. Then define $U^\dagger = \overline{U}^T$. Further, $U$ is normal iff
  $U^\dagger U =UU^\dagger$ and unitary iff $U^\dagger U =UU^\dagger = I$.
\end{definition}

\begin{theorem}[Spectral decomposition]
  Any normal operator $M$ on a vector space $V$ is diagonal with respect to some orthonormal
  basis for $V$.

  Conversely, any diagonalisable operator is normal.
\end{theorem}

\begin{definition}[Tensor product]
  \label{def:tensor-product}
  Let $\mathcal H_A, \mathcal H_B$ be Hilbert spaces with bases $\mathcal B_A$ and
  $\mathcal B_B$, respectively. Then $\mathcal H_A\otimes\mathcal H_B$ is the Hilbert
  space with basis
  \begin{align*}
    B_{A\otimes B} = \cc{a \otimes b : a \in \mathcal B_A, b\in\mathcal B_B}.
  \end{align*}

  Let
  \begin{align*}
    x = \sum_{a\in\mathcal B_A} x_a a\in\mathcal H_A, \hs
    y = \sum_{b\in\mathcal B_B} y_b b\in\mathcal H_B
  \end{align*}
  be vectors. Then define
  \begin{align*}
    x \otimes y = \sum_{a\in\mathcal B_A}\sum_{b\in\mathcal B_B} x_a y_b a \otimes b.
  \end{align*}

  Let $A\in\mathcal L\rr{\mathcal H_A}$ and $B\in\mathcal L\rr{\mathcal H_B}$ be linear
  operators. Then $A\otimes B\in\mathcal L\rr{\mathcal H_A\otimes\mathcal H_B}$ is the
  unique linear operator such that, for all $a\in\mathcal H_A$, $b\in\mathcal H_B$,
  \begin{align*}
    \rr{A\otimes B}\rr{a \otimes b} = A a \otimes B b.
  \end{align*}
  Explicitly,
  \begin{align*}
    A \otimes B = \begin{pmatrix}
      A_{11} B & \cdots & A_{1n}B \\
      \vdots   & \ddots & \vdots  \\
      A_{n1} B & \cdots & A_{nn}B
    \end{pmatrix}.
  \end{align*}
\end{definition}

\section{Postulates}
\label{sec:postulates}

\subsection{State space}
\label{sec:quantum-states}

\begin{definition}
  \label{def:quantum-state}
  Let $\mathcal H$ be a Hilbert space with an finite ordered orthonormal basis
  $\rr{v^1, ..., v^d}$. A quantum state is a unit vector $\va{\psi}\in\mathcal H$.
  We write
  \begin{align*}
    \va{\psi} = \sum_{i\in I} \psi_i v^i
  \end{align*}
  where $\psi_1,...,\psi_d\in\C$ are the probability amplitudes of the $d$ outcomes.
  In particular, the probability of the outcome $1\leq i\leq d$, is $P \rr{ i } = \abs{ \psi_i }^2$.
  For $n$ bits there are $d=2^n$ outcomes.
\end{definition}

\begin{axiom}
  Associated to any isolated physical system is a Hilbert space $\mathcal H$. The state of the system
  is completely described by a state vector $\va{\psi}\in\mathcal H$.
\end{axiom}

\begin{definition}
  \label{def:qubit}
  A qubit is a quantum state $ \va{\psi}\in\C^2$. We define
  \begin{align*}
    \va{ 0 } = \rr{ 1, 0 }^\top \\
    \va{ 1 } = \rr{ 0, 1 }^\top
  \end{align*}
\end{definition}

\begin{definition}
  \label{def:superposition}
  \begin{align*}
    \va{ + } = \frac{1}{\sqrt{2}} \rr{ \va{ 0 } + \va{ 1 } } \\
    \va{ - } = \frac{1}{\sqrt{2}} \rr{ \va{ 0 } - \va{ 1 } }
  \end{align*}
\end{definition}

\subsection{Quantum operations}
\label{sec:quantum-operations}

\begin{axiom}
  \label{def:quantum-operation}
  The evolution of a closed quantum system with associated Hilbert space $\mathcal H$
  is described by a unitary transformation $U:\mathcal H\to\mathcal H$.

  That is, the state $\va{\psi}$ of the system at time $t_1$ is related to the state
  $\va{\psi'}$ of the system at time $t_2$ by a unitary operator $U$ which depends only
  on the times $t_1$ and $t_2$,
  \begin{align*}
    \va{\psi'}=U\va{\psi}.
  \end{align*}
\end{axiom}

\subsection{Measurement}
\label{sec:measurement}

\begin{axiom}
  Quantum measurements are described by a collection $\cc{M_m}$ of measurement operators.
  There are operators acting on the state space of the system being measured. The index $m$
  refers to the measurement outcomes that may occur in the experiment. If the state of the
  quantum system is $\va{\psi}$ immediately before the measurement then the probability
  that result $m$ occurs is given by
  \begin{align*}
    \pr{m}=\av{\psi}M_m^\dagger M_m \va{\psi},
  \end{align*}
  and the state of the system after the measurement is
  \begin{align*}
    \frac{M_m \va{\psi}}{\sqrt{\av{\psi}M_m^\dagger M_m \va{\psi}}}.
  \end{align*}
  The measurement operators satisfy the completeness equation,
  \begin{align*}
    \sum_{m} M_m^\dagger M_m = I.
  \end{align*}
\end{axiom}

\begin{theorem}
  \label{thm:probability-of-outcome}
  Let $\cc{\va{v_i}}$ be a basis of a Hilbert space $\mathcal{H}$ and an associated measurement $\psi$.
  Then the probability of outcome $i$ is
  \begin{align*}
    P\rr{i} = \abs{\ava{v_i}{\psi}}^2.
  \end{align*}
  The quantum state is updated to $\va{v_i}$.
\end{theorem}

\section{Quantum circuits}
\label{sec:circuits}

\subsection{Gates}
\label{sec:gates}

\begin{definition}
  \label{def:single-qubit-gates}
  An $n$ qubit gate is a unitary matrix $U\in\Mat\rr{2^n;\C}$.
\end{definition}

\begin{corollary}
  Every quantum gate is invertible.
\end{corollary}

\subsubsection{Single qubit gates}
\label{sec:single-qubit-gates}

\begin{definition}
  \label{def:identity}
  The identity gate $I$ is the matrix
  \begin{align*}
    I = \begin{pmatrix}
      1 & 0 \\ 0 & 1
    \end{pmatrix}.
  \end{align*}
\end{definition}

\begin{definition}
  \label{def:not}
  The not gate $X$ is given by
  \begin{align*}
    X = \begin{pmatrix}
      0 & 1 \\ 1 & 0
    \end{pmatrix}
  \end{align*}
\end{definition}

\begin{definition}
  \label{def:phase-flip}
  The phase flip gate $Z$ is given by
  \begin{align*}
    Z = \begin{pmatrix} 1 & 0 \\ 0 & -1 \end{pmatrix}.
  \end{align*}
  It transforms as follows:
  \begin{align*}
    Z \rr{\alpha\va{0} + \beta\va{1}} = \alpha\va{0} - \beta\va{1}.
  \end{align*}
\end{definition}


\begin{definition}
  \label{def:hadamard}
  The Hadamard gate $H$ is given by
  \begin{align*}
    H = \frac{1}{\sqrt{2}} \begin{pmatrix} 1 & 1 \\ 1 & -1 \end{pmatrix}.
  \end{align*}
  It acts as a change of basis from $ \cc{ \va{ 0 }, \va{ 1 } } $ to $ \cc{ \va{ + }, \va{ - } }$.
\end{definition}

\begin{theorem}
  \label{thm:infinite-gates}
  There are infinitely many single-qubit quantum gates.
\end{theorem}

\subsubsection{Multiple qubit gates}
\label{sec:multiple-qubit-gates}

\begin{definition}
  \label{def:cnot}
  The \text{CNOT} gate $U_{CN}$ is a two qubit quantum gate defined by
  \begin{align*}
    U_{CN} = \begin{pmatrix}
      1 & 0 & 0 & 0 \\
      0 & 1 & 0 & 0 \\
      0 & 0 & 0 & 1 \\
      0 & 0 & 1 & 0
    \end{pmatrix}
  \end{align*}
\end{definition}

\begin{definition}
  \label{def:controlled-u}
  Let $U=\rr{u_{ij}}$ be a single qubit quantum gate. Then the controlled-$U$ gate is a two qubit
  quantum gate defined by
  \begin{align*}
    U_C = \begin{pmatrix}
      1 & 0 & 0 & 0 \\
      0 & 1 & 0 & 0 \\
      0 & 0 & u_{00} & u_{01} \\
      0 & 0 & u_{10} & u_{11}
    \end{pmatrix}
  \end{align*}
\end{definition}

\begin{definition}
  Let $n\in\N$. Then the Walsh-Hadamard transform is given by the unitary
  \begin{align*}
    H^{\otimes n} = H\otimes \cdots \otimes H
  \end{align*}
  and, for all $\va{x}\in\cc{0,1}^n$,
  \begin{align*}
    H^{\otimes n} \va{x} = \frac{1}{\sqrt{2^n}} \sum_{y\in\cc{0,1}^n} \rr{-1}^{x\cdot y}\va{y}.
  \end{align*}
\end{definition}

\begin{definition}
  Let $f:\cc{0,1}^n\to \cc{0,1}$ be a function. Then the phase-kickback unitary
  $U_f$ is such that, for all $x\in\cc{0,1}^n$,
  \begin{align*}
    U_f \va{x}= \rr{-1}^{f\rr{x}}\va{x}.
  \end{align*}
\end{definition}

\section{Algorithms}

\begin{algorithm}[Deutsch-Jozsa]\label{alg:deutsch-jozsa}
  Given a function $f:\cc{0,1}^n\to\cc{0,1}$ that is either constant or balanced,
  we decide with certainty which is the casae.

  The circuit is given by
  \begin{align}
    \label{eq:deutsch-jozsa}
    H^{\otimes n} U_f H^{\otimes n}
  \end{align}
  and on input $\va{0}^{\otimes n}$ the output is $0^n$ iff $f$ is constant.
\end{algorithm}

\begin{algorithm}[Bernstein-Vazirani]\label{alg:bernstein-vazirani}
  Given a function $f:\cc{0,1}^n\to\cc{0,1}$ that is given by $x\mapsto a\cdot x$ for some
  $a\in\cc{0,1}^n$ we find $a$ with certainty. The circuit is identical to the one in
  (\ref{eq:deutsch-jozsa}).

  On input $\va{0}^{\otimes n}$ the output is $a$.
\end{algorithm}

\end{document}
