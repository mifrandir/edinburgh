%! TEX program = lualatex
\documentclass{article}
\usepackage{notes-preamble}
\mkthmstwounified

\title{Introduction to Quantum Computing (SEM7)}
\author{Franz Miltz}

\begin{document}
\maketitle
\tableofcontents
\pagebreak

\section{Postulates}

\subsection{State space}

\begin{definition}
  Let $\mathcal H$ be a Hilbert space with an finite ordered orthonormal basis
  $\rr{v^1, ..., v^d}$. A quantum state is a unit vector $\va{\psi}\in\mathcal H$.
  We write
  \begin{align*}
    \va{\psi} = \sum_{i\in I} \psi_i v^i
  \end{align*}
  where $\psi_1,...,\psi_d\in\C$ are the probability amplitudes of the $d$ outcomes.
  In particular, the probability of the outcome $1\leq i\leq d$, is $P \rr{ i } = \abs{ \psi_i }^2$.
  For $n$ bits there are $d=2^n$ outcomes.
\end{definition}

\begin{axiom}
  Associated to any isolated physical system is a Hilbert space $\mathcal H$. The state of the system
  is completely described by a state vector $\va{\psi}\in\mathcal H$.
\end{axiom}

\begin{definition}
  \label{def:qubit}
  A qubit is a quantum state $ \va{\psi}\in\C^2$. We define
  \begin{align*}
    \va{ 0 } = \rr{ 1, 0 }^\top \\
    \va{ 1 } = \rr{ 0, 1 }^\top
  \end{align*}
\end{definition}

\begin{definition}
  \label{def:superposition}
  \begin{align*}
    \va{ + } = \frac{1}{\sqrt{2}} \rr{ \va{ 0 } + \va{ 1 } } \\
    \va{ - } = \frac{1}{\sqrt{2}} \rr{ \va{ 0 } - \va{ 1 } }
  \end{align*}
\end{definition}

\subsection{Quantum operations}
\label{sec:quantum-operations}

\begin{axiom}
  \label{def:quantum-operation}
  The evolution of a closed quantum system with associated Hilbert space $\mathcal H$
  is described by a unitary transformation $U:\mathcal H\to\mathcal H$.

  That is, the state $\va{\psi}$ of the system at time $t_1$ is related to the state
  $\va{\psi'}$ of the system at time $t_2$ by a unitary operator $U$ which depends only
  on the times $t_1$ and $t_2$,
  \begin{align*}
    \va{\psi'}=U\va{\psi}.
  \end{align*}
\end{axiom}

\subsection{Measurement}
\label{sec:measurement}

\begin{axiom}
  Quantum measurements are described by a collection $\cc{M_m}$ of measurement operators.
  There are operators acting on the state space of the system being measured. The index $m$
  refers to the measurement outcomes that may occur in the experiment. If the state of the
  quantum system is $\va{\psi}$ immediately before the measurement then the probability
  that result $m$ occurs is given by
  \begin{align*}
    \pr{m}=\av{\psi}M_m^\dagger M_m \va{\psi},
  \end{align*}
  and the state of the system after the measurement is
  \begin{align*}
    \frac{M_m \va{\psi}}{\sqrt{\av{\psi}M_m^\dagger M_m \va{\psi}}}.
  \end{align*}
  The measurement operators satisfy the completeness equation,
  \begin{align*}
    \sum_{m} M_m^\dagger M_m = I.
  \end{align*}
\end{axiom}

\begin{theorem}
  \label{thm:probability-of-outcome}
  Let $\cc{\va{v_i}}$ be a basis of a Hilbert space $\mathcal{H}$ and an associated measurement $\psi$.
  Then the probability of outcome $i$ is
  \begin{align*}
    P\rr{i} = \abs{\ava{v_i}{\psi}}^2.
  \end{align*}
  The quantum state is updated to $\va{v_i}$.
\end{theorem}

\section{Gates}
\label{sec:gates}

\begin{definition}
  \label{def:single-qubit-gates}
  An $n$ qubit gate is a unitary matrix $U\in\Mat\rr{2^n;\C}$.
\end{definition}

\begin{corollary}
  Every quantum gate is invertible.
\end{corollary}

\subsection{Single qubit gates}
\label{sec:single-qubit-gates}

\begin{definition}
  We have the Pauli gates $X,Y,Z$:
  \begin{align*}
    X &=
    \begin{pmatrix}
      0 & 1 \\ 1 & 0
    \end{pmatrix},
      &X\va{0} &= \va{1},
      &X\va{1}&=\va{0},
      &X\va{+}&=\va{+},
      &X\va{-}&=-\va{-};\\
    Y &=
    \begin{pmatrix}
      0 & i \\
      -i & 0
    \end{pmatrix},
      &Y\va{0} &= -i\va{1},
      &Y\va{1} &= i\va{0},
      &Y\va{+} &= i\va{-},
      &Y\va{-} &=-i\va{+};\\
    Z &= \begin{pmatrix} 1 & 0 \\ 0 & -1 \end{pmatrix},
      &Z\va{0} &= \va{0},
      &Z\va{1} &= -\va{1},
      &Z\va{+} &= \va{-},
      &Z\va{-} &= \va{+}.
  \end{align*}
\end{definition}

\begin{lemma}
  $Y = iXZ$
\end{lemma}

\begin{definition}
  \label{def:hadamard}
  The Hadamard gate $H$ is given by
  \begin{align*}
    H &= \frac{1}{\sqrt{2}} \begin{pmatrix} 1 & 1 \\ 1 & -1 \end{pmatrix},
      & H\va{0} &=\va{1},
      & H\va{1} &=\va{0},
      & H\va{+} &=\va{-},
      & H\va{-} &=\va{+}.
  \end{align*}
\end{definition}

\begin{definition}
  Let $\theta\in\br{0,2\pi}$. Then
  \begin{align*}
    \va{\pm_\theta} = \frac{1}{\sqrt 2} \rr{\va{0}\pm e^{i\theta}\va{1}}.
  \end{align*}
\end{definition}


\begin{definition}
  For $\theta\in\br{0,2\pi}$, the rotation gate $R(\theta)$ is
  \begin{align*}
    R(\theta) &= \frac{1}{\sqrt{2}}
    \begin{pmatrix}
      1 & 0 \\
      0 & e^{i\theta}
    \end{pmatrix},&
    R(\theta)\va{0} &=\va{0},&
    R(\theta)\va{1} &=e^{i\theta}\va{1},&
    R(\theta)\va{+_\phi} &=\va{+_{\theta+\phi}},&
    R(\theta)\va{-_\phi} &=\va{-_{\theta+\phi}}.
  \end{align*}

\end{definition}

\subsection{Multiple qubit gates}
\label{sec:multiple-qubit-gates}

\begin{definition}
  \label{def:controlled-u}
  Let $j,k$ be distinct qubits and let $U$ be a single qubit quantum gate.
  Then the controlled-$U$ gate on qubit $k$ with control $j$ is
  given by
  \begin{align*}
    \wedge U_{jk}\va{0}_j\va{\phi}_k &= \va{0}_j\va{\phi}_k,&
    \wedge U_{jk}\va{1}_j\va{\phi}_k &= \va{1}_j U_k\va{\phi}_k.
  \end{align*}
\end{definition}

\begin{lemma}
  We have the following actions:
  \begin{align*}
    \wedge X_{12}\va{00} &= \va{00},
     &\wedge X_{12}\va{01} &= \va{01},\hs
     &\wedge X_{12}\va{10} &= \va{11},
     &\wedge X_{12}\va{11} &= \va{10}\\
     \wedge X_{12}\va{++} &= \va{++},
     &\wedge X_{12}\va{+-} &= \va{--},
     &\wedge X_{12}\va{-+} &= \va{-+},
     &\wedge X_{12}\va{--} &= \va{-+}\\
    \wedge Z_{12}\va{00} &= \va{00},
     \wedge Z_{12}\va{01} &= \va{01},\hs
     \wedge Z_{12}\va{10} &= \va{10},
     \wedge Z_{12}\va{11} &= -\va{11}
   \end{align*}
   \begin{align*}
     \wedge Z_{12}\va{++} = \frac{1}{2}\rr{\va{00}+\va{01}+\va{10}-\va{11}}\\
     \wedge Z_{12}\va{+-} = \frac{1}{2}\rr{\va{00}+\va{01}-\va{10}+\va{11}}\\
     \wedge Z_{12}\va{-+} = \frac{1}{2}\rr{\va{00}-\va{01}+\va{10}+\va{11}}\\
     \wedge Z_{12}\va{--} = \frac{1}{2}\rr{-\va{00}+\va{01}+\va{10}+\va{11}}
  \end{align*}
\end{lemma}

\begin{definition}
  Let $n\in\N$. Then the Walsh-Hadamard transform is given by the unitary
  \begin{align*}
    H^{\otimes n} = H\otimes \cdots \otimes H.
  \end{align*}
\end{definition}

\begin{lemma}
  For all $\va{x}\in\cc{0,1}^n$,
  \begin{align*}
    H^{\otimes n} \va{x} = \frac{1}{\sqrt{2^n}} \sum_{y\in\cc{0,1}^n} \rr{-1}^{x\cdot y}\va{y}.
  \end{align*}
\end{lemma}

\section{Algorithms}

\begin{definition}
  Let $f:\cc{0,1}^n\to \cc{0,1}$ be a function. Then the phase-kickback unitary
  $U_f$ is such that, for all $x\in\cc{0,1}^n$,
  \begin{align*}
    U_f \va{x}= \rr{-1}^{f\rr{x}}\va{x}.
  \end{align*}
\end{definition}

\begin{definition}
  Let $f:\cc{0,1}^n\to\cc{0,1}^n$. The quantum oracle $O_f$ is the unitary
  such that, for all $x,y\in\cc{0,1}^n$,
  \begin{align*}
    O_f \va{xy} = \va{x}\va{y \oplus f(x)}.
  \end{align*}
\end{definition}

\begin{algorithm}
  Deutsch-Jozsa algorithm:
  \begin{itemize}
    \item input: $f:\cc{0,1}^n\to\cc{0,1}$;
    \item promise: $f$ is either constant or balanced;
    \item circuit: $H^{\otimes n} U_f H^{\otimes n}$;
    \item output: $\va{0^n}$ if, and only if, $f$ is constant.
  \end{itemize}
\end{algorithm}

\begin{algorithm}
  Bernstein-Vazirani algorithm:
  \begin{itemize}
    \item input: $f:\cc{0,1}^n\to\cc{0,1}$;
    \item promise: for some $a\in\cc{0,1}^n$, $f(x) = a\cdot x$;
    \item circuit: $H^{\otimes n} U_f H^{\otimes n}$;
    \item output: $a$
  \end{itemize}
\end{algorithm}

\begin{algorithm}
  Grover search:
  \begin{itemize}
    \item input: $f:\cc{0,1}^n\to\cc{0,1}$;
    \item promise: for some $s\in\cc{0,1}^n$, $f(x)=1$ if, and only if,
      $x=s$;
    \item circuit: $H^{\otimes n} U_0 H^{\otimes n} U_f$.
    \item output: after approximately $\pi/4 \sqrt{2^n}$ iterations,
      $\va{s}$.
  \end{itemize}
\end{algorithm}

\begin{algorithm}
  Simon's algorithm:
  \begin{itemize}
    \item input: $f:\cc{0,1}^n\to\cc{0,1}$;
    \item promise: for some $a\in\cc{0,1}^n$, if $f(x) = f(x')$ then
      either $x=x'$ or $x = x' \oplus a$;
    \item circuit:
      $(H^{\otimes n}\otimes I^{\otimes n}) O_f (H^{\otimes n}\otimes I^{\otimes n})$
    \item output: after measuring qubits $n+1$ to $2n$, measuring qubits
      $1$ to $n$ yields a sample $z$ with $az=0$; $n$ queries yield
      independent equations with $p>1/4$; $O(n)$ queries yield solution.
  \end{itemize}
\end{algorithm}

\section{Measurement-based quantum computing}

\subsection{The $J(\theta)$ gate}

\begin{definition}
  Let $\theta\in\br{0,2\pi}$ and let $j\geq 1$. Then the
  \emph{measurement of qubit $j$ at angle $\theta$} is the measurement
  of qubit $j$ in the basis
  \begin{align*}
    M_j^\theta = \cc{\va{\pm_\theta}},\hs
    M_j^Z = \cc{\va{0},\va{1}}
  \end{align*}
  The outcome is denoted by $s_j$.
\end{definition}

\begin{definition}
  Let $\theta\in\br{0,2\pi}$. Define the \emph{Hadamard rotated
  phase gate} $J(\theta)$ by
  \begin{align*}
    J(\theta) = HR(\theta) = \frac{1}{\sqrt{2}}\begin{pmatrix}
      1 & e^{i\theta} \\
      1 & -e^{i\theta}
    \end{pmatrix}
  \end{align*}
\end{definition}

\begin{lemma}
  \begin{align*}
    J(0) = H, \hs J(-\pi) = HZ.
  \end{align*}
\end{lemma}

\begin{theorem}
  The set of gates $\cc{\wedge Z}\cap\cc{J(\theta) : \theta\in\br{0,2\pi}}$
  is universal.
  \begin{proof}
    Any single qubit unitary $U$ may be written as
    $U=J(0)J(\theta_1)J(\theta_2)J(\theta_3)$ for some
    $\theta_1,\theta_2,\theta_3$.
    Universality is achieved with the two qubit gate $\wedge Z$.
  \end{proof}
\end{theorem}


\subsection{Corrections}

\begin{theorem}
  Consider a graph $(G,I,O)$, a flow function $f:O^C\to I^C$, measurement
  outcomes $s_1,\ldots,s_n$, and measurement angles $\phi_1,\ldots,\phi_n$.
  Consider the set $S_Z(i) = \cc{1\leq j\leq n : i\in N_G(f(j)), i\neq j}$.
  Define
  \begin{align*}
    s_X(i) = s_{\inv f(i)}, \hs s_Z(i) = \sum_{j\in S_Z(i)} s_j.
  \end{align*}
  For $i\in O^C$, the corrected measurement angles are given by
  \begin{align*}
    \phi_i' = (-1)^{s_X}\phi_i + \pi s_Z
  \end{align*}
  and output qubits $i\in O$ may be corrected by applying
  $X^{s_X(i)}Z^{s_Z(i)}$.
\end{theorem}

\end{document}
