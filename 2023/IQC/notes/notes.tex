 %! TEX program = lualatex
\documentclass{article}
\usepackage{notes-preamble}
\mkthmstwounified

\title{Introduction to Quantum Computing (SEM7)}
\author{Franz Miltz}

\begin{document}
\maketitle
\tableofcontents
\pagebreak

\section{Mathematical prerequisites}
\label{sec:mathetmatical-prerequisites}

\begin{definition}
	\label{def:hilbert-space}
	A Hilbert space is a complex inner product space which is complete with respect to the induced metric.
\end{definition}

\begin{definition}
	\label{def:unitary-matrix}
	Let $U\in\Mat\rr{n; \C}$. Then define $U^\dagger = \overline{U}^T$. Further, $U$ is unitary iff
	$U^\dagger U =UU^\dagger = I$.
\end{definition}

\begin{definition}[Tensor product]
	\label{def:tensor-product}
	Let $\mathcal H_A, \mathcal H_B$ be Hilbert spaces with bases $\mathcal B_A$ and
	$\mathcal B_B$, respectively. Then $\mathcal H_A\otimes\mathcal H_B$ is the Hilbert
	space with basis
	\begin{align*}
		B_{A\otimes B} = \cc{a \otimes b : a \in \mathcal B_A, b\in\mathcal B_B}.
	\end{align*}

	Let
	\begin{align*}
		x = \sum_{a\in\mathcal B_A} x_a a\in\mathcal H_A, \hs
		y = \sum_{b\in\mathcal B_B} y_b b\in\mathcal H_B
	\end{align*}
	be vectors. Then define
	\begin{align*}
		x \otimes y = \sum_{a\in\mathcal B_A}\sum_{b\in\mathcal B_B} x_a y_b a \otimes b.
	\end{align*}

	Let $A\in\mathcal L\rr{\mathcal H_A}$ and $B\in\mathcal L\rr{\mathcal H_B}$ be linear
	operators. Then $A\otimes B\in\mathcal L\rr{\mathcal H_A\otimes\mathcal H_B}$ is the
	unique linear operator such that, for all $a\in\mathcal H_A$, $b\in\mathcal H_B$,
	\begin{align*}
		\rr{A\otimes B}\rr{a \otimes b} = A a \otimes B b.
	\end{align*}
	Explicitly,
	\begin{align*}
		A \otimes B = \begin{pmatrix}
			              A_{11} B & \cdots & A_{1n}B \\
			              \vdots   & \ddots & \vdots  \\
			              A_{n1} B & \cdots & A_{nn}B
		              \end{pmatrix}.
	\end{align*}
\end{definition}

\section{Postulates}
\label{sec:postulates}

\subsection{Quantum States}
\label{sec:quantum-states}

\begin{definition}
	\label{def:quantum-state}
	Let $\mathcal H$ be a Hilbert space with an finite ordered orthonormal basis
	$\rr{v^1, ..., v^d}$. A quantum state is a vector $\va{\psi}\in\mathcal H$.
	We write
	\begin{align*}
		\va{\psi} = \sum_{i\in I} \psi_i v^i
	\end{align*}
	where $\psi_1,...,\psi_d\in\C$ are the probability amplitudes of the $d$ outcomes.
	In particular, the probability of the outcome $1\leq i\leq d$, is $P \rr{ i } = \abs{ \psi_i }^2$.
	For $n$ bits there are $d=2^n$ outcomes.
\end{definition}

\begin{definition}
	\label{def:qubit}
	A qubit is a quantum state $ \va{\psi}\in\C^2$. We define
	\begin{align*}
		\va{ 0 } = \rr{ 1, 0 }^\top \\
		\va{ 1 } = \rr{ 0, 1 }^\top
	\end{align*}
\end{definition}

\begin{definition}
	\label{def:superposition}
	\begin{align*}
		\va{ + } = \frac{1}{\sqrt{2}} \rr{ \va{ 0 } + \va{ 1 } } \\
		\va{ - } = \frac{1}{\sqrt{2}} \rr{ \va{ 0 } - \va{ 1 } }
	\end{align*}
\end{definition}

\subsection{Quantum operations}
\label{sec:quantum-operations}

\begin{definition}
	\label{def:quantum-operation}
	An operation on a quantum state $ \va{ \psi } \in \mathcal H = \C^d$ is given by
	a unitary transformation $U:\mathcal H \to \mathcal H$, i.e.
	\begin{align*}
		UU^\dagger = U^\dagger U = I_d.
	\end{align*}
\end{definition}

\subsection{Measurement}
\label{sec:measurement}

\begin{definition}
	\label{def:probability-of-outcome}
	Let $\cc{\va{v_i}}$ be a basis of a Hilbert space $\mathcal{H}$ and an associated measurement $\psi$.
	Then the probability of outcome $i$ is
	\begin{align*}
		P\rr{i} = \abs{\ava{v_i}{\psi}}^2.
	\end{align*}
	The quantum state is updated to $\va{v_i}$.
\end{definition}

\section{Quantum circuits}
\label{sec:circuits}

\subsection{Gates}
\label{sec:gates}

\begin{definition}
	\label{def:single-qubit-gates}
	An $n$ qubit gate is a unitary matrix $U\in\Mat\rr{2^n;\C}$.
\end{definition}

\begin{corollary}
	Every quantum gate is invertible.
\end{corollary}

\subsubsection{Single qubit gates}
\label{sec:single-qubit-gates}

\begin{definition}
	\label{def:identity}
	The identity gate $I$ is the matrix
	\begin{align*}
		I = \begin{pmatrix}
			    1 & 0 \\ 0 & 1
		    \end{pmatrix}.
	\end{align*}
\end{definition}

\begin{definition}
	\label{def:not}
	The not gate $X$ is given by
	\begin{align*}
		X = \begin{pmatrix}
			    0 & 1 \\ 1 & 0
		    \end{pmatrix}
	\end{align*}
\end{definition}

\begin{definition}
	\label{def:phase-flip}
	The phase flip gate $Z$ is given by
	\begin{align*}
		Z = \begin{pmatrix} 1 & 0 \\ 0 & -1 \end{pmatrix}.
	\end{align*}
	It transforms as follows:
	\begin{align*}
		Z \rr{\alpha\va{0} + \beta\va{1}} = \alpha\va{0} - \beta\va{1}.
	\end{align*}
\end{definition}


\begin{definition}
	\label{def:hadamard}
	The Hadamard gate $H$ is given by
	\begin{align*}
		H = \frac{1}{\sqrt{2}} \begin{pmatrix} 1 & 1 \\ 1 & -1 \end{pmatrix}.
	\end{align*}
	It acts as a change of basis from $ \cc{ \va{ 0 }, \va{ 1 } } $ to $ \cc{ \va{ + }, \va{ - } }$.
\end{definition}

\begin{theorem}
	\label{thm:infinite-gates}
	There are infinitely many single-qubit quantum gates.
\end{theorem}

\subsubsection{Multiple qubit gates}
\label{sec:multiple-qubit-gates}

\begin{definition}
	\label{def:cnot}
	The \text{CNOT} gate $U_{CN}$ is a two qubit quantum gate defined by
	\begin{align*}
		U_{CN} = \begin{pmatrix}
			         1 & 0 & 0 & 0 \\
			         0 & 1 & 0 & 0 \\
			         0 & 0 & 0 & 1 \\
			         0 & 0 & 1 & 0
		         \end{pmatrix}
	\end{align*}
\end{definition}

\begin{definition}
	\label{def:controlled-u}
	Let $U$ be a quantum gate. Then the controlled-$U$ gate
\end{definition}

\end{document}
