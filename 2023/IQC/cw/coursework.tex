\documentclass{article}
\usepackage{homework-preamble}

\title{Introduction to Quantum Computing: Coursework}
\author{Franz Miltz}
\begin{document}
\maketitle

\section*{Question 1}

Consider the 4-qubit QFT circuit $U_{QFT}$ given in the assignment.

\begin{claim*}[a]
  Let $y_1,\ldots,y_4$ be the four output qubits of the QFT circtuit:
  \begin{align*}
    \va{y_1}\otimes \va{y_2}\otimes \va{y_3}\otimes\va{y_4}=U_{QFT}\va{1101}.
  \end{align*}
  Then
  \begin{align*}
    y_1 = \frac{1}{\sqrt{2}} \rr{\va{0}+e^{2\pi i 0.1101}\va{1}},\hs
    y_2 = \frac{1}{\sqrt{2}} \rr{\va{0}+e^{2\pi i 0.101}\va{1}},\\
    y_3 = \frac{1}{\sqrt{2}} \rr{\va{0}+e^{2\pi i 0.01}\va{1}},\hs
    y_4 = \frac{1}{\sqrt{2}} \rr{\va{0}+e^{2\pi i 0.1}\va{1}}.
  \end{align*}
  \begin{proof}
    We consider the states $\va{\psi_1},\ldots,\va{\psi_{10}}$ after applying each
    of the ten gates.
    We find
    \begin{align*}
      \va{\psi_1}&=H\va{1}\otimes \va{101} = \rr{\frac{\va{0}+e^{2\pi i 0.1}\va{1}}{\sqrt{2}}}\otimes\va{101},
    \end{align*}
    and then
    \begin{align*}
      \va{\psi_2}
      &=R_2\rr{\frac{\va{0}+e^{2\pi i 0.1}\va{1}}{\sqrt{2}}}\otimes\va{101}
      =\rr{\frac{\va{0}+e^{2\pi i 0.11}\va{1}}{\sqrt{2}}}\otimes\va{101}
    \end{align*}
    Now $x_3=0$ so $\va{\psi_3}=\va{\psi_2}$. Then
    \begin{align*}
      \va{\psi_4}
      &=R_4\rr{\frac{\va{0}+e^{2\pi i 0.11}\va{1}}{\sqrt{2}}}\otimes\va{101}
    \end{align*}
    Similarly,
    \begin{align*}
      \va{\psi_5}
      &=\rr{\frac{\va{0}+e^{2\pi i 0.1101}\va{1}}{\sqrt{2}}}
      \otimes H\va{1}
      \otimes \va{01}
      =\rr{\frac{\va{0}+e^{2\pi i 0.1101}\va{1}}{\sqrt{2}}}
      \otimes \rr{\frac{\va{0}+e^{2\pi i 0.1}\va{1}}{\sqrt{2}}}
      \otimes \va{01}.
    \end{align*}
    Now, $\va{\psi_6}=\va{\psi_5}$ as $x_2=0$. Then
    \begin{align*}
      \va{\psi_7}
      &= \rr{\frac{\va{0}+e^{2\pi i 0.1101}\va{1}}{\sqrt{2}}}
      \otimes R_3\rr{\frac{\va{0}+e^{2\pi i 0.1}\va{1}}{\sqrt{2}}}
      \otimes \va{01} \\
      &= \rr{\frac{\va{0}+e^{2\pi i 0.1101}\va{1}}{\sqrt{2}}}
      \otimes \rr{\frac{\va{0}+e^{2\pi i 0.101}\va{1}}{\sqrt{2}}}
      \otimes \va{01}.
    \end{align*}
    Further,
    \begin{align*}
      \va{\psi_8}
      &= \rr{\frac{\va{0}+e^{2\pi i 0.1101}\va{1}}{\sqrt{2}}}\otimes
      \rr{\frac{\va{0}+e^{2\pi i 0.101}\va{1}}{\sqrt{2}}} \otimes H\va{0} \otimes \va{1}\\
      &= \rr{\frac{\va{0}+e^{2\pi i 0.1101}\va{1}}{\sqrt{2}}}
      \otimes \rr{\frac{\va{0}+e^{2\pi i 0.101}\va{1}}{\sqrt{2}}}
      \otimes \rr{\frac{\va{0}+e^{2\pi i 0.0}}{\sqrt{2}}}
      \otimes \va{1},
    \end{align*}
    and
    \begin{align*}
      \va{\psi_9}
      &= \rr{\frac{\va{0}+e^{2\pi i 0.1101}\va{1}}{\sqrt{2}}}
      \otimes \rr{\frac{\va{0}+e^{2\pi i 0.101}\va{1}}{\sqrt{2}}}
      \otimes R_2\rr{\frac{\va{0}+e^{2\pi i 0.0}}{\sqrt{2}}}
      \otimes \va{1} \\
      &= \rr{\frac{\va{0}+e^{2\pi i 0.1101}\va{1}}{\sqrt{2}}}
      \otimes \rr{\frac{\va{0}+e^{2\pi i 0.101}\va{1}}{\sqrt{2}}}
      \otimes \rr{\frac{\va{0}+e^{2\pi i 0.01}}{\sqrt{2}}}
      \otimes \va{1}.
    \end{align*}
    Finally,
    \begin{align*}
      \va{\psi_{10}}
      &= \rr{\frac{\va{0}+e^{2\pi i 0.1101}\va{1}}{\sqrt{2}}}
      \otimes \rr{\frac{\va{0}+e^{2\pi i 0.101}\va{1}}{\sqrt{2}}}
      \otimes \rr{\frac{\va{0}+e^{2\pi i 0.01}}{\sqrt{2}}}
      \otimes H\va{1}\\
      &= \rr{\frac{\va{0}+e^{2\pi i 0.1101}\va{1}}{\sqrt{2}}}
      \otimes \rr{\frac{\va{0}+e^{2\pi i 0.101}\va{1}}{\sqrt{2}}}
      \otimes \rr{\frac{\va{0}+e^{2\pi i 0.01}}{\sqrt{2}}}
      \otimes \rr{\frac{\va{0}+e^{2\pi i 0.1}\va{1}}{\sqrt{2}}},
    \end{align*}
    as required.
  \end{proof}
\end{claim*}

\begin{claim*}[b]
  Let $\mathcal H$ be a two dimensional Hilbert space and let $x\in\mathcal H$.
  Then $R_k\va{x}\to\va{x}$ as $k\to\infty$.
  \begin{proof}
    Let $\va{x}=x_1\va{0}+x_2\va{1}$. Then
    \begin{align*}
      \lim_{k\to\infty} \rr{R_k\va{x}}
      = \lim_{k\to\infty} \rr{x_1\va{0} + e^{2\pi i/2^k}x_2\va{1}}.
    \end{align*}
    It remains to show that $e^{2\pi i / 2^k} \to 1$ as $k\to\infty$. We note that
    $2^k\to\infty$ as $k\to\infty$ so $2\pi i / 2^k\to 0$. By continuity of
    $x\mapsto e^x$ we have
    \begin{align*}
      \lim_{k\to\infty} \rr{R_k\va{x}} = x_1\va{0} + \lim_{k\to\infty} \rr{e^{2\pi i/2^k} x_2\va{1}} = x_1\va{0}+x_2\va{1} = \va{x}.
    \end{align*}
  \end{proof}
\end{claim*}

\begin{claim*}[b']
  Let $\mathcal H$ be a two dimensional Hilbert space and let $x\in\mathcal H$
  with $\vabs{\va{x}}_{\mathcal H}=1$
  where $\vabs{v}_{\mathcal H}=\ava{v}{v}$ is the induced norm on $\mathcal H$.
  Then, for all $k>2$,
  \begin{align*}
    \vabs{R_k \va{x}-\va{x}}_{\mathcal H} < \tan\rr{\frac{2\pi}{2^k}}.
  \end{align*}
  \begin{proof}
    Let $\va{x}=x_1\va{0}+x_2\va{1}$. Then
    \begin{align*}
      \vabs{R_k\va{x}-\va{x}}_{\mathcal H}
      = \vabs{\rr{e^{2\pi i / 2^k}-1}x_2\va{1}}_{\mathcal H}
      = \abs{\rr{e^{2\pi i / 2^k}-1}x_2}
    \end{align*}
    As $\vabs{x}_{\mathcal H}=1$ we have $x_2\leq 1$ so
    \begin{align*}
      \vabs{R_k\va{x}-\va{x}}_{\mathcal H}
      \leq \abs{e^{2\pi i / 2^k}-1}.
    \end{align*}
    The inequality $e^{\theta i}-1<\tan\theta$, for $0<\theta < \pi/2$, holds due to
    basic trigonometry. Thus the claim follows.
  \end{proof}
\end{claim*}

Consider the 4-qubit lazy QFT circuit $U_{LQFT}^{\rr{k_0}}$ which differs from $U_{QFT}$
only by replacing $R_k$ with $I$ whenever $k\geq k_0$.

\begin{claim*}[c]
  Let $k_0=3$ and let $y_1,\ldots,y_4$ be the four output qubits of the lazy QFT circuit:
  \begin{align*}
    \va{y_1}\otimes \va{y_2}\otimes \va{y_3}\otimes\va{y_4}=U^{\rr{3}}_{LQFT}\va{1101}.
  \end{align*}
  Then
  \begin{align*}
    y_1 = \frac{1}{\sqrt{2}} \rr{\va{0}+e^{2\pi i 0.11}\va{1}},\hs
    y_2 = \frac{1}{\sqrt{2}} \rr{\va{0}+e^{2\pi i 0.10}\va{1}},\\
    y_3 = \frac{1}{\sqrt{2}} \rr{\va{0}+e^{2\pi i 0.01}\va{1}},\hs
    y_4 = \frac{1}{\sqrt{2}} \rr{\va{0}+e^{2\pi i 0.1}\va{1}}.
  \end{align*}
  \begin{proof}
    We consider the states $\va{\psi'_1},\ldots,\va{\psi'_{10}}$ after applying each of the
    ten gates (including identities for consistency).
    The proof is analogous to the normal QFT case except that we omit all
    steps involving $R_3$ or $R_4$ (as they are replaced by $I$). Thus
    \begin{align*}
      \va{\psi'_4} = \va{\psi'_3} &=\rr{\frac{\va{0}+e^{2\pi i 0.11}\va{1}}{\sqrt{2}}}\otimes\va{101},
    \end{align*}
    and
    \begin{align*}
      \va{\psi'_7} = \va{\psi'_6} = \rr{\frac{\va{0}+e^{2\pi i 0.11}\va{1}}{\sqrt{2}}}
      \otimes \rr{\frac{\va{0}+e^{2\pi i 0.1}\va{1}}{\sqrt{2}}}
      \otimes \va{01}.
    \end{align*}
    Following the previous derivation we then obtain
    \begin{align*}
      \va{\psi'_{10}}
      &= \rr{\frac{\va{0}+e^{2\pi i 0.11}\va{1}}{\sqrt{2}}}
      \otimes \rr{\frac{\va{0}+e^{2\pi i 0.10}\va{1}}{\sqrt{2}}}
      \otimes \rr{\frac{\va{0}+e^{2\pi i 0.01}}{\sqrt{2}}}
      \otimes \rr{\frac{\va{0}+e^{2\pi i 0.1}\va{1}}{\sqrt{2}}},
    \end{align*}
    as claimed.
  \end{proof}
\end{claim*}

We note that only the first two qubits were affected, and only insignificant bits
with values $1$ in the input are faulty.

\begin{claim*}
  The lazy QFT circuit with parameter $k_0$ for $n$ qubits requires $O(k_0 n)$ gates.
  \begin{proof}
    We note that each qubit only gets transformed by each $R_k$ gate at most once.
    There are less than $k_0$ gates of the form $R_k$ by construction, so
    we require less than $k_0 n$ gates in total. The claimed asymptotic bound follows.
  \end{proof}
\end{claim*}

\section*{Question 2}

\paragraph{a.} Given that $x\cdot y=0$ in $3/4$ of the cases, Alice and Bob may simply
agree to choose $a=b=0$ all the time.

\paragraph{b.} We choose $a = x$ and $b = \overline{y}$. Then the only case where $x \oplus\overline{y} \neq x \cdot y$ is when $x=0$ and $y=0$.

\paragraph{c.} Given that $x\cdot y=1$ in $1/4$ of the cases, Alice and Bob may simply
agree to choose $a=b=1$ all the time.

\paragraph{d.} We claim that any deterministic strategy achieves a winning probability
of either $1/4$ or $3/4$. We note that there are only four functions $\cc{0,1}\to\cc{0,1}$.
Namely, the constants $x\mapsto 0$ and $x \mapsto 1$, the identity $x\mapsto x$, and the
bit flip $x\mapsto\overline x$.

Let $X,Y$ be uniform random variables over $\cc{0,1}$. We then check
\begin{align*}
  \pr{0 = X \cdot Y} &= 3/4, \\
  \pr{1 = X \cdot Y} &= 1/4, \\
  \pr{X = X \cdot Y} &= 3/4, \\
  \pr{Y = X \cdot Y} &= 3/4, \\
  \pr{X \oplus Y = X \cdot Y} &= 1/4,\\
  \pr{X \oplus \overline Y = X \cdot Y} &= 3/4 \\
  \pr{\overline X \oplus Y = X \cdot Y} &= 3/4,\\
  \pr{\overline X \oplus \overline Y = X \cdot Y} &= 1/4.
\end{align*}

\paragraph{e.} Let $F,G$ be random variables over the set of functions $\cc{0,1}\to\cc{0,1}$.
Alice's and Bob's randomness simply amounts to changing the distributions of $F$ and $G$
depending on $X$ and $Y$ respectivel. I.e. choosing values for
\begin{align*}
  \prc{F=f}{X=x}, \hs \prc{G=g}{Y=y}.
\end{align*}
Note that $F$ does not depend on the value of $Y$ and $G$ does not depend on the value of
$X$ (as Alice does not know $y$ and Bob does not know $x$).
Thus, regardless of what Alice and Bob agree on, we have the winning probability
\begin{align*}
  \pr{F\rr{X}\oplus G\rr{Y}=X\cdot Y} = \sum_{f,g}\pr{F=f\cap G=g}\pr{f\rr{X}\oplus g\rr{Y}=X\cdot Y}.
\end{align*}
We note that, for all $f,g:\cc{0,1}\to\cc{0,1}$,
\begin{align*}
  \pr{f\rr{X}\oplus g\rr{Y}=X\cdot Y}\leq 3/4
\end{align*}
so, by the law of total probability,
\begin{align*}
  \pr{F\rr{X}\oplus G\rr{Y}=X\cdot Y} \leq 3/4.
\end{align*}

\paragraph{f.} We note
\begin{align*}
  \av{0}\otimes \av{v_0} = \cos\rr{\pi/8} \av{00} + \sin\rr{\pi/8}\av{01} , \hs
  \av{1}\otimes \av{v_1} = - \sin\rr{\pi/8}\av{10} + \cos\rr{\pi/8} \av{11}.
\end{align*}
Thus
\begin{align*}
  \abs{\rr{\av{0}\otimes\av{v_0}}\va{\psi}}^2
  &= \abs{\rr{\cos\rr{\pi/8} \av{00} + \sin\rr{\pi/8}\av{01}}\rr{\frac{\va{00}+\va{11}}{\sqrt{2}}}}^2 \\
  &=\frac{1}{2}\abs{\rr{\cos\rr{\pi/8} \av{00} + \sin\rr{\pi/8}\av{01}}\rr{\va{00}+\va{11}}}^2 \\
  &=\frac{1}{2}\cos^2\rr{\pi/8}.
\end{align*}
Similarly,
\begin{align*}
  \abs{\rr{\av{1}\otimes\av{v_1}}\va{\psi}}^2
  &= \abs{\rr{-\sin\rr{\pi/8} \av{10} + \cos\rr{\pi/8}\av{11}}\rr{\frac{\va{00}+\va{11}}{\sqrt{2}}}}^2 \\
  &=\frac{1}{2}\abs{\rr{-\sin\rr{\pi/8} \av{10} + \cos\rr{\pi/8}\av{11}}\rr{\va{00}+\va{11}}}^2 \\
  &=\frac{1}{2}\cos^2\rr{\pi/8}.
\end{align*}
Thus we have
\begin{align*}
  \prc{\text{Win}}{X=0\cap Y=0} = \cos^2\rr{\pi/8}.
\end{align*}

\paragraph{g.} We note
\begin{align*}
  \av{0}\otimes \av{w_0} = \cos\rr{\pi/8} \av{00} - \sin\rr{\pi/8}\av{01} , \hs
  \av{1}\otimes \av{w_1} = \sin\rr{\pi/8}\av{10} + \cos\rr{\pi/8} \av{11}.
\end{align*}
Once again, we calculate
\begin{align*}
  \abs{\rr{\av{0}\otimes\av{w_0}}\va{\psi}}^2
  = \abs{\rr{\cos\rr{\pi/8} \av{00} - \sin\rr{\pi/8}\av{01}}\rr{\frac{\va{00}+\va{11}}{\sqrt{2}}}}^2
  =\frac{1}{2}\cos^2\rr{\pi/8}.
\end{align*}
Finally,
\begin{align*}
  \abs{\rr{\av{1}\otimes\av{w_1}}\va{\psi}}^2
  = \abs{\rr{\sin\rr{\pi/8}\av{10} + \cos\rr{\pi/8} \av{11}}\rr{\frac{\va{00}+\va{11}}{\sqrt{2}}}}^2
  =\frac{1}{2}\cos^2\rr{\pi/8}.
\end{align*}
Thus we have
\begin{align*}
  \prc{\text{Win}}{X=0\cap Y=1} = \cos^2\rr{\pi/8}.
\end{align*}

\paragraph{h.}

We calculate
\begin{align*}
  \va{\psi}
  &= \frac{\va{00} + \va{11}}{\sqrt{2}} \\
  &= \frac{2\rr{\va{00} + \va{11}}}{\sqrt{2^3}} \\
  &= \frac{\rr{\va{00}-\va{01}-\va{10}+\va{11}} + \rr{\va{00}+\va{01}+\va{10}+\va{11}}}{\sqrt{2^3}} \\
  &= \frac{\rr{\va{0}-\va{1}}\otimes \rr{\va{0}-\va{1}} + \rr{\va{0}+\va{1}}\otimes \rr{\va{0}+\va{1}}}{\sqrt{2^3}} \\
  &= \frac{\rr{\frac{\va{0}-\va{1}}{\sqrt{2}}}\otimes \rr{\frac{\va{0}-\va{1}}{\sqrt{2}}} + \rr{\frac{\va{0}+\va{1}}{\sqrt{2}}}\otimes \rr{\frac{\va{0}+\va{1}}{\sqrt{2}}}}{\sqrt{2}} \\
  &= \frac{\va{--} + \va{++}}{\sqrt{2}}
\end{align*}

\paragraph{i.}

We find
\begin{align*}
  \abs{\rr{\av{+}\otimes \av{v_0}}\va{\psi}}^2
  &= \abs{\rr{\rr{\frac{\av{0}+\av{1}}{\sqrt{2}}}\otimes\rr{\cos\rr{\pi/8}\av{0}+\sin\rr{\pi/8}\av{1}}}\rr{\frac{\va{00}+\va{11}}{\sqrt{2}}}}^2 \\
  &= \frac{1}{4}\abs{\rr{\cos\rr{\pi/8}\av{00} + \sin\rr{\pi/8}\av{10} + \cos\rr{\pi/8}\av{10}+\sin\rr{\pi/8}\av{11}}\rr{\va{00}+\va{11}}}^2 \\
  &= \frac{1}{4}\abs{\cos\rr{\pi/8}+\sin\rr{\pi/8}}^2 \\
\end{align*}
Analogously,
\begin{align*}
  \abs{\rr{\av{-}\otimes \av{v_1}}\va{\psi}}
  &= \abs{\rr{\rr{\frac{\av{0}-\av{1}}{\sqrt{2}}}\otimes\rr{-\sin\rr{\pi/8}\av{0}+\cos\rr{\pi/8}\av{1}}}\rr{\frac{\va{00}+\va{11}}{\sqrt{2}}}}^2 \\
  &= \frac{1}{4}\abs{\rr{-\sin\rr{\pi/8}\av{00} + \cos\rr{\pi/8}\av{10} + \sin\rr{\pi/8}\av{10}-\cos\rr{\pi/8}\av{11}}\rr{\va{00}+\va{11}}}^2 \\
  &= \frac{1}{4}\abs{\cos\rr{\pi/8}+\sin\rr{\pi/8}}^2
\end{align*}
Thus we have
\begin{align*}
  \prc{\text{Win}}{X=1\cap Y=0} = \frac{1}{2}\abs{\cos\rr{\pi/8}+\sin\rr{\pi/8}}^2.
\end{align*}

\paragraph{j.}

We find
\begin{align*}
  \abs{\rr{\av{+}\otimes\av{w_0}}} &= \abs{\rr{\rr{\frac{\av{0}+\av{1}}{\sqrt{2}}}\otimes\rr{\cos\rr{\pi/8}\av{0}-\sin\rr{\pi/8}\av{1}}}\rr{\frac{\va{00}+\va{11}}{\sqrt{2}}}}^2 \\
                                   &= \frac{1}{4} \abs{\rr{\cos\rr{\pi/8}\av{00}-\sin\rr{\pi/8}\av{01}+\cos\rr{\pi/8}\av{10}-\sin\rr{\pi/8}\av{11}}\rr{\va{00}+\va{11}}}^2 \\
                                   &= \frac{1}{4} \abs{\cos\rr{\pi/8}-\sin\rr{\pi/8}}^2
\end{align*}
Analogously,
\begin{align*}
  \abs{\rr{\av{-}\otimes\av{w_1}}} &= \abs{\rr{\rr{\frac{\av{0}-\av{1}}{\sqrt{2}}}\otimes\rr{\sin\rr{\pi/8}\av{0}+\cos\rr{\pi/8}\av{1}}}\rr{\frac{\va{00}+\va{11}}{\sqrt{2}}}}^2 \\
                                   &= \frac{1}{4} \abs{\rr{\sin\rr{\pi/8}\av{00}+\cos\rr{\pi/8}\av{01}-\sin\rr{\pi/8}\av{10}-\cos\rr{\pi/8}\av{11}}\rr{\va{00}+\va{11}}}^2 \\
                                   &= \frac{1}{4} \abs{\cos\rr{\pi/8}-\sin\rr{\pi/8}}^2
\end{align*}
Thus we have
\begin{align*}
  \prc{\text{Win}}{X=1\cap Y=1} = \frac{1}{2} \abs{\cos\rr{\pi/8}-\sin\rr{\pi/8}}^2.
\end{align*}

\paragraph{k.}

The cases discussed above all occur with equal probability. Moreover, we have
\begin{align*}
  \pr{\text{Win}} &= \frac{1}{4}\prc{\text{Win}}{X=0\cap Y=0} +\frac{1}{4}\prc{\text{Win}}{X=0\cap Y=1}\\
                  &+\frac{1}{4}\prc{\text{Win}}{X=1\cap Y=0} +\frac{1}{4}\prc{\text{Win}}{X=1\cap Y=1}\\
                  &=\frac{1}{2}\cos^2\rr{\pi/8}+\frac{1}{8}\abs{\cos\rr{\pi/8}+\sin\rr{\pi/8}}^2+\frac{1}{8}\abs{\cos\rr{\pi/8}-\sin\rr{\pi/8}}^2
\end{align*}

\end{document}
