\documentclass{article}
\usepackage{homework-preamble}
\mkanonthms
\begin{document}
\title{General Topology: Assignment 8}
\author{Franz Miltz (UUN: S1971811)}
\date{24 November 2022}
\maketitle

\begin{claim*}
  Let $f:T^\omega\to\R$ be a continuous map. Then there exist points $x,y\in T^\omega$ such that
  \begin{enumerate}
    \item for every $z\in T^\omega$, one has $f\rr{x}\leq z\leq f\rr{y}$, and
    \item for every $t\in\R$ with $f\rr{x}\leq t \leq f\rr{y}$, there eixsts $z\in T^\omega$ such that
      $t=f\rr{z}$.
  \end{enumerate}
  \begin{proof}
    We note that each $S^1$ is compact so $T^\omega$ is compact. As $f$ is continuous, $f\rr{T^\omega}$
    must be compact. In particular, $f\rr{T^\omega}$ is a subset of Euclidean space so it must be closed and bounded.

    For a contradiction, assume $T^\omega$ is not connected. Then $T^\omega$ is the disjoint union
    of non-empty opens $U,V\subseteq T^\omega$. Consider $x=\rr{x_i}_{i\in\N}\in U$ and fix $n\in\N$.
    We note that the inclusion
    \begin{align*}
      i:S^1 &\to T^\omega \\
      y     &\mapsto \rr{x_1,\ldots,x_{n-1},y,x_n,\ldots}
    \end{align*}
    is continuous.
    Moreover, $S^1$ is connected so the image $i\rr{S^1}$ is connected. Thus
    $i\rr{S^1}\subseteq U$. We have shown that any element of $T^\omega$ differing from $x$ in exactly
    one coordinate is in $U$. It may then be shown by induction that the same holds for any
    element of $T^\omega$ differing from $x$ in finitely many coordinates. We note that there
    are opens $U_1,\ldots,U_n\subseteq S^1$ such that
    \begin{align*}
      U_1\times\cdots\times U_n\times T^\omega\subseteq U.
    \end{align*}
    In particular, for any $x\in V$, it is possible to find a $u\in U_1\times\cdots\times U_n\times T^\omega$
    that differs from $x$ in only finitely many coordinates. Thus $x\in U$. Contradiction.
    We conclude that $T^\omega$ is connected.

    Continuous maps preserve connectedness, so $f\rr{T^\omega}$ is connected. Thus
    $f\rr{T^\omega}\subseteq\R$ is a closed and bounded interval and the claim follows.
  \end{proof}
\end{claim*}

\begin{claim*}
  The solenoid $\hat S$ is compact and Hausdorff.
  \begin{proof}
    We show that $\hat S\subseteq T^\omega$ is closed. The procedure is analogous to showing
    that $\hat Z$ is closed.

    We note that, for each $n\in\N$, the choice of $s_n$ in any $\rr{s_n}_{n\in\N}\in\hat S$
    is only restricted by choice of $s_1,\ldots,s_{n-1}$. Thus we may consider the set of
    choices that remain at each step.
    Let $s=\rr{s_n}_{n\in\N}\in\hat\Z$ and $N\in\N$. Define the set
    \begin{align*}
      C_N\rr{s} = \cc{x \in S^1 : \forall m<N. \: \text{if $m|n$ then }s_m= x^{n/m} }\subseteq S^1.
    \end{align*}
    If $N$ is prime then $C_N\rr{s}=S^1$. Otherwise $C_N\rr{s}$ is finite and thus. In either
    case $C_N\rr{s}$ is closed.

    In order to show that $\hat S$ is closed we may show that its complement is open. We thus define the
    set of sequences that have the same prefix as $s\in\hat S$ but violate the condition above
    at some $N\in\N$:
    \begin{align*}
      D_N\rr{s}=\prod_{n=1}^{N-1}\cc{s_n}\times\rr{S^1\setminus C_N\rr{s}}\times T^\omega \subseteq T^\omega.
    \end{align*}
    This allows us to consider all the sequences that are constructed in such manner:
    \begin{align*}
      D = \bigcup_{N\in\N}\bigcup_{s\in\hat S} D_N\rr{s} \subseteq T^\omega.
    \end{align*}
    We note that $D\subseteq T^\omega$ is open. This follows as, for any $N\in\N$ and
    $s\in\hat S$, the set $C_N\rr{s}\subseteq S^1$ is open. Now
    $S^1\setminus C_N\rr{s}$ is open and so is $D_N\rr{s}$. Thus $D$ is a union of opens.

    Now fix $s=\rr{s_n}_{n\in\N}\in T^\omega\setminus\hat S$ and
    \begin{align*}
      N=\min\cc{n : \text{$m|n$ and $s_m\neq s_n^{n/m}$}},
    \end{align*}
    i.e. $s_N$ is the first element of the sequence that violates the condition that defines $\hat S$.
    In particular, there exists a sequence $t=\rr{t_n}_{n\in\N}\in\hat S$ such that,
    for $n<N$, $s_n=t_n$ and $s_N\in S^1\setminus C_N\rr{t}$. Thus
    \begin{align*}
      s\in \prod_{n=1}^{N-1} \cc{t_n} \times \rr{T^\omega\setminus C_N\rr{t}} \times T^\omega= D_N\rr{t}\subseteq D
    \end{align*}
    We have shown $T^\omega\setminus\hat S\subseteq D$ so $D=T^\omega\setminus\hat S$. Now
    $D$ is open so $\hat S$ is closed.

    The product of compact Hausdorff spaces $T^\omega$ is compact and Hausdorff.
    Thus the closed subspace $\hat S\subseteq T^\omega$ is compact and Hausdorff.
  \end{proof}
\end{claim*}

\begin{claim*}
  Let $f:\hat S\to S$ be the restriction of the projection $T^\omega\to S^1$.
  The fibre $F$ of $f$ over $1\subseteq S^1$ is homeomorphic to $\hat Z$.
  \begin{proof}
    Consider $\rr{z_n}_{n\in\N}\in F$. Then $z_1=1$ so, for all $n\in\N$,
    $z_n^n=1$. In particular, for some unique $0\leq k_n<n$, $z_n=\exp\rr{2\pi ik_n/n}$.
    We construct the map
    \begin{align*}
      \phi : F &\to \hat Z\\
      \rr{\exp\rr{2\pi ik_n/n}}_{n\in\N} &\mapsto \rr{k_n}_{n\in\N}
    \end{align*}
    We note that this is well defined: Fix $\rr{\exp\rr{2\pi ik_n/n}}_{n\in\N}\in F$ and
    let $m,n\in\N$ such that $m$ divides $n$. Then
    \begin{align*}
      \rr{\exp\rr{2\pi i k_n/n}}^{n/m} = \exp\rr{2\pi i k_m/m}.
    \end{align*}
    Equivalently, $k_n\equiv k_m\mod m$ which is the defining property of $\hat Z$.

    The map $\phi$ is injective by uniqueness of the $k_n$. Surjectivity follows as
    the conditions are equivalent, i.e. each $\rr{k_n}_{n\in\N}\in\hat Z$ defines an element
    $\rr{\exp\rr{2\pi i k_n/n}}_{n\in\N}\in F$.

    Consider open sets $U_1,\ldots,U_n\subseteq S^1$. As $\phi$ is bijective,
    \begin{align}
      \label{eq:image}
      \phi\rr{\rr{U_1\times\cdots\times U_n\times T^\omega}\cap F} \subseteq \hat Z
    \end{align}
    may be written as
    \begin{align*}
      V_1\times\cdots\times V_n\times \Z/\rr{n+1}\times\cdots\subseteq\hat Z
    \end{align*}
    where $V_k\subseteq \Z/k$. In particular, each $\Z/k$ has the discrete topology so
    each $V_k$ is open. This makes the product in $\ref{eq:image}$ open. As the topology is
    generated by such open sets, $\phi$ is open.

    Now consider open sets $U_1\subseteq \Z/1,\ldots,U_n\subseteq\Z/n$. Once again, we may
    write
    \begin{align*}
      \inv \phi\rr{\rr{U_1\times\cdots\times U_n\times\Z/\rr{n+1}\times\cdots}\cap\hat Z} \subseteq F
    \end{align*}
    as
    \begin{align*}
      W_1\times\cdots\times W_n\times T^\omega
    \end{align*}
    where $W_k\subseteq S^1$. We then note that each $W_k$ is finite. By Hausdorffness of $S^1$,
    for each $1\leq k\leq n$, we have open sets $V^k_1,\ldots,V_k^k\subseteq S^1$ such that
    \begin{align*}
      V\cap F = W_1\times\cdots\times W_n\times T^\omega
    \end{align*}
    where
    \begin{align*}
      V = \bigcup_{i=1}^1 V^1_i \times \bigcup_{i=2}^2 V^2_i\times\cdots\times\bigcup_{i=1}^n V^n_i\times T^\omega\subseteq T^\omega.
    \end{align*}
    In particular, $V$ is open in $T^\omega$ so $V\cap F$ is open in $\hat S$. Thus $\phi$ is continuous.

    We have shown that $\phi:\hat S\to\hat Z$ is an open and continuous bijection, i.e. a homeomorphism.
  \end{proof}
\end{claim*}

\end{document}
