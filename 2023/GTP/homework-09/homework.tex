\documentclass{article}
\usepackage{homework-preamble}
\mkanonthms
\begin{document}
\title{General Topology: Assignment 8}
\author{Franz Miltz (UUN: S1971811)}
\date{24 November 2022}
\maketitle

\begin{claim*}[1]
  Let $f:T^\omega\to\R$ be a continuous map. Then there exist points $x,y\in T^\omega$ such that
  \begin{enumerate}
    \item for every $z\in T^\omega$, one has $f\rr{x}\leq z\leq f\rr{y}$, and
    \item for every $t\in\R$ with $f\rr{x}\leq t \leq f\rr{y}$, there eixsts $z\in T^\omega$ such that
      $t=f\rr{z}$.
  \end{enumerate}
  \begin{proof}
    We note that each $S^1$ is compact so $T^\omega$ is compact. As $f$ is continuous, $f\rr{T^\omega}$
    must be compact. In particular, $f\rr{T^\omega}$ is a subset of Euclidean space so it must be closed and bounded.

    For a contradiction, assume $T^\omega$ is not connected. Then $T^\omega$ is the disjoint union
    of non-empty opens $U,V\subseteq T^\omega$. Consider $x=\rr{x_i}_{i\in\N}\in U$ and fix $n\in\N$.
    We note that the inclusion
    \begin{align*}
      i:S^1 &\to T^\omega \\
      y     &\mapsto \rr{x_1,\ldots,x_{n-1},y,x_n,\ldots}
    \end{align*}
    is continuous.
    Moreover, $S^1$ is connected so the image $i\rr{S^1}$ is connected. Thus
    $i\rr{S^1}\subseteq U$. We have shown that any element of $T^\omega$ differing from $x$ in exactly
    one coordinate is in $U$. It may then be shown by induction that the same holds for any
    element of $T^\omega$ differing from $x$ in finitely many coordinates. We note that there
    are opens $U_1,\ldots,U_n\subseteq S^1$ such that
    \begin{align*}
      U_1\times\cdots\times U_n\times T^\omega\subseteq U.
    \end{align*}
    In particular, for any $x\in V$, it is possible to find a $u\in U_1\times\cdots\times U_n\times T^\omega$
    that differs from $x$ in only finitely many coordinates. Thus $x\in U$. Contradiction.
    We conclude that $T^\omega$ is connected.

    Continuous maps preserve connectedness so $f\rr{T^\omega}$ is connected. Thus $f\rr{T^\omega}$
    is a closed and bounded interval and the claim follows.
  \end{proof}
\end{claim*}

\end{document}
