\documentclass{article}
\usepackage{homework-preamble}
\mkanonthms
\begin{document}
\title{General Topology: Assignment 2}
\author{Franz Miltz (UUN: S1971811)}
\date{04 October 2022}
\maketitle

\begin{claim*}[1]
  Consider the real line $\R$ and the maps
  \begin{align*}
    \tau : \P\rr\R \to \P\rr\R, \hs
    \kappa : \P\rr\R \to \P\rr\R
  \end{align*}
  taking subsets to their closures and complements, respectively.
  There exists a subset $S\subseteq\R$ such that the following sets are distinct:
  \begin{center}
    $S$,
    $\kappa\rr S$,
    $\tau\rr S$,
    $\kappa\tau\rr S$,
    $\tau\kappa\rr S$,
    $\kappa\tau\kappa\rr S$,
    $\tau\kappa\tau\rr S$,
    $\kappa\tau\kappa\tau\rr S$,
    $\tau\kappa\tau\kappa\rr S$,
    $\kappa\tau\kappa\tau\kappa\rr S$,\\
    $\tau\kappa\tau\kappa\tau\rr S$,
    $\kappa\tau\kappa\tau\kappa\tau\rr S$,
    $\tau\kappa\tau\kappa\tau\kappa\rr S$,
    $\kappa\tau\kappa\tau\kappa\tau\kappa\rr S$.
  \end{center}
  \begin{proof}
    Let \begin{align*}
      S = \rr{0,1} \cup \rr{1,2} \cup \cc{3} \cup \rr{\bb{4,5}\cap\Q}.
    \end{align*}
    Then we obtain the following sets:
    \begin{enumerate}
      \item $S=\rr{0,1} \cup \rr{1,2} \cup \cc{3} \cup \rr{\bb{4,5}\cap\Q}$
      \item $\kappa\rr S=\rb{-\infty,0}\cup\cc{1}\cup\br{2,3}\cup\rr{3,4}\cup\rr{\bb{4,5}\setminus\Q}\cup\rr{5,\infty}$
      \item $\tau\rr S=\bb{0,2}\cup\cc{3}\cup\bb{4,5}$
      \item $\kappa\tau\rr S=\rr{-\infty,0}\cup\rr{2,3}\cup\rr{3,4}\cup\rr{5,\infty}$
      \item $\tau\kappa\rr S=\rb{-\infty,0}\cup\cc{1}\cup\rr{2,\infty}$
      \item $\kappa\tau\kappa\rr S=\rr{0,1}\cup\rr{1,2}$
      \item $\tau\kappa\tau\rr S=\rb{-\infty,0}\cup\bb{2,4}\cup\br{5,\infty}$
      \item $\kappa\tau\kappa\tau\rr S=\rr{0,2}\cup\rr{4,5}$
      \item $\tau\kappa\tau\kappa\rr S=\bb{0,2}$
      \item $\kappa\tau\kappa\tau\kappa\rr S=\rr{-\infty,0}\cup\rr{2,\infty}$
      \item $\tau\kappa\tau\kappa\tau\rr S=\bb{0,2}\cup\bb{4,5}$
      \item $\kappa\tau\kappa\tau\kappa\tau\rr S=\rr{-\infty,0}\cup\rr{2,4}\cup\rr{5,\infty}$
      \item $\tau\kappa\tau\kappa\tau\kappa\rr S=\rb{-\infty,0}\cup\br{2,\infty}$
      \item $\kappa\tau\kappa\tau\kappa\tau\kappa\rr S=\rr{0,2}$
    \end{enumerate}
    It is straightforward to check that these are distinct.
  \end{proof}
\end{claim*}

\begin{claim*}[2a]
  The Cantor set $C$ is uncountable.
  \begin{proof}
    It is known that every element $x\in C$ may be uniquely represented as
    \begin{align} \label{ternary}
      x = \sum_{n=1}^{ \infty } \frac{x_n}{3^n}
    \end{align}
    for some $\rr{x_n}_{n\in\N}$ where all $x_n\in\cc{0,2}$. We thus construct a map
    \begin{align*}
      \phi : C &\to \cc{0,1}^{\N}\\
      \sum_{n=1}^{ \infty } \frac{x_n}{3^n} &\mapsto \rr{x_n/2}_{n\in\N}
    \end{align*}
    This map is injective as its inverse is given by
    \begin{align*}
      \inv\phi : \cc{0,1}^\N &\to C \\
      \rr{x_n}_{n\in\N} &\mapsto \sum_{n=1}^{\infty} \frac{2x_n}{3^n}.
    \end{align*}
    We see this as, for all $\rr{x_n}_{n\in\N}\in \cc{0,1}^\N$,
    \begin{align*}
      \rr{\phi\circ\inv\phi}\rr{\rr{x_n}_{n\in\N}} = \phi\rr{\sum_{n=1}^{\infty} \frac{2x_n}{3^n}} = \rr{2x_n/2}_{n\in\N} = \rr{x_n}_{n\in\N}
    \end{align*}
    and
    \begin{align*}
      \rr{\inv\phi\circ\phi}\rr{\sum_{n=1}^{\infty} \frac{x_n}{3^n}} = \inv\phi\rr{\rr{x_n/2}_{n\in\N}} = \sum_{n=1}^{\infty} \frac{2x_n/2}{3^n} = \sum_{n=1}^{\infty} \frac{x_n}{3^n}.
    \end{align*}
    Thus $\phi$ is a bijection to the uncountable set of binary strings $\cc{0,1}^\N$.
  \end{proof}
\end{claim*}

\begin{claim*}[2b]
  As a subset of the topological space $\rr{\bb{0,1},\tau}$, the Cantor set $C$ is closed.
  \begin{proof}
    We note
    \begin{align*}
      C = \bigcap_{n\in\N} C_n.
    \end{align*}
    In particular, for all $n\in\N$, $C_n$ is a finite union closed intervals. Thus every
    $C_n$ is closed. It now follows that the intersection of closed sets $C$ is closed, too.
  \end{proof}
\end{claim*}

\begin{claim*}[2c]
  As a subset of the topological space $\rr{\bb{0,1},\tau}$, the Cantor set $C$ is codense.
  \begin{proof}
    Suppose $x\in C$ is an interior point and let $\rr{x_n}_{n\in\N}$ be as in (\ref{ternary}).
    Let $\e>0$. Choose $N\in\N$ such that
    $3^{-N}<\e$. Then
    \begin{align} \label{middle}
      \abs{\sum_{n=1}^{\infty} \frac{x_n}{3^n} - \sum_{n=1}^{N} \frac{x_n}{3^n} - \sum_{n=N+1}^{\infty} \frac{1}{3^n}}
      = \abs{\sum_{n =N+1}^{\infty} \frac{x_n - 1}{3^n}}
      \leq \sum_{n=N+1}^{\infty}\frac{1}{3^n}
      <\frac{1}{3^N}
      <\e.
    \end{align}
    However,
    \begin{align*}
      \sum_{n=1}^N \frac{x_n}{3^n} + \sum_{n=N+1}^{\infty} \frac{1}{3^n} \not\in C.
    \end{align*}
    Thus there does not exist an $\e>0$ such that $B\rr{x,\e}\cap\bb{0,1}\subseteq C$
    so $x$ is not an interior point. Contradiction. We conclude that there are no interior
    points $x\in C$.
  \end{proof}
\end{claim*}

\begin{claim*}[2d]
  Consider the Cantor space $C$. For every $x\in C$ and every open subset $U\subseteq C$
  with $x\in U$, there exists a subset $V\subseteq U$ that is both closed and open in $C$
  with $x\in V$.
  \begin{proof}
    Let $x\in C$ and let $U\subseteq C$ such that $x\in U$. We note that $C\subseteq\bb{0,1}$
    is a subset so there exists a set, open in $\bb{0,1}$, $U'\subseteq\bb{0,1}$ such that
    $U=U'\cap C$. In particular, $x\in U'$ so there exists an $r>0$ such that $B\rr{x,r}\subseteq U'$.

    We choose $N\in\N$ such that $3^{-N-1}<r$. Then note, for any $y,z\in C$ such that
    $y_n=z_n$ whenever $n\leq N$,
    \begin{align*}
      \abs{y-z} &= \abs{\sum_{n=1}^{\infty} \frac{y_n}{3^n} - \sum_{n=1}^{\infty} \frac{z_n}{3^n}} \\
                &\leq \sum_{n=N+1}^{\infty} \frac{\abs{y_n-z_n}}{3^n} \\
                &\leq \sum_{n=N+1}^{\infty} \frac{2}{3^n} \\
                &\leq 3^{-N-1} \\
                &<r.
    \end{align*}
    Let $V\subseteq C$ be the set of $v'\in C$ for which $x_n=v_n$ whenever $n\leq N$.
    Then $V\subseteq B\rr{x,r}$ and thereby $V\subseteq U$. Consider once again
    \begin{align*}
      c=\sum_{n=1}^{N} \frac{x_n}{3^n} + \sum_{n=N+1}^{\infty} \frac{1}{3^n} \in \bb{0,1}.
    \end{align*}
    We noted in (\ref{middle}) that, for all $v\in V$,
    \begin{align*}
      \abs{v-c}<3^{-N}.
    \end{align*}
    Therefore $V\subseteq B\rr{c,3^{-N}}$. Further, let $y\in C\setminus V$. Then
    $x_k\neq y_k$ for some $k<N$. Thus
    \begin{align*}
      \abs{c-y} &= \abs{\sum_{n=1}^{N} \frac{x_n}{3^n}+\sum_{n=N+1}^{\infty} \frac{1}{3^n} - \sum_{n=1}^{\infty} y_n}
      \geq \frac{2}{3^k} > \frac{1}{3^k} > 3^{-N}.
    \end{align*}
    Therefore $V=C\cap B\rr{c,3^{-N}}$. As $B\rr{c,3^{-N}}$ is open in $\bb{0,1}$, $V$ must be open in $C$.
    By similar reasoning, $V=C\cap B\bb{c,3^{-N}}$ where $B\bb{c,3^{-N}}$ is the closed ball with radius $3^{-N}$
    centred at $c$. Thus $V$ is closed in $C$.
  \end{proof}
\end{claim*}

\end{document}
