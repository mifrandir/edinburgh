\documentclass{article}
\usepackage{homework-preamble}

\begin{document}
\title{Algebraic Geometry: Homework 1}
\author{Franz Miltz (UUN: S1971811)}
\date{1 February 2023}
\maketitle

\section*{Exercise 1}

\begin{claim*}
  Let $\Sigma$ be a set of 8 distinct points in $\mathbf P^2$. Let $n$ be the largest number
  of points in $\Sigma$ that lie on a line. If $n\in\cc{5,6,7}$ then there is a line
  $L\subseteq\mathbf P^2$ that contains exactly two points in $\Sigma$.
  \begin{proof}
    Let $\Sigma=\cc{x_1,\ldots,x_8}$ and assume without loss of generality that
    $x_1,\ldots,x_n$ are collinear and let $L_0\subseteq\mathbf P^2$ be the line such
    that $x_1,\ldots,x_n\in L_0$. This line is uniquely determined by $x_1,x_2\in L_0$.
    Similarly, for $j<8$, let $L_j$ be the unique line passing through $x_j$ and $x_8$.

    Assume $n=7$. If, for any $1<j<8$, $x_j\in L_1$ then $L_1=L_0$ and $x_8\in L_0$.
    This contradicts $n=7$. Thus $L_1$ contains exactly two elements of $\Sigma$.

    Assume $n=6$. If $x_7\not\in L_1$ then we are done. If $x_7\in L_1$ then consider
    $L_2$. We have $x_7\not\in L_2$ as otherwise $L_1=L_2$, thus $x_1,x_2\in L_1$, and finally
    $L_1=L_2=L_0$, once again contradicting $n=6$. Thus $x_j\in L_2$ if and only if
    $j=1$ or $j=8$.

    Assume $n=5$. If $x_6,x_7\not\in L_1$ then we are done. Otherwise, assume without loss
    of generality $x_7\in L_1$. By the same argument as above, $x_7\not\in L_j$ for $j\neq 1$.
    If $x_6\not\in L_2$ then we are done. Otherwise we have $x_6,x_7\not\in L_3$
    so $L_3$ is the required line.
  \end{proof}
\end{claim*}

\section*{Exercise 2}

Let $f\rr{x,y,z}=xy^3+yz^3+x^3z$ and consider the quartic
$C\subseteq\mathbf P^2_{\C}$ given by $f\rr{x,y,z}=0$.

\begin{claim*}[1]
  The gradient vanishes at no point on $C$ and, moreover, $f$ is irreducible.
  \begin{proof}
    We have the system of equations
    \begin{align*}
      y^3 + 3x^2z &= 0\\
      z^3 + 3xy^2 &= 0\\
      x^3 + 3yz^2 &= 0\\
      xy^3 + yz^3 + x^3z &= 0
    \end{align*}
    Assume there is a solution $\bb{x_0:y_0:z_0}$. As this system is symmetric in all variables
    and all equations are homogeneous, we assume without loss of generality $z_0=1$.
    Thus we obtain the equations
    \begin{align*}
      y_0^3 + 3x_0^2 &= 0\\
      1 + 3x_0y_0^2 &= 0\\
      x_0^3 + 3y_0 &= 0\\
      x_0y_0^3 + y_0 + x_0^3 &= 0
    \end{align*}
    Using the first and third equation, we perform the substitutions $y_0=-x_0^3/3$
    and $y_0^3=-3x^2$ to obtain the equation
    \begin{align*}
      x_0(-3x_0^2) - \frac{1}{3}x_0^3 + x^3 = 0
    \end{align*}
    which implies $x_0=0$. However, this contradicts the second equation
    $1 + 3x_0y_0^2 = 0$. Thus there is no solution for the system.

    Now assume there are homogeneous polynomials $g,h\in\C\bb{x,y,z}$ with
    $\deg g,\deg h\geq 1$ such that $f\rr{x,y,z}=g\rr{x,y,z}h\rr{x,y,z}$.
    By Bezout's theorem, the system
    \begin{align*}
      g\rr{x,y,z}=h\rr{x,y,z}=0
    \end{align*}
    has at least one solution $P$. Note that $f\rr{P}=0$ so $P\in C$. Moreoever,
    \begin{align*}
      \eval{\frac{\partial f}{\partial x}}{P}
      =\frac{\partial g}{\partial x}\rr{P}h\rr{P}
      +\frac{\partial h}{\partial x}\rr{P}g\rr{P}
      = 0
    \end{align*}
    so the gradient of $f$ must vanish at $P$. This contradicts the developments above
    so $f$ must be irreducible.
  \end{proof}
\end{claim*}

\begin{claim*}[2]
  The conic $C$ intersects its tangent at $P=\bb{0:0:1}$ in $P$ with multiplicity $3$
  and in $Q=\bb{1:0:0}$ with multiplicity $1$.
  \begin{proof}
    The tangent line at $\bb{0:0:1}$ is given by the equation
    \begin{align*}
      g\rr{x,y,z}=
      \frac{\partial f}{\partial x}\rr{0,0,1}x
      +\frac{\partial f}{\partial y}\rr{0,0,1}y
      +\frac{\partial f}{\partial z}\rr{0,0,1}z
      = y
      =0.
    \end{align*}
    Thus the intersection points are given by the solutions to the system $x^3z=y=0$.
    This leaves us with two solutions $\bb{1:0:0}$ and $\bb{0:0:1}$.

    Let $f'\rr{x,y}=f\rr{x,y,1}$ and $g'\rr{x,y}=g\rr{x,y,1}$. The intersection multiplicity
    of $C$ and $L$ at $P=\bb{0:0:1}$ is
    \begin{align*}
      I_P\rr{C,L}
      &=\dim\rr{\frac{\C\bb{\bb{x,y}}}{\aa{f',g'}}}\\
      &=\dim\rr{\frac{\C\bb{\bb{x,y}}}{\aa{xy^3+y+x^3,y}}}\\
      &=\dim\rr{\frac{\C\bb{\bb{x,y}}/\aa{y}}{\aa{xy^3+y+x^3,y}/\aa{y}}}\\
      &=\dim\rr{\frac{\C\bb{\bb{x}}}{\aa{x^3}}}\\
      &= 3.
    \end{align*}
    To compute the intersection multiplicity at $Q=\bb{1:0:0}$ we apply the
    linear change of coordinates $\rr{x,y,z}\mapsto\rr{z,y,x}$. We then obtain
    the polynomials $f'\rr{x,y}=f\rr{1,y,x}$ and $g'\rr{x,y}=g\rr{1,y,x}$.
    Now
    \begin{align*}
      I_Q\rr{C,L}
    &=\dim\rr{\frac{\C\bb{\bb{x,y}}}{\aa{y^3+yx^3+x,y}}} \\
    &=\dim\rr{\frac{\C\bb{\bb{x,y}}/\aa{y}}{\aa{y^3+yx^3+x,y}/\aa{y}}} \\
    &=\dim\rr{\frac{\C\bb{\bb{x}}}{\aa{x}}} \\
    &= 1
    \end{align*}
    Thus $I_P\rr{C,L}+I_Q\rr{C,L}=4$, as predicted by Bezout's theorem.
  \end{proof}
\end{claim*}

\section*{Exercise 3}

Let $f\rr{x,y,z}=x^2+4xy-6xz+14y^2-28yz+35z^2$ and consider the conic
$C\subseteq \mathbf P^2_{\C}$ given by $f\rr{x,y,z}=0$.

\begin{claim*}[1]
  Let $B$ be the matrix of the symmetric bilinear form associated with $f$.
  Then $\det B = 196$.
  \begin{proof}
    We find
    \begin{align*}
      B=
      \begin{pmatrix}
        1 & 2 & -3 \\
        2 & 14 & -14 \\
        -3 & -14 & 35
      \end{pmatrix}.
    \end{align*}
    Direct computation then yields
    \begin{align*}
      \det B = 14 \times 35-14^2
      - 2\times\rr{2\times 35-3\times 14}
      - 3\times\rr{-2\times 14+3\times 14}
      =196.
    \end{align*}
  \end{proof}
\end{claim*}

\begin{claim*}[2]
  The linear change of coordinates
  \begin{align*}
    \begin{pmatrix}
      x \\ y \\ z
    \end{pmatrix}
    =
    \begin{pmatrix}
      1 & 1/\sqrt{20} & 21/\sqrt{980}\\
      0 & 1/\sqrt{20} & -9/\sqrt{980}\\
      0 & 1/\sqrt{20} & 1/\sqrt{980}
    \end{pmatrix}
    \begin{pmatrix}
      x' \\ y' \\ z'
    \end{pmatrix}
  \end{align*}
  gives a projective transformation that takes $C$ to $\rr{x'}^2+\rr{y'}^2+\rr{z'}^2=0$.
  \begin{proof}
    We include the full derivation analogously to Example 1.48 in the notes.
    Consider the symmetric bilinear form $\aa{-,-}$ associated with $f$.
    We choose $v_1=\rr{1,0,0}$ and note
    \begin{align*}
      \aa{v_1,v_1}=\begin{pmatrix}
        1 & 0 & 0
        \end{pmatrix}B\begin{pmatrix}
        1 & 0 & 0
      \end{pmatrix}^\top = 1.
    \end{align*}
    A vector $\rr{x,y,z}$ is orthogonal to $v_1$ if $x+2y-3z=0$.
    We choose $v_2=1/\sqrt{20}\rr{1,1,1}$ to obtain $\aa{v_2,v_2}=1$.
    Finally, a vector $\rr{x,y,z}$ is orthogonal to $v_1$ and $v_2$ if
    \begin{align*}
      x+2y-3z  &= 0,\\
      2y+18z &= 0.
    \end{align*}
    A solution is $\rr{21,-9,1}$. We normalise and choose $v_3=1/\sqrt{980}\rr{21,-9,1}$.
    We thus obtain the matrix
    \begin{align*}
      A=
      \begin{pmatrix}
        1 & 1/\sqrt{20} & 21/\sqrt{980}\\
        0 & 1/\sqrt{20} & -9/\sqrt{980}\\
        0 & 1/\sqrt{20} & 1/\sqrt{980}
      \end{pmatrix}.
    \end{align*}
  We verify by direct computation that $A^\top BA = I$.   \end{proof}
\end{claim*}

\end{document}
