\documentclass{article}
\usepackage{notes-preamble}
\usepackage{enumitem}
\usepackage{multicol}
\mkthmstwounified

\title{Automated Reasoning (SEM5)}
\author{Franz Miltz}
\begin{document}
\maketitle
\tableofcontents
\pagebreak
\section{Introduction}
\begin{definition}
	\emph{Automated Reasoning} refers to reasoning in a computer using logic.
	\begin{itemize}
		\item active area of research since the 50s
		\item part of artificial intelligence
	\end{itemize}
\end{definition}
\begin{theorem}
	Formalised mathematics is neither
	\begin{itemize}
		\item \emph{complete} (see G\"odel's Incompleteness Theorems), nor
		\item \emph{decidable} (see Church and Turing).
	\end{itemize}
\end{theorem}

\section{Propositional Logic}

\begin{definition}
	A syntactically correct formula is called a \emph{well-formed formula}.

	Given an alphabet of propositional symbols $\mathcal{L}$, the
	set of wffs is the smallest set such that
	\begin{itemize}
		\item any symbol $A\in\mathcal{L}$ is a wff;
		\item if $P$ and $Q$ are wffs, so are $\neg P, P\vee Q, P\wedge Q, P\rightarrow Q$, and $P\leftrightarrow Q$;
		\item if $P$ is a wff, then $(P)$ is a wff.
	\end{itemize}
\end{definition}

\begin{definition}
	An \emph{interpretation} $\mathcal{I}$ is a truth assignment
	to the symbols in the alphabet $\mathcal{L}$, i.e. a function
	\begin{align*}
		V:\mathcal{L}\to \{\top,\bot\}.
	\end{align*}
\end{definition}

\begin{definition}
	An interpretation $\mathcal{I}$ \emph{satisfies} a wff $P$ if $[\![ P]\!]_\mathcal{I}=\top$.
\end{definition}

\begin{definition}
	A wff is \emph{satisfiable} if there exists an interpretation that satisfies it.
	Otherwise it's unsatisfiable.
\end{definition}

\begin{definition}
	A wff is \emph{valid} if every interpretation satisfies it.
\end{definition}

\begin{definition}
	The wffs $P_1, P_2,...,P_n$ \emph{entail} $Q$ if any interpretation
	which satisfies all of $P_1, P_2,...,P_n$ also satisfies $A$.
	We write
	\begin{align*}
		P_1, P_2, ..., P_n \vDash Q.
	\end{align*}
\end{definition}

\subsection{Natural deduction}


\begin{definition}
	\emph{Natural deduction} uses the following rules:
	\begin{align*}
		\begin{array}{c c c}
			\infer[(\text{conjI})]{P\wedge Q}{P\hs Q}                     &
			\infer[(\text{conjunct1})]{P}{P\wedge Q}                      &
			\infer[(\text{conjunct2})]{Q}{P\wedge Q}                                   \\
			\infer[(\text{disjI1})]{P\vee Q}{P}                           &
			\infer[(\text{disjI2})]{P\vee Q}{Q}                           &
			\infer[(\text{disjE})]{R}{P\vee Q \hs \begin{array}[b]{c}
					                                      [P] \\ \vdots \\ R
				                                      \end{array}
			\hs \begin{array}[b]{c}
					    [Q] \\ \vdots \\ R
				    \end{array}}                                                     \\
			\infer[(\text{impI})]{P\to Q}{\begin{array}[b]{c}
					                              [P] \\\vdots\\Q
				                              \end{array}}             &
			\infer[(\text{impE})]{R}{P\to Q\hs P \hs \begin{array}[b]{c}
					                                         [Q] \\\vdots\\R
				                                         \end{array}}  &
			\infer[(\text{mp})]{Q}{P\to Q\hs P}                                        \\
			\infer[(\text{iffI})]{P\leftrightarrow Q}{\begin{array}[b]{c}
					                                          [Q] \\\vdots\\P
				                                          \end{array}\hs\begin{array}[b]{c}
					                                                        [P] \\\vdots\\Q
				                                                        \end{array}} &
			\infer[(\text{iffD1})]{Q}{P\leftrightarrow Q\hs P}            &
			\infer[(\text{iffD2})]{P}{P\leftrightarrow Q\hs Q}                         \\
			\infer[(\text{notI})]{\neg P}{\begin{array}[b]{c}
					                              P \\ \vdots \\ \bot
				                              \end{array}}             &
			\infer[(\text{notE})]{\bot}{P\hs\neg P}                       &            \\
			\infer[(\text{excluded\_middle})]{\neg P \vee P}{}            &
			\infer[\text{(ccontr)}]{P}{\begin{array}[b]{c}
					                           [\neg P] \\ \vdots \\ \bot
				                           \end{array}}
		\end{array}
	\end{align*}
\end{definition}

\begin{theorem}[Soundness]
	If $Q$ is provable from assumptions $P_1,...,P_2$ then $P_1,...,P_n\vDash Q$.
\end{theorem}
\begin{theorem}[Completeness]
	If $P_1,...,P_n\vDash Q$ then $Q$ is provable from assumptions $P_1,...,P_2$.
\end{theorem}

\subsection{Propositional Reasoning in Isabelle}

\newcommand{\db}[1]{[\![#1]\!]}

\begin{definition}
	Applying \texttt{rule someRule} where
	\begin{align*}
		\texttt{someRule} : \db{P_1;...,;P_m} \Rightarrow Q
	\end{align*}
	to the goal
	\begin{align*}
		\db{A_1; ...; A_n} \Rightarrow C
	\end{align*}
	where $Q$ and $C$ can be \emph{unified}, we generate the goals
	\begin{align*}
		\db{A'_1; ...; A'_n} & \Rightarrow P'_1 \\
		                     & \vdots           \\
		\db{A'_1; ...; A'_n} & \Rightarrow P'_m
	\end{align*}
	where \begin{align*}
		A'_1, ..., A'_n,P'_1,...,P'_m
	\end{align*}
	are the results of applying the
	substitution which unfies $Q$ and $C$ to
	\begin{align*}
		A_1,...,A_n,P_1,...,P_m.
	\end{align*}
\end{definition}

\begin{definition}
	Applying \texttt{erule someRule} where
	\begin{align*}
		\texttt{someRule} : \db{P_1;...,;P_m} \Rightarrow Q
	\end{align*}
	to the goal
	\begin{align*}
		\db{A_1; ...; A_n} \Rightarrow C
	\end{align*}
	where $Q$ and $C$ are unifiable and $P_1$ and $A_1$ are unifiable, we generate the goals
	\begin{align*}
		\db{A'_2; ...; A'_n} & \Rightarrow P'_2 \\
		                     & \vdots           \\
		\db{A'_2; ...; A'_n} & \Rightarrow P'_m
	\end{align*}
	where \begin{align*}
		A'_2, ..., A'_n,P'_2,...,P'_m
	\end{align*}
	are the results of applying the
	substitution which unfies $Q$ to $C$ and $P_1$ to $A_1$ to
	\begin{align*}
		A_2,...,A_n,P_2,...,P_m.
	\end{align*}
\end{definition}

\begin{definition}
	Applying \texttt{drule someRule} where
	\begin{align*}
		\texttt{someRule} : \db{P_1;...,;P_m} \Rightarrow Q
	\end{align*}
	to the goal
	\begin{align*}
		\db{A_1; ...; A_n} \Rightarrow C
	\end{align*}
	where $P_1$ and $A_1$ can be \emph{unified}, we generate the goals
	\begin{align*}
		\db{A'_2; ...; A'_n}    & \Rightarrow P'_2 \\
		                        & \vdots           \\
		\db{A'_2; ...; A'_n}    & \Rightarrow P'_m \\
		\db{Q',A'_2; ...; A'_n} & \Rightarrow C'
	\end{align*}
	where \begin{align*}
		A'_2, ..., A'_n,P'_2,...,P'_m,Q',C'
	\end{align*}
	are the results of applying the
	substitution which unfies $P_1$ and $A_1$ to
	\begin{align*}
		A_2,...,A_n,P_2,...,P_m,Q,C.
	\end{align*}
\end{definition}
\begin{definition}
	Applying \texttt{frule someRule} where
	\begin{align*}
		\texttt{someRule} : \db{P_1;...,;P_m} \Rightarrow Q
	\end{align*}
	to the goal
	\begin{align*}
		\db{A_1; ...; A_n} \Rightarrow C
	\end{align*}
	where $P_1$ and $A_1$ can be \emph{unified}, we generate the goals
	\begin{align*}
		\db{A'_1; ...; A'_n}    & \Rightarrow P'_1 \\
		                        & \vdots           \\
		\db{A'_1; ...; A'_n}    & \Rightarrow P'_m \\
		\db{Q',A'_1; ...; A'_n} & \Rightarrow C'
	\end{align*}
	where \begin{align*}
		A'_1, ..., A'_n,P'_1,...,P'_m,Q',C'
	\end{align*}
	are the results of applying the
	substitution which unfies $P_1$ and $A_1$ to
	\begin{align*}
		A_1,...,A_n,P_1,...,P_m,Q,C.
	\end{align*}
\end{definition}

\begin{definition}
	Applying \texttt{cut\_tac lemmaName} adds the conclusion of \texttt{lemmaName} as a
	new assumption, and its assumptions as new subgoals.
\end{definition}

\begin{definition}
	Applying \texttt{subgoal\_tac $P$} adds $P$ as a new assumption, and introduces $P$
	as a new subgoal.
\end{definition}

\begin{theorem}
	Propositional logic in Isabelle is sound.
\end{theorem}
\begin{proof}
	By the following properties.
	\begin{enumerate}
		\item Every proof is broken down into primitive rules applications which are checked
		      by a small piece of hand-verified code. These rules are sound.
		\item New concepts are introduced by definition rather than axiomatisation. New
		      definitions cannot introduce unsoundness.
	\end{enumerate}
\end{proof}

\section{First-Order Logic}

\subsection{Syntax}

\begin{definition}
	Let
	\begin{itemize}
		\item $\mathcal{V}$ a countable set of variables, and
		\item $\mathcal{F}$ an at most countable set of function letters each assigned a unique arity.
	\end{itemize}
	Then the set of \emph{well-formed terms} is the smallest set such that
	\begin{itemize}
		\item any variable $v\in \mathcal{V}$ is a term, and
		\item if $f\in \mathcal{F}$ has arity $n$, and $t_1,...,t_n$ are terms, so is $f(t_1, ..., t_n)$.
	\end{itemize}
\end{definition}

\begin{definition}
	Let $P$ an at most countable set of predicates, each assigned a unique arity,
	then the set of \emph{well-formed formulas} is the smallest set such that
	\begin{itemize}
		\item if $A\in \mathcal{P}$ has arity $n$, and $t_1,...,t_n$ are terms, then $A(t_1, ..., t_n)$ is a wff,
		\item if $P$ and $Q$ are wffs, so are $\neg P,P\vee Q,P\wedge Q,P\rightarrow Q,P\leftrightarrow Q$,
		\item if $P$ is a wff, so are $\exists x. P$ and $\forall x. P$ for any $x\in \mathcal{V}$, and
		\item if $P$ is a wff, then $(P)$ is a wff.
	\end{itemize}
\end{definition}

\begin{definition}
	A variable occurence of $x$ is \emph{in the scope of} a quantifier
	occurence $\forall x$ or $\exists x$ if the quantifier occurence is
	the first occurence of the quantifier over $x$ in a traversal from the
	variable occurence position to the root of the formula tree.

	An occurence of a variable $x$ in a formula $P$ is \emph{bound} if it
	is in the scope of a $\forall x$ or $\exists x$ quantifier.
\end{definition}

\begin{definition}
	If $P$ is a formula, $s$ is a term and $x$ is a variable, then
	\begin{align*}
		P[s/x]
	\end{align*}
	is the formula obtained by \emph{substituting $s$ for all free occurences of $x$}
	throughout $P$.
\end{definition}

\subsection{Semantics}

\begin{definition}[Interpretation]
	An \emph{interpretation} $\mathcal{I}$ consists of a non-empty set $\mathcal{D}$ of concrete values
	called the \emph{domain} together with the following mappings:
	\begin{itemize}
		\item 0-ary predicate symbols $P\in\mathcal{P}$ are mapped to $P^\mathcal{I}\in\{\bot, \top\}$,
		\item $n$-ary predicate symbols $P\in\mathcal{P}$ are mapped to predicates $P^\mathcal{I}\subseteq \mathcal{D}^n$, and
		\item $n$-ary function symbols $f\in\mathcal{F}$ are mapped to functions $f^\mathcal{I}:\mathcal{D}^n\to\mathcal{D}$.
	\end{itemize}
\end{definition}

\begin{definition}[Assignment]
	Given an interpretation $\mathcal{I}$, an \emph{assignment $s$} assigns a value
	from the domain $\mathcal{D}$ to each variable $\mathcal{V}$ i.e. $s:\mathcal{V}\to\mathcal{D}$.
	We extend this assignment $s$ to all terms inductively by saying that
	\begin{enumerate}
		\item if $\mathcal{I}$ maps the $n$-ary function letter $f$ to the function $f^\mathcal{I}$, and
		\item if terms $t_1, ..., t_n$ have been assigned concrete values $a_1,...,a_n\in D$
	\end{enumerate}
	then we can assign value $f^\mathcal{I}(a_1,...,a_n)\in\mathcal{D}$ to the term
	$f(t_1,...,t_n)$.
\end{definition}

\begin{definition}[Satisfaction]
	Given an interpretation $\mathcal{I}$ and an assignment $s:\mathcal{V}\to\mathcal{D}$
	\begin{enumerate}
		\item any wff which is a nullary predicate letter $A$ is satisfied iff $A^\mathcal{I}=\top$,
		\item suppose we have a wff $P$ of the form $A(t_1,..., t_n)$, where $A$ is
		      interpreted as relation $A^\mathcal{I}$ and $t_1,...,t_n$ have been assigned
		      concrete values $a_1,...,a_n$ by $s$. Then $P$ is satisfied iff \begin{align*}
			      (a_1,...,a_n)\in A^\mathcal{I},
		      \end{align*}
		\item any wff of the form $\forall x.P$ is satisfied iff $P$ is satisfied with
		      with respect to the assignment $s[x\mapsto a]$ for all $a\in D$,
		\item any wff of the form $\exists x.P$ is satisfied iff $P$ is satisfied with
		      with respect to the assignment $s[x\mapsto a]$ for some $a\in D$,
		\item any wffs of the form $P\vee Q$, $P\wedge Q$, $P\rightarrow Q$, $P\leftrightarrow Q$,
		      $\neg P$ are satisfied according to the truth-tables for each connective.
	\end{enumerate}
\end{definition}

\begin{definition}[Satisfiability, validity]
	A wff is \emph{satisfiable} in an interpretation iff there exists an assignment that satisfies it.
	A wff is \emph{valid} in an interpretation iff every assignment satisfies it.
\end{definition}

\begin{definition}[Entailment]
	We write $\mathcal{I}\vDash_s P$ to mean that wff $P$ is satisfied by interpretation $\mathcal{I}$ and
	assignment $s$.  We say that the wffs $P_1,...,P_n$ \emph{entail} wff $Q$ and write
	\begin{align*}
		P_1,...,P_n\vDash Q
	\end{align*}
	if, for any interpretation $\mathcal{I}$ and assignment $s$ for which $\mathcal{I}\vDash_s P^i$ for all $i$,
	we also have $\mathcal{I}\vDash_s Q$.
\end{definition}

\subsection{Deduction}

\begin{definition}
	The introduction rules for quantifiers are:
	\begin{itemize}
		\item Universal quantification, provided that $x_0$ is not free in the assumptions, \begin{align*}
			      \infer[(\text{allI})]{\forall x. P}{P[x_0/x]}
		      \end{align*}
		\item Existential qunatification \begin{align*}
			      \infer[(\text{exI})]{\exists x.P}{P[t/x]}
		      \end{align*}
	\end{itemize}
\end{definition}

\begin{definition}
	The elimination rules for quantifiers are:
	\begin{itemize}
		\item Existential elimination: \begin{align*}
			      \infer[(\text{exE})]{Q}{\exists x.P\hs \begin{array}[b]{c}
					                                             [P[x_0/x]] \\
					                                             \vdots     \\
					                                             Q
				                                             \end{array}}
		      \end{align*}
		\item Specialisation: \begin{align*}
			      \infer[(\text{spec})]{P[t/x]}{\forall x.P}
		      \end{align*}
		\item Universal elimination: \begin{align*}
			      \infer[(\text{allE})]{Q}{\forall x.P\hs \begin{array}[b]{c}
					                                              [P[t/x]] \\
					                                              \vdots   \\
					                                              Q
				                                              \end{array}}
		      \end{align*}
	\end{itemize}
\end{definition}

\section{Higher-Order Logic}

\begin{definition}
	A \emph{lambda abstraction} is a term that denotes a function directly by the
	rules which define it.
\end{definition}

\begin{definition}
	The basic types in Isabelle are \emph{bool}, \emph{ind} and $\alpha\Rightarrow\beta$.
	The two primitive families of functions are:
	\begin{align*}
		 & (=_\alpha): \alpha\Rightarrow \alpha \Rightarrow \textit{bool}                \\
		 & (\rightarrow): \textit{bool} \Rightarrow\textit{bool}\Rightarrow\textit{bool}
	\end{align*}
\end{definition}

\begin{definition}
	The following representations and definitions are used:
	\begin{itemize}
		\item Formulas are terms of type \textit{bool}.
		\item Predicates are functions of type $\alpha\Rightarrow\textit{bool}$.
		\item Sets are represented as predicates.
		\item True is defined as $\top \equiv (\lambda x.x)=(\lambda x.x)$.
		\item Universal quantification as function equality: \begin{align*}
			      \forall x. \phi \equiv (\lambda x. \phi) = (\lambda x. \top).
		      \end{align*}
		\item Conjunction and disjunction are defined as: \begin{align*}
			      P\wedge Q & \equiv \forall R. (P\rightarrow Q \rightarrow R) \rightarrow R               \\
			      P\vee Q   & \equiv \forall R. (P\rightarrow R)\rightarrow (Q\rightarrow R) \rightarrow R
		      \end{align*}
	\end{itemize}
\end{definition}

\section{Structured Proofs}

Isabelle's procedural proofs are
\begin{itemize}
	\item difficult to read,
	\item hard to maintain, and
	\item do not scale.
\end{itemize}
Yet they remain useful for proof exploration!

\begin{definition}[Lemma]
	A lemma can be declared as follows:
	\begin{minted}{isabelle}
lemma lemma_name:
  fixes x1 :: t1 and x2 :: b2 and ...
  assumes a1: P1 and a2: P2 and ... 
  shows R
	\end{minted}
\end{definition}

\begin{definition}
	The following proof patterns may be used:
	\begin{center}
		\begin{tabular}{l | l | l }
			\textbf{Case split}                 &
			\textbf{Contradiction}              &
			\textbf{Iff}                          \\
			\hline
			\begin{minipage}[t]{0.25\textwidth}
				\begin{minted}{isabelle}
have "P \<or> Q"
then show "R"
proof
  assume "P"
  ...
  show "R"
next
  assume "Q"
  ... 
  show "R"
qed
				\end{minted}
			\end{minipage} &
			\begin{minipage}[t]{0.25\textwidth}
				\begin{minted}{isabelle}
show "P"
proof (rule ccontr)
  assume "~P"
  ...
  show "False"
qed
				\end{minted}
			\end{minipage} &
			\begin{minipage}[t]{0.25\textwidth}
				\begin{minted}{isabelle}
show "P <--> Q"
proof
  assume "P"
  ... 
  show "Q"
next
  assume "Q"
  ... 
  show "P"
qed
					\end{minted}
			\end{minipage}
		\end{tabular}
	\end{center}
\end{definition}

\begin{definition}
	To proof existentially or universally quantified statements
	use the following proof patterns:\par
	\begin{center}
		\begin{tabular}{l | l}
			\textbf{$\forall$ introduction}    &
			\textbf{$\exists$ introduction}      \\
			\hline
			\begin{minipage}[t]{0.4\textwidth}
				\begin{minted}{isabelle}
show "\<forall> x. P x"
proof
  fix x 
  show "P x" ...
qed
				\end{minted}
			\end{minipage} &
			\begin{minipage}[t]{0.4\textwidth}
				\begin{minted}{isabelle}
show "\<exists> x. P x"
proof
  ...
  show "P witness" ..
qed
				\end{minted}
			\end{minipage}
		\end{tabular}
	\end{center}
\end{definition}

\begin{definition}
	To eliminate an extential qunatification use
	\begin{minted}{isabelle}
have "\<exists> x. P x"
then obtain "x" where p: "P x" by blast
...	
	\end{minted}
\end{definition}

\section{Unification}

\subsection{Substitutions}

\begin{example}
	\begin{align*}
		(s(X)+Y)[0/X,s(0)/Y]\equiv (s(0)+s(0))
	\end{align*}
\end{example}

\begin{definition}
	We say two subsitutions $\phi$, $\psi$ are equal if for all variables $x$,
	\begin{align*}
		x[\phi] = x[\psi].
	\end{align*}
\end{definition}

\begin{definition}
	To combine subsitutions $\phi$ and $\psi$
	\begin{enumerate}
		\item Replace each pair $s/X$ in $\phi$ by $s\psi/X$ to form $\phi'$.
		\item Delete from $\phi$ each pair $t_1/Y$ such that $\phi'$ contains a pair $s/Y$ to form $\psi'$.
		\item Output the union of $\phi'$ and $\psi'$.
	\end{enumerate}
\end{definition}

\begin{definition}
	If $\phi$ and $\psi$ are subsitutions then their \emph{composition $\phi\circ\psi$}
	is also a subsitution which, for any term $t$, satisfies the following property:
	\begin{align*}
		t[\phi\circ\psi] \equiv (t[\phi])[\psi].
	\end{align*}
\end{definition}

\begin{definition}
	Given any two terms $s$ and $t$, a subsitution $\phi$ is their \emph{most general unifier}
	if
	\begin{align*}
		s[\phi]\equiv t[\phi] \hs\text{and}\hs
		\forall \psi.\: s[\psi] \equiv t[\psi]  \Rightarrow \exists \theta.\: \psi = \phi \circ \theta.
	\end{align*}
\end{definition}

\begin{theorem}
	The set of all subsitutions together with the operation $\circ$ forms a monoid. In particular,
	for any substitutions $\phi,\psi,\theta$ and any variable $x$, we have
	\begin{itemize}
		\item \emph{associativity}: $(\phi \circ \psi) \circ \theta = \phi \circ (\psi \circ \theta)$,
		\item \emph{unit}: $\phi\circ[] = \phi = []\circ\phi$.
	\end{itemize}
\end{theorem}

\begin{theorem}
	The \emph{most general unifier} is unique up to alphabetic variance.
\end{theorem}

\subsection{Matching}

\begin{definition}
	\emph{Matching} is the process of finding an instance of a pattern that is equal to another term.

	Problem: Given two expressions, \emph{pattern} $s$ and \emph{target} $t$, find a subsitution
	\begin{align*}
		s[\phi] \equiv t
	\end{align*}
	where $\equiv$ means that the terms are identical.
\end{definition}

\begin{definition}
	A \emph{disagreement pair} of two expressions $s$ and $t$ is a pair of subexpressions
	which disagree.
\end{definition}

\paragraph{Matching Algorithm}

To match pattern and target given substitution $\phi$:

\begin{enumerate}
	\item If pattern and target are identical then succeed with output $\phi$.
	\item Otherwise, let $\lra{t_1, t_2}$ be the first disagreement pair.
	\item If $t_1$ is a variable then call match on pattern $\{t_2/t_1\}$ and target given $\phi\{t_2/t_1\}$.
	\item Else fail.
\end{enumerate}

\paragraph*{Matching rules}
$s$ and $t$ are arbitrary terms, standardised apart.
\begin{center}
	\begin{tabular}{c | c | c | c}
		\textbf{Name} & \textbf{Before}                        & \textbf{After}                        & \textbf{Condition}     \\\hline
		Decompose     & $P\wedge (f(\vec s) \equiv f(\vec t))$ & $P\wedge\bigwedge_i (s_i \equiv t_i)$ &                        \\\hline
		Conflict      & $P\wedge (f(\vec s) \equiv g(\vec t))$ & fail                                  & $f\not\equiv g$        \\\hline
		Switch        & $P\wedge (s\equiv X)$                  & $P\wedge(X\equiv s)$                  & $X\in V$, $s\not\in V$ \\\hline
		Delete        & $P\wedge (s\equiv s)$                  & P                                     &
	\end{tabular}
\end{center}

\subsection{General unification}

\begin{definition}
	\emph{Unification} is the process of finding a common instance of two terms.

	Problem: Given two expressions, $s$ and $t$, find subsitutions
	$\phi$ and $\psi$ such that
	\begin{align*}
		s[\phi] \equiv t[\psi]
	\end{align*}
	where $\equiv$ means that the terms are identical.
\end{definition}

\paragraph{Unification algorithm}

To generally unify $S$ given $\phi$
\begin{enumerate}
	\item If $\abs{S}=1$, succeed and output $\phi$.
	\item Otherwise, let $D$ be the first disagreement set of $S$.
	\item If $D$ contains a variable $V$ and a term $t_1$ and $V$
	      does not occur in $t_1$ then generally unify $S[\{t_1/V\}]$ given $\phi[\{t_1/V\}]$.
	\item Else fail.
\end{enumerate}

\begin{definition}
	We have the following types of unification:
	\begin{enumerate}
		\item \emph{unitary}: A single unique mgu, or none. (predicate logic)
		\item \emph{finitary}: Finite number of mgus. (predicate logic with commutativity)
		\item \emph{infinitary}: Possibly infinite number of mgus. (predicate logic with associativity)
		\item \emph{nullary}: No mgus exist, although unifiers may exist.
		\item \emph{undecidable}: Unification is not decidable.
	\end{enumerate}
\end{definition}

\paragraph*{Unification rules}
$s$ and $t$ are arbitrary terms, standardised apart and $V=V(s)\cup V(t)$.
\begin{center}
	\begin{tabular}{c | c | c | c}
		\textbf{Name} & \textbf{Before}                        & \textbf{After}                        & \textbf{Condition}                         \\\hline
		Decompose     & $P\wedge (f(\vec s) \equiv f(\vec t))$ & $P\wedge\bigwedge_i (s_i \equiv t_i)$ &                                            \\\hline
		Conflict      & $P\wedge (f(\vec s) \equiv g(\vec t))$ & fail                                  & $f\not\equiv g$                            \\\hline
		Switch        & $P\wedge (s\equiv X)$                  & $P\wedge(X\equiv s)$                  & $X\in V$, $s\not\in V$                     \\\hline
		Delete        & $P\wedge (s\equiv s)$                  & P                                     &                                            \\\hline
		Eliminate     & $P\wedge (X\equiv s)$                  & $P[s/X]\wedge (X\equiv s)$            & $X\in V(P)$, $X\not\in V(s)$, $s\not\in V$ \\\hline
		Occurs Check  & $P\wedge (X\equiv s)$                  & fail                                  & $X\in V(s)$, $s\not\in V$                  \\\hline
		Coalesce      & $P\wedge (X\equiv Y)$                  & $P[Y/X]\wedge(X\equiv Y)$             & $X,Y\in V(P)$, $X\not\equiv Y$
	\end{tabular}
\end{center}

\section{Rewriting}

\subsection{Definition}

\begin{definition}
	\emph{Rewriting} is a technique for replacing terms in an expression with equivalent terms.
\end{definition}

\begin{theorem}[Rewrite rule of inference]
	\begin{align*}
		\infer{P\{R[\theta]\}}{P\{t\}\hs L\Rightarrow R\hs L[\theta]=t}
	\end{align*}
\end{theorem}

\begin{definition}
	A rewrite rule $\alpha\Rightarrow\beta$ must satisfy the following restrictions:
	\begin{itemize}
		\item $\alpha$ is not a variable.
		\item $V(\beta)\subseteq V(\alpha)$.
	\end{itemize}
\end{definition}

\subsection{Termination}

\begin{definition}
	We say that a set of rewrite rules is \emph{terminating} if starting with any expression,
	successively applying rewrite rules eventually brings us to a state where no rule applies.
\end{definition}

\begin{example}
	Rules that may cause non-termination:
	\begin{itemize}
		\item reflexive rules, e.g. $0\Rightarrow 0$,
		\item self-commuting rewrites, e.g. $X\times Y\Rightarrow Y\times X$, without lexiographical measure,
		\item commuting pairs of rewrites, e.g. $X \Rightarrow Y$ and $Y\Rightarrow X$.
	\end{itemize}
\end{example}

\begin{theorem}
	Suppose we have a measure $M$ that assigns to every expression a nonnegative integer and
	a set of rewrite rules $S$. If every rule in $S$ decreases the value of $M$, then $S$
	is terminating.
\end{theorem}

\subsection{Confluence}

\begin{definition}
	An expression where no rewrite rules apply is said to be in \emph{normal form}.
\end{definition}

\begin{definition}
	Let $s$ be an expression and let $t_1, ..., t_n$ be the normal forms obtained by
	applying rewrite rules exhaustively. If $t_i=t_j$ for all $1\leq i,j\leq n$, then we
	say we have a \emph{canonical normal form} for $s$.
\end{definition}

\begin{definition}
	A set of rewrite rules is \emph{confluent} if for all terms $r,s_1,s_2$ such that
	$r\to^*s_1$ and $r\to^*s_2$ there exists a term $t$ such that
	$s_1\to^*t$ and $s_2\to^*t$.
\end{definition}

\begin{definition}
	A set of rewrite rules is \emph{Church-Rosser} if for all terms $s_1$ and $s_2$
	such that $s_1\leftrightarrow^* s_2$, there exists a term $t$ such that $s_1\to^*t$
	and $s_2\to^* t$.
\end{definition}

\begin{theorem}
	A term is confluent if and only if it is Church-Rosser.
\end{theorem}

\begin{theorem}
	Let $S$ be a terminating rewrite set. If $S$ is confluent, then every expression has a
	canonical normal form under $S$.
\end{theorem}

\begin{definition}
	A set of rewrite rules is \emph{locally confluent} if for all terms $r,s_1,s_2$ such that
	$r\to s_1$ and $r\to s_2$ there exists a term $t$ such that
	$s_1\to^*t$ and $s_2\to^*t$.
\end{definition}

\begin{lemma}[Newman]
	Any terminating and locally confluent rewrite set is confluent.
\end{lemma}

\begin{theorem}
	Local confluence is decidable.
\end{theorem}

\begin{definition}
	Let $L_1\Rightarrow R_1$ and $L_2\Rightarrow R_2$ be rewrite rules. A \emph{critical pair} is
	a pair of expressions
	\begin{align*}
		\lra{R_1[\theta], L_1[\theta]\{R_2[\theta]/s[\theta]\}}
	\end{align*}
	where $s$ is a non-variable sub-term of $L_1$ such that $s[\theta]=L_2[\theta]$ with
	most general unifier $\theta$.
\end{definition}

\begin{theorem}
	An algorithm to test for local confluence works as follows:
	\begin{enumerate}
		\item Find all the critical pairs in set of rewrite rules $R$
		\item For each critical pair $\lra{s_1, s_2}$: \begin{enumerate}
			      \item Find a normal form $s_1'$ of $s_1$
			      \item Find a normal form $s_2'$ of $s_2$
			      \item Check $s_1'=s_2'$, if not then fail
		      \end{enumerate}
	\end{enumerate}
\end{theorem}

\begin{theorem}
	The \emph{Knuth-Bendix Completion Algorithm} works as follows:
	\begin{enumerate}
		\item While there are non-conflatable critical pairs in $R$: \begin{enumerate}
			      \item Take a critical pair $\lra{s_1, s_2}$ in $R$
			      \item Normalise $s_1$ to $s_1'$ and $s_2$ to $s_2'$
			      \item If $R\cup\{s_1'\Rightarrow s_2'\}$ is terminating then \begin{align*}
				            R := R \cup \{s_1'\Rightarrow s_2'\}
			            \end{align*}
			      \item Otherwise, if $R\cup\{s_2'\Rightarrow s_1'\}$ is terminating then \begin{align*}
				            R := R \cup \{s_2' \Rightarrow s_1'\}
			            \end{align*}
			      \item Otherwise, fail.
		      \end{enumerate}
	\end{enumerate}
\end{theorem}

\subsection{The Isabelle simplifier}

\paragraph*{Tactics}

\begin{itemize}
	\item \texttt{simp} rewrites the first subgoal using the \texttt{simpset},
	\item \texttt{auto} simplifies all subgoals and applies basic natural deduction steps.
\end{itemize}
It is also possible to control the simplification, e.g.
\begin{itemize}
	\item \texttt{simp add: ... del: ...} adds and deletes specific rules,
	\item \texttt{simp (no\_asm)} ignores the assumptions,
	\item \texttt{simp (no\_asm\_simp)} uses the assumptions but does not simplify them,
	\item \texttt{simp (no\_asm\_use)} simplifies the assumptions but does not use them.
\end{itemize}

\paragraph*{More features}

\begin{itemize}
	\item Conditional rewriting: A rewrite rule
	      \begin{align*}
		      [[P_1;...;P_n]]\Longrightarrow s = t
	      \end{align*}
	      is applied if
	      \begin{enumerate}
		      \item the left-hand side $s$ matches some expression, and
		      \item Isabelle can recursively prove $P1, ..., P_n$.
	      \end{enumerate}
	\item Ordered rewriting: A lexiographical ordering is used to prevent some rewrite loops.
	\item Case splitting: A rule \begin{align*}
		      ?P (\text{case $?x$ of True} \Rightarrow ?f_1 | \text{False} \Rightarrow f_2)
		      = ((?x = \text{True}\longrightarrow ?P?f_1)\wedge (?x = \text{False} \longrightarrow ?P?f_2))
	      \end{align*}
	      is applied whenever there is an explicit case-split.
	\item Definitions: Every definition \texttt{defn} generates an associated
	      rewrite rule \texttt{defn\_def}.
\end{itemize}

\section{Inductive Proof}

\begin{definition}
	An \emph{inductive datatype} is defined by constructors, some of which involve the datatype
	itself. An inductive datatype is called \emph{free} if terms are only equal if they are
	syntactically identical.
\end{definition}

\begin{example}
	Consider the datatypes \emph{nat} and \emph{int} with the following definitions:
	\begin{minted}{isabelle}
datatype nat = Zero | Succ nat
datatype int = Zero | Succ int | Pred int
    \end{minted}
	Here \emph{nat} is freely generated but \emph{int} is not (e.g. \texttt{Succ (Pred Zero) == Zero}).
\end{example}

\begin{definition}
	Let $(<) \subseteq S\times S$ be an ordering on a type $S$. We call $<$ \emph{well-founded} or
	\emph{noetherian} if there does not exist an infinite sequence $(x_n)$ of $x_n\in S$ such that
	\begin{align*}
		x_1 > x_2 > \cdots > x_n\hs\text{ for all $n\in\N$.  }
	\end{align*}
\end{definition}

\begin{theorem}[Complete Induction]
	Let $P:X\to\{\top,\bot\}$ be a predicate and $(<)\subseteq X\times X$ well-founded ordering.
	If
	\begin{align*}
		\forall y.\:(\forall z.\:z < y \to P\:z) \to P\:y
		\hs\Rightarrow\hs
		\forall x.\:P\:x
	\end{align*}
\end{theorem}

\begin{theorem}
	Consider the $L$-system with left- and right-introduction rules. This system has two
	convenient properties
	\begin{enumerate}
		\item Cut elimination: the cut rule is unnecessary.
		\item Sub-formula property: every cut-free proof only contains formulas which are sub-formulas of the original goal.
	\end{enumerate}
	Introducing the induction rule
	\begin{align*}
		\infer{\Gamma \vdash \forall n. P(n)}{\Gamma \vdash P(0) \hs \Gamma,P(n)\vdash P(n+1) \hs \vdash n\not\in V(\Gamma, P)}
	\end{align*}
	then cut elimination fails. Furthermore, it is impossible to extend the system such that
	the sub-formula property is re-established.
\end{theorem}

\paragraph{Challenges in automating inductive proofs}

Theoretically, and practically, to do inductive proofs we need
\begin{itemize}
	\item lemma speculation
	\item generalisation
\end{itemize}

Automated techniques include
\begin{itemize}
	\item Boyer-Moore approach
	\item Rippling, "Productive Use of Failure"
	\item Up-front speculation
	\item Cycling proofs
	\item Only doing a few cases
\end{itemize}

\section{Hoare Logic}

\subsection{Formal verification}

\begin{definition}
	\emph{Formal specification} uses mathematical notation to give a precise description
	of what a program should do. \emph{Formal verification} uses logical rules to mathematically
	prove that a program satisfies a formal specification.

	There are different types of semantics to describe programs:
	\begin{itemize}
		\item \emph{denotational}: construct mathematical objects that describe the meaning.
		      E.g. functions: $[\![C]\!]:S\to S$.
		\item \emph{operational}: describe the steps of computation during execution. Either
		      small-step ($\lra{C,\sigma}\to\lra{C',\sigma'}$) or big-step ($\lra{C,\sigma}\Downarrow\sigma'$).
		\item \emph{axiomatic}: define axioms and rules of some logic of program. E.g. $\{P\}C\{Q\}$.
	\end{itemize}
\end{definition}

\begin{definition}[Hoare triple]
	In Hoare logic, a program specification states that
	given a state that satisfies preconditions $P$, executing a program $C$, if it terminates,
	will result in a state that satisfies postconditions $Q$. We write
	\begin{align*}
		\hoare{P}{C}{Q}.
	\end{align*}
\end{definition}

\subsection{Hoare logic rules}

\begin{definition}[Assignment axiom]
	\begin{align*}
		\infer{\hoare{Q[E/V]}{V:=E}{Q}}{}
	\end{align*}
	More intuitive statements are wrong, e.g.
	\begin{itemize}
		\item $\hoare{Q}{E:=V}{Q[E/V]}$ allows for $\hoare{X=0}{X:=1}{X=0}$;
		\item $\hoare{Q}{E:=V}{Q[V/E]}$ allows for $\hoare{X=0}{X := 1}{1 = 0}$.
	\end{itemize}
\end{definition}

\begin{definition}[Sequencing rule]
	\begin{align*}
		\infer{\hoare{P}{C_1; C_2}{R}}{\hoare{P}{C_1}{Q} \hs \hoare{Q}{C_2}{R}}
	\end{align*}
\end{definition}

\begin{definition}[Skip axiom]
	\begin{align*}
		\infer{\hoare{P}{\texttt{SKIP}}{P}}{}
	\end{align*}
\end{definition}

\begin{definition}[Conditional rule]
	\begin{align*}
		\infer{\hoare{P}{\texttt{IF $S$ THEN $C_1$ ELSE $C_2$ FI}}{Q}}{
			\hoare{P\wedge S}{C_1}{Q}\hs\hoare{P\wedge\neg S}{C_2}{Q}
		}
	\end{align*}
\end{definition}

\begin{definition}[Precondition strengthening]
	\begin{align*}
		\infer{\hoare{P}{C}{Q}}{P\to P'\hs\hoare{P'}{C}{Q}}
	\end{align*}
\end{definition}

\begin{definition}[Postcondition weakening]
	\begin{align*}
		\infer{\hoare{P}{C}{Q}}{\hoare{P}{C}{Q'}\hs Q'\to Q}
	\end{align*}
\end{definition}

\begin{definition}[While rule]
	\begin{align*}
		\infer{\hoare{P}{\texttt{WHILE $S$ DO $C$ OD}}{P\wedge\neg S}}{\hoare{P\wedge S}{C}{P}}
	\end{align*}
	The loop invariant $P$ should
	\begin{itemize}
		\item hold at each iteration of the loop,
		\item ecapsulate the work that has been done and the work that remains, and
		\item give the desired result when the loop terminates.
	\end{itemize}
\end{definition}

\subsection{Meta-theory}

\begin{theorem}
	Hoare logic is
	\begin{itemize}
		\item sound,
		\item undecidable, and
		\item complete, but only for simple languages.
	\end{itemize}
\end{theorem}

\end{document}