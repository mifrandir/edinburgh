\documentclass{article}
\usepackage{homework-preamble}

\begin{document}
\title{Geometry: Hand-in 2}
\author{Franz Miltz (UUN: S1971811)}
\date{25 October 2021}
\maketitle
\noindent Let $c:[0,2\pi] \to \R^3$ be a curve and $\alpha:\R^3\to\R$ be a $1$-form
such that
\begin{align*}
	c(t)   = (\cos t, \sin t, t), \hs
	\alpha = 2x^1x^2dx^1 + (x^1)^2dx^2 + x^3dx^3.
\end{align*}

\begin{claim*}
	Then the integral of $\alpha$ along $c$ is $\int_c \alpha = 2\pi^2$.
\end{claim*}
\begin{proof}
	Firstly, we find the tangent vector field $c'$:
	\begin{align*}
		c'(t) = -\sin t\frac{\p }{\p x^1} + \cos t\frac{\p }{\p x^2} + \frac{\p }{\p x^3} .
	\end{align*}
	Secondly, we observe that along $c$ we have
	\begin{align*}
		x^1(t) = \cos t, \hs x^2(t)= \sin t, \hs x^3(t) = t.
	\end{align*}
	Thirdly, we use the definition of integrating $\alpha$ along $c$ to obtain
	\begin{align*}
		\int_c \alpha & = \int_a^b \alpha(c'(t))\:dt                                                        \\
		              & = \int_0^{2\pi}\left( (2\sin t\cos t)(-\sin t) + (\cos t)^2 (\cos t) + t\right)\:dt \\
		              & = \int_0^{2\pi} \left(3\cos^3 t - 2 \cos t + t\right)\:dt                           \\
		              & = 3\int_0^{2\pi} \cos^3 t\:dt - 2 \int_0^{2\pi} \cos t\: dt + \int_0^{2\pi} t\:dt   \\
		              & = 2\pi^2.
	\end{align*}
\end{proof}

\begin{claim*}
	$d\alpha = 0$.
\end{claim*}
\begin{proof}
	By definition of $d:\Omega^1(\R^3) \to\Omega^2(\R^3)$ we have
	\begin{align*}
		d\alpha =\sum_I d\alpha_I\wedge dx^I
		= d\alpha_1\wedge dx^1 + d\alpha_2\wedge dx^2 + d\alpha_3\wedge dx^3.
	\end{align*}
	We find
	\begin{align*}
		d\alpha_1 = 2x^2dx^1 + 2x^1dx^2, \hs
		d\alpha_2 = 2x^1dx^1, \hs
		d\alpha_3 = dx^3.
	\end{align*}
	Using $dx^i\wedge dx^i=0$ for all $i$, we find
	\begin{align*}
		d\alpha = 2x^1dx^2\wedge dx^1 + 2x^1dx^1\wedge dx^2.
	\end{align*}
	Since $dx^2\wedge dx^1=-dx^1\wedge dx^2$, the claim follows.
\end{proof}

\begin{claim*}
	Let $f:\R^3\to \R$ be defined by
	\begin{align*}
		f = \left(x^1\right)^2x^2 + 2\left(x^3\right)^2.
	\end{align*}
	Then $df=\alpha$.
\end{claim*}

\begin{proof}
	We have the partial derivatives
	\begin{align*}
		\frac{\p f}{\p x^1}= 2x^1x^2, \hs
		\frac{\p f}{\p x^2}= \left(x^1\right)^2, \hs
		\frac{\p f}{\p x^3}= x^3.
	\end{align*}
	Therefore, by definition of $d:\Omega^0(\R^3)\to\Omega^1(\R^3)$,
	\begin{align*}
		df = \sum_{i=1}^3 \frac{\p f}{\p x^i}dx^i
		= 2x^1x^2dx^1 + \left(x^1\right)^2dx^2 + x^3dx^3 = \alpha.
	\end{align*}
\end{proof}

\end{document}