\documentclass{article}
\usepackage{homework-preamble}

\begin{document}
\title{Honours Analysis: Homework 5}
\author{Franz Miltz (UUN: S1971811)}
\date{21 October 2021}
\maketitle

Let
\begin{align*}
	S(x) =\sum_{k=0}^\infty \frac{(-1)^k x^{2k+1}}{(2k+1)!},\hs \text{and}\hs
	C(x) = \sum_{k=0}^\infty \frac{(-1)^k x^{2k}}{(2k)!}.
\end{align*}

\section*{Workshop 5 - Question 2}

\begin{claim*}
	$S$ is differentiable on $\R$ and $S'=C$.
\end{claim*}

\begin{proof}
	We shall not prove that the interval of convergence is $(-\infty,\infty)$
	as this is part of another exercise, but it can be shown easily by
	applying the \emph{Alternating Series Test}. Let us now focus on the claim above.

	Firstly, observe that we can write
	\begin{align*}
		S(x) = \sum_{n=0}^\infty a_n(x-x_0)^n
	\end{align*}
	with $x_0 = 0$ and
	\begin{align*}
		a_n = \begin{cases}
			\frac{(-1)^{(n-1)/2}}{n!} & \text{if $n$ is odd},  \\
			0                         & \text{if $n$ is even}.
		\end{cases}
	\end{align*}
	By \emph{Wade, Theorem 7.30}, it follows that $S$ is differentiable on $\R$
	and
	\begin{align*}
		S'(x) = \sum_{n=1}^\infty na_k(x-x_0)^{n-1}
		= \sum_{n=0}^\infty (n+1)a_{n+1}x^n.
	\end{align*}
	However, observe that for all odd $n$ we have $a_{n+1}=0$
	and thus we may substitute $n=2k$ and write
	\begin{align*}
		S'(x) = \sum_{k=0}^\infty (2k+1)a_{2k+1}x^{2k}
		= \sum_{k=0}^\infty (2k+1)\frac{(-1)^k x^{2k}}{(2k+1)!}
		= \sum_{k=0}^\infty \frac{(-1)^k x^{2k}}{(2k)!} = C(x).
	\end{align*}
\end{proof}

\subsection*{Workshop 5 - Question 3}


\begin{claim*}
	$C$ is differentiable on $\R$ and $C'=-S$.
\end{claim*}
\begin{proof}
	Let $(a_n)_{n\geq 0}$ be the sequence such that
	\begin{align*}
		a_n = \frac{(-1)^{k}x^{2k}}{(2k)!}.
	\end{align*}
	Fix $x\not=0$. Then note that
	\begin{align*}
		\frac{\abs{a_{n+1}}}{\abs{a_n}} = \frac{x^2}{(2k+2)(2k+1)} \leq \frac{x^2}{4k^2} = \frac{x^2}{4}\left(\frac{1}{k^2}\right)
	\end{align*}
	goes to zero. It follows that the series
	\begin{align*}
		\sum_{k=0}^\infty a_k = C(x)
	\end{align*}
	converges absolutely on $\R\setminus\{0\}$.
	Sinced the power series always converges at $x=0$, we know that the interval of convergence is $(-\infty,\infty)$.
	We can now write
	\begin{align*}
		C(x) = \sum_{n=0}^\infty \frac{(-1)^{n}x^{2n}}{(2n)!} = \sum_{n=0}^\infty b_n(x-x_0)^n
	\end{align*}
	where $x_0=0$ and
	\begin{align*}
		b_n = \begin{cases}
			0                     & \text{if $n$ is odd},  \\
			\frac{(-1)^{n/2}}{n!} & \text{if $n$ is even}.
		\end{cases}
	\end{align*}
	By \emph{Wade, Theorem 7.30}, it follows that $C$ is differentiable on $\R$ and
	\begin{align*}
		C'(x) = \sum_{n=1}^\infty nb_n(x-x_0)^{n-1} = \sum_{n=0}^\infty (n+1)b_{n+1}x^n.
	\end{align*}
	However, for every even $n$ we have $b_{n+1}=0$ and thus we may substitute $n=2k+1$ and write
	\begin{align*}
		C'(x) = \sum_{k=0}^\infty (2k+2)b_{2k+2}x^{2k+1}
		= \sum_{k=0}^\infty \frac{(-1)^{k+1}x^{2k+1}}{(2k+1)!}
		= -S(x).
	\end{align*}
\end{proof}

\begin{claim*}
	Then $(C(x))^2 + (S(x))^2 = 1$ for all $x$ and $\abs{S(x)},\abs{C(x)}\leq 1$
	for all $x$.
\end{claim*}

\begin{proof}
	Let $f:\R\to\R$ be such that
	\begin{align*}
		f(x) = (C(x))^2 + (S(x))^2.
	\end{align*}
	Firstly, observe that $f(0)=1$.
	Since $C,S\in C^\infty(\R)$, $f$ is differentiable and by the rules of differentiation
	\begin{align*}
		f'(x) = 2C(x)C'(x) + 2S(x)S'(x).
	\end{align*}
	We now note that $S'=C$ and $C'=-S$, so $f'(x)=0$ everywhere. By the \emph{Mean Value Theorem}
	it follows that $f$ is constant and since $f(0)=1$, we conclude that $f(x)=1$ for all $x\in\R$.

	Finally, note that
	\begin{align*}
		0 \leq (S(x))^2, (C(x))^2 \leq f(x) = 1.
	\end{align*}
	Since $\sqrt{x}$ is strictly increasing on $\R_{\geq 0}$ and $\sqrt{1}=1$, it follows that
	\begin{align*}
		0 \leq \abs{S(x)}, \abs{C(x)} \leq 1.
	\end{align*}
\end{proof}

\section*{Workshop 5 - Question 5}

\begin{claim*}
	Let $0<x\leq \sqrt{6}$. Then $S(x)>0$.
\end{claim*}
\begin{proof}
	Fix $0<x\leq\sqrt{6}$. We regroup
	\begin{align*}
		S(x) = x - \frac{x^3}{3!} + \frac{x^5}{5!} - \frac{x^7}{7!} + \cdots
		= \left(x - \frac{x^3}{3!}\right) + \left(\frac{x^5}{5!} - \frac{x^7}{7!}\right) + \cdots
	\end{align*}
	We consider a pair of adjacent terms and use $0<x\leq\sqrt{6}$ to find
	\begin{align*}
		\frac{x^{2k+1}}{(2k+1)!} - \frac{x^{2k+3}}{(2k+3)!}
		\geq \frac{x^{2k+1}}{(2k+1)!}-\frac{6x^{2k+1}}{(2k+3)!}
		= ((2k+3)(2k+2)-6)\frac{x^{2k+1}}{(2k+3)!}
	\end{align*}
	where $k\geq 0$.
	Observe that the second factor is strictly greater than zero. Consider the first factor,
	\begin{align*}
		(2k+3)(2k+2) - 6.
	\end{align*}
	Note that this is strictly increasing with $k$ and we have
	\begin{align*}
		k=0 \hs & \Rightarrow\hs (2k+3)(2k+2) - 6 = 0,  \\
		k=1 \hs & \Rightarrow\hs (2k+3)(2k+2) - 6 = 14. \\
	\end{align*}
	Therefore,
	\begin{align*}
		\frac{x^{2k+1}}{(2k+1)!} - \frac{x^{2k+3}}{(2k+3)!} \geq 0
	\end{align*}
	for all $k$. Since the strict inequality holds for $k=1$, the claim follows.
\end{proof}

\begin{claim*}
	Let $0<x\leq \sqrt{2}$. Then $C(x)>0$.
\end{claim*}
\begin{proof}
	Fix $0<x\leq\sqrt{2}$. We regroup
	\begin{align*}
		C(x) = 1 - \frac{x^2}{2!} + \frac{x^4}{4!} - \frac{x^6}{6!} + \cdots
		= \left(1 - \frac{x^2}{2!}\right) + \left(\frac{x^4}{4!} - \frac{x^6}{6!}\right) + \cdots
	\end{align*}
	We consider a pair of adjacent terms and use $0<x\leq\sqrt{2}$ to find
	\begin{align*}
		\frac{x^{2k}}{(2k)!} - \frac{x^{2k+2}}{(2k+2)!}
		\geq \frac{x^{2k}}{(2k)!}-\frac{2x^{2k}}{(2k+2)!}
		= ((2k+2)(2k+1)-2)\frac{x^{2k}}{(2k+2)!}
	\end{align*}
	where $k\geq 0$.
	Observe that the second factor is strictly greater than zero. Consider the first factor,
	\begin{align*}
		(2k+2)(2k+1) - 2.
	\end{align*}
	Note that this is strictly increasing with $k$ and we have
	\begin{align*}
		k=0 \hs & \Rightarrow\hs (2k+2)(2k+1) - 2 = 0,  \\
		k=1 \hs & \Rightarrow\hs (2k+2)(2k+1) - 2 = 12. \\
	\end{align*}
	Therefore,
	\begin{align*}
		\frac{x^{2k}}{(2k)!} - \frac{x^{2k+2}}{(2k+2)!} \geq 0
	\end{align*}
	for all $k$. Since the strict inequality holds for $k=1$, the claim follows.
\end{proof}
\begin{claim*}
	Let $0\leq x\leq \sqrt{56}$ such that $1-x^2/2!+x^4/4!<0$. Then $C(x)<0$.
\end{claim*}
\begin{proof}
	Fix $0\leq x\leq\sqrt{56}$. We regroup
	\begin{align}
		\label{regroup3}
		C(x)= \left(1 - \frac{x^2}{2!} + \frac{x^4}{4!} \right)
		- \left(\frac{x^6}{6!}-\frac{x^8}{8!}\right)
		- \left(\frac{x^{10}}{10!}-\frac{x^{12}}{12!}\right)
		- \cdots
	\end{align}
	Consider a pair of adjacent terms and use $x^2\leq 56$ to find
	\begin{align*}
		\frac{x^{2k}}{(2k)!}-\frac{x^{2k+2}}{(2k+2)!} \geq
		\frac{x^{2k}}{(2k)!}-\frac{56x^{2k}}{(2k+2)!} =
		((2k+2)(2k+1)-56)\frac{x^{2k}}{(2k+2)!}
	\end{align*}
	where $k\geq 3$. Note again that the factor on the right is strictly
	greater than zero. When considering the left factor,
	\begin{align*}
		(2k+2)(2k+1) - 56,
	\end{align*}
	we note that it is strictly increasing with $k$ and we have
	\begin{align*}
		k=3 \hs\Rightarrow\hs (2k+2)(2k+1) - 56 = 0.
	\end{align*}
	Therefore, for $k\geq 3$,
	\begin{align*}
		\frac{x^{2k}}{(2k)!}-\frac{x^{2k+2}}{(2k+2)!}\geq 0.
	\end{align*}
	By the premise,
	\begin{align*}
		1 - \frac{x^2}{2!} + \frac{x^4}{4!} < 0.
	\end{align*}
	Thus all the summands in (\ref{regroup3}) are less than or equal to zero
	where the first one is guaranteed to be non-zero. $C(x)<0$ follows.
	Further, we have
	\begin{align*}
		0 < 8/5 < \sqrt{49} < \sqrt{56}
	\end{align*}
	and
	\begin{align*}
		x = 8/5 \hs\Rightarrow\hs 1-\frac{x^2}{2!}+\frac{x^4}{4!} = -\frac{13}{1875} < 0.
	\end{align*}
	Thus $C(8/5)<0$.
\end{proof}
\end{document}