\documentclass{article}
\usepackage{homework-preamble}

\begin{document}
\title{Honours Analysis: Homework 4}
\author{Franz Miltz (UUN: S1971811)}
\date{21 October 2021}
\maketitle

\section*{Workshop 4 - Question 2}

\begin{claim*}
	Let $f_n:\left[0,1\right)\to\R$ be defined by $f_n(x)=nx^n$.
	Then $f_n\to 0$ pointwise but $\int_0^1 f_n(x)\:dx\to 1$.
\end{claim*}

\begin{proof}
	We show $f_n(x)\to 0$.
	Note if $x=0$, then $f_n(x)=0$ for all $n$. Consider $x\in(0,1)$.
	Consider the limit
	\begin{align*}
		r = \lim_{n\to\infty} \frac{\abs{f_{n+1}(x)}}{\abs{f_n(x)}}
		= \lim_{n\to\infty}\frac{\abs{(n+1)x^{n+1}}}{\abs{nx^n}}
		= \lim_{n\to\infty}\frac{(n+1)x}{n} = x < 1.
	\end{align*}
	By the \emph{Ratio Test} this implies the series $\sum_{n=1}^\infty
		f_n(x)$ converges absolutely for all $x$ which implies that
	$f_n\to 0$ pointwise. However,
	\begin{align*}
		\lim_{n\to\infty} \int_0^1 f_n(x)\:dx
		= \lim_{n\to\infty}\int_0^1 nx^n\:dx
		= \lim_{n\to\infty}\frac{n}{n+1} = 1.
	\end{align*}
	This demonstrates that uniform convergence is required as a premise for
	\emph{Theorem 2.2} in the notes since
	\begin{align*}
		\lim_{n\to\infty} \int_0^1 f_n(x)\:dx
		\not = \int_0^1\left(\lim_{n\to\infty} f_n(x)\right)\:dx.
	\end{align*}
\end{proof}

\section*{Workshop 4 - Question 5}

\begin{claim*}
	Let $f_n:[0,\infty)\to\R$ be defined by
	\begin{align*}
		f_n(x) = \frac{x^n}{1+x^n}.
	\end{align*}
	Then $f_n\to f$ pointwise where
	\begin{align*}
		f(x) = \begin{cases}
			       1,   & \text{if }x > 1, \\
			       1/2, & \text{if }x = 1, \\
			       0,   & \text{if }x < 1.
		       \end{cases}
	\end{align*}
	This convergence is not uniform on any interval $[0, a)$ where
	$a\geq 1/2$ is an extended real number but it
\end{claim*}
\begin{proof}
	Consider $x>1$. Let $\e>0$. Then pick $N\in\R$ as follows
	\begin{align*}
		N = \begin{cases}
			    0,              & \text{if }1/\e \leq 1, \\
			    \log_x(1/\e-1), & \text{otherwise.}
		    \end{cases}
	\end{align*}
	Observe that now, for all $n>N$,
	\begin{align*}
		\frac{1}{\e} - 1 < x^n,
		\hs\text{or equivalently}\hs
		\frac{1}{1+x^n}< \e.
	\end{align*}
	Since $x^n\geq 0$ for all $x,n$, we have
	\begin{align*}
		\abs{\frac{1}{1+x^n}}=\abs{\frac{x^n}{1+x^n}-\frac{1+x^n}{1+x^n}}
		=\abs{\frac{x^n}{1+x^n}-1}<\e.
	\end{align*}
	Now observe that $f_n(1)=1/2$ for all $n$.
	This shows that $f_n\to f$ pointwise on $[1,\infty)$. Consider $0\leq x<1$.
	We observe
	\begin{align*}
		0 \leq f_n(x)=\frac{x^n}{1 + x^n} \leq x^n\hs\text{for all }n.
	\end{align*}
	Since $x^n\to 0$, we can use the \emph{Squeeze Theorem} to show that
	$f_n\to 0$ pointwise as $n\to\infty$. Combining all the cases, we have shown that
	$f_n\to f$ pointwise on $[0, \infty)$. Consider an interval $[0,a)$ where $a\geq 1$
	is an extended real number. If $f_n$ converges uniformly then it must be the
	case that the limit is $f$. Consider a sequence $(x_k)$ such that $x_k\to 1$ as
	$k\to\infty$ and $x_k\in[0,1)$. Let $\e=1/4$. If $f_n\to f$ uniformly on $[0, a)$, then
	there exists an $N$ such that for all $n>N$,
	\begin{align}
		\label{cond1}
		\abs{f_n(x)-f(x)} < \e \hs\text{for all }x\in [0, 1) \subseteq [0, a).
	\end{align}
	However, observe that for all $n$, $f_n$ is a rational function. Thus it is
	continuous on its domain. Since $f_n(1)=1/2$, we find
	\begin{align*}
		\lim_{x\to 1} f_n(x) = \frac{1}{2}.
	\end{align*}
	Since $x_k\to 1$ as $k\to\infty$, this implies $f_n(x_k)\to 1/2$ as $k\to\infty$ for all $n$.
	By definition, this means that there exists at least one $k_0$ such that
	\begin{align}
		\label{cond2}
		\abs{f_n(x_{k_0}) - \frac{1}{2}} < \e, \hs\text{or equivalently}\hs \abs{f_n(x_{k_0})} > \frac{1}{4}.
	\end{align}
	Assume that $f_n\to f$ on $[0, a)$. Then (\ref{cond1}) holds. Using $x=x_{k_0}$, this implies
	\begin{align*}
		\abs{f_n(x_{k_0})} < \e = \frac{1}{4}
	\end{align*}
	which directly contradicts (\ref{cond2}). Therefore $f_n$ cannot converge uniformly on $[0, a)$.
	Now consider the case where $a<1$.
	Note that the rational functions $f_n$ are differentiable on $[0, \infty)$ with
	\begin{align*}
		f_n'(x) =  \frac{nx^{n-1}}{(1+x^n)^2}.
	\end{align*}
	Note that $f'_n(x)>0$ for all $x\in(0,\infty)$. Additionally, $f_n(0) = 0 < f_(x)$ on $x\in(0, \infty)$.
	Therefore all $f_n$ are strictly increasing on $[0, \infty)$. Therefore, for all $x\in[0,a)$,
	\begin{align}
		f_n(x) < f_n(a).
	\end{align}
	Observe that $f_n\to f$ pointwise and thus $f_n(a)\to f(a)$. Let $\e>0$. Choose $N$ such that
	for all $n>N$,
	\begin{align*}
		\abs{f_n(a)-f(a)} < \e.
	\end{align*}
	With $f_n(x)\geq 0$ and $f(x)=0$ for all $x\in[0,1)$, we have
	\begin{align*}
		\abs{f_n(x)-f(x)}=f_n(x)<f_n(a)=\abs{f_n(a) - f(a)} < \e.
	\end{align*}
	This shows $f_n\to f$ uniformly on any interval $[0, a)$ where $a<1$.
\end{proof}

\section*{Workshop 4 - Question 7}

\begin{claim*}
	Let $f_n:\R\to\R$ be a sequence of continuous functions which converges uniformly to a
	function $f:\R\to\R$. Let $(x_n)$ be a sequence of real numbers which converges to
	$x\in\R$. Then $f_n(x_n)\to f(x)$.
\end{claim*}
\begin{proof}
	Let $\e>0$. Since $f_n\to f$ uniformly, we can choose an $N_1$ such that for all
	$n>N_1$ and all $x\in\R$,
	\begin{align}
		\label{uniform}
		\abs{f_n(x)-f(x)} < \frac{\e}{3}.
	\end{align}
	Further, since $f_k$ is continuous for all $k\in\N$, there exists an $N_2$ such that
	for all $n>N_2$,
	\begin{align}
		\label{continuous}
		\abs{f_k(x_n) - f_k(x)} < \frac{\e}{3}.
	\end{align}
	The above shows that $f_k(x_n)\to f_k(x)$ for all $k\in\N$. Therefore \emph{Cauchy's Criterion}
	holds: We can pick an $N_3$ such that for all $k\in\N$ and $m,n>N_3$,
	\begin{align}
		\label{cauchy}
		\abs{f_k(x_n) - f_k(x_m)} < \frac{\e}{3}.
	\end{align}
	We choose $N=\max\{N_1,N_2,N_3\}$. Then for all $k,n>N$, the following inequalities hold
	due to (\ref{uniform}), (\ref{continuous}) and (\ref{cauchy}) respectively
	\begin{align*}
		\abs{f_n(x) - f(x)} < \frac{\e}{3}, \hs\abs{f_n(x_k)-f_n(x)} < \frac{\e}{3},
		\hs\abs{f_n(x_n)-f_n(x_k)} < \frac{\e}{3}.
	\end{align*}
	We sum them up to get
	\begin{align*}
		\abs{f_n(x_n)-f_n(x_k)} + \abs{f_n(x_k)-f_n(x)} + \abs{f_n(x)-f(x)} < \e.
	\end{align*}
	Due to the \emph{Triangle Inequality}, it follows that
	\begin{align*}
		\abs{f_n(x_n)-f(x)}<\e.
	\end{align*}
\end{proof}

\end{document}