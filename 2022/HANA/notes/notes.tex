\documentclass{article}
\usepackage{notes-preamble}
\usepackage{enumitem}
\begin{document}
\mkthmstwounified
\title{Honours Analysis (SEM5)}
\author{Franz Miltz}
\maketitle
\tableofcontents
\pagebreak

\section{Revision}

\subsection{Real numbers}

\subsubsection{Nested intervals}

\begin{definition}
	A sequence $(I_n)_{n\in\N}$ of sets is said to be nested if
	\begin{align*}
		I_1 \supset I_2 \supset I_3 \supset \dots
	\end{align*}
\end{definition}

\begin{theorem}[Nested interval property]
	If $(I_n)_{n\in\N}$ is a nested sequence of nonempty closed bounded intervals then
	\begin{align*}
		E=\bigcap_{n\in\N} I_n = \{x\in\R : x \in I_n \text{ for all } n\in\N\}
	\end{align*}
	is nonempty.
	Moreover if $\lambda(I_n)\to 0$, where $\lambda(I_n)$ denotes the length of the
	interval $I_n$, then $E$ contains exactly one number.
\end{theorem}

\subsubsection{Compactness of closed bounded intervals}

\begin{definition}
	Let $E=[a,b]$ for some real numbers $a\leq b$. Suppose that $(I_\alpha)_{\alpha\in\mathcal{A}}$
	is a collection of intervals. Then we say that $(I_\alpha)_{\alpha\in\mathcal{A}}$ \textbf{covers}
	$E$ if
	\begin{align*}
		E \subset \bigcup_{\alpha \in \mathcal{A}} I_\alpha.
	\end{align*}
\end{definition}

\begin{theorem}[Notes 1.2]
	Let $E=[a,b]$ for some real numbers $a\leq b$. Suppose that $(I_\alpha)_{\alpha\in\mathcal{A}}$
	is an arbitrary collection of open intervals that cover $E$. Then there exists a finite
	set of indices $\{\alpha_1, \alpha_2, ..., \alpha_n\}$ such that
	\begin{align*}
		E \subset I_{\alpha_1} \cup I_{\alpha_2}\cup \cdots \cup I_{\alpha_n}.
	\end{align*}
	We say that $(I_{\alpha_i})_{i=1,2,...,n}$ is a finite subcover of $E$.
\end{theorem}

\subsection{Sequences in $\R$}

\begin{definition}
	A sequence of real numbers $(x_n)$ is said to \emph{converge} to a real number
	$a\in\R$ iff for every $\e>0$ there is an $N\in\N$ such that
	\begin{align*}
		n\geq N \text{ implies } \abs{x_n-a} < \e.
	\end{align*}
	The number $a$ is called the limit of the sequence $(x_n)$. A sequence that
	does not converge to a real number is said to \emph{diverge}.
\end{definition}

\begin{definition}
	A sequence $(x_n)$ of numbers $x_n\in\R$ is said to be \emph{Cauchy} if for every $\e>0$ there
	exists an $N\in\N$ such that
	\begin{align*}
		\abs{x_n-x_m}<\e \hs \text{for all }n,m\geq N.
	\end{align*}
\end{definition}

\begin{theorem}[Notes 1.3]
	If $(x_n)$ is a convergent sequence of real numbers, then $(x_n)$ is a Cauchy sequence.
\end{theorem}

\begin{theorem}[Cauchy]
	Let $(x_n)$ be a sequence of real numbers. Then $(x_n)$ is a Cauchy sequence iff $(x_n)$ is a convergent sequence.
\end{theorem}

\begin{definition}
	Suppose $(x_n)_{n\in\N}$ is a sequence. A \emph{subsequence} of $(x_n)$ is a sequence of the form
	$(x_{n_k})_k\in\N$ where for each $k$ there is a positive integer $n_k$ such that
	\begin{align*}
		n_1 < n_2 < \cdots n_k < n_{k+1} < ...
	\end{align*}
\end{definition}

\begin{theorem}[Bolzano-Weierstrass]
	Every bounded sequence of real numbers has a convergent subsequence.
\end{theorem}

\begin{definition}
	If $(x_n)$ is a bounded sequence of real numbers we denote by
	\begin{align*}
		\limsup_{n\to\infty} x_n = \lim_{n\to\infty}\left(\sup_{k\geq n}x_k\right),\hs
		\liminf_{n\to\infty} x_n = \lim_{n\to\infty}\left(\inf_{k\geq n}x_k\right).
	\end{align*}
\end{definition}

\begin{theorem}[Notes 1.6]
	A sequence $(x_n)$ of real numbers is convergent iff $\limsup_{n\to\infty}x_n$
	and $\liminf_{n\to\infty}x_n$ are real numbers and
	\begin{align*}
		\limsup_{n\to\infty}x_n =\liminf_{n\to\infty} x_n.
	\end{align*}
\end{theorem}

\subsection{Infinite series of real numbers}

\begin{definition}
	Let $S=\sum_{k=1}^\infty a_k$ be an infinite series. For each $n\in\N$, the partial
	sum of $S$ of order $n$ is defined by
	\begin{align*}
		s_n = \sum_{k=1}^n a_k.
	\end{align*}
	$S$ is said to \emph{converge} iff its sequence of partial sums $(s_n)$
	converges to some $s\in\R$ as $n\to\infty$.
\end{definition}

\begin{theorem}[Cauchy criterion for series]
	Let $S=\sum_{k=1}^\infty a_k$ be a series. Then the series $S$ is convergent
	iff for any $\e>0$ there exists $N$ such that for all $m\geq n\geq N$
	we have that
	\begin{align*}
		\abs{\sum_{k=n+1}^m}<\e.
	\end{align*}
\end{theorem}

\begin{theorem}[Notes 1.8]
	Let $S=\sum_{k=1}^\infty a_k$ be an absolutely convergent series. Then
	\begin{enumerate}
		\item The series $S$ is convergent.
		\item Let $z:\N\to\N$ be a bijection. Then the series $\sum_{k=1}^\infty a_{z(k)}$
		      is convergent and \begin{align*}
			      \sum_{k=1}^\infty a_k = \sum_{k=1}^\infty a_{z(k)}.
		      \end{align*}
	\end{enumerate}
\end{theorem}

\begin{theorem}[Notes 1.9]
	Let $S=\sum_{k=1}^\infty a_k$ be any conditionally convergent series. Then there
	exist rearangements $z:\N\to\N$ such that
	\begin{enumerate}
		\item For any $r\in\R$ the series $\sum_{k=1}^\infty a_{z(k)}$ is conditionally
		      convergent and its sum is $r$.
		\item The series $\sum_{k=1}^\infty a_{z(k)}$ diverges to $+\infty$.
		\item The series $\sum_{k=1}^\infty a_{z(k)}$ diverges to $-\infty$.
		\item The partial sums of the series $\sum_{k=1}^\infty a_{z(k)}$ oscillate between any two real numbers.
	\end{enumerate}
\end{theorem}

\subsection{Continuity of real functions}

\begin{definition}
	Let $f$ be a function $f:\dom(f)\to\R$ where $\dom(f)\subset\R$. We say that $f$ is
	continuous at some $a\in\dom(f)$ if for any sequence $(x_n)$ whose terms lie in
	$\dom(f)$ and which converges to $a$, we have $\lim_{n\to\infty}f(x_n)=f(a)$. If
	$f$ is continuous at each $a\in\S\subset\dom(f)$ then we say $f$ is continuous on $S$.
	If $f$ is continuous on $\dom(f)$ then we say that $f$ is continuous.
\end{definition}

\begin{theorem}[Notes 1.10]
	Let $f,g:D\to\R$ be continuous on $D$, and let $\alpha\in\R$ then the following functions
	are continuous on $D$.
	\begin{enumerate}
		\item $\alpha f$;
		\item $f+g$;
		\item $fg$.
	\end{enumerate}
\end{theorem}

\begin{definition}
	Let $A,B\subseteq\R$ be nonempty, let $f:A\to\R$, $g:B\to\R$ and $f(A)\subseteq B$. The
	composition of $g$ with $f$ is the function $g\circ f:A\to\R$ defined by
	\begin{align*}
		(g\circ f)(x) = g(f(x)),\hs \text{for all }x\in A.
	\end{align*}
\end{definition}

\begin{theorem}[Notes 1.11]
	If $f$ is continuous at $a\in\R$ and $g$ is continuous at $f(a)$ then the composition
	$g\circ f$ is continuous at $a$.
\end{theorem}

\begin{theorem}[Notes 1.12]
	Let $f$ be a function $f:\dom(f)\to\R$ where $\dom(f)\subset\R$. Then $f$ is continuous
	at $a\in\dom(f)$ iff for any $\e>0$ there exists $\delta>0$ such that
	whenever $x\in\dom(f)$ and $\abs{x-a}<\delta$ we have $\abs{f(x)-f(a)}<\e$.
\end{theorem}

\begin{theorem}[Intermediate Value]
	Let $a<b$ be real numbers and $f:[a,b]\to\R$ be a continuous on $[a,b]$.
	If $f(a)f(b)>0$ then there eixists at least one $c\in(a,b)$ such that $f(c)=0$.
\end{theorem}

\begin{theorem}[Extreme Value]
	Let $a<b$ be real numbers and $f:[a,b]\to\R$ be continuous on $[a,b]$. Then
	there exists points $c,d\in[a,b]$ such that
	\begin{align*}
		f(c) = \inf\{f(x):x\in[a,b]\},\hs f(d)=\sup\{f(x):x\in[a,b]\}.
	\end{align*}
	That is the function $f$ on the interval $[a,b]$ is bounded and attains its minimal
	and maximal values at some points $c,d\in[a,b]$ respectively.
\end{theorem}

\section{Uniform convergence}

\subsection{Uniform convergence of sequences of functions}

\begin{definition}
	Let $E$ be a nonempty subset of $\R$. A sequence of functions $f_n:E\to\R$ is said to
	\emph{converge poinwise} on $E$ iff $f(x)=\lim_{n\to\infty}f_n(x)$ exists
	for each $x\in E$.
\end{definition}

\begin{theorem}
	The pointwise limit does not preserve
	\begin{itemize}
		\item continuouity, i.e. all $f_n$ are continuous but $f$ is not,
		\item differentiability, i.e. all $f_n$ are differentiable but $f$ is not,
		\item itegrability, i.e. all $f_n$ are Lebesgue integrable but $f$ is not,
		\item derivatives, i.e. all $f_n$ and $f$ are differentiable, but $(f_n)'\not\to f'$,
		\item integrals, i.e. all $f_n$ and $f$ are Lebesgue integrable, but $\int_a^b f_n\:d\lambda\not\to \int_a^b f\:d\lambda$.
	\end{itemize}
\end{theorem}

\begin{definition}
	Let $E$ be a nonempty subset of $\R$. A sequence of functions $f_n:E\to\R$ is said to
	\emph{converge uniformly} on $E$ to a function $f$ iff for every $\e>0$ there
	is an $N\in\N$ such that
	\begin{align*}
		n\geq N\hs\text{implies}\hs \abs{f_n(x)-f(x)}<\e
	\end{align*}
	for all $x\in E$.
\end{definition}

\begin{proposition}
	The following are equivalent concerning a sequence of functions $f_n:E\to\R$ and $f:E\to\R$:
	\begin{enumerate}
		\item $f_n\to f$ uniformly on $E$
		\item $\sup_{x\in E}\abs{f_n(x)-f(x)}\to 0$ as $n\to\infty$
		\item there exists a sequence $a_n\to 0$ such that $\abs{f_n(x)-f(x)}\leq a_n$ for all $x\in E$.
	\end{enumerate}
\end{proposition}

\begin{definition}
	A sequence of functions $f_n$ is said to be \emph{uniformly bounded} on a set $E$ if there
	is an $M>0$ such that $\abs{f_n(x)}\leq M$ for all $x\in E$ and all $n\in\N$.
\end{definition}

\begin{lemma}
	Let $f_n:E\to\R$ be a sequence of real bounded functions. If $f_n\to f$ uniformly on $E$
	then $f$ is bounded and $f_n$ is uniformly bounded on $E$.
\end{lemma}

\begin{theorem}[Notes 2.2]
	Suppose that $f_n\to f$ uniformly on a closed interval $[a,b]$. If each $f_n$ is Lebesgue integrable
	on $[a,b]$, then so is $f$ and
	\begin{align*}
		\lim_{n\to\infty} \int_a^b f_n(x)\:d\lambda = \int_a^b\left(\lim_{n\to\infty}f_n(x)\right)\:d\lambda.
	\end{align*}
	In fact, $\lim_{n\to\infty} \int_a^x f_n(t)\:d\lambda = \int_a^x f(t)\:d\lambda$ uniformly for $x\in[a,b]$.
\end{theorem}

\begin{theorem}[Notes 2.3]
	Let $(a,b)$ be a bounded interval and suppose that $f_n$ is a sequence of functions which converges
	at some $x_0\in(a,b)$. If each $f_n$ is differentiable on $(a,b)$, and $f'_n$ converges uniformly
	on $(a,b)$ as $n\to\infty$, then $f_n$ converges uniformly on $(a,b)$ and
	\begin{align*}
		\lim_{n\to\infty}f'_n(x)= \left(\lim_{n\to\infty}f_n(x)\right)'
	\end{align*}
	for each $x\in(a,b)$.
\end{theorem}

\subsection{Uniform convergence of series of functions}

\begin{definition}
	Let $f_k$ be a sequence of real functions defined on some set $E$ and set
	\begin{align*}
		s_n(x) = \sum_{k=1}^n f_k(x),\hs x\in E,n\in\N.
	\end{align*}
	\begin{enumerate}
		\item The series $\sum_{k=1}^\infty f_k$ is said to \emph{converge pointwise} on $E$ iff
		      the sequence $s_n(x)$ converges pointwise on $E$ as $n\to\infty$.
		\item The series $\sum_{k=1}^\infty f_k$ is said to \emph{converge uniformly} on $E$ iff
		      the sequence $s_n(x)$ converges uniformly on $E$ as $n\to\infty$.
		\item The series $\sum_{k=1}^\infty f_k$ is said to \emph{converge absolutely (pointwise)} on $E$ iff
		      $\sum_{k=1}^\infty \abs{f_k(x)}$ converges for each $x\in E$.
	\end{enumerate}
\end{definition}

\begin{theorem}[Notes 2.4]
	Let $E\subseteq\R$ be non-empty and let $(f_k)$ be a sequence of functions $E\to\R$.
	\begin{enumerate}
		\item Suppose that $x_0\in E$ and that each $f_k$ is continuous at $x_0\in E$.
		      If $f=\sum_{k=1}^\infty f_k$ converges uniformly on $E$, then $f$ is continuous at $x_0\in E$.
		\item Term-by-term integration: Suppose that $E=[a,b]$ and that each $f_k$ is Lebesgue integrable on $E$.
		      If $f=\sum_{k=1}^\infty f_k$ converges uniformly on $E$, then $f$ is Lebesgue integrable on $E$
		      and \begin{align*}
			      \int_E\sum_{k=1}^\infty f_k(x)\:d\lambda = \sum_{k=1}^\infty \int_E f_k(x)\:d\lambda.
		      \end{align*}
		\item Term-by-term differentiation: Suppse that $E$ is a bounded, open interval and that each
		      $f_k$ is differentiable on $E$. If $\sum_{k=1}^\infty f_k(x_0)$ converges at some $x_0\in E$,
		      and $g=\sum_{k=1}^\infty f'_k$ converges uniformly on $E$, then $f=\sum_{k=1}^\infty f_k$
		      converges uniformly on $E$, is differentiable on $E$, and, for all $x\in E$, \begin{align*}
			      f'(x) = \left(\sum_{k=1}^\infty f_k(x)\right)' = \sum_{k=1}^\infty f'_k(x)= g(x).
		      \end{align*}
	\end{enumerate}
\end{theorem}

\begin{theorem}[Weierstrass M-test]
	Let $E$ be a nonempty subset of $\R$, let $f_k:E\to\R$, $k\in\N$, and suppose that $M_k\geq 0$
	satisfies $\sum_{k=1}^\infty M_k<\infty$. If $\abs{f_k(x)}\leq M_k$ for $k\in\N$ and $x\in E$,
	$f=\sum_{k=1}^\infty f_k$ converges absolutely and uniformly on $E$.
\end{theorem}

\begin{lemma*}
	\begin{align*}
		\sum_{k=1}^\infty \frac{1}{k(k+1)} = \sum_{k=1}^\infty \left(\frac{1}{k+1}-\frac{1}{k}\right)=1.
	\end{align*}
\end{lemma*}

\section{Power series}

\subsection{Introduction}

\begin{definition}
	The \emph{radius of convergence $R$} of the power series
	\begin{align}
		\label{ps}
		\sum_{n=0}^\infty a_n(x-c)^n
	\end{align}
	is defined by
	\begin{align*}
		R = \sup\{r\geq 0:(a_nr^n) \text{ is bounded}\},
	\end{align*}
	unless $(a_nr^n)$ is bounded for all $r\geq 0$, in which case we declare $R=\infty$.
\end{definition}

\begin{theorem}[Notes 3.1]
	Suppose the radius of convergence $E$ of (\ref{ps}) satisfies $0<R<\infty$.
	If $\abs{x-c}<R$, the power series (\ref{ps}) converges absolutely. If
	$\abs{x-c}>R$, the power series (\ref{ps}) diverges.
\end{theorem}

\begin{lemma}[Example 2.2]
	Consider a power series as in (\ref{ps}).
	\begin{itemize}
		\item If $\lim_{n\to\infty}\abs{\frac{a_n}{a_{n+1}}}$ exists then it equals $R$.
		\item If $\lim_{n\to\infty}\abs{a_n}^{-1/n}$ exists then it equals $R$.
	\end{itemize}
\end{lemma}

\subsection{Continuity and differentiability of power series}

\begin{theorem}[Notes 3.2]
	Assume that $R>0$. Suppose that $0<r<R$. Then the series (\ref{ps}) converges
	uniformly and absolutely on $\abs{x-c}\leq r$ to a continuous function $f$.
	Hence
	\begin{align*}
		f(x) = \sum_{n=0}^\infty a_n(x-c)^n
	\end{align*}
	defines a continuous function $f:(c-R, c+R)\to\R$.
\end{theorem}

\begin{lemma}[Notes 3.1]
	The two poer series $\sum_{n=1}^\infty a_n(x-c)^n$ and $\sum_{n=1}^\infty (x-c)^{n-1}$
	have the same radius of convergence.
\end{lemma}

\begin{theorem}[Notes 3.3]
	Suppose the radius of convergence of the power series (\ref{ps}) is $R$. Then
	the function
	\begin{align*}
		f(x) = \sum_{n=0}^\infty a_n(x-c)^n
	\end{align*}
	is infinitely differentiable on $\abs{x-c} < R$, and for such $x$,
	\begin{align*}
		f'(x) = \sum_{n=0}^\infty na_n(x-c)^{n-1}
	\end{align*}
	and the series converges absolutely, and also uniformly on $[c-r, c+r]$ for any
	$r<R$. Moreover,
	\begin{align*}
		a_n = \frac{f^{(n)}(c)}{n!}.
	\end{align*}
\end{theorem}

\subsection{Analytic functions}

\begin{definition}
	A function $f$ is \emph{analytic} on $\{x: \abs{x-c}<r\}$ if there is a power
	series (\ref{ps}) which converges to $f$ on $\{x: \abs{x-c}<r\}$.
\end{definition}

\begin{definition}
	The \emph{exponential function} $\exp : \R\to\R$ is defined by the power series
	\begin{align*}
		\exp x := \sum_{n=0}^\infty \frac{x^n}{n!}.
	\end{align*}
\end{definition}

\begin{corollary*}
	\begin{align*}
		\lim_{n\to\infty}\left(1+\frac{x}{n}\right)^n = \exp x.
	\end{align*}
\end{corollary*}

\begin{theorem}
	For all $x,y\in\R$, and $q\in\Q$,
	\begin{enumerate}
		\item $\exp(x)\exp(-x)=1$,
		\item $\exp x > 0$,
		\item if $f:\R\to\R$ such that $f=f'$ and $f(0)=1$ then $f=\exp$,
		\item $\exp(x+y)=\exp x + \exp y$
		\item $\exp x = 1/\exp x$,
		\item $\exp (qx) = (\exp x)^q$.
	\end{enumerate}
\end{theorem}

\begin{definition}
	The trigonometric functions sine and cosine are defined by the power series
	\begin{align*}
		\sin x = \sum_{k=0}^\infty \frac{(-1)^k x^{2k+1}}{(2k+1)!},\hs
		\cos x = \sum_{k=0}^\infty \frac{(-1)^k x^{2k}}{(2k)!}.
	\end{align*}
\end{definition}

\section{Integration}

\subsection{Step functions}

\begin{definition}
	We say that $\phi:\R\to\R$ is a \emph{step function} if there exist real numbers
	$x_0<x_1<\cdots < x_n$ (for some $n\in\N$) such that
	\begin{enumerate}
		\item $\phi(x)=0$ for $x<x_0$ and $x>x_n$,
		\item $\phi$ is constant on $(x_{j-1}, x_j)$, $1\leq j\leq n$.
	\end{enumerate}
\end{definition}

\begin{definition}
	Let $f:\R\to\R$. Then $f$ has \emph{bounded support} if there exists a bounded
	interval $I\subset\R$ such that $f(x)=0$ for all $x\not\in I$.
\end{definition}

\begin{lemma}
	Let $\phi:\R\to\R$ a step function. Then
	\begin{enumerate}
		\item $\phi$ has bounded support,
		\item $\phi$ is continuous everywhere except possibly at finitely many points
		      $x_1,...,x_n\in\R$.
	\end{enumerate}
\end{lemma}

\begin{lemma}[Example 4.2]
	Let $\phi:\R\to\R$. Then $\phi$ is a step function iff there exist $c_1,...,c_n\in\R$
	and bounded intervals $J_1,...,J_n\subset\R$ such that
	\begin{align*}
		\phi = \sum_{j=1}^n c_j\chi_{J_j}.
	\end{align*}
\end{lemma}

\begin{definition}
	Let $J\subset\R$ be a bounded interval. Then we define the length of the interval $\lambda(J)\in\R$ by
	\begin{align*}
		\lambda(J) := \sup J - \inf J.
	\end{align*}
\end{definition}

\begin{definition}
	\label{step_integration}
	If $\phi$ is a step function with respect to $\{x_0, x_1, ..., x_n\}$ which takes the
	value $c_j$ on $(x_{j-1}, x_j)$, then we define
	\begin{align*}
		\int \phi = \sum_{j=1}^n c_j\lambda((x_{j-1},x_j)).
	\end{align*}
\end{definition}

\subsection{Lebesgue integral}

\begin{definition}
	\label{lebesgue}
	A function $f:I\to\R$ is said to be \emph{Lebesgue integrable} on an interval
	$I$ if there exist values $c_j$ and bounded intervals $J_j\subset I$, $j=1,2,3,...$
	such that
	\begin{align*}
		\sum_{j=1}^\infty \abs{c_j}\lambda (J_j) < \infty,
	\end{align*}
	and the equality
	\begin{align*}
		f(x) = \sum_{j=1}^\infty \abs{c_j}\chi_{J_j}(x)
	\end{align*}
	holds for all $x\in I$ at which
	\begin{align*}
		\sum_{j=1}^\infty \abs{c_j} X_{J_j}(x) < \infty.
	\end{align*}
	We denote by $\int_I f\:d\lambda$ the number
	\begin{align*}
		\int_I f\:d\lambda = \sum_{j=1}^\infty c_j\lambda(J_j)
	\end{align*}
	and call it the Lebesgue integral of $f$ over the interval $I$.
\end{definition}

\begin{theorem}[Notes 4.1]
	Suppose that $c_j,d_j$ are real numbers and $J_j,K_j$ are bounded intervals
	for all $j=1,2,3,...$ and
	\begin{align*}
		\sum_{j=1}^\infty \abs{c_j}\lambda(J_j)<\infty, \hs
		\sum_{j=1}^\infty \abs{d_j}\lambda(K_j)<\infty.
	\end{align*}
	If
	\begin{align*}
		\sum_{j=1}^\infty \abs{c_j}\chi_{J_j}(x) < \infty\hs\text{and}\hs
		\sum_{j=1}^\infty \abs{d_j}\chi_{K_j}(x) <\infty,
	\end{align*}
	then
	\begin{align*}
		\sum_{j=1}^\infty c_j\lambda(J_j)= \sum_{j=1}^\infty d_j\lambda(K_j).
	\end{align*}
\end{theorem}

\begin{corollary}
	Let $\phi:\R\to\R$ be a step function. Then $\phi$ is Lebesgue integrable on $\R$
	and $\int\phi\:d\lambda = \int\phi$.
\end{corollary}

\begin{corollary}
	The function $\chi_{\Q\cap[0,1]}$ is Lebesgue integrable on $[0,1]$ and its integral is zero.
\end{corollary}

\begin{theorem}[Notes 4.2]
	Suppose $f,g:I\to\R$ are Lebesgue integrable and $\alpha,\beta\in\R$. Then
	\begin{enumerate}
		\item $\alpha f + \beta g$ is Lebesgue integrable and \begin{align*}
			      \int_I (\alpha f + \beta g)\:d\lambda = \alpha \int_I f\:d\lambda + \beta \int_I g\:d\lambda.
		      \end{align*}
		\item If $f\geq 0$ then $\int_I f\:d\lambda \geq 0$; if $f\geq g$ then
		      $\int_I f\:d\lambda\geq \int_I g\:d\lambda$.
		\item $\abs f$ is Lebesgue integrable and $\abs{\int_I f\:d\lambda}\leq \int_I \abs f\:d\lambda$.
		\item $\max\{f,g\}$ and $\min\{f,g\}$ are Lebesgue integrable.
		\item If one of the functions is bounded then the product $fg$ is Lebesgue integrable.
		\item If $f\geq 0$ with $\int_I f\:d\lambda = 0$ then any function $h$ such that $0\leq h<f$ is Lebesgue integrable.
	\end{enumerate}
\end{theorem}

\begin{lemma}[Exercise 4.6]
	Let $f,g:I\to\R$ for some interval $I\subseteq\R$ such that $f=g$ almost everywhere.
	If $f$ is Lebesgue integrable then so is $g$ and
	\begin{align*}
		\int_I f\:d\lambda = \int_I g\:d\lambda.
	\end{align*}
\end{lemma}

\subsection{Riemann integration}

\begin{definition}
	Let $f:\R\to\R$. We say that $f$ is \emph{Riemann integrable} if for every
	$\e>0$ there exist step functions $\phi$ and $\psi$ such that
	\begin{align*}
		\phi \leq f \leq \psi
	\end{align*}
	and
	\begin{align*}
		\int \psi - \int \phi < \e
	\end{align*}
	where $\int\psi$, $\int\phi$ are as in (\ref{step_integration}).
\end{definition}

\begin{theorem}[Notes 4.5]
	A function $f:\R\to\R$ is \emph{Riemann integrable} iff
	\begin{align*}
		\sup\left\lbrace\int\phi : \phi\text{ is a step function and }\phi\leq f\right\rbrace
		= \inf\left\lbrace\int\psi : \psi\text{ is a step function and }\psi \geq f\right\rbrace.
	\end{align*}
\end{theorem}

\begin{definition}
	If $f$ is Riemann integrable we define its Riemann integral $\int f\:dx$ as the common
	value
	\begin{align*}
		\int f\: dx
		=\sup\left\lbrace\int\phi : \phi\text{ is a step function and }\phi\leq f\right\rbrace
		= \inf\left\lbrace\int\psi : \psi\text{ is a step function and }\psi \geq f\right\rbrace.
	\end{align*}
\end{definition}

\begin{theorem}[Notes 4.6]
	Suppose that $f:\R\to\R$ is Riemann integrable. Then $f$ is also Lebesgue integrable on $\R$
	and moreover
	\begin{align*}
		\int f\:dx = \int f\:d\lambda.
	\end{align*}
\end{theorem}

\begin{lemma}[Notes 4.1]
	Let $f:\R\to\R$ be a bounded function with bounded support $[a,b]$. The following are equivalent:
	\begin{enumerate}
		\item $f$ is Riemann integrable.
		\item for every $\e>0$ there exist $a=x_0<\cdots x_n=b$ such that if $M_j$ and $m_j$
		      denote the supremum and infimum vallues of $f$ on $(x_{j-1}, x_j)$ respectively, then \begin{align*}
			      \sum_{j=1}^n (M_j-m_j)(x_j-x_{j-1})<\e.
		      \end{align*}
		\item for every $\e>0$ there exist $a=x_0<\cdots<x_n=b$ such that, with $I_j=(x_{j-1},x_j)$
		      for every $j\geq 1$, \begin{align*}
			      \sum_{j=1}^n \sup_{x,y\in I_j} \abs{f(x)-f(y)}\lambda(I_j)<\e.
		      \end{align*}
	\end{enumerate}
\end{lemma}

\begin{theorem}[Notes 4.7]
	Suppose that $g:[a,b]\to\R$ and let $f$ be defined by $f(x)=g(x)$ for $x\in[a,b]$ and $f(x)=0$
	otherwise.
	\begin{enumerate}
		\item If $g$ is continuous on $[a,b]$, then $f$ is Riemann integrable.
		\item If $g$ is a monotone function, then $f$ is Riemann integrable.
	\end{enumerate}
\end{theorem}

\begin{corollary*}
	Let $I$ be a bounded interval and suppose that there exists points
	\begin{align*}
		\inf I=x_0<x_1<\cdots<x_n=\sup I
	\end{align*}
	such that a function $f:I\to\R$ is bounded and continuous on each subinterval $(x_j,x_{j+1})$,
	$j=0,1,...,n$. Then such a function is Riemann integrable.
\end{corollary*}

\subsection{Dependence on an interval}

\begin{theorem}[Notes 4.8]
	Let $I$ and $J$ be two intervals such that $J\subset I$.
	\begin{enumerate}
		\item If $f$ is Lebesgue integrable on $I$ then $f$ is Lebesgue integrable on $J$.
		\item If $f$ is Lebesgue integrable on $J$ and simultaneously $f(x)=0$ for all $x\in I\setminus J$
		      then $f$ is Lebesgue integrable on $I$ and \begin{align*}
			      \int_J f\:d\lambda = \int_I f\:d\lambda.
		      \end{align*}
		\item If $f$ is Lebesgue integrable on $I$ and $f(x)\geq 0$ for all $x\in I$ then \begin{align*}
			      \int_J f\:d\lambda \leq \int_I f\:d\lambda.
		      \end{align*}
		\item Suppose that $I$ can be written as the union of disjoint intervals $I_n$ and let $f$
		      be Lebesgue integrable on each of the intervals $I_n$. Then $f$ is Lebesgue integrable on $I$ iff
		      \begin{align*}
			      \sum_{n=1}^\infty \int_{I_n} \abs f\:d\lambda < \infty.
		      \end{align*}
		      It then follows that \begin{align*}
			      \int_I f\:d\lambda = \sum_{n=1}^\infty \int_{I_n}f\:d\lambda.
		      \end{align*}
	\end{enumerate}
\end{theorem}

\begin{corollary}
	Let $f:\R\to\R$ be Lebesgue integrable on an interval $I$ with $a=\inf I$ and $b=\sup I$.
	Then $f$ is Lebesgue integrable on $(a,b)$, $(a,b]$, $[a,b)$, and $[a,b]$, and
	\begin{align*}
		\int_I f\:d\lambda
		= \int_{(a,b)} f\:d\lambda
		= \int_{(a,b]} f\:d\lambda
		= \int_{[a,b)} f\:d\lambda
		= \int_{[a,b]} f\:d\lambda.
	\end{align*}
	We define
	\begin{align*}
		\int_a^b f\:d\lambda := \int_If\:d\lambda.
	\end{align*}
\end{corollary}

\subsection{Fundamental Theorem of Calculus}

\begin{theorem}
	Let $I$ be an interval and let $g:I\to\R$ be Lebesgue integrable on $I$. For all $x\in I$ and
	some fixed $x_0\in I$ let $G(x) = \int_{x_0}^x g\:d\lambda$. Suppose $g$ is continuous at $x$
	for some $x\in I$. Then $G$ is differentiable at $x$ and $G'(x)=g(x)$.
\end{theorem}

\begin{theorem}
	Suppose $f:I\to\R$ has continuous derivative $f'$ on the interval $I$. Then, for any
	$a,b\in I$,
	\begin{align*}
		\int_a^b f'\:d\lambda= f(b) - f(a).
	\end{align*}
\end{theorem}

\subsection{Integrability of sequences of functions}

\begin{theorem}[Notes 4.3]
	Suppose that $(f_n)_{n\in\N}$ is a sequence of functions each of which is Lebesgue integrable on $I$.
	\begin{enumerate}
		\item Assume that \begin{align*}
			      \sum_{n=1}^\infty \int_I \abs{f_n}\:d\lambda < \infty.
		      \end{align*}
		      Let $f$ be a function on the interval $I$ such that \begin{align*}
			      f(x) = \sum_{n=1}^\infty f_n(x)\hs\text{for all $x\in I$ such that }\sum_{n=1}^\infty \abs{f_n(x)}<\infty.
		      \end{align*}
		      Then $f$ is Lebesgue integrable on $I$ and \begin{align*}
			      \int_I f\:d\lambda= \sum_{n=1}^\infty \int_I f_n\:d\lambda.
		      \end{align*}
		\item Assume that each $f_n\geq 0$ on $I$ and let $f(x)=\sum_{n=1}^\infty f_n(x)$ for all $x\in I$. Then $f$
		      is Lebesgue integrable on $I$ iff \begin{align*}
			      \sum_{n=1}^\infty \int_I f_n\:d\lambda< \infty.
		      \end{align*}
	\end{enumerate}
\end{theorem}

\begin{theorem}[Monotone Convergence]
	Suppose that $(f_n)$ is a monotone nondecreasing sequence of Lebesgue integrable functions on an interval $I$. That is
	$f_1(x) \leq f_2(x) \leq \cdots$ for all $x\in I$. Then let
	\begin{align*}
		f(x) = \lim_{n\to\infty}f_n(x),
	\end{align*}
	where we allow for the possibility that at some points this limit is infinite. Then $f$ is Lebesgue integrable on $I$
	iff
	\begin{align*}
		\sup_{n\in\N}\int_I f_n\:\lambda = \lim_{n\to\infty}\int_I f_n\:d\lambda< \infty.
	\end{align*}
	Then further,
	\begin{align*}
		\int_I f\:d\lambda= \lim_{n\to\infty} \int_I f_n\:d\lambda.
	\end{align*}
\end{theorem}

\begin{lemma}[Fatoux]
	Let $(f_n)$ be a sequence of nonnegative Lebesgue integrable functions on an interval $I$. Let
	\begin{align*}
		f(x) = \liminf_{n\to\infty} f_n(x), \hs\text{for all $x\in I$}.
	\end{align*}
	If $\liminf_{n\to\infty}\int_I f_n\:d\lambda<\infty$ then $f$ is Lebesgue integrable on $I$ and
	\begin{align*}
		\int_I f\:d\lambda\leq \liminf_{n\to\infty} \int_I f_n\:d\lambda.
	\end{align*}
\end{lemma}

\begin{theorem}[Dominated convergence]
	Let $(f_n)$ be a sequence of Lebesgue integrable functions on an interval $I$ and assume that
	\begin{align*}
		f(x) = \lim_{n\to\infty} f_n(x), \hs \text{for all $x\in I$}.
	\end{align*}
	Assume also that the sequence $(f_n)$ is dominated by some Lebesgue integrable function $g$, that is
	\begin{align*}
		\abs{f_n(x)} \leq g(x), \hs \text{for all $x\in I$ and $n\in\N$.}
	\end{align*}
	Then the function $f$ is Lebesgue integrable on $I$ and
	\begin{align*}
		\int_I f\:d\lambda= \lim_{n\to\infty} \int_I f_n\:d\lambda.
	\end{align*}
\end{theorem}

\begin{theorem}[Notes 4.13]
	Let $(a,b)$ be a bounded interval and suppose that $f_n:(a,b)\to\R$ are Lebesgue integrable functions
	which converge uniformly to a function $f$. Then $f$ is Lebesgue integrable on $(a,b)$ and
	\begin{align*}
		\int_a^b f\:d\lambda= \lim_{n\to\infty} \int_a^b f_n\:d\lambda.
	\end{align*}
\end{theorem}

\subsection{Cursed baby steps towards measure theory}

\begin{definition}
	Let $I\subseteq\R$ be an interval and $f:I\to\R$. We say that $f$ is measurable on $I$ if
	there exists a sequence of Lebesgue integrable functions $(f_n)$ on $I$ such that
	\begin{align*}
		f(x)=\lim_{n\to\infty} f_n(x) \hs\text{for all $x\in I\setminus E$}
	\end{align*}
	where $E$ is a set of measure zero.

	A set $E\subseteq\R$ is said to be measurable if $\chi_E$ is a measurable function
	on $\R$.
\end{definition}

\begin{definition}
	Let $E$ be a measruable set. Then we define the Lebesgue measure $\lambda(E)\in\R_{\geq 0}\cup\{\infty\}$ by
	\begin{align*}
		\lambda(E) := \int \chi_E\:d\lambda.
	\end{align*}
\end{definition}

\section{Fourier series and orthogonality}

\subsection{The space $L^2$}

\begin{definition}
	Let $g,h:\R\to\R$ be Lebesgue integrable on an interval $I$ and let $f=g+ih$. Then we
	define
	\begin{align*}
		\int_I	f\:d\lambda = \int_I g\:d\lambda + i\int_I h\:d\lambda.
	\end{align*}
\end{definition}

\begin{definition}
	Define the space $L^2=L^2([a,b])$ as the set of measurable functions $f:[a,b]\to\C$
	so that the function $x\mapsto \abs{f(x)}^2$ is Lebesgue integrable, i.e.
	\begin{align*}
		\vabs{f}_2^2 := \int_a^b \abs{f}^2 d\lambda < \infty.
	\end{align*}
	The quantity $\vabs{f}_2$ is called the \emph{$L^2$-norm} of $f$. If $\vabs{f}_2=1$,
	then we say that $f$ is \emph{$L^2$-normalised}.
\end{definition}

\begin{definition}
	For two functions $f,g\in L^2([a,b])$ we define the \emph{inner product} by
	\begin{align*}
		\lra{f,g} := \int_a^b f(x)\overline{g(x)}\:d\lambda(x).
	\end{align*}
\end{definition}

\begin{theorem}[Cauchy-Schwarz inequality]
	Let $f,g\in L^2([a,b])$. Then the function $x\mapsto f(x)\overline{g(x)}$ is Lebesgue
	integrable and we have
	\begin{align*}
		\abs{\lra{f,g}}\leq \vabs f_2\vabs g_2.
	\end{align*}
\end{theorem}

\begin{definition}
	Let $f,f_1,f_2,...$ be functions in $L^2([a,b])$. We say that the sequence $(f_n)_{n\in\N}$ converges
	to $f$ in $L^2$ if the sequence
	\begin{align*}
		\vabs{f_n - f}_2 = \left(\int_a^b \abs{f_n-f}^2\:d\lambda\right)^{1/2}
	\end{align*}
	converges to zero as $n\to\infty$. We also write $f_n\to f$ in $L^2$.
\end{definition}

\subsection{Orthonormal systems}

\begin{definition}
	A countable family $(\phi_j)_{j\in J}$ of functions $\phi_j\in L^2([a,b])$ is
	called an \emph{orthonormal system} iff, for all $j,j'\in J$,
	\begin{align*}
		\lra{\phi_j, \phi_{j'}} = \delta_{jj'}.
	\end{align*}
	Without loss of generality, we assume $J=\N_{\leq N}$ for some $N\in\N\cup\{\infty\}$.
\end{definition}

\begin{lemma}
	The following are orthonormal systems:
	\begin{enumerate}
		\item $\phi_n(x)=\chi_{[n-1,n)}$ for $N\in\N$ and $n\leq N$ on $[0,N]$,
		\item $\phi_n(x)=e^{2\pi inx}$ for $n\in\Z$ on $[0,1]$,
		\item $\phi_n(x)=\sqrt 2\cos(2\pi nx)$ for $n\in\Z$ on $[0,1]$,
		\item $\phi_n(x)=\sqrt 2\sin(2\pi nx)$ for $n\in\Z$ on $[0,1]$,
		\item $r_n(x)=\sgn(\sin(2^n\pi x))$ for $n\in\N$ on $[0,1]$.
	\end{enumerate}
\end{lemma}

\begin{theorem}[Notes 5.2]
	Let $(\phi_n)$ be an orthonormal system on $[a,b]$ and $f\in L^2$. Consider
	\begin{align}
		\label{sn}
		s_N(x) := \sum_{n=1}^N \lra{f,\phi_n}\phi_n(x).
	\end{align}
	Denote the linear span of the functions $(\phi_n)_{n\leq N}$ by $X_N$. Then
	\begin{align*}
		\vabs{f-s_N}_2 \leq \vabs{f-g}_2
	\end{align*}
	holds for all $g\in X_N$ with equality iff $g=s_N$.
\end{theorem}

\begin{theorem}[Bessel's inequality]
	If $(\phi_n)_{j\in J}$ is an orthonormal system on $[a,b]$ and $f\in L^2$ then
	\begin{align*}
		\sum_{j\in J} \abs{\lra{f,\phi_j}}^2 \leq \vabs f_2^2.
	\end{align*}
\end{theorem}

\begin{corollary}[Riemann-Lebesgue lemma in $L^2$]
	Let $(\phi_n)_{n\in\N}$ be an orthonormal system and $f\in L^2$, then
	\begin{align*}
		\lim_{n\to\infty} \lra{f,\phi_n}=0.
	\end{align*}
\end{corollary}

\begin{theorem}[Notes 5.4]
	Let $(\phi_n)$ be an orthonormal system on $[a,b]$. Let $(s_N)$ be as in (\ref{sn}).
	Then $(\phi_n)$ is complete iff $(s_N)$ converges to $f$ in the $L^2$-norm
	for every $f\in L^2$.
\end{theorem}

\subsection{Trigonometric polynomials}

\begin{definition}
	A \emph{trigonometric polynomial} is a function $f:\R\to\C$ given by
	\begin{align*}
		f(x) = \sum_{n=-N}^N c_ne^{2\pi i n x},
	\end{align*}
	where $N\in\N$ and $c_n\in\C$. If $c_N$ or $c_{-N}$ is non-zero, then $N$ is called the
	degree of $f$.
\end{definition}

\begin{definition}
	A function $f:\R\to\C$ is $\lambda$-periodic for some $k\in\R$ iff,
	for all $x\in\R$,
	\begin{align*}
		f(x) = f(x + \lambda).
	\end{align*}
\end{definition}

\begin{lemma}
	Every trigonometric polynomial is 1-periodic.
\end{lemma}

\begin{lemma}[Notes 5.1]
	$(e^{2\pi inx})_{n\in\Z}$ forms an orthonormal system on $[0,1]$. In particular,
	\begin{enumerate}
		\item for all $n\in\Z$, \begin{align*}
			      \int_0^1 e^{2\pi inx}\:d\lambda(x) = \delta_{0,n},
		      \end{align*}
		\item if $f(x)=\sum_{n=-N}^N c_ne^{2\pi inx}$ is a trigonometric polynomial
		      then \begin{align*}
			      c_n = \lra{f,\phi_n}.
		      \end{align*}
	\end{enumerate}
\end{lemma}

\begin{definition}
	For a $1$-periodic Lebesgue integrable function $f$ and $n\in\Z$ we define the \emph{nth Fourier coefficient}
	by
	\begin{align*}
		\hat f(n) = \lra{f, \phi_n}.
	\end{align*}
	The double infinite series
	\begin{align*}
		\sum_{n=-\infty}^\infty \hat f(n) e^{2\pi inx}
	\end{align*}
	is called the \emph{Fourier series} of $f$.
\end{definition}

\begin{definition}
	For a $1$-periodic Lebesgue integrable function $f$ we define the partial sums
	\begin{align*}
		S_Nf(x):=\sum_{n=-N}^N \hat f(n) e^{2\pi inx}.
	\end{align*}
\end{definition}

\subsection{Convolutions and kernels}

\begin{definition}
	For two $1$-periodic functions $f,g\in L^2$ we define their \emph{convolutions}
	by
	\begin{align*}
		(f*g)(x)=\int_0^1 f(t)g(x-t)\:d\lambda(t).
	\end{align*}
\end{definition}

\begin{lemma}[Notes 5.2]
	For $1$-periodic functions $f,g\in L^2$,
	\begin{align*}
		f*g=g*f.
	\end{align*}
\end{lemma}

\begin{definition}
	The \emph{Dirichlet kernel} is the sequence of functions
	$(D_N)$ defined by
	\begin{align*}
		D_N(x):=\sum_{n=-N}^N e^{2\pi inx} \hs \text{for all }N\in\N.
	\end{align*}
\end{definition}

\begin{lemma}
	Let $f\in L^2$ be 1-periodic. Then
	\begin{align*}
		S_Nf(x) = (f*D_N)(x).
	\end{align*}
\end{lemma}

\begin{lemma}[Notes 5.3]
	For all $N\in\N$,
	\begin{align*}
		D_N(x)=\frac{\sin(2\pi(N+1/2)x)}{\sin(\pi x)}.
	\end{align*}
\end{lemma}

\begin{definition}
	The \emph{Fej\'er kernel} is the sequence of functions
	$(K_N)$ defined by
	\begin{align*}
		K_N(x)=\frac{1}{N+1}\sum_{n=0}^ND_n(x).
	\end{align*}
\end{definition}

\begin{lemma}[Notes 5.4]
	For all $N\in\N$,
	\begin{align*}
		K_N(x)=\frac{1}{2(N+1)}\frac{1-\cos(2\pi(N+1)x)}{\sin^2(\pi x)}
		=\frac{1}{N+1}\left(\frac{\sin(\pi(N+1)x)}{\sin(\pi x)}\right)^2.
	\end{align*}
\end{lemma}

\begin{theorem}[Fej\'er]
	For every 1-periodic continuous function $f$,
	\begin{align*}
		K_N*f\to f
	\end{align*}
	uniformly on $\R$ as $N\to \infty$.
\end{theorem}

\begin{corollary*}
	Every 1-periodic continuous function can be uniformly approximated by trigonometric
	polynomials. That is, for every 1-periodic continuous $f$ there exists a sequence
	$(f_n)$ of trigonometric polynomials such that $f_n\to f$ uniformly.
\end{corollary*}

\subsection{Approximations of unity}

\begin{definition}
	A sequence of 1-periodic Lebesgue integrable functions $(k_n)_{n\in\N}$ is called \emph{approximation
		of unity} if for all 1-periodic continuous functions $f$ we have that $f*k_n$ converges
	uniformly to $f$ on $\R$. That is, as $n\to\infty$,
	\begin{align*}
		\sup_{x\in\R}\abs{(f*k_n-f)(x)}\to 0.
	\end{align*}
\end{definition}

\begin{lemma}
	There is no continuous $k:\R\to\C$ such that, for all continuous, 1-periodic $f:\R\to\C$,
	$k*f=f$.
\end{lemma}

\begin{lemma}
	Let $(k_n)_{n\in\N}$ an approximation of unity and $f:\R\to\C$ continuous and 1-periodic. Then
	\begin{align*}
		\lim_{n\to\infty} k_n * f = f.
	\end{align*}
\end{lemma}

\begin{theorem}[Notes 5.6]
	Let $(k_n)_{n\in\N}$ a sequence of Lebesgue integrable 1-periodic functions such that
	\begin{enumerate}
		\item $k_n(x)>0$ for all $n\in\N$ and $x\in\R$,
		\item $\int_{-1/2}^{1/2} k_n\:d\lambda = 1$ for all $n\in\N$, and
		\item for all $0<\delta\leq 1/2$, \begin{align*}
			      \lim_{n\to\infty}\int_{-\delta}^\delta k_n\:d\lambda = 1.
		      \end{align*}
	\end{enumerate}
	Then $(k_n)_{n\in\N}$ is an approximation of unity.
\end{theorem}

\begin{corollary}
	The Fej\'er kernel $(K_N)_{n\in\N}$ is an approximation of unity.
\end{corollary}

\subsection{Convergence of Fourier series}

\begin{lemma}[Notes 5.5]
	Let $f:\R\to\C$ be 1-periodic and continuous. Then
	\begin{align*}
		\lim_{N\to\infty} \vabs{S_Nf-f}_2 = 0.
	\end{align*}
\end{lemma}

\begin{theorem}[Notes 5.7]
	The trigonometric system is complete. In particular, for all 1-periodic $f\in L^2$,
	\begin{align*}
		\lim_{N\to\infty}\vabs{S_Nf-f}_2=0.
	\end{align*}
\end{theorem}

\begin{theorem}[Parseval]
	If $f,g\in L^2$ are 1-periodic then
	\begin{align*}
		\lra{f,g} = \sum_{n=-\infty}^\infty \hat f(n)\overline{g(n)}.
	\end{align*}
\end{theorem}

\begin{corollary*}
	If $f,g\in L^2$ are 1-periodic then
	\begin{align*}
		\vabs{f}_2^2 = \sum_{n=-\infty}^\infty \abs{\hat f(n)}^2.
	\end{align*}
\end{corollary*}

\begin{theorem}
	Let $f$ be a continuous 1-periodic function and let $x\in\R$. If $f$ is differentiable
	at $x$ then $S_Nf(x)\to f(x)$ as $N\to\infty$.
\end{theorem}

\end{document}
