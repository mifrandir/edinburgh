\documentclass{article}
\usepackage{homework-preamble}

\begin{document}
\title{Geometry: Hand-in 4}
\author{Franz Miltz (UUN: S1971811)}
\date{22 November 2021}
\maketitle
\noindent Let $D=\{u,v,z\in\R : u,v\not=0\}$ and let $\vec x:D\to\E^3$ be such that
\begin{align*}
	\vec x(u,v,z) = \begin{pmatrix}
		uv \\ \frac{1}{2}(v^2 - u^2) \\ z
	\end{pmatrix}
\end{align*}

\section*{Question 1}

\begin{claim*}
	\begin{align}
		\label{dx}
		d\vec x = \sqrt{u^2 + v^2}du\vec e_1
		+ \sqrt{u^2 + v^2}dv \vec e_2
		+ dz\vec e_3
	\end{align}
	where
	\begin{align*}
		\vec e_1 = \frac{1}{\sqrt{u^2 + v^2}}\begin{pmatrix}
			v \\ -u \\ 0
		\end{pmatrix} , \hs
		\vec e_2 = \frac{1}{\sqrt{u^2 + v^2}}\begin{pmatrix}
			u \\ v \\ 0
		\end{pmatrix}, \hs
		\vec e_3 = \begin{pmatrix}
			0 \\ 0 \\ 1
		\end{pmatrix}
	\end{align*}
	are a moving frame on $D$.
\end{claim*}
\begin{proof}
	Firstly, we calculate
	\begin{align*}
		d\vec x = \begin{pmatrix}
			v du + u dv  \\
			-u du + v dv \\
			dz
		\end{pmatrix}.
	\end{align*}
	It's straightforward to verify algebraically that (\ref{dx}) holds. Secondly,
	we note that $u,v\not=0$ and therefore that the $\vec e_i$ are well-defined on $D$
	and smooth. Lastly, we observe that
	\begin{align}
		\vec e_i \cdot \vec e_j = \delta_i^j,
	\end{align}
	making $\{\vec e_1, \vec e_2, \vec e_3\}$ a moving frame.
\end{proof}

\section*{Question 2}

\begin{claim*}
	The connection 1-forms are
	\begin{align*}
		\omega_2^1 = \frac{vdu - udv}{u^2 + v^2},\hs
		\omega_3^1 = \omega_3^2 = 0.
	\end{align*}
\end{claim*}
\begin{proof}
	We begin by computing
	\begin{align*}
		d\vec e_2 = \frac{1}{(u^2+v^2)^{3/2}}\begin{pmatrix}
			v^2du - uvdv   \\
			- uvdu + u^2dv \\
			0
		\end{pmatrix}.
	\end{align*}
	This lets us determine
	\begin{align*}
		\omega_2^1 = \vec e_1 \cdot d\vec e_2 = \frac{vdu - udv}{u^2 + v^2}.
	\end{align*}
	Finally, we note that $d\vec e_3 = \vec 0$ and thus
	\begin{align}
		\label{omega0}
		\omega_3^1 = \vec e_1 \cdot d\vec e_3=0, \hs \omega_3^2 = \vec e_2 \cdot d\vec e_3=0.
	\end{align}
\end{proof}

\begin{claim*}
	The structure equations,
	\begin{align}
		\label{se1}
		d\theta^1 + \omega_2^1 \wedge \theta^2 = 0,              \\
		\label{se2}
		d\theta^2 + \omega_1^2 \wedge \theta^1 = 0,              \\
		\label{se3}
		\omega_1^3\wedge\theta^1 + \omega_2^3\wedge\theta^2 = 0, \\
		\label{se4}
		d\omega_2^1+\omega_3^1 \wedge\omega_2^3 = 0,             \\
		\label{se5}
		d\omega_3^1 + \omega_2^1 \wedge \omega_3^2 = 0,          \\
		\label{se6}
		d\omega_3^2 + \omega_1^2 \wedge \omega_3^1 = 0,
	\end{align}
	are satisfied.
\end{claim*}
\begin{proof}
	Consider (\ref{se1}). We compute
	\begin{align*}
		d\theta^1 = \frac{v}{\sqrt{u^2+v^2}}dv\wedge du,\hs
		\omega_2^1 \wedge \theta^2 = \frac{v}{\sqrt{u^2+v^2}}du\wedge dv.
	\end{align*}
	Since $dv\wedge du = - du \wedge dv$, the equation holds.
	Consider (\ref{se2}). We compute
	\begin{align*}
		d\theta^2 = \frac{u}{\sqrt{u^2+v^2}}du\wedge dv, \hs
		\omega_1^2 \wedge \theta^1 = - \frac{u}{\sqrt{u^2 + v^2}}du\wedge dv
	\end{align*}
	where we used $\omega_1^2 = - \omega_2^1$. The equation clearly holds.
	Now we observe that (\ref{se3}), (\ref{se5}) and (\ref{se6}) depend on
	$\omega_3^2=\omega_3^1=0$ in such a way that they are trivially satisfied.
	Further, (\ref{se4}) simplifies to
	\begin{align*}
		d\omega_2^1 = 0.
	\end{align*}
	We calculate
	\begin{align*}
		d\omega_2^1 = \frac{u^2-v^2}{u^2+v^2}dv\wedge du + \frac{u^2-v^2}{u^2+v^2}du\wedge dv.
	\end{align*}
	Again, by $dv\wedge du = - du \wedge dv$ this clearly holds.
\end{proof}

\end{document}