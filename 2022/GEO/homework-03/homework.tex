\documentclass{article}
\usepackage{homework-preamble}

\begin{document}
\title{Geometry: Hand-in 3}
\author{Franz Miltz (UUN: S1971811)}
\date{8 November 2021}
\maketitle

\section*{Question 1}

\emph{The proof below is obviously wrong as $x^n$ may be zero,
	but I have not been able to find smooth parameters for $\beta$ that are defined everywhere.}

\begin{claim*}
	Let $\alpha\in\Omega^1(\R^n)$ be a $1$-form such that
	\begin{align*}
		\alpha=\sum_{i=1}^n x^idx^i.
	\end{align*}
	Then there exists $\beta\in\Omega^{n-1}(\R^n\setminus\{\vec 0\})$ such that
	\begin{align*}
		\alpha\wedge\beta = dx^1\wedge\cdots\wedge dx^n.
	\end{align*}
\end{claim*}
\begin{proof}
	Let
	\begin{align*}
		\beta = \frac{(-1)^{n-1}}{x^n}dx^1\wedge\cdots\wedge dx^{n-1}.
	\end{align*}
	Then we have
	\begin{align*}
		\alpha\wedge\beta & = \left(\sum_{i=1}^n x^idx^i\right) \wedge \left(\frac{(-1)^{n-1}}{x^n}dx^1\wedge\cdots\wedge dx^{n-1}\right) \\
		                  & = \sum_{i=1}^n \frac{(-1)^{n-1}x^i}{x^n}dx^i\wedge dx^1 \wedge \cdots dx^{n-1}.
	\end{align*}
	However, for all $1\leq i\leq n-1$ the wedge product is zero due to the repeated $dx^i$. Thus
	\begin{align*}
		\alpha\wedge\beta = \frac{(-1)^{n-1}x^n}{x^n}dx^n\wedge dx^1\wedge\cdots\wedge dx^{n-1}=dx^1\wedge\cdots\wedge dx^n.
	\end{align*}
\end{proof}

\section*{Question 2}

Let $\vec x: D\to\E^3$ be a local surface defined by $D=\{z\in\R,\phi\in(0,2\pi)\}\subset\R^3$ and
\begin{align*}
	\vec x(z,\phi)=\begin{pmatrix}
		\cosh z \cos\phi \\ \cosh z \sin\phi \\ z
	\end{pmatrix}.
\end{align*}

\begin{claim*}
	$\vec x$ is regular and the first fundamental form is
	\begin{align*}
		I= (\cosh^2 z)(dz)^2 + (\cosh^2 z) (d\phi)^2.
	\end{align*}
\end{claim*}
\begin{proof}
	We calculate
	\begin{align}
		\label{dirderiv1}
		\vec x_z    & = \sinh z \cos\phi \frac{\p }{\p x^1}
		+ \sinh z \sin\phi \frac{\p }{\p x^2}
		+ \frac{\p }{\p x^3},                                \\
		\label{dirderiv2}
		\vec x_\phi & = -\cosh z \sin\phi \frac{\p }{\p x^1}
		+ \cosh z\cos\phi\frac{\p }{\p x^2}.
	\end{align}
	Firstly, we observe that both $\vec x_z$ and $\vec x_\phi$ are non-zero on $D$.
	Further,
	\begin{align*}
		\vec x_z \cdot \vec x_\phi = -\sinh z\cosh z\sin\phi\cos\phi + \sinh z\cosh z\sin\phi\cos\phi
		= 0.
	\end{align*}
	Therfore $\vec x_z$ and $\vec x_\phi$ are non-zero and perpendicular everywhere, making them linearly
	independent. By definition, this shows that $\vec x$ is a regular local surface.
	Further, we have
	\begin{align*}
		(dx^1)^2 & = (\sinh z\cos\phi dz - \cosh z \sin\phi d\phi)^2                                                     \\
		         & =\sinh^2 z \cos^2\phi(dz)^2
		-2\sinh z\cosh z\sin\phi\cos\phi dzd\phi+\cosh^2z\sin^2\phi(d\phi)^2,                                            \\
		(dx^2)^2 & = (\sinh z\sin\phi dz + \cosh z \cos\phi d\phi)^2                                                     \\
		         & =\sinh^2 z\sin^2\phi (dz)^2 + 2\sinh z\cosh z\sin\phi\cos\phi dzd\phi + \cosh^2 z\cos^2\phi(d\phi)^2, \\
		(dx^3)^2 & =(dz)^2.
	\end{align*}
	Therefore, by definition,
	\begin{align*}
		I=d\vec x\cdot d\vec x = (\cosh^2 z)(dz)^2 + (\cosh^2 z) (d\phi)^2.
	\end{align*}
\end{proof}

\begin{claim*}
	The unit normal of $\vec x$ at a point $(z,\phi)\in D$ is
	\begin{align*}
		\vec N(z, \phi) = \frac{-\cos\phi \frac{\p }{\p x^1}-\sin\phi\frac{\p }{\p x^2}+\sinh z\frac{\p }{\p x^3}}{\cosh z}.
	\end{align*}
\end{claim*}
\begin{proof}
	By definition, we have
	\begin{align*}
		\vec N(p) = \frac{\vec x_z \times \vec x_\phi}{\abs{\vec x_z \times \vec x_\phi}}.
	\end{align*}
	We use (\ref{dirderiv1}) and (\ref{dirderiv2}) to claculate
	\begin{align*}
		\vec x_z\times \vec x_\phi = -\cosh z \cos\phi \frac{\p }{\p x^1}-\cosh z \sin\phi\frac{\p }{\p x^2}+\sinh z\cosh z \frac{\p }{\p x^3}
	\end{align*}
	and then
	\begin{align*}
		\abs{\vec x_z\times \vec x_\phi} = \sqrt{(1 + \sinh^2 z)\cosh^2 z} = \cosh^2 z.
	\end{align*}
	The claim follows after cancellation.
\end{proof}

\begin{claim*}
	The second fundamental form of $\vec x$ at a point $(z,\phi)\in D$ is
	\begin{align*}
		\I = -(dz)^2+(d\phi)^2.
	\end{align*}
\end{claim*}

\begin{proof}
	We calculate
	\begin{align*}
		dN_1 & = d\left(-\frac{\cos\phi}{\cosh z}\right) = \left(\frac{\sinh z\cos\phi}{\cosh^2z}\right)dz + \left(\frac{\sin\phi}{\cosh z}\right)d\phi,  \\
		dN_2 & = d\left(-\frac{\sin\phi}{\cosh z}\right) = \left(\frac{\sinh z\sin\phi}{\cosh^2 z}\right)dz - \left(\frac{\cos\phi}{\cosh z}\right)d\phi, \\
		dN_3 & = d\left(\frac{\sinh z}{\cosh z}\right)   = \left(\frac{1}{\cosh^2 z}\right)dz
	\end{align*}
	We use this and our calculations for $d\vec x$ to find
	\begin{align*}
		\I = -d\vec x \cdot d\vec N = -(dz)^2 +(d\phi)^2.
	\end{align*}
\end{proof}

\begin{claim*}
	The principal curvatures of $\vec x$ at $(z,\phi)\in D$ are
	\begin{align*}
		k_1 = \frac{1}{\cosh^2 z}, \hs k_2 = -\frac{1}{\cosh^2 z}
	\end{align*}
	and thus
	\begin{align*}
		H=0,\hs K=-\frac{1}{\cosh^4 z}.
	\end{align*}
\end{claim*}
\begin{proof}
	The first and second fundamental form are represented by the following
	matrices:
	\begin{align*}
		I = \begin{pmatrix}
			\cosh^2 z & 0 \\ 0 & \cosh^2 z
		\end{pmatrix}, \hs
		\I = \begin{pmatrix}
			-1 & 0 \\ 0 & 1
		\end{pmatrix}.
	\end{align*}
	Thus the principal curvatures are the solutions to the equation
	\begin{align*}
		\det(\I-kI)=0
	\end{align*}
	which holds if and only if
	\begin{align*}
		(1+k\cosh^2 z)(1-k\cosh^2 z) = 0.
	\end{align*}
	Thus $k_1=1/\cosh^2 z$ and $k_2=-1/\cosh^2 z$.
	By definition, we then have
	\begin{align*}
		H & =\frac{1}{2}(k_1+k_2)=0,      \\
		K & =k_1k_2=-\frac{1}{\cosh^4 z}.
	\end{align*}
\end{proof}

\section*{Question 3}

\begin{claim*}
	Let $\Sigma\subset\E^3$ be a regular surface, that meets a plane $P\subset\E^3$ in a
	single point $\vec p$. Then $P$ coincides with the tangent plane of $\Sigma$ at $\vec p$,
	$T_{\vec p}\Sigma$.
\end{claim*}
\begin{proof}
	Let $P$ be given by the equation
	\begin{align*}
		\vec n \cdot (\vec x - \vec p) = 0 \hs\text{for all }\vec x\in P
	\end{align*}
	where we assume without loss of generality that $\vec n$ is a unit vector. Then the
	distance of any point $\vec x\in\E^3$ the distance to $P$ is
	\begin{align*}
		D(\vec x) = \vec n \cdot (\vec x - \vec p).
	\end{align*}
	Further, since $\Sigma$ is regular, there exists a local regular surface
	$\vec y: D\subset\R^3\to \E^3$ such that $\vec p \in\vec y(D)$ and $\vec y(D)=U\cap\Sigma$
	for some open set $U\subset\R^3$.
	The distance of a point on this surface given by $(u,v)\in D$ then is
	\begin{align*}
		D(\vec y(u,v)) = \vec n \cdot (\vec y(u,v) - \vec p) =: g(u,v).
	\end{align*}
	Consider the Jacobian matrix
	\begin{align*}
		J_g =
		\begin{bmatrix}
			\frac{\p g}{\p u} & \frac{\p g}{\p v}
		\end{bmatrix}.
	\end{align*}
	Note that the $1\times1$ matrix $\left[\eval{\frac{\p g}{\p v}}{q}\right]$ is invertible
	at $q\in D$ where $\vec y(q) = \vec p$ if and only if
	\begin{align}
		\label{non_tangent}
		\eval{\frac{\p g}{\p v}}{q} = \vec n \cdot \vec y_v(q)\not= 0.
	\end{align}
	Assume, for contradiction, that $P$ does not coincide with the tangent space
	$T_{\vec p}\Sigma$. Then we know that at $q$ either
	\begin{align*}
		\vec n \cdot \vec y_u(q) \not=0, \hs\text{or}\hs \vec n \cdot \vec y_v(q) \not=0.
	\end{align*}
	This is because $\vec y$ is regular so $\vec y_u$ and $\vec y_v$ must be linearly independent
	and span $T_{\vec p} \Sigma$. Since we could reparameterise to swap $\vec y_u$ and $\vec y_v$,
	we can assume without loss of generality that the latter inequality holds.
	We therefore have (\ref{non_tangent}).

	By the \emph{Implicit Function Theorem} there must exist a function
	\begin{align*}
		\Phi : E \subset \R \to \R
	\end{align*}
	defined on an open set $E\subset \R$ such that there exists an open set $\tilde D$
	with $q\in \tilde D\subset D$ such that
	\begin{align*}
		\tilde D \cap (\R \subset \R^2) = E,
	\end{align*}
	and such that for all $\tilde q \in \tilde D$ we have
	\begin{align*}
		g(\tilde q) = g(q) \hs\Leftrightarrow\hs \tilde q = (r, \Phi(r)) \hs \text{for some $r\in E$}.
	\end{align*}
	Since $E$ is open, there exists more than one such $r$. This implies that there
	is more than one point on $\Sigma$ where the distance to $P$ is zero which contradicts the premise.
	Therefore $P$ and $T_{\vec p}\Sigma$ must coincide.
\end{proof}


\end{document}