\documentclass{article}
\usepackage{notes-preamble}
\usepackage{enumitem}
\usepackage{tikz}
\usetikzlibrary{shapes,positioning}
\mkthmstwounified

\title{Compiling Techniques (SEM6)}
\author{Franz Miltz}
\begin{document}
\maketitle
\tableofcontents
\pagebreak

\section{Introduction}

\begin{definition}
	An \emph{interpreter} is a program that directly executes an executable program,
	producing the results of executing that program.
\end{definition}
\begin{definition}
	A \emph{compiler} is a program that translates an executable program in one language
	to an executable program in another language. The compiler may improve the program
	in some way.
\end{definition}

\paragraph{Problems posed by compiler construction}
\begin{itemize}
	\item compile-time performance: How long does it take to compile a program?
	\item run-time performance: How fast is the resulting executable?
	\item programming language complexity: Are special features being handled well?
	\item machine complexity: Does the compiler adapt to underlying hardware?
	\item user interface: Is the user exposed to the compiler infrastructure?
\end{itemize}

\paragraph{Specification}
Any compiler must
\begin{enumerate}
	\item recognise legal and illegal programs,
	\item generate correct code,
	\item manage storage of all variables and code,
	\item agree with OS \& linker on format for object code, and
	\item use higher level notations.
\end{enumerate}

\paragraph{Related programs} to the compiler are
\begin{itemize}
	\item \emph{pre-processor}: processes macros and directives,
	\item \emph{assembler}: translates assembly language to actual machine code,
	\item \emph{linker}: links together various compiled files and/or libraries,
	\item \emph{debugger}: tightly integrated with compiler; uses meta-information,
	\item \emph{virtual machines}: executes virtual assembly.
\end{itemize}

\section{Structure}

\tikzset{
	interface/.style={
			minimum width=0cm,
			minimum height=1cm
		},
	component/.style={
			draw,
			minimum width=2cm,
			minimum height=1cm
		}
}

\begin{definition}
	A \emph{two-pass compiler} consists of a frontend which converts the source code
	into an \emph{intermediate representation} (IR), catching all the errors that may occur,
	and a backend which converts the IR into machine code.

	\begin{center}
		\begin{tikzpicture}[font=\small,thick]
			\node[interface] (sourcecode) {Source Code};
			\node[component, right=of sourcecode] (frontend) {Front-End};
			\node[component, right=of frontend] (backend) {Back-End};
			\node[interface, right=of backend] (machinecode) {Machine Code};
			\node[interface, below=0cm of machinecode] (errors) {Errors};
			\draw[-latex]
			(sourcecode) edge (frontend)
			(frontend) edge node[above]{IR} (backend)
			(backend) edge (machinecode)
			(frontend) |- (errors);
		\end{tikzpicture}
	\end{center}
\end{definition}

\begin{definition}
	A \emph{three-pass compiler} consists of the usual frontend and backend as well as
	a middle-end which is tasked with improving efficiency (e.g. with respect to time,
	memory, ...) of the compiled code.

	\begin{center}
		\begin{tikzpicture}[font=\small,thick]
			\node[interface] (sourcecode) {Source Code};
			\node[component, right=of sourcecode] (frontend) {Front-End};
			\node[component, right=of frontend] (midend) {Middle-End};
			\node[component, right=of midend] (backend) {Back-End};
			\node[interface, right=of backend] (machinecode) {Machine Code};
			\node[interface, below=0cm of machinecode] (errors) {Errors};
			\draw[-latex]
			(sourcecode) edge (frontend)
			(frontend) edge node[above]{IR} (midend)
			(midend) edge node[above]{IR} (backend)
			(backend) edge (machinecode)
			(frontend) |- (errors);
		\end{tikzpicture}
	\end{center}
\end{definition}

\subsection{Frontend}

\begin{center}
	\begin{tikzpicture}[font=\small,thick]
		\node[interface] (sourcecode) {Source};
		\node[component, right=.5cm of sourcecode] (lexer) {Lexer};
		\node[component, right=.5cm of lexer] (parser) {Parser};
		\node[component, right=.5cm of parser] (semanalyser) {Semantic Analyser};
		\node[component, right=.5cm of semanalyser] (irgenerator) {IR generator};
		\node[interface, right=.5cm of irgenerator] (ir) {IR};
		\node[interface, below=0cm of ir] (errors) {Errors};
		\draw[-latex]
		(sourcecode) edge (lexer)
		(lexer) edge node[above of=1cm]{token} (parser)
		(parser) edge node[above of=1cm]{AST} (semanalyser)
		(semanalyser) edge node[above of=1cm]{AST} (irgenerator)
		(irgenerator) edge (ir)
		(lexer) |- (errors)
		(parser) |- (errors)
		(semanalyser) |- (errors)
		(irgenerator) |- (errors);
	\end{tikzpicture}
\end{center}

\paragraph{Specification} The front-end must
\begin{enumerate}
	\item Recoginise legal and illegal programs,
	\item report errors in a useful way,
	\item produce IR and preliminary storage map,
	\item shape the code for the back end.
\end{enumerate}

\begin{definition}
	An \emph{abstract syntax tree} (AST) is a tree representing an expression with
	respect to a grammar. It is typically much more concise than an a parse tree and
	does not contain any unnecessary information.
\end{definition}

\paragraph{Components} Typically, a front-end consists of
\begin{enumerate}
	\item \emph{lexer}: converts source code to token stream and collects identifier
	      information; typically consists of \emph{scanner} and \emph{tokeniser}
	\item \emph{parser}: recognises context-free syntax and converts token stream
	      into AST
	\item \emph{semantic analyser}: performs context-sensitive analysis, e.g. variable names, types, parentheses
	\item \emph{IR generator}: converts AST to IR; may not require any work
\end{enumerate}

\subsection{Backend}
\begin{center}
	\begin{tikzpicture}[font=\small,thick]
		\node[interface] (ir) {IR};
		\node[component, right=.5cm of ir] (selection) {Instruction Selection};
		\node[component, right=.5cm of selection] (registers) {Register Allocation};
		\node[component, right=.5cm of registers] (scheduling) {Instruction Scheduling};
		\node[interface, right=.5cm of scheduling] (machinecode) {Machine Code};
		\node[interface, below=0cm of machinecode] (errors) {Errors};
		\draw[-latex]
		(ir) edge (selection)
		(selection) edge node[above of=1cm]{IR} (registers)
		(registers) edge node[above of=1cm]{IR} (scheduling)
		(scheduling) edge (machinecode)
		(selection) |- (errors)
		(registers) |- (errors)
		(scheduling) |- (errors);
	\end{tikzpicture}
\end{center}

\paragraph{Components} Typically, a backend contains the following phases:
\begin{enumerate}
	\item \emph{instruction selection}: takes advantage of target features such as addressing modes,
	\item \emph{register allocaiton}: manages limited number of registers for optimal performance,
	\item \emph{instruction scheduling}: see instruction selection.
\end{enumerate}

\end{document}