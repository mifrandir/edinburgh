\documentclass{article}
\usepackage{homework-preamble}

\begin{document}
\title{Honours Analysis: Homework 9}
\author{Franz Miltz (UUN: S1971811)}
\date{25 November 2021}
\maketitle

\section*{Workshop 9 - Question 2}

\begin{claim*}
	Let $f:[a,b]\to\R$. If $f$ is Riemann integrable, then so is $\abs f$.
\end{claim*}
\begin{proof}
	Let $\e>0$. Assume $f$ is Riemann integrable. Then there exists a partition
	$a=x_0<\cdots<x_n=b$ such that
	\begin{align*}
		\sum_{j=1}^n \sup_{x,y\in I_j} \abs{f(x) - f(y)} \lambda(I_j) < \e,\hs
		\text{where }I_j = (x_{j-1}, x_j).
	\end{align*}
	Observe for all $x,y\in[a,b]$,
	\begin{align*}
		\abs{f(x)-f(y)} \geq \abs{\abs{f(x)} - \abs{f(y)}}.
	\end{align*}
	Thus
	\begin{align*}
		\sum_{j=1}^n \sup_{x,y\in I_j} \abs{\abs{f(x)} - \abs{f(y)}} \lambda(I_j)
		\leq \sum_{j=1}^n \sup_{x,y\in I_j} \abs{f(x) - f(y)} \lambda(I_j) < \e.
	\end{align*}
	Therefore $\abs f$ is Riemann integrable.
\end{proof}

\begin{claim*}
	There exists $f:[a,b]\to\R$ such that $\abs f$ is integrable but $f$ is not.
\end{claim*}
\begin{proof}
	Consider $f = 2\chi_{\Q\cap[0,1]}-1$. Observe that $\abs{f(x)} = 1$ for all $x\in[0,1]$.
	Therefore $\abs f$ is Riemann integrable. However, $f$ itself cannot be Riemann integrable
	as $\chi_{\Q\cap[0,1]} = (f+1)/2$ and we know that the Dirichlet function is not Riemann
	integrable.
\end{proof}

\section*{Workshop 9 - Question 3 \& 6}

\begin{claim*}
	Let $-\infty \leq a < b \leq \infty$. Suppose that $f:(a,b)\to\R$ is integrable
	on $(u,b)$ for all $u\in(a,b)$. Then $f$ is integrable on $(a,b)$ if and only if
	there exists $M<\infty$ such that
	\begin{align}
		\label{cond}
		\int_u^a \abs f \leq M, \hs \forall u \in (a,b).
	\end{align}
	Additionally, if this holds then
	\begin{align}
		\label{intval}
		\int_a^b f = \lim_{u\to b^-}\int_u^b f.
	\end{align}
\end{claim*}
\begin{proof}
	($\Rightarrow$) Assume $f$ is integrable on $(a,b)$. Immediately, $\abs f$ must be integrable
	on $(a,b)$ as well. By \emph{Notes, Theorem 4.8 (c)} we then have
	\begin{align*}
		\int_u^b \abs f \leq \int_a^b \abs f = M,
	\end{align*}
	for all $u\in(a,b)$ which is precisely (\ref{cond}).

	($\Leftarrow$) Assume (\ref{cond}) holds for some $M<\infty$. Consider a sequence of numbers
	$b>u_1>u_2>\cdots>a$ such that $u_n\to a$ as $n\to\infty$. Let $u_0=b$.
	By \emph{Notes, Theorem 4.8 (d)}, for all $n$ we have
	\begin{align*}
		\sum_{j=1}^n \int_{u_j}^{u_{j-1}} \abs f = \int_{u_n}^b \abs f.
	\end{align*}
	Using \emph{Notes, Theorem 4.8 (c)}, we then observe that (\ref{cond}) implies
	\begin{align}
		\label{newcond}
		\sum_{j=1}^n \int_{u_j}^{u_{j-1}} \abs f \leq M, \hs \text{for }n=1,2,...
	\end{align}
	Further, since the above holds, we also have
	\begin{align*}
		\sum_{j=1}^\infty \int_{u_j}^{u_{j-1}} \abs f \leq M,
	\end{align*}
	and hence by \emph{Notes, Theorem 4.8 (d)} the function $f$ is integrable on
	the union of intervals $(u_j, u_{j-1})$ which is $(a,b)$. By the same theorem,
	we also get (\ref{intval}).
\end{proof}

\section*{Workshop 9 - Question 10}

\begin{claim*}
	The function $f(x)=(-1)^{\floor{x}} / \floor{x}$ for $x\geq 1$ is not integrable
	on $[1, \infty)$.
\end{claim*}
\begin{proof}
	Consider the sequence $(v_n)$ given by $v_n = n$. Observe that in order for
	$f$ to be integrable on $[1,\infty)$, we require
	\begin{align*}
		\sum_{j=1}^n \int_{v_j}^{v_{j+1}} \abs f \leq M, \hs \text{for }n = 1,2,...
	\end{align*}
	for some $M<\infty$ as proven on the worksheet. However, for all $j$,
	\begin{align*}
		\sum_{j=1}^n \int_{v_j}^{v_{j+1}} \abs f
		= \sum_{j=1}^n \frac{1}{j}
	\end{align*}
	which is clearly unbounded as $n\to\infty$.
\end{proof}

\emph{Note: So the problem with the given function is that we require the integrals of the
	absolute values to be bounded which is not the case. The given limit exists due to the alternating
	signs.}

\end{document}