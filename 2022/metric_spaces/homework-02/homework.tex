\documentclass{article}
\usepackage{homework-preamble}
\mkanonthms
\begin{document}
\title{Metric Spaces: Assignment 2}
\author{Franz Miltz (UUN: S1971811)}
\date{14 February 2022}
\maketitle

\begin{claim*}[1a]
	Let $X=\R$ and $d$ the standard metric. Then $\dist(\sqrt 2, \Q)=0$.
	\begin{proof}
		Note, for all $x,y\in X$, $d(x,y) \geq 0$ so $0$ is a lower bound for the set
		\begin{align*}
			S=\{d(\sqrt 2,x):x\in\Q\}.
		\end{align*}
		Assume $m=\inf S>0$. As a corollary of the \emph{Archimedean property}, we know that
		there must exist $x\in\Q$ such that $\sqrt 2 < x < \sqrt{2}+m$.
		However, now
		\begin{align*}
			d(\sqrt 2, x) = \abs{\sqrt 2 - x} = x - \sqrt{2} < m.
		\end{align*}
		Thus $m$ is not a lower bound of $S$, contradicting our assumption. The claim follows.
	\end{proof}
\end{claim*}

\begin{claim*}[1b]
	Let $X=\R$ and $d$ the discrete metric. Then $\dist(\sqrt 2, \Q)=1$.
	\begin{proof}
		Note $d(x,y)\not=1$ if and only if $x=y$ by definition of $d$. However,
		$x\not=\sqrt{2}$, for all $x\in\Q$. Thus
		\begin{align*}
			\dist(\sqrt 2,\Q) = \inf\left\lbrace d(\sqrt 2, x) : x\in\Q\right\rbrace = \inf\{1\} = 1.
		\end{align*}
	\end{proof}
\end{claim*}

\begin{claim*}[2]
	Let $X=\R$ and $d$ the standard metric. Then, for all $x\in\Q$, $d(\sqrt 2, x)\not=\dist(\sqrt 2, \Q)$.
	\begin{proof}
		Note $\dist(\sqrt 2, \Q)=0$ as proven above. Further, $d(\sqrt 2, x)=0$ if and only if
		$x=\sqrt 2$. Since $\sqrt 2\not\in\Q$ the claim follows.
	\end{proof}
\end{claim*}

\begin{claim*}[2]
	Let $X=\R$ and $d$ the discrete metric. Then there exists $x\in\Q$ such that
	$d(\sqrt 2, x) = \dist(\sqrt 2, \Q)$.
	\begin{proof}
		Note $\dist(\sqrt 2, \Q)=1$ as proven above. Let $x=1$. Then $d(\sqrt 2, x)=1=\dist(\sqrt 2, \Q)$.
	\end{proof}
\end{claim*}

\begin{lemma*}
	Let $(X,d)$ be a metric space, $x\in X$ and $A\subseteq X$ non-empty. Then there exists a sequence
	$(a_n\in A)_{n\in\N}$ such that $d(x,a_n)\to\dist(x,A)$ as $n\to\infty$.
	\begin{proof}
		Let $m=\dist(x,A)$ and
		\begin{align*}
			S=\left\lbrace d(x,a):a\in A\right\rbrace.
		\end{align*}
		By the approximation property of infima, for all $n$, there exists $d_n\in S$
		such that $m\leq d_n\leq m+1/n$. For each $d_n\in S$ there must exist $a_n\in A$
		such that $d_n=d(x,a_n)$. By the \emph{Squeeze Theorem}, $d(x,a_n)\to m$ as
		$n\to\infty$.
	\end{proof}
\end{lemma*}

\begin{claim*}[3]
	Let $X=\R$, $d$ be the standard metric, $x\in X$ and $K\subseteq X$ be non-empty and compact.
	Then there exists $y\in X$ such that $\dist(x,K)=d(x,y)$.
	\begin{proof}
		Let $m=\dist(x,K)$ and
		\begin{align*}
			S=\{d(x,y):y\in K\}.
		\end{align*}
		Using the above lemma, we fix a sequence $(y_n\in K)_{n\in\N}$ such that
		$d(x,y_n)\to m$.
		Because $K$ is compact, there exists a subsequence $(y_{n_k})_{k\in\N}$ of $(y_n)_{n\in\N}$ such that
		$y_{n_k} \to y\in K$ as $k\to\infty$. By continuity of the function given by $z\mapsto\abs{x-z}$,
		we find $\abs{x-y_{n_k}}\to\abs{x-y}=d(x,y)$. However, $\abs{x-y_{n_k}}=d(x,y_{n_k})\to m$. By uniqueness of the
		limit, we have
		\begin{align*}
			d(x,y)=\abs{x-y}=m=\dist(x,K).
		\end{align*}
	\end{proof}
\end{claim*}

\begin{claim*}[4]
	Let $(X,d)$ be a metric space and $A\subseteq X$ non-empty. Then, for all $x,y\in X$,
	\begin{align*}
		\dist(y,A)\leq d(y,x)+\dist(x,A).
	\end{align*}
	\begin{proof}
		Fix $x,y\in X$.
		We use the above lemma to fix $(z_n\in A)_{n\in\N}$ such that $d(x,z_n)\to\dist(x,A)$.
		Secondly, we observe $\dist(y,A)\leq d(y,z_n)$ for all $n$. Thus
		by \emph{Notes, Proposition 0.50}
		\begin{align*}
			\dist(y,A)\leq\lim_{n\to\infty} d(y,z_n).
		\end{align*}
		Similarly,
		\begin{align*}
			d(y,z_n) \leq d(y,x) + d(x,z_n)
		\end{align*}
		by definition of a metric space so
		\begin{align*}
			\dist(y,A)  \leq\lim_{n\to\infty} d(y,z_n)
			\leq \lim_{n\to\infty}\left(d(y,x)+d(x,z_n)\right)
			=d(y,z)+\dist(x,A).
		\end{align*}
	\end{proof}
\end{claim*}

\end{document}