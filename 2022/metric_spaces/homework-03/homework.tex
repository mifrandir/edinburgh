\documentclass{article}
\usepackage{homework-preamble}
\mkanonthms
\begin{document}
\title{Metric Spaces: Assignment 3}
\author{Franz Miltz (UUN: S1971811)}
\date{7 March 2022}
\maketitle

Let $(X,d_X)$ and $(Y,d_Y)$ be metric spaces.

\begin{claim*}[1a]
	If $f:X\to Y$ is an isometry then $f$ is injective and continuous.
	\begin{proof}
		Let $x,x'\in X$ such that $f(x)=f(x')$. Then $d_Y(f(x),f(x'))=0$ since $d_Y$ is a metric.
		Thus $d_X(x,x')=0$ as $f$ is an isometry. Now, because $d_X$ is a metric, $x=x'$. This proves
		that $f$ is injective.

		Let $x\in X$ and $\e>0$. Choose $\delta=\e$. Then, for all $x'\in X$ with $d_X(x,x')<\delta$,
		\begin{align*}
			d_Y(f(x),f(x')) = d_X(x,x') < \delta = \e.
		\end{align*}
		This proves that $f$ is continuous.
	\end{proof}
\end{claim*}

\begin{claim*}[1b]
	Let $f:X\to Y$ be surjective and an isometry. Then $\inv f:Y\to X$ is an isometry.
	\begin{proof}
		Let $y,y'\in Y$. Then, as $f$ is surjective, there exist $x,x'\in X$ such that
		$f(x)=y$ and $f(x')=y'$. Further, since $f$ is an isometry, $d_X(x,x')=d_Y(y,y')$.
		However, by definition of $\inv f$, $x=\inv f(y)$ and $x'=\inv f(y')$. The claim is
		now immediate.
	\end{proof}
\end{claim*}

\begin{claim*}[2]
	Let $f:\R\to \R$ be given by $x\mapsto 0$. Then, with respect to the standard metric
	$d$ on $\R$, $f$ is continuous but not an isometry.
	\begin{proof}
		Let $x,x'\in X$ be distinct. Then
		\begin{align*}
			d(x,x') \not= 0 = \abs{0-0} = d(0, 0) = d(f(x), f(x')).
		\end{align*}
		Thus $f$ is not an isometry.

		Let $x\in X$ and $\e>0$. Then, for all $x'\in X$, $d(f(x),f(x'))=0<\e$.
		Thus $f$ is continuous.
	\end{proof}
\end{claim*}

\begin{claim*}[3]
	Let $(X,d_X)$ be a metric space and $C>0$ such that, for all $x,x'\in X$, $d_X(x,x')\leq C$.
	Then, for each $x\in X$, we define $f_x:X\to\R$ by $t\mapsto d_X(x,t)$. Then each $f_x$ is a
	bounded and continuous function with respect to the standard metric on $\R$.
	\begin{proof}
		Let $x,x'\in X$. Choose $M=C$. Then $\abs{f_x(x')} = \abs{d_X(x,x')} = d_X(x,x') \leq C$. Thus $f_x$ is bounded.

		Let $x\in X$ and $\e>0$. Choose $\delta=\epsilon$. Then, for all $x',x''\in X$
		with $d_X(x',x'')<\delta$,
		\begin{align*}
			\abs{f_x(x')-f_x(x'')} = \abs{d_X(x,x')-d_X(x,x'')} \leq d_X(x',x'') < \delta = \e
		\end{align*}
		where we used the reverse triangle inequality. Thus $f_x$ is continuous with respect to the
		standard metric.
	\end{proof}
\end{claim*}

\begin{claim*}[4]
	Let $(X,d_X)$ be a metric space and $C>0$ such that, for all $x,x'\in X$, $d_X(x,x')\leq C$.
	Let $Y$ be the set of all bounded continuous functions $f:X\to\R$ and define the metric
	$d_Y:Y\times Y\to\R$ by
	\begin{align*}
		d_Y(f,g) = \sup\{\abs{f(t)-g(t)}:t\in X\}.
	\end{align*}
	Then, for each $x\in X$,  let $f_x:X\to\R$ be given by $t\mapsto d_X(x,t)$. Then the function
	$g:X\to Y$ given by $x\mapsto f_x$ is an isometry.
	\begin{proof}
		Let $x,x'\in X$. By definition of $d_Y$ we have
		\begin{align}
			\label{sup}
			d_Y(f_x,f_{x'}) = \sup\{\abs{f_x(t)-f_{x'}(t)}:t\in X\}
			= \sup\{\abs{d(x,t)-d(x',t)}:t\in X\}.
		\end{align}
		We note that, for all $t\in X$,
		\begin{align*}
			\abs{d(x,t)-d(x',t)}\leq d(x,t)
		\end{align*}
		by the reverse triangle inequality. Moreover, if $t=x'$,
		\begin{align*}
			\abs{d(x,t)-d(x',t)}=d(x,t).
		\end{align*}
		Therefore $d(x,x')$ is an upper bound of the set in (\ref{sup}) which is
		attained. We conclude that
		\begin{align*}
			d_Y(f_x,f_{x'}) = d(x,x'),
		\end{align*}
		thereby proving the claim.
	\end{proof}
\end{claim*}

\end{document}