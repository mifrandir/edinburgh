\documentclass{article}
\usepackage{homework-preamble}

\title{PPS: Homework 9 (Workshop 33)}
\author{Franz Miltz (UUN: S1971811)}
\date{6 April 2020}

\begin{document}
\maketitle

\section*{Problem 1}
\begin{claim}
	There are $120$ integers from $1$ to $200$ that are divisible by $3$, $4$ or $5$.
\end{claim}
\begin{proof}
	By using the \emph{Inclusion-Exclusion Principle}.
	Let $A$, $B$ and $C$ be finite sets such that
	\begin{align*}
		A & = \{x\in [1,200] : 3 | x\}, \\
		B & = \{x \in [1,200] : 4|x\},  \\
		C & = \{x\in [1,200] : 5|x\}
	\end{align*}
	Observe that there are $66$ integers in the given range divisible by $3$, $50$ divisible by $4$ and $40$ divisible by $5$.
	Thus we get
	\begin{align*}
		c_1 = |A| + |B| + |C| = 66 + 50 + 40 = 156.
	\end{align*}
	To find $c_2$ we need to consider integers in $[1,200]$ divisible by $12$ ($A\cap B$), $15$ ($A\cap C$) and $20$ ($B\cap C$). This leads to
	\begin{align*}
		c_2 = |A\cap B| + |A\cap C| + |B \cap C| = 16 + 13 + 10 = 39.
	\end{align*}
	Finally, $c_3$ is the number of integers in $[1,200]$ divisible by $3$, $4$ and $5$, i.e. $60$. We get
	\begin{align*}
		c_3 = |A\cap B \cap C| = 3.
	\end{align*}
	So finally we prove our claim by applying the \emph{Inclusion-Exclusion Principle}:
	\begin{align*}
		|A\cup B \cup C| = c_1- c_2 + c_3 = 156 - 39 + 3 = 120.
	\end{align*}
\end{proof}
\noindent\emph{Note: To find the number of integers in $[1,200]$ divisible by $x$, you can use integer division $200 / x$}.

\section*{Problem 2}
\begin{claim}
	Let $S$ be a set of size $n$ and $A\subseteq S$ of size $k$. Then there are $2^{n-k}$ subsets of $S$ that contain (all elements of) $A$.
\end{claim}
\begin{proof}
	Let $T=S\setminus A$ be a finite set. Then for any $X\subseteq S$ such that $A \subseteq X$, i.e. $X$ contains $A$, there exists a unique set $B$ such that
	\begin{align*}
		X =  A \cup B \text{ and } B\subseteq T.
	\end{align*}
	Therefore the number of subsets of $S$ that contain $A$ is equal to the number of subsets of $S$ that are disjoint to $A$, i.e. the number of subsets of $T$.
	Since $|S|=n$ and $|A|=k$, $|T| = n - k$. Using \emph{Liebeck, Proposition 17.4} we know that $T$ has $2^{n-k}$ subsets.
\end{proof}

\section*{Problem 3}
\begin{claim}
	Let $S=\{1,...,50\}$ and let $W\subseteq S$ be a set with $6$ randomly selected elements.
	Then the chance $p$ of a set $P\subseteq S$ with $10$ randomly selected elements containing $W$ is
	\begin{align*}
		p = \frac{\binom{42}{4}}{\binom{50}{10}} = \frac{1}{75670} \approx 0.0013\%.
	\end{align*}
\end{claim}
\begin{proof}
	This is equivalent to a \emph{Hypergeometric Distribution} with $N=50$ as there are $50$ possible numbers, $K=6$ as there are $6$ correct numbers, $n=10$ as $10$ numbers are picked by the player and $k=6$ as we require the player to pick all the correct numbers in one draw.
	Using the formula for the \emph{Hypergeometric Distribution} we get
	\begin{align*}
		p = \frac{\binom{6}{6}\binom{50-6}{10-6}}{\binom{50}{10}}= \frac{\binom{44}{4}}{\binom{50}{10}} = \frac{1}{75670} \approx 0.0013 \%.
	\end{align*}
\end{proof}
\noindent\emph{Note: I know we have not covered the Hypergeometric Distribution in this course but I don't think it makes any sense to reinvent the wheel in this case.
	This distribution is very well known, I've been taught about it in highschool and it should therefore be allowed to be used in homeworks.}
\end{document}