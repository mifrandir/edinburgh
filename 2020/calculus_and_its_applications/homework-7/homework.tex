\documentclass{article}
\usepackage{homework-preamble}

\title{CAP: Homework 7 (Workshop 71)}
\author{Franz Miltz (UUN: S1971811)}
\begin{document}
\maketitle
\section*{Question 1}
\subsection*{a)}
At first, we solve the indefinite integral
\begin{align*}
	I(x) = \int x^4e^{-x^5}dx.
\end{align*}
With $u=e^{-x^5}$ and $du=-5x^4e^{-x^5}dx$, we get
\begin{align*}
	I(x) = -\frac{1}{5}\int du = -\frac{1}{5}e^{-x^5} + C.
\end{align*}
Now we can calculate the improper integral
\begin{align*}
	\int_0^\infty x^4e^{-x^5}dx & = \lim_{x\to\infty}\left(I(x)-I(0)\right)
	= \frac{1}{5}-\frac{1}{5}\lim_{x\to\infty}e^{-x^5}=\frac{1}{5}.
\end{align*}
Therefore the integral converges to $\frac{1}{5}$.
\subsection*{b)}
We can solve the indefinite integral
\begin{align*}
	I(t) = \int \cos(\pi t)\:dt
\end{align*}
by substituting $u=\pi t$ and $du=\pi dt$ to get
\begin{align*}
	I(t) = \frac{1}{\pi} \sin(\pi t) + C.
\end{align*}
Since $\sin x$ does not converge as $x\to-\infty$, the improper integral
\begin{align*}
	\int_{-\infty}^0\cos(\pi t)\:dt
\end{align*}
is divergent and may therefore not be evaluated.
\subsection*{c)}
We can solve the indefinite integral
\begin{align*}
	I(x)=\int_0^{\sqrt{3}}\frac{x}{\sqrt{9-x^4}}\:dx
\end{align*}
using the trigonometric substitution
\begin{align*}
	x^2 = 3\sin\theta \Rightarrow 2xdx = 3\cos \theta d\theta
\end{align*}
to get
\begin{align*}
	I(x)=\frac{3}{2}\int \frac{\cos\theta}{3\cos\theta}\:d\theta = \frac{1}{2}\int d\theta = \frac{1}{2}\theta + C.
\end{align*}
To revert the substitution, we use $\theta = \arcsin\left(\frac{x^2}{3}\right)$:
\begin{align*}
	I(x)=\frac{1}{2}\arcsin\left(\frac{x^2}{3}\right)+C.
\end{align*}
This lets us evaluate the improper integral
\begin{align*}
	\int_0^{\sqrt{3}}\frac{x}{\sqrt{9-x^4}}dx = I(\sqrt{3})-I(0)=\frac{1}{2}\left(\frac{\pi}{2}-0\right)=\frac{\pi}{4}.
\end{align*}
Since the integral is well-defined, it is also convergent.
\section*{Question 2}
\subsection*{a)}
\begin{claim}
	The improper integral $\int_0^{\infty}\frac{\arctan x}{2+e^x}\:dx$ is convergent.
\end{claim}
\begin{proof}
	We will prove this by showing that
	\begin{align*}
		0\leq \frac{\arctan x}{2+e^x} < \frac{\pi}{2}e^{-x} \text{ for all } x \geq 0
	\end{align*}
	and then using the \emph{Comparison Theorem}.\\
	Firstly, observe that since
	\begin{align*}
		 & \forall x \geq 0,\: \arctan x \geq 0, \\
		 & \forall x \geq 0,\: e^x \geq 0,
	\end{align*}
	and therefore $\forall x \geq 0,\: \frac{\arctan x}{2+e^x}\geq 0$.\\
	Secondly, note that $\forall x \in \R,\: \arctan x < \frac{\pi}{2}$. Thus we can say that
	\begin{align*}
		\frac{\arctan x}{2+e^x}<\frac{\pi}{2}\left(\frac{1}{2+e^x}\right).
	\end{align*}
	Further,
	\begin{align*}
		\frac{1}{2+e^x}<\frac{1}{e^x}
	\end{align*}
	and thus
	\begin{align*}
		\frac{\arctan x}{2+e^x}<\frac{\pi}{2}e^{-x}.
	\end{align*}
	Since
	\begin{align*}
		\int_0^\infty \frac{\pi}{2}e^{-x}dx = \frac{\pi}{2}\int_0^\infty e^{-x}dx = -\frac{\pi}{2}\left[e^{-x}\right]^\infty_0=\frac{\pi}{2},
	\end{align*}
	the improper integral $\int_0^\infty \frac{\pi}{2}e^{-x}dx$ is convergent. Using the \emph{Comparison Theorem} with
	\begin{align*}
		f(x) & =\frac{\pi}{2}e^{-x},     \\
		g(x) & =\frac{\arctan x}{2+e^x},
	\end{align*}
	we can conclude that
	\begin{align*}
		\int_0^\infty \frac{\arctan x}{2+e^x}\:dx
	\end{align*}
	has to be convergent, too.
\end{proof}
\subsection*{b)}
\begin{claim}
	The improper integral $\int_2^\infty \frac{dx}{(1+x^5)^{1/6}}$ is divergent.
\end{claim}
\begin{proof}
	We will prove this by showing that
	\begin{align*}
		0\leq \frac{1}{x}<\frac{1}{(1+x^5)^{1/6}} \text{ for all } x \geq 2
	\end{align*}
	and then using the \emph{Comparison Theorem}.\\
	Firstly, observe that for $x\geq 2$
	\begin{align*}
		0 \leq \frac{1}{x}.
	\end{align*}
	Secondly, note that
	\begin{align*}
		\frac{1}{x}         & <\frac{1}{(1+x^5)^{1/6}} \\
		\Leftrightarrow x   & >(1+x^5)^{1/6}           \\
		\Leftrightarrow x^6 & > 1+x^5
	\end{align*}
	which is true for all $x\geq 2$, because $2^6 > 1 + 2^5$, both $x^6$ and $1+x^5$ are continuous on $\R$ and the points of intersection are at $x\approx -0.88$ and $x\approx 1.29$, both of which are less than $2$.\\
	Since we know the improper integral $\int_2^\infty \frac{1}{x}\:dx$ is divergent, we can use the \emph{Comparison Theorem} with
	\begin{align*}
		g(x) & =\frac{1}{x},             \\
		f(x) & =\frac{1}{(1+x^5)^{1/6}},
	\end{align*}
	to show that $\int_2^\infty \frac{dx}{(1+x^5)^{1/6}}$ is divergent, too.
\end{proof}
\end{document}