\documentclass{article}
\usepackage{homework-preamble}

\title{ILA: H4 (Workshop 08)}
\author{Franz Miltz (UUN: s1971811)}
\date{October 22, 2019}
\begin{document}
\maketitle
\section*{Q43 a}
Let's suppose $S'=\{\vec u+\vec v, \vec u+\vec w, \vec v+\vec w\}$ is linearly dependent
while $S=\{\vec u, \vec v, \vec w\}$ isn't. Then there exists a solution to
\begin{align}
	a_1(\vec u+\vec v) + a_2(\vec u + \vec w) + a_3(\vec v + \vec w) = \vec 0
\end{align}
such that one $a_i\not=0$. Let's expand the brackets and rewrite the equation:
\begin{align}
	(a_1+a_2)\vec u + (a_1+a_3) \vec v + (a_2 + a_3) \vec w = \vec 0
\end{align}
Since we know the factors in front of the vectors have to be zero (otherwise $S$ would not be linearly independent), we get the following system of linear equations:
\begin{align}
	\begin{bmatrix}
		1 & 1 & 0 \\
		1 & 0 & 1 \\
		0 & 1 & 1
	\end{bmatrix}
	\begin{bmatrix}
		a_1 \\a_2\\a_3
	\end{bmatrix}
	= \begin{bmatrix}
		  0 \\ 0 \\ 0
	  \end{bmatrix}
\end{align}
Since we can reduce the coefficient matrix to $I_3$ using elementary row operations which do not influence the vector on the right-hand side, we know the only solution to this system is the trivial one.\\
This contraticts our assumption, that $S'$ is linearly dependent. Therefore, $S'$ has to be linearly independent as well.
\section*{Q46}
Let $S$ be a linearly independent set of vectors:
\begin{align}
	S=\{\vec v_0, \vec v_1, ..., \vec v_n\}.
\end{align}
Therefore
\begin{align}
	c_0\vec v_0 + c_1\vec v_1 + ... + c_n\vec v_n = \vec 0
\end{align}
if and only if
\begin{align}
	c_0 = c_1 = ... = c_n = 0.
\end{align}
Let's consider the subset $S'$:
\begin{align}
	S'=\{\vec v_1, \vec v_2, ..., \vec v_n\}.
\end{align}
Suppose $S'$ is linearly dependent. Therefore there exists a solution to
\begin{align}
	a_1\vec v_1 + a_2\vec v_2 + ... + a_n\vec v_n = \vec 0
\end{align}
where at least one $a_i\not=0$. Now we add the vector $c_0\vec v_0$ to this equation and let $c_0 = 0$.
Effectively, this is the same as adding $\vec 0$.
Thus we get
\begin{align}
	0\vec v_0 + a_1\vec v_1 + a_2\vec v_2 + ... + a_n\vec v_n = \vec 0.
\end{align}
This is obviously true and therefore provides a non-trivial solution of the linear combination in $(5)$, making $S$ linearly dependent and contradicting our definition.
Therefore $S'$ also has to be linearly independent. $\square$
\end{document}