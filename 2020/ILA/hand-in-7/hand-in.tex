\documentclass{article}
\usepackage{homework-preamble}

\title{ILA: H7 (Workshop 08)}
\author{Franz Miltz (UUN: s1971811)}
\date{November 12, 2019}
\begin{document}
\maketitle
\section*{Q8}
We know that for $T$ to be linear we need the following to be true for every $k\in\mathbb{R}$ and $\vec u \in\mathbb{R}^n$.
\begin{align*}
	T(k\vec u) = k T(\vec u)
\end{align*}
Let $\vec u = \vec e_1$ and $k = -1$. Then
\begin{align*}
	T(k\vec u) = T\left(-\begin{bmatrix}
		1 \\ 0
	\end{bmatrix}\right) =
	\begin{bmatrix}
		|-1| \\ 0
	\end{bmatrix} =
	\begin{bmatrix}
		1 \\0
	\end{bmatrix}.
\end{align*}
Now let's look at $kT(\vec u)$:
\begin{align*}
	kT(\vec u) = -T\left(\begin{bmatrix}
		1 \\ 0
	\end{bmatrix}\right)
	= \begin{bmatrix}
		-1 \\ 0
	\end{bmatrix}.
\end{align*}
Therefore
\begin{align*}
	T(k\vec u) \not= kT(\vec u)
\end{align*}
and thus $T$ is not linear.
\section*{Q53}
A transformation $T:\mathbb{R}^n\to\mathbb{R}^m$ is linear iff
\begin{enumerate}
	\item $T(k\vec u) = kT(\vec u)$ for all $k\in R$ and $\vec u\in\mathbb{R}^n$
	\item $T(\vec u + \vec v) = T(\vec u) + T(\vec v)$ for all $\vec u, \vec v \in\mathbb{R}^n$
\end{enumerate}
The given equation,
\begin{align*}
	T(c_1\vec v_1 + c_2\vec v_2) = c_1T(\vec v_1) + c_2T(\vec v_2),
\end{align*}
seems to follow directly from that definition:
\begin{align*}
	T(c_1\vec v_1 + c_2\vec v_2) = T(c_1\vec v_1) + T(c_2\vec v_2)
\end{align*}
by the second part of the definition iff $T$ is linear and
\begin{align*}
	T(c_1\vec v_1) + T(c_2\vec v_2) = c_1T(\vec v_1) + c_2T(\vec v_2)
\end{align*}
by the first part of the definition iff $T$ is linear. $\square$\\\\
Note that we used both conditions in the definition. Therefore we know that all the conditions posed by the definition are definetly held by any $T$ that satisfies our single new equation. Additionally we did not use anything else but the definition and therefore can be sure that any $T$ which satisfies the definition will satisfy our equation.
\end{document}