\documentclass{article}
\usepackage{homework-preamble}

\title{ILA: H6 (Workshop 08)}
\author{Franz Miltz (UUN: s1971811)}
\date{November 05, 2019}

\begin{document}
\maketitle
\section*{Q52}
This question can be answered by finding a solution to the following system:
\begin{align*}
	\begin{bmatrix}
		3 & 5 \\ 1 & 1\\ 4 &6
	\end{bmatrix}
	[\vec w]_{\mathcal{B}} =
	\begin{bmatrix}
		1 \\ 3 \\4
	\end{bmatrix}
\end{align*}
To do this, let's find an RREF to the augmented matrix
\begin{align*}
	\begin{amatrix}{1}
		\mathcal{B} &\vec w
	\end{amatrix} =
	\begin{amatrix}{2}
		3 &5 &1 \\
		1 &1 & 3\\
		4 &6 & 4
	\end{amatrix}.
\end{align*}
We get
\begin{align*}
	rref\left(
	\begin{amatrix}{1}
		\mathcal{B} &\vec w
	\end{amatrix}\right)=
	\begin{amatrix}{2}
		1 &0 &7\\
		0 &1 &-4\\
		0 &0 &0
	\end{amatrix}.
\end{align*}
Therefore
\begin{align*}
	[\vec w]_\mathcal{B} = \begin{bmatrix}
		7 \\ -4
	\end{bmatrix}.
\end{align*}
This means that a) $\vec w$ is in $span(\mathcal{B})$ because it is a linear combination of of the vectors in $\mathcal{B}$ and b) that the coordinate vector that presents $\vec w$ in terms of the vectors in $\mathcal{B}$, $[\vec w]_\mathcal{B}$, is the solution of the linear system.
\section*{Q62}
We have two properties of $A$:
\begin{enumerate}
	\item $P$: $rank(A)=1$
	\item $Q$: $\exists (\vec u, \vec v) \in \mathbb{R}^m \times \mathbb{R}^n.\: \vec u\vec v^T = \vec u \otimes \vec v = A$
\end{enumerate}
To show that $P \Leftrightarrow Q$, we need to show that $P$ implies $Q$ and vice versa.\\
Let's look at $Q\Rightarrow P$: If $A=\vec u \otimes \vec v$ for some vectors $\vec u$ and $\vec v$, we know that
\begin{align*}
	A = \vec u \otimes \vec v = \begin{bmatrix}
		u_1v_1 & u_1v_2 & \cdots & u_1v_n \\
		u_2v_1 & u_2v_2 & \cdots & u_2v_n \\
		\vdots & \vdots & \ddots & \vdots \\
		u_mv_1 & u_mv_2 & \cdots & u_mv_n
	\end{bmatrix}.
\end{align*}
Therefore we can write $A$ in terms of its column vectors:
\begin{align*}
	A = \begin{bmatrix}
		v_1\vec u & v_2 \vec u & \cdots & v_n\vec u
	\end{bmatrix}.
\end{align*}
Since all of the column vectors are a multiple of $\vec u$, we know that $col(A)=span(\vec u)$.
Since $rank(A)=dim(col(A))$ in general, $rank(A)=1$ in this case.
Therefore $Q\Rightarrow P$.\\
We can apply the same procedure to $P\Rightarrow Q$: Because of $P$ we know that $rank(A)=dim(col(A))=1$ and may thus choose to write $A$ in terms of it's collinear column vectors:
\begin{align*}
	A = \begin{bmatrix}
		v_1\vec u & v_2 \vec u & \cdots & v_n\vec u
	\end{bmatrix}.
\end{align*}
As you can see, we have conveniently named the common factor in the columns $\vec u$ and the scalars $v_i$.
This obviously means we can write $A$ as
\begin{align*}
	A = \begin{bmatrix}
		u_1v_1 & u_1v_2 & \cdots & u_1v_n \\
		u_2v_1 & u_2v_2 & \cdots & u_2v_n \\
		\vdots & \vdots & \ddots & \vdots \\
		u_mv_1 & u_mv_2 & \cdots & u_mv_n
	\end{bmatrix} = \vec u \otimes \vec v.
\end{align*}
Since $P\Rightarrow Q$ and $Q\Rightarrow P$, $P\Leftrightarrow Q$. $\square$
\end{document}