\documentclass{article}
\usepackage[a4paper]{geometry}
\geometry{tmargin=3cm, bmargin=3cm, lmargin=2cm, rmargin=2cm}
\usepackage[british]{babel}
\usepackage{amsmath}
\usepackage{amssymb}
\usepackage{amsthm}
\usepackage{nicefrac}
\usepackage{siunitx}
\usepackage{mathtools}
\usepackage{hyperref}
\usepackage{fontspec}
\setmainfont{arial}
\usepackage{minted}
\hypersetup{
	colorlinks,
	citecolor=black,
	filecolor=black,
	linkcolor=black,
	urlcolor=black
}
\DeclareMathOperator{\csch}{csch}
\DeclareMathOperator{\arccot}{\text{cot}^{-1}}
\DeclareMathOperator{\arccsc}{\text{csc}^{-1}}
\DeclareMathOperator{\arccosh}{\text{cosh}^{-1}}
\DeclareMathOperator{\arcsinh}{\text{sinh}^{-1}}
\DeclareMathOperator{\arctanh}{\text{tanh}^{-1}}
\DeclareMathOperator{\arcsech}{\text{sech}^{-1}}
\DeclareMathOperator{\arccsch}{\text{csch}^{-1}}
\DeclareMathOperator{\arccoth}{\text{coth}^{-1}} 
\DeclareMathOperator{\dom}{dom}
\DeclareMathOperator{\st}{s.t.}
\DeclareMathOperator{\sech}{sech}
\newtheoremstyle{sltheorem} {}                % Space above
{}                % Space below
{\upshape}        % Theorem body font % (default is "\upshape")
{}                % Indent amount
{\bfseries}       % Theorem head font % (default is \mdseries)
{.}               % Punctuation after theorem head % default: no punctuation
{ }               % Space after theorem head
{}                % Theorem head spec
\theoremstyle{sltheorem}
\newtheorem{definition}{Definition}[section]
\newtheorem{theorem}{Theorem}[section]
\newtheorem{lemma}[theorem]{Lemma}
\newtheorem{corollary}[theorem]{Corollary}
\newtheorem{proposition}[theorem]{Proposition}
\newcommand{\R}{\mathbb{R}}
\newcommand{\N}{\mathbb{N}}
\newcommand{\Z}{\mathbb{Z}}
\renewcommand{\C}{\mathbb{C}} % \C defined in `hyperref'
\DeclareMathOperator{\lub}{LUB}
\DeclareMathOperator{\glb}{GLB}
\DeclareMathOperator{\hcf}{hcf}
\DeclareMathOperator{\lcm}{lcm}
\DeclareMathOperator{\sgn}{sgn}
\DeclareMathOperator{\cl}{cl}
\newcommand*\lneg[1]{\overline{#1}}
\newcommand*\B[1]{\textbf{#1}}
\newcommand*\T[1]{\texttt{#1}}
\usepackage{expl3}[2012-07-08]
\ExplSyntaxOn
\cs_new_eq:NN \fpeval \fp_eval:n
\ExplSyntaxOff
\begin{document}
\title{Introduction to Algorithms and Data Structures (YEAR2)}
\author{Franz Miltz}
\maketitle
\tableofcontents  
\section{Asymptotics Analysis}
\B{Asymptotic theory} makes precise quantitive statements about efficiency of algorithms themselves.
\begin{definition}
	Let $f,g:\N\to\R_{\geq 0}$ be functions. Then
	\text{$f\in o(g)$} if, and only if, 
	\begin{align*}
		\forall c>0,\:\exists N\st \forall n \geq N, f(n)<cg(n)
	\end{align*}
	where $c\in\R$ and $N,n\in\N$.
\end{definition}
\begin{theorem}
	Let $f:\N\to\R_{\geq 0}$ and let $o(f)$ refer to some
	function within the set $o(f)$. Then
	\begin{itemize}
		\item $co(f)=o(f)$ where $c\in\R$,
		\item $o(f) + o(f) = o(f)$.
	\end{itemize}
\end{theorem}
\begin{theorem}
	Let $f,r:\N\to\R_{\geq 0}$ and let $a,b\in\R$. Then
	\begin{align*}
		f=o(g) \Leftrightarrow af=o(bg).	
	\end{align*}
\end{theorem}
\begin{definition}
	Let $f,g:\N\to\R_{\geq 0}$. Then $f=\omega(g)$ if, and only if, $g=o(f)$.
\end{definition}
\begin{definition}
	Let $f,g:\N\to\R_{\geq 0}$. Then $f\in O(g)$ if, and only if,
	\begin{align*}
		\exists C > 0\st \exists N \st \forall n \geq N,\: f(n) \geq Cg(n)
	\end{align*}
	where $C\in\R$ and $N,n\in\N$.\\
	We call $g$ an \B{asymptotic upper bound} for $f$.
\end{definition}
\begin{definition}
	Let $f,g:\N\to\R_{\geq 0}$. Then $f\in\Omega(g)$ if, and only if, $g\in O(f)$.\\
	We call $g$ an \B{asymptotic lower bound} for $f$.
\end{definition}
\begin{definition}
	Let $f,g:\N\to\R_{\geq 0}$. Then $f\in\Theta(g)$ if, and only if, $f\in O(f) \cap \Omega(f)$.\\
	We call $g$ an \B{asymptotic tight bound} for $f$.
\end{definition}
\begin{theorem}
	Let $f,g:\N\to\R_{\geq 0}$. Then $f\in\Theta(g)$ if, and only if, $g\in\Theta(f)$.
\end{theorem}
\end{document}
