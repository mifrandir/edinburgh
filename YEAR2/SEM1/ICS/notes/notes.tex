\documentclass{article}
\usepackage[a4paper]{geometry}
\geometry{tmargin=3cm, bmargin=3cm, lmargin=2cm, rmargin=2cm}
\usepackage[british]{babel}
\usepackage{amsmath}
\usepackage{amssymb}
\usepackage{amsthm}
\usepackage{nicefrac}
\usepackage{siunitx}
\usepackage{mathtools}
\usepackage{hyperref}
\usepackage{newpxtext}
\usepackage{minted}
\hypersetup{
	colorlinks,
	citecolor=black,
	filecolor=black,
	linkcolor=black,
	urlcolor=black
}
\DeclareMathOperator{\csch}{csch}
\DeclareMathOperator{\arccot}{\text{cot}^{-1}}
\DeclareMathOperator{\arccsc}{\text{csc}^{-1}}
\DeclareMathOperator{\arccosh}{\text{cosh}^{-1}}
\DeclareMathOperator{\arcsinh}{\text{sinh}^{-1}}
\DeclareMathOperator{\arctanh}{\text{tanh}^{-1}}
\DeclareMathOperator{\arcsech}{\text{sech}^{-1}}
\DeclareMathOperator{\arccsch}{\text{csch}^{-1}}
\DeclareMathOperator{\arccoth}{\text{coth}^{-1}} 
\DeclareMathOperator{\dom}{dom}
\DeclareMathOperator{\st}{s.t.}
\DeclareMathOperator{\sech}{sech}
\newtheoremstyle{sltheorem} {}                % Space above
{}                % Space below
{\upshape}        % Theorem body font % (default is "\upshape")
{}                % Indent amount
{\bfseries}       % Theorem head font % (default is \mdseries)
{.}               % Punctuation after theorem head % default: no punctuation
{ }               % Space after theorem head
{}                % Theorem head spec
\theoremstyle{sltheorem}
\newtheorem{definition}{Definition}[section]
\newtheorem{theorem}{Theorem}[section]
\newtheorem{lemma}[theorem]{Lemma}
\newtheorem{corollary}[theorem]{Corollary}
\newtheorem{proposition}[theorem]{Proposition}
\newcommand{\R}{\mathbb{R}}
\newcommand{\N}{\mathbb{N}}
\newcommand{\Z}{\mathbb{Z}}
\renewcommand{\C}{\mathbb{C}} % \C defined in `hyperref'
\DeclareMathOperator{\lub}{LUB}
\DeclareMathOperator{\glb}{GLB}
\DeclareMathOperator{\hcf}{hcf}
\DeclareMathOperator{\lcm}{lcm}
\DeclareMathOperator{\sgn}{sgn}
\DeclareMathOperator{\cl}{cl}
\newcommand*\lneg[1]{\overline{#1}}
\newcommand*\B[1]{\textbf{#1}}
\newcommand*\T[1]{\texttt{#1}}
\usepackage{expl3}[2012-07-08]
\ExplSyntaxOn
\cs_new_eq:NN \fpeval \fp_eval:n
\ExplSyntaxOff
\begin{document}
\title{Introduction to Computer Systems (SEM3)}
\author{Franz Miltz}
\maketitle
\tableofcontents
\section{Data Representation}
Data representation affects
\begin{itemize}
	\item complexity of circuits
	\item cost
	\item speed 
	\item reliability
\end{itemize}
\subsection{The bit}
Advantages:
\begin{itemize}
	\item easy to use
	\item very reliable
	\item simple \& reusable circuits
\end{itemize}
Downside: Low information density
\subsection{Fixed bit-length arithmetic}
Issue: Hardware cannot operate on infinitely big numbers.
\begin{definition}
	An \textbf{Overflow} happens when a result does not fit the representation.
	Arithmetic therefore is module $2^N$, where $N$ is the number of bits.
\end{definition}
\subsection{Negative numbers}
\subsubsection{Sign-magnitude representation}
\begin{itemize}
	\item use MSB as the sign
	\item $1$ means negative, $0$ means positive
\end{itemize}
Problems: \begin{itemize}
	\item Complicates addition and substraction
	\item has positive and negative zero
\end{itemize}
\subsubsection{Two's complement representation}
Idea:
\begin{itemize}
	\item Want: $X + (-X) = 0$
	\item With fixed length arithmetic: $X + (-X) \equiv 0 \mod 2^N$
	\item Approach:
	\begin{itemize}
		\item Represent $-X$ as $2^N-X$
		\item Then $X + (-X) = X + (2^N - X) = 2^N$ and $2^N \equiv 0 \mod 2^N$
	\end{itemize}
\end{itemize}
Properties:
\begin{itemize}
	\item The MSB is the sign
	\item Change $X$ to $-X$ is easy: Flip the bits and add $1$
	\item $A-B$ is the same as $A+(-B)$; same circuit
	\item Operations do not depend on the operands' signs
	\item Range is asymmetric: $[-2^{n-1}, 2^{n-1}-1]$
	\item Two kinds of overflows 
\end{itemize}
\subsubsection{Sign extension}
Process of converting from a smaller to a larger representation.
MSB is extended to all the new bits.
\subsubsection{Overflow detection}
\begin{enumerate}
	\item If numbers with \B{opposite signs} are \B{added}, no overflow is possible.
	\item If numbers with \B{identical signs} are \B{substracted}, no overflow is possible.
\end{enumerate}
\begin{tabular}{| c | c | c | c |}
	\hline
	Operation & Operand A & Operand B & Result of Overflow\\
	\hline
	$A + B$ & $\geq 0$ & $\geq 0$ & $<0$\\
	\hline
	$A + B$ & $< 0$    & $< 0$    & $\geq 0$\\
	\hline
	$A - B$ & $\geq 0$ & $< 0$    & $<0$\\
	\hline
	$A - B$ & $< 0$    & $\geq 0$ & $\geq 0$\\
	\hline
\end{tabular}
\subsection{Shifting}
Moving the bits of a number left or right.
\begin{itemize}
	\item left shifts: \begin{itemize}
		\item zeros fill in the empty bits
		\item equivalent to multiplication by $2$
	\end{itemize}
	\item right shifts \begin{itemize}
		\item logical shift: fill with zero, for non-numerical data
		\item arithmetic shift: fill with MSB, for two's complement numbers
		\item equivalent to integer division by $2$
	\end{itemize}
\end{itemize}
\subsection{Floating point numbers}
\subsubsection{Normalised scientific notation}
All floating point binary numbers are represented in the form
\begin{align*}
	(-1)^S 1.a_1a_2\cdots a_n \cdot 2^E.
\end{align*}
Advantages:
\begin{itemize}
	\item Simplifies machine representation.
	\item Simplifies comparisons.
	\item More compact for very small/large numbers.
\end{itemize}
\subsubsection{IEEE 754 Floating Point standard}
32 bit:
\begin{tabular}{| l | l | l | l |}
	\hline
	word length & sign ($S$) & exponent ($E$) & mantissa ($M$)\\
	\hline
	\hline
	32 bit & 1 bit & 8 bits & 23 bits\\
	\hline
	64 bit & 1 bit & 11 bits & 52 bits\\
	\hline
\end{tabular}\\\\
This results in the number
\begin{align*}
	(-1)^S \times (1.M) \times 2^{E-127}.
\end{align*}
\subsection{Characters}
\subsubsection{ASCII}
\begin{itemize}
	\item most common
	\item each character is one byte (just 256 characters)
\end{itemize}
\subsubsection{Unicode}
\begin{itemize}
	\item used by Java
	\item two bytes per character
	\item enough for all languages on earth
\end{itemize}
\subsubsection{Length}
\begin{itemize}
	\item variable length
	\item computer needs to know where it ends
	\item options:
	\begin{itemize}
		\item last character is \T{NULL} (C)
		\item length is stored with the string itself (Java)
		\item stored in separate variable 
	\end{itemize}
\end{itemize}
\section{Design Principles}
\begin{enumerate}
	\item Simplicity favours regularity.
	\item Smaller is faster.
	\item Good design demands good compromises.
	\item Make the common case fast.
\end{enumerate}
\section{MIPS}
\begin{definition}
	The \B{instruction set architecture} (ISA) is the interface between
	the software and the hardware.
	\begin{itemize}
		\item Collection of all machine instructions recognised by
					a particular processor.
		\item privilege levels, memory managment, etc.
		\item abstracts away implementation details away from the user
		\item enables multiple implementations of the same ISA
	\end{itemize}
\end{definition}
\begin{definition}
	\B{Assembly language} is a symbolic representation of machine instructions
	\begin{itemize}
		\item makes it easy for humans to read and write machine code
		\item 1-to-1 mapping between a machine instructions and a corresponding
			    assembly instruction
	\end{itemize}
\end{definition}
\subsection{Instructions}
\begin{itemize}
	\item Separate instructions to \B{access} data in memory \& operate on it.
\end{itemize}

\begin{tabular}{| l | l | l | l |}
	\hline
	Category & Instruction & Example & Meaning	\\
	\hline
	\hline
	Arithmetic & add & \T{add \$s1,\$s2,\$s3} & \T{\$s1 = \$s2 + \$s3} \\
	\hline
	Arithmetic & sub & \T{sub \$s1,\$s2,\$s3} & \T{\$s1 = \$s2 - \$s3} \\
	\hline
\end{tabular}
\subsection{Fields}
\begin{itemize}
	\item \B{op}: Basic operation of the instruction
	\item \B{rs}: The first register source operand
	\item \B{rt}: The second register source operand
	\item \B{rd}: The register destination operand
	\item \B{shamt}: Shift amount
	\item \B{funct}: Function
\end{itemize}
\newcommand{\cellw}{1.3cm}	
\begin{tabular}{l l}
	\B{R-type:} &
	\begin{tabular}{| p{\cellw} | p{\cellw} | p{\cellw} | p{\cellw} | p{\cellw} | p{\cellw} | p{\cellw} |}
		\hline
		op & rs & rt & rd & shamt & funct \\
		\hline 
		6 bits & 5 bits & 5 bits & 5 bits & 5 bits & 6 bits\\
		\hline	
	\end{tabular}\\\\
	\B{I-type:} &
	\begin{tabular}{| p{\cellw} | p{\cellw} | p{\cellw} | p{\cellw} p{\cellw} p{\cellw} |}
		\hline
		op & rs & rt & address & & \\
		\hline
		6 bits & 5 bits & 5 bits & 16 bits & & \\
		\hline
	\end{tabular}\\\\
	\B{J-type:} &
	\begin{tabular}{| p{\cellw} | p{\cellw} p{\cellw} p{\cellw} p{\cellw} p{\cellw} |}
		\hline
		op & address & & & & \\
		\hline
		6 bits & 26 bits & & & & \\
		\hline
	\end{tabular}
\end{tabular}
\subsection{Procedures}
Should follow the following steps:
\begin{enumerate}
	\item Place parameters in a place where the procedure can access them.
	\item Transfer control to the procedure.
	\item Acquire the storage resources needed for the procedure.
	\item Perform the desired task.
	\item Place the result value in a place where the calling program can access it.
	\item Return control to the point of origin.
\end{enumerate}
The allocated registers are:
\begin{itemize}
	\item \T{\$a0-\$a3}: argument registers to pass parameters
	\item \T{\$v0-\$v1}: two value registers in which to return values
	\item \T{\$ra}: return address register to return to the point of origin
\end{itemize}
\subsection{Addressing}
\subsubsection{Register addressing}
Operands are registers.\\ Example: \T{add \$v0,\$a0,\$a1}.
\subsubsection{Base addressing}
Operand is a memory location whose address is the sum of a register and a constant in the instruction.\\
Example: \T{add 0(\$a0),4(\$a0),8(\$a0)}
\subsubsection{Immediate addressing}
Operand is a constant within the instruction.\\
Example: \T{addi \$v0,\$v0,2}
\subsubsection{PC-relative addressing}
Address is the sum of the PC and a constant in the instruction.\\
Example: \T{beq \$a0,\$a1,L1}
\subsubsection{Pseudodirect addressing}
The 26-bits of the address are concatenated with the upper 4 bits of the PC.\\
Example: \T{j L1}
\subsection{Idioms}
\subsubsection{Conditionals}
\B{Simple equality:}
\begin{minted}{C}
if (a == b) {
	a = a + b
}
\end{minted}
Assume \T{a} is in \T{\$t0} and \T{b} is in \T{\$t1}.
\begin{verbatim}
      bne      $t0,$t1,Exit
      add      $t0,$t1,$t0
Exit: # more code
\end{verbatim}
\B{\T{max(a,b)} function:}
\begin{minted}{C}
if (a > b) {
	c = a
} else {
	c = b
}
\end{minted}
Assume \T{a,b,c} are in \T{\$t0-\$t2}.
\begin{verbatim}
       slt $t3,$t1,$t0
       beq $t3,$zero,ELSE
       add $t2,$t0,$zero
       j ENDIF
ELSE:  add $t2,$t1,$zero
ENDIF: 
\end{verbatim}
\subsubsection{Loops}
\subsubsection{Procedures}
\B{Pushing/Poppping values:}
\begin{verbatim}
push: li  $t0,4
      sub $sp,$sp,$t0
      sw  $v0,0($sp)
pop:  lw  $v0,0($sp)
      li  $t0,4
      add $sp,$sp,$t0
\end{verbatim}
\section{Assembler, Linker and Loader}
To run a set of \emph{pseudoinstructions}, you need three programs:
\begin{enumerate}
	\item Assembler: Turns human readable assembly into machine code.
	\item Linker: Turns machine code module and library routines into executable.
	\item Loader: Loads executable into memory and starts execution.
\end{enumerate}
\subsection{Assembler}
\begin{enumerate}
	\item expands syntactic sugar (e.g. \T{move}, \T{blt}, ...)
	\item validates references (and replaces big branches)
	\item converts instructions to binary
\end{enumerate}
The output of the assembler is called an \emph{object file}. A Unix object file contains
\begin{enumerate}
	\item object header
	\item text segment
	\item data segment
	\item relocation information
	\item symbol table
	\item debugging information
\end{enumerate}
More info at \emph{P\&H, Section 3.9, Page 157}.
\subsection{Linker}
\begin{enumerate}
	\item Places code and data modules symbolically in memory.
	\item Determines the addresses of data and instruction labels.
	\item Patches both the internal and external references.
\end{enumerate}
More info at \emph{P\&H, Section 3.9, Page 158}.
\subsection{Loader}
\begin{enumerate}
	\item Reads header to determine size of text and data segments.
	\item Creates address space large enough for the text and data.
	\item Copies the instructions and data into memory.
	\item Copies the parameters (if any) to the main program onto the stack.
	\item Initialises machine registers and sets stack pointer.
	\item Jumps to start-up routine. When \T{main} exits, returns with \T{exit} system call.
\end{enumerate}
\end{document}
