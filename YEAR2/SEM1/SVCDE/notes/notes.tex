\documentclass{article}
\usepackage[a4paper]{geometry}
\geometry{tmargin=3cm, bmargin=3cm, lmargin=2cm, rmargin=2cm}
\usepackage[british]{babel}
\usepackage{amsmath}
\usepackage{amssymb}
\usepackage{amsthm}
\usepackage{nicefrac}
\usepackage{siunitx}
\usepackage{mathtools}
\usepackage{fontspec}
\usepackage{hyperref}
\hypersetup{
	colorlinks,
	citecolor=black,
	filecolor=black,
	linkcolor=black,
	urlcolor=black
}
\setmainfont{arial}
\newtheoremstyle{sltheorem} {}                % Space above
{}                % Space below
{\upshape}        % Theorem body font % (default is "\upshape")
{}                % Indent amount
{\bfseries}       % Theorem head font % (default is \mdseries)
{.}               % Punctuation after theorem head % default: no punctuation
{ }               % Space after theorem head
{}                % Theorem head spec
\theoremstyle{sltheorem}
\newtheorem{definition}{Definition}[section]
\newtheorem{theorem}{Theorem}[section]
\newtheorem{lemma}[theorem]{Lemma}
\newtheorem{corollary}[theorem]{Corollary}
\newtheorem{proposition}[theorem]{Proposition}
\newcommand{\R}{\mathbb{R}}
\newcommand{\N}{\mathbb{N}}
\renewcommand{\C}{\mathbb{C}}
\newcommand{\Z}{\mathbb{Z}}
\newcommand{\st}{\text{ s.t }}
\newcommand{\ih}{\widehat i}
\newcommand{\jh}{\widehat j}
\newcommand{\kh}{\widehat k}
\newcommand{\p}{\partial}
\newcommand{\grad}{\vec\triangledown}
\DeclareMathOperator{\lub}{LUB}
\DeclareMathOperator{\glb}{GLB}
\DeclareMathOperator{\hcf}{hcf}
\DeclareMathOperator{\lcm}{lcm}
\DeclareMathOperator{\sgn}{sgn}
\DeclareMathOperator{\cl}{cl}
\newcommand{\dv}[1]{\vec #1\:'}
%\DeclareMathOperator{\mod}{mod}
\newcommand*\lneg[1]{\overline{#1}}
\newcommand*\B[1]{\textbf{#1}}
\newcommand*\binco[2]{\begin{pmatrix}
    #1\\#2
\end{pmatrix}}
\begin{document}
\title{Several Variable Calculus and Differential Equations (SEM3)}
\author{Franz Miltz}
\maketitle
\tableofcontents
\section{Vectors and geometry}
(see ILA notes)
\setcounter{subsection}{4}
\subsection{Vector functions}
\begin{definition}
    A function $\vec f$ of the form $\vec f: \R \to \R^3$ is called a \B{vector function}.
\end{definition}
\begin{definition}
    The limit $\vec b$ of a vector function $\vec f$ at $a$ exists if, and only if,
    \begin{align*}
        \forall \epsilon > 0,\: \exists \delta > 0 \st 0 < |t-a| < \delta \Rightarrow |\vec f(t) - \vec b|<\epsilon.
    \end{align*}
    This is the case if, and only if, the limits for all the components of $\vec f$ exist.\\
    A vector function $\vec f(t)$ is \B{continuous} at $t=a$ if
    \begin{align*}
        \lim_{t\to a} \vec f (t) = \vec f (a). 
    \end{align*}
\end{definition}
\subsection{Derivatives and integrals}
\subsubsection{Derivatives}
\begin{definition}
    The \B{derivative of a vector function} $\vec f(t)$ is defined as
    \begin{align*}
        \vec f(t) = \frac{d\vec f}{dt} = \lim_{\delta \to 0} \frac{\vec f(t+\delta)-\vec f(t)}{\delta}.
    \end{align*}
\end{definition}
\begin{theorem}
    Let $\vec v = f(t)\ih + g(t)\jh + h(t)\kh$ where $f,g,h: \R\to\R$. Then
    \begin{align*}
        \frac{d\vec v(t)}{dt} = \frac{d\vec f(t)}{dt}\ih + \frac{d\vec g(t)}{dt}\jh + \frac{d\vec h(t)}{dt}\kh.
    \end{align*}
\end{theorem}
\begin{definition}
    $\vec r'(a)$ is the \B{tangent vector} at the point $t=a$ to the curve traced out by the position vector function $\vec r(t)$.
    The \B{unit tangent vector} at $t=a$ is
    \begin{align*}
        \vec T(a) = \frac{\vec r'(a)}{|\vec r'(a)|}
    \end{align*}
    if $|\vec r'(a)| \not= 0$.
\end{definition}
\subsubsection{Integrals}
\begin{definition}
    The \B{definite integral} of the continuous vector function $\vec v(t)=f(t)\ih + g(t)\jh + h(t)\kh$ is given by
    \begin{align*}
        \int_a^b \vec v(t)dt = \left[\int_a^b f(t)dt\right]\ih + \left[\int_a^b g(t) dt\right]\jh + \left[\int_a^b h(t)dt\right] \kh.
    \end{align*}
\end{definition}
\begin{theorem}{Fundamental Theorem of Calculus}
    \begin{align*}
        \int_a^b \vec v(t) dt = \left[\vec V(t)\right]^b_a = \vec V(b) - \vec V(a)
    \end{align*} 
    where $\vec V'(t) = \vec v(t)$.
\end{theorem}
\subsection{Arc Length}
\begin{theorem}
    The length $L$ of the curve traced out by the position vector function $\vec r(t)$ in three-dimensional space ove the interval
    $[a,b]$ is given by
    \begin{align*}
        L = \int_a^b \left|\frac{d\vec r(t)}{dt}\right| dt
    \end{align*}
\end{theorem}
\subsection{Curvature}
\begin{definition}
    A parameterisation $\vec v(t)$ is \B{smooth} on an interval $I$ if $\vec v'(t)$ is continuous and $\vec v'(t) \not = \vec 0$ on $I$.
\end{definition}
\begin{definition}
    A curve is \B{smooth} if it has a smooth parameterisation.
\end{definition}
\begin{definition}
    For a smooth curved traced out by the position vector $\vec r(t)$ with the unit tangent vector 
    $\vec T(t)$ the \B{curvature} $\kappa$ is defined as
    \begin{align*}
        \kappa = \left|\frac{d\vec T}{ds}\right|=\frac{\left| \vec T'(t)\right|}{\left|\vec r'(t)\right|}
    \end{align*}
    where $s$ is the arc length.
    Alternatively
    \begin{align*}
        \kappa = \frac{\left|\dv r(t) \times \dv r'(t)\right|}{|\dv r(t)|^3}
    \end{align*}
\end{definition}
\subsection{Normal and binormal vectors}
\begin{definition}
    Let the \B{principal unit normal vector} of the curve traced out by the position vector $\vec r(t)$
    where $\vec r'(t)$ is smooth be defined as
    \begin{align*}
        \vec N (t) = \frac{\vec T'(t)}{\left| \vec T'(t) \right|}.
    \end{align*}
\end{definition}
\begin{definition}
    The \B{binormal vector} is defined as
    \begin{align*}
        \vec B(t) = \vec T(t) \times \vec N(t).
    \end{align*}
\end{definition}
\begin{theorem}
    \begin{align*}
        \vec N(t) &= \vec B(t) \times \vec T(t)\\
        \vec T(t) &= \vec N(t) \times \vec B(t)
    \end{align*}
\end{theorem}
\section{Partial differentiation}
\subsection{Functions of several variables}
\begin{definition}
    A function of $n$ variables $f(x_1, x_2, ..., x_n)$ maps an $n$-tuple
    of real numbers $(x_1, x_2, ..., x_n)$ in domain $D\subseteq\R^n$ to
    a unique real number in the range
    \begin{align*}
        \{f(x_1, x_2, ..., x_n)|(x_1, x_2, ..., x_n)\in D\} \subseteq \R.
    \end{align*}
\end{definition}
\begin{definition}
    The \B{level curves} of a function $f(x,y)$ are curves with equations
    $f(x,y)=k$ where $k$ is a constant in the range of $f$.
\end{definition}
\begin{definition}
    The \B{level surfaces} of a function $f(x,y,z)$ are surfaces with
    equations $f(x,y,z)=k$ where $k$ is a constant in the range of $f$.
\end{definition}
\subsection{Limits \& continuity}
\subsubsection{Limits}
\begin{definition}
    Suppose the domain $D$ of a function $f:D\to\R$ includes points
    arbitrarily close to the point $P$ with position vector $\vec p$.
    Then
    \begin{align*}
        \lim_{\vec x\to \vec p}f(\vec x)=L
    \end{align*}
    if for every $\epsilon>0$ there exists a $\delta > 0$ such that
    \begin{align*}
        0 < |\vec x - \vec p| < \delta \:\Rightarrow\: |f(\vec x)-L|<\epsilon
    \end{align*}
    for $\vec x \in D$.\\\\
    Note: The definition is independent of the direction of approach to the
    limit point $\vec p$. Thus, if
    \begin{align*}
        &f(\vec x)\to L_1 \text{ as } \vec x\to\vec p \text{ along a path } C_1, \text{ and}\\
        &f(\vec x)\to L_2 \text{ as } \vec x\to\vec p \text{ along a path } C_2
    \end{align*}
    where $L_1 \not= L_2$, then $\lim_{\vec x\to\vec p} f(\vec x)$ does not exist.
\end{definition}
\begin{theorem}
    The limit laws carry over from functions of one variable case:
    \begin{enumerate}
        \item $\lim_{\vec x\to\vec p}\left(f(\vec x)\pm g(\vec x)\right) =
               \lim_{\vec x\to\vec p}f(\vec x)\pm\lim_{\vec x\to\vec p}g(\vec x)$.
        \item $\lim_{\vec x\to\vec p}\left(cf(\vec x)\right) =
              c\lim_{\vec x\to\vec p}f(\vec x)$ where $c$ is a constant.
        \item $\lim_{\vec x\to\vec p}\left(f(\vec x)g(\vec x)\right) =
               \left(\lim_{\vec x\to\vec p}f(\vec x)\right)
               \left(\lim_{\vec x\to\vec p}g(\vec x)\right)$.
        \item $\lim_{\vec x\to\vec p}\frac{f(\vec x)}{g(\vec x)}
               =\frac{\lim_{\vec x\to\vec p}f(\vec x)}{\lim_{\vec x\to\vec p}g(\vec x)}$
               provided that $\lim_{\vec x\to\vec p}g(\vec x)\not=0$.
    \end{enumerate}
\end{theorem}
\subsubsection{Continuity}
\begin{definition}
    A function $f:D\to R$ with $D\subseteq\R^n$ and $R\subseteq\R$
    is called \B{continuous} at a point $P$ with position vector $\vec p\in \R^n$
    \begin{align*}
        \lim_{\vec x\to \vec p}f(\vec x) = f(\vec p);
    \end{align*}
    and $f$ is \B{continuous on $S\subseteq D$} if it is continuous at 
    every point in $S$.
\end{definition}
\subsection{Partial derivatives}
\begin{definition}
    Consider a function $f:\R^n\to\R$. The $n$ \B{partial derivatives} of
    $f$ are defined by
    \begin{align*}
        f_{x_i}(\vec x)\equiv\frac{\p f}{\p x_i}=\lim_{\delta\to 0}\frac{f(\vec x + \delta \vec e_i)-f(\vec x)}{\delta}
    \end{align*}
    where $i\in[1,n]$, $x_i$ is the $i$th component of $\vec x$ and $\vec e_i$ is the
    $i$th standard unit vector.
\end{definition}
Notation for higher derivatives:
\begin{align*}
    (f_x)_x \equiv f_{xx} \equiv f_{11} 
    &\equiv \frac{\p}{\p x}\frac{\p f}{\p x} 
    \equiv \frac{\p^2 f}{\p x^2}
    \equiv \frac{\p^2 z}{\p x^2}\\
    (f_x)_y \equiv f_{xy} \equiv f_{12} 
    &\equiv \frac{\p}{\p y}\frac{\p f}{\p x} 
    \equiv \frac{\p^2 f}{\p y \p x}
    \equiv \frac{\p^2 z}{\p y \p x}\\
\end{align*}
\begin{theorem}
    Suppose $f(x,y)$ is defined on a disk $D$ containing the point $(a,b)$.
    If the partial derivatives $f_{xy}$ and $f_{yx}$ are both continuous
    on $D$ then
    \begin{align*}
        f_{xy}(a,b)=f_{yx}(a,b).
    \end{align*}
\end{theorem}
\subsection{Tangent planes and linear approximations}
\subsubsection{Tangent planes}
\begin{definition}
    Consider a surface $S$ with equation $z=f(x,y)$ where $f_x$ and $f_y$
    are both continuous. Then the \B{tangent plane} to the surface $S$
    is given by the equation
    \begin{align*}
        z - z_0 = f_x(x_0, y_0)(x-x_0) + f_y(x_0,y_0)(y-y_0).
    \end{align*}
\end{definition}
\subsubsection{Linear approximation}
\begin{definition}
    The linear function
    \begin{align*}
        L(x,y)=f(a,b)+f_x(a,b)(x-a)+f_y(a,b)(y-a),
    \end{align*}
    is called the \B{linearisation} of $f(x,y)$ at $(a,b)$, and the
    approximation
    \begin{align*}
        f(x,y)\approx f(a,b)+f_x(a,b)(x-a)+f_y(a,b)(y-a)
    \end{align*}
    is called the \B{linear approximation} of $f(x,y)$ at $(a,b)$.\\
    The graph of the linearisation is the tangent plane of $f(x,y)$ at
    $(a,b)$.
\end{definition}
\begin{definition}
    Consider the change in $z=f(x,y)$ as we move from the point
    $(a,b)$ to the point $(a+\Delta x, b+\Delta y)$. The
    corresponding \B{increment} in $z$ is
    \begin{align*}
        \Delta z = f_x(a,b)\Delta x + f_y(a,b)\Delta y 
        + \epsilon_1\Delta x + \epsilon_2\Delta y.
    \end{align*}
    The function $f(x,y)$ is \B{differentiable} if $\Delta z$
    may be expressed as
    \begin{align*}
        \Delta z = f_x(a,b)\Delta x + f_y(a,b)\Delta y
        + \epsilon_1\Delta x + \epsilon_2\Delta y
    \end{align*}
    where $\epsilon_1, \epsilon_2 \to 0$ as 
    $(\Delta x, \Delta y)\to(0,0)$.
\end{definition}
\begin{theorem}
    If the partial derivatives $f_x$ and $f_y$ exist near
    $(a,b)$ and are continuous at $(a,b)$, then $f(x,y)$
    is differentiable at $(a,b)$.
\end{theorem}
\subsubsection{Differentials}
\begin{definition}
    For the function $z=f(x,y)$, \B{differentials} $dx$ and $dy$ are
    defined to be independent variables. The \B{total differential} $dz$
    is then defined as
    \begin{align*}
        dz = f_x(x,y)dx + f_y(x,y)dy 
        = \frac{\p z}{\p x}dx + \frac{\p z}{\p y}dy.
    \end{align*}
\end{definition}
\subsection{The chain rule}
\subsubsection{Formulations}
\begin{theorem}[Chain rule - case 1]
    Suppose that $z=f(x,y)$ is a differentiable function of $x$ and $y$,
    where $x=g(t)$ and $y=h(t)$ are both differentiable functions of $t$.
    Then $z$ is a differentiable function of $t$ and
    \begin{align*}
        \frac{dz}{dt}=
        \frac{\p f}{\p x}\frac{dx}{dt}+\frac{\p f}{\p y}\frac{dy}{dt}
        \equiv \frac{\p z}{\p x}\frac{dx}{dt}+\frac{\p z}{\p y}\frac{dy}{dt}
    \end{align*}
\end{theorem}
\begin{theorem}[Chain rule - case 2]
    Suppose that $z=f(x,y)$ is a differentiable function of $x$ and $y$,
    where $x=g(s,t)$ and $y=h(s,t)$ are both differentiable functions of
    $s$ and $t$. Then
    \begin{align*}
        \frac{\p z}{\p s}=\frac{\p z}{\p x}\frac{\p x}{\p s}
        + \frac{\p z}{\p y}\frac{\p y}{\p s}
        \text{\hspace{1cm}and\hspace{1cm}}
        \frac{\p z}{\p t}=\frac{\p z}{\p x}\frac{\p x}{\p t}
        + \frac{\p z}{\p y}\frac{\p y}{\p t}.
    \end{align*} 
\end{theorem}
\begin{theorem}[Chain rule - general case]
    Suppose that $z$ is a differentiable function of the $n$ variables
    $x_1, x_2, ..., x_n$, and each $x_j$ is a differentiable function of
    the $m$ variables $t_1,t_2, ...,t_m$. Then
    \begin{align*}
        \frac{\p z}{\p t_i}=
        \frac{\p z}{\p x_1}\frac{\p x_1}{\p t_i}
        + \frac{\p z}{\p x_2}\frac{\p x_2}{\p t_i}
        + \cdots
        + \frac{\p z}{\p x_n}\frac{\p x_n}{\p t_i}
    \end{align*} 
\end{theorem}
\subsubsection{Implicit differentiation}
\begin{definition}
    Suppose that $z$ is given implicitly as a function $z=f(x,y)$ by an
    equation of the form
    \begin{align*}
        F(x,y,z)=0.
    \end{align*}
    Provided that $F$ and $f$ are differentiable, use the Chain rule to
    differentiate paritally w.r.t. $x$:
    \begin{align*}
        \frac{\p F}{\p x}\frac{\p x}{\p x}+\frac{\p F}{\p y}\frac{\p y}{\p x}
        + \frac{\p F}{\p z}\frac{\p z}{\p x} &= 0\\
        \frac{\p F}{\p x} + \frac{\p F}{\p z}\frac{\p z}{\p x} &= 0
    \end{align*}
\end{definition}
\begin{theorem}[Implicit Function Theorem]
    If $F(x,y,z)$ is defined within a sphere containing $(a,b,c)$
    where $F(a,b,c)=0, F_z(a,b,c)\not=0$ and $F_x,F_y$ and $F_z$ are
    continuous inside the sphere, then
    \begin{align*}
        F(x,y,z)=0
    \end{align*} 
    defines $z$ as a differentiable function of $x$ and $y$ near the
    point $(a,b,c)$.
\end{theorem}
\subsection{Directional derivatives and the gradient vector}
\subsubsection{Directional derivatives}
\begin{definition}
    The \B{directional derivative} of $f(x,y)$ at $(x_0,y_0)$ in the
    direction of a unit vector $\vec u=a\ih + b\jh$ is defined as
    \begin{align*}
        D_{\vec u}f(x_0, y_0)= \lim_{\delta \to 0}\frac{f(x_0+\delta a, y_0 + \delta b)-f(x_0, y_0)}{\delta}.
    \end{align*}
\end{definition}
\begin{theorem}
    If $f$ is a differentiable function of $x$ and $y$, then $f$
    has a directional derivative in the direction of any unit vector
    $\vec u=a\ih + b\jh$ and
    \begin{align*}
        D_{\vec u}f(x,y) = af_x(x,y) + bf_y(x,y).
    \end{align*}
\end{theorem}
\subsubsection{Gradient vector}
\begin{definition}
    The \B{gradient} of the scalar function $f(x,y)$ is the vector
    function defined as
    \begin{align*}
        \grad f(x,y)=\frac{\p f}{\p x}\ih + \frac{\p f}{\p y}\jh.
    \end{align*}
\end{definition}
\begin{theorem}
    The directional derivative may be expressed in terms of gradients
    as 
    \begin{align*}
        D_{\vec u}f(x,y)=\grad f(x,y)\cdot \vec u
    \end{align*}
\end{theorem}
\end{document}
