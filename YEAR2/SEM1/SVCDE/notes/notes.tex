\documentclass{article}
\usepackage[a4paper]{geometry}
\geometry{tmargin=3cm, bmargin=3cm, lmargin=2cm, rmargin=2cm}
\usepackage[british]{babel}
\usepackage{amsmath}
\usepackage{amssymb}
\usepackage{amsthm}
\usepackage{nicefrac}
\usepackage{siunitx}
\usepackage{mathtools}
\usepackage{fontspec}
\setmainfont{arial}
\newtheoremstyle{sltheorem} {}                % Space above
{}                % Space below
{\upshape}        % Theorem body font % (default is "\upshape")
{}                % Indent amount
{\bfseries}       % Theorem head font % (default is \mdseries)
{.}               % Punctuation after theorem head % default: no punctuation
{ }               % Space after theorem head
{}                % Theorem head spec
\theoremstyle{sltheorem}
\newtheorem{definition}{Definition}[section]
\newtheorem{theorem}{Theorem}[section]
\newtheorem{lemma}[theorem]{Lemma}
\newtheorem{corollary}[theorem]{Corollary}
\newtheorem{proposition}[theorem]{Proposition}
\newcommand{\R}{\mathbb{R}}
\newcommand{\N}{\mathbb{N}}
\newcommand{\C}{\mathbb{C}}
\newcommand{\Z}{\mathbb{Z}}
\newcommand{\st}{\text{ s.t }}
\newcommand{\ih}{\widehat i}
\newcommand{\jh}{\widehat j}
\newcommand{\kh}{\widehat k}
\DeclareMathOperator{\lub}{LUB}
\DeclareMathOperator{\glb}{GLB}
\DeclareMathOperator{\hcf}{hcf}
\DeclareMathOperator{\lcm}{lcm}
\DeclareMathOperator{\sgn}{sgn}
\DeclareMathOperator{\cl}{cl}
%\DeclareMathOperator{\mod}{mod}
\newcommand*\lneg[1]{\overline{#1}}
\newcommand*\B[1]{\textbf{#1}}
\newcommand*\binco[2]{\begin{pmatrix}
    #1\\#2
\end{pmatrix}}
\begin{document}
\title{Several Variable Calculus and Differential Equations (SEM3)}
\author{Franz Miltz}
\maketitle
\tableofcontents
\section{Vectors and geometry}
(see ILA notes)
\setcounter{subsection}{4}
\subsection{Vector functions}
\begin{definition}
    A function $\vec f$ of the form $\vec f: \R \to \R^3$ is called a \B{vector function}.
\end{definition}
\begin{definition}
    The limit $\vec b$ of a vector function $\vec f$ at $a$ exists if, and only if,
    \begin{align*}
        \forall \epsilon > 0,\: \exists \delta > 0 \st 0 < |t-a| < \delta \Rightarrow |\vec f(t) - \vec b|<\epsilon.
    \end{align*}
    This is the case if, and only if, the limits for all the components of $\vec f$ exist.\\
    A vector function $\vec f(t)$ is \B{continuous} at $t=a$ if
    \begin{align*}
        \lim_{t\to a} \vec f (t) = \vec f (a). 
    \end{align*}
\end{definition}
\subsection{Derivatives and integrals}
\subsubsection{Derivatives}
\begin{definition}
    The \B{derivative of a vector function} $\vec f(t)$ is defined as
    \begin{align*}
        \vec f(t) = \frac{d\vec f}{dt} = \lim_{\delta \to 0} \frac{\vec f(t+\delta)-\vec f(t)}{\delta}.
    \end{align*}
\end{definition}
\begin{theorem}
    Let $\vec v = f(t)\ih + g(t)\jh + h(t)\kh$ where $f,g,h: \R\to\R$. Then
    \begin{align*}
        \frac{d\vec v(t)}{dt} = \frac{d\vec f(t)}{dt}\ih + \frac{d\vec g(t)}{dt}\jh + \frac{d\vec h(t)}{dt}\kh.
    \end{align*}
\end{theorem}
\begin{definition}
    $\vec r'(a)$ is the \B{tangent vector} at the point $t=a$ to the curve traced out by the position vector function $\vec r(t)$.
    The \B{unit tangent vector} at $t=a$ is
    \begin{align*}
        \vec T(a) = \frac{\vec r'(a)}{|\vec r'(a)|}
    \end{align*}
    if $|\vec r'(a)| \not= 0$.
\end{definition}
\subsubsection{Integrals}
\begin{definition}
    The \B{definite integral} of the continuous vector function $\vec v(t)=f(t)\ih + g(t)\jh + h(t)\kh$ is given by
    \begin{align*}
        \int_a^b \vec v(t)dt = \left[\int_a^b f(t)dt\right]\ih + \left[\int_a^b g(t) dt\right]\jh + \left[\int_a^b h(t)dt\right] \kh.
    \end{align*}
\end{definition}
\begin{theorem}{Fundamental Theorem of Calculus}
    \begin{align*}
        \int_a^b \vec v(t) dt = \left[\vec V(t)\right]^b_a = \vec V(b) - \vec V(a)
    \end{align*} 
    where $\vec V'(t) = \vec v(t)$.
\end{theorem}
\subsection{Arc Length}
\begin{theorem}
    The length $L$ of the curve traced out by the position vector function $\vec r(t)$ in three-dimensional space ove the interval
    $[a,b]$ is given by
    \begin{align*}
        L = \int_a^b \left|\frac{d\vec r(t)}{dt}\right| dt
    \end{align*}
\end{theorem}
\subsection{Curvature}
\begin{definition}
    A parameterisation $\vec v(t)$ is \B{smooth} on an interval $I$ if $\vec v'(t)$ is continuous and $\vec v'(t) \not = \vec 0$ on $I$.
\end{definition}
\begin{definition}
    A curve is \B{smooth} if it has a smooth parameterisation.
\end{definition}
\begin{definition}
    For a smooth curved traced out by the position vector $\vec r(t)$ with the unit tangent vector 
    $\vec T(t)$ the \B{curvature} $\mathcal{K}$ is defined as
    \begin{align*}
        \mathcal{K} = \left|\frac{d\vec T}{ds}\right|=\frac{\left| \vec T'(t)\right|}{\left|\vec r'(t)\right|}
    \end{align*}
    where $s$ is the arc length.
\end{definition}
\subsection{Normal and binormal vectors}
\begin{definition}
    Let the \B{principal unit normal vector} of the curve traced out by the position vector $\vec r(t)$
    where $\vec r'(t)$ is smooth be defined as
    \begin{align*}
        \vec N (t) = \frac{\vec T'(t)}{\left| \vec T'(t) \right|}.
    \end{align*}
\end{definition}
\begin{definition}
    The \B{binormal vector} is defined as
    \begin{align*}
        \vec B(t) = \vec T(t) \times \vec N(t).
    \end{align*}
\end{definition}
\begin{theorem}
    \begin{align*}
        \vec N(t) &= \vec B(t) \times \vec T(t)\\
        \vec T(t) &= \vec N(t) \times \vec B(t)
    \end{align*}
\end{theorem}
\end{document}
