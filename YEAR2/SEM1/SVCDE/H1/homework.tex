\documentclass{article}
\usepackage[a4paper]{geometry}
\usepackage[british]{babel}
\usepackage{amsmath}
\usepackage{amssymb}
\usepackage{mathtools}
\usepackage{nicefrac}
\usepackage{amsthm}
\usepackage{changepage}
\geometry{tmargin=2cm, bmargin=3cm}
\DeclarePairedDelimiter{\floor}{\lfloor}{\rfloor}
\title{SVCDE: Hand-in 1}
\author{Franz Miltz (UNN: S1971811)}
\newcommand*\lneg[1]{\overline{#1}}
\newcommand{\R}{\mathbb{R}}
\newcommand{\N}{\mathbb{N}}
\newcommand{\Z}{\mathbb{Z}}
\newcommand{\C}{\mathbb{C}}
\newcommand{\st}{\text{ s.t. }}
%newenvironment{claim}[1]{\noindent\emph{Claim.}\space#1}{}
\newenvironment{claimproof}[1]{\par\noindent\emph{Proof.}\space#1}{\hfill $\blacksquare$}
\newtheorem{claim}[section]{Claim}
\newtheorem{lemma}{Lemma}[section]
\DeclareMathOperator{\lub}{\text{LUB}}
\DeclareMathOperator{\hcf}{hcf}
\DeclareMathOperator{\lcm}{lcm}
\setcounter{MaxMatrixCols}{20}
\newcommand*\binco[2]{\begin{pmatrix}
  #1\\#2
\end{pmatrix}}
\newcommand{\ih}{\widehat i}
\newcommand{\jh}{\widehat j}
\newcommand{\kh}{\widehat k}
\newcommand{\K}{\kappa}
\newcommand{\dv}[1]{\vec #1\, '}
\renewcommand{\d}[1]{#1'}
\begin{document}
\date{27th September 2020}
\maketitle
\section*{Problem 1}
Let $P$ be a plane given by the equation
\begin{align*}
  x - 2y + 4z = 5.
\end{align*}
Then the vector
\begin{align*}
  \vec n = \ih - 2 \jh + 4 \kh
\end{align*}
is normal to $P$. Thus a line $L$ perpendicular to $P$ has to be
parallel to $\vec n$. The line parallel to $\vec n$ that goes through
the point $(1,-2,3)$ with the position vector $\vec p$ is traced by the position vector
\begin{align*}
  \vec r(t) = \vec p + t \vec n
  = (\ih - 2\jh + 3\kh) + t(\ih - 2\jh + 4\kh).
\end{align*}
Therefore we get
\begin{align*}
  \vec r (t) = f(t)\ih + g(t)\jh + h(t)\kh = (1+t)\ih+(-2-2t)\jh+(3+4t)\kh.
\end{align*}
\section*{Problem 2}
The curve with the equation $y=f(x)$ has the parametrisation
\begin{align*}
  \vec r(x) = x \ih + f(x) \jh.
\end{align*}
Then the curvature $\K$ at a point $(x, f(x))$ is given by
\begin{align*}
  \K(x) = \frac{\left|\dv T(x)\right|}{\left|\dv r(x)\right|}.
\end{align*}
We know that 
\begin{align}
  \label{eq1}
  \dv r (x) = \ih + \d f(x) \jh.
\end{align}
Further, the tangent unit vector $\vec T(x)$ is defined by
\begin{align*}
  \vec T(x) = \frac{\dv r(x)}{\left|\dv r(x)\right|}.
\end{align*}
Therefore we get
\begin{align*}
  \K(x) = \frac{\left|\frac{\dv r(x)}{\left|\dv r(x)\right|}\right|}{\left|\dv r(x)\right|}
\end{align*}
Using the fact that for all $\vec v \in \R^n$ and $r\in\R$
\begin{align*}
  \left|r\vec v\right| = r\left|\vec v\right|,
\end{align*}
we get
\begin{align*}
  \K(x) = \frac{\left|\dv r(x)\right|}{\left|\dv r(x)\right|^2} = \frac{1}{|\dv r(x)|}.
\end{align*}
With (\ref{eq1}) we get
\begin{align*}
  \K(x) = \frac{1}{\sqrt{1+\left(\d f(x)\right)^2}}.
\end{align*}
\section*{Problem 3}
Let
\begin{align*}
  \dv r(t) = t\ih + \frac{4}{3}t^{3/2}\jh + \frac{1}{2}t^2\kh
\end{align*}
be a position vector function. 
Then the length $L$ of the curve traced out by $\vec r$ for $0\leq t \leq 2$ is given by
\begin{align*}
  L = \int_0^2 \left|\dv r(t)\right|dt.
\end{align*}
We find
\begin{align*}
  \dv r(t) = \ih + 2\sqrt{t}\:\jh + t^2\kh
\end{align*}
and thus
\begin{align*}
  |\dv r(t)| = \sqrt{1 + 4t + t^2}=|t+1|.
\end{align*}
Since for $t\in[0,2]$ the value $|\dv r(t)|=|t+1|$ is never
negative, $L$ is given by
\begin{align*}
  L = \int_0^2 (t+1)dt = \left[\frac{1}{2}t^2+t\right]_0^2=4.
\end{align*}
\end{document}