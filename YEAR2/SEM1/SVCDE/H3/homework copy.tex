\documentclass{article}
\usepackage[a4paper]{geometry}
\usepackage[british]{babel}
\usepackage{amsmath}
\usepackage{amssymb}
\usepackage{mathtools}
\usepackage{nicefrac}
\usepackage{amsthm}
\usepackage{changepage}
\geometry{tmargin=2cm, bmargin=3cm}
\DeclarePairedDelimiter{\floor}{\lfloor}{\rfloor}
\title{SVCDE: Hand-in 3}
\author{Franz Miltz (UNN: S1971811)}
\newcommand*\lneg[1]{\overline{#1}}
\newcommand{\R}{\mathbb{R}}
\newcommand{\N}{\mathbb{N}}
\newcommand{\Z}{\mathbb{Z}}
\newcommand{\C}{\mathbb{C}}
\newcommand{\st}{\text{ s.t. }}
\newcommand{\p}{\partial }
%newenvironment{claim}[1]{\noindent\emph{Claim.}\space#1}{}
\newenvironment{claimproof}[1]{\par\noindent\emph{Proof.}\space#1}{\hfill $\blacksquare$}
\newtheorem{claim}[section]{Claim}
\newtheorem{lemma}{Lemma}[section]
\DeclareMathOperator{\lub}{\text{LUB}}
\DeclareMathOperator{\hcf}{hcf}
\DeclareMathOperator{\lcm}{lcm}
\setcounter{MaxMatrixCols}{20}
\newcommand*\binco[2]{\begin{pmatrix}
  #1\\#2
\end{pmatrix}}
\newcommand{\ih}{\widehat i}
\newcommand{\jh}{\widehat j}
\newcommand{\kh}{\widehat k}
\newcommand{\K}{\mathcal{K}}
\newcommand{\dv}[1]{\vec #1\, '}
\renewcommand{\d}[1]{#1'}
\newenvironment{itpars}{\par\itshape}{\par}
\newenvironment{itquote}{\begin{quote}\itshape}{\end{quote}\ignorespacesafterend}
\begin{document}
\date{11th October 2020}
\maketitle
\section*{Problem 1}
We have
\begin{align*}
  z &= f(x,y),\\
  x &= r\cos t,\\
  y &= r\sin t.
\end{align*}
We want to show that
\begin{align*}
  \frac{\p^2 z}{\p x^2}+\frac{\p^2 z}{\p y^2}
  = \frac{\p^2 z}{\p r^2}+a(r)\frac{\p z}{\p r}+b(r)\frac{\p^2 z}{\p t^2},
\end{align*}
or equivalently
\begin{align}
  \label{eq1}
  z_{xx} + z_{yy} = z_{rr} + a(r)z_r + b(r)z_{tt}.
\end{align}
To do that, we need to find $z_{xx}$ and $z_{yy}$ in terms 
of $z_{rr}$, $z_r$, $z_{tt}$ and $r$.\\
To do that, we determine $z_r$ and $z_t$:
\begin{align*}
  z_r &= z_x x_r + z_y y_r\\
  z_t &= z_x x_t + z_y y_t.
\end{align*}
This follows directly from the \emph{Chain Rule}. Notice that these equations
are linear with respect to $z_x$ and $z_y$. Solving for both of them gives
\begin{align*}
  z_x &= \frac{z_r y_t - z_t y_r}{r},\\
  z_y &= \frac{z_r x_t - z_t x_r}{r}.
\end{align*}
To get $z_{xx}$ and $z_{yy}$ we can differentiate further. This leads to
\begin{align*}
  z_{xx} &= \frac{(z_r y_t-z_t y_r)_x r - (z_r y_t - z_t y_r)r_x}{r^2},\\
  z_{yy} &= \frac{(z_r x_t - z_t x_r)_y r + (z_r x_t - z_t x_r)r_y}{r^2}.
\end{align*}
By adding this, we obtain
\begin{align*}
  z_{xx}+ z_{yy} =\frac{1}{r^2}( 
   (z_r y_t-z_t y_r)_x r +(z_r x_t - z_t x_r)_y r - (z_r y_t - z_t y_r)r_x
   + (z_r x_t - z_t x_r)r_y).
\end{align*}
We will now evaluate all the remaining derivatives. We get
\begin{align*}
  (z_r x - z_t y_r)_x &= (z_r y_t)_x - (z_t y_r)_x = z_{rx} y_t - z_{tx} y_r,\\
  (z_r y - z_t x_r)_y &= (z_r x_t)_y - (z_t y_r)_y = z_{ry} x_t - z_{ty} x_r.
\end{align*}
Note that $x$ and $y$ are independent and therefore we do not need to apply the
Chain Rule. If we did, we would get the same result as the summands containing
$y_{rx}=(y_x)_r$ and $x_{ry}=(x_y)_r$ respectively would would be $0$.\\
Furthermore, we know that
\begin{align*}
  r = \frac{x}{\cos t} &\Rightarrow r_x = \frac{1}{\cos t},\\
  r = \frac{y}{\sin t} &\Rightarrow r_y = \frac{1}{\sin t}.
\end{align*}
Thus we get
\begin{align}
  \label{eq2}
  z_{xx} + z_{yy} = \frac{1}{r^2}\left(
    (z_{rx}y_t - z_{tx}y_r)r + (z_{ry}x_t-z_{ty}x_r)r
    -\frac{z_r y_t -z_t y_r}{\cos t} - \frac{z_r x_t - z_t x_r}{\sin t}
  \right).
\end{align}
This still contains many things we want to get rid off. Let's get rid off 
the mixed second order partial derivatives $z_{rx}$, $z_{tx}$, $z_{ry}$ and $z_{ty}$. 
Due to \emph{Clairaut's theorem} we know that
\begin{align*}
  z_{rx} &= (z_x)_r = \left(\frac{z_r y_t - z_t y_r}{r}\right)_r,\\
  z_{ry} &= (z_y)_r = \left(\frac{z_r x_t - z_t x_r}{r}\right)_r.
\end{align*}
Applying the quotient rule we obtain
\begin{align*}
  z_{rx} &= \frac{1}{r^2}\left((z_r y_t - z_t y_r)_r r - (z_r y_t - z_t y_r)r_r\right),\\
  z_{ry} &= \frac{1}{r^2}\left((z_r x_t - z_t x_r)_r r - (z_r x_t - z_t x_r)r_r\right).
\end{align*}
Further
\begin{align*}
  (z_r y_t - z_t y_r)_r 
  &= (z_r y_t)_r - (z_t y_r)_r
  = z_{rr}y_t + z_r y_{rt} - z_{rt} y_r - z_t y_{rr},\\ 
  (z_r x_t - z_t x_r)_r
  &= (z_r x_t)_r - (z_t x_r)_r
  = z_{rr} x_t + z_r x_{rt} - z_{rt} x_r - z_t y_{rr}.
\end{align*}
With $r_r=1$ and $x_{rr}=y_{rr}=0$, we find
\begin{align*}
  z_{rx} &= 
  z_{ry} &= \frac{1}{r^2}\left((z_{rr} x_t + z_r x_{rt} - z_{rt} x_r) r - z_r x_t + z_t x_r\right).
\end{align*}
Similarly, we can evaluate
\begin{align*}
  z_{tx} &= \left(\frac{z_r y_t - z_t y_r}{r}\right)_t 
  = \frac{(z_r y_t)_t - (z_t y_r)_t}{r}
  = \frac{z_{rt}y_t + z_ry_{tt} - z_{tt}y_{r}}{r},\\
  z_{ty} &= \left(\frac{z_r x_t - z_t x_r}{r}\right)_t
  = \frac{(z_r x_t)_t - (z_t x_r)_t}{r}
  = \frac{z_{rt}x_t + z_r x_{tt} - z_{tt}x_r}{r}.
\end{align*}
To evaluate equation \ref{eq2} further, we are going to have to insert our
results into the summands to the left of the right-hand side. We get
\begin{align*}
  (z_{rx} y_t - z_{tx} y_r)r &= \left(
    z_{rr}y_t + z_r y_{rt} - z_{rt} y_r - \frac{z_r y_t + z_t y_r}{r}\right y_t
    -z_{rt}y_t + z_r y_{tt} - z_{tt} y_r
  \right)
  &
\end{align*}
\end{document}
