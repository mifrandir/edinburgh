\documentclass{article}
\usepackage[a4paper]{geometry}
\usepackage[british]{babel}
\usepackage{amsmath}
\usepackage{amssymb}
\usepackage{mathtools}
\usepackage{nicefrac}
\usepackage{amsthm}
\usepackage{changepage}
\geometry{tmargin=2cm, bmargin=3cm}
\DeclarePairedDelimiter{\floor}{\lfloor}{\rfloor}
\title{SVCDE: Hand-in 3}
\author{Franz Miltz (UNN: S1971811)}
\newcommand*\lneg[1]{\overline{#1}}
\newcommand{\R}{\mathbb{R}}
\newcommand{\N}{\mathbb{N}}
\newcommand{\Z}{\mathbb{Z}}
\newcommand{\C}{\mathbb{C}}
\newcommand{\st}{\text{ s.t. }}
\newcommand{\p}{\partial }
%newenvironment{claim}[1]{\noindent\emph{Claim.}\space#1}{}
\newenvironment{claimproof}[1]{\par\noindent\emph{Proof.}\space#1}{\hfill $\blacksquare$}
\newtheorem{claim}[section]{Claim}
\newtheorem{lemma}{Lemma}[section]
\DeclareMathOperator{\lub}{\text{LUB}}
\DeclareMathOperator{\hcf}{hcf}
\DeclareMathOperator{\lcm}{lcm}
\setcounter{MaxMatrixCols}{20}
\newcommand*\binco[2]{\begin{pmatrix}
  #1\\#2
\end{pmatrix}}
\newcommand{\ih}{\widehat i}
\newcommand{\jh}{\widehat j}
\newcommand{\kh}{\widehat k}
\newcommand{\K}{\mathcal{K}}
\newcommand{\dv}[1]{\vec #1\, '}
\renewcommand{\d}[1]{#1'}
\newenvironment{itpars}{\par\itshape}{\par}
\newenvironment{itquote}{\begin{quote}\itshape}{\end{quote}\ignorespacesafterend}
\begin{document}
\date{11th October 2020}
\maketitle
\section*{Problem 1}
We have
\begin{align*}
  z &= f(x,y),\\
  x &= r\cos t,\\
  y &= r\sin t.
\end{align*}
We want to show that
\begin{align*}
  \frac{\p^2 z}{\p x^2}+\frac{\p^2 z}{\p y^2}
  = \frac{\p^2 z}{\p r^2}+a(r)\frac{\p z}{\p r}+b(r)\frac{\p^2 z}{\p t^2},
\end{align*}
or equivalently
\begin{align}
  \label{eq1}
  z_{xx} + z_{yy} = z_{rr} + a(r)z_r + b(r)z_{tt}.
\end{align}
To do that, we need to find $z_{xx}$ and $z_{yy}$ in terms 
of $z_{rr}$, $z_r$, $z_{tt}$ and $r$.\\
To do that, we determine $z_r$ and $z_t$:
\begin{align*}
  z_r &= z_x x_r + z_y y_r\\
  z_t &= z_x x_t + z_y y_t.
\end{align*}
This follows directly from the \emph{Chain Rule}. Notice that these equations
are linear with respect to $z_x$ and $z_y$. Solving for both of them gives
\begin{align*}
  z_x &= \frac{z_r y_t - z_t y_r}{r},\\
  z_y &= \frac{z_r x_t - z_t x_r}{r}.
\end{align*}
To get $z_{xx}$ and $z_{yy}$ we can differentiate further. This leads to
\begin{align*}
  z_{xx} &= \frac{(z_r y_t-z_t y_r)_x r - (z_r y_t - z_t y_r)r_x}{r^2},\\
  z_{yy} &= \frac{(z_r x_t - z_t x_r)_y r + (z_r x_t - z_t x_r)r_y}{r^2}.
\end{align*}
By adding this, we obtain
\begin{align*}
  z_{xx}+ z_{yy} =\frac{1}{r^2}( 
   (z_r y_t-z_t y_r)_x r +(z_r x_t - z_t x_r)_y r - (z_r y_t - z_t y_r)r_x
   + (z_r x_t - z_t x_r)r_y).
\end{align*}
We will now evaluate all the remaining derivatives. We get
\begin{align*}
  (z_r x - z_t y_r)_x &= (z_r y_t)_x - (z_t y_r)_x = z_{rx} y_t - z_{tx} y_r,\\
  (z_r y - z_t x_r)_y &= (z_r x_t)_y - (z_t y_r)_y = z_{ry} x_t - z_{ty} x_r.
\end{align*}
Note that $x$ and $y$ are independent and therefore we do not need to apply the
Chain Rule. If we did, we would get the same result as the summands containing
$y_{rx}=(y_x)_r$ and $x_{ry}=(x_y)_r$ respectively would would be $0$.\\
Furthermore, we know that
\begin{align*}
  r = \frac{x}{\cos t} &\Rightarrow r_x = \frac{1}{\cos t},\\
  r = \frac{y}{\sin t} &\Rightarrow r_y = \frac{1}{\sin t}.
\end{align*}
Thus we get
\begin{align}
  \label{eq2}
  z_{xx} + z_{yy} = \frac{1}{r^2}\left(
    (z_{rx}y_t - z_{tx}y_r)r + (z_{ry}x_t-z_{ty}x_r)r
    -\frac{z_r y_t -z_t y_r}{\cos t} - \frac{z_r x_t - z_t x_r}{\sin t}
  \right).
\end{align}
This still contains many things we want to get rid off. Let's get rid off 
the mixed second order partial derivatives $z_{rx}$, $z_{tx}$, $z_{ry}$ and $z_{ty}$. 
Due to \emph{Clairaut's theorem} we know that
\begin{align*}
  z_{rx} &= (z_x)_r = \left(\frac{z_r y_t - z_t y_r}{r}\right)_r,\\
  z_{ry} &= (z_y)_r = \left(\frac{z_r x_t - z_t x_r}{r}\right)_r.
\end{align*}
Applying the quotient rule we obtain
\begin{align*}
  z_{rx} &= \frac{1}{r^2}\left((z_r y_t - z_t y_r)_r r - (z_r y_t - z_t y_r)r_r\right),\\
  z_{ry} &= \frac{1}{r^2}\left((z_r x_t - z_t x_r)_r r - (z_r x_t - z_t x_r)r_r\right).
\end{align*}
Further
\begin{align*}
  (z_r y_t - z_t y_r)_r 
  &= (z_r y_t)_r - (z_t y_r)_r
  = z_{rr}y_t + z_r y_{rt} - z_{rt} y_r - z_t y_{rr},\\ 
  (z_r x_t - z_t x_r)_r
  &= (z_r x_t)_r - (z_t x_r)_r
  = z_{rr} x_t + z_r x_{rt} - z_{rt} x_r - z_t y_{rr}.
\end{align*}
With $r_r=1$ and $x_{rr}=y_{rr}=0$, we find
\begin{align*}
  z_{rx} &= \frac{1}{r^2}\left((z_{rr}y_t + z_r y_{rt} - z_{rt} y_r) r - z_r y_t + z_t y_r\right),\\
  z_{ry} &= \frac{1}{r^2}\left((z_{rr} x_t + z_r x_{rt} - z_{rt} x_r) r - z_r x_t + z_t x_r\right).
\end{align*}
Similarly, we can evaluate
\begin{align}
  \label{eqztx}
  z_{tx} &= \left(\frac{z_r y_t - z_t y_r}{r}\right)_t 
  = \frac{(z_r y_t)_t - (z_t y_r)_t}{r}
  = \frac{z_{rt}y_t + z_ry_{tt} - z_{tt}y_{r}}{r},\\
  \label{eqzty}
  z_{ty} &= \left(\frac{z_r x_t - z_t x_r}{r}\right)_t
  = \frac{(z_r x_t)_t - (z_t x_r)_t}{r}
  = \frac{z_{rt}x_t + z_r x_{tt} - z_{tt}x_r}{r}.
\end{align}
To evaluate equation \ref{eq2} further, we are going to have to insert our
results into the summands to the left of the right-hand side. We get
\begin{align*}
  (z_{rx} y_t - z_{tx} y_r)r &= \left(
    z_{rr}y_t + z_r y_{rt} - z_{rt} y_r - \frac{z_r y_t + z_t y_r}{r}\right) y_t
    -\left(z_{rt}y_t + z_r y_{tt} - z_{tt} y_r\right)y_r\\
    &= z_{rr}y^2_t + z_r y_{rt}y_r-z_{rt}y_ry_t - \frac{z_ry^2_t + z_t y_r y_t}{r} - z_{rt}y_t y_r + z_r y_{tt} y_r - z_{tt} y^2_r\\
    &= z_{rr}y^2_t + z_r \left(y_{rt} y_r - \frac{y^2_t}{r}-y_{tt} y_r\right) - z_{rt} \left(y_r y_t + y_t y_r\right) 
    + z_t \left(\frac{y_r y_t}{r}\right) - z_{tt} y^2_r
\end{align*}
By inserting $y_r = \sin t$, $y_t = r\cos t$, $y_{tt} = -r \sin t$ and $y_{rt} = \cos t$, we get
\begin{align*}
  (z_{rx} y_t - z_{tx} y_r)r &= z_{rr}(r^2 \cos^2 t) + z_r (r\sin^2 t) - z_{rt} (2r \sin t \cos t) + z_t (\sin t \cos t) - z_{tt} (\sin^2 t).
\end{align*}
Similarly
\begin{align*}
  (z_{ry}x_t - z_{ty}x_r)r &= \left(
    z_{rr}x_t + z_r x_{rt} - z_{rt}x_r - \frac{z_r x_t + z_t x_r}{r}
  \right) x_t 
  -\left(z_{rt}x_t + z_r x_{tt} - z_{tt} x_r\right)x_r\\
  &=z_{rr}x^2_t + z_r x_{rt} x_t - z_{rt}x_r x_t - \frac{z_r x^2_t + z_t x_r x_t}{r}-z_{rt}x_t x_r + z_r x_{tt} x_r - z_{tt} x^2_r\\
  &=z_{rr} x^2_t + z_r \left(x_{rt}x_t - \frac{x^2_t}{r}+x_{tt} x_r\right) - z_{rt} (x_r x_t + x_t x_r) + z_t\left(\frac{x_r x_t}{r}\right)
  - z_{tt} x^2_r.
\end{align*}
With $x_r = \cos t$, $x_t = -r \sin t$, $x_{tt}=-r\cos t$ and $x_{rt}=-\sin t$
\begin{align*}
  (z_{ry}x_t - z_{ty}x_r)r= z_{rr}(r^2 \sin^2 t) - z_r (r\cos^2 t) + z_{rt}(2r\sin t\cos t) - z_t(\sin t \cos t) - z_{tt}(\cos^2 t).
\end{align*}
Now we can add both of these expressions together:
\begin{align*}
  (z_{rx}y_t - z_{tx}y_r)r &+ (z_{ry}x_t-z_{ty}x_r)r\\
  &= z_{rr}(r^2 \cos^2 t) + z_r (r\sin^2 t) - z_{rt} (2r \sin t \cos t) + z_t (\sin t \cos t) - z_{tt} (\sin^2 t)\\
  &+z_{rr}(r^2 \sin^2 t) - z_r (r\cos^2 t) + z_{rt}(2r\sin t\cos t) - z_t(\sin t \cos t) - z_{tt}(\cos^2 t)\\
  &=z_{rr}r^2+z_r r(\sin^2 t - \cos^2 t) - z_{tt}.
\end{align*}
We can plug that back into equation \ref{eq2} to get
\begin{align*}
  z_{xx} + z_{yy} &= \frac{1}{r^2}\left(
    z_{rr}r^2+z_r r(\sin^2 t - \cos^2 t) - z_{tt}
    -\frac{z_r y_t -z_t y_r}{\cos t} - \frac{z_r x_t - z_t x_r}{\sin t}
  \right)\\
  &= z_{rr} + \frac{z_r}{r^2}\left(r(\sin^2 t - \cos^2 t) - \frac{y_t}{\cos t} - \frac{x_t}{\sin t}\right)-\frac{z_{tt}}{r^2}
  +\frac{z_t}{r^2}\left(\frac{y_r}{\cos t} + \frac{x_r}{\sin t}\right)
\end{align*}
This results in
\begin{align}
  \label{eq2z}
  z_{xx} + z_{yy}= z_{rr} + z_r\left(\frac{\sin^2 t - \cos^2 t}{r}\right)-\frac{z_{tt}}{r^2}+\frac{z_t}{r^2\sin t\cos t}.
\end{align}
As you can see this is not quite as required. Firstly, $a$ still depends on $t$ and secondly, $z_t$ should not appear whatsoever.
Therefore, we need to try and write $z_{tt}$ in terms of $z_r$ and $z_t$ so that we can replace $z_t$. Note that
\begin{align*}
  z_{tt} = (z_t)_t = (z_x x_t)_t + (z_y y_t)_t = z_{tx}x_t + z_x x_{tt} + z_{ty}y_t + z_y y_{tt}.
\end{align*}
We can express $z_x$ and $z_y$ by using the Chain rule once again:
\begin{align*}
  z_x &= z_r r_x + z_t t_x\\
  z_y &= z_r r_y + z_t t_y
\end{align*}
With $t=\cos^{-1}\left(\frac{x}{r}\right)$ and $t=\sin^{-1}\left(\frac{y}{r}\right)$, we find
\begin{align*}
  t_x &= -\frac{1}{r\sqrt{1-\left(\frac{x}{r}\right)^2}}  = - \frac{1}{r\sqrt{1-\cos^2 t}} = -\frac{1}{y},\\
  t_y &= \frac{1}{r\sqrt{1-\left(\frac{y}{r}\right)^2}}  = \frac{1}{r\sqrt{1-\sin^2 t}} = -\frac{1}{y}.
\end{align*}
We can now write
\begin{align*}
  z_{tt} &=
  \left(\frac{z_{rt}y_t + z_ry_{tt} - z_{tt}y_{r}}{r}\right)x_t + (z_r r_x + z_t t_x)x_{tt}\\
  &+\left(\frac{z_{rt}x_t + z_r x_{tt} - z_{tt}x_r}{r}\right)y_t + (z_r r_y + z_t t_y)y_{tt}\\
\end{align*}
Reorganising gives
\begin{align}
  \label{eqztt}
  z_{tt} =z_{rt}\left(\frac{2 x_t y_t}{r}\right)+z_r\left(\frac{y_{tt}x_t+x_{tt}y_t}{r}+r_x x_{tt}+r_y y_{tt}\right)
  +z_t(t_x x_{tt} + t_y y_{tt})-z_{tt}\left(\frac{x_r+y_r}{r}\right).
\end{align}
We know that
\begin{align*}
  z_{rt} = z_{tx}x_r + z_x x_{rt} + z_{ty} y_r + z_y y_{rt}.
\end{align*}
Using equations \ref{eqztx} and \ref{eqzty} we get
\begin{align*}
  z_{rt} &= \left(\frac{z_{rt} y_t + z_r y_{tt} - z_{tt}y_r}{r}\right)x_r
  + \left(z_r r_x + z_t t_x\right)x_{rt}\\
  &+ \left(\frac{z_{rt}x_t + z_r x_{tt} - z_{tt}x_r}{r}\right)y_r
  + \left(z_r r_y + z_t t_y\right)y_{rt}
\end{align*}
Let's evaluate the summands individually:
\begin{align*}
  \left(\frac{z_{rt} y_t + z_r y_{tt} - z_{tt}y_r}{r}\right)x_r
  &= z_{rt}\left(\frac{x_r y_t}{r}\right) + z_r \left(\frac{y_{tt}x_r}{r}\right)-z_{tt}\left(\frac{x_r y_r}{r}\right)\\
  &= z_{rt}\left(\cos^2 t\right) + z_r \left(-\sin t\cos t\right)-z_{tt}\left(\frac{\sin t\cos t}{r}\right),\\
  \left(\frac{z_{rt}x_t + z_r x_{tt} - z_{tt}x_r}{r}\right)y_r
  &= z_{rt}\left(\frac{x_t y_r}{r}\right)+z_r\left(\frac{x_{tt} y_r}{r}\right)-z_{tt}\left(\frac{x_r y_r}{r}\right)\\
  &= z_{rt}\left(-\sin^2 t\right) + z_r \left(-\sin t\cos t\right) - z_{tt}\left(\frac{\sin t\cos t}{r}\right),\\
  \left(z_r r_x + z_t t_x\right)x_{rt}
  &=z_r (r_x x_{rt}) + z_t (t_x x_{rt})
  =z_r \left(-\frac{\sin t}{\cos t}\right) + z_t \left(\frac{1}{r}\right),\\
  \left(z_r r_y + z_t t_y\right)y_{rt}
  &=z_r \left(r_y y_{rt}\right) + z_t \left(t_y y_{rt}\right)
  =z_r \left(\frac{\cos t}{\sin t}\right) + z_t \left(\frac{1}{r}\right).
\end{align*}
Now we can add the two halves individually:
\begin{align*}
  \left(\frac{z_{rt} y_t + z_r y_{tt} - z_{tt}y_r}{r}\right)x_r
  &+\left(\frac{z_{rt}x_t + z_r x_{tt} - z_{tt}x_r}{r}\right)y_r\\
  &=z_{rt}\left(\cos^2 t - \sin^2 t\right) - z_r(2\sin t\cos t)-z_{tt}\left(\frac{2\sin t\cos t}{r}\right),\\
  \left(z_r r_x + z_t t_x\right)x_{rt}
  +\left(z_r r_y + z_t t_y\right)y_{rt} 
  &=z_r\left(\frac{\cos^2 t - \sin^2 t}{\sin t \cos t}\right) - z_t\left(\frac{2}{r}\right).
\end{align*}
Therefore
\begin{align*}
  z_{rt} = z_r\left(\frac{\cos^2 t - \sin^2 t}{\sin t \cos t}-2\sin t\cos t\right)+ z_{rt}\left(\cos^2 t - \sin^2 t\right)-z_t\left(\frac{2}{r}\right)
  -z_{tt}\left(\frac{2\sin t\cos t}{r}\right).
\end{align*}
Equivalently
\begin{align*}
  z_{rt}(2\sin^2 t) &= z_r\left(\frac{\cos^2 t - \sin^2 t}{\sin t \cos t}-2\sin t\cos t\right)-z_t\left(\frac{2}{r}\right)
  -z_{tt}\left(\frac{2\sin t\cos t}{r}\right)\\
  \Leftrightarrow z_{rt}&=z_r\left(\frac{\cos^2 t - \sin^2 t}{\sin^3 t \cos t}\right)
  -z_t\left(\frac{1}{r\sin^2 t}\right)-z_{tt}\left(\frac{\cos t}{r\sin t}\right).
\end{align*}
To insert this back into equation \ref{eqztt} we need to evaluate
\begin{align*}
  z_{rt}\left(\frac{2x_t y_t}{r}\right)
  &=z_{rt}\left(-2r\sin t\cos t\right)
  =z_r\left(\frac{2r(\sin^2 t - \cos^2 t)}{\sin^2 t}\right)+z_t\left(\frac{2\cos t}{\sin t}\right)+z_{tt}\left(2\cos^2 t\right).
\end{align*}
Further, we need to evaluate the other summands of equation \ref{eqztt} in terms of $\sin$ and $\cos$:
\begin{align*}
  \frac{y_{tt}x_t+x_{tt}y_t}{r}+r_x x_{tt}+r_y y_{tt}&=r(\sin^2 t - \cos^2 t - 2),\\
  t_x x_{tt} + t_y y_{tt}&=\frac{\cos^2 t-\sin^2 t}{\sin t\cos t},\\
  \frac{x_r+y_r}{r}&=\frac{\cos t + \sin t}{r}.
\end{align*}
Inserting into equation \ref{eqztt} gives
\begin{align*}
  z_{tt} &= z_r\left(\frac{2r(\sin^2 t - \cos^2 t)}{\sin^2 t}\right)
  +z_t\left(\frac{2\cos t}{\sin t}\right)+z_{tt}\left(2\cos^2 t\right)\\
  &+z_r\left(r(\sin^2 t - \cos^2 t - 2)\right)+z_t\left(\frac{\cos^2 t-\sin^2 t}{\sin t\cos t}\right)-z_{tt}\left(\frac{\cos t + \sin t}{r}\right) \\
  &=z_r\left(r\left(\sin^2 t - \cos^2 t - \frac{2\cos^2 t}{\sin^2 t}\right)\right)
  +z_t\left(\frac{3\cos^2 t-\sin^2 t}{\sin t\cos t}\right)
  + z_{tt}\left(\frac{2r\cos^2 t - \cos t + \sin t}{r}\right).
\end{align*}
This allows us to write $z_t$ in terms of $z_r$ and $z_{tt}$:
\begin{align*}
  z_t = z_{tt}\left(\frac{(2r\cos^2 t - \cos t + \sin t)\sin t\cos t}{r(3\cos^2 t - \sin^2 t)}\right)
  + z_r \left(r\left(\frac{\sin^2 t\cos t(2\cos^2 -1)+2\cos^2 t}{\sin(3\cos^2 t - \sin^2 t)}\right)\right)
\end{align*}
To insert this back into equation \ref{eq2z} we need to evaluate
\begin{align*}
  \frac{z_t}{r^2\sin t\cos t}
  =z_{tt}\left(\frac{(2r\cos^2 t - \cos t + \sin t)}{r^3(3\cos^2 t - \sin^2 t)}\right)
  +z_{r}\left(\frac{2\sin^2 t\cos^2 t+2\cos t-\sin^2 t}{r\sin^2 t\left(3\cos^2 t - \sin^2 t\right)}\right)
\end{align*}
and then
\begin{align*}
  \frac{z_t}{r^2\sin t\cos t} - \frac{z_{tt}}{r^2}=
  z_{tt}\left(\frac{1}{r^2}\left(\frac{2r\cos^2 t - \cos t + \sin t}{r(3\cos^2 t - \sin^2 t)}-1\right)\right)
  +z_{r}\left(\frac{2\sin^2 t\cos^2 t+2\cos t-\sin^2 t}{r\sin^2 t\left(3\cos^2 t - \sin^2 t\right)}\right)
\end{align*}
and finally
\begin{align*}
  z_{xx}+z_{yy} &= z_{rr}+\frac{z_t}{r^2\sin t\cos t} - \frac{z_{tt}}{r^2}+ z_r\left(\frac{\sin^2 t - \cos^2 t}{r}\right)\\
  &=z_{rr}+z_{tt}\left(\frac{1}{r^2}\left(\frac{2r\cos^2 t - \cos t + \sin t}{r(3\cos^2 t - \sin^2 t)}-1\right)\right)\\
  &+z_r\left(\frac{2\sin^2 t\cos^2 t+2\cos t-\sin^2 t}{r\sin^2 t\left(3\cos^2 t - \sin^2 t\right)}+\frac{\sin^2 t - \cos^2 t}{2} \right)
\end{align*}
You can see that now we got rid of the summand containing $z_t$ but now both $a$ and $b$ depend on $t$.
\pagebreak
\section*{Question 2}
Let $f(x,y)=x^2+y^2-2(x+y)$. Then we find its first order partial derivatives
\begin{align*}
  \frac{\p f}{\p x}=2x-2 \text{\hspace{0.5cm}and\hspace{0.5cm}}
  \frac{\p f}{\p y}=2y-2
\end{align*}
as well as its second order partial derivatives
\begin{align*}
  \frac{\p^2 f}{\p x^2}=2\text{\hspace{0.5cm}and\hspace{0.5cm}}
  \frac{\p^2 f}{\p y^2}=2\text{\hspace{0.5cm}and\hspace{0.5cm}}
  \frac{\p^2 f}{\p y\p x}=0.
\end{align*}
We can see that
\begin{align*}
  \frac{\p f}{\p x} = \frac{\p f}{\p y} = 0
\end{align*}
if, and only if, $x=y=1$. Thus the only critical point is $(1,1)$.\\
Further, observe that
\begin{align*}
  \frac{\p^2 f}{\p x^2}=2>0
\end{align*}
and
\begin{align*}
  D = \det \begin{pmatrix}
    2 & 0\\
    0 & 2
  \end{pmatrix}
  = 4 > 0.
\end{align*}
Applying the test from \emph{Section 2.7.1} of the notes, we find 
that the function $f$ has a local minimum at $(1,1)$.
\end{document}
