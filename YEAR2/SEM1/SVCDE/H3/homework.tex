\documentclass{article}
\usepackage[a4paper]{geometry}
\usepackage[british]{babel}
\usepackage{amsmath}
\usepackage{amssymb}
\usepackage{mathtools}
\usepackage{nicefrac}
\usepackage{amsthm}
\usepackage{changepage}
\geometry{tmargin=2cm, bmargin=3cm}
\DeclarePairedDelimiter{\floor}{\lfloor}{\rfloor}
\title{SVCDE: Hand-in 3}
\author{Franz Miltz (UNN: S1971811)}
\newcommand*\lneg[1]{\overline{#1}}
\newcommand{\R}{\mathbb{R}}
\newcommand{\N}{\mathbb{N}}
\newcommand{\Z}{\mathbb{Z}}
\newcommand{\C}{\mathbb{C}}
\newcommand{\st}{\text{ s.t. }}
\newcommand{\p}{\partial }
%newenvironment{claim}[1]{\noindent\emph{Claim.}\space#1}{}
\newenvironment{claimproof}[1]{\par\noindent\emph{Proof.}\space#1}{\hfill $\blacksquare$}
\newtheorem{claim}[section]{Claim}
\newtheorem{lemma}{Lemma}[section]
\DeclareMathOperator{\lub}{\text{LUB}}
\DeclareMathOperator{\hcf}{hcf}
\DeclareMathOperator{\lcm}{lcm}
\setcounter{MaxMatrixCols}{20}
\newcommand*\binco[2]{\begin{pmatrix}
  #1\\#2
\end{pmatrix}}
\newcommand{\ih}{\widehat i}
\newcommand{\jh}{\widehat j}
\newcommand{\kh}{\widehat k}
\newcommand{\K}{\mathcal{K}}
\newcommand{\dv}[1]{\vec #1\, '}
\renewcommand{\d}[1]{#1'}
\newenvironment{itpars}{\par\itshape}{\par}
\newenvironment{itquote}{\begin{quote}\itshape}{\end{quote}\ignorespacesafterend}
\begin{document}
\date{11th October 2020}
\maketitle
\section*{Problem 1}
We have
\begin{align*}
  z &= f(x,y),\\
  x &= r\cos t,\\
  y &= r\sin t.
\end{align*}
We want to show that
\begin{align*}
  \frac{\p^2 z}{\p x^2}+\frac{\p^2 z}{\p y^2}
  = \frac{\p^2 z}{\p r^2}+a(r)\frac{\p z}{\p r}+b(r)\frac{\p^2 z}{\p t^2},
\end{align*}
or equivalently
\begin{align}
  \label{eq1}
  z_{xx} + z_{yy} = z_{rr} + a(r)z_r + b(r)z_{tt}.
\end{align}
Note the following derivatives of $x$ and $y$:
\begin{align*}
  \begin{array}{l l}
  x_r = \cos t &\hspace{1cm} y_r = \sin t\\
  x_t = -r\sin t &\hspace{1cm} y_t = r\cos t\\
  x_{tt} = -r\cos t &\hspace{1cm} y_{tt} = -r\sin t
  \end{array}
\end{align*}
We start by determining $z_r$ and $z_t$:
\begin{align*}
  z_r &= z_x x_r + z_y y_r = z_x \cos t + z_y \sin t,\\
  z_t &= z_x x_t + z_y y_t = z_x (-r\sin t) + z_y r\cos t
\end{align*}
This follows directly from the \emph{Chain Rule}. 
Let us now find $z_{rr}$ and $z_{tt}$. We know
\begin{align*}
    z_{rr} &= (z_x x_r + z_y y_r)_r = (z_x x_r)_r + (z_y y_r)_r
\end{align*}
and since $x_r = \cos t$ and $y_r = \cos t$ do not depend on $r$
\begin{align*}
  z_{rr} &= (z_{xr})\cos t + (z_y)_r \sin t \\
  &=(z_{xx}x_r + z_{xy}y_r)\cos t + (z_{xy}x_r + z_{yy}y_r)\sin t\\
  &=z_{xx}\cos^2 t + 2 z_{xy}\sin t\cos t + z_{yy}\sin^2 t.
\end{align*}
Note that \emph{Clairaut's Theorem} is being used a lot and without special notice in this hand-in. Further
\begin{align*}
  z_{tt} &= (z_x x_t + z_y y_t)_t = (z_x x_t)_t + (z_y y_t)_t\\
  &=(z_{xx}x_t + z_{xy}y_t)x_t + z_x x_{tt}
  + (z_{xy}x_t + z_{yy}y_t)y_t + z_y y_{tt}\\
  &= z_{xx}r^2\sin^2 t - 2z_{xy}r^2\sin t\cos t + z_{yy}r^2\cos^2 t -z_x r\cos t - z_y r\sin t.
\end{align*}
To show equation \ref{eq1} we need to evaluate the right-hand side of the equation by inserting
$z_{rr}$, $z_r$ and $z_{tt}$. We get
\begin{align*}
  &z_{rr} + a(t)z_r + b(t)z_{tt}\\
  &=z_{xx}\cos^2 t + 2 z_{xy}\sin t\cos t + z_{yy}\sin^2 t\\
  &+a(r)\left(z_x \cos t + z_y \sin t\right)\\
  &+b(r)\left(z_{xx}r^2\sin^2 t - 2z_{xy}r^2\sin t\cos t + z_{yy}r^2\cos^2 t-z_x r\cos t - z_y r\sin t\right).
\end{align*}
At this point we observe that, if $b(r)=1/r^2$, both $z_{xx}$ and $z_{yy}$ will
be without any factor. We therefore try it and obtain
\begin{align*}
  z_{rr} + a(t)z_r + \frac{1}{r^2}z_{tt} &= z_{xx}+z_{yy} + a(r)(z_x\cos t + z_y\sin t)
  -z_x\frac{\cos t}{r}-z_y \frac{\sin t}{r}\\
  &= z_{xx}+z_{yy} + \left(a(r)-\frac{1}{r}\right)(z_x\cos t+z_y \sin t).
\end{align*}
Now we see that $a(r)=1/r$ is required to get
\begin{align*}
  z_{rr} + \frac{1}{r}z_r + \frac{1}{r^2}z_r = z_{xx} + z_{yy}
\end{align*}
as desired.
\section*{Question 2}
Let $f(x,y)=x^2+y^2-2(x+y)$. Then we find its first order partial derivatives
\begin{align*}
  \frac{\p f}{\p x}=2x-2 \text{\hspace{0.5cm}and\hspace{0.5cm}}
  \frac{\p f}{\p y}=2y-2
\end{align*}
as well as its second order partial derivatives
\begin{align*}
  \frac{\p^2 f}{\p x^2}=2\text{\hspace{0.5cm}and\hspace{0.5cm}}
  \frac{\p^2 f}{\p y^2}=2\text{\hspace{0.5cm}and\hspace{0.5cm}}
  \frac{\p^2 f}{\p y\p x}=0.
\end{align*}
We can see that
\begin{align*}
  \frac{\p f}{\p x} = \frac{\p f}{\p y} = 0
\end{align*}
if, and only if, $x=y=1$. Thus the only critical point is $(1,1)$.\\
Further, observe that
\begin{align*}
  \frac{\p^2 f}{\p x^2}=2>0
\end{align*}
and
\begin{align*}
  D(1,1) = \det \begin{pmatrix}
    2 & 0\\
    0 & 2
  \end{pmatrix}
  = 4 > 0.
\end{align*}
Applying the test from \emph{Section 2.7.1} of the notes, we find 
that the function $f$ has a local minimum at $(1,1)$.
\end{document}
