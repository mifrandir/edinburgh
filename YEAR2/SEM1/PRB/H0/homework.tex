\documentclass{article}
\usepackage[a4paper]{geometry}
\usepackage[british]{babel}
\usepackage{amsmath}
\usepackage{amssymb}
\usepackage{mathtools}
\usepackage{nicefrac}
\usepackage{amsthm}
\usepackage{changepage}
\geometry{tmargin=2cm, bmargin=3cm}
\DeclarePairedDelimiter{\floor}{\lfloor}{\rfloor}
\title{PRB: Workshop 1 Hand-in}
\author{Franz Miltz (UNN: S1971811)}
\newcommand*\lneg[1]{\overline{#1}}
\newcommand{\R}{\mathbb{R}}
\newcommand{\N}{\mathbb{N}}
\newcommand{\Z}{\mathbb{Z}}
\newcommand{\C}{\mathbb{C}}
\newcommand{\st}{\text{ s.t. }}
\renewcommand{\P}{\mathbb{P}}
%newenvironment{claim}[1]{\noindent\emph{Claim.}\space#1}{}
\newenvironment{claimproof}[1]{\par\noindent\emph{Proof.}\space#1}{\hfill $\blacksquare$}
\newtheorem{claim}[section]{Claim}
\newtheorem{lemma}{Lemma}[section]
\DeclareMathOperator{\lub}{\text{LUB}}
\DeclareMathOperator{\hcf}{hcf}
\DeclareMathOperator{\lcm}{lcm}
\setcounter{MaxMatrixCols}{20}
\newcommand*\binco[2]{\begin{pmatrix}
  #1\\#2
\end{pmatrix}}
\newcommand{\ih}{\widehat i}
\newcommand{\jh}{\widehat j}
\newcommand{\kh}{\widehat k}
\newcommand{\K}{\mathcal{K}}
\newcommand{\dv}[1]{\vec #1\, '}
\renewcommand{\d}[1]{#1'}
\begin{document}
\date{29th September 2020}
\maketitle
\noindent Let the experiment $E$ be to pick a random number in the sample space
$S=[1,600]$. Then, let us define the following events:
\begin{align*}
  A:\:&\text{The outcome is divisible by 2.}\\
  B:\:&\text{The outcome is divisible by 3.}\\
  C:\:&\text{The outcome is divisible by 5.}
\end{align*}
This gives us the following probabilities:
\begin{align*}
  \P(A)=\frac{300}{600}=\frac{1}{2},\\
  \P(B)=\frac{200}{600}=\frac{1}{3},\\
  \P(C)=\frac{120}{600}=\frac{1}{5}.
\end{align*}
Observe that the events
\begin{align*}
  A\cap B:\:&\text{The outcome is divisible by 6.}\\
  A\cap C:\:&\text{The outcome is divisible by 10.}\\
  B\cap C:\:&\text{The outcome is divisible by 15.}
\end{align*}
have the probabilities
\begin{align*}
  \P(A\cap B)&=\frac{100}{600}=\frac{1}{6}=\P(A)\P(B),\\
  \P(A\cap C)&=\frac{60}{600}=\frac{1}{10}=\P(A)\P(C),\\
  \P(A\cap B)&=\frac{40}{600}=\frac{1}{15}=\P(A)\P(B).
\end{align*}
Therefore all of these events are pairwise independent. Therefore the
probability we are looking for which corresponds to the event
\begin{align*}
  A\cap B^c \cap C:\:\text{The outcome is divisible by 2 and 5 but not 3.}
\end{align*}
is given by
\begin{align*}
  \P(A\cap B^c\cap C)= \P(A)\P(B^c)\P(c) = \frac{1}{2}\left(1-\frac{1}{3}\right)\frac{1}{5}=\frac{1}{15}.
\end{align*}
\end{document}