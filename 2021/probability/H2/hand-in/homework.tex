\documentclass{article}
\usepackage{homework-preamble}

\begin{document}
\title{Probability: Week 6 Hand-in}
\author{Franz Miltz (UUN: S1971811)}
\date{27th October, 2020}
\maketitle


\section*{P 4.3}


Let $X$ be as described. Then for the event $X=k$ we know that
the $k$th roll has to be a six. Further, for the $k-1$ rolls before
that, there need to be exactly $r-1$ sixes. If and only if both of these
conditions are met, $X=k$. Therefore, we can write
\begin{align*}
	\P(X=k) = \frac{1}{6}\P(Y=r-1)
\end{align*}
where $Y$ is the number of sixes in $k-1$ rolls, i.e. $Y\sim\Binom(k-1,1/6)$. Using the binomial
distribution, we find
\begin{align*}
	\P(X=k) = \frac{1}{6}\binom{k-1}{r-1}\left(\frac{1}{6}\right)^ {r-1}
	\left(\frac{5}{6}\right)^{k-r}
	= \frac{(k-1)!}{(r-1)!(k-r)!}\left(\frac{1}{6}\right)^r\left(\frac{5}{6}\right)^{k-r}
\end{align*}
as required. \\
We can write $X=X_1+X_2+\cdots X_r$ with $X_i\sim \Geom(r)$ for all $i$.
Using the \emph{Law of Total Probability for Expected Values}, we find
\begin{align*}
	\E(X) = \E(X_1) + \E(X_2) + \cdots + \E(X_r) = r\E(X_1).
\end{align*}
Using \emph{Proposition 4.1.3} from the nodes, we find $\E(X_1) = 1/p$ and thus
\begin{align*}
	\E(X) = r/p.
\end{align*}


\section*{P 5.1}


We know that, for our random variable $W$, the pdf is
\begin{align*}
	f_W(x) = 6x(1-x).
\end{align*}
Using
\begin{align*}
	\E(W) = \int_0^1 xf_W(x) dx,
\end{align*}
we find
\begin{align*}
	\E(W)=\int_0^1 6x^2-6x^3 dx = \left[2x^3 - \frac{3}{2}x^4\right]^1_0 = \frac{1}{2}.
\end{align*}
Further, we know that
\begin{align*}
	\V(W) = \E(W^2) - (\E(W))^2.
\end{align*}
Since we have determined $\E(W)=1/2$, we only need to find $\E(W^2)$. We know
\begin{align*}
	\E(W^2) = \int_0^1 x^2 f_{W}(x)dx = \int_0^1 6x^3(1-x)dx.
\end{align*}
Thus
\begin{align*}
	\E(W^2) = \int_0^1 6x^3-6x^4 dx = \left[\frac{3}{2} x^4 - \frac{6}{5}x^5\right]_0^1=\frac{3}{10}.
\end{align*}
Now we know
\begin{align*}
	\V(W) = \frac{3}{10}-\frac{1}{4} = \frac{1}{20}.
\end{align*}
It makes sense that $\E(W) = \E(X) = \E(Y) = \E(Z)$ if we define $W$ to be the middle one of $X$, $Y$ and $Z$ because
it means that $W$ is most likely to be the closest to the middle of the intervall and thus the expected value.\\

\emph{Note: I could not use the Beta distribution as I don't have
	a clue how it works or what its parameters $\alpha$ and $\beta$ are.
	As far as I can tell it hasn't been explained in the notes so I don't
	think it's required to do this.}


\section*{P 5.4}


We have a random variable $X_n\sim\Binom(n,1/2)$. We define
$A_n = X_n / n$ to be the average of all the $n$ coin flips.
We know
\begin{align*}
	\E(A_n) = \frac{1}{2} \text{ and }
	\V(A_n) = \V\left(\frac{1}{n}X_n\right) = \frac{npq}{n^2} = \frac{1}{4n}.
\end{align*}
Using \emph{Chebyshev's inequality} with $a=k\sigma$, we know
\begin{align*}
	\P(|A_n - \mu| \leq k\sigma) \geq 1 - \frac{1}{k^2}.
\end{align*}
We choose $k=5$ to get
\begin{align*}
	\P(|A_n - \mu| \leq 5\sigma) \geq 0.96
\end{align*}
which is greater than $0.95$ and thus suitable for our purpose.
We want $a=1/100$ and thus
\begin{align*}
	5\sigma                       & = \frac{1}{100}  \\
	\Leftrightarrow \frac{25}{4n} & = \frac{1}{10^4} \\
	\Leftrightarrow 4n            & = 25\cdot 10^4   \\
	\Leftrightarrow n             & = 62500.
\end{align*}
Note that this is not the lowest $n$ can be, but rather a conservative
estimate that is guranteed to work.
\end{document}