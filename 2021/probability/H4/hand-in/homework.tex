\documentclass{article}
\usepackage{homework-preamble}
\usepackage{slashbox}

\begin{document}
\title{Probability: Week 10 Hand-in}
\author{Franz Miltz (UUN: S1971811)}
\date{24 November, 2020}
\maketitle

\section*{P8.2}

Let $X_n$ be the position after $n$ moves. Then we know from \emph{P7.2}
that
\begin{align*}
	X_n = 2B_n-n
\end{align*}
where $B_n\sim\Binom(n,1/2)$. By finding the appropriate interval $I$
for $B_{1000}$ such that
\begin{align*}
	B_{1000}\in I\Rightarrow X_{1000}\in [-20, 20]
\end{align*}
we can use this to simplify:
\begin{align*}
	\P(-20\leq X_{1000} \leq 20) = \P(490 \leq B_{1000} \leq 510)
\end{align*}
where we used $I=[490, 510]$ which may be found by inspection.\\
With $\E(B_{1000})=500$ and $\V(B_{1000})=250$, we can approximate this value
by utilising $\Phi$:
\begin{align*}
	\P(-20\leq X_{1000} \leq 20) = \Phi\left(\frac{510.5-500}{\sqrt{250}}\right)
	- \Phi\left(\frac{489.5-500}{\sqrt{250}}\right)
	= 2\Phi\left(\frac{10.5}{\sqrt{250}}\right)-1\approx 0.493.
\end{align*}

\section*{P9.1}

We find the following table:
\begin{center}
	\begin{tabular}{| c | c c c c c | c |}
		\hline
		\backslashbox{S}{D} & -2  & -1  & 0   & 1   & 2   &     \\
		\hline
		2                   & 0   & 0   & 1/9 & 0   & 0   & 1/9 \\
		3                   & 0   & 1/9 & 0   & 1/9 & 0   & 2/9 \\
		4                   & 1/9 & 0   & 1/9 & 0   & 1/9 & 3/9 \\
		5                   & 0   & 1/9 & 0   & 1/9 & 0   & 2/9 \\
		6                   & 0   & 0   & 1/9 & 0   & 0   & 1/9 \\
		\hline
		                    & 1/9 & 2/9 & 3/9 & 2/9 & 1/9 & 1   \\
		\hline
	\end{tabular}
\end{center}
For example, we observe that $D=-2$ can only ever occur when $(X,Y)=(1,3)$.
Thus $D=-2\Rightarrow S=4$. Therefore the we find the probabilities of the first column:
\begin{align*}
	\P(D=-2 \text{ and } S=4) = 1/9 \text{ and }\forall k \not=4,\: \P(D=-2 \text{ and } S = k) = 0 .
\end{align*}
We get the value $1/9$ because $X$ and $Y$ can take $3$ values each,
thus there are $9$ values the pair $(X,Y)$
can take and there is only one such pair that leads to $D=-2$.\\
You can see that $\E(S)=4$ and $\E(D)=0$. We know that
\begin{align*}
	\E(S\:D) = \sum_{a\in [2,6]} \sum_{b\in[-2,2]} ab\P(S=a\text{ and }D=b).
\end{align*}
Using the symmetry of the table and the fact that all the summands
where $b=0$ will end up being $0$, we find
\begin{align*}
	\E(S\:D)=0.
\end{align*}
Two random variables $X$ and $Y$ are, by definition, independent
if and only if, for all $x,y$,
\begin{align*}
	F_{X,Y}(x,y)=F_X(x)F_Y(y).
\end{align*}
Since
\begin{align*}
	F_{D,S}(2,-2) = 0 \not= F_D(2)F_S(-2) = 1/81,
\end{align*}
we know that $D$ and $S$ are not independent.

\section*{P9.2}

Let the unit rod break in two places $X_1$ and $X_2$. Then there
exist the following conditions that imply that at least one of the pieces is shorter
than $a$:
\begin{enumerate}
	\item $A_1:X_1<a$,
	\item $A_2:X_1>1-a$,
	\item $B_1:X_2<a$,
	\item $B_2:X_2>1-a$,
	\item $C:|X_1-X_2|<a$.
\end{enumerate}
The cases 1 to 4 are the ones where at least one of the outer
pieces is shorter than $a$. The fifth cases is where the middle section
is shorter than $a$. The probabilities of these events are
\begin{align*}
	\P(A) & = \P(A_1) + \P(A_2) = 2a,                  \\
	\P(B) & = \P(B_1) + \P(B_2) = 2a,                  \\
	\P(C) & = \P(X_1-a\leq X_2 \leq X_1+a) \approx 2a.
\end{align*}
Note that the probability for $C$ is only $2a$ if $X_2$ is not too close to
either edge of the interval. Since $a\to 0$, we assume this is not the case.
Also, that would imply $B$ which is a case that has already been included.
Since the events $A,B,C$ are not pairwise mututally exclusive,
we have to use the \emph{Inclusion-exclusion principle}
to determine the total probability:
\begin{align*}
	\P(A\cup B\cup C) = \P(A) + \P(B) + \P(C) - \P(A \cap B) - \P(B\cap C) - \P(C\cap A) + \P(A\cap B\cap C).
\end{align*}
Notice that the probabilities will be, in terms of orders of magnitude,
similar to $a$ to the power of the number of events involved. For example
\begin{align*}
	\log_{10}(\P(A\cap B)) = \log_{10}(4a^2) \approx \log_{10}(a^2).
\end{align*}
The argument is more complicated for the intersections involving $C$ but
it works in a similar way.\\
Since $a\to 0$, we can assume that $\forall k>1,\:a\gg a^k$ and thus
approximate the probability by omiting all the intersection:
\begin{align*}
	\P(A\cup B\cup C) \approx \P(A) + \P(B) + \P(C) \approx 6a.
\end{align*}
Therefore the required constant is $k=6$.


\end{document}