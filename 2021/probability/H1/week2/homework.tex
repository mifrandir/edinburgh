\documentclass{article}
\usepackage{homework-preamble}

\begin{document}
\title{Probability: Week 2 Solutions}
\author{Franz Miltz (UUN: S1971811)}
\date{6th October 2020}
\maketitle


\section*{P 2.1}


Let $X$ be the number of times that a run of 4 consecutive tosses
contains an equal number of heads and tails when tossing a coin 99 times.
Observe that
\begin{align*}
	X=X_1 + X_2 + \cdots X_{96}
\end{align*}
where each $X_i$ is a Bernoulli random variable that is $1$ if,
and only if, there are exactly two heads and two tails in the tosses
$[i, i+3]$. Furthermore, we can define $Y_i$ as the number of heads
in each of these intervalls. Since $Y_i\sim \text{Binom}(4,0.5)$,
we get
\begin{align*}
	\E(X_i) = \P(Y=2) = \binom{4}{2}\left(\frac{1}{2}\right)^2\left(\frac{1}{2}\right)^2=\frac{3}{8}.
\end{align*}
Since all $X_i$ are discrete variables, we can apply \emph{Theorem 2.5.6}
to get
\begin{align*}
	\E(X) = \E(X_1) + \E(X_2) + \cdots \E(X_{96}).
\end{align*}
Using the fact that all $\E(X_i)$ are equal, we can simplify this to get
\begin{align*}
	\E(X) = 96\cdot\E(X_1) = \frac{96\cdot3}{8}=36.
\end{align*}


\section*{P 2.2}


Using the binomial distribution we get
\begin{align*}
	p_n = \binom{2n}{n}\left(\frac{1}{2}\right)^n\left(\frac{1}{2}\right)^n
	=\binom{2n}{n}\frac{1}{2^{2n}}.
\end{align*}
We can rewrite this by inserting the definition of the binomial
coefficient:
\begin{align*}
	p_n = \frac{(2n)!}{(n!)^2}\cdot\frac{1}{2^{2n}}=\frac{(2n)!}{(n!)^2\cdot 2^{2n}}.
\end{align*}
To approximate $p_n$ for large $n$, we use insert \emph{Stirlings formula}
to get
\begin{align*}
	p_n \approx \frac{
		\frac{(2n)^{2n}}{e^{2n}}\sqrt{4\pi n}
	}{
		\left(\frac{n^n}{e^n}\sqrt{2\pi n}\right)^2\cdot 2^{2n}}.
\end{align*}
Fortunately, we can simplify this to get
\begin{align*}
	p_n\approx \frac{
		\frac{2^{2n}\cdot n^{2n}}{e^{2n}}\cdot2\sqrt{\pi n}
	}{
		\frac{n^{2n}}{e^{2n}}\cdot 2\pi n\cdot 2^{2n}}=\frac{1}{\sqrt{\pi n}}.
\end{align*}
Rewriting this into the desired form, we get
\begin{align*}
	p_n\approx C / n^\alpha = \pi^{-\nicefrac{1}{2}}/n^{\nicefrac{1}{2}}.
\end{align*}
So $C=\pi^{-\nicefrac{1}{2}}$ and $\alpha=\nicefrac{1}{2}$.
\section*{P 2.3}
Let $X$ be the number of points $A$ scores when playing five more points.
For the probability of A winning the game in question
to be $\nicefrac{1}{2}$, we equivalently require
\begin{align*}
	\P(X\geq 4) = \nicefrac{1}{2}.
\end{align*}
We can split this up to get
\begin{align*}
	\P(X = 5) + \P(X = 4) = \nicefrac{1}{2}.
\end{align*}
Since $X\sim\text{Binom(5,p)}$, we get
\begin{align*}
	\binom{5}{5}p^5q^0+\binom{5}{4}p^4q^1=\nicefrac{1}{2},
\end{align*}
which may be simplified to
\begin{align*}
	p^5+5p^4(1-p)             & = \nicefrac{1}{2} \\
	\Leftrightarrow 5p^4-4p^5 & =\nicefrac{1}{2}
\end{align*}
Finding an exact solution to this polynomial is unrealistic.
Therefore we can use Wolfram Alpha to compute three possible real solutions:
\begin{align*}
	p_n \in \{-0.516..., 0.686..., 1.187...\}.
\end{align*}
Since we also know $p\in[0,1]$ simply because we are dealing with a
probability, we get
\begin{align*}
	p\in \{-0.516..., 0.686..., 1.187...\} \cap [0,1] \Rightarrow p \approx 0.686=68.6\%.
\end{align*}


\section*{P 2.4}


Let there be two players: Alice and Bob. Let Alice be the one to pick
a coin and let Bob be the one to guess. Then we define the following
events:
\begin{align*}
	A: & \:\text{"Alice picks a 10 pence coin.", and}  \\
	B: & \:\text{"Bob guesses it's a 20 pence coin."}.
\end{align*}
Therefore $\P(A)=p$ and $\P(B)=s$.
Now, let $X$ be Bobs winnings. Then we find
\begin{align*}
	\P(X=10) & = \P(A\cap B^c) = \P(A)\P(B^c) = p(1-s),\text{ and} \\
	\P(X=20) & = \P(A^c\cap B) = \P(A^c)\P(B) = (1-p)s.
\end{align*}
Thus the expected value of $X$ is
\begin{align*}
	\E(X) = 10p(1-s)+20(1-p)s = 10(p - 3ps + 2s).
\end{align*}
Note that we can omit the case where $X=0$ because it does
not contribute anything to the weighted sum.

\subsection*{Challenge}

Let $f=\E(X)$ be a function of the independent variables $p$ and $s$.
Then we get
\begin{align*}
	f_p(p,s)=\frac{\partial f(p,s)}{\partial p} = 10(1-3s)
\end{align*}
and
\begin{align*}
	f_s(p,s)=\frac{\partial f(p,s)}{\partial s} = 10(2-3p).
\end{align*}
This lets us observe some interesting properties:
\begin{enumerate}
	\item If $s=\nicefrac{1}{3}$, $f_p$ is $0$ which means the choice of
	      $p$ does not influence the value of $\E(X)$.
	\item Similarly, if $p=\nicefrac{2}{3}$, $f_s$ is $0$.
\end{enumerate}
Therefore, for all $p$ and $s$, the extreme values lie on the boundaries
of the intervall $[0,1]$. Thus we can classify the choices of $(p,s)$ with
respect to $(\nicefrac{2}{3}, \nicefrac{1}{3}$). Since the extreme values
in the quadrants around that point are at the corners of the range of
$f$ and we assume that both players are trying to maximise their
winnings/minimise their losses it only makes sense to choose either
$0$ or $1$ if they think their opponent is more likely to pick one of the
sides (with respect to $\nicefrac{1}{3}$ or $\nicefrac{2}{3}$) for their
probability. Let us therefore investigate the following values of $f$:
\begin{align*}
	f(0,0) & =0,  \\
	f(0,1) & =20, \\
	f(1,0) & =10, \\
	f(1,1) & =0.
\end{align*}
This shows that picking $p$ and $s$ is the same as picking either
coin from both players point of view. Assuming that neither player
can predict what the other will choose, this leaves us with the following
suggestions, if we assume both players are aware of their sensible options.\\
Alice should not pick $p=0$ as this gives her a $50\%$ chance of losing
(again, assuming both $p<\nicefrac{2}{3}$ and $p>\nicefrac{2}{3}$
are equally likely values for $s$)
which is the same for $p=1$ but here she only puts 10 pence on the line.
We don't need to consider Bob picking $s=\nicefrac{1}{3}$ from Alices
point of view because Alice cannot influence $\E(X)$ in that case anyways.
This leaves the last option for Alice as $p=\nicefrac{2}{3}$.
By Choosing this, Alice can force $\E(X)=\nicefrac{20}{3}$.
We can see that this is worse than the expected value of the
expected value when choosing $p=1$ which is $10\cdot\nicefrac{1}{2}=5$.
Thus \emph{if Alice has no reason to believe that Bob is more likely
	to choose either $p\leq\nicefrac{2}{3}$ or $p\leq\nicefrac{2}{3}$
	she should pick $s=1$}.\\
Similarly, it does not make a lot of sense for Bob to choose $s=0$.
From his point of view, it is not sensible to force
$\E(X)=\nicefrac{20}{3}$ with $s=\frac{1}{3}$ because he has to
assume that the expected value of the expected value
after choosing $s=1$ is $20\cdot \frac{1}{2}=10$.\\
This line of thinking leads to the possibility that Alice \emph{and}
Bob are equally as likely to choose either $0$ or $1$. This lets us
estimate an expected value of the expected value as $7.5$. Interestingly
this is greater than what Alice could have forced the expected value to
be.
\end{document}