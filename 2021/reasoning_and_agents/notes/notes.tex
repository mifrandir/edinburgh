\documentclass{article}
\usepackage{notes-preamble}
\usepackage{enumitem}
\mkfpmthms

\title{INF2D: Reasoning and Agents (SEM4)}
\author{Franz Miltz}
\date{10 May 2021}
\begin{document}
\maketitle
\noindent If you have questions, or want to contribute, email me at \texttt{notes@miltz.me}.
\tableofcontents
\pagebreak

\section{Intelligent agents and their environments}

\begin{definition}[Agent, Rationality, Architecture; R\&N p. 37]
	An \emph{agent} is an entity that perceives and
	acts based on a sequence of observations in a system.
	An agent is called \emph{rational} if it acts to achieve
	the best possible outcome or the best possible expected
	outcome, if uncertainty is involved.

	The computing device an agent runs on combined with all sensors and actuators
	is called the agent's \emph{architecture}.
\end{definition}

\subsection{Types of environments}

\begin{definition}[R\&N p. 42]
	An environment is called
	\begin{itemize}
		\item \emph{fully observable} if the agent's sensors describe it fully.
		\item \emph{deterministic} if the next state is fully
		      determined by the current state and the agent's actions.
		\item \emph{episodic} if the next action does not depend on the previous
		      actions. Otherwise \emph{sequential}.
		\item \emph{static} if it does not change while the agent determines
		      next action.
		\item \emph{discrete} if the perecepts, actions and episodes within
		      are discrete.
	\end{itemize}
\end{definition}

\subsection{Types of intelligent agents}

\begin{definition}[R\&N p. 47]
	There are four types of agents:
	\begin{enumerate}
		\item \emph{simple reflex}: React only to the immediate percepts and ignore
		      the enitrety of their percept history.
		\item \emph{model-based reflex}: Actions may depend on the percept history
		      as well as unperceived aspects of the world. Require \emph{internal world
			      model}.
		\item \emph{goal-based}: Have variable goals depending on the state of their
		      environment.
		\item \emph{utility-based}: Prioritise between multiple, potentially conflicting
		      goals at the same time.
	\end{enumerate}
\end{definition}

\section{Problem solving and search}


\subsection{Types of problems}

\begin{definition}
	We define the following types of problems:
	\begin{itemize}
		\item \emph{single-state problem}: deterministic, fully-observable environment; solution is sequence
		\item \emph{sensorless problem}: non-observable environment; solution is sequence
		\item \emph{contingency problem}: non-deterministic and/or partially observable environment
		\item \emph{exploration problem}: state space is unknown
	\end{itemize}
\end{definition}

\subsection{Problem formulation}

\begin{definition}[Problem formulation, R\&N p. 66]
	A single-state problem in state space $S$
	and action space $A$ is defined by
	\begin{enumerate}
		\item \emph{initial state}
		\item \emph{successor function}, $f:S\to A\times S$
		\item \emph{goal test}, explicit or implicit
		\item additive \emph{path cost}, $c:S\times A\times S\to\R_{\geq 0}$
	\end{enumerate}
\end{definition}

\subsection{Selecting a state space}

Complexity of real world needs to be abstracted away.
\begin{itemize}
	\item \emph{(abstract) state}: set of real states
	\item \emph{(abstract) action}: complex combination of real actions
	\item \emph{(abstract) solution}: set of real paths, solutions in the real world
\end{itemize}

\subsection{Search algorithms}

\emph{Idea}: offline, simulated exploration of state space by generating successors of already
explored states.

\begin{pseudo}
	\textbf{function} TREE-SEARCH(problem) \textbf{returns} a solution or failure   \\+
	initialise the frontier using the initial state of the problem              \\
	\textbf{loop do}                                                            \\+
	\textbf{if} the frontier is empty \textbf{then}                          \\+
	\textbf{return} failure                                             \\-
	choose a leaf node and remove it from the frontier                      \\
	\textbf{if} the node contains a goal state \textbf{then}                \\+
	\textbf{return} the corresponding solution                          \\-
	expand the chosen node, adding the resulting nodes to the frontier
\end{pseudo}

\begin{pseudo}
	\textbf{function} GRAPH-SEARCH(problem) \textbf{returns} a solution or failure  \\+
	initialise the frontier using the initial state of the problem              \\
	\textbf{loop do}                                                            \\+
	\textbf{if} the frontier is empty \textbf{then}                          \\+
	\textbf{return} failure                                             \\-
	choose a leaf node and remove it from the frontier                      \\
	\textbf{if} the node contains a goal state \textbf{then}                \\+
	\textbf{return} the corresponding solution                          \\-
	\emph{add the node to the explored set}                                 \\
	expand the chosen node, adding the resulting nodes to the frontier      \\+
	\emph{\textbf{only if} not in the frontier or explored set}
\end{pseudo}



\subsection{Uninformed search strategies}

\begin{definition}
	A \emph{search strategy} defines the order of expansion in a tree search algorithm.
\end{definition}

\begin{definition}[R\&N pp. 81-91]
	There are five uninformed search strategies:
	\begin{enumerate}
		\item \emph{breadth-first search}: frontier is a FIFO queue
		\item \emph{uniform-cost search}: frontier is a priority queue with path cost
		\item \emph{depth-first search}: frontier is a LIFO queue
		\item \emph{depth-limited search}: frontier is a LIFO queue and nodes at depth $l$ have no successors
		\item \emph{iterative deepening search}: depth-limited search is repeated with increasing $l$
	\end{enumerate}
\end{definition}




\subsection{Informed search strategies}

\begin{definition}[R\&N p. 92]
	A \emph{heuristic} $h:S\to \R_{\geq 0}$ is a problem-specific function
	that evaluates every element of a search space $S$ to a nonnegative value.
	For all goal states $s_0\in S$ we have $h(s_0)=0$.
	A heuristic $h$ is called
	\begin{itemize}
		\item \emph{admissible} if for every node $n$, $h(n)\leq h^*(n)$,
		      where $h^*(n)$ is the true cost to reach the goal node from $n$.
		\item \emph{consistent} if for every node $n$, every
		      successor $n'$ of $n$ generated by any action $a$,
		      \begin{align*}
			      h(n) \leq c(n, a, n') + h(n').
		      \end{align*}
	\end{itemize}
	If $h_1,h_2:S\to \R_{\geq 0}$ are admissible heuristics for a state
	space $S$ and
	\begin{align*}
		\forall s\in S,\: h_1(s) \leq h_2(s).
	\end{align*}
	Then \emph{$h_2$ dominates $h_1$}.
\end{definition}

\begin{theorem}
	Let $h_1,h_2$ be admissible heuristics. If $h_2$ dominates $h_1$ it is
	better for tree and graph search.
\end{theorem}

\begin{definition}
	We consider the following informed search strategies:
	\begin{itemize}
		\item \emph{best-first}: next node to expand is chosen based on desirability assigned by a function $f$.
		\item \emph{greedy best-first}: $f=h$ where $h$ is a heuristic.
		\item \emph{A*}: $f=g+h$ where $h$ is a heuristic and $g$ is the cumulative cost of reaching a given node.
	\end{itemize}
\end{definition}

\begin{theorem}
	Let $P$ be a problem and let $P'$ be a relaxed problem of $P$, i.e.
	a problem with fewer restrictions. Then the cost of an optimal solution of
	$P'$ is a admissible heuristic for $P$.
\end{theorem}


\subsection{Adversarial search}


\begin{definition}[R\&N p. 161]
	\emph{Adversarial search} is required when multiple agents work to achieve
	opposing goals. Its purpose is to ensure the best possible outcome for the
	agent applying it, taking into account all possible actions by other agents.
\end{definition}

\begin{definition}[R\&N p. 165]
	\emph{Minimax} is an adversarial search strategy in deterministic, perfect
	information games, where the agent chooses the action that guarantees the
	highest lower bound for the utility value of the final state.

	\begin{pseudo}
		\textbf{function} MINIMAX-DECISION(\emph{state}) \textbf{returns} \emph{an action}\\+
		\textbf{return} $\argmax_{a\in\textsf{ACTIONS}(s)}$(MIN-VALUE(RESULT(\emph{state}, \emph{a})))\\-
		\\
		\textbf{function} MAX-VALUE(\emph{state}) \textbf{return} \emph{a utility value}\\+
		\textbf{if} TERMINAL-TEST(\emph{state}) \textbf{then}\\+
		\textbf{return} UTILITY(\emph{state})\\-
		$v\leftarrow-\infty$\\
		\textbf{for each} $a\in\textsf{ACTIONS(\emph{state})}$ \textbf{do}\\+
		$v\leftarrow \textsf{MAX}(v, \textsf{MIN-VALUE}(\textsf{RESULT}(s, a)))$\\-
		\textbf{return} $v$\\-
		\\
		\textbf{function} MIN-VALUE(\emph{state}) \textbf{return} \emph{a utility value}\\+
		\textbf{if} TERMINAL-TEST(\emph{state}) \textbf{then}\\+
		\textbf{return} UTILITY(\emph{state})\\-
		$v\leftarrow\infty$\\
		\textbf{for each} $a\in\textsf{ACTIONS(\emph{state})}$ \textbf{do}\\+
		$v\leftarrow \textsf{MIN}(v, \textsf{MAX-VALUE}(\textsf{RESULT}(s, a)))$\\-
		\textbf{return} $v$
	\end{pseudo}
\end{definition}

\begin{definition}[R\&N p. 167]
	\emph{\alpha-\beta-pruning} is a minimax-like adversarial search
	strategy, where upper and lower bounds are used to identify
	sub-trees that do not affect the solution.
	\begin{pseudo}
		\textbf{function} ALPHA-BETA-SEARCH(\emph{state}) \textbf{returns} \emph{an action}\\+
		$v\leftarrow$MAX-VALUE(\emph{state}, $-\infty$, $+\infty$)\\
		\textbf{return} the \emph{action} in ACTIONS(\emph{state}) with value $v$\\-
		\\
		\textbf{function} MAX-VALUE(\emph{state}, \alpha, \beta) \textbf{return} \emph{a utility value}\\+
		\textbf{if} TERMINAL-TEST(\emph{state}) \textbf{then}\\+
		\textbf{return} UTILITY(\emph{state})\\-
		$v\leftarrow-\infty$\\
		\textbf{for each} $a\in\textsf{ACTIONS(\emph{state})}$ \textbf{do}\\+
		$v\leftarrow \textsf{MAX}(v, \textsf{MIN-VALUE}(\textsf{RESULT}(\textsf{\emph{state}}, a), \alpha, \beta))$\\
		\textbf{if} $v\geq \beta$ \textbf{then}\\+
		\textbf{return} $v$\\-
		$\alpha\leftarrow\textsf{MAX}(\alpha, v)$\\-
		\textbf{return} $v$\\-
		\\
		\textbf{function} MIN-VALUE(\emph{state}, \alpha, \beta) \textbf{return} \emph{a utility value}\\+
		\textbf{if} TERMINAL-TEST(\emph{state}) \textbf{then}\\+
		\textbf{return} UTILITY(\emph{state})\\-
		$v\leftarrow\infty$\\
		\textbf{for each} $a\in\textsf{ACTIONS(\emph{state})}$ \textbf{do}\\+
		$v\leftarrow \textsf{MIN}(v, \textsf{MAX-VALUE}(\textsf{RESULT}(\textsf{\emph{state}}, a), \alpha, \beta))$\\
		\textbf{if} $v\leq \alpha$ \textbf{then}\\+
		\textbf{return} $v$\\-
		$\beta\leftarrow\textsf{MIN}(\beta, v)$\\-
		\textbf{return} $v$\\-
	\end{pseudo}
\end{definition}

\subsection{Summary}

Let $b$ be the branching factor, $d$ be the optimal depth, $m$ the maximum depth, $l$ be the
depth limit$n$ be any node and $g$ be the goal node.

\begin{center}
	\begin{tabular}{l | l | l | l | l}
		\textbf{Strategy}   & \textbf{Complete?}                     & \textbf{Optimal?}                & \textbf{Time} & \textbf{Space}
		\\\hline
		Breadth-first       & yes, if $b$ is finite                  & yes, if each step has equal cost & $O(b^d)$      & $O(b^d)$       \\
		Depth-first         & no                                     & no                               & $O(b^m)$      & $O(bm)$        \\
		Depth-limited       & yes, if $l\geq d$                      & no                               & $O(b^l)$      & $O(bl)$        \\
		Iterative deepening & yes                                    & yes, if each step has equal cost & $O(b^d)$      & $O(bd)$        \\
		Greedy best-first   & no, can get stuck in loops             & no                               & $O(b^m)$      & $O(b^m)$       \\
		A* (Tree)           & yes, unless $f(n)\leq f(g)$ infinitely & yes, if $h$ is admissible        & $O(b^d)$      & $O(b^d)$       \\
		A* (Graph)          & yes, unless $f(n)\leq f(g)$ infinitely & yes, if $h$ is consistent        & $O(b^d)$      & $O(b^d)$       \\
		Minimax             & yes, if tree is finite                 & yes, against optimal opponent    & $O(b^m)$      & $O(bm)$
	\end{tabular}
\end{center}

\section{Constraint satisfaction problems}

\begin{definition}[R\&N p. 202]
	A \emph{constraint satisfaction problem} consists of three components:
	a set $X=\{X_1, ..., X_n\}$ of variables, a set $D=\{D_1, ..., D_n\}$ of
	domains and a set $C$ of constraints that specify allowable combinations
	of values.\\
	Each $D_i$ consists of a set of allowable values, $\{v_1, ..., v_k\}$
	for variable $X_i$. Each constraint $C_i$ consists of a pair
	$\angles{\text{\emph{scope}}, \text{\emph{rel}}}$ where \emph{scope} is a
	tuple of all the variables involved in the constraint and \emph{rel} is a
	relation that needs to hold.
\end{definition}

\subsection{Constraint propagation}

\begin{definition}[R\&N p. 208]
	A variable is \emph{node-consistent} if all the values in the variable's
	domain satisfy the variable's unary constraints.
\end{definition}

\begin{definition}
	Let $X_i,X_j\in X$ be variables. Then $X_i$ is \emph{arc-consistent} with
	respect to $X_j$ if for every value in the current domain $D_i$ there is some
	value in the domain $D_j$ that satisfies the binary constraint on the arc
	$(X_i, X_j)$.
\end{definition}

\begin{definition}[R\&N p. 210]
	A variable $X_i$ is \emph{generalised arc consistent} with respect to
	an $n$-ary constraint if for every value $v$ in the domain of $X_i$ there
	exists a tuple of values that is a member of the constraint, has all its values
	taken from the corresponding variables and has its $X_i$ component equal to $v$.
\end{definition}

\begin{definition}
	A two-variable set $\{X_i, X_j\}$ is \emph{path-consistent} with respect to a
	third variable $X_m$ if, for every assignment $\{X_i=a,X_j=b\}$ consistent with
	the constraints on $\{X_i,X_j\}$, there is an assignment to $X_m$ that satisfies
	the constraints on $\{X_i,X_m\}$ and $\{X_m,X_j\}$.
\end{definition}

\begin{definition}[R\&N p. 211]
	A CSP is \emph{$k$-consistent} if, for any set of $k-1$ variables and for any consistent
	assignment to those variables, a consistent value can always be assigned to any
	$k$-th variable.
	A CSP is \emph{strongly $k$-consistent} if for all $k'\leq k$ it is $k'$-consistent.
\end{definition}

\subsection{Backtracking search}

\begin{definition}[R\&N p. 215]
	\emph{Backtracking search} is a search strategy where at every step only a single
	variable and all its possible values are considered.
\end{definition}

\begin{lemma}
	A backtracking search tree for a problem with $n$ variables with $d$ possible values
	has $d^n$ leaves.
\end{lemma}

\begin{definition}
	We consider the following heuristics for backtracking search:
	\begin{itemize}
		\item \emph{minimum-remaining-values}: chooses the variable with the fewest legal values
		\item \emph{degree}: chooses the variable involved in the largest number of constraints
		\item \emph{least-constraining-value}: chooses the value that
		      rules out the fewest values for the remaining variables
	\end{itemize}
\end{definition}

\begin{definition}[R\&N p. 217]
	\emph{Forward checking} is an inference strategy that is applied after every
	step in backtracking search to ensure arc-consistency for all neighbouring variables.
\end{definition}

\begin{definition}[R\&N p. 218]
	\emph{Maintaining Arc Consistency (MAC)} is an algorithm that, after a value has
	been selected for a variable $X_i$ in backtracking search, applies the arc
	consistency algorithm for all neighbouring variables $X_j$ that have not yet
	been assigned a value.
\end{definition}

\section{Logical agents}

\subsection{Knowledge-based agents}

\begin{definition}[R\&N p. 235]
	An agent's \emph{knowledge base} is a set of \emph{sentences} representing assertions
	about the world.

	A sentence is called an \emph{axiom} if it is not derived from other sentences.
\end{definition}

\begin{pseudo}
	\textbf{function} KB-AGENT(\emph{percept}) \textbf{returns} an \emph{action} \\+
	\textbf{persistent:} $KB$, $t$\\
	TELL($KB$,MAKE-PERCEPT-SENTENCE(\emph{percept}, $t$))\\
	$\textsf{\emph{action}}\leftarrow\textsf{ASK}(KB,\textsf{MAKE-ACTION-QUERY}(t))$\\
	$\textsf{TELL}(KB, \textsf{MAKE-ACTION-SENTENCE}(\textsf{\emph{action}},t))$\\
	$t\leftarrow t+1$\\
	\textbf{return} \emph{action}
\end{pseudo}

\subsection{Propositional logic}

\begin{definition}[R\&N p. 240]
	A \emph{representation language} must have a \emph{syntax} that defines which
	sentences are well formed and \emph{semantics}, i.e. the meaning of sentences.
\end{definition}

\begin{definition}[R\&N p. 241]
	Let $m$ be a model and $\alpha$ be a sentence. If $\alpha$ is true in $m$, then
	$m$ \emph{satisfies} $\alpha$. The set of all models that satisfy $\alpha$ is
	denoted $M(\alpha)$.\\
	Let $\alpha,\beta$ be sentences. Then we define the following:
	\begin{itemize}
		\item $a\vDash b$ if and only if $M(\alpha)\subseteq M(\beta)$
		\item $a\equiv b$ if and only if $M(\alpha)=M(\beta)$
		\item $\alpha$ is \emph{satisfiable} only if $M(\alpha)\not=\emptyset$
		\item $\alpha$ is \emph{valid} only if $M(\neg\alpha)=\emptyset$
	\end{itemize}
\end{definition}

\begin{definition}
	Let $i$ be an inference algorithm, $\alpha$ be a sentence and $KB$ be a knowlege base.
	If $i$ can derive $\alpha$ from $KB$ then we write $KB\vdash_i \alpha$.
	An inference algorithm is called
	\begin{itemize}
		\item \emph{sound} if it only derives entailed sentences.
		\item \emph{complete} if it can derive any sentence that is entailed.
	\end{itemize}
\end{definition}

\begin{lemma}[R\&N p. 249]
	Let $\alpha,\beta$ be sentences. Then
	\begin{itemize}
		\item $\alpha\vDash\beta$ if and only if $\alpha\Rightarrow\beta$ is valid. (Deduction theorem)
		\item $\alpha\not\vDash\beta$ if and only if $(\alpha\wedge\neg\beta)$ is satisfiable.
	\end{itemize}
\end{lemma}

\begin{definition}[R\&N p. 251]
	A knowledge base is called \emph{monotonous} if the set of entailed sentences can
	only increase as information is added, i.e.
	\begin{align*}
		\text{if}\hs KB \vDash \alpha \hs\text{then}\hs KB\wedge \beta \vDash \alpha
	\end{align*}
	for all sentences $\alpha,\beta$.
\end{definition}

\subsection{Satisfiability \& resolution}

\begin{theorem}[Resolution rule, R\&N p. 253]
	Let $l_1,...,l_k,m_1,...,m_n$ be literals and let $l_i\Leftrightarrow\neg m_j$. Then
	\begin{align*}
		\infer{l_1\vee \cdots\vee l_{i-1}\vee l_{i+1}\vee\cdots\vee l_k\vee m_1\vee\cdots\vee m_{j-1}\vee m_{j+1}\vee \cdots\vee m_n}{
			l_1\vee\cdots \vee l_k\hs m_1\vee\cdots\vee m_n
		}.
	\end{align*}
\end{theorem}

\begin{theorem}[Ground resolution theorem]
	If a set of clauses is unsatisfiable, then the resolution closure of
	those clauses contains the empty clause.
\end{theorem}

\begin{definition}
	We define the following types of special clauses:
	\begin{itemize}
		\item \emph{unit clause}: disjunction of a single literal
		\item \emph{definite clause}: disjunction of literals of which exactly one is positive
		\item \emph{Horn clause}: disjunction of literals of which at most one is positive
	\end{itemize}
\end{definition}


\begin{pseudo}
	\textbf{function} PL-RESOLUTION$(KB,\alpha)$ \textbf{returns} $true$ or $false$\\+
	$\textsf{\emph{clauses}}\leftarrow \text{the set of clauses in the CNF representation of }KB\wedge\neg\alpha$\\
	$\textsf{\emph{new}}\leftarrow\emptyset$\\
	\textbf{loop do}\\+
	\textbf{for each} pair of clauses $(C_i,C_j)$ \textbf{in} \emph{clauses} \textbf{do}\\+
	$\textsf{\emph{resolvents}}\leftarrow\textsf{PL-RESOLVE}(C_i,C_j)$\\
	\textbf{if} \emph{resolvents} contains an empty clause \textbf{then return} $true$\\
	$\textsf{\emph{new}}\leftarrow \textsf{\emph{new}}\cup \textsf{\emph{resolvents}}$\\-
	\textbf{if} $\textsf{\emph{new}} \subseteq \textsf{\emph{clauses}}$ \textbf{then return} $false$\\
	$\textsf{\emph{clauses}}\leftarrow\textsf{\emph{clauses}}\cup\textsf{\emph{new}}$
\end{pseudo}

\begin{pseudo}
	\textbf{function} DPLL-SATISFIABLE?$(s)$ \textbf{returns} $true$ or $false$\\+
	$\textsf{\emph{clauses}}\leftarrow\textsf{the set of clauses in the CNF representation of }s$\\
	$\textsf{\emph{symbols}}\leftarrow\textsf{a list of the proposition symbols in }s$\\
	\textbf{return} DPLL$(\textsf{\emph{clauses}}, \textsf{\emph{symbols}}, \emptyset)$
	\\\\-
	\textbf{function} DPLL$(\textsf{\emph{clauses}}, \textsf{\emph{symbols}}, \textsf{\emph{model}})$ \textbf{returns} $true$ or $false$\\+
	\textbf{if} every clause in \emph{clauses} is true in \emph{model} \textbf{then return} $true$\\
	\textbf{if} some clause in \emph{clauses} is false in \emph{model} \textbf{then return} $false$\\
	$P,\textsf{\emph{value}}\leftarrow\textsf{FIND-PURE-SYMBOL}(\textsf{\emph{symbols}},\textsf{\emph{clauses}},\textsf{\emph{model}})$\\
	\textbf{if} $P$ is non-null \textbf{then return} DPLL$(\textsf{\emph{clauses}}, \textsf{\emph{symbols}}-P,\textsf{\emph{model}}\cup\{P=\textsf{\emph{value}}\})$\\
	$P,\textsf{\emph{value}}\leftarrow\textsf{FIND-UNIT-CLAUSE}(\textsf{\emph{clauses}}, \textsf{\emph{model}})$\\
	\textbf{if} $P$ is non-null \textbf{then return} DPLL$(\textsf{\emph{clauses}}, \textsf{\emph{symbols}}-P,\textsf{\emph{model}}\cup\{P=\textsf{\emph{value}}\})$\\
	$P\leftarrow\textsf{FIRST}(\textsf{\emph{symbols}})$; $\textsf{\emph{rest}}\leftarrow\textsf{REST}(\textsf{\emph{symbols}})$\\
	\textbf{return} $\textsf{DPLL}(\textsf{\emph{clauses}}, \textsf{\emph{rest}}, \textsf{\emph{model}}\cup\{P=true\})$ \textbf{or}
	$\textsf{DPLL}(\textsf{\emph{clauses}}, \textsf{\emph{rest}}, \textsf{\emph{model}}\cup\{P=false\})$
\end{pseudo}

\paragraph{Optimisations in SAT solvers}
(R\&N p. 261)
\begin{enumerate}
	\item \emph{component analysis}: Find disjoint components and solve separately
	\item \emph{variable and value ordering}: Find most occuring literal and set to true
	\item \emph{intelligent backtracking}: record conflicts to not repeat them later
	\item \emph{random restarts}: restarting with learned information may improve progress
	\item \emph{clever indexing}: optimise queries like "the set of clauses in which variable $X_i$ appears as a positive literal"
\end{enumerate}

\subsection{SAT-based planning}

\begin{theorem}
	To create a plan using a SAT solver, we need to provide the following
	\begin{itemize}
		\item \emph{initial state}
		\item \emph{goal}
		\item \emph{successor-state axioms}
		\item \emph{precondition axioms}: avoid illegal actions by imposing constraints
		\item \emph{action-exclusion axioms}: avoid illegal combinations of actions
	\end{itemize}
\end{theorem}


\section{First-order logic}


\subsection{Basics}

\begin{definition}[R\&N p. 290]
	A model in first-order logic consists of a set of objects and an interpretation
	that maps constant symbols to objects, predicate symbols to relations on
	those objects and function symbols to functions on those objects.
\end{definition}

\begin{definition}[R\&N p. 293]
	An \emph{atomic sentence} is formed from a predicate optionally followed
	by a parenthesised list of terms. Such a sentence is true in a given model
	if the relation referred to by the predicate symbol holds among the objects
	referred to by the arguments.
\end{definition}

\begin{theorem}[R\&N p. 298]
	Let $P$ be a predicate. Then
	\begin{align*}
		\forall x,\: P(x) \equiv \neg(\exists x,\: \neg P(x)).
	\end{align*}
\end{theorem}

\subsection{Knowledge engineering}

\paragraph{Steps}(R\&N p. 307)
\begin{enumerate}
	\item Identify the task
	\item Assemble the relevant knowledge
	\item Decide on a vocabulary of predicates, functions and constants
	\item Encode general knowledge about the domain
	\item Encode a description of the specific problem instance
	\item Pose queries to the inference procedure and get answers
	\item Debug the knowledge base
\end{enumerate}

\subsection{Propositional inference}

\begin{theorem}[Universal Instantiation, R\&N p. 323]
	Let $\subst(\theta, \alpha)$ denote the result of applying
	the substitution $\theta$ to the sentence $\alpha$. Then
	\begin{align*}
		\infer{\subst(\{v/g\},\alpha)}{\forall v,\:\alpha}
	\end{align*}
	for any variable $v$ and ground term $g$, i.e. a term without variables.
\end{theorem}

\begin{theorem}[Existential Instantiation]
	Let $\alpha$ be a sentence, $v$ be a variable and $k$ be a
	symbol that does not appear elsewhere in the knowledge base
	(called \emph{Skolem constant}), then
	\begin{align*}
		\infer{\subst(\{v/k\},\alpha)}{\exists v,\:\alpha}
	\end{align*}
\end{theorem}

\begin{theorem}[R\&N p. 325]
	The question of entailment for first-order logic is \emph{semidecidable}.
\end{theorem}

\subsection{Unification and lifting}

\begin{theorem}[Generalised Modus Ponens]
	For atomic sentences $p_i$, $p_i'$ and $q$, where there is a substitution
	$\theta$ such that $\subst(\theta,p_i')=\subst(\theta,p_i)$, for
	all $i$,
	\begin{align*}
		\infer{\subst(\theta,q)}{p_1',\: p_2',\: ...,\: p_n',\:(p_1\wedge p_2 \wedge ... \wedge p_n\Rightarrow q)}.
	\end{align*}
\end{theorem}

\begin{definition}[R\&N p. 326]
	The \emph{unification} alogrithm takes two sentences and returns a unifier
	for them if one exists:
	\begin{align*}
		\textsf{UNIFY}(p,q)=\theta \hs\text{where}\hs \subst(\theta,p)=\subst(\theta,q).
	\end{align*}
\end{definition}

\begin{theorem}[R\&N p. 327]
	For every unifiable pair of expressions, there exists a single \emph{most general
		unifier (MGU)} that is unique up to renaming and substitution of variables.
\end{theorem}

\subsection{Forward-chaining}

\begin{definition}[R\&N p. 331]
	\emph{Datalog} is a language of knowledge bases that is restricted to first-order
	definite clauses with no function symbols.
\end{definition}

\paragraph{Simple forward-chaining alogrithm}

\begin{enumerate}
	\item Start with a set of known facts $KB$
	\item For all clauses in $KB$ whose premises are satisfied, add the conclusion to $KB$.
	\item Repeat until no new facts can be added \emph{or} the query is answered.
\end{enumerate}

\begin{theorem}
	The algorithm above is
	\begin{itemize}
		\item \emph{sound} and \emph{complete} for Datalog knowledge bases.
		\item \emph{sound} for general knowledge bases.
	\end{itemize}
\end{theorem}

\begin{theorem}[R\&N p. 334]
	Every finite domain CSP can be expressed as a
	single definite clause with some ground facts.
\end{theorem}

\begin{theorem}
	Matching a definite clause against a set of facts is NP-hard.
\end{theorem}

\subsection{Backward-chaining}

\begin{definition}[R\&N p. 337]
	A \emph{generator} is a function that returns multiple times, each time giving one
	possible result.
\end{definition}

\begin{definition}[R\&N p. 338]
	\newcommand{\func}{\textbf{function }}
	\newcommand{\return}{\textbf{return }}
	\newcommand{\returns}{\textbf{returns }}
	\newcommand{\generator}{\textbf{generator }}
	\newcommand{\foreach}[2]{\textbf{for each} #1 \textbf{in} #2 \textbf{do}}
	A simple first-order logic backward-chaining algorithm works as follows:
	\begin{pseudo}
		\func FOL-BC-ASK$(KB,query)$ \returns a generator of substitutions\\+
		\return FOL-BC-OR$(KB,goal,\{\})$\\-
		\\
		\generator FOL-BC-OR$(KB, goal, \theta)$ \textbf{yields} a substitution\\+
		\foreach{rule $(lhs\Rightarrow rhs)$}{FETCH-RULES-FOR-GOAL$(KB, goal)$}\\+
		$(lhs,rhs)\leftarrow\textsf{STANDARDISE-VARIABLES}((lhs, rhs))$\\
		\foreach{$\theta'$}{FOL-BC-AND$(KB,lhs,\textsf{UNIFY}(rhs,goal,\theta))$}\\+
		\textbf{yield} $\theta'$\\---
		\\
		\generator FOL-BC-AND$(KB,goals,\theta)$ \textbf{yields} a substitution\\+
		\textbf{if} $\theta=failure$ \textbf{then return}\\
		\textbf{else if} LENGTH$(goals)=0$ \textbf{then yield} $\theta$\\
		\textbf{else do}\\+
		$first,rest\leftarrow \textsf{FIRST}(goals),\textsf{REST}(goals)$\\
		\foreach{$\theta'$}{FOL-BC-OR$(KB,\subst(\theta,first),\theta)$}\\+
		\foreach{$\theta''$}{FOL-BC-AND$(KB,rest,\theta')$}\\+
		\textbf{yield} $\theta''$

	\end{pseudo}
\end{definition}

\begin{definition}
	There are two types of parallelism that may be used to optimise CSP solvers:
	\begin{itemize}
		\item \emph{OR-parallelism}: goal may unify with many different clauses;
		      branches can be computed in parallel
		\item \emph{AND-parallelism}: each conjunct in the body of an implication
		      may be solved individually (though each branch needs to have bindings
		      consistent with all the other branches)
	\end{itemize}
\end{definition}

\subsection{Resolution}

\begin{theorem}[R\&N p. 346]
	Every sentence of first-oder logic can be converted into an inferentially equivalent
	CNF sentence. This may be achieved as follows:
	\begin{enumerate}
		\item eliminate implications
		\item move $\neg$ inwards
		\item standardise variables
		\item skolemise
		\item drop universal quantifiers
		\item distribute $\vee$ over $\wedge$
	\end{enumerate}
\end{theorem}


\begin{theorem}[Binary resolution, factoring; R\&N p. 347]
	\emph{Binary resolution} is given by the rule
	\begin{align*}
		\infer{\subst(\theta,l_1\vee \cdots\vee l_{i-1}\vee l_{i+1}\vee\cdots\vee l_k\vee m_1\vee\cdots\vee m_{j-1}\vee m_{j+1}\vee \cdots\vee m_n)}{
			l_1\vee\cdots \vee l_k\hs m_1\vee\cdots\vee m_n
		}
	\end{align*}
	where $\theta=\textsf{UNIFY}(l_i,\neg m_j)$.\\
	\emph{Factoring} reduces two literals to one if they are unifiable. The unifier must be applied to the
	entire clause.\\
	The combination of binary resolution and factoring is complete.
\end{theorem}

\begin{theorem}[R\&N p. 350]
	If $S$ is an unsatisfiable set of clauses, then the application of a finite number of
	resolution steps to $S$ will yield a contradiction.
\end{theorem}

\begin{definition}[Herbrand universe, R\&N p. 351]
	If $S$ is a set of clauses, then $H_S$, the \emph{Herbrand universe} of $S$, is
	the set of all ground terms constructable from the following
	\begin{enumerate}
		\item The function symbols in $S$, if any.
		\item The constant symbols in $S$, if any; if none, then the constant symbol $A$.
	\end{enumerate}
\end{definition}

\begin{definition}[Saturation]
	If $S$ is a set of clauses and $P$ is a set of ground terms, then $P(S)$, the
	saturation of $S$ with respect to $P$, is the set of all ground clauses obtained
	by applying all possible consistent substitutions of ground terms in $P$ with variables in $S$.
\end{definition}

\begin{definition}[Herbrand base]
	The saturation of a set $S$ of clauses with respect to its Herbrand universe is called the Herbrand
	base of $S$, written $H_S(S)$.
\end{definition}

\begin{theorem}[Herbrand's theorem]
	If a set $S$ of clauses is unsatisfiable, then there exists a finite subset of $H_S(S)$
	that is also unsatisfiable.
\end{theorem}

\begin{lemma}[Lifting lemma, R\&N p. 353]
	Let $C_1$ and $C_2$ be two clauses with no shared variables,
	and let $C_1'$ and $C_2'$ be ground instances of $C_1$ and $C_2$.
	If $C'$ is a resolvent of $C_1'$ and $C_2'$, then there exists a clause
	$C$ such that (1) $C$ is a resolvent of $C_1$ and $C_2$ and (2) $C'$ is a
	ground instance of $C$.
\end{lemma}

\begin{definition}[Demodulation, R\&N p. 354]
	For any terms $x,y,z$ where $z$ appears somewhere in literal $m_i$ and where
	$\textsf{UNIFY}(x,z)=\theta$,
	\begin{align*}
		\infer{
		\textsf{SUB}(\subst(\theta,x),\subst(\theta,y),m_1\wedge\cdots\wedge m_n)
		}{
			x=y,\hs m_1\wedge\cdots\wedge m_n
		}
	\end{align*}
	where $\subst$ is the usual substitution of a binding list, and SUB$(x,y,m)$
	means to replace $x$ with $y$ everywhere that $x$ occurs within $m$.
\end{definition}

\begin{definition}[Paramodulation]
	For any terms $x,y,z$ where $z$ appears somewhere in literal $m_i$,
	and where $\textsf{UNIFY}(x,z)=\theta$,
	\begin{align*}
		\infer{
		\textsf{SUB}(\subst(\theta,x),\subst(\theta,y),\subst(\theta,l_1\wedge\cdots\wedge l_k\wedge m_1\wedge\cdots\wedge m_n))
		}{
			l_1\wedge \cdots\wedge l_k\wedge x=y,\hs m_1\wedge\cdots\wedge m_n
		}.
	\end{align*}
\end{definition}

\begin{theorem}[R\&N p. 355]
	Paramodulation yields a complete inference procedure for first-order logic
	with equality.
\end{theorem}

\subsection{Resolution strategies}

\begin{proposition}[R\&N p. 355]
	We consider the following resolution strategies
	\begin{itemize}
		\item \emph{unit preference}: prefers resolution with at least one unit clause
		\item \emph{unit resolution}: restricted; every step must involve a unit clause; complete for Horn clauses
		\item \emph{set of support}: every resolution must involve one element of a special
		      set of clauses and resolvent is added to this set; may be incomplete depending on $S$
		\item \emph{input resolution}: is a strategy where every resolution combines one of the
		      input sentences with some other sentence; not complete
		\item \emph{linear resolution}: $P$ and $Q$ may be resolved if $P$ is in the original
		      $KB$ or if $P$ is an ancestor of $Q$; complete
		\item \emph{subsumption}: eliminates al sentences that are more specific than an existing
		      sentence in the $KB$
	\end{itemize}
\end{proposition}


\subsection{Situation calculus}

\begin{definition}
	\emph{Situations} are logical terms consisting of the initial
	situation (usually $S_0$) and all situations that are generated
	by applying an action to a situation. The function $\textsf{Result}
		(a,s)$ names the situation that results when action $a$ is executed
	in situation $s$.\\
	We observe two types of predicates and function:
	\begin{itemize}
		\item \emph{fluents}: vary from one situation to the next,
		\item \emph{atemporal}: do not vary from one situation to the next.
	\end{itemize}
	Further, there are new types of tasks:
	\begin{itemize}
		\item \emph{projection}: deduction of outcome of a given sequence of actions,
		\item \emph{planning}: finding a sequence of actions that achieves desired effect.
	\end{itemize}
\end{definition}

\begin{definition}
	Each action is described by two axioms: a \emph{possibility axiom}
	that states when it is possible to execute the action, and an
	\emph{effect axiom} that says what happens when a possible action
	is executed.
\end{definition}

\begin{definition}[Frame problem, R\&N p. 266]
	The fact that the possibility and effect axioms only describe what changes
	and not what stays the same gives rise to the \emph{frame problem}. In a world
	with $m$ actions and $n$ fluents of which at most $k$ are affected by an action,
	we have the following specialisations:
	\begin{itemize}
		\item \emph{representational}: Finding a set of frame axioms of size $O(mn)$.
		\item \emph{inferential}: Projecting forward the result of a $t$ step plan in $O(kt)$ time.
	\end{itemize}
\end{definition}

\begin{theorem}
	\emph{Frame axioms} state what stays the same for each possible action and
	solve the frame problem by introducing $O(mn)$ axioms.
\end{theorem}

\begin{theorem}[R\&N p. 267]
	\emph{Successor-state axioms} have the form
	\begin{align*}
		\textsf{Action } & \textsf{is possible} \Rightarrow                         \\
		                 & (\textsf{Fluent is true in result state} \Leftrightarrow
		\textsf{Action's effect made it true}\vee \textsf{It was true before
			and action left it alone}).
	\end{align*}
	and solve the representational
	frame problem because the total size of the axioms is $O(mk)$ where
	$k$ is the number of effects of each of the $m$ actions.
\end{theorem}

\begin{definition}[Unique names]
	The \emph{unique names axiom} states disequality for every pair of
	constants in the knowledge base. When this is assumed by the theorem
	prover, rather than written down in the knowledge base, it is called
	a \emph{unique names assumption}.
\end{definition}

\begin{definition}
	\emph{Unique action axioms} have the form
	\begin{align*}
		A(x_1,...,x_m)=A(y_1,...,y_m)\Leftrightarrow x_1 = y_1\wedge\cdots \wedge x_m=y_m
	\end{align*}
	and enforce that two action terms with the same action name refer to the
	same action only if they involve all the same objects.
\end{definition}

\paragraph{Solution to the inferential frame problem}

\begin{enumerate}
	\item Index actions by fluent, effect (positive or negative) and time.
	\item Index axioms so that if $F_i$ is an effect, you can find the axiom for $F_i$ in $O(1)$ time.
	\item Represent each situation as a previous state plus a delta.
\end{enumerate}
Runtime: $O(kt)$

\section{Planning}

\subsection{Definition of classical planning}

\begin{definition}[R\&N, p. 367]
	\emph{PDDL}, the Planning Domain Definition Language, is a language
	that allows us to express many actions in one action schema.
	Each state is represented as a conjunction of fluents that are ground,
	functionless atoms. Database semantics is used, i.e. each unmentioned
	fluent is false and variables with distinct names refer to distinct
	objects.\\
	An \emph{action schema} is a lifted representation of a set of ground
	actions. Every action is defined by a precondition and an effect. An
	action $a$ is \emph{applicable} in state $s$ if $s$ entails the precondition
	of $a$. An action $a$ involving a set of variables $x_1,...,x_n$ with
	a precondition $P$ and an effect $E$ is written as follows:
	\begin{align*}
		Action & (a(x_1, ..., x_n),                   \\
		       & \textsf{Precond: } P(x_1, ..., x_n), \\
		       & \textsf{Effect: } E(x_1, ..., x_n)).
	\end{align*}
\end{definition}

\begin{lemma}
	If $a$ has $v$ variables, then, in a domain with $k$ unique names of objects,
	it takes $O(k^v)$ time to find the applicable ground actions.
\end{lemma}

\begin{definition}
	The \emph{result} of executing an action $a$ in state $s$ is defined
	as a state $s'$ which is represented by
	\begin{align*}
		\textsf{RESULT}(s,a)=(s\setminus\textsf{DEL}(a))\cup\textsf{ADD}(a)
	\end{align*}
	where DEL is the \emph{delete list}, i.e. the set of literals that appear
	as negative in the action's effect, and ADD is the \emph{add list}, i.e.
	the set of literals that appear as positive in the action's effect.
\end{definition}

\subsection{Planning as state-space search}

\begin{theorem}[R\&N, p. 373]
	\emph{Forward state-space search}, i.e. iterating possible sequences of
	actions until a goal state is reached, has the following drawbacks:
	\begin{itemize}
		\item too many iterations required,
		\item irrelevant actions may be chosen.
	\end{itemize}
\end{theorem}

\begin{definition}
	\emph{Backward relevant-states search} starts at the goal state and applies
	relevant actions backward until a sequence of steps that reaches the initial
	state has been found. An action is relevant if
	\begin{enumerate}
		\item at least one of the action's effects unifies with an element of the goal, and
		\item none of its effects negate an element of the goal.
	\end{enumerate}
	Formally, assume a goal description $g$
	which contains a goal literal $g_i$ and an action schema $A$ that is standardised
	to produce $A'$. If $A'$ has an effect literal $e_j'$ where $\textsf{UNIFY}(g_i,e_j')=\theta$
	and where we define $a'=\textsf{SUBST}(\theta, A')$ and if there is no effect in
	$a'$ that is the negation of a literal in $g$, then $a'$ is a relevant action towards
	$g$.
\end{definition}

\begin{theorem}[R\&N p. 376]
	The \emph{ignore preconditions heuristic} drops all preconditions from actions
	and thus makes every action applicable in every state. This may be relaxed in
	such a way that only certain preconditions are ignored.

	Using this heuristic to find the minimum number of
	actions required such that the union of those actions'
	effects satisfies the goal is an instance of the \emph{set-cover problem}. This
	is NP-hard. There exists a greedy alogrithm that is guaranteed to find a solution
	within a factor of $O(\lg n)$ of the optimum.
\end{theorem}

\begin{theorem}[R\&N p. 377]
	The \emph{ignore delete lists} heuristic ignores the negative literals from all
	effects.

	Using the ignore delete lists heuisitic to find the minimum number of actions
	required such that the union of those actions' effects satisfies the goal
	is NP-hard but an approximate solution can be found in polynomial time using
	hill-climbing.
\end{theorem}

\begin{definition}
	A \emph{state abstraction} is a many-to-one mapping from states in the ground
	representation of the problem to abstract states.
\end{definition}

\begin{definition}
	The \emph{subgoal independence} assumption is that the cost of solving a conjunction
	of subgoals is approximated by the sum of the costs of solving each subgoal
	\emph{independently}.
\end{definition}

\subsection{SAT for planning}

\begin{theorem}
	Every PDDL description may be translated into a form that can be
	processed by SATPlan as follows:
	\begin{enumerate}
		\item Propositionalise the actions: replace each action schema with a set of ground actions
		      formed by substituting constants for each of the variables.
		\item Define the intial state: Assert $F^0$ for every fluent $F$ in the problem's intial state
		      and $\neg F^0$ for every fluent not mentioned in the initial state.
		\item Propositionalise the goal: the goal becomes a disjunction over all of its ground instances
		      where variables are replaced by constants.
		\item Add successor-state axioms: For each fluent $F$, add an axiom of the form \begin{align*}
			      F^{t+1} \Leftrightarrow \textsf{ActionCausesF}^t \vee (F^t \wedge \neg \textsf{ActionCausesNotF}^t)
		      \end{align*}
		      where $ACF$ is a disjunction of all the ground actions that have $F$ in
		      their add list, and $ACNF$ is a disjunction of all the ground actions that have $F$
		      in their delete list.
		\item Add precondition axioms: For each ground action $A$, add the axiom $A^t\Rightarrow\textsf{PRE}(A)^t$,
		      that is, if an action is taken at time $t$, then the preconditions must have been true.
		\item Add action exclusion axioms: say that exactly one ground action can occur at each step.
	\end{enumerate}
\end{theorem}

\subsection{Partially ordered plans}

\begin{definition}
	A \emph{plan} is a set of actions and a set of constraints of the form $\textsf{Before}(a_i, a_j)$ saying
	that one action occurs before another. A \emph{flaw} is anything that keeps a partial plan from
	being a solution.
	Such a partially ordered plan may be constructed by making the least commitment possible to fix the
	current flaw.
\end{definition}

\subsection{Planning in nondeterministic domains}

\subsubsection{Sensorless planning}

\begin{itemize}
	\item Sensorless planning requires an open-world assumption.
	\item The belief state corresponds to all the states that satisfy the formula.
	\item In a given belief state $b$, the agent can consider any action whose preconditions are satisfied by $b$.
\end{itemize}

\begin{theorem}[R\&N p. 418]
	The family of belief states defined as conjunctions of literals is closed under updates defined by PDDL
	action schemas.
\end{theorem}

\begin{definition}
	\emph{Conditional effects} occur in action schemas of the form
	\begin{align*}
		Action(A,
		\textsf{Effect:}\textbf{ when }c_1: e_1 \wedge \cdots \wedge \textbf{when }c_n: e_n)
	\end{align*}
\end{definition}

\subsubsection{Online replanning}

\begin{definition}[R\&N p. 423]
	Online replanning is required if the agent's model is incomplete.
	This can be caused by the following flaws in the model:
	\begin{itemize}
		\item \emph{missing precondition}
		\item \emph{missing effect}
		\item \emph{missing state variable}
		\item missing provision for \emph{exogenous events}
	\end{itemize}
\end{definition}

\begin{definition}[R\&N p. 423]
	There are three levels of monitoring the environment:
	\begin{itemize}
		\item \emph{action monitoring}: before executing an action, the agent verifies that
		      all the preconditions still hold
		\item \emph{plan monitoring}: before executing an action, the agent verifies that the
		      remaining plan will still succed
		\item \emph{goal monitoring}: before executing an action, the agent checks to see
		      if there is a better set of goals it could be trying to achieve
	\end{itemize}
\end{definition}

\subsection{Hierarchical planning}

\subsubsection{High-level actions}

\begin{definition}[HLA, R\&N p. 406]
	A \emph{high-level action} (HLA) is an action that has one or more possible
	\emph{refinements}, into a sequence of actions, each of which may be high-level
	or primitive.\\
	An HLA refinement that contains only primitive actions is called an
	\emph{implementation}.
\end{definition}

\begin{definition}[R\&N p. 407]
	A high-level plan (a sequence of HLAs) achieves the goal from a given state
	if at least one of its implementations achieves the goal from that state.
\end{definition}

\subsubsection{Search for primitive solutions}

\begin{theorem}[R\&N p. 409]
	Suppose each HLA has $r$ possible refinement, each into $k$ actions at the
	next lower level. Then, if there are $d$ actions at the primitive level,
	there are
	\begin{align*}
		r^{(d-1)/(k-1)}
	\end{align*}
	possible decomposition trees.
\end{theorem}

\subsubsection{Searching for abstract solutions}

\begin{definition}[R\&N p. 410]
	The \emph{downward refinement property} for HLA descriptions states that
	a high-level plan must have at least one implementations that achieves the
	goal.
\end{definition}

\begin{definition}[R\&N p. 411]
	Given a state $s$, the \emph{reachable set} for an HLA $h$, written as
	$\textsf{REACH}(s,h)$, is the set of states reachable by any of the HLA's
	implementations.
\end{definition}

\begin{definition}
	In \emph{demonic nondeterminism} a high-level plan achieves the goal if
	its reachable set is fully contained in the set of goal states.
	On the other hand, \emph{angelic nondeterminism} only requires the two
	sets to intersect for the plan to achieve the goal.
\end{definition}

\begin{theorem}[R\&N p. 412]
	An HLA under angelic semantics can have one of nine different effects on
	a variable $A$. There are two cases: $A$ is true or $A$ is false. In both
	cases there are three different effects:
	\begin{enumerate}
		\item $A$ is always true afterwards
		\item $A$ is never true afterwards
		\item the agent can choose the truth value of $A$
	\end{enumerate}
	The effects in both cases can be combined independently.
\end{theorem}

\begin{definition}
	Each HLA has two approximations:
	\begin{itemize}
		\item an \emph{optimistic description} $\textsf{REACH}^+(s,h)$ that may
		      overstate the reachable set, and
		\item a \emph{pessimistic description} $\textsf{REACH}^-(s,h)$ that may
		      understate the reachable set.
	\end{itemize}
	We thus have
	\begin{align*}
		\textsf{REACH}^-(s,h) \subseteq\textsf{REACH}(s,h)\subseteq\textsf{REACH}^+(s,h).
	\end{align*}
\end{definition}


\section{Quantifying uncertainty}


\subsection{Acting under uncertainty}

\begin{definition}[R\&N p. 480]
	\emph{Uncertainty} arises when an agent may not know for certain what state
	it's in or where it will end up after a sequence of actions. There are
	two main reasons for uncertainty:
	\begin{itemize}
		\item partial observability
		\item nondeterminism
	\end{itemize}
\end{definition}

\begin{lemma}
	Keeping track of a belief state as a recipe for creating agent programs
	has the following drawbacks:
	\begin{itemize}
		\item The agent must consider every logically possible explanation, regardless of likelihood.
		\item Handling every eventuality will lead to an arbitrarily large plan.
		\item If no plan exists that is guaranteed to achieve the goal, the agent still has to find some action.
	\end{itemize}
\end{lemma}

\begin{definition}[R\&N p. 481]
	The \emph{rational decision} depends on both the relative importance of various goals and the
	likelihood that, and degree to which, they will be achieved.
\end{definition}

\begin{theorem}
	In general, there are three main reasons why trying to use logic
	to cope with some specific domain may fail:
	\begin{enumerate}
		\item \emph{Laziness}: It is too much work to construct an
		      exceptionless rule and too hard to use such rules.
		\item \emph{Theoretical ignorance}: There exists no complete theory for the domain.
		\item \emph{Practical ignorance}: Not all the necessary observations
		      in a particular case have been or even can be made.
	\end{enumerate}
	An agent in such a domain can at best provide only a \emph{degree of belief}.
\end{theorem}

\begin{definition}[Maximum expected utility, R\&N p. 483 ]
	The \emph{principle of maximum expected utility} (MEU) states that an agent is rational
	if and only if it chooses the action that yieldsd the highest expected utility,
	averaged over all possible outcomes of the action.
\end{definition}

\subsection{Introduction to probability theory}

\renewcommand{\P}{\mathbf{P}}
\renewcommand{\vec}[1]{\mathbf{#1}}

\begin{definition}[R\&N p. 484]
	The \emph{sample space} is the set of all possible worlds.
	The possible worlds are mutually exclusive and exhaustive, i.e.
	two possible worlds cannot both be the case and one possible world must
	be the case.
\end{definition}

\begin{definition}
	A fully specified \emph{probability model} associates a numerical probability
	$P(\omega)$ with each possible world. The following axioms hold for every
	model with sample space $\Omega$:
	\begin{enumerate}[label=P\arabic*.]
		\item $\forall \omega\in\Omega.\:0\leq P(\omega)\leq 1$
		\item $\sum_{\omega\in\Omega} P(\omega)=1$
	\end{enumerate}
\end{definition}

\begin{definition}
	An \emph{event} is a set of worlds $\phi\subseteq \Omega$ such that
	for some proposition $Q$ and for all $\omega\in\Omega$
	\begin{align*}
		\omega\in \phi\hs\Leftrightarrow\hs Q(\omega).
	\end{align*}
	The probability associated with $\phi$ is
	\begin{align*}
		P(\phi)=\sum_{\omega\in\phi} P(\omega).
	\end{align*}
\end{definition}

\begin{definition}
	A \emph{possible world} is an assignment of values to all of the RV's
	under consideration.
\end{definition}

\begin{theorem}[De Finetti]
	If an agent $A$ expresses a set of degrees of belief that violate
	the axioms of probability theory then there is a combination of
	bets by another agent $B$ that guarantees that $A$ will lose money
	every time.
\end{theorem}

\subsection{Probabilistic inference}

\begin{lemma}[R\&N p. 493]
	Let $\vec E$ be the list of evidence variables, let $\vec e$ be the list of observed
	values for them, and let $\vec Y$ be the remaining unobserved variables. The query
	$\P(X|\vec e)$ can be evaluated as
	\begin{align*}
		\P(X|\vec e) = \alpha\P(X, \vec e) = \alpha \sum_{\vec y} \P(X,\vec e, \vec y),
	\end{align*}
	where the summation is over all possible combinations of values of the variables in
	$\vec Y$.
\end{lemma}

\begin{definition}[R\&N p. 494]
	Two propositions $a$ and $b$ are called \emph{independent} if
	\begin{align*}
		P(a|b) = P(a).
	\end{align*}
	Similarly, two variables are independent if
	\begin{align*}
		\P(X|Y) = \P(X).
	\end{align*}
\end{definition}

\begin{theorem}[General Bayes' rule, R\&N p. 497]
	Let $X$ and $Y$ be random variables. Then
	\begin{align*}
		\P(Y|X) = \alpha \P(X|Y)\P(Y).
	\end{align*}
\end{theorem}

\begin{definition}[R\&N p. 498]
	Given a variable $Z$, two variables $X$ and $Y$ are conditionally independent
	if
	\begin{align*}
		\P(X,Y|Z) = \P(X|Z)\P(Y|Z).
	\end{align*}
\end{definition}

\subsection{Bayesian networks}

\begin{definition}[R\&N p. 511]
	A \emph{Bayesian network} is a directed graph in which each node is
	annotated with quantitative probability information. The full sepcification
	is as follows:
	\begin{enumerate}
		\item Each node corresponds to a random variable, which may be discrete or cointinuous.
		\item A set of directed links or arrows connects pairs of nodes. If there is an arrow
		      from node $X$ to node $Y$, $X$ is said to be a parent of $Y$. The graph has no directed cycles.
		\item Each node $X_i$ has a conditional probability distribution $\P(X_i|Parents(X_i))$ that
		      quantifies the effect of the parents on the node.
	\end{enumerate}
\end{definition}

\begin{proposition}[R\&N p. 513]
	Let $x_1, ..., x_n$ be values of the random variables $X_1,...,X_n$. Then
	\begin{align*}
		P(x_1, ..., x_n) = \prod_{i=1}^n P(x_i | \textsf{parents}(X_i))
	\end{align*}
	where $\textsf{parents}(X_i)$ denotes the values of $\textsf{Parents}(X_i)$ that
	appear in $x_1, ..., x_n$.
\end{proposition}

\begin{definition}[R\&N p. 518]
	The \emph{noisy-OR} relation is a generalisation of the logical OR. Let $X_i$ be a boolean
	random variable with parents $X_1, ..., X_k$. Then
	\begin{align*}
		P(x_i|X_1, ..., X_k) = 1 - \prod_{\{j:X_j = true\}} q_j
	\end{align*}
	where $q_1,...,q_k$ are the \emph{inhibition probabilities} for each parent
	\begin{align*}
		q_j = P(\neg x_i | \neg x_1, ..., \neg x_{j-1}, x_j, \neg x_{j+1}, ..., \neg x_k).
	\end{align*}
\end{definition}

\begin{definition}[R\&N p. 522]
	Let $X$ be a random variable called the \emph{query variable}, let $\vec E=\{E_1,...,E_m\}$
	be a set of \emph{evidence variables}, let $\vec e$ be a particular observed event and let
	$\vec Y=\{Y_1, ..., Y_l\}$ denote the set of nonevidence, nonquery variables (also
	called the \emph{hidden variables}). Then the complete set of variables is
	$\vec X = \{X\}\cup \vec E \cup \vec Y$.\\
	A \emph{probabilistic inference system} then computes the posterior probability distribution
	$\P(X|\vec e)$.
\end{definition}

\begin{theorem}[R\&N p. 523]
	Any query can be answered using a Bayesian network by computing sums of products of
	conditional probabilities from the network.
\end{theorem}

\begin{theorem}[Complexity of exact inference, R\&N p. 528]
	The time and space complexity of exact inference in \emph{polytrees}, i.e. networks where there exists at most one
	undirected path between any two nodes, is linar in the size of the network.

	In general, inference in Bayesian networks is NP-hard.
\end{theorem}

\begin{definition}
	A \emph{consistent} estimate $\hat p_N$ of a probability $p$ becomes exact in the large sample limit. I.e.
	\begin{align*}
		\lim_{N\to\infty} \hat p_N = p
	\end{align*}
	where $N$ is the sample size used to calculate $\hat p_N$.
\end{definition}

\paragraph{Rejection sampling} (R\&N p. 532) is a general method of producing samples from a hard-to-sample
distribution given an easy-to-sample distribution. It involves generating samples from the
prior distribution specified by the network and then removing all the samples that do not
match the evidence. The estimate is obtained by counting how often $X=x$ occurs in the remaining
samples.

\paragraph{Likelihood weighting} fixes the values for the evidence variables and samples
only the nonevidence variables. Before tallying the counts in the distribution for the query
variable, each even is weighted by the product of the conditional probabilities for each evidence
variable, given its parents.

\paragraph{Gibbs sampling} (R\&N p. 536) starts with an arbitrary state and generates a next state by randomly
sampling a value for one of the nonevidence variables $X_i$. The sampling of $X_i$ is done
conditioned on the current values of the variables in the Markov blanket of $X_i$.

\section{Probabilisitc reasoning over time}

\subsection{Time and uncertainty}

\begin{definition}[R\&N p. 568]
	A \emph{stationary process} is a process of change that is governed by laws
	that do not themselves change over time.
\end{definition}

\begin{definition}
	The \emph{Markov assumption} states that the current
	state depends on only a finite fixed number of previous states.
\end{definition}

\subsection{Inference in temporal models}

\begin{proposition}[R\&N p. 570]
	The basic inference tasks that must be solved are
	\begin{enumerate}
		\item \emph{Filtering}: Computing the belief state given all the evidence to date.
		      I.e. \[\P(\vec X_t | \vec e_{1:t}) \text{ for some } t>0.\]
		\item \emph{Prediction}: Computing the posterior distribution over the future
		      state given all evidence to date. I.e. \[\P(\vec X_{t+k}|\vec e_{1:t})\text{ for some }t,k>0.\]
		\item \emph{Smoothing}: Computing the posterior distribution over a past state, given all evidence
		      up to the present. I.e. \[\P(\vec X_k | \vec e_{1:t}) \text{ for some } t>k>0.\]
		\item \emph{Most likely explanation}: Computing the sequence of states that is mostlikely to have
		      generated a sequence of observations. I.e. \[\P(\vec x_{1:t}|\vec e_{1:t}).\]
	\end{enumerate}
\end{proposition}


\begin{definition}
	\emph{Recursive estimation} is the process of computing the distribution $\P(\vec X_{t+1}|\vec e_{1:t+1})$
	given the new evidence $e_{t+1}$ and the previous distribution $\P(\vec X_t|\vec e_{1:t})$. In other words,
	it is given by a function $f$ such that
	\begin{align*}
		\P(\vec X_{t+1}|\vec e_{1:t+1})=f(\vec e_{t+1}, \P(\vec X_t|\vec e_{1:t})).
	\end{align*}
\end{definition}

\begin{theorem}[R\&N p. 572]
	By conditioning on the current state $\vec X_t$ we obtain the one-step prediction for the
	next state
	\begin{align*}
		\P(\vec X_{t+1}|\vec e_{1:t+1}) = \alpha \P(\vec e_{t+1}|\vec X_{t+1})\sum_{\vec x_t} \P(\vec X_{t+1}|\vec x_t)P(\vec x_t|e_{1:t}).
	\end{align*}
\end{theorem}

\begin{theorem}
	The time and space requirements for updating must be constant if
	an agent with limited memory is to keep track of the current state
	distribution over an unbounded sequence of observations.
\end{theorem}

\begin{theorem}[R\&N p. 574]
	Given a time $t>k>0$ and evidence $\vec e_{1:t}$ we can compute the
	past state $\vec X_k$ as follows
	\begin{align*}
		\P(\vec X_k|\vec e_{1:t}) = \alpha\P(\vec X_k|\vec e_{1:k})
		\P(\vec e_{k+1:t} | \vec X_k) =: \alpha\vec f_{1:k}\times\vec b_{k+1:t}
	\end{align*}
	where $\times$ is pointwise multiplication and $\vec b_{k+1:t}$ can be
	computed recursively with
	\begin{align*}
		\P(\vec e_{k+1:t}|\vec X_k) = \sum_{x_{k+1}}
		P(\vec e_{k+1}|\vec x_{k+1})P(\vec e_{k+2:t})\P(\vec x_{k+1}|\vec X_k).
	\end{align*}
\end{theorem}

\subsection{Matrix algorithms}

\begin{theorem}[R\&N p. 579]
	Let the state variable $X_t$ have values denoted by
	integers $1,...,S$. The transition model $\P(X_t|X_{t-1})$
	becomes an $S\times S$ matrix $\vec T$ where
	\begin{align*}
		T_{ij} = P(X_t=j|X_{t-1}=i).
	\end{align*}
	Let $\vec O_t$ be an $S\times S$ diagonal matrix whose $i$th
	diagonal entry is $P(e_t|X_t=i)$. Then
	\begin{align*}
		\vec f_{1:t+1} = \alpha \vec O_{t+1}\vec T^\top \vec f_{1:t}
		\hs\text{and}\hs
		\vec b_{k+1:t} = \vec T\vec O_{k+1}\vec b_{k+2:t}.
	\end{align*}
\end{theorem}

\subsection{Dyanamic Bayesian networks}

\begin{definition}[R\&N p. 591]
	To construct a \emph{dynamic Bayesian network} three kinds of information
	are required:
	\begin{enumerate}
		\item $\P(\vec X_0)$, the prior distribution over the state variables,
		\item $\P(\vec X_{t+1}|\vec X_t)$, the transition model,
		\item $\P(\vec E_t | \vec X_t)$, the sensor model.
	\end{enumerate}
	Since the transition and sensor model are assumed to be stationary they only
	need to be specified for the first slice.
\end{definition}

\begin{definition}[R\&N p. 593]
	The \emph{transient failure model} handles transient failures, i.e. occasional invalid
	values, of a sensor $S$ by assigning a non-zero probability for receiving an invalid
	value at each point in time, regardless of the evidence.
\end{definition}

\begin{definition}
	The \emph{persistent failure model} handles persistent failures, i.e. consistent invalid
	outputs from some point in time onwards, of a sensor $S$ by introducing
	a variable $B_{S,t}$ that indicates whether $S$ is in a failure state. Between each pair
	$B_{S,t}, B_{S,t+1}$ there exists a \emph{persistence arc} that specifies that the sensor can
	fail with a small probability at each time step but once  it has failed it will never recover.
\end{definition}

\begin{theorem}[R\&N p. 595]
	Inference in dynamic Bayesian networks at some point in time $t$ can be achieved by unrolling
	in to the finite Bayesian network up to and including $t$ and applying standard Bayesian
	inference in $O(t)$ time.
\end{theorem}

\begin{definition}[R\&N p. 597]
	\emph{Particle filtering} describes a family of algorithms that are designed
	for approximate inference in dynamic Bayesian networks. Initially, a population of
	$N$ initial-state samples is created by sampling from $\P(\vec X_0)$. Then at each
	step the following update cycle is repeated:
	\begin{enumerate}
		\item Each sample is propagated forward by sampling the next state value $\vec x_{t+1}$
		      given the current state value $\vec x_t$ based on the transition model $\P(\vec X_{t+1}|\vec x_t)$.
		\item Each sample is weighted by the likelihood it assigns to the new evidence, $P(\vec e_{t+1}|\vec x_{t+1})$.
		\item The population is resampled to generate a new population of $N$ samples. Each new sample is
		      selected from the current population; the probability that a particular sample is selected is proportional
		      to its weight. The new samples are unweighted.
	\end{enumerate}
\end{definition}

\begin{theorem}[R\&N p. 599]
	Particle filtering is consistent.
\end{theorem}

\section{Making simple decisions}

\subsection{Basics of utility theory}

\begin{definition}[R\&N p. 611]
	A \emph{utility function} $U:S\to\R$ assigns each state $s\in S$ a desirability
	$U(s)$. The \emph{expected utility} of an action given the evidence, $EU(a|\vec e)$,
	is the weighted average of the outcomes
	\begin{align*}
		EU(a|\vec e)=\sum_{s'} P(\textsf{RESULT}(a)=s'|a,\vec e)U(s').
	\end{align*}
	The principle of \emph{maximum expected utility} states that a rational agent
	should choose the action that maximised the agent's expected utility.
\end{definition}

\begin{definition}[R\&N p. 612]
	A lottery $L$ with possible outcomes $S_1,...,S_n$ that occur with probabilities
	$p_1,...,p_n$ is written
	\begin{align*}
		L=[p_1,S_1;\:p_2,S_2;\:... p_n,S_n].
	\end{align*}
\end{definition}

\begin{definition}[Axioms of Utility Theory]
	Any reasonable preference relation should obey the following constraints
	\begin{enumerate}
		\item \emph{Orderability}: Exactly one of $A\prec B$, $A\succ B$, or $A\sim B$ holds.
		\item \emph{Transitivity}: $(A \succ B)\wedge(B\succ C) \Rightarrow A\succ C$.
		\item \emph{Continuity}: $A\succ B\succ C\Rightarrow \exists p.\: [p,A;\:1-p,C]\sim B$.
		\item \emph{Substitutability}: $A\sim B\Rightarrow [p,A;\:1-p,C]\sim [p,A;\:1-p,C]$.
		\item \emph{Monotonicity}: $A\succ B\Rightarrow (p>q\Leftrightarrow [p,A;\:1-p,B]\succ[q,A;\:1-q,B])$.
		\item \emph{Decomposability}: $[p,A;\:1-p,[q,B;\:1-q,C]]\sim[p,A;\:(1-p)q,B;\:(1-p)(1-q),C]$.
	\end{enumerate}
\end{definition}

\begin{theorem}[R\&N p. 613]
	If an agent's preferences obey the axioms of utility, then there
	exists a utility function $U$ such that for any two outcomes $A$
	and $B$
	\begin{align*}
		U(A) & >U(B)\Leftrightarrow A\succ B, \\
		U(A) & =U(B)\Leftrightarrow A\sim B.
	\end{align*}
\end{theorem}

\begin{theorem}
	The utility of a lottery is given by
	\begin{align*}
		U([p_1,S_1;...;p_n,S_n]) = \sum_i p_i U(S_i).
	\end{align*}
\end{theorem}

\subsection{Utility functions}

\begin{definition}[R\&N p. 615]
	The values of a utility functions can be fixed by $U(S)=u_\top$ for the
	best possible outcome and $U(S)=u_\bot$ for the worst possible outcome.
	If a utility function is \emph{normalised} it uses $u_\top = 1$ and
	$u_\bot = 0$.
\end{definition}

\subsection{Decision networks}

\begin{definition}[R\&N p. 626]
	A \emph{decision network} uses three types of nodes:
	\begin{itemize}
		\item \emph{Chance nodes} (ovals) represent random variables, like in Bayesian networks.
		\item \emph{Decision nodes} (rectangles) represent points where the decision maker
		      has a choice of actions.
		\item \emph{Utility nodes} (diamonds) represent the agent's utility function.
	\end{itemize}
\end{definition}

\begin{theorem}[R\&N p. 628]
	The algorithm for evaluating decision networks is the following:
	\begin{enumerate}
		\item Set the evidence variables for the current state.
		\item For each possible value of the decision node: \begin{enumerate}
			      \item Set the decision node to that value.
			      \item Calculate the posterior probabilities for the parent nodes of
			            the utility node, using a standard probabilistic inference algorithm.
			      \item Calculate the resulting utility for the action.
		      \end{enumerate}
		\item Return the action with the highest utility.
	\end{enumerate}
\end{theorem}


\section{Making complex decisions}


\subsection{Sequential decision problems}

\begin{definition}[MDP and policy, R\&N p. 647]
	A sequential decision problem for a fully observable, stochastic environment
	with a Markovian transition model and additive rewards is called a \emph{Markovian
		decision problem} and consists of a set of states (with an initial state $s_0$),
	a set $\textsf{ACTIONS}(s)$ of
	actions in each state $s$, a transition model $P(s'|s,a)$, and a reward function
	$R(s)$.\\
	A solution for such a problem must specify the agent's behaviour for any state
	and is called a \emph{policy}.
\end{definition}

\begin{definition}[R\&N p. 648]
	Let $U_h$ be a utility function on environment histories. Then, if there exists
	a fixed time $N$ such that for all $k\in\N$
	\begin{align*}
		U([s_0,s_1,...,s_{N+k}]) = U([s_0,s_1,...,s_N])
	\end{align*}
	we say there is a \emph{finite horizon}. Otherwise there is a \emph{infinite
		horizon}.
\end{definition}

\begin{theorem}
	With a finite horizon the optimal action in a given state can change over time.
\end{theorem}

\begin{definition}[R\&N p. 649]
	An agent's preferences are stationary if for two state sequences
	$[s_0, s_1, s_2, ...]$ and $[s_0', s_1', s_2'...]$ with $s_0=s_0'$ we have
	\begin{align*}
		[s_0, s_1, s_2, ...] \prec [s_0', s_1', s_2'...]\hs & \Leftrightarrow\hs
		[s_1, s_2, ...] \prec [s_1', s_2', ...],                                 \\
		[s_0, s_1, s_2, ...] \sim [s_0', s_1', s_2'...]\hs  & \Leftrightarrow\hs
		[s_1, s_2, ...] \sim [s_1', s_2', ...].
	\end{align*}
\end{definition}

\begin{theorem}
	Let an agent's preferences be stationary. Then there are just two coherent
	ways to assign utilities to a sequence of states $[s_0, s_1, s_2, ...]$
	\begin{enumerate}
		\item \emph{Additive rewards}: The utility is \begin{align*}
			      U_h([s_0, s_1, s_2, ...]) = R(s_0) + R(s_1) + R(s_2) + \cdots.
		      \end{align*}
		\item \emph{Discounted rewards}: For some $\gamma\in(0,1)$ the utility is \begin{align*}
			      U_h([s_0, s_1, s_2, ...]) = R(s_0) + \gamma R(s_1) + \gamma^2 R(s_2) + \cdots.
		      \end{align*}
	\end{enumerate}
\end{theorem}

\begin{theorem}[R\&N p. 650]
	The utility of an infinite sequence $[s_0, s_1, s_2, ...]$ with discounted
	rewards and a reward function bounded by $\pm \hat R$ is bounded by
	\begin{align*}
		U_h([s_0, s_1, s_2, ...]) = \sum_{t=0}^\infty \gamma^t R(s_t) \leq \frac{\hat R}{1-\gamma}.
	\end{align*}
\end{theorem}

\begin{definition}
	A policy that is guaranteed to reach a terminal state is called \emph{proper}.
\end{definition}

\begin{definition}
	The expected utility obtained by executing a policy $\pi$ starting in $s$ is given by
	\begin{align*}
		U^\pi(s) = E\left[\sum_{t=0}^\infty \gamma^t R(S_t)\right]
	\end{align*}
	where the expectation is with respect to the probability distribution over state
	sequences determined by $s$ and $\pi$. The policy $\pi_s^*$ given by
	\begin{align*}
		\pi_s^*=\argmax\limits_\pi U^\pi(s)
	\end{align*}
	is the one that has higher expected utilities than all the others.
\end{definition}

\begin{corollary}
	The policy $\pi_s^*$ is optimal when $s$ is the starting state.
\end{corollary}

\begin{theorem}
	Consider an environment with a state space $S$, an infinite horizon and discounted utilities. The
	policy $\pi_s^*$ is optimal for some $s\in S$ is optimal for all $s'\in S$.
\end{theorem}

\subsection{Value iteration}

\begin{theorem}[Bellman equation, R\&N p. 652]
	The utility of a state is given by
	\begin{align*}
		U(s) = R(s) + \gamma \max_{a\in A(s)}\sum_{s'} P(s'|s,a)U(s').
	\end{align*}
\end{theorem}

\begin{theorem}
	The value iteration algorithm given by the Bellman update
	\begin{align*}
		U_{i+1}(s) \leftarrow R(s) + \gamma\max_{a\in A(s)} \sum_{s'} P(s'|s,a)U_i(s')
	\end{align*}
	is guaranteed to reach an equilibrium after some finite number of steps
	and the resulting policy is optimal.
\end{theorem}

\subsection{Policy iteration}

\begin{theorem}[R\&N p. 656]
	The \emph{policy iteration} algorithm alternates the following two steps beginning
	from some initial policy $\pi_0$:
	\begin{itemize}
		\item \emph{Policy evaluation}: given a policy $\pi_i$, calculate $U_i=U^{\pi_i}$,
		      the utility of each state if $\pi_i$ were to be executed.
		\item \emph{Policy improvement}: Calculate a new policy $\pi_{i+1}$, using one step
		      look-ahead based on $U_i$ as in
		      \begin{align*}
			      \pi^*(s) = \argmax_{a\in A(s)}\sum_{s'} P(s'|s,a)U(s')
		      \end{align*}
	\end{itemize}
\end{theorem}

\end{document}