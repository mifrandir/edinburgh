\documentclass{article}
\usepackage{preamble}

\title{FDS CW2: Critical Evaluation of a Data Science Study}
\author{Franz Miltz}
\begin{document}
\maketitle

\section*{Chosen study}

\begin{itemize}
	\item Paper: Menni C, Valdes, AM, Freidin MB. et al. (2020).
	      \emph{\href{https://www.nature.com/articles/s41591-020-0916-2?fbclid=IwAR0tkHR2rBZ9vPGhoc8pm7ZreHZ0HdkVOzLGU2mxvYhgZ2n9DkWDZLRfys4}
		      {Real-time tracking of self-reported symptoms to predict potential COVID-19}} Nat Med26, 1037–1040. doi: 10.1038/s41591-020-0916-2
	\item Article: \emph{\href{https://www.bbc.co.uk/news/health-52770313}{Coronavirus: Five things a Covid-19 symptom-tracking app tells us}} (BBC website, 23 May 2020)
\end{itemize}


\section{Paper-related questions}


\subsection{What is the scientific goal of the study?}

The aim of the study is to analyse the correlation between COVID-19 symptoms and test results
to decide which symptoms, or combinations thereof, are most indicative of a positive case.

\subsection{What is the type of study?}

The study is observational since no experiments have been conducted. More precisely, we are
dealing with a cross-sectional study because the researchers are trying to gather information
from the entire population. The study is trying to establish a prediction model.

\subsection{What are the hypotheses of the study?}

The main hypothesis the study is trying to prove is that out of the COVID-19 symptoms that have
so far been observed, some have more predictive value than others. In particular, loss of smell
and taste is hypothesised to be caused by COVID-19. Another, less obvious, hypothesis is
that the main stream media reports have affected awareness of loss of taste and
smell as a COVID-19 symptom.

\subsection{What is the methodology of the study and what is the result?}

The data was gathered through an app where participants self-reported their symptoms and other
health information. This has been done in two separate regions, the United Kingdom and the United
States, and for about a month from 24 March to 21 April 2020.
The app was, and still is, freely available so anyone could document their symptoms.
The symptoms that were tracked are loss of smell
and taste, fatigue, shortness of breath, fever, persistent cough, diarrhea, delirium, skipped meals,
abdominal pain, chest pain and hoarse voice. Additionally, data on demographics,
test results, hospitalisations and more was recorded.\\
\indent The data was then analysed using various methods (cf. 1.5) to create a linear
prediction model. The resulting model suggests that loss of smell and taste is, in
addition to other symptoms like high temperature and persistent cough, a
potential predictor of COVID-19. Furthermore, to examine the influence of the main
stream media on the reports of loss of smell and taste, the researchers compared
different time intervals and found that, in the UK, the awareness of loss of smell
and taste as a symptom of COVID-19 has increased over the span of the study.

\subsection{What are the statistical methods used in the study and how are they applied to the data?}

Firstly, the researchers calculated basic characteristics of the data, like the percentages
of each symptom within the entire population and tested people with either result. Further,
they used odds ratios to characterise the relation between positive tests and loss of
smell and taste in particular. The corresponding confidence intervals were determined by
using a $P$-value of $0.01\%$.\\
\indent To find out more about the association of certain symptoms with positive tests, logistic
regression was used. In order to increase the validity,
they first adjusted for values like age, sex and BMI. Further, step-wise logistic regression
with 80\% training data was applied to generate
a linear prediction model. The set of variables to include was
chosen based on the Akaike information criterion of the resulting model. The researchers also
provided sensitivities, specificities and areas under the curve of the receiver operating
characteristic curve (ROC-AUC) for different models and the two cohorts (UK and US).

\subsection{What are the stated conclusions of the study?}

The study concludes that loss of smell and taste is a potential predictor of COVID-19
and that awareness of loss of smell and taste as a symptom has increased due to
an increase in media reports. To be precise, the paper provides a linear model which includes age,
sex, loss of smell and taste, persistent cough, severe fatigue and skipped meals to
predict COVID-19 test results. The researchers claim that said model is similarly accurate
accross different sex and age groups. However, it is acknowledged that the resulting
values may overestimate the likelihood of a positive test result due to the non-representative
group of participants.

\subsection{Impact, errors \& potential flaws}

As highlighted by the researchers who conducted the study, the actual values
obtained are unlikely to be entirely accurate for the whole population. This is
due to the self-selected group of participants who are unlikely to be representative
of the population in general. Further, it is rightfully pointed out that
some symptoms seem to be more indicative of COVID-19 in one region than the other.
This begs the question which other factors may be involved and how this affects
the general validity of the study. This question was not adressed in the paper.\\
\indent Since the flaws presented above are unlikely to change the central conclusion,
i.e. that loss of smell and taste is indeed a predictor for COVID-19, the
conducted research work is still very impactful. The World Health Organization
is urged to add the symptom to their list. This could significantly increase
awareness and make diagnosis more accurate.

\subsection{Are there any ehtical implications of the study? How well do the authors relate these implications?}

The main, if not only, sources of ethical implications for this study seem to be acquisition and
usage of data. Without further investigation into the terms and conditions
of the app and their communication to the participants, it seems difficult
to fully assess the ethicality of the study. Since participantion was completely
voluntary, all the data was directly entered by the users and the application
used to collect the data does not seem to have a purpose other than data collection,
I would argue that there are no significant ethical concerns in this regard.
This is supported by the approval by the KCL ethics Committee as well as the
Partners Human Research Committee.

\section{Article-related questions}

\subsection{Brief summary}

The report primarily focusses on the predictive abilities of the model in the paper and
the values it generates over the course of the study compared to actual recorded numbers.
A data scientist gives a basic overview of how the figures were obtained and how accurate
they are expected to be. Finally, the report mentions that a wide range of symptoms
are associated with COVID-19, but crucially loss of smell and taste has been shown to
have the strongest corrolation with an infection.

\subsection{How accurately did the report summarise the study?}

It is important to point out that the goal of a media report is to provide
a summary of the study and its conclusions, focussing on the information that
is relevant to the audience and presenting it in a way that is appropriate for the
average reader. From this perspective, the report does a very good job at presenting
the methods, the conclusions and the shortcomings of the study and the resulting model.
It includes the relevant numbers and helpful visualisations to put them into perspective.
A more detailed explanation of how the predictions were made or even how the model itself
works were left out.\\
\indent The most noticable difference between the paper and the report are the focal points.
While the paper primarily focuses on the predictive value of the symptoms,
from the hypotheses to the conclusions, the media report chooses to present
various collected as well as predicted values. Throughout the paper, individual symptoms, especially
loss of smell and taste, are mentioned often whereas in the report only the very last
section deals with this. This is not so much a factual inaccuracy as it is a
change in tone and priority.


\end{document}