\documentclass{article}
\usepackage{homework-preamble}

\begin{document}

\title{SVCDE: Hand-in 9}
\author{Franz Miltz}
\date{22 November 2020}
\maketitle


\section*{Question 1}


Consider the ODE
\begin{align}
	\label{ode1}
	2(y+4)+3(2x+1)y' = 0.
\end{align}


\subsection*{(a)}

\begin{claim*}
	The ODE (\ref{ode1}) is not exact.
\end{claim*}
\begin{proof}
	Assume the equation is exact. Then, as proven in \emph{Section 6.5} of the notes,
	\begin{align}
		\label{condex}
		\frac{\p M}{\p y} = \frac{\p N}{\p x}.
	\end{align}
	We find
	\begin{align*}
		M(x,y) = 2y+8,\hs N(x,y)=6x+3.
	\end{align*}
	and therefore
	\begin{align*}
		\frac{\p M}{\p y} = 2 \not= 6 = \frac{\p N}{\p x}.
	\end{align*}
	Since (\ref{condex}) does not hold, (\ref{ode1}) cannot be exact.
\end{proof}


\subsection*{(b)}


\begin{claim*}
	There exists an integrating factor $\mu:\R\to\R$ such that the ODE
	\begin{align}
		\label{ode2}
		2\mu(y+4)+3\mu(2x+1)y'=0
	\end{align}
	is exact.
\end{claim*}

\begin{proof}
	Consider $\mu:\R\to\R$ where
	\begin{align*}
		\mu(x) = (6x+3)^{-2/3}.
	\end{align*}
	From \emph{Section 6.5.1} of the notes, we know that (\ref{ode2}) is exact
	if and only if
	\begin{align}
		\label{condint}
		M\mu_y-N\mu_x + (M_y-N_x)\mu = 0.
	\end{align}
	By inserting all the functions, we find
	\begin{align*}
		4(6x+3)(6x+3)^{-5/3}+(2-6)(6x+3)^{-2/3}         & = 0 \\
		\Leftrightarrow 4(6x+3)^{-2/3} - 4(6x+3)^{-2/3} & =0  \\
		\Leftrightarrow 0                               & = 0
	\end{align*}
	Thus the condition (\ref{condint}) holds and $\mu$ is an integrating factor
	of (\ref{ode1}).
\end{proof}


\subsection*{(c)}


To solve the ODE (\ref{ode1}) we multiply by $\mu$ to obtain
\begin{align*}
	2(6x+3)^{-2/3}(y+4)+(6x+3)^{1/3}y'=0.
\end{align*}
We therefore have
\begin{align*}
	\mu M & = 2(6x+3)^{-2/3}(y+4), \\
	\mu N & = (6x+3)^{1/3}.
\end{align*}
We want to find $\psi(x,y)$ such that
\begin{align*}
	\frac{\p\psi}{\p x} = \mu M \hs\text{and}\hs \frac{\p\psi}{\p y}=\mu N.
\end{align*}
By integrating, we obtain
\begin{align*}
	\int \mu M\:dx & = \int 2(6x+3)^{-2/3}(y+4)\: dx                     \\
	               & = 2(y+4) \left(\frac{1}{2}(6x+3)^{1/3}\right) + C_1 \\
	               & = (y+4)(6x+3)^{1/3} + C_1
\end{align*}
and
\begin{align*}
	\int \mu N\:dy & = \int (6x+3)^{1/3}\:dy = (6x+3)^{1/3}y + C_2.
\end{align*}
By inspection we find a suitable function
\begin{align*}
	\psi(x,y) = (y+4)(6x+3)^{1/3}.
\end{align*}
This yields the implicit solution
\begin{align*}
	(y+4)(6x+3)^{1/3} = C
\end{align*}
and thus the explicit form
\begin{align*}
	y(x) = C(6x+3)^{-1/3}-4.
\end{align*}


\section*{Question 2}


Consider the second order linear homogeneous ODE with constant coefficients
\begin{align*}
	4(y'' + \pi y') + \pi^2 y = 0.
\end{align*}
We find the characteristic equation
\begin{align*}
	4r^2 + 4\pi r + \pi^2      & = 0  \\
	\Leftrightarrow (2r+\pi)^2 & = 0.
\end{align*}
Thus there exists a repeated root $r=-\pi/2$. As shown in \emph{Section 7.2}
of the notes, the general solution is
\begin{align*}
	y = (C_1 + tC_2)\exp\left(-\frac{\pi}{2}t\right).
\end{align*}
With
\begin{align*}
	y(1)=1\hs\text{and}\hs y'(1)=1
\end{align*}
we find the equations
\begin{align*}
	1 & = e^{-\pi/2}(C_1 + C_2)                                                       \\
	1 & = e^{-\pi/2}\left(-\frac{\pi}{2}C_1 + C_2\left(1-\frac{\pi}{2}\right)\right).
\end{align*}
By solving this system of linear equations, we get
\begin{align*}
	C_1 & = -\frac{\pi}{2}e^{\pi/2},                \\
	C_2 & = \left(1+\frac{\pi}{2}\right) e^{\pi/2}.
\end{align*}
Therefore, the particular solution to the IVP is
\begin{align*}
	y & = (-\frac{\pi}{2}e^{\pi/2} + t\left(1+\frac{\pi}{2}\right) e^{\pi/2})e^{-\pi t/2} \\
	  & = \left(1-\frac{\pi(1-t)}{2}\right)\exp\left(\frac{\pi(1-t)}{2}\right)
\end{align*}
\end{document}