\documentclass{article}
\usepackage{homework-preamble}

\begin{document}

\title{SVCDE: Hand-in 8}
\author{Franz Miltz}
\date{15 November 2020}
\maketitle


\section*{Question 1}


We want to evaluate
\begin{align*}
	I = \di_S x^2y^2+y^2z^2+z^2x^2 dS
\end{align*}
where $S$ is the surface of the unit sphere around the origin.
By the definition of surface integrals, we know that
\begin{align*}
	I=\di_S \vec F \cdot \vec{\hat n}\:dS = \di_S \vec F \cdot d\vec S
\end{align*}
for some vector field $\vec F$. We know that
\begin{align*}
	\vec{\hat n} = x\ih + y\jh + z\kh
\end{align*}
by definition of $S$ (centered around the origin and radius $r=1$).
Let
\begin{align*}
	\vec F = xy^2\ih + yz^2\jh + zx^2\kh.
\end{align*}
Then
\begin{align*}
	\vec F \cdot \vec{\hat n} = x^2y^2+y^2z^2+z^2x^2
\end{align*}
and thus our $\vec F$ is valid.
We use the \emph{Divergence Theorem} to find
\begin{align*}
	I = \di_S \vec F \cdot d\vec S=\ti_E \grad\cdot\vec F\:dV
	= \ti_E x^2 + y^2 + z^2\:dV
\end{align*}
where $E$ is the unit sphere around the origin.
By using spherical coordinates $(\rho, \theta, \phi)$ and inserting
the limits we find
\begin{align*}
	I = \int_0^{2\pi}\int_0^\pi\int_0^1 r^4\sin\phi\:dr\,d\phi\,d\theta.
\end{align*}
This lets us evaluate $I$ with relative ease:
\begin{align*}
	I & = \int_0^{2\pi}\int_0^\pi \left[\frac{r^5}{5}\sin\phi\right]_0^1\:d\phi\,d\theta
	= \int_0^{2\pi}\int_0^\pi \frac{\sin\phi}{5}\:d\phi\,d\theta                         \\
	  & =\int_0^{2\pi} \left[-\frac{\cos\phi}{5}\right]_0^\pi\:d\theta
	= \int_0^{2\pi} \frac{2}{5}\:d\theta = \frac{4\pi}{5}.
\end{align*}

\section*{Question 2}


We want to find an implicit solution of
\begin{align*}
	\frac{dy}{dx}=\frac{x+y}{x-y}.
\end{align*}
Firstly, observe that
\begin{align}
	\label{dydx}
	\frac{dy}{dx}=\frac{1+\frac{y}{x}}{1-\frac{y}{x}} = g\left(\frac{y}{x}\right)
\end{align}
where
\begin{align}
	\label{g}
	g(v) = \frac{1+v}{1-v}.
\end{align}
We let
\begin{align*}
	v(x)=\frac{y}{x}.
\end{align*}
Thus $y = v(x)x$ and via the \emph{Product Rule}
\begin{align*}
	\frac{dy}{dx}=v+x\frac{dv}{dx}.
\end{align*}
By inserting (\ref{dydx}) and (\ref{g}) we get
\begin{align*}
	v+x\frac{dv}{dx}               & = \frac{1+v}{1-v}   \\
	\Leftrightarrow x\frac{dv}{dx} & = \frac{1+v^2}{1-v}
\end{align*}
Separating the variables leads us to
\begin{align*}
	\frac{1-v}{1+v^2}\frac{dv}{dx} = \frac{1}{x}.
\end{align*}
Integrating both sides w.r.t. $x$ gives
\begin{align*}
	\int \frac{1-v}{1+v^2}\:dv                                          & = \int \frac{1}{x}\: dx \\
	\Leftrightarrow \int \frac{1}{1+v^2}\:dv - \int \frac{v}{1+v^2}\:dv & = \int \frac{1}{x}\:dx  \\
	\Leftrightarrow \arctan(v)-\frac{1}{2}\ln(1+v^2)                    & =\ln(x)+C
\end{align*}
By resubstituting $v$ and simplifying, we get the implicit expression
\begin{align*}
	\arctan\left(\frac{y}{x}\right)-\frac{1}{2}\ln\left(x^2+y^2\right)=C.
\end{align*}
\end{document}