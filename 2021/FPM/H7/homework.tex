\documentclass{article}
\usepackage{homework-preamble}

\begin{document}
\title{FPM: Hand-in 7}
\author{Franz Miltz (UUN: S1971811)}
\date{15 March 2021}
\maketitle
\mkthms
\section*{Problem 1}

The actions are
\begin{enumerate}
	\item left action,
	\item right action,
	\item conjugate action,
	\item the symmetries acting on the vertices of the graph,
	\item acting on the cosets.
\end{enumerate}

\subsection*{Left, right and conjugate action}

Let $(G, *)$ be a group. Then let $g\in G$ and $x\in X = G$.
The left, right and conjugate actions of $G$ on itself are defined as follows:

\begin{itemize}
	\item left action: $\alpha_l: (g,x) \mapsto g * x$,
	\item right action: $\alpha_r: (g,x) \mapsto x * \inv g$,
	\item conjugate action: $\alpha_c: (g,x) \mapsto g * x * \inv g$.
\end{itemize}

\noindent These actions are well-defined for $D_5$.

\subsection*{Symmetries acting on the vertices}

$D_5$ is the set of symmetries of a pentagon. If we think of the pentagon as
a graph, all the element $g\in D_5$ are graph isomorphisms. Since every
graph isomorphism for a graph with vertices $V$ is a map $\phi:V\to V$,
we can define the action $\beta$ of $D_5$ on $V$ as follows:
\begin{align*}
	\beta: D_5\times V\to V,(\phi, v)\mapsto \phi(v).
\end{align*}

\subsection*{Acting on the cosets}

Let $h\in D_5$ be the reflection through one of the vertices in the pentagon
and let $g\in D_5$ be the rotation by one fifth of a turn.
Then we have the proper subgroup $H=\{e, h\}<D_5$. Now consider the set of
cosets
\begin{align*}
	D_5/H=\{H, gH, g^2H, g^3H, g^4H\}.
\end{align*}
We define the group action
\begin{align*}
	\gamma: D_5\times D_5/H\to D_5/H,(a, bH)\mapsto abH.
\end{align*}
This also works for any other subset $H\leq D_5$.

\section*{Problem 2}

\begin{claim*}
	Let $G$ be the group of rotational symmetries of the octahedron. Then
	$|G|=24$.
\end{claim*}

\begin{proof}
	Let $G$, $X$ and $t\in X$ be as described. Then, since all the vertices
	have identical valency, we have
	\begin{align*}
		\orb_G(t) = X
	\end{align*}
	which implies $\abs{\orb_g(t)} = 6$. Further, the only rotations that
	do not affect $t$ are the identity and the ones that have an axis that
	goes through $t$. The only such axis goes through $t$ and the vertex on
	the bottom where faces $E$, $F$, $G$ and $H$ meet. Let the rotation
	around this axis by one fourth of a turn clockwise (i.e. the one that
	maps the face $A$ onto $B$) be called $g$. Then we have
	\begin{align*}
		\stab_G(t) = \{e, g, g^2, g^3\}
	\end{align*}
	and thus $\abs{\stab_G(t)} = 4$. By the \emph{Orbit-Stabiliser Theorem}
	we have
	\begin{align*}
		\abs{G} = \abs{\orb_G(t)}\cdot \abs{\stab_G(t)}
	\end{align*}
	and thus
	\begin{align*}
		\abs{G} = 6 \cdot 4 = 24.
	\end{align*}
\end{proof}

\end{document}