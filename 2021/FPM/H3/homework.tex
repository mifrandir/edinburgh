\documentclass{article}
\usepackage{homework-preamble}

\begin{document}
\title{FPM: Hand-in 3}
\author{Franz Miltz (UUN: S1971811)}
\date{8 February 2021}
\maketitle

\section*{Problem 1}

\begin{claim*}
	Let $G$ be a group such that $|G|>1$ is not prime. Then
	there exists a proper nontrivial subgroup $H<G$.
\end{claim*}

\begin{proof}
	Suppose $G$ is not cyclic. Then, by \emph{Lemma 2.2.14} in the notes,
	for all $g\in G$ we have $o(g)<\abs G$. Since $|G|>1$ there exists
	a $g\in G$ such that $g\not=e$ which implies $o(g)>1$.
	Let $H=\lra g$ for one such element. Then
	\begin{align*}
		\abs H = \abs{\lra g} = o(g) \hs\text{which implies}\hs 1 < \abs H < \abs G
	\end{align*}
	which makes $H$ a subgroup of $G$ with
	the desired characteristics.\\\\
	Now suppose $G$ is cyclic. Then let $g\in G$ be such that
	$G=\lra g$. Let $n=\abs{G}$. Then, since $n>1$ and $n$ is not prime,
	there exist integers $k,l>1$ such that $n=kl$.
	Now consider the cyclic subgroup $H=\lra{g^k}\leq G$. We have
	\begin{align*}
		H =\{e, g^k, g^{2k}, ..., g^{(l-1)k}, g^{lk}, g^{(l+1)k}, ...\}.
	\end{align*}
	Observe that
	\begin{align*}
		\left(g^k\right)^l = g^n = e
	\end{align*}
	because $o(g)=n$ by \emph{Lemma 2.2.14} in the notes (and the proof
	thereof). Therefore
	\begin{align*}
		g^n = e,\hs g^{(l+1)k} = eg^k = g^k, \hs g^{(l+2)k}=eg^{2k}=g^{2k},\hs \cdots
	\end{align*}
	More explicitly
	\begin{align*}
		\forall a \geq l,\: \exists b<l,\: g^a = g^b
	\end{align*}
	where $a,b\in\Z_{\geq 0}$ and $g^0 = e$. Therefore
	\begin{align*}
		H = \{e, g^k, g^{2k}, ..., g^{(l-1)k}\}
	\end{align*}
	and thus
	\begin{align*}
		\abs H = l \hs\text{which implies}\hs 1 < \abs H < |G|
	\end{align*}
	which makes $H$ a subgroup with the desired characteristics.
\end{proof}

\section*{Problem 2}

Let $G=S_3$ and $H=\lra{(1\: 2)}$.

\subsection*{(a)}

Then the left cosets of $H$ are
\begin{align*}
	\{e, (1\:2)\},\hs \{(1\:3), (1\:2\:3)\},\hs \{(2\:3), (1\:3\:2)\}.
\end{align*}
Similarly, the right cosets of $H$ are
\begin{align*}
	\{e, (1\:2)\},\hs \{(1\:3), (1\:3\:2)\},\hs \{(2\:3),(1\:2\:3)\}.
\end{align*}

\subsection*{(b)}

\begin{claim*}
	$G$ is not normal.
\end{claim*}

\begin{proof}
	Let $g=(1\:3)\in G$. Then
	\begin{align*}
		gH=\{(1\:3), (1\:2\:3)\}\hs\text{and}\hs Hg=\{(1\:3), (1\:3\:2)\}.
	\end{align*}
	Thus $gH\not=Hg$ and therefore $G$ cannot be normal.
\end{proof}

\end{document}