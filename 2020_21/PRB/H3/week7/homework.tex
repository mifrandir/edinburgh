\documentclass{article}
\usepackage[a4paper]{geometry}
\usepackage[british]{babel}
\usepackage{amsmath}
\usepackage{amssymb}
\usepackage{mathtools}
\usepackage{nicefrac}
\usepackage{amsthm}
\usepackage{changepage}
\geometry{tmargin=2cm, bmargin=3cm}
\DeclarePairedDelimiter{\floor}{\lfloor}{\rfloor}
\newcommand*\lneg[1]{\overline{#1}}
\newcommand{\R}{\mathbb{R}}
\newcommand{\N}{\mathbb{N}}
\newcommand{\Z}{\mathbb{Z}}
\newcommand{\C}{\mathbb{C}}
\newcommand{\st}{\text{ s.t. }}
\renewcommand{\P}{\mathbb{P}}
%newenvironment{claim}[1]{\noindent\emph{Claim.}\space#1}{}
\newenvironment{claimproof}[1]{\par\noindent\emph{Proof.}\space#1}{\hfill $\blacksquare$}
\newtheorem{claim}[section]{Claim}
\newtheorem{lemma}{Lemma}[section]
\DeclareMathOperator{\lub}{\text{LUB}}
\DeclareMathOperator{\hcf}{hcf}
\DeclareMathOperator{\lcm}{lcm}
\setcounter{MaxMatrixCols}{20}
\newcommand{\E}{\mathbb{E}}
\newcommand{\V}{\mathbb{V}}
\DeclareMathOperator{\Geom}{Geom}
\DeclareMathOperator{\Binom}{Binom}
\DeclareMathOperator{\Bern}{Bern}
\DeclareMathOperator{\Unif}{Unif}
\DeclareMathOperator{\Exp}{Exp}
\DeclareMathOperator{\GD}{Gamma}
\newcommand*\binco[2]{\begin{pmatrix}
  #1\\#2
\end{pmatrix}}
\newcommand{\ih}{\widehat i}
\newcommand{\jh}{\widehat j}
\newcommand{\kh}{\widehat k}
\newcommand{\K}{\mathcal{K}}
\newcommand{\dv}[1]{\vec #1\, '}
\renewcommand{\d}[1]{#1'}
\begin{document}
\title{PRB: Week 7 Solutions}
\author{Franz Miltz (UNN: S1971811)}
\date{10 November, 2020}
\maketitle
\section*{P7.1}
Let $X_1,X_2,...$ be as described in the questions. We find 
\begin{align*}
  \E(X) &= \P(X_1^c)(\E(X)+1)\\
        &+\P(X_1\cap X_2^c)(\E(X)+2)\\
        &+\P(X_1\cap X_2\cap X_3^c)(\E(X)=3)\\
        &+3\P(X_1\cap X_2\cap X_3)
\end{align*}
With $\E(X)=s$ and $q=1-p$ we find
\begin{align}
  \label{eqs}
  s = q(s+1) + pq(s+2) + p^2q(s+3) +3p^3
\end{align}
because after any of the first three starts to the sequence the chances 
are the same as they were in the beginning.
We can rewrite equation (\ref{eqs}) to find
\begin{align*}
  s &= qs + q + pqs + 2pq + p^2qs + 3p^2q + 3p^3\\
    &= s(q+pq+p^2q)+q + 2pq + 3p^2q + 3p^3.\\
\end{align*}
This lets us write $s$ in terms of $p$ and $q$:
\begin{align*}
  s &= \frac{q+2pq+3p^2q+3p^3}{1-q-pq-p^2q} =\frac{1+p+p^2}{p^3}.
\end{align*}
Thus the expected number of trials up to and including the first occurrence of
three consecutive successes is
\begin{align*}
  \E(X) = \frac{1+p+p^2}{p^3}.
\end{align*}
\section*{P7.2}
Observe that the the $i$th step $S_i=x_i-x_{i-1}$ for $i\in\N$ is either $-1$ or
$1$ with equal probability. Therefore we can define $S_i$ in terms of the
random variable $W_i\sim\Bern(1/2)$ as follows:
\begin{align*}
  S_i = 2W_i - 1.
\end{align*}
Notice that if $W_i=0$ then $S_i=-1$ and if $W_i=1$ then $S_i=1$. This is as
desired. Let us now consider the position at some point in time $t$:
\begin{align*}
  X_n = x_0 + \sum_{i=1}^n S_i = \sum_{i=1}^n \left(2W_i - 1\right).
\end{align*}
We can move the constants out of the sum to find
\begin{align*}
  X_n = 2\sum_{i=1}^n W_i - n.
\end{align*}
Since a binomial random variable with $n$ trials is, by definition, the 
sum of $n$ Bernoulli variables, we can define $B_n\sim\Binom(n,1/2)$
and write 
\begin{align*}
  X_n = 2B_n - n
\end{align*}
as required. This lets us find
\begin{align*}
  \E(X_n) = \E(2B_n-n) = 2\E(B_n)-\E(n) = 2\left(\frac{n}{2}\right)-n = 0
\end{align*}
and
\begin{align*}
  \V(X_n) = \V(2B_n-n) = 4\V(B_n) = 4\left(\frac{n}{4}\right)=n.
\end{align*}
Therefore
\begin{align*}
  \sigma_{X_n} = \sqrt{\V(X_n)} = \sqrt{n}.
\end{align*}
For any sequence of $n$ steps, we have
\begin{align*}
  x_n = x_0 + a - b
\end{align*}
where $a+b=n$. Since $x_0=0$ we require $a=b$ for $x_n=0$. Using this,
we can write $a + a = 2a = n$ which is equivalent to $n$ being even.
Therefore $x_n=0$ cannot occur for odd $n$.
\section*{P7.3}
Let $X_k$ be the number of steps before a random walk starting at a node
$k$ steps away from the desired target terminates. Then we find the following
expected values for $k\in[1,3]$ with $E(X_k) = s_k$ and $s_0=0$:
\begin{align}
  \label{s1}
  s_1 &= 1 + \frac{s_0}{2} + \frac{s_2}{2} = 1 + \frac{s_2}{2},\\
  \label{s2}
  s_2 &= 1 + \frac{s_1}{2} + \frac{s_3}{2},\\
  \label{s3}
  s_3 &= 1 + s_2.
\end{align}
By inserting (\ref{s3}) into (\ref{s2}) we find 
\begin{align*}
  s_2 = 3 + s_1.
\end{align*}
Using this in (\ref{s1}) gives $s_1=5$ and thus $s_2=8$ and $s_3=9$.
By using the fact that $\sigma_{X_n}\to\infty$ as $n\to\infty$, we find that
$\P(X_n)\to 0$ at the same time.
\section*{P7.4}
\end{document}