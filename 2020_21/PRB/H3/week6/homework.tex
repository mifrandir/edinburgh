\documentclass{article}
\usepackage{homework-preamble}

\begin{document}
\title{Probability: Week 6 Solutions}
\author{Franz Miltz (UNN: S1971811)}
\date{10 November, 2020}
\maketitle


\section*{P6.1}


Let $X\sim\Poiss(1.5)$ be the number of patients that arrive in half an hour. Then
\begin{align*}
  \P(X \geq 4) = 1 - \P(X \leq 3)
\end{align*}
and
\begin{align*}
  \P(X \leq 3) & = \sum_{i=0}^3 \P(X=i)=e^{-3/2}\sum_{i=0}^3 \frac{(3/2)^i}{i!} \\
               & =e^{-3/2} (1+\frac{3}{2}+\frac{3^2}{2^3}+\frac{3^2}{2^4})
  \approx 0.934.
\end{align*}
Thus
\begin{align*}
  \P(X\geq 4) \approx 1 - 0.934 \approx 0.066.
\end{align*}
There is a chance of about $6.6\%$ that four or more patients arrive in
any given half hour period.


\section*{P6.2}


Let $B$ be the even that the coin is biased. Then $\P(B)=1/1000$. Further,
let $A_n$ be the event that a coin comes down heads $n$ times in a row.
Then we want to find some $N\in\R$ such that
\begin{align*}
  \forall n > N,\: \P(B|A_n)\geq 0.99.
\end{align*}
We can use \emph{Bayes theorem} to write
\begin{align*}
  \P(B|A_n) = \P(A_n|B)\frac{\P(B)}{\P(A_n)}.
\end{align*}
Since, by definition, $\P(A_n|B)=1$, this simplifies to
\begin{align}
  \label{ban}
  \P(B|A_n) = \frac{\P(B)}{\P(A_n)}.
\end{align}
Further, we know $\P(B)=1/1000$. We can use the \emph{Law of Total
  Probability} to find
\begin{align*}
  \P(A_n)=\P(A_n|B)\P(B) + \P(A_n|B^c)\P(B^c)
  =\frac{1}{1000}+\frac{999}{1000}\left(\frac{1}{2}\right)^n.
\end{align*}
By inserting both of these into (\ref{ban}) we obtain
\begin{align*}
  \P(B|A_n) = \frac{1}{1+999\cdot 2^{-n}}.
\end{align*}
This lets us solve
\begin{align*}
  \frac{1}{1+999\cdot 2^{-N}} = 0.99
\end{align*}
to find $N\approx 16.59$. Therefore, Guildenstern can be certain after
17 tosses.\\
We can verify this:
\begin{align*}
  \P(B|A_{17}) \approx 0.992.
\end{align*}


\section*{P6.3}


Let the number of occurrences be $X\sim\Poiss(\lambda t)$. Then
\begin{align*}
  \P(X=0) & =e^{-\lambda t},          \\
  \P(X=1) & =\lambda te^{-\lambda t}.
\end{align*}
We can add both of these to find
\begin{align*}
  \P(X\leq 1) = (1+\lambda t)e^{-\lambda t}.
\end{align*}
Then $T_2$ has the cdf
\begin{align*}
  F(t) = 1 - \P(X\leq 1) = 1-(1+\lambda t)e^{-\lambda t}
\end{align*}
and therefore
\begin{align*}
  f(t) = \frac{d}{dt}F(t) = \lambda^2te^{-\lambda t}.
\end{align*}
Let $g$ be the pdf of a random variable $Y\sim\Gamma(2,\lambda)$. Then
\begin{align*}
  g(t) = \frac{\lambda^2}{(2-1)!}te^{-\lambda t} = \lambda^2te^{-\lambda t} = f(t).
\end{align*}
Thus $T_2$ has a Gamma distribution.


\section*{P6.4}


Observe that we are only interested in the sequences containing even numbers.
Therefore, there are three possible outcomes $S=\{2,4,6\}$. Further,
we stop the experiment once we roll a six. Therefore we can model the
number of rolls up to and including the first six with a geometric random
variable $X\sim\Geom(1/3)$. We find
\begin{align*}
  \E(X)=\frac{1}{1/3} = 3.
\end{align*}
If I was asked to carry out an experiment to estimate this expected value
I would roll the D6 many times. I would count the number of rolls. If I
encountered an odd number, I would start anew without noting my result.
If I encountered a "6", I would start anew and write down the number of
rolls it took me from the last restart. At the end I would then be able
to take the average of the noted numbers to find an estimate.
\end{document}