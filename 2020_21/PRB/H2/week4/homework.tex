\documentclass{article}
\usepackage{homework-preamble}

\begin{document}
\title{Probability: Week 4 Solutions}
\author{Franz Miltz (UNN: S1971811)}
\date{13th October, 2020}
\maketitle


\section*{P 4.1} 


Let $f_X$ be the pdf of a random variable $X$ with
\begin{align*}
  f_X(x) = k x^n \text{ for some } n\in\N.
\end{align*}
Then we require
\begin{align*}
  \int_0^1 f_X(x) dx = k\int_0^1 x^n dx = 1.
\end{align*}
We find
\begin{align*}
   k\int_0^1 x^n dx = k\left[\frac{1}{n+1}x^{n+1}\right]_0^1 = \frac{k}{n+1}.
\end{align*}
Therefore, $k=n+1$ is required to make $f_X$ a pdf.\\
Now 
\begin{align*}
  \P(X\leq 1/2) = \int_0^{1/2} (n+1)x^n dx = \left[x^{n+1}\right]_0^{1/2} 
  =\frac{1}{2^{n+1}},
\end{align*}
and
\begin{align*}
  \E(X) = \int_0^1 x f_X(x) dx = \int_0^1 (n+1)x^{n+1} dx
  = \left[\frac{n+1}{n+2}x^{n+2}\right]_0^1 = \frac{n+1}{n+2}.
\end{align*}
We see that 
\begin{align*}
  \lim_{n\to\infty} \P(X\leq 1/2) = 0
\end{align*}
and
\begin{align*}
  \lim_{n\to\infty} \E(X) = 1.
\end{align*}
This is intuitively true because $f_X$ gets steeper with increasing
$n$ and thus the fraction of the area under the graph to the left 
gets smaller and smaller while the center of gravity moves closer to
$x=1$.


\section*{P 4.2}


We know that
\begin{align*}
  \P(a \leq X \leq b) = \int_a^b f_X(x) dx 
\end{align*}
and therefore
\begin{align*}
  \P(a \leq X \leq b) = \left[F_X(x)\right]_a^b = F_X(b) - F_X(a).
\end{align*}
Addditionally, we know by definition that
\begin{align*}
  F_X(x) = \P(X \leq x).
\end{align*}
Therefore
\begin{align*}
  \P(a \leq X \leq b) = \P(x \leq b) - \P(x < a).
\end{align*}
Obsesrve that, since $X$ is continuous,
\begin{align*}
  \P(X \leq x) = \P(X < x). 
\end{align*}
Thus, by applying the definition, we find
\begin{align*}
  \P(a \leq X \leq b) = \P(x \leq b) - \P(x \leq a) = F_X(b) - F_X(b).
\end{align*}


\section*{P 4.3}


Let $X$ be as described. Then for the event $X=k$ we know that
the $k$th roll has to be a six. Further, for the $k-1$ rolls before
that, there need to be exactly $r-1$ sixes. If and only if both of these
conditions are met, $X=k$. Therefore, we can write
\begin{align*}
  \P(X=k) = \frac{1}{6}\P(Y=r-1)
\end{align*}
where $Y$ is the number of sixes in $k-1$ rolls. Using the binomial
distribution, we find
\begin{align*}
  \P(X=k) = \frac{1}{6}\binom{k-1}{r-1}\left(\frac{1}{6}\right)^ {r-1}
  \left(\frac{5}{6}\right)^{k-r}
  = \frac{(k-1)!}{(r-1)!(k-r)!}\left(\frac{1}{6}\right)^r\left(\frac{5}{6}\right)^{k-r}
\end{align*}
as required. \\
We can write $X=X_1+X_2+\cdots X_r$ with $X_i\sim \Geom(r)$ for all $i$. 
Using the \emph{Law of Total Probability for Expected Values}, we find
\begin{align*}
  \E(X) = \E(X_1) + \E(X_2) + \cdots + \E(X_r) = r\E(X_1).
\end{align*}
Using \emph{Proposition 4.1.3} from the nodes, we find $\E(X_1) = 1/p$ and thus
\begin{align*}
  \E(X) = r/p.
\end{align*}


\section*{P 4.4}


Let the random variables $X$, $Y$ and $Z$ have 
the pdfs $f_X(x)=f_Y(x)=f_Z(x)=1$. Let $A$, $B$ and $C$
be the same random variables but assigned in such a way that
$A\leq B\leq C$ all the time. Then, since $X,Y,Z$ have identical pdfs,
there are $6$ equally likely mappings to $A,B,C$.
You can see that $B=W$ as specified in the question. For $f_W$, 
we thus find
\begin{align*}
  f_W(x) = 6\left(\P(A \leq x)\P(x \leq C)\right).
\end{align*}
By definition, we get
\begin{align*}
  f_W(x) = 6\left(\int_0^x 1dx\right)\left(\int_x^1 1dx\right) = 6x(1-x).
\end{align*}
We can now determine
\begin{align*}
  F_W(x) = 6\int_0^x x - x^2 dx = 3x^2-2x^3.
\end{align*}
We can check $F_W(1) = 3\cdot 1^2 - 2 \cdot 1^3 = 1$. Further, we can
calculate
\begin{align*}
  \P\left(\frac{1}{3}\leq W \leq \frac{2}{3}\right)
  &= F_W\left(\frac{2}{3}\right) - F_W\left(\frac{1}{3}\right)
  = 1 - \frac{14}{27} = \frac{13}{27}.
\end{align*}
\end{document}