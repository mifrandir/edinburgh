\documentclass{article}
\usepackage[a4paper]{geometry}
\usepackage[british]{babel}
\usepackage{amsmath}
\usepackage{amssymb}
\usepackage{mathtools}
\usepackage{nicefrac}
\usepackage{amsthm}
\usepackage{changepage}
\geometry{tmargin=2cm, bmargin=3cm}
\DeclarePairedDelimiter{\floor}{\lfloor}{\rfloor}
\newcommand*\lneg[1]{\overline{#1}}
\newcommand{\R}{\mathbb{R}}
\newcommand{\N}{\mathbb{N}}
\newcommand{\Z}{\mathbb{Z}}
\newcommand{\C}{\mathbb{C}}
\newcommand{\st}{\text{ s.t. }}
\renewcommand{\P}{\mathbb{P}}
%newenvironment{claim}[1]{\noindent\emph{Claim.}\space#1}{}
\newenvironment{claimproof}[1]{\par\noindent\emph{Proof.}\space#1}{\hfill $\blacksquare$}
\newtheorem{claim}[section]{Claim}
\newtheorem{lemma}{Lemma}[section]
\DeclareMathOperator{\lub}{\text{LUB}}
\DeclareMathOperator{\hcf}{hcf}
\DeclareMathOperator{\lcm}{lcm}
\setcounter{MaxMatrixCols}{20}
\newcommand{\E}{\mathbb{E}}
\newcommand{\V}{\mathbb{V}}
\DeclareMathOperator{\Geom}{Geom}
\DeclareMathOperator{\Binom}{Binom}
\DeclareMathOperator{\Bern}{Bernoulli}
\DeclareMathOperator{\Unif}{Unif}
\DeclareMathOperator{\Exp}{Exponential}
\DeclareMathOperator{\GD}{Gamma}
\newcommand*\binco[2]{\begin{pmatrix}
  #1\\#2
\end{pmatrix}}
\newcommand{\ih}{\widehat i}
\newcommand{\jh}{\widehat j}
\newcommand{\kh}{\widehat k}
\newcommand{\K}{\mathcal{K}}
\newcommand{\dv}[1]{\vec #1\, '}
\renewcommand{\d}[1]{#1'}
\begin{document}
\title{PRB: Week 4 Solutions}
\author{Franz Miltz (UNN: S1971811)}
\date{27th October, 2020}
\maketitle
\section{P 5.1}
We know that, for our random variable $W$, the pdf is
\begin{align*}
  f_W(x) = 6x(1-x).
\end{align*}
Using
\begin{align*}
  \E(W) = \int_0^1 xf_W(x) dx,
\end{align*}
we find
\begin{align*}
  \E(W)=\int_0^1 6x^2-6x^3 dx = \left[2x^3 - \frac{3}{2}x^4\right]^1_0 = \frac{1}{2}.
\end{align*}
Further, we know that
\begin{align*}
  \V(W) = \E(W^2) - (\E(W))^2.
\end{align*}
Since we have determined $\E(W)=1/2$, we only need to find $\E(W^2)$. We know
\begin{align*}
  \E(W^2) = \int_0^1 x f_{W^2}(x)dx.
\end{align*}
Further, observe that
\begin{align*}
  f_{W^2}(x) = f_W\left(\sqrt{x}\right) = 6\sqrt{x}\left(1-\sqrt{x}\right)
\end{align*}
is the pdf of $W^2$. Thus
\begin{align*}
  \E(W^2) = \int_0^1 6 x^{3/2}-6x^2 dx = \left[\frac{12}{5}x^{5/2}-2x^3\right]_0^1=\frac{2}{5}.
\end{align*}
Now we know
\begin{align*}
  \V(W) = \frac{2}{5}-\frac{1}{4} = \frac{3}{20}.
\end{align*}
\section{P 5.2}
We have a random variable $X$ with a generating function 
\begin{align*}
  G_X(s) = \frac{1}{2-s^2}.
\end{align*}
Observe that
\begin{align*}
  \frac{dG_X}{ds} = \frac{2s}{(2-s^2)^2}
\end{align*}
and thus
\begin{align*}
  \frac{dG_X}{ds}(1) = 2 = \E(X).
\end{align*}
Further
\begin{align*}
  \frac{d^2G_X}{ds^2}=\frac{6s^2+4}{(2-s^2)^3}
\end{align*}
and therefore
\begin{align*}
  \frac{d^2G_X}{ds^2}(1) = 10.
\end{align*}
We find
\begin{align*}
  \V(X) = \frac{d^2G_X}{ds}(1) + \frac{dG_X}{ds}(1) - \left(\frac{dG_X}{ds}(1)\right)^2
  = 10 + 2 - 4 = 8.
\end{align*}
Since
\begin{align*}
  G_X(s) = \frac{1}{2-s^2} = \frac{1}{2} + \frac{s^2}{4} + \frac{s^4}{8} + \frac{s^6}{16} + \cdots
\end{align*}
we can observe
\begin{align*}
  \P(X=3) = 0
\end{align*}
and
\begin{align*}
  \P(X=4) = 1/8.
\end{align*}
\section{P 5.3}
We have a random variable $Y\sim\Geom(1/6)$. Therefore 
\begin{align*}
  \P(Y=k) = \left(\frac{5}{6}\right)^{k-1}\left(\frac{1}{6}\right).
\end{align*}
Also note
\begin{align*}
  \P(|Y-6|\geq 6) = \P(Y \geq 12) = \frac{1}{6}\sum_{i=1}^{11} \left(\frac{5}{6}\right)^i.
\end{align*}
We can compute this sum to find
\begin{align*}
  \P(|Y-6|\geq 6) = \frac{6069373681}{2176782336} \approx 0.279.
\end{align*}
We know the \emph{Chebyshev Inequality}
\begin{align*}
  \P(|X-\mu|\geq k\sigma) \leq \frac{1}{k^2}.
\end{align*}
The variance of a geometric random variable $X\sim\Geom(p)$ is given by
\begin{align*}
  \V(X) = \frac{1-p}{p^2}.
\end{align*}
Thus
\begin{align*}
  \V(X) = \V(Y) = \frac{1-1/6}{(1/6)^2} = 30
\end{align*}
and
\begin{align*}
  \sigma_X = \sqrt{30}.
\end{align*}
We solve
\begin{align*}
  6 = k\sqrt{30} \Leftrightarrow k = \sqrt{\frac{6}{5}}
\end{align*}
and insert this result to find
\begin{align*}
  \P(|X-6| \geq 6) \leq \frac{5}{6}.
\end{align*}
\end{document}