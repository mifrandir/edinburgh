\documentclass{article}
\usepackage{homework-preamble}

\begin{document}
\title{Probability: Week 4 Hand-in}
\author{Franz Miltz (UNN: S1971811)}
\date{6th October 2020}
\maketitle


\section*{P 2.2}


Using the binomial distribution we get
\begin{align*}
  p_n = \binom{2n}{n}\left(\frac{1}{2}\right)^n\left(\frac{1}{2}\right)^n
  =\binom{2n}{n}\frac{1}{2^{2n}}.
\end{align*}
We can rewrite this by inserting the definition of the binomial
coefficient:
\begin{align*}
  p_n = \frac{(2n)!}{(n!)^2}\cdot\frac{1}{2^{2n}}=\frac{(2n)!}{(n!)^2\cdot 2^{2n}}.
\end{align*}
To approximate $p_n$ for large $n$, we use insert \emph{Stirlings formula}
to get
\begin{align*}
  p_n \approx \frac{
    \frac{(2n)^{2n}}{e^{2n}}\sqrt{4\pi n}
  }{
    \left(\frac{n^n}{e^n}\sqrt{2\pi n}\right)^2\cdot 2^{2n}}.
\end{align*}
Fortunately, we can simplify this to get
\begin{align*}
  p_n\approx \frac{
    \frac{2^{2n}\cdot n^{2n}}{e^{2n}}\cdot2\sqrt{\pi n}
  }{
    \frac{n^{2n}}{e^{2n}}\cdot 2\pi n\cdot 2^{2n}}=\frac{1}{\sqrt{\pi n}}.
\end{align*}
Rewriting this into the desired form, we get
\begin{align*}
  p_n\approx C / n^\alpha = \pi^{-\nicefrac{1}{2}}/n^{\nicefrac{1}{2}}.
\end{align*}
So $C=\pi^{-\nicefrac{1}{2}}$ and $\alpha=\nicefrac{1}{2}$.


\section*{P 2.3}


Let $X$ be the number of points $A$ scores when playing five more points.
For the probability of A winning the game in question
to be $\nicefrac{1}{2}$, we equivalently require
\begin{align*}
  \P(X\geq 4) = \nicefrac{1}{2}.
\end{align*}
We can split this up to get
\begin{align*}
  \P(X = 5) + \P(X = 4) = \nicefrac{1}{2}.
\end{align*}
Since $X\sim\text{Binom(5,p)}$, we get
\begin{align*}
  \binom{5}{5}p^5q^0+\binom{5}{4}p^4q^1=\nicefrac{1}{2},
\end{align*}
which may be simplified to
\begin{align*}
  p^5+5p^4(1-p)             & = \nicefrac{1}{2} \\
  \Leftrightarrow 5p^4-4p^5 & =\nicefrac{1}{2}
\end{align*}
Finding an exact solution to this polynomial is unrealistic.
Therefore we can use Wolfram Alpha to compute three possible real solutions:
\begin{align*}
  p \in \{-0.516..., 0.686..., 1.187...\}.
\end{align*}
Since we also know $p\in[0,1]$ simply because we are dealing with a
probability, we get
\begin{align*}
  p\in \{-0.516..., 0.686..., 1.187...\} \cap [0,1] \Rightarrow p \approx 0.686=68.6\%.
\end{align*}


\section*{P 3.3}


Let $X_i$ be a Bernoulli random variable for the state of the switch described in the
problem on a given day $i$ and let $A_i$ be the event that $X_i=1$ on the $i$th day,
i.e. the switch is "on" on that day.
Using the \emph{law of total probability} we find for the probability of the switch
being "on" on day $i$
\begin{align*}
  \P(A_{i+1}) = \P(A^c_i)\P(A^c_i | A_{i+1}) + \P(A_i)\P(A_i|A_{i+1})
\end{align*}
Since all days are claimed to be
identical, we assume that there exists a $p\in[0,1]$ such that for all $i$,
$\P(A_i)=p$. Thus
\begin{align*}
  p = (1-p)\P(A^c_i|A_{i+1}) + p\P(A_i|A_{i+1}).
\end{align*}
The remaining probabilities, namely $\P(A^c_i|A_{i+1})=1/4$ and $\P(A_i|A_{i+1})=1/2$,
are given in the question. Inserting them results in the equation
\begin{align*}
  p=\frac{1-p}{4} + \frac{p}{2}.
\end{align*}
We can solve for $p$ to find
\begin{align*}
  p=\frac{1}{3}
\end{align*}
which shows that the observation in the question is indeed true.
\end{document}