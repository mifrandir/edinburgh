\documentclass{article}
\usepackage{homework-preamble}

\begin{document}
\title{Probability: Week 3 Solutions}
\author{Franz Miltz (UNN: S1971811)}
\date{13th October, 2020}
\maketitle


\section*{P 3.1}


Let $E$ be the experiment of rolling three D6 and let $X$ be the number of 6s rolled.
Then let $A$ be the event that $X\geq 1$ and let $B$ be the event that $X=2$. 
We are interested in $\P(A|B)$. By definition, we find
\begin{align*}
  \P(A|B) = \frac{\P(A\cap B)}{\P(A)}.
\end{align*}
Observe that 
\begin{align*}
  \P(A\cap B) = \P(X\geq 1 \text{ and } X = 2) = \P(X=2) = \P(B)
\end{align*}
and
\begin{align*}
  \P(B) = \binom{3}{2}\left(\frac{1}{6}\right)^2\frac{5}{6}=\frac{15}{216}.
\end{align*}
Further
\begin{align*}
  \P(A) = 1-\P(A^c) = 1-\binom{3}{0}\left(\frac{5}{6}\right)^3=\frac{91}{216}.
\end{align*}
Therefore
\begin{align*}
  \P(A\cap B) = \frac{\P(B)}{\P(A)}= \frac{15/216}{91/216} = \frac{15}{91}.
\end{align*}


\section*{P 3.2}


Let the $X$ be as described in the problem. Let the events
$A_i$ for $i\in[4,5,6]$ be that a specific D$i$ is chosen.
Then we can use \emph{the law of total probability} to find
\begin{align*}
  \P(X=1)=\sum_{i=4}^6\P(A_i)\P(A_i|X=1).
\end{align*}
For each $i$ we know
\begin{align*}
  \P(A_i) = \frac{1}{3} 
\end{align*}
since there are three dice that are equally likely to be chosen and
\begin{align*}
  \P(A_i|X=1) = \frac{1}{i}
\end{align*}
by definition of D$i$.
Thus 
\begin{align*}
  \P(X=1) = \frac{1}{3}\left(\frac{1}{4}+\frac{1}{5}+\frac{1}{6}\right)=\frac{37}{180}.
\end{align*}
Furthermore, we know from \emph{the law of total probability for expectations} that
\begin{align*}
  \E(X) = \sum_{i=4}^6 \E(X|A_i)\P(A_i) = \frac{1}{3}\sum_{i=4}^6\E(X|A_i).
\end{align*}
Inserting the expected values for each D$i$ we find
\begin{align*}
  \E(X) = \frac{1}{3}\left(\frac{5 + 6 + 7}{2}\right)=3.
\end{align*}


\section*{P 3.3}


Let $X$ be a Bernoulli random variable for the state of the switch described in the
problem and let $A_i$ be the event that $X=1$ on the $i$th day, i.e. the switch is 
"on" on that day. Lets consider the transition between two days where $\P(E\to F)$ 
means the probability given $E$ occurred on the first day that $F$ occurs on the
second day. We find for the probability of the switch being on on day $i$
\begin{align*}
  p=\P(A_{i+1}) = \P(A^c_i)\P(A^c_i | A_{i+1}) + \P(A_i)\P(A_i|A_{i+1})
\end{align*}
where $p$ is the probability that $A$ occurs on any given day $i$. Since all days are
identical, $\P(A_{i+1})=p=\P(A_i)$. Thus
\begin{align*}
  p = (1-p)\P(A^c_i|A_{i+1}) + p\P(A_i|A_{i+1}).
\end{align*}
The remaining probabilities, i.e. $\P(A^c_i|A_{i+1})=1/4$ and $\P(A_i|A_{i+1})=1/2$, 
are given in the question. Inserting them results in the equation
\begin{align*}
  p=\frac{1-p}{4} + \frac{p}{2}.
\end{align*}
We can solve for $p$ to find 
\begin{align*}
  p=\frac{1}{3}
\end{align*}
which shows that the observation in the question is indeed true.


\section*{P 3.4}


Let $A$ be the event that a person has the condition and let $B$ be the event that
a person gets tested positive. Then we know from \emph{the law of total probability}
that
\begin{align*}
  \P(B) = \P(A|B)\P(A) + \P(A^c|B)\P(A^c).
\end{align*}
Inserting the values from the problem gives
\begin{align*}
  \P(B) = \frac{99}{100}\frac{1}{1000}+\frac{1}{200}\frac{999}{1000}=\frac{1197}{2\cdot 10^5}\approx 0.599\%.
\end{align*}
Let $X$ be the number of people that get a positive result if $1000$ people are tested.
Since the probability of getting a positive result is identical for all of them, we
find that
\begin{align*}
  \E(X) = 1000 \P(B)=\frac{1197}{200}\approx 5.99
\end{align*}
but we only expect $1$ in $1000$ to actually have the condition. Thus
we expect only $1/5.99\approx 16.7\%$ of the positive results to be correct.
\end{document}