\documentclass{article}
\usepackage[a4paper]{geometry}
\usepackage[british]{babel}
\usepackage{amsmath}
\usepackage{amssymb}
\usepackage{mathtools}
\usepackage{nicefrac}
\usepackage{amsthm}
\usepackage{changepage}
\geometry{tmargin=2cm, bmargin=3cm}
\DeclarePairedDelimiter{\floor}{\lfloor}{\rfloor}
\newcommand*\lneg[1]{\overline{#1}}
\newcommand{\R}{\mathbb{R}}
\newcommand{\N}{\mathbb{N}}
\newcommand{\Z}{\mathbb{Z}}
\newcommand{\C}{\mathbb{C}}
\newcommand{\st}{\text{ s.t. }}
\newcommand{\p}{\partial }
\newcommand{\di}{\iint\limits}
\newcommand{\ti}{\iiint\limits}
\newcommand{\grad}{\vec\nabla}
%newenvironment{claim}[1]{\noindent\emph{Claim.}\space#1}{}
\newenvironment{claimproof}[1]{\par\noindent\emph{Proof.}\space#1}{\hfill $\blacksquare$}
\newtheorem{claim}[section]{Claim}
\newtheorem{lemma}{Lemma}[section]
\DeclareMathOperator{\lub}{\text{LUB}}
\DeclareMathOperator{\hcf}{hcf}
\DeclareMathOperator{\lcm}{lcm}
\setcounter{MaxMatrixCols}{20}
\newcommand*\binco[2]{\begin{pmatrix}
  #1\\#2
\end{pmatrix}}
\newcommand{\ih}{\widehat i}
\newcommand{\jh}{\widehat j}
\newcommand{\kh}{\widehat k}
\newcommand{\K}{\mathcal{K}}
\renewcommand{\vec}{\underline}
\newcommand{\dv}[1]{\vec #1'}
\renewcommand{\d}[1]{#1'}
\renewcommand{\t}{\theta}
\newenvironment{itpars}{\par\itshape}{\par}
\newenvironment{itquote}{\begin{quote}\itshape}{\end{quote}\ignorespacesafterend}
\begin{document}
\title{SVCDE: Hand-in 5}
\author{Franz Miltz (UNN: S1971811)}
\date{25th October 2020}
\maketitle
\section*{Question 1}
Observe that we can write the height of the volume $V$ in terms of $x$ and $y$ as 
\begin{align*}
  h(x,y) = \sqrt{x^2 + y^2}
\end{align*}
because it is bounded below by $z=0$ and above by $z=\sqrt{x^2 + y^2}$.
Thus we need to integrate the height over the projection onto the $xy$ plane. We find
\begin{align*}
  V = \di_D h(x,y)dA = \di_D \sqrt{x^2+y^2}dA
\end{align*}
with
\begin{align*}
  D = \{(x,y) : x^2 + y^2 = 2x\}.
\end{align*}
We can see that $D$ is the unit circle around the point $(1,0)$. Therefore we can 
change to polar coordinates with $x=r\cos\t$ and $y=r\sin\t$ to find
\begin{align*}
  V = \int_{-\pi/2}^{\pi/2}\int_0^{2\cos\t} r^2\: dr\,d\t.
\end{align*}
We can simplify a bit by observing that $z(r,\t)=z(r,-\t)$. This lets us write
\begin{align*}
  V = 2\int_{0}^{\pi/2}\int_0^{2\cos \t} r^2\: dr\,d\t.
\end{align*}
Solving gives 
\begin{align*}
  V &= 2\int_0^{\pi/2}\left[\frac{1}{3}r^3\right]_0^{2\cos \t}d\t
  =\frac{16}{3}\int_0^{\pi/2}\cos^3 \t\:d\t\\
  &=\frac{16}{3}\left[\sin \t - \frac{1}{3}\sin^3 \t\right]_0^{\pi/2}
  =\frac{16}{3}\left(\frac{2}{3}\right) = \frac{32}{9}
\end{align*}
as required.
\section*{Question 2}
We have to solve
\begin{align*}
  I = \ti_D \left(\frac{z}{1-|x|y}\right)^2 dV
\end{align*}
with $D=\{(x,y,z) : x\in(-1,2), y\in(-1,0), z\in(0,3)\}$. Since all the limits are constants,
we can use \emph{Fubini's Theorem} and evaluate
\begin{align*}
  I = \int_{-1}^2 \int_{-1}^0 \int_0^3 \left(\frac{z}{1-|x|y}\right)^2\:dz\,dy\,dx. 
\end{align*}
We start with
\begin{align*}
  I_1 (x,y) = \int_0^3 \left(\frac{z}{1-|x|y}\right)^2\:dz = \frac{9}{(1-|x|y)^2}.
\end{align*}
Reinserting this, we find
\begin{align*}
  I = \int_{-1}^2 \int_{-1}^0 \frac{9}{(1-|x|y)^2}\:dy\,dx.
\end{align*}
Now we find
\begin{align*}
  I_2 (x) = \int_{-1}^0 \frac{9}{(1-|x|y)^2}\:dy 
  = \frac{9}{|x|}\left[\frac{1}{1-|x|y}\right]_{-1}^0
  = \frac{9}{|x|}\left(1-\frac{1}{1+|x|}\right).
\end{align*}
This may be simplified to
\begin{align*}
  I_2(x)=\frac{9}{|x|+1}
\end{align*}
and so, by separating positive and negative $x$ values, 
\begin{align*}
  I &= \int_{-1}^2 \frac{9}{|x|+1}\: dx\\
  &=-\int_{-1}^0 \frac{9}{1-x}\:dx
  + \int_0^2 \frac{9}{1+x}\: dx\\
  &=-9\left[\ln(1-x)\right]_{-1}^0 + 9\left[\ln(1+x)\right]_0^2\\
  &=9\ln 2 + 9\ln 3 = 9\ln 6
\end{align*} 
as required.
\end{document}
