\documentclass{article}
\usepackage{homework-preamble}

\begin{document}

\title{SVCDE: Hand-in 8}
\author{Franz Miltz}
\date{15 November 2020}
\maketitle


\section*{Question 1}


We want to evaluate
\begin{align*}
    I = \di_S x^2y^2+y^2z^2+z^2x^2 dS
\end{align*}
where $S$ is the surface of the unit sphere around the origin. 
We use the \emph{Divergence Theorem} to find
\begin{align*}
    I = \ti_E \grad\cdot\vec F\:dV = \ti_E 2xy^2 + 2yz^2 + 2xz^2\:dV
\end{align*}
where $E$ is the unit sphere around the origin. We can insert the limits to get
\begin{align*}
    I = \int_0^1 \int_{-\sqrt{1-x^2}}^{\sqrt{1-x^2}} \int_{-\sqrt{1-x^2-y^2}}^{\sqrt{1-x^2-y^2}}
    2xy^2 + 2yz^2 + 2zx^2\:dz\,dy\,dx.
\end{align*}
By evaluating the inner integral we obtain
\begin{align*}
    I &= \int_0^1 \int_{-\sqrt{1-x^2}}^{\sqrt{1-x^2}}
    \left[\frac{2}{3}yx^3 + z^2x^2\right]_{-\sqrt{1-x^2-y^2}}^{\sqrt{1-x^2-y^2}}\:dy\,dx\\
    &= \int_0^1 \int_{-\sqrt{1-x^2}}^{\sqrt{1-x^2}}
    \frac{4}{3}y(1-x^2-y^2)^{3/2}\: dy\,dx.
\end{align*}
We can now convert this integral to polar coordinates:
\begin{align*}
    I &= \int_0^1 \int_0^{2\pi}\frac{4}{3}r^2\sin\theta (1-r^2)^{3/2}\:d\theta\,dr\\
    &=\int_0^1\left[-\frac{4}{3}r^2\cos\theta(1-r^2)^{3/2}\right]_0^{2\pi}dr
\end{align*}
Note that $\theta$ only appears within the cosine. Since $\cos(0)=\cos(2\pi)=1$,
we know that
\begin{align*}
    I = \int_0^1 0\:dr = 0.
\end{align*}


\section*{Question 2}


We want to find an implicit solution of
\begin{align*}
    \frac{dy}{dx}=\frac{x+y}{x-y}.
\end{align*}
Firstly, observe that
\begin{align}
    \label{dydx}
    \frac{dy}{dx}=\frac{1+\frac{y}{x}}{1-\frac{y}{x}} = g\left(\frac{y}{x}\right)
\end{align}
where
\begin{align}
    \label{g}
    g(v) = \frac{1+v}{1-v}.
\end{align}
We let
\begin{align*}
    v(x)=\frac{y}{x}.
\end{align*}
Thus $y = v(x)x$ and via the \emph{Product Rule}
\begin{align*}
    \frac{dy}{dx}=v+x\frac{dv}{dx}.
\end{align*}
By inserting (\ref{dydx}) and (\ref{g}) we get
\begin{align*}
    v+x\frac{dv}{dx} &= \frac{1+v}{1-v}\\
    \Leftrightarrow x\frac{dv}{dx} &= \frac{1+v^2}{1-v}
\end{align*}
Separating the variables leads us to
\begin{align*}
    \frac{1+v^2}{1-v}\frac{dv}{dx} = \frac{1}{x}.
\end{align*}
Integrating both sides w.r.t. $x$ gives
\begin{align*}
    \int \frac{1+v^2}{1-v}\:dv &= \int \frac{1}{x}\: dx\\
    \Leftrightarrow -\frac{v^2}{2}-v-2\ln(v-1) &= \ln(x)+C
\end{align*}
By resubstituting $v$, we get the implicit expression
\begin{align*}
    \ln(x)-2\ln(y-x)-\frac{y^2+2xy}{2x^2} = C.
\end{align*}
\end{document}