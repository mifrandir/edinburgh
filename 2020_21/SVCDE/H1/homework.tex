\documentclass{article}
\usepackage{homework-preamble}

\begin{document}
\title{SVCDE: Hand-in 1}
\author{Franz Miltz (UUN: S1971811)}
\date{27th September 2020}
\maketitle


\section*{Problem 1}


Let $P$ be a plane given by the equation
\begin{align*}
	x - 2y + 4z = 5.
\end{align*}
Then the vector
\begin{align*}
	\vec n = \ih - 2 \jh + 4 \kh
\end{align*}
is normal to $P$. Thus a line $L$ perpendicular to $P$ has to be
parallel to $\vec n$. The line parallel to $\vec n$ that goes through
the point $(1,-2,3)$ with the position vector $\vec p$ is traced by the position vector
\begin{align*}
	\vec r(t) = \vec p + t \vec n
	= (\ih - 2\jh + 3\kh) + t(\ih - 2\jh + 4\kh).
\end{align*}
Therefore we get
\begin{align*}
	\vec r (t) = f(t)\ih + g(t)\jh + h(t)\kh = (1+t)\ih+(-2-2t)\jh+(3+4t)\kh.
\end{align*}


\section*{Problem 2}


The curve with the equation $y=f(x)$ has the parametrisation
\begin{align*}
	\vec r(x) = x \ih + f(x) \jh.
\end{align*}
Then the curvature $\kappa$ at a point $(x, f(x))$ is given by
\begin{align*}
	\kappa(x) = \frac{\left|\dv T(x)\right|}{\left|\dv r(x)\right|}.
\end{align*}
We know that
\begin{align}
	\label{eq1}
	\dv r (x) = \ih + \d f(x) \jh.
\end{align}
Further, the tangent unit vector $\vec T(x)$ is defined by
\begin{align*}
	\vec T(x) = \frac{\dv r(x)}{\left|\dv r(x)\right|}.
\end{align*}
Therefore we get
\begin{align*}
	\kappa(x) = \frac{\left|\frac{\dv r(x)}{\left|\dv r(x)\right|}\right|}{\left|\dv r(x)\right|}
\end{align*}
Using the fact that for all $\vec v \in \R^n$ and $r\in\R$
\begin{align*}
	\left|r\vec v\right| = r\left|\vec v\right|,
\end{align*}
we get
\begin{align*}
	\kappa(x) = \frac{\left|\dv r(x)\right|}{\left|\dv r(x)\right|^2} = \frac{1}{|\dv r(x)|}.
\end{align*}
With (\ref{eq1}) we get
\begin{align*}
	\kappa(x) = \frac{1}{\sqrt{1+\left(\d f(x)\right)^2}}.
\end{align*}


\section*{Problem 3}


Let
\begin{align*}
	\dv r(t) = t\ih + \frac{4}{3}t^{3/2}\jh + \frac{1}{2}t^2\kh
\end{align*}
be a position vector function.
Then the length $L$ of the curve traced out by $\vec r$ for $0\leq t \leq 2$ is given by
\begin{align*}
	L = \int_0^2 \left|\dv r(t)\right|dt.
\end{align*}
We find
\begin{align*}
	\dv r(t) = \ih + 2\sqrt{t}\:\jh + t^2\kh
\end{align*}
and thus
\begin{align*}
	|\dv r(t)| = \sqrt{1 + 4t + t^2}=|t+1|.
\end{align*}
Since for $t\in[0,2]$ the value $|\dv r(t)|=|t+1|$ is never
negative, $L$ is given by
\begin{align*}
	L = \int_0^2 (t+1)dt = \left[\frac{1}{2}t^2+t\right]_0^2=4.
\end{align*}
\end{document}