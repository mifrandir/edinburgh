\documentclass{article}
\usepackage{homework-preamble}

\begin{document}
\title{SVCDE: Hand-in 2}
\author{Franz Miltz (UNN: S1971811)}
\date{4th October 2020}
\maketitle
\section*{Problem 1}
Since the required plane $P$ is supposed to be parallel to the $yz$ plane, it
has to have the form $x=a$ where $a\in\R$ is a constant.
Additionally, both partial derivatives of $x$ need to be zero at the
point $(a,b,c)$ that lies within the plane and the given surface.
With
\begin{align*}
  x = y^2 - 4yz - 2z^2 + 8y + 2z -3
\end{align*}
we get the partial derivatives
\begin{align*}
  \frac{\p x}{\p y}&=2y - 4z + 8,\text{ and}\\
  \frac{\p x}{\p z}&=-4y-4z+2.
\end{align*}
Since we require $\frac{\p x}{\p y}=\frac{\p x}{\p z}=0$ for
some $y=b$ and $z=c$, we get
\begin{align*}
  2b - 4c + 8 &= 0,\\
  4b + 4c - 2 &= 0.
\end{align*}
Adding these gives $6b + 6 = 0$ and thus $b=-1$.
By inserting this solution we get $c=\nicefrac{3}{2}$. 
To find $a$ we just need to evaluate $x$ for $y=b$ and $z=c$:
\begin{align*}
  a = x(b,c) = (-1)^2 - 4(-1)\frac{3}{2} - 2\left(\frac{3}{2}\right)^2
      + 8(-1) + 2\left(\frac{3}{2}\right) - 3 = - \frac{11}{2}. 
\end{align*}
Therefore the plane $P$ has the equation
\begin{align*}
  x = -\frac{11}{2}.
\end{align*}
\section*{Problem 2}
Let $ z = \ln\frac{x^2}{1-y^3}$. Then the partial derivatives are
\begin{align*}
  \frac{\p z}{\p x}
  &=\frac{1-y^3}{x^2}\left[\frac{\p}{\p x}\left(\frac{x^2}{1-y^3}\right)\right]\\
  &=\frac{1-y^3}{x^2}\left[\frac{2x}{1-y^3}\right]\\
  &=\frac{2x(1-y^3)}{x^2(1-y^3)}=\frac{2}{x}
\end{align*}
and
\begin{align*}
  \frac{\p z}{\p y}
  &=\frac{1-y^3}{x^2}\left[\frac{\p}{\p y}\left( \frac{x^2}{1-y^3}\right)\right]\\
  &=\frac{1-y^3}{x^2}\left[\frac{3x^2y^2}{(1-y^3)^2}\right]\\
  &=\frac{3x^2(1-y^3)y^2}{x^2(1-y^3)^2} =\frac{3y^2}{1-y^3}.
\end{align*}
\section*{Problem 3}
We are looking for the differential $dQ$ where
\begin{align*}
  Q(p, \theta, R)=p^3\sin^2\theta R^{-2}.
\end{align*}
By definition (cf. \emph{Notes, Section 2.4.3}), $dQ$ is
\begin{align*}
  dQ=\frac{\p Q}{\p p}dp + \frac{\p Q}{\p\theta}d\theta + \frac{\p Q}{\p R}dR.
\end{align*}
We find
\begin{align*}
  \frac{\p Q}{\p p}&= 3p^2\sin^2\theta R^{-2},\\
  \frac{\p Q}{\p\theta}&= 2p^3R^{-2}\sin\theta\cos\theta,\\
  \frac{\p Q}{\p R}&=-2p^3\sin^2\theta R^{-3}.
\end{align*}
Since $p=1$, $\theta=\nicefrac{\pi}{4}$, $R=10$, $dp=0.01$, we get
\begin{align*}
  \frac{\p Q}{\p p}(1, \nicefrac{\pi}{4}, 10)
  &=3\cdot 1 \cdot \frac{1}{2} \cdot \frac{1}{100}=0.015,\\
  \frac{\p Q}{\p\theta}(1, \nicefrac{\pi}{4}, 10)
  &=2\cdot 1 \cdot \frac{1}{100}\cdot\frac{1}{\sqrt{2}}\cdot\frac{1}{\sqrt{2}} =0.01,\\
  \frac{\p Q}{\p R}(1, \nicefrac{\pi}{4}, 10)
  &= -2\cdot 1 \cdot \frac{1}{2}\frac{1}{1000}=-0.001.
\end{align*}
Additionally $dp = 0.01$, $d\theta = 0.002$ and $dR=0.05$.
Therefore the maximum error may be estimated as
\begin{align*}
  dQ(1,\pi/4,10)=0.015\cdot 0.01+0.01\cdot 0.002  -0.001 \cdot 0.05=1.2\cdot10^{-4}.
\end{align*}
With a value of $Q(1,\nicefrac{\pi}{4},10)=5\cdot10^{-3}$ we therefore get a
relative error of approximately 
\begin{align*}
  \frac{dQ(1,\nicefrac{\pi}{4},10)}{Q(1,\nicefrac{\pi}{4},10)}=\frac{2.7\cdot10^{-4}}{5\cdot 10^{-3}}=2.4\%.
\end{align*}
\end{document}
