\documentclass{article}
\usepackage{notes-preamble}
\begin{document}
\title{Several Variable Calculus and Differential Equations (SEM3)}
\author{Franz Miltz}
\maketitle
\tableofcontents
\pagebreak



\section{Vectors and geometry}



(see ILA notes)
\setcounter{subsection}{4}


\subsection{Vector functions}


\begin{definition}
    A function $\vec r$ of the form $\vec r: \R \to \R^3$ is called a \B{vector function}.
\end{definition}
\begin{definition}
    The limit $\vec b$ of a vector function $\vec r$ at $a$ exists if, and only if,
    \begin{align*}
        \forall \epsilon > 0,\: \exists \delta > 0 \st 0 < |t-a| < \delta \Rightarrow |\vec r(t) - \vec b|<\epsilon.
    \end{align*}
    This is the case if, and only if, the limits for all the components of $\vec f$ exist.\\
    A vector function $\vec r(t)$ is \B{continuous} at $t=a$ if
    \begin{align*}
        \lim_{t\to a} \vec r (t) = \vec r (a). 
    \end{align*}
\end{definition}


\subsection{Derivatives and integrals}


\subsubsection{Derivatives}

\begin{definition}
    The \B{derivative of a vector function} $\vec r(t)$ is defined as
    \begin{align*}
        \vec r(t) = \frac{d\vec r}{dt} = \lim_{\delta \to 0} \frac{\vec r(t+\delta)-\vec r(t)}{\delta}.
    \end{align*}
\end{definition}
\begin{theorem}
    Let $\vec v = f(t)\ih + g(t)\jh + h(t)\kh$ where $f,g,h: \R\to\R$. Then
    \begin{align*}
        \frac{d\vec v}{dt} = \frac{df}{dt}\ih + \frac{dg}{dt}\jh + \frac{dh}{dt}\kh.
    \end{align*}
\end{theorem}
\begin{definition}
    $\vec r'(a)$ is the \B{tangent vector} at the point $t=a$ to the curve traced out by the position vector function $\vec r(t)$.
    The \B{unit tangent vector} at $t=a$ is
    \begin{align*}
        \vec T(a) = \frac{\vec r'(a)}{|\vec r'(a)|}
    \end{align*}
    if $|\vec r'(a)| \not= 0$.
\end{definition}

\subsubsection{Integrals}

\begin{definition}
    The \B{definite integral} of the continuous vector function $\vec v(t)=f(t)\ih + g(t)\jh + h(t)\kh$ is given by
    \begin{align*}
        \int_a^b \vec v(t)dt = \left[\int_a^b f(t)dt\right]\ih + \left[\int_a^b g(t) dt\right]\jh + \left[\int_a^b h(t)dt\right] \kh.
    \end{align*}
\end{definition}
\begin{theorem}{Fundamental Theorem of Calculus}
    \begin{align*}
        \int_a^b \vec v(t) dt = \left[\vec V(t)\right]^b_a = \vec V(b) - \vec V(a)
    \end{align*} 
    where $\vec V'(t) = \vec v(t)$.
\end{theorem}


\subsection{Arc Length}


\begin{theorem}
    The length $L$ of the curve traced out by the position vector function $\vec r(t)$ in three-dimensional space ove the interval
    $[a,b]$ is given by
    \begin{align*}
        L = \int_a^b \left|\frac{d\vec r(t)}{dt}\right| dt
    \end{align*}
\end{theorem}


\subsection{Curvature}


\begin{definition}
    A parameterisation $\vec v(t)$ is \B{smooth} on an interval $I$ if $\vec v'(t)$ is continuous and $\vec v'(t) \not = \vec 0$ on $I$.
\end{definition}
\begin{definition}
    A curve is \B{smooth} if it has a smooth parameterisation.
\end{definition}
\begin{definition}
    For a smooth curved traced out by the position vector $\vec r(t)$ with the unit tangent vector 
    $\vec T(t)$ the \B{curvature} $\kappa$ is defined as
    \begin{align*}
        \kappa = \left|\frac{d\vec T}{ds}\right|=\frac{\left| \vec T'(t)\right|}{\left|\vec r'(t)\right|}
    \end{align*}
    where $s$ is the arc length.
    Alternatively
    \begin{align*}
        \kappa = \frac{\left|\dv r(t) \times \dv r'(t)\right|}{|\dv r(t)|^3}
    \end{align*}
\end{definition}


\subsection{Normal and binormal vectors}


\begin{definition}
    Let the \B{principal unit normal vector} of the curve traced out by the position vector $\vec r(t)$
    where $\vec r'(t)$ is smooth be defined as
    \begin{align*}
        \vec N (t) = \frac{\vec T'(t)}{\left| \vec T'(t) \right|}.
    \end{align*}
\end{definition}
\begin{definition}
    The \B{binormal vector} is defined as
    \begin{align*}
        \vec B(t) = \vec T(t) \times \vec N(t).
    \end{align*}
\end{definition}
\begin{theorem}
    \begin{align*}
        \vec N(t) &= \vec B(t) \times \vec T(t)\\
        \vec T(t) &= \vec N(t) \times \vec B(t)
    \end{align*}
\end{theorem}



\section{Partial differentiation}



\subsection{Functions of several variables}


\begin{definition}
    A function of $n$ variables $f(x_1, x_2, ..., x_n)$ maps an $n$-tuple
    of real numbers $(x_1, x_2, ..., x_n)$ in domain $D\subseteq\R^n$ to
    a unique real number in the range
    \begin{align*}
        \{f(x_1, x_2, ..., x_n)|(x_1, x_2, ..., x_n)\in D\} \subseteq \R.
    \end{align*}
\end{definition}
\begin{definition}
    The \B{level curves} of a function $f(x,y)$ are curves with equations
    $f(x,y)=k$ where $k$ is a constant in the range of $f$.
\end{definition}
\begin{definition}
    The \B{level surfaces} of a function $f(x,y,z)$ are surfaces with
    equations $f(x,y,z)=k$ where $k$ is a constant in the range of $f$.
\end{definition}


\subsection{Limits \& continuity}


\subsubsection{Limits}

\begin{definition}
    Suppose the domain $D$ of a function $f:D\to\R$ includes points
    arbitrarily close to the point $P$ with position vector $\vec p$.
    Then
    \begin{align*}
        \lim_{\vec x\to \vec p}f(\vec x)=L
    \end{align*}
    if for every $\epsilon>0$ there exists a $\delta > 0$ such that
    \begin{align*}
        0 < |\vec x - \vec p| < \delta \:\Rightarrow\: |f(\vec x)-L|<\epsilon
    \end{align*}
    for $\vec x \in D$.\\\\
    Note: The definition is independent of the direction of approach to the
    limit point $\vec p$. Thus, if
    \begin{align*}
        &f(\vec x)\to L_1 \text{ as } \vec x\to\vec p \text{ along a path } C_1, \text{ and}\\
        &f(\vec x)\to L_2 \text{ as } \vec x\to\vec p \text{ along a path } C_2
    \end{align*}
    where $L_1 \not= L_2$, then $\lim_{\vec x\to\vec p} f(\vec x)$ does not exist.
\end{definition}
\begin{theorem}
    The limit laws carry over from functions of one variable case:
    \begin{enumerate}
        \item $\lim_{\vec x\to\vec p}\left(f(\vec x)\pm g(\vec x)\right) =
               \lim_{\vec x\to\vec p}f(\vec x)\pm\lim_{\vec x\to\vec p}g(\vec x)$.
        \item $\lim_{\vec x\to\vec p}\left(cf(\vec x)\right) =
              c\lim_{\vec x\to\vec p}f(\vec x)$ where $c$ is a constant.
        \item $\lim_{\vec x\to\vec p}\left(f(\vec x)g(\vec x)\right) =
               \left(\lim_{\vec x\to\vec p}f(\vec x)\right)
               \left(\lim_{\vec x\to\vec p}g(\vec x)\right)$.
        \item $\lim_{\vec x\to\vec p}\frac{f(\vec x)}{g(\vec x)}
               =\frac{\lim_{\vec x\to\vec p}f(\vec x)}{\lim_{\vec x\to\vec p}g(\vec x)}$
               provided that $\lim_{\vec x\to\vec p}g(\vec x)\not=0$.
    \end{enumerate}
\end{theorem}

\subsubsection{Continuity}

\begin{definition}
    A function $f:D\to R$ with $D\subseteq\R^n$ and $R\subseteq\R$
    is called \B{continuous} at a point $P$ with position vector $\vec p\in \R^n$
    \begin{align*}
        \lim_{\vec x\to \vec p}f(\vec x) = f(\vec p);
    \end{align*}
    and $f$ is \B{continuous on $S\subseteq D$} if it is continuous at 
    every point in $S$.
\end{definition}


\subsection{Partial derivatives}


\begin{definition}
    Consider a function $f:\R^n\to\R$. The $n$ \B{partial derivatives} of
    $f$ are defined by
    \begin{align*}
        f_{x_i}(\vec x)\equiv\frac{\p f}{\p x_i}=\lim_{\delta\to 0}\frac{f(\vec x + \delta \vec e_i)-f(\vec x)}{\delta}
    \end{align*}
    where $i\in[1,n]$, $x_i$ is the $i$th component of $\vec x$ and $\vec e_i$ is the
    $i$th standard unit vector.
\end{definition}
Notation for higher derivatives:
\begin{align*}
    (f_x)_x \equiv f_{xx} \equiv f_{11} 
    &\equiv \frac{\p}{\p x}\frac{\p f}{\p x} 
    \equiv \frac{\p^2 f}{\p x^2}
    \equiv \frac{\p^2 z}{\p x^2}\\
    (f_x)_y \equiv f_{xy} \equiv f_{12} 
    &\equiv \frac{\p}{\p y}\frac{\p f}{\p x} 
    \equiv \frac{\p^2 f}{\p y \p x}
    \equiv \frac{\p^2 z}{\p y \p x}\\
\end{align*}
\begin{theorem}[Clairaut's Theorem]
    Suppose $f(x,y)$ is defined on a disk $D$ containing the point $(a,b)$.
    If the partial derivatives $f_{xy}$ and $f_{yx}$ are both continuous
    on $D$ then
    \begin{align*}
        f_{xy}(a,b)=f_{yx}(a,b).
    \end{align*}
\end{theorem}


\subsection{Tangent planes and linear approximations}


\subsubsection{Tangent planes}

\begin{definition}
    Consider a surface $S$ with equation $z=f(x,y)$ where $f_x$ and $f_y$
    are both continuous. Then the \B{tangent plane} to the surface $S$
    is given by the equation
    \begin{align*}
        z - z_0 = f_x(x_0, y_0)(x-x_0) + f_y(x_0,y_0)(y-y_0).
    \end{align*}
\end{definition}

\subsubsection{Linear approximation}

\begin{definition}
    The linear function
    \begin{align*}
        L(x,y)=f(a,b)+f_x(a,b)(x-a)+f_y(a,b)(y-a),
    \end{align*}
    is called the \B{linearisation} of $f(x,y)$ at $(a,b)$, and the
    approximation
    \begin{align*}
        f(x,y)\approx f(a,b)+f_x(a,b)(x-a)+f_y(a,b)(y-a)
    \end{align*}
    is called the \B{linear approximation} of $f(x,y)$ at $(a,b)$.\\
    The graph of the linearisation is the tangent plane of $f(x,y)$ at
    $(a,b)$.
\end{definition}
\begin{definition}
    Consider the change in $z=f(x,y)$ as we move from the point
    $(a,b)$ to the point $(a+\Delta x, b+\Delta y)$. The
    corresponding \B{increment} in $z$ is
    \begin{align*}
        \Delta z = f_x(a,b)\Delta x + f_y(a,b)\Delta y 
        + \epsilon_1\Delta x + \epsilon_2\Delta y.
    \end{align*}
    The function $f(x,y)$ is \B{differentiable} if $\Delta z$
    may be expressed as
    \begin{align*}
        \Delta z = f_x(a,b)\Delta x + f_y(a,b)\Delta y
        + \epsilon_1\Delta x + \epsilon_2\Delta y
    \end{align*}
    where $\epsilon_1, \epsilon_2 \to 0$ as 
    $(\Delta x, \Delta y)\to(0,0)$.
\end{definition}
\begin{theorem}
    If the partial derivatives $f_x$ and $f_y$ exist near
    $(a,b)$ and are continuous at $(a,b)$, then $f(x,y)$
    is differentiable at $(a,b)$.
\end{theorem}

\subsubsection{Differentials}

\begin{definition}
    For the function $z=f(x,y)$, \B{differentials} $dx$ and $dy$ are
    defined to be independent variables. The \B{total differential} $dz$
    is then defined as
    \begin{align*}
        dz = f_x(x,y)dx + f_y(x,y)dy 
        = \frac{\p z}{\p x}dx + \frac{\p z}{\p y}dy.
    \end{align*}
\end{definition}


\subsection{The chain rule}


\subsubsection{Formulations}

\begin{theorem}[Chain rule - case 1]
    Suppose that $z=f(x,y)$ is a differentiable function of $x$ and $y$,
    where $x=g(t)$ and $y=h(t)$ are both differentiable functions of $t$.
    Then $z$ is a differentiable function of $t$ and
    \begin{align*}
        \frac{dz}{dt}=
        \frac{\p f}{\p x}\frac{dx}{dt}+\frac{\p f}{\p y}\frac{dy}{dt}
        \equiv \frac{\p z}{\p x}\frac{dx}{dt}+\frac{\p z}{\p y}\frac{dy}{dt}
    \end{align*}
\end{theorem}
\begin{theorem}[Chain rule - case 2]
    Suppose that $z=f(x,y)$ is a differentiable function of $x$ and $y$,
    where $x=g(s,t)$ and $y=h(s,t)$ are both differentiable functions of
    $s$ and $t$. Then
    \begin{align*}
        \frac{\p z}{\p s}=\frac{\p z}{\p x}\frac{\p x}{\p s}
        + \frac{\p z}{\p y}\frac{\p y}{\p s}
        \text{\hs and\hs}
        \frac{\p z}{\p t}=\frac{\p z}{\p x}\frac{\p x}{\p t}
        + \frac{\p z}{\p y}\frac{\p y}{\p t}.
    \end{align*} 
\end{theorem}
\begin{theorem}[Chain rule - general case]
    Suppose that $z$ is a differentiable function of the $n$ variables
    $x_1, x_2, ..., x_n$, and each $x_j$ is a differentiable function of
    the $m$ variables $t_1,t_2, ...,t_m$. Then
    \begin{align*}
        \frac{\p z}{\p t_i}=
        \frac{\p z}{\p x_1}\frac{\p x_1}{\p t_i}
        + \frac{\p z}{\p x_2}\frac{\p x_2}{\p t_i}
        + \cdots
        + \frac{\p z}{\p x_n}\frac{\p x_n}{\p t_i}
    \end{align*} 
\end{theorem}

\subsubsection{Implicit differentiation}

\begin{definition}
    Suppose that $z$ is given implicitly as a function $z=f(x,y)$ by an
    equation of the form
    \begin{align*}
        F(x,y,z)=0.
    \end{align*}
    Provided that $F$ and $f$ are differentiable, use the Chain rule to
    differentiate paritally w.r.t. $x$:
    \begin{align*}
        \frac{\p F}{\p x}\frac{\p x}{\p x}+\frac{\p F}{\p y}\frac{\p y}{\p x}
        + \frac{\p F}{\p z}\frac{\p z}{\p x} &= 0\\
        \frac{\p F}{\p x} + \frac{\p F}{\p z}\frac{\p z}{\p x} &= 0
    \end{align*}
\end{definition}
\begin{theorem}[Implicit Function Theorem]
    If $F(x,y,z)$ is defined within a sphere containing $(a,b,c)$
    where $F(a,b,c)=0, F_z(a,b,c)\not=0$ and $F_x,F_y$ and $F_z$ are
    continuous inside the sphere, then
    \begin{align*}
        F(x,y,z)=0
    \end{align*} 
    defines $z$ as a differentiable function of $x$ and $y$ near the
    point $(a,b,c)$.
\end{theorem}


\subsection{Directional derivatives and the gradient vector}


\subsubsection{Directional derivatives}

\begin{definition}
    The \B{directional derivative} of $f(x,y)$ at $(x_0,y_0)$ in the
    direction of a \emph{unit vector} $\vec u=a\ih + b\jh$ is defined as
    \begin{align*}
        D_{\vec u}f(x_0, y_0)=\lim_{\delta \to 0}
        \frac{f(x_0+\delta a, y_0 + \delta b)-f(x_0, y_0)}{\delta}.
    \end{align*}
\end{definition}
\begin{theorem}
    If $f$ is a differentiable function of $x$ and $y$, then $f$
    has a directional derivative in the direction of any unit vector
    $\vec u=a\ih + b\jh$ and
    \begin{align*}
        D_{\vec u}f(x,y) = af_x(x,y) + bf_y(x,y).
    \end{align*}
\end{theorem}

\subsubsection{Gradient vector}

\begin{definition}
    The \B{gradient} of the scalar function $f(x,y)$ is the vector
    function defined as
    \begin{align*}
        \grad f(x,y)=\frac{\p f}{\p x}\ih + \frac{\p f}{\p y}\jh.
    \end{align*}
\end{definition}
\begin{theorem}
    The directional derivative may be expressed in terms of gradients
    as 
    \begin{align*}
        D_{\vec u}f(x,y)=\grad f(x,y)\cdot \vec u
    \end{align*}
\end{theorem}

\subsubsection{Functions of 3 variables}

\begin{definition}
    The gradient vector of $f(x,y,z)$ is
    \begin{align*}
        \vec\nabla f(x,y,z)=\frac{\p f}{\p x}\ih 
        + \frac{\p f}{\p y}\jh + \frac{\p f}{\p z}\kh
    \end{align*}
\end{definition}

\subsubsection{Maximising the directional derivative}

\begin{theorem}
    Suppose $f$ is a differentiable function of two or three variables.
    The maximum value of the directional derivative $D_{\vec u}f$ is
    $|\grad f|$ and it occurs when $\vec u$ is aligned with the gradient
    vector $\grad f$.
\end{theorem}

\subsubsection{Tangent planes to level surfaces}

\begin{theorem}
    Suppose $S$ is a surface described by the equation $F(x,y,z)=k$ where
    $k\in\R$ is a constant. Let $C$ be any curve that lies on $S$, passing
    through the point $(x_0,y_0,z_0)$. We describe $C$ by the position vector
    function
    \begin{align*}
        \vec r(t) = x(t)\ih + y(t)\jh + z(t)\kh
    \end{align*}
    such that the parameter value $t=t_0$ corresponds to $(x_0, y_0, z_0)$.
    Provided that $x,y,z$ are all differentiable functions of $t$ and $F$
    is a differentiable function of $x,y,z$, it holds that
    \begin{align*}
        \grad F \cdot \dv r(t) = 0.
    \end{align*}
    In particular, at $t=t_0$ we have
    \begin{align*}
        \grad F(x_0, y_0, z_0)\cdot \dv r(t_0) = 0.
    \end{align*}
    Thus the gradient vector at $(x_0, y_0, z_0)$ is perpendicular to the
    tangent vector $\dv r(t)$ for any curve $C$ that passes through 
    $(x_0, y_0, z_0)$.
\end{theorem}
\begin{definition}
    Provided that $\grad F(x_0, y_0, z_0)\not=\vec 0$, we define the
    \B{tangent plane to the level surface} $F(x,y,z)=k$ at $(x_0, y_0, z_0)$
    as the plane that passes through $(x_0, y_0, z_0)$ and has normal vector
    $\grad F(x_0, y_0, z_0)$.
\end{definition}
\begin{theorem}
    The equation of the tangent plane to the level surface may be written as
    \begin{align*}
        F_x(x_0, y_0, z_0)(x-x_0)+F_y(x_0, y_0, z_0)(y-y_0)+F_z(x_0, y_0, z_0)(z-z_0)=0.
    \end{align*}
\end{theorem}
\begin{definition}
    The \B{normal line} to surface $S$ at $(x_0, y_0, z_0)$ is the line
    passing through $(x_0, y_0, z_0)$ that is perpendicular to the tangent
    plane. Thus its direction is given by the gradient vector 
    $\grad F(x_0, y_0, z_0)$.
\end{definition}
\begin{theorem}
    The equation of the normal plane to the level surface may be written as
    \begin{align*}
        \frac{x-x_0}{F_x(x_0, y_0, z_0)}
        =\frac{y-y_0}{F_y(x_0, y_0, z_0)}
        =\frac{z-z_0}{F_z(x_0, y_0, z_0)}.
    \end{align*}
\end{theorem}


\subsection{Maximum and minimum values}


\subsubsection{Functions of 2 variables}

\begin{definition}
    A function $f(x,y)$ has a \B{local maximum} at $(a,b)$ if $f(x,y)\leq f(a,b)$
    for all $(x,y)$ within a disk centred on $(a,b)$. The number $f(a,b)$
    is then called the \B{local maximum value}.
\end{definition}
\begin{definition}
    A function $f(x,y)$ has a \B{local minimum} at $(a,b)$ if $f(x,y)\geq f(a,b)$
    for all $(x,y)$ within a disk centred on $(a,b)$. The number $f(a,b)$
    is then called the \B{local minimum value}.
\end{definition}
\begin{definition}
    If the inequalities in these two definitions hold for all points in the
    domain of $f(x,y)$, then $f(x,y)$ has an \B{absolute maximum} or an
    \B{absolute minimum} at $(a,b)$.
\end{definition}
\begin{theorem}
    If $f(x,y)$ has a local maximum or minimum at $(a,b)$ and the first-order
    partial derivatives of $f(x,y)$ exist there, then
    \begin{align*}
        f_x(a,b) = f_y(a,b) = 0.
    \end{align*}
\end{theorem}
\begin{lemma}
    If the graph of $f(x,y)$ has a tangent plane at a local maximum or minimum
    then this tangent plane must be parallel to the $xy$ plane.
\end{lemma}
\begin{definition}
    A point $(a,b)$ is called a \B{critical point} of $f(x,y)$ if
    $f_x(a,b)=f_y(a,b)=0$, or if one of these partial derivatives does not exist.
\end{definition}
\begin{theorem}[Second Derivatives Test]
    Suppose that the second partial derivatives of $f(x,y)$ are continuous
    on a disk centred at a critical point $(a,b)$ Let
    \begin{align*}
        D = f_{xx}(a,b)f_{yy}(a,b) -\left(f_{xy}(a,b)\right)^2
        = \det\begin{bmatrix}
            f_{xx}(a,b) &f_{xy}(a,b)\\
            f_{xy}(a,b) &f_{yy}(a,b) 
        \end{bmatrix}
    \end{align*}
    \begin{enumerate}
        \item If $D>0$ and $f_{xx}(a,b)>0$ then $f(a,b)$ is a local minimum.
        \item If $D>0$ and $f_{xx}(a,b)<0$ then $f(a,b)$ is a local maximum.
        \item If $D<0$ then $(a,b)$ is a saddle point.
        \item If $D=0$ then the test is inconclusive.
    \end{enumerate}
\end{theorem}

\subsubsection{Absolute maximum and minimum values}

\begin{definition}
    Properties of sets in $\R^2$:
    \begin{itemize}
        \item A point $(a,b)$ is a \B{boundary point} of a set $D$ if
        every disk with centre $(a,b)$ contains points in $D$ and also points
        not in $D$.
        \item A \B{closed set} is one that contains all its boundary points.
        \item A \B{bounded set} is one taht is contained within some disk, i.e.
        one that is finite in extent.
    \end{itemize}
\end{definition}
\begin{theorem}[Extreme Value Theorem]
    If $f(x,y)$ is continuous on a closed, bounded set $D$ in $\R^2$, then
    then $f(x,y)$ attains an absolute maximum value $f(x_1, y_1)$ and an
    absolute minimum value $f(x_2, y_2)$ at some points $(x_1, y_1)$ and
    $(x_2, y_2)$ in $D$. 
\end{theorem}


\subsection{Lagrange multipliers}


\begin{theorem}
    Suppose we seek the local maximum and minimum values of a differentiable
    function $f(x,y)$ along the smooth curve $C$ given by $g(x,y)=0$ where
    $\grad g(x,y)\not=\vec 0$.\\
    If $f(x,y)$ has a local maximum or minumum along $C$ at the point
    $(x_0, y_0)$ then there exists $\lambda\in\R$ such that
    \begin{align*}
        \grad f(x_0, y_0) + \lambda\grad g(x_0, y_0) = \vec 0.
    \end{align*}
\end{theorem}
\begin{theorem}[Method of Lagrange Multipliers]
    The local maximum and minimum values of the differentiable function
    $f(x,y)$ subject to the constraint $g(x,y)=0$ (assuming that these
    max/min values exist and $\grad g(x,y)\not=\vec 0$) are to be found
    among the critical points of the \B{Lagrangian function}
    \begin{align*}
        L(x,y,\lambda)=f(x,y)+\lambda g(x,y).
    \end{align*} 
    The number $\lambda$ is called a \B{Lagrange multiplier}.
\end{theorem}



\section{Multiple integrals}



\subsection{Double integrals over rectangles}


\subsubsection{Volumes}

\begin{definition}
    The \B{double integral} of $f(x,y)$ over the rectangle $R$ is defined as
    \begin{align*}
        \di_R f(x,y)dA = \lim_{m,n\to\infty}\sum_{i=1}^m\sum_{j=1}^n f(x_{ij}^*, y_{ij}^*)\Delta A
    \end{align*}
    and $f(x,y)$ is called \B{integrable} if this limit exists.
\end{definition}
\begin{lemma}
    All continuous functions are integrable, and many bouned but
    discontinuous functions are also integrable.
\end{lemma}


\subsubsection{Average value}


\begin{definition}
    The \B{average value} of a function $f(x,y)$ on a rectangle $R$
    is defined as
    \begin{align*}
        f_{ave} = \frac{1}{A(R)}\di_R f(x,y)dA
    \end{align*}
    where $A(R)$ is the area of $R$.
\end{definition}


\subsubsection{Properties}


\begin{theorem}[Sum of integrals]
    \begin{align*}
        \di_R\left[f(x,y)+g(x,y)\right]dA = \di_R f(x,y)dA + \di_R g(x,y)dA
    \end{align*}
\end{theorem}
\begin{theorem}[Scaled integrals]
    Let $c\in\R$ be a constant. Then
    \begin{align*}
        \di_R cf(x,y)dA = c\di_R f(x,y)dA.
    \end{align*} 
\end{theorem}
\begin{theorem}
    If $f(x,y)\geq g(x,y)$ for all $(x,y)\in R$ then 
    \begin{align*}
        \di_R f(x,y)dA \geq \di_R g(x,y) dA.
    \end{align*}
\end{theorem}

\subsection{Iterated integrals}

\begin{theorem}[Fubini's Theorem]
    If $f(x,y)$ is continuous on the rectangle $R=[a,b]\times[c,d]$ then 
    \begin{align*}
        \di_r f(x,y)dA = \int_a^b \int_c^d f(x,y)\:dydx 
        = \int_c^d \int_a^b f(x,y)\:dxdy.
    \end{align*}
    More generally, this holds if $f(x,y)$ is bounded on $R$ and is
    discontinuous only on a finite number of smooth curves.
\end{theorem}
\begin{lemma}
    Suppose that $f(x,y)$ can be written as $f(x,y)=g(x)h(y)$ 
    on $R=[a,b]\times[c,d]$. Then
    \begin{align*}
        \di_R f(x,y)dA = \int_a^b g(x)dx \int_c^d h(y)dy.
    \end{align*}
\end{lemma}


\subsection{Double integrals over general regions}


\begin{definition}
    Let $D$ be a region in the $xy$ plane of general shape.
    Suppose that $D$ is bounded. Now let $R$ be a rectangle that
    encloses $D$. Then we define a new function $F(x,y)$
    with domain $R$ per
    \begin{align*}
        F(x,y)=\begin{cases}
            f(x,y) &\text{if } (x,y)\in D\\
            0      &\text{if } (x,y)\not\in D
        \end{cases}
    \end{align*}
    Thus the double integral of $f(x,y)$ over $D$ is defined as
    \begin{align*}
        \di_D f(x,y)dA = \di_R F(x,y)dA.
    \end{align*}
\end{definition}
\begin{theorem}
    Suppse that $D$ lies between the graphs of two continuous
    functions of $x$, i.e.
    \begin{align*}
        D = \{(x,y) | a \leq x \leq b, g_1(x)\leq y\leq g_2(x)\}.
    \end{align*}
    Then
    \begin{align*}
        \di_D f(x,y)dA f(x,y)dA = \int_a^ b \int_{g_1(x)}^{g_2(x)} f(x,y)dy dx.
    \end{align*}
\end{theorem}

\subsubsection{Properties}

\begin{theorem}
    If $D=D_1\cup D_2$ where $D_1$ and $D_2$ do not overlap
    except perhaps on their boundaries, then
    \begin{align*}
        \di_D f(x,y) dA = \di_{D_1} f(x,y)dA + \di_{D_2}f(x,y)dA.
    \end{align*}
\end{theorem}
\begin{theorem}
    If $f(x,y)=1$, then 
    \begin{align*}
        \di_D f(x,y)dA = A(D).
    \end{align*}
\end{theorem}
\begin{theorem}
    If $m\leq f(x,y) \leq M$ for all $(x,y)\in D$, then
    \begin{align*}
        mA(D) \leq \di_D f(x,y) dA \leq MA(D).
    \end{align*}
\end{theorem}


\subsection{Double integrals in polar coordinates}


\begin{theorem}
    If $f(x,y)$ is continuous on the \B{polar rectangle} $R$
    given by $a\leq r \leq b$, $\alpha \leq \theta\leq \beta$ where
    $\beta-\alpha \in [0,2\pi)$, then
    \begin{align*}
        \di_R f(x,y)dA = \int_\alpha^\beta \int_a^b
        f(r\cos\theta, r\sin\theta)r\:drd\theta
    \end{align*}
\end{theorem}


\subsection{Triple integrals} 


\begin{definition}
    The \B{triple integral} over a box $B=[a,b]\times[c,d]\times[v,w]$
    is defined as
    \begin{align*}
        \ti_B f(x,y,z)dV = \lim_{l,m,n\to\infty}\sum_{i=1}^l \sum_{j=1}^m \sum_{k=1}^n
        f(x_i,y_j,z_k)\Delta V.
    \end{align*}
\end{definition}
\begin{theorem}[Fubini's Theorem for triple integrals]
    If $f(x,y,z)$ is continuous on the rectangular box $B=[a,b]\times[b,c]\times[w,v]$,
    then 
    \begin{align*}
        \ti_B f(x,y,z) dV = \int_v^w\int_c^d \int_a^b f(x,y,z)\:dxdydx.
    \end{align*} 
    The iterated integrals can be evaluated in any order.
\end{theorem}
\begin{definition}
    The \B{average value} of a function $f(x,y,z)$ over a three-dimensional region $E$
    is defined as
    \begin{align*}
        f_{ave} = \frac{1}{V(E)}\ti_E f(x,y,z)dV.
    \end{align*} 
\end{definition}


\subsection{Triple integrals in cylindrical coordinates}


\begin{definition}
    The point $P$ described by the Cartesian coordinates $(x,y,z)$ is specified by the
    ordered triple $(\rho,\theta,z)$ in the \B{cylindrical coordinate system}. The
    $z$ coordinate is the same in both systems, the other coordinates are related as
    \begin{align*}
        x=\rho\cos\theta,\hs y=\rho\sin\theta,\hs
        \rho^2 = x^2+y^2,\hs\tan\theta = \frac{y}{x}.
    \end{align*}
\end{definition}
\begin{theorem}
    Suppose the function $f(x,y,z)$ is continuous on the 3-dimensional region $E$, where $E$
    may be expressed as
    \begin{align*}
        E=\{(x,y,z) : (x,y,z) \in D, u_1(x,y) \leq z \leq u_2(x,y)\}
    \end{align*}
    where $D$ is specified in therms of polar coordinates as
    \begin{align*}
        \alpha \leq \rho \leq \beta,\:\: h_1(\theta) \leq \rho \leq h_2(\theta).
    \end{align*}
    The formula for \B{triple integration in cylindrical coordinates} is
    \begin{align*}
        \ti_E f(x,y,z) dV = \int_\alpha^\beta \int_{h_1(\theta)}^{h_2(\theta)}
        \int_{u_1(\rho\cos\theta,\rho\sin\theta)}^{u_2(\rho\cos\theta,\rho\sin\theta)}
        f(\rho\cos\theta,\rho\sin\theta,z)\rho\,dz\,dp\,d\theta.
    \end{align*}
\end{theorem}


\subsection{Triple integrals in spherical coordinates}


\begin{definition}
    The point $P$ described by the Cartesian coordinates $(x,y,z)$ is specified by the ordered
    triple $(\rho,\theta,\phi)$ in the \B{spherical coordinate system}. The coordinates are
    related as
    \begin{align*}
        x=\rho\sin\phi\cos\theta,\hs y = \rho \sin\phi\sin\theta,\hs z = \rho\cos\phi
    \end{align*}
    where the radial distance $\phi=\sqrt{x^2+y^2+z^2}$.
\end{definition}
\begin{theorem}
    The formula for \B{triple integration in spherical coordinates} over a region $E$ 
    given by
    \begin{align*}
        E=\{(\rho,\theta,\phi):a\leq\rho\leq b,\alpha\leq\theta\leq\beta, c\leq\phi\leq d\}
    \end{align*}
    is as follows:
    \begin{align*}
        \ti_E f(x,y,z)\:dV = \int_c^d \int_\alpha^\beta \int_a^b f(\rho\sin\phi\cos\theta,\rho\sin\phi\sin\theta,\rho\cos\phi)\rho^2\sin\phi\:d\rho\,d\theta\,d\phi.
    \end{align*}
\end{theorem}


\subsection{Change of variable in multiple integrals}


\subsubsection{Double integrals}

\begin{definition}
    Consider the \B{transformation} from the points $(u,v)$ in the $uv$-plane to the points $(x,y)$
    in the $xy$-plane, represented by
    \begin{align*}
        x = x(u,v),\hs y=y(u,v).
    \end{align*}
    The \B{Jacobian} for this transformation is defined as
    \begin{align*}
        \frac{\partial(x,y)}{\partial(u,v)}
        =\frac{\partial x}{\partial u}\frac{\partial y}{\partial v}
        - \frac{\partial x}{\partial v}\frac{\partial y}{\partial u}
        =\begin{vmatrix}
            \frac{\partial x}{\partial u} &\frac{\partial x}{\partial v}\\
            \frac{\partial y}{\partial u} &\frac{\partial y}{\partial v}
        \end{vmatrix}.
    \end{align*}
    We regard $x$ and $y$ as independent functions of $u$ and $v$ if the Jacobian is nonzero.
\end{definition}
\begin{theorem}
    Suppose that $x=x(u,v), y=y(u,v)$ is a one-to-one transformation that maps a region
    $S$ in the $uv$-plane onto a region $D$ in the $xy$-plane. If the partial derivatives
    of $x(u,v)$ and $y(u,v)$ are continuous in $S$ and the corresponding Jacobian is nonzero,
    then
    \begin{align*}
        \di_D f(x,y)\:dx\,dy = \di_S f(x(u,v), y(u,v))\left|\frac{\partial(x,y)}{\partial(u,v)}\right|\:du\,dv
    \end{align*}
    where $f$ is continuous on $D$.
\end{theorem}

\subsubsection{Triple integrals}

\begin{definition}
    Consider a transformation that maps a region $S$ in $uvw$-space onto a region $D$ in
    $xyz$-space represented by
    \begin{align*}
        x=x(u,v,w),\hs y=y(u,v,w),\hs z=z(u,v,w).
    \end{align*}
    The Jacobian of this transformation is the $3\times3$ matrix determinant
    \begin{align*}
        \frac{\partial(x,y,z)}{\partial(u,v,w)}=\begin{vmatrix}
            \frac{\partial x}{\partial u}&\frac{\partial x}{\partial v}&\frac{\partial x}{\partial w}\\
            \frac{\partial y}{\partial u}&\frac{\partial y}{\partial v}&\frac{\partial y}{\partial w}\\
            \frac{\partial z}{\partial u}&\frac{\partial z}{\partial v}&\frac{\partial z}{\partial w}
        \end{vmatrix}.
    \end{align*}
\end{definition}
\begin{theorem}
    Suppose that $x=x(u,v,w),y=y(u,v,w),z=z(u,v,w)$ is a one-to-one transformation that maps a
    region $S$ in $uvw$-space onto a region $D$ in $xyz$-space. If the partial derivatives of
    $x$, $y$ and $z$ are continuous in $S$ and the corresponding Jacobian is nonzero,
    then
    \begin{align*}
       \ti_D f(x,y,z)\:dx\,dy\,dz = \ti_S f(x(u,v,w), y(u,v,w), z(u,v,w))\left|\frac{\partial(x,y,z)}{\partial(u,v,w)}\right|\:dv\,du\,dw. 
    \end{align*}
\end{theorem}



\section{Vectors and calculus}



\subsection{Vector fields}

\begin{definition}
    Let $D\subseteq \R^n$. A \B{vector field on $\R^n$} is a function $\vec F$ that assigns
    to each point $\vec p\in D$ a $n$-dimensional vector $\vec F(\vec p)$.
\end{definition}
\begin{definition}
    A vector field $\vec F$ is called \B{conservative} if there exists some scalar function
    $f$ such that
    \begin{align*}
        \vec F = \grad f.
    \end{align*}
    If $\vec F$ is conservative then $f$ is called a \B{potential function} of $\vec F$.
\end{definition}


\subsection{Line integrals}


\subsubsection{Line inetrals in 2 dimension}

\begin{definition}
    Let $C$ be the curve in the $xy$-plane described by the position vector equation
    \begin{align*}
        \vec r(t) = x(t)\ih + y(t)\jh
    \end{align*}
    where the parameter $t\in[a,b]$. Assume that $C$ is smooth. 
    If $f(x,y)$ is continuous on $C$, then the \B{line integral of $f$ along $C$} is
    defined as
    \begin{align*}
        \int_C f(x,y) ds = \lim_{n\to\infty} \sum_{i=1}^n f(x(t_i^*), y(t_i^*))\Delta s_i.
    \end{align*}
\end{definition}
\begin{theorem}
    The line integral of $f$ along the curve $C$ may be caculated as
    \begin{align*}
        \int_C f(x,y)ds = \int_a^b f(x(t), y(t))\sqrt{\left(\frac{dx}{dt}\right)^2 + 
        \left(\frac{dy}{dt}\right)^2}dt.
    \end{align*}
\end{theorem}
\begin{theorem}
    If $C$ is a \B{piecewise-smooth curve}, i.e. $C$ is a union of a finite number of smooth
    curves $C_1, C_2, ..., C_n$ where the terminal point of $C_{i-1}$ is the starting point
    of $C_i$, then the integral of $f(x,y)$ along $C$ is
    \begin{align*}
        \int_C f(x,y)ds = \sum_{i=1}^n \int_{C_i} f(x,y)ds.
    \end{align*}
\end{theorem}
\begin{lemma}
    If a curve $C$ is parameterised by $t\in[a,b]$ such that its start point is $A$
    (at $t=a$) and its end point is $B$ (at $t=b$), then
    \begin{align*}
        \int_{-C} f(x,y)dx = -\int_C f(x,y)dx,\hs 
        \int_{-C}f(x,y)dy = -\int_C f(x,y)dy
    \end{align*}
    This is not the case if we integrate with respect to the arc length, i.e.
    \begin{align*}
        \int_{-C}f(x,y)ds = \int_C f(x,y)ds.
    \end{align*}
\end{lemma}

\subsubsection{Line integrals in 3 dimensions}

\begin{definition}
    If $f(x,y,z)$ is continuous on a smooth curve $C$ described by the
    position vector equation
    \begin{align*}
        \vec r(t) = x(t)\ih + y(t)\jh + z(t)\kh
    \end{align*}
    wher the parameter $t\in[a,b]$, then the \B{line integral of $f$ along $C$}
    is defined as
    \begin{align*}
        \int_C f(x,y,z)ds = \lim_{n\to\infty}\sum_{i=1}^n f(x(t_i^*), y(t_i^*), z(t_i^*))\Delta s_i
    \end{align*}
    which may be evaluated as
    \begin{align*}
        \int_C f(x,y,z)ds = \int_a^b f(x(t), y(t), z(t))\sqrt{\left(\frac{dx}{dt}\right)^2 + \left(\frac{dy}{dt}\right)^2 + \left(\frac{dz}{dt}\right)^2}dt.
    \end{align*}
\end{definition}
\begin{lemma}
    More compactly,
    \begin{align*}
        \int_C f ds = \int_a^b f(\vec r(t))|\dv r(t)|dt.
    \end{align*}
\end{lemma}

\subsubsection{Line integrals of vector fields}

\begin{definition}
    If the vecotr field $\vec F$ is continuous on $C$, then the
    \B{line integral of $\vec F$ along $C$} is defined as
    \begin{align*}
        \int_C \vec F\cdot d\vec r = \int_a^b \vec F(\vec r(t))\cdot \dv r(t)dt
        = \int_C \vec F\cdot \vec T ds.
    \end{align*}
\end{definition}
\begin{theorem}
    Suppse that the vector field $\vec F$ on $\R^3$ is given in component
    form as
    \begin{align*}
        \vec F(x,y,z) = P(x,y,z)\ih + Q(x,y,z)\jh + R(x,y,z)\kh.
    \end{align*}
    Then
    \begin{align*}
        \int_C \vec F\cdot d\vec r = \int_C P\:dx + Q\:dy + R\:dz.
    \end{align*}
\end{theorem}


\subsection{The Fundamental Theorem for line integrals}


\begin{theorem}
    Let $C$ be a smooth curve specified by the position vector function 
    $\vec r(t)$, $t\in[a,b]$. If $f$ is a diffferentiable function
    whose gradient vector $\grad f$ is continuous on $C$, then
    \begin{align*}
        \int_C \grad f\cdot d\vec r = f(\vec r(b)) - f(\vec r(a)).
    \end{align*}
\end{theorem}
\begin{lemma}
    Suppose that $C_\alpha$ and $C_\beta$ are two piecewise-smooth curves
    - which are called \B{paths} - that have the same start point and end
    point. Then
    \begin{align*}
        \int_{C_\alpha} \grad f \cdot d\vec r = \int_{C_\beta} \grad f \cdot d\vec r
    \end{align*}
    provided $\grad f$ is continuous.
\end{lemma}

\subsubsection{Independence of path}

\begin{definition}
    If $\vec F$ is a continous vector field with domain $D$, we say that the
    line integral $\int_C\vec F\cdot d\vec r$ is
    \B{independent of path} if
    \begin{align*}
        \int_{C_\alpha} \vec F\cdot d\vec r = \int_{C_\beta}\vec F\cdot d\vec r
    \end{align*}
    for any two paths $C_\alpha, C_\beta$ in $D$ that have the same start and
    end points.
\end{definition}
\begin{theorem}
    Line integrals of conservative vector fields are independent of path.
\end{theorem}
\begin{definition}
    A \B{closed} curve is one for which the start point coincides with the end point.
\end{definition}
\begin{theorem}
    \begin{align*}
        \int_C \vec F\cdot d\vec r \text{ is independent of path in $D$}
        \Leftrightarrow
        \int_C \vec F\cdot d\vec r = 0 \text{ for every closed path $C$ in $D$}
    \end{align*}
\end{theorem}
\begin{definition}
    A set $D$ is \B{connected} if any two points in $D$ can be joined by a
    path that lies in $D$.
\end{definition}
\begin{theorem}
    Let $\vec F$ be a vector field that is continuous on an open connected
    region $D$. If $\int_C\vec F\cdot d\vec r$ is independent of path in $D$
    then $\vec F$ is a conservative vector field on $D$.
\end{theorem}
\begin{theorem}
    If $\vec F(x,y) = P(x,y)\ih + Q(x,y)\jh$ is a conservative vector field,
    where $P(x,y)$ and $Q(x,y)$ have continuous first-order partial derivatives
    on a domain $D$, then throughout $D$
    \begin{align*}
        \frac{\partial P}{\partial y}=\frac{\partial Q}{\partial x}.
    \end{align*}
\end{theorem}
\begin{definition}
    \B{Simple} properties:
    \begin{itemize}
        \item A \B{simple curve} is one that does not intersect itself anywhere
              between its start and end.
        \item A \B{simply-connected region} in the plane is a connected region $D$
              such that every simple closed curve in $D$ encloses only points that 
              are in $D$.
    \end{itemize}
\end{definition}
\begin{theorem}
    If $\vec F(x,y)=P(x,y)\ih + Q(x,y)\jh$ is a vector field on an open simply-connected
    region $D$, where $P(x,y)$ and $Q(x,y)$ have continuous first-order partial
    derivatives and
    \begin{align*}
        \frac{\partial P}{\partial y}=\frac{\partial Q}{\partial x}
    \end{align*}
    throughout $D$, then $\vec F(x,y)$ is conservative.
\end{theorem}


\subsection{Green's Theorem}


\begin{theorem}[Green's Theorem in the plane]
    Let $C$ be a positively-oriented, piecewise-smooth, simple closed curve in
    the $xy$-plane and let $D$ be the region enclosed by $C$. If $P(x,y)$ and
    $Q(x,y)$ have continuous partial derivatives on an open region that contains
    $D$, then
    \begin{align*}
        \int_C P\:dx + Q\:dy = \di_D \left(\frac{\partial Q}{\partial x}-\frac{\partial P}{\partial y}\right)dA.
    \end{align*} 
\end{theorem}


\subsection{Curl and Divergence}


\subsubsection{Curl}
\begin{definition}
    If $\vec F(x,y,z) = P(x,y,z)\ih + Q(x,y,z)\jh + R(x,y,z)\kh$ is a vector field
    on $\R^3$, then the \B{curl} of $\vec F$ is defined by the vector function
    \begin{align*}
        \curl\vec F = \left(\frac{\partial R}{\partial y}-\frac{\partial Q}{\partial z}\right)\ih
        +\left(\frac{\partial P}{\partial z}-\frac{\partial R}{\partial x}\right)\jh
        +\left(\frac{\partial Q}{\partial x}-\frac{\partial P}{\partial y}\right)\kh.
    \end{align*} 
    We can write curl as
    \begin{align*}
        \curl\vec F = \grad \times \vec F.
    \end{align*}
\end{definition}
\begin{theorem}
    If the scalar function $f(x,y,z)$ has continuous second order partial derivatives,
    then
    \begin{align*}
        \curl(\grad f) = \vec 0.
    \end{align*}
\end{theorem}
\begin{corollary}
    If $\vec F$ is conservative then $\curl\vec F = \vec 0$. The converse is true
    provided that $\vec F$ is defined on all $\R^3$.
\end{corollary}
\begin{theorem}[Laplace's equation]
    \begin{align*}
        \nabla^2 f = 0.
    \end{align*}
\end{theorem}
\begin{definition}
    The Laplace operator can be applied to a vector field $\vec F$ where 
    \begin{align*}
        \vec F(x,y,z) =P(x,y,z)\ih + Q(x,y,z)\jh + R(x,y,z)\kh
    \end{align*}
    per 
    \begin{align*}
        \nabla^2 \vec F = \nabla^2 P\ih + \nabla^2 Q\jh + \nabla^2 R\kh.
    \end{align*}
\end{definition}

\subsubsection{Divergencve}

\begin{definition}
    If $\vec F(x,y,z)=P(x,y,z)\ih + Q(x,y,z)\jh + R(x,y,z)\kh$ is a vector field on $\R^3$, then
    the \B{divergence} of $\vec F$ is defined by the scalar function
    \begin{align*}
        \div \vec F = \frac{\partial P}{\partial x}+\frac{\partial Q}{\partial y} + \frac{\partial R}{\partial z}.
    \end{align*}
\end{definition}

\subsubsection{Vector form of Green's Theorem}

\begin{theorem}
    Suppose that the plane region $D$, its boundary cruve $C$, and the functions
    $P(x,y)$ and $Q(x,y)$ satisfy the hypotheses of \emph{Green's Theorem}. Now
    consider the vector field $\vec F = P\ih + Q\jh$ which has the line integral
    \begin{align*}
        \int_C\vec F \cdot d\vec r = \int_C Pdx + Qdy.
    \end{align*}
    If we regard $\vec F$ as a vector field on $\R^3$ with thrid component $0$, 
    we can write Green's Theorem as
    \begin{align*}
        \int_C\vec F \cdot d\vec r = \di_D\left(\curl\vec F(x,y)\right)\cdot\kh dA.
    \end{align*}
    If the boundary curve $C$ is specified by the position vector function
    \begin{align*}
        \vec r(t) = x(t)\ih + y(t)\jh
    \end{align*}
    for $t\in[a,b]$, then Greens Theorem can be written as
    \begin{align*}
        \int_C \vec F\cdot \vec{\hat n} ds = \di_D \div\vec F(x,y)dA.
    \end{align*}
\end{theorem}


\subsection{Parametric descriptions of surfaces and their areas}


\begin{definition}
    A surface $S$ in $\R^3$ may be described by the position vector function
    \begin{align*}
        \vec r(u,v) = x(u,v)\ih + y(u,v)\jh + z(u,v)\kh
    \end{align*}
    in terms of two parameters $u$ and $v$ which lie in a region $D$ in the $uv$-plane.
    The \B{parametric equations}
    \begin{align*}
        x = x(u,v),\hs y=y(u,v),\hs z=z(u,v)
    \end{align*}
    describe all point on the surface $S$ as $(u,v)$ moves through $D$.
\end{definition}

\subsubsection{Tangent planes}

\begin{definition}
    Consider the point $P_0$ with position vector $\vec r(u_0, v_0)$. This point lies
    on the surface traced out by the position vector function
    \begin{align*}
        \vec r(u,v) = x(u,v)\ih + y(u,v)\jh + z(u,v)\kh.
    \end{align*}
    If $\vec r_u \times\vec r_v\not=\vec 0$ then the surface is called \B{smooth}.\\
     For a smooth surface, the \B{tangent plane} at $P_0$ is the plane containing
              $\vec r(u_0,v_0)+\vec r_u(u_0,v_0)$ and $\vec r(u_0,v_0)+\vec r_v(u_0,v_0)$ and the vector
              $\vec r_u(u_0,v_0)\times \vec r_v(u_0,v_0)$ is normal to the tangent plane.

\end{definition}

\subsubsection{Surface area}

\begin{definition}
    If a smooth surface $S$ is described by the position vector function
    \begin{align*}
        \vec r(u,v) = x(u,v)\ih + y(u,v)\jh + z(u,v)\kh,\hs (u,v)\in D
    \end{align*}
    and $S$ is covered just once as $(u,v)$ ranges through the parameter domain $D$,
    then the \B{surface area} of $S$ is defined as
    \begin{align*}
        A(S)=\di_D |\vec r_u \times \vec r_v|\:dA.
    \end{align*}
\end{definition}
\begin{theorem}
    Consider the surface $S$ with equation $z=f(x,y)$ where $x,y$ lie in $D$ and $f$
    has continuous partial derivatives. Then the surface area of $S$ is
    \begin{align*}
        A(S) = \di_D \sqrt{1+\left(\frac{\partial z}{\partial x}\right)^2 + \left(\frac{\partial z}{\partial y}\right)^2}dA.
    \end{align*}
\end{theorem}


\subsection{Surface integrals}


\begin{definition}
    \label{si_def}
    Consider the function $f(x,y,z)$ defined on the surface $S$ which is described by the
    position vector function
    \begin{align*}
        \vec r(u,v) = x(u,v)\ih + y(u,v)\jh + z(u,v)\kh,\hs (u,v)\in D.
    \end{align*}
    Then the \B{surface integral} of $f(x,y,z)$ over the surface $S$ is defined as
    \begin{align*}
        \di_S f(x,y,z)dS = \lim_{m,n\to\infty}\sum_{i=1}^m\sum_{j=1}^n f(P_{ij}^*)\Delta S_{ij}.
    \end{align*}
\end{definition}
\begin{theorem}
    The surface integral in definition \ref{si_def} may be evaluated by
    \begin{align*}
        \di_S f(x,y,z)dS = \di_D f(\vec r(u,v)) |\vec r_u\times \vec r_v|dA.
    \end{align*}
\end{theorem}

\subsubsection{Graphs}

\begin{theorem}
    Consider the surface $S$ which is given by $z=g(x,y)$. Then the surface integral 
    of $f(x,y,z)$ has the formula
    \begin{align*}
        \di_S f(x,y,z)dS = \di_D f(x,y,g(x,y))\sqrt{1+\left(\frac{\partial z}{\partial x}\right)^2+\left(\frac{\partial z}{\partial y}\right)^2}\:dA
    \end{align*}
\end{theorem}

\subsubsection{Oriented surfaces}

\begin{definition}
    A smooth surface $S$ in $\R^3$ is \B{orientable} if there exists a unit normal vector
    field $\hat n(x,y,z)$ defined on $S$ that varies continuously as $(x,y,z)$ ranges over $S$
    and is everywhere normal to $S$.\\
    Any such unit nromal vector field $\hat n(x,y,z)$ determines an \B{orientation} of $S$.
    The side which $\hat n(x,y,z)$ points to is called the \B{positive side}; the other side is the
    \B{negative side}.
\end{definition}
\begin{definition}
    An \B{oriented surface} is a smooth surface together with a particular choice of unit
    normal vector field $\hat n(x,y,z)$.
\end{definition}
\begin{lemma}
    For the surface described by the graph $z=g(x,y)$, a natural orientation is provided by
    the unit normal fector field
    \begin{align*}
        \hat n(x,y,z)=\frac{-\frac{\partial g}{\partial x}\ih -\frac{\partial g}{\partial y}\jh + \kh}{\sqrt{1+\left(\frac{\partial g}{\partial x}\right)^2 + \left(\frac{\partial g}{\partial y}\right)^2}}.
    \end{align*}
\end{lemma}
\begin{lemma}
    For the surfaces described by the position vector function $\vec r(u,v)$ in parametric form,
    a natural orientation is provided by the unit normal vector field
    \begin{align*}
        \hat n(u,v) = \frac{\vec r_u \times \vec r_v}{|\vec r_u \times \vec r_v|}.
    \end{align*}
\end{lemma}
\begin{definition}
    For a \B{closed surface} (i.e., a surface without boundary):
    \begin{itemize}
        \item the \B{positive orientation} is one for which the unit normal vectors point outwards;
        \item the \B{negative orientation} is one for which the unit normal vector points inwards.
    \end{itemize}
\end{definition}
\begin{definition}
    A piecewise smooth surface is orientable if, whenever two smooth component surfaces join
    at a boundary curve $C$, they induce opposite orientations along $C$.
\end{definition}

\subsubsection{Surface integrals of vector fields}

\begin{definition}
    If $\vec F(x,y,z)$ is a continuous vector field defined on an oriented surface $S$ with
    unit normal vector field $\hat n(x,y,z)$, then the \B{surface integral of $\vec F$ over 
    $S$} is
    \begin{align*}
        \di_S \vec F\cdot d\vec S = \di_S \vec F\cdot \hat n\:dS.
    \end{align*}
    This integral is also called the \B{flux} of $\vec F$ over $S$.
\end{definition}


\subsection{Stokes' Theorem}

\begin{theorem}[Stokes' Theorem]
    Let $S$ be an oriented piecewise-smooth surface that is bounded by a simple closed
    piecewise smooth boundary curve $C$ with orientation inherited from $S$. Let $\vec F$
    be a vector field whose components have continuous partial derivatives on an open
    region in $\R^3$ that contains $S$. Then
    \begin{align*}
        \int_C \vec F\cdot d\vec r = \di_S \curl \vec F\cdot d\vec S.
    \end{align*} 
\end{theorem}
\begin{corollary}
    If $S_1$ and $S_2$ are oriented surfaces with the same oriented boundary $C$ then
    \begin{align*}
        \di_{S_1}\curl \vec F\cdot d\vec S = \int_C \vec F\cdot d\vec r = \di_{S_2} \curl\vec F\cdot d\vec S
    \end{align*}
    where $S_{1,2}$ and $\vec F$ satisfy the hypotheses of Stokes' Theorem.
\end{corollary}


\subsection{The Divergence Theorem}


\begin{definition}
    A \B{simple} 3-dimensional region $E$ may be specified in three different ways:
    \begin{align*}
        E=\begin{cases}
            \{(x,y,z)|(x,y)\in D_{xy}, u_1(x,y)\leq z \leq u_2(x,y)\}\\
            \{(x,y,z)|(y,z)\in D_{yz}, u_3(y,z)\leq x \leq u_4(y,z)\}\\
            \{(x,y,z)|(x,z)\in D_{xz}, u_5(x,z)\leq y \leq u_6(x,z)\}
        \end{cases}
    \end{align*}
    where $u_{1,2,3,4,5,6}$ are continuous functions and $D_{xy,yz,xz}$ are the projections
    of $E$ onto the $xy$-,$yz$-,$xz$-planes repsectively.
\end{definition}
\begin{theorem}(The Diveregence Theorem)
    Let $E$ be a simple 3-dimensional region whose boundary surface $S$ has positive
    orientation. Let $\vec F$ be a vector field whose component functions have continuous
    partial derivatives on an open region that contains $E$. Then
    \begin{align*}
        \di_S \vec F\cdot d\vec S = \ti_E \div\vec F\:dV.
    \end{align*}
\end{theorem}



\section{Differential equation classifications}



\begin{definition}
    A \B{differential equation} is one invovling one or more derivatives of an unknown function.
\end{definition}

\begin{definition}
    Any function which satisfies a differential equation on an interval is called a \B{soltution} on that interval.
\end{definition}

\begin{definition}
    An \B{ordinary differential equation (ODE)} involves derivatives with respect to only one variable.
\end{definition}

\begin{definition}
    A \B{partial differential equation (PDE)} involves derivatives with respect to more than one variable.
\end{definition}

\begin{definition}
    The \B{order} of a differential equation is the order of the highest-order derivative present.
\end{definition}

\begin{definition}
    A $n$th-order \B{linear} ODE has the general form
    \begin{align*}
        a_n(x) \frac{d^ny}{dx^n}+a_{n-1}(x)\frac{d^{n-1}y}{dx^{n-1}}+\cdots a_1(x)\frac{dy}{dx}+a_0(x)y(x)= f(x)
    \end{align*}
    where $a_n(x), a_{n-1}(x), ..., a_0(x), f(x)$ are functions of $x$.
    \begin{itemize}
        \item If $f(x)=0$ the ODE is \B{homogeneous},
        \item if $f(x)\not=0$ the ODE is \B{nonhomogeneous}.
    \end{itemize}
\end{definition}

\begin{definition}
    An \B{initial-value problem} comprises
    \begin{itemize}
        \item a differential equation
        \item prescribed values for the solution and enough of its derivatives at a paritcular point to yield a solution.
    \end{itemize}
\end{definition}



\section{First-order ODEs}



\setcounter{subsection}{1}
\subsection{Linear equations}


\begin{theorem}
    The linear ODE
    \begin{align*}
        \frac{dy}{dx}+P(x)y = Q(x)
    \end{align*}
    may be solved by the \B{method of integrating factors}; i.e.
    \begin{itemize}
        \item multiply both sides of the ODE by the \B{integrating factor} $e^{\int P(x)dx}$ to give\begin{align*}
            e^{\int P(x)dx}\left[\frac{dy}{dx}+P(x)y\right]&= e^{\int P(x)dx}Q(x)\\
            \Leftrightarrow \frac{d}{dx}\left(e^{\int P(x)dx}y\right)&=e^{\int P(x)dx}Q(x)
        \end{align*}
        \item then integrating both sides w.r.t. $x$ yields \begin{align*}
            y = \frac{1}{e^{\int P(x)dx}}\left[\int e^{\int P(x)dx}Q(x)dx + C\right].
        \end{align*}
    \end{itemize}
\end{theorem}


\subsection{Separable equations}


\begin{theorem}
    A \B{separable} first-order ODE can be written in the form
    \begin{align*}
        \frac{dy}{dx}=g(x)h(y)
    \end{align*}
    and its solution can be found by integrating per
    \begin{align*}
        \int \frac{1}{h(y)}\frac{dy}{dx}dx &= \int g(x)dx, \\
        \int \frac{1}{h(y)}dy &= \int g(x)dx,
    \end{align*}
    if $h(y)\not=0$.
\end{theorem}


\subsection{Homogeneous equations}


\begin{theorem}
    A \B{homogeneous} first-order ODE can be written in the form
    \begin{align*}
        \frac{dy}{dx} = g\left(\frac{y}{x}\right)
    \end{align*}
    and its solution can be found by
    \begin{itemize}
        \item changing the dependent variable to $v(x)=\frac{y}{x}$,
        \item then using separation of variables.
    \end{itemize}
\end{theorem}


\subsection{Exact equations}


\begin{definition}
    The ODE
    \begin{align*}
        M(x,y)+N(x,y)\frac{dy}{dx}
    \end{align*}
    is called \B{exact} if there exists a function $\psi(x,y)$ such that
    \begin{align*}
        \frac{\p \psi}{\p x}=M(x,y)\hs\text{and}\hs\frac{\p\psi}{\p y}=N(x,y).
    \end{align*}
\end{definition}

\begin{theorem}
    If an ODE is exact, then it may be written as
    \begin{align*}
        \frac{\p\psi}{\p x}+\frac{\p\psi}{\p y}\frac{dy}{dx}&=0\\
        \Leftrightarrow \frac{d}{dx}\psi(x,y)&=0.
    \end{align*}
    Integrating gives the solution implicitly as
    \begin{align*}
        \psi(x,y)=C.
    \end{align*}
\end{theorem}

\begin{theorem}
    If $M,N$ and their partial derivatives w.r.t. $x,y$ are continuous
    on a domain $R:\alpha<x<\beta$, $\gamma<y<\delta$, then the ODE is
    exact on $R$ if and only if
    \begin{align*}
        \frac{\p M}{\p y}=\frac{\p N}{\p x}
    \end{align*}
    at each point in $R$.
\end{theorem}


\subsubsection{Integrating factors}


\begin{theorem}
    The ODE
    \begin{align*}
        \mu(x,y)M(x,y)+\mu(x,y)N(x,yu)\frac{dy}{dx}=0
    \end{align*}
    is exact if and only if
    \begin{align*}
        \frac{\p}{\p y}(\mu M)=\frac{\p}{\p x}(\mu N),
    \end{align*}
    i.e., if and only if
    \begin{align*}
        M\mu_y - N\mu_x + (M_y + N_x)\mu = 0.
    \end{align*}
\end{theorem}

\begin{theorem}
    If 
    \begin{align*}
        \frac{M_y-N_x}{N}=f(x),
    \end{align*}
    then an integrating factor $\mu(x)$
    can be found by solving the linear and separable ODE
    \begin{align*}
        \frac{d\mu}{dx}=\frac{M_y-N_x}{N}\mu.
    \end{align*}
\end{theorem}


\subsection{Existence and uniqueness}


\begin{theorem}
    If the functions $P(t)$, $Q(t)$ are continuous on an open interval
    $\alpha<t<\beta$ containing the point $t_0$, then there exists a
    unique function $y=\phi(t)$ which satisfies the linear ODE
    \begin{align*}
        \frac{dy}{dt}+P(t)y=Q(t)
    \end{align*}
    for all $t\in(\alpha,\beta)$, and that satisfies the initial condition.
\end{theorem}

\begin{theorem}
    Let $f(t,y)$ and $\p f/\p y$ be continuous on a domain $t\in(\alpha,\beta)$,
    $y\in(\gamma,\delta)$ containing the point $(t_0, y_0)$. Then, in some
    interval $(t_0-h, t_0+h)$ contained in $(\alpha, \beta)$, there exists
    a unique solution $y=\phi(t)$ of the IVP
    \begin{align*}
        \frac{dy}{dx}=f(t,y),\hs y(t_0) = y_0.
    \end{align*}
\end{theorem}


\subsection{Numerical approach}


\begin{theorem}
    Suppose we know that the IVP
    \begin{align*}
        \frac{dy}{dt} = f(t,y),\hs y(t_0) = y_0
    \end{align*}
    has a unique solution $y=\phi(t)$ on some interval containing $t_0$.\\
    An approximate numerical solution ot the IVP can be found using
    numerical methods, i.e., for a sequence of points
    \begin{align*}
        t_0,\hs t_1=t_0+h,\hs t_2=t_1+h,\hs...
    \end{align*}
    sparated by the \B{step size} $h>0$, we seek approximate values
    $y_n$ for $\phi(t_n)$.
\end{theorem}

\begin{theorem}[Euler's method]
    The solution curve $y=\phi(t)$ is approximated by a sequence
    of straight line segments, joined end to end, where the gradient
    of each segment is given by $f(t,y)$ at the end of the previous
    segment, i.e.
    \begin{align*}
        y_1 = y_0+f(t_0, y_0)h,\hs y_2=y_1+f(t_1,y_1)h,\hs ...
    \end{align*} 
\end{theorem}



\section{Second-order linear ODEs}



\subsection{General theory}


\begin{theorem}[Existence and Uniqueness Theorem]
    Consider the IVP
    \begin{align*}
        y'' + p(t)y'+q(t)y=g(t),\hs y(t_0)=\alpha,\hs y'(t_0)=\beta
    \end{align*}
    where $p$, $q$, $g$ are continuous on an open interval $I$ containing
    the point $t_0$. There is a unique solution $y=\phi(t)$ of this IVP
    and the solution exists throughout $I$.
\end{theorem}
\begin{definition}
    Two functions $y_1$ and $y_2$ arre \B{linearly dependent} on an
    interval $I$ if there exists two constants $C_1$ and $C_2$, not
    both zero, such that
    \begin{align*}
        C_1y_1(t)+C_2y_2(t)=0
    \end{align*}
    for all $t\in I$.\\
    The functions are \B{linearly independent} on $I$ if they are not
    linearly dependent.
\end{definition}
\begin{definition}
    The \B{Wronskian} of two functions $y_1$ and $y_2$ is defined as the
    determinant
    \begin{align*}
        W(y_1, y_2) = \begin{vmatrix}
            y_1 & y_2 \\
            y'_1& y'_2\\
        \end{vmatrix} = y_1y'_2 - y'_1y_2.
    \end{align*}
\end{definition}
\begin{theorem}
    Let $y_1(t)$ and $y_2(t)$ be differentiable functions on an open
    interval $I$:
    \begin{enumerate}
        \item If $W(y_1,y_2)(t_0)\not=0$ at some point $t_0\in I$ then $y_1$, $y_2$ are linearly independent on $I$.
        \item If $y_1$, $y_2$ are linearly dependent on $I$ then $W(y_1, y_2)(t)=0$ for all $t\in I$.
    \end{enumerate}
\end{theorem}
\begin{theorem}[Abel's Theorem]
    Suppose that $y_1$, $y_2$ are two solutions to the homogeneous ODE
    \begin{align*}
        y'' + p(t)y' + q(t)y = 0
    \end{align*} 
    where $p$, $q$ are continuous on an open interval $I$. Then the
    Wronskian
    \begin{align*}
        W(y_1, y_2) = C\exp \left[-\int p(t)\:dt\right]
    \end{align*}
    where the constant $C$ depends on $y_1$, $y_2$ but not $t$.
    Furthermore, there are only two possibilities:
    \begin{align*}
        W(y_1,y_2)&=0\hs \forall t \in I, \text{ or}\\
        W(y_1,y_2)&\not=0\hs \forall t \in I.
    \end{align*}
\end{theorem}
\begin{theorem}
    Suppose that $y_1$, $y_2$ are two solutions of the homogeneous ODE
    \begin{align*}
        y''+p(t)y'+q(t)y =0
    \end{align*}
    where $p$, $q$ are continuous on an open interval $I$. Then
    \begin{enumerate}
        \item $y_1$, $y_2$ are linearly dependent on $I$ $\Leftrightarrow$ $W(y_1,y_2)(t)=0$ for all $t\in I$,
        \item $y_1$, $y_2$ are linearly independent on $I$ $\Leftrightarrow$ $W(y_1,y_2)(t)\not=0$ for all $t\in I$.
    \end{enumerate}
\end{theorem}
\begin{theorem}
    Suppse that $y_1$, $y_2$ are two \B{linearly independent} solutions to the
    homogeneous ODE
    \begin{align*}
        y''+p(t)y'+q(t)y =0
    \end{align*}
    where $p$, $q$ are contiuous. Then every solution to the ODE has the
    form
    \begin{align*}
        y = C_1y_1(t)+C_2y_2(t)
    \end{align*}
    where $C_1$, $C_2$ are constants.
\end{theorem}
\begin{definition}
    The solution
    \begin{align*}
        y=C_1y_1(t)+C_2y_2(t)
    \end{align*}
    is called the \B{general solution}; every \B{particular solution}
    to the homogeneous ODE can be found by assigning values to the 
    constants $C_1$ and $C_2$.
\end{definition}
\begin{definition}
    The functions $y_1$ and $y_2$ in the general solution $y=C_1y_2+C_2y_2$
    are said to form a \B{fundamental set of solutions}.
\end{definition}
\begin{lemma}
    If we know one solution $y_1(t)$ to a linear homogeneous ODE, another
    linearly independent solution may be found by substituting $y=v(t)y_1(t)$.
    This leads to a first-order seperable ODE for $v'(t)$ from which
    $v(t)$ may be found.\\
    This is called the \B{Method of Reduction of Order}.
\end{lemma}


\subsection{Linear homogeneous ODEs with constant coefficients}

\begin{theorem}
Consider the linear homogeneous ODE with constant coefficients
\begin{align}
    \label{linode}
    ay'' + by' + cy = 0
\end{align}
where $a$, $b$, $c$ are real-valued constants and $a\not=0$.
Then 
\begin{align*}
    y(t)=e^{rt}
\end{align*}
is a solution to (\ref{linode}) provided that $r$ is a root of the
\B{characteristic equation}
\begin{align*}
    ar^2 + br + c = 0.
\end{align*}
\end{theorem}
\begin{theorem}
    If the characteristic equation has real-valued and distinct roots,
    then the general solution to (\ref{linode}) is
    \begin{align*}
        y(t)=C_1\exp(r_1 t) + C_2 \exp(r_2 t).
    \end{align*}
\end{theorem}
\begin{theorem}
    If the characteristic equation has equal roots, then $r=-b/2a$ and the
    general solution to (\ref{linode}) is
    \begin{align*}
        y(t)=C_1\exp\left(rt\right)+C_2t\exp(rt).
    \end{align*}
\end{theorem}
\begin{theorem}
    If the characteristic equation has complex-valued roots, then the
    roots must be the complex-conjugate pair
    \begin{align*}
        r_1 = \alpha + i\beta,\hs r_2 = \alpha - i\beta
    \end{align*}
    where $\alpha,\beta\in\R$ and $i=\sqrt{-1}$. The general soltuion
    to (\ref{linode}) then is
    \begin{align*}
        y(t) = \exp(\alpha t)\left(C_1 \cos\left(\alpha t\right)+C_2\sin\left(\beta t\right)\right).
    \end{align*}
\end{theorem}


\subsection{Linear nonhomogeneous ODEs with constant coefficients}


\begin{definition}
    Consider the linear nonhomogeneous ODE with constant coefficients
    \begin{align}
        \label{linoden}
        ay'' + by' + cy = g(t)
    \end{align}
    where $a$, $b$, $c$ are real-valued constants and $a\not=0$. Then 
    the corresponding homogeneous ODE (\ref{linode}) is called a
    \B{complementary equation}.
\end{definition}

\begin{theorem}
    If $Y_1$ and $Y_2$ are any two solutions of the nonhomogeneous ODE
    (\ref{linoden}) then their difference $Y_1-Y_2$ is  a solution of the
    corresponding complementary equation. Thus
    \begin{align*}
        Y_1(t) - Y_2(t) = C_1y_1(t) + C_2y_2(t)
    \end{align*}
    where $C_1$, $C_2$ are certain constants and $y_1$, $y_2$ are a
    fundamental set of solutions of the corresponding complementary equation.
\end{theorem}

\begin{theorem}
    Let $Y_1$ be an arbitrary solution $y(t)$ of the nonhomogeneous ODE
    and let $Y_2$ be a particular soltuion $y_p(t)$ of the nonhomogeneous
    ODE, then we see that the general solution of the nonhomogeneous ODE
    may be epxressed as
    \begin{align*}
        y(t) = y_c(t) + y_p(t)
    \end{align*}
    where $y_c(t) = C_1y_1(t) + C_2y_2(t)$ is the general solution of the
    corresponding complementary equation.\\
    $y_c(t)$ is called the \B{complementary function}. 
\end{theorem}

\subsubsection{Method of undetermined coefficients}

\begin{itemize}
    \item Case 1: $g(t)=Me^{kt}$ \begin{itemize}
        \item If $k$ is not a root of the characteristic equation, try $y_p(t)=Ce^{kt}$,
        \item if $k$ is a root of the characteristic equation, try $y_p(t)=Cte^{kt}$,
        \item if $k$ is a repeated root of the characteristic equation, try $y_p(t)=Ct^2e^{kt}$.
    \end{itemize}
    \item Case 2: $g(t)=M\cos(kt) + N\sin(kt)$ \begin{itemize}
        \item If $\pm ik$ are not roots of the characteristic equation, try $y_p(t)=C\cos(kt)+D\sin(kt)$,
        \item if $\pm ik$ are roots of the characteristic equation, try $y_p(t)=t(C\cos(kt)+D\sin(kt))$.
    \end{itemize}
    \item Case 3: $g(t)=a_nt^n+a_{n-1}t^{n-1}+\cdots + a_1t+a_0$ \begin{itemize}
        \item If $0$ is not a root of the characteristic equation, try $y_p(t)=b_nt^n+b_{n-1}t^{n-1}+\cdots+b_1t+b_0$,
        \item if $0$ is a root of the characteristic equation, try $y_p(t) = t(b_n t^n + b_{n-1}t^{n-1} + \cdots + b_1 t + b_0)$,
        \item if $0$ is a repeated root of the characteristic equation, try $y_p(t)=t^2(b_nt^n+b_{n-1}t^{n-1}+\cdots+b_1 t+b_0)$.
    \end{itemize}
    \item Case 4: $g(t)=\Pi_{i=0}^n g_i(t)$ where all $g_i$ are of the form in cases 1-3 \begin{itemize}
        \item Try the product of the trial functions in the corresponding cases.
    \end{itemize}
\end{itemize}

\subsubsection{Method of variation of parameters}

\begin{theorem}
    If $g(t)$ for the nonhomogeneous ODE is a general function and we know
    the general solution
    \begin{align*}
        y_c(t) = C_1y_1(t) + C_2y_2(t)
    \end{align*}
    to the corresponding homogeneous ODE, the general solution to the
    nonhomogeneous ODE is
    \begin{align*}
        y(t) = y_c(t)+y_p(t)
    \end{align*}
    where
    \begin{align*}
        y_p(t) = -y_1(t)\int\frac{y_2(t)g(t)}{aW(y_1,y_2)}\:dt 
        + y_2(t) \int \frac{y_1(t)g(t)}{aW(y_1,y_2)}\:dt.
    \end{align*}
\end{theorem}
\end{document}
