\documentclass{article}
\usepackage{homework-preamble}

\begin{document}
\title{SVCDE: Hand-in 6}
\author{Franz Miltz (UUN: S1971811)}
\date{1 November 2020}
\maketitle


\section*{Question 1}


To evaluate
\begin{align*}
  I=\int_C \left(y\ih - x\jh\right)\cdot d\vec r
\end{align*}
with $C$ as described in the question, we need to split $I$ into its
parts:
\begin{align*}
  I = \int_{C_\alpha} \vec G \cdot d\vec r + \int_{C_\beta} \vec G \cdot d\vec r = I_1 + I_2
\end{align*}
where $C_\alpha$ is the the part of the circle and $C_\beta$ is the line segment.
For $I_1$ we find
\begin{align*}
  I_1 = \int_{C_\alpha} ydx-xdy.
\end{align*}
We parameterise with $x=\cos\theta$ and $y=\sin\theta$.
This let's us write
\begin{align*}
  I_1 = \int_0^{3\pi/2} \left(-\sin^2\theta - \cos^2\theta\right) d\theta
  =-\int_0^{3\pi/2}d\theta=-\frac{3\pi}{2}.
\end{align*}
Similarly, we find
\begin{align*}
  I_2 = \int_{C_\beta} ydx-xdy.
\end{align*}
We use the parameterisation $x=t$ and $y=t-1$ to get
\begin{align*}
  I_2 = \int_0^1 (t-1-t)dt=-\int_0^1 dt = -1.
\end{align*}
Inserting this let's us find $I$:
\begin{align*}
  I = -\frac{3\pi}{2}-1 = -\frac{3\pi + 2}{2}.
\end{align*}


\section*{Question 2}


We have the transformation $(u,v,w)\to(x,y,z)$ defined by the equations
\begin{align}
  \label{eq1}
  u   & = x + y + z, \\
  \label{eq2}
  uv  & = y + z,     \\
  \label{eq3}
  uvw & = z.
\end{align}
To find the Jacobian, we need to find $x$, $y$ and $z$ in terms of $u$, $v$ and $w$.
The equation for $z(u,v,w)$ follows from (\ref{eq3}) directly.
We can insert this into (\ref{eq2}) to find
$y(u,v,w)$ and with (\ref{eq1}) this leads to $x(u,v,w)$. We get
\begin{align*}
  x(u,v,w) = u(1-v),\hs
  y(u,v,w) = uv(1-w),\hs
  z(u,v,w) = uvw.
\end{align*}
This lets us write
\begin{align*}
  \frac{\partial(x,y,z)}{\partial(u,v,w)}
  = \begin{vmatrix}
    1-v    & -u     & 0   \\
    v(1-w) & u(1-w) & -uv \\
    vw     & uw     & uv
  \end{vmatrix}
\end{align*}
We expand via the first row to get
\begin{align*}
  \frac{\partial(x,y,z)}{\partial(u,v,w)}
  = (1-v)\begin{vmatrix}
    u(1-w) & -uv \\
    uw     & uv
  \end{vmatrix}
  + (-u)\begin{vmatrix}
    v(1-w) & -uv \\
    vw     & uv
  \end{vmatrix}.
\end{align*}
By applying the formula for $2\times 2$ determinants and
simplifying the result, we find
\begin{align*}
  \frac{\partial(x,y,z)}{\partial(u,v,w)} = u^2 v.
\end{align*}
Now we can evaluate
\begin{align*}
  I = \ti_T e^{-(x+y+z)^3}dx\,dy\,dz.
\end{align*}
We apply the transformation to get
\begin{align*}
  I = \ti_U e^{-u^3}u^2v\:du\,dv\,dw
  = \int_0^1\int_0^1\int_0^1 e^{-u^3}u^2\:du\,dv\,dw.
\end{align*}
Using \emph{Fubini's Theorem} this may be solved with relative ease:
\begin{align*}
  I & = \int_0^1 \int_0^1 \left[e^{-u^3}u^2 v w\right]_0^1dv\,du
  =\int_0^1 \int_0^1 e^{-u^3}u^2 v\:dv\,du                       \\
    & =\int_0^1 \left[\frac{1}{2}e^{-u^3}u^2v^2\right]_0^1du
  =\frac{1}{2}\int_0^1 e^{-u^3}u^2\:du                           \\
    & =\frac{1}{2}\left[-\frac{1}{3}e^{-u^3}\right]_0^1
  =\frac{e-1}{6e}.
\end{align*}
\end{document}
