\documentclass{article}
\usepackage{homework-preamble}

\begin{document}
\title{SVCDE: Hand-in 7}
\author{Franz Miltz (UNN: S1971811)}
\date{8 November 2020}
\maketitle


\section*{Question 1}


We have
\begin{align*}
  \vec F = (1+2xy+y^2+yz^2)\ih + (x^2+e^y+2xy+xz^2)\jh + 2xyz\kh.
\end{align*}

\subsection*{(a) and (b)}

Since finding a potential function $\phi$ such that $\vec F = \grad\phi$
will show that $\vec F$ is conservative, we can solve (a) implicitly.
To find a suitable $\phi$ we need to find anti-derivatives of all the components
of $\vec F$. We find
\begin{align*}
  \int \vec F\cdot \ih\: dx & = x + x^2y + xy^2 + xyz^2 + C_x, \\
  \int \vec F\cdot \jh\: dy & = x^2y+e^y+xy^2+xyz^2+C_y,       \\
  \int \vec F\cdot \kh\: dz & = xyz^2 + C_z.
\end{align*}
Notice how every summand that depends on more than one of $x,y,z$ appears
in all the relevant anti-derivatives. Thus we can inspect to find
\begin{align}
  \label{eqphi}
  \phi(x,y,z) = x + x^2y + xy^2 + xyz^2 + e^y.
\end{align}
Partial differentiation shows that
\begin{align*}
  \grad \phi = (1+2xy+y^2+yz^2)\ih + (x^2+e^y+2xy+xz^2)\jh + 2xyz\kh = \vec F.
\end{align*}
Thus $\vec F$ is conservative and $\phi$ is a potential function.

\subsection*{(c)}

We observe that $\vec F$ is conservative and thus the integral is independent of
path. We therefore let $P$ be some smooth curve from $(0,0,0)$ to $(2,0,2)$ with the
position vector $\vec s:\R\to\R^3$ such that $\vec s(a)=\vec 0$ and $\vec s(b)=2\ih + 2\kh$
and write
\begin{align*}
  I=\int_C \vec F \cdot d\vec r = \int_P \vec F \cdot d\vec s.
\end{align*}
Furthermore, we know $\vec F=\grad\phi$ for the function $\phi$ given in (\ref{eqphi}).
Using this and applying the \emph{Fundamental Theorem of Line Integrals}, we find
\begin{align*}
  I & = \int_P \grad \phi\cdot d\vec s = \phi(\vec s(b))-\phi(\vec s(a)) \\
    & = \phi(2,0,2)-\phi(0,0,0) = 3 - 1 = 2.
\end{align*}
\emph{Note: We don't specify all of $P$ because we're only interested in the start
  and end points. If required, we could define $P$ as the straight line between
  the two points of interest.}


\section*{Question 2}


Consider the plane $P$ where
\begin{align*}
  \frac{x}{a}+\frac{y}{b}+\frac{z}{c}=1.
\end{align*}
This may be rewritten as
\begin{align*}
  z=c\left(1-\frac{x}{a}-\frac{y}{b}\right)=g(x,y).
\end{align*}
Therefore the area of the triangle described in the question is given by
\begin{align*}
  A = \di_D \sqrt{1 + \left(\frac{\partial z}{\partial x}\right)^2+\left(\frac{\partial z}{\partial y}\right)^2}\:dA.
\end{align*}
We find the projection of $P$ onto the $xy$-plane, i.e. $D$, to be the right triangle
given by the points $(0,0,0)$, $(a,0,0)$ and $(0,b,0)$.
Thus $D$ may be described by the region between the $x$-axis and the line segment from $(0,b,0)$ to
$(a,0,0)$ which is given by the equation
\begin{align*}
  y = b-\frac{b}{a}x
\end{align*}
in the interval $x\in[0,a]$.
By inserting the partial derivatives and the correct boundaries with respect to
$x$ and $y$, we find
\begin{align*}
  A = \int_0^a \int_0^{b-bx/a}u\:dy\,dx
\end{align*}
where $u$ is the constant
\begin{align}
  \label{equ}
  u = \sqrt{1+\frac{c^2}{a^2}+\frac{c^2}{b^2}}.
\end{align}
Thus
\begin{align*}
  A & = u\int_0^a [y]_0^{b-bx/a}=u\int_0^a\left(b - \frac{b}{a}x\right)dx \\
    & = bu \left[x-\frac{x^2}{2a}\right]_0^a = \frac{1}{2}abu.
\end{align*}
Re-inserting the substitution (\ref{equ}) then yields
\begin{align*}
  A & = \frac{1}{2}ab\sqrt{1+\frac{c^2}{a^2}+\frac{c^2}{b^2}}
  = \frac{1}{2}\sqrt{a^2b^2+c^2b^2+c^2a^2}
\end{align*}
as required.
\end{document}
