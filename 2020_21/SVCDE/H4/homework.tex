\documentclass{article}
\usepackage{homework-preamble}

\begin{document}
\title{SVCDE: Hand-in 4}
\author{Franz Miltz (UUN: S1971811)}
\date{18th October 2020}
\maketitle


\section*{Question 1}


To find the minimum and maximum of
\begin{align*}
	f(x,y,z)=x + y + z
\end{align*}
on the elipsoid with the condition
\begin{align*}
	g(x,y,z)=\frac{x^2}{a^2} + \frac{y^2}{b^2} + \frac{z^2}{c^2}-1=0.
\end{align*}
we can write the Lagrangian function
\begin{align*}
	L(x,y,z,\lambda) = f(x,y,z)+\lambda g(x,y,z) =  x + y + z + \lambda\left(\frac{x^2}{a^2}+\frac{y^2}{b^2}+\frac{z^2}{c^2}-1\right).
\end{align*}
Differentiation lets us find $\grad L$ as follows:
\begin{align*}
	\grad L(x,y,z,\lambda)=\begin{pmatrix}
		                       1+\frac{2\lambda x}{a^2} \\
		                       1+\frac{2\lambda y}{b^2} \\
		                       1+\frac{2\lambda z}{c^2} \\
		                       \frac{x^2}{a^2}+\frac{y^2}{b^2}+\frac{z^2}{c^2}-1
	                       \end{pmatrix}
\end{align*}
Setting $\grad L(x,y,z,\lambda)=\vec 0$ gives us the equation
\begin{align}
	\label{eq1}
	-\frac{1}{2\lambda} = \frac{x}{a^2} = \frac{y}{b^2} = \frac{z}{c^2}.
\end{align}
Additionally, we can rewrite
\begin{align*}
	\frac{x^2}{a^2}+\frac{y^2}{b^2}+\frac{z^2}{c^2}-1=0
\end{align*}
as
\begin{align*}
	\frac{x^2}{a^2}=1-\frac{y^2}{b^2}-\frac{z^2}{c^2}.
\end{align*}
We can use equation \ref{eq1} to express $y$ and $z$ in terms of $x$ and
insert the results to find
\begin{align*}
	\frac{x^2}{a^2}=1-\frac{b^2x^2}{a^4}-\frac{c^2x^2}{a^4}.
\end{align*}
This equation is linear in $x^2$ so we can solve for it easily. We obtain
\begin{align*}
	x^2=\frac{a^2}{a^2+b^2+c^2} \Rightarrow |x|=x_0=\frac{a}{\sqrt{a^2+b^2+c^2}}.
\end{align*}
Notice that we can find $y^2$ and $z^2$ in much the same way to get
\begin{align*}
	y^2 & =\frac{b^2}{a^2+b^2+c^2}\Rightarrow |y|=y_0=\frac{b}{\sqrt{a^2+b^2+c^2}}, \\
	z^2 & =\frac{c^2}{a^2+b^2+c^2}\Rightarrow |z|=z_0=\frac{c}{\sqrt{a^2+b^2+c^2}}.
\end{align*}
Furthermore, we observe that $f$ is linear in all of $x$, $y$ and $z$.
So we need to find the largest possible values for each argument to maximise
the function and do the opposite to minimise it. Therefore $f$ has a maximum
on the ellipsoid $g=0$ at
\begin{align*}
	f(x_0,y_0,z_0) = \frac{a+b+c}{\sqrt{a^2+b^2+c^2}}
\end{align*}
and a minimum at
\begin{align*}
	f(-x_0, -y_0, -z_0) = -\frac{a+b+c}{\sqrt{a^2+b^2+c^2}}
\end{align*}
where $x_0$, $y_0$ and $z_0$ are as defined above.\\\\
\emph{Note: If we were interested in all critical points we would need to find
	$\lambda$ in terms of $a$, $b$ and $c$. Since we're only interested in maximum
	and minimum values, we can skip other potential solutions.}


\section*{Question 2}


To evaluate
\begin{align*}
	\iint_R\frac{xy^2}{(4x^2+y^2)^2}dA
\end{align*}
where $R$ is the finite region enclosed by $y=x^2$ and $y=2x$,
we need to find $R$ first. To do that we find the intersections between the
two curves as follows
\begin{align*}
	y=x^2 \text{ and } y = 2x \\
	\Leftrightarrow x(x-2)=0.
\end{align*}
Therefore the intersections are at $x=0$ and $x=2$. Observe that
$\forall x\in[0,2], 2x\geq x^2$. Therefore
\begin{align*}
	\iint_R \frac{xy^2}{(4x^2+y^2)^2} dA
	= \int_0^2\int_{x^2}^{2x} \frac{xy^2}{(4x^2+y^2)^2}dy\: dx.
\end{align*}
To solve these nested integrals, we first need to evaluate the
integral
\begin{align*}
	I(x) = \int_{x^2}^{2x} \frac{xy^2}{(4x^2+y^2)^2}dy.
\end{align*}
We can use $u=xy$ and $dv=y/(4x^2+y^2)^2$ and integration by
parts to get
\begin{align*}
	I(x) = \left[-\frac{xy}{2(4x^2+y^2)}+\frac{x}{2}\int \frac{1}{4x^2+y^2}dy\right]_{x^2}^{2x}.
\end{align*}
To solve the new integral we use $u=y/2x$ to find
\begin{align*}
	I(x) & = \left[-\frac{xy}{2(4x^2+y^2)}+\frac{x}{2}\int \frac{2x}{4x^2(1+u^2)}du\right]_{x^2}^{2x}    \\
	     & = \left[-\frac{xy}{2(4x^2+y^2)}+\frac{1}{4}\arctan(u)\right]_{x^2}^{2x}                       \\
	     & = \left[-\frac{xy}{2(4x^2+y^2)}+\frac{1}{4}\arctan\left(\frac{y}{2x}\right)\right]_{x^2}^{2x} \\
	     & =-\frac{x^2}{4x^2+4x^2}+\frac{1}{4}\arctan(1)+\frac{x^3}{2(4x^2+x^4)}
	-\frac{1}{4}\arctan\left(\frac{x}{2}\right)                                                          \\
	     & =-\frac{1}{8}+\frac{\pi}{16}+\frac{x}{2(4+x^2)}-\frac{1}{4}\arctan\left(\frac{x}{2}\right).
\end{align*}
Thus
\begin{align*}
	\di_R\frac{xy^2}{(4x^2+y^2)^2}dA & =\int_0^2\left(\frac{\pi-2}{16}+\frac{x}{2(4+x^2)}-\frac{1}{4}\arctan\left(\frac{x}{2}\right)\right)dx                   \\
	                                 & =\frac{1}{16}\int_0^2(\pi-2)dx+\frac{1}{2}\int_0^2\frac{x}{4+x^2}dx-\frac{1}{4}\int_0^2\arctan\left(\frac{x}{2}\right)dx
\end{align*}
We find
\begin{align*}
	\frac{1}{16}\int_0^2 (\pi-2)dx = \frac{\pi-2}{8},
\end{align*}
and
\begin{align*}
	\frac{1}{4}\int_0^2 \arctan\left(\frac{x}{2}\right)dx
	=\frac{1}{4}\left[x\arctan\left(\frac{x}{2}\right)
		-\ln (x^2+4)\right]^2_0=
	\frac{\pi}{8}-\frac{1}{4}\ln(2).
\end{align*}
Finally, we obtain
\begin{align*}
	\frac{1}{2}\int_0^2\frac{x}{4+x^2}dx=\frac{1}{2}\left[\frac{1}{2}\ln(x^2+4)\right]_0^2=\frac{1}{4}\ln(2)
\end{align*}
by using $u=4+x^2$ as a substitution. We can combine those results
to get
\begin{align*}
	\di_R\frac{xy^2}{(4x^2+y^2)^2}dA
	 & =\frac{\pi-2}{8}+\frac{1}{4}\ln(2)-\frac{\pi}{8}+\frac{1}{4}\ln(2)
	=\frac{1}{2}\ln(2)-\frac{1}{4}   a.
\end{align*}
\end{document}
