\documentclass{article}
\usepackage{homework-preamble}

\begin{document}
\title{FPM: Hand-in 5}
\author{Franz Miltz (UNN: S1971811)}
\date{1 March 2021}
\maketitle
\mkthms

\section*{Problem 1}

\begin{claim*}
    Let $h:G\to H$ be a group homomorphism. Then
    \begin{align}
        \label{iff}
        h\text{ is injective}\hs\Leftrightarrow\hs \gke h = \{e_G\}.
    \end{align}
\end{claim*}
\begin{proof}
    First, assume $h$ is injective. Then, by definition, we have
    \begin{align}
        \label{inj}
        h(g_1)=h(g_2) \hs\Rightarrow\hs g_1 = g_2
    \end{align}
    for all $g_1,g_2\in G$. From \emph{Lemma 3.1.5} from the notes
    we know that $h(e_G) = e_H$ where $e_G$. By definition of the kernel of
    a group homomorphism this implies $e_G\in\gke h$. Now let
    $g\in\gke h$. Thus $h(g) = e_H$. Then, as shown above,
    \begin{align*}
        h(g) = e_H = h(e_G)
    \end{align*}
    and by (\ref{inj}) $g=e_G$ follows. Therefore
    \begin{align*}
        \gke h = \{e_G\}.
    \end{align*}
    We have shown
    \begin{align}
        \label{if}
        h\text{ is injective}\hs\Rightarrow\hs \gke h = \{e_G\}.
    \end{align}
    Now assume $\gke h = \{e_G\}$. Let $g_1,g_2\in G$ such that
    \begin{align*}
        h(g_1) = h(g_2).
    \end{align*}
    Then
    \begin{align*}
        h(g_1)h(\inv g_1) = h(g_2)h(\inv g_1).
    \end{align*}
    Since $h(g_1)h(\inv g_1) = h(g_1\inv g_1) = h(e_G)$, we have
    \begin{align*}
        h(e_G) = h(g_2)h(\inv g_1)= h(g_2\inv g_1).
    \end{align*}
    By \emph{Lemma 3.1.5} from the notes this shows
    \begin{align*}
        e_H = h(g_2\inv g_1).
    \end{align*}
    By the premise $\gke h = \{e_G\}$ and the definition of $\gke h$,
    we have
    \begin{align*}
        h(g) = e_H \hs\Leftrightarrow\hs g=e_G
    \end{align*}
    and therefore
    \begin{align*}
        e_G = g_2\inv g_1.
    \end{align*}
    By multiplication with $g_1$,
    \begin{align*}
        g_1 = g_2
    \end{align*}
    follows. This shows
    \begin{align}
        \label{onlyif}
        h\text{ is injective}\hs\Leftarrow\hs \gke h = \{e_G\}.
    \end{align}
    Since (\ref{if}) and (\ref{onlyif}) hold, we have shown (\ref{iff}).
\end{proof}

\section*{Problem 2}


\begin{claim*}
    Let
    \begin{align*}
        G_1 & = \Z_2 \times \Z_5 \times \Z_2 \times \Z_5, \\
        G_2 & = \Z_4 \times \Z_{25},                      \\
        G_3 & = \Z_{100}
    \end{align*}
    be groups where $\Z_n$ is a group under addition. Then
    $G_2 \cong G_3$ but $G_1\not\cong G_2,G_3$.
\end{claim*}

\begin{proof}
    Firstly, observe that $\abs{G_1}=\abs{G_2}=\abs{G_3}=100$.
    Seconldy, note that $G_3=\lra{1}$. Further, we know from
    \emph{Theorem 2.2.16} that $G_2$ must be cyclic, too. That is,
    there exists an element $g\in G_2$ such that $o(g)=100$. The
    same holds for $G_3$. Further, consider any element $g\in G_1$.
    We have
    \begin{align*}
        g = (g_1, g_2, g_3, g_4).
    \end{align*}
    Consider $g^{10}$. Observe that $o(g_1)$ and $o(g_3)$ are either
    $1$ or $2$ and $o(g_2)$ and $o(g_4)$ are either $1$ or $5$. Therefore
    \begin{align*}
        g^{10}=(g_1^{10},g_2^{10},g_3^{10},g_4^{10})=(e,e,e,e)=e.
    \end{align*}
    This shows that for all $g\in G_1$ we have $o(g)\leq 10$. Thus
    $G_1$ cannot be isomorphic to either $G_2$ or $G_3$.\\
    \indent As noted above, both $G_2$ and $G_3$ are cyclic and their
    generators have the same order. By the lemma below, $G_2\cong G_3$
    follows.
\end{proof}


\begin{lemma*}
    Let $G=\lra{g}$ and $H=\lra{h}$ be cyclic groups such that
    $o(g)=o(h)=n$. Then $G\cong H$.
\end{lemma*}

\begin{proof}
    Let $G$ and $H$ be as described. Note that we can write
    \begin{align*}
        G=\{e, g, g^2, ..., g^{n-1}\},\hs
        H=\{e, h, h^2, ..., h^{n-1}\}.
    \end{align*}
    Consider the mapping $i:G\to H, g^k\mapsto h^k$ where
    $k\in\{0,1,...,n-1\}$. We then have
    \begin{align*}
        i(g^{a+b}) = h^{a+b} = h^a h^b = i(g^a)i(g^b).
    \end{align*}
    This shows that $i$ is a group homomorphism. Further, we observe
    that for any $g^a,g^b\in G$ we have
    \begin{align*}
        i(g^a) = i(g^b) \hs\Leftrightarrow\hs h^a=h^b.
    \end{align*}
    Since $a,b\in\{0,1,...,n-1\}$ this is true if and only if $a=b$.
    Therefore we have
    \begin{align*}
        i(g^a) = i(g^b) \hs\Leftrightarrow\hs g^a = g^b,
    \end{align*}
    i.e. $i$ is bijective. This shows that $i$ is an isormorphism
    and $G$ and $H$ are isomorphic.
\end{proof}
\end{document}