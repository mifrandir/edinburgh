\documentclass{article}
\usepackage{homework-preamble}

\begin{document}
\title{FPM: Hand-in 2}
\author{Franz Miltz (UUN: S1971811)}
\date{1 February 2021}
\maketitle

\section*{Problem 1}

\begin{claim*}
	Let $A=\{\frac{n}{n+1}:n\in\N\}$. Then $\sup A = 1$.
\end{claim*}
\begin{proof}
	We know
	\begin{align*}
		0 \leq 1
	\end{align*}
	and therefore
	\begin{align*}
		n \leq n + 1
	\end{align*}
	and thus
	\begin{align}
		\label{ub}
		\frac{n}{n+1}\leq 1.
	\end{align}
	This shows that $1$ is an upper bound of $A$.\\
	Let $x<1$ be a real number. Then let $N\in\R$ be such that
	\begin{align*}
		N = \frac{x}{1-x}.
	\end{align*}
	Using the \emph{Archimedian Property, Ross 4.6} we know that there exists
	an $n\in\N$ such that $n > N$. This results in the inequality
	\begin{align*}
		n > \frac{x}{1-x}.
	\end{align*}
	We can rearrange this to obtain
	\begin{align*}
		n(1-x) > x
	\end{align*}
	and
	\begin{align*}
		0 > x + xn - n
	\end{align*}
	and therefore we have
	\begin{align}
		\label{sup}
		\frac{n}{n+1} > x.
	\end{align}
	This shows that any $x<1$ is not an upper bound of $A$.
	From (\ref{ub}) we know that $1$ is an upper bound and (\ref{sup}) shows
	that any $x<1$ is not. Therefore we conclude that
	$1$ is the least upper bound of $A$, i.e.
	\begin{align*}
		\sup A = \sup\left\{\frac{n}{n+1}:n\in\N\right\} = 1.
	\end{align*}
\end{proof}

\section*{Problem 2}

\begin{claim*}
	\begin{align*}
		\lim_{n\to\infty}\frac{7n+8}{9n+10}=\frac{7}{9}.
	\end{align*}
\end{claim*}
\begin{proof}
	Let $\e>0$ be a real number. Then, using the \emph{Archimedian Property,
		Ross 4.6} we know there exists an $N\in\N$ such that
	\begin{align*}
		N > \frac{1}{\e}.
	\end{align*}
	For any such $N$, let $n\in\N$ be such that $n>N$. We therefore have the
	inequality
	\begin{align}
		\label{e}
		n > \frac{1}{\e}\hs\Leftrightarrow\hs \e > \frac{1}{n}=\abs{\frac{1}{n}}.
	\end{align}
	Since $2/81<1$ and $90>0$, we have
	\begin{align*}
		\abs{\frac{1}{n}}>\abs{\frac{2}{81n}} > \abs{\frac{2}{81n+90}}.
	\end{align*}
	This lets us rearrange to find
	\begin{align*}
		\abs{\frac{2}{81n+90}} = \abs{\frac{63n+72-(63n-70)}{81n+90}}
		= \abs{\frac{9(7n+8)-7(9n+10)}{9(9n+10)}}=\abs{\frac{7n+8}{9n+10}-\frac{7}{9}}.
	\end{align*}
	We therefore have
	\begin{align*}
		\abs{\frac{1}{n}}>\abs{\frac{7n+8}{9n+10}-\frac{7}{9}}.
	\end{align*}
	Using (\ref{e}) this shows that
	\begin{align*}
		\e > \abs{\frac{7n+8}{9n+10}-\frac{7}{9}}
	\end{align*}
	which by \emph{Ross 7.1} shows that
	\begin{align*}
		\lim_{n\to\infty}\frac{7n+8}{9n+10}=\frac{7}{9}.
	\end{align*}
\end{proof}

\end{document}