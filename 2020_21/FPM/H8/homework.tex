\documentclass{article}
\usepackage{homework-preamble}

\begin{document}
\title{FPM: Hand-in 8}
\author{Franz Miltz (UUN: S1971811)}
\date{22 March 2021}
\maketitle
\mkthms
\section*{Problem 1}

\begin{claim*}
    \begin{align*}
        \lim_{x\to\infty} \frac{2x^2-3}{x^2+x}=2.
    \end{align*}
\end{claim*}

\begin{proof}
    Let $M\in\R$. Then choose $\alpha=\max\{1, 5/\e\}$. Now let $x>\alpha$.
    We have
    \begin{align*}
        \abs{f(x)-L} = \abs{\frac{2x^2-3}{x^2+x}-2} = \abs{\frac{2x+3}{x^2+x}}.
    \end{align*}
    Observe that $x>1$ and therefore
    \begin{align*}
        \abs{f(x)-L} = \frac{2x+3}{x^2+x}< \frac{5x}{x^2+x} = \frac{5}{x+1} < \frac{5}{x}.
    \end{align*}
    Since $x>5/\e$, we find
    \begin{align*}
        \abs{f(x)-L} < \frac{5}{x} < \frac{5}{5/\e} = \e.
    \end{align*}
    By \emph{Ross, Discussion 20.9} the claim follows.
\end{proof}

\section*{Problem 2}

\begin{claim*}
    Let $f:[0,\infty)\to[0,\infty)$ be continuous and whenever
    $(x_n)$ is a sequence of positive real numbers diverging to
    infinity we have $f(x_n)\to 0$. Then $f$ is bounded on $\dom f$.
\end{claim*}

\begin{proof}
    By \emph{Ross, Definitions 20.1 and 20.3} it follows directly that
    \begin{align*}
        \lim_{x\to \infty} f(x) = 0.
    \end{align*}
    Using \emph{Ross, Discussion 20.9} we know that there
    exists $\alpha\in \dom f$ such that for all $x>\alpha$, $f(x)<1$.
    Thus the function $f$ is bounded above on the interval
    $(\alpha, \infty)$. Consider the interval $I=[0,\alpha]$. Since $I$
    is closed and bounded, \emph{Theorem 4.2.2} from the notes implies
    that $f$ is bounded (above) on $I$. Since $f$ is bounded below by
    $0$ on $\dom f$ and above on $[0,\alpha]\cup(\alpha,\infty)=\dom f$,
    $f$ is bounded on its entire domain.
\end{proof}

\end{document}