\documentclass{article}
\usepackage{homework-preamble}

\begin{document}
\title{FPM: Hand-in 1}
\author{Franz Miltz (UNN: S1971811)}
\date{25 January 2021}
\maketitle

\section*{Problem 1}
\begin{claim}
   Consider the graph $G$ described in the question. Then
   let $F$ be the set of symmetries of $G$. Then $|F|=20$.
\end{claim}
\begin{proof}
   Note that there exist two disconnected components
   $G_1$ and $G_2$ where $G_1$ contains $10$ vertices
   and $G_2$ contains $2$ vertices. \\
   Consider $G_1$ and its symmetries $F_1$. There are $5$ vertices with valency
   $3$, call them $c_1, ..., d_5$. Further, there are
   five nodes with valency $2$, call them $d_1, ..., c_5$
   such that
   \begin{align*}
      \forall i\in[1,5],\: c_i \text{ is connected to } d_i.
   \end{align*}
   Let $f$ be a symmetry of $G_1$. Then notice that
   \begin{align*}
      \forall i\in[1,5],\: f(c_i)=c_j \Rightarrow f(d_i)=d_j
   \end{align*}
   in order to preserve the edge between $c_i$ and $d_i$.
   Therefore, the mapping of the inner vertices fully
   determines the symmetry. It follows that there are as
   many symmetries of $G_1$ as there are elements in the
   dihedgral group $D_5$, i.e.
   \begin{align}
      \label{f1}
      |F_1|=|D_5|.
   \end{align}
   Consider the subgraph $G_2$ and its symmetries $F_2$.
   Notice that $G_2$ is a graph with $2$ vertices and
   no edges. By \emph{Lemma 1.3.3} in the notes, and the
   proof thereof, it follows that
   \begin{align}
      \label{f2}
      |F_2|=|S_2|
   \end{align}
   where $S_2$ is a symmetric group.\\
   Now consider the entire graph $G$. Since every symmetry may
   only map a vertex to another vertex of the same subgraph
   (due $G_1$ and $G_2$ not being isomorphic),
   each symmetry $f\in F$ of $G$ is fully determined by
   a pair of symmetries $(f_1, f_2)\in F_1\times F_2$ of
   the subgraphs $G_1$ and $G_2$. Thus
   \begin{align*}
      |F| = |F_1\times F_2| = |F_1||F_2|.
   \end{align*}
   Using (\ref{f1}) and (\ref{f2}) we get
   \begin{align*}
      |F| = |D_5||S_2|.
   \end{align*}
   Here we can use the formulas given in \emph{Sections 1.3.3
      and 1.3.4} of the notes to get
   \begin{align*}
      |F| = 2\cdot 5\cdot 2! = 20.
   \end{align*}
\end{proof}

\section*{Problem 2}

\begin{claim}
   Consider the graph $G$ described in the question. Then
   let $F$ be the set of symmetries of $G$. Then
   $|F|=28800$.
\end{claim}

\begin{proof}
   Note that $G$ consists of two disconnected components.
   Both of those subgraphs contain $5$ vertices and are
   fully connected. Let the vertices of one component be
   $V=\{v_1, ..., v_5\}$ and the vertices of the other be
   $U=\{u_1, ..., u_5\}$. Consider a bijection $f: V\to U$, e.g.
   \begin{align*}
      \forall i\in[1,5],\: f(v_i) = u_i.
   \end{align*}
   Since all the nodes $V$ and $U$ are pairwise connected by an edge, for
   all $i,j\in[1,5]$,
   \begin{align*}
      v_i \text{ is connected to } v_j \Leftrightarrow f(v_i) \text{ is connected to } f(v_j)
   \end{align*}
   holds. Thus $f$ is an isomorphism between the two components and
   hence the components are isomorphic.\\
   Consider either one of the subgraphs $G'$ with vertices $V'$ and
   symmetries $F'$. Since $G'$
   is fully connected, every bijection $f:V'\to V'$ is a symmetry (the argument
   is analogous to the one above). Thus
   \begin{align}
      \label{f'}
      |F'| = |S_5| = 5!.
   \end{align}
   Consider the entire graph $G$ with vertices $V\cup U$, where
   $V$ and $U$ are as above, and symmetries $F$. Then, since the
   two subgraphs are disconnected, we know
   \begin{align*}
      f(v_1)\in V \hs \Leftrightarrow \hs \forall v\in V,\: f(v)\in V \text{ and } \forall u\in U,\: f(u_i)\in U
   \end{align*}
   and, conversely,
   \begin{align*}
      f(v_1)\in U \hs \Leftrightarrow \hs \forall v\in V,\: f(v)\in U \text{ and } \forall u\in U,\: f(u)\in V
   \end{align*}
   where $v_1\in V$ and $f\in F$.
   In other words, the mapping of $v_1$ to either $V$ or $U$ fully
   determines which subgraph each vertex gets mapped to. This only leaves
   the mapping within each subgraph undecided. Therefore
   \begin{align*}
      |F| = 2|F'\times F'|.
   \end{align*}
   Applying (\ref{f'}) yields
   \begin{align*}
      |F| = 2|F'|^2 = 2(5!)^2 = 28800.
   \end{align*}
\end{proof}
\end{document}