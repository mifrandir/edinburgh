\documentclass{article}
\usepackage{notes-preamble}
\usepackage{enumitem}
\begin{document}
\mkfpmthms
\title{Fundamentals of Pure Mathematics - Algebra (SEM4)}
\author{Franz Miltz}
\maketitle
\tableofcontents
\pagebreak

\setcounter{section}{-1}


\section{Revision}


\subsection{Functions}

\begin{definition}
    A function $f:X\to Y$ is called
    \begin{itemize}
        \item \emph{injective} if $f(x_1)=f(x_2)$ implies that $x_1=x_2$;
        \item \emph{surjective} if for every $y\in Y$, there exists $x\in X$ such that $f(x)=y$;
        \item \emph{bijective} if it is both injective and surjective.
    \end{itemize}
\end{definition}

\begin{lemma}
    Let $f,g,h$ be functions such that the composition $h \circ (g \circ f)$ exists. Then
    \begin{align*}
        h \circ (g \circ f) = (h \circ g) \circ f = h \circ g \circ f.
    \end{align*}
\end{lemma}

\subsection{Making counting arguments rigorous}

\begin{theorem}
    Let $S$ and $T$ be two finite sets. Let $f:S\to T$ be a map.
    Then the cardinality of $S$ is equal to the sum, over the elements
    $t$ of $T$, of the cardinalities of the inverse images $f^{-1}(\{t\})\subseteq S$:
    \begin{align*}
        \abs{S} = \sum_{t\in T}\abs{f^{-1}(\{t\})}
    \end{align*}
\end{theorem}

\begin{corollary}
    Let $S$ and $T$ be two finite sets. Then $\abs{S\times T}=\abs{S}\abs{T}$.
\end{corollary}

\begin{corollary}
    Let $S$ and $T$ be two finite sets, and let $f:S\to T$ be a map. If $f$ is
    an injection, then $\abs{S}\leq\abs{T}$. If $f$ is a surjection, then
    $\abs{S}\geq\abs{T}$. If $f$ is a bijection, then $\abs{S}=\abs{T}$.
\end{corollary}

\begin{definition}
    For any two sets $S$ and $T$, let us write $\sum(S,T)$ for the set of
    bijections $S\to T$.
\end{definition}

\begin{theorem}
    Let $S$ and $T$ be two finite sets, and assume that $\abs{S}=\abs{T}=n$.
    Then there are $n!$ bijections $S\to T$. 
\end{theorem}

\subsection{Permutations}

\begin{definition}
    A \emph{permutation} of a finite set $X$ is a bijection $X\to X$. We
    write $S_n$ for the set of permutations of $\{1,2,...,n\}$.
\end{definition}

\begin{definition}
    More generally, if $\sigma : X\to X$ is a permutation, then the \emph{order}
    of $\sigma$ is the smallest natural number $k>0$ such that $\sigma^k = e$.
\end{definition}

\begin{proposition}
    Every element of $S_n$ has a finite order.
\end{proposition}


\section{Symmetries and groups}


\subsection{Symmetries of graphs}

\begin{definition}
    A \emph{graph} is a finite set of vertices joined by edges. We will
    assume that there is at most one edge joining two given vertices and
    no edge joins a vertex to itself. The \emph{valency} of a vertex
    is the number of edges emerging from it.
\end{definition}

\setcounter{theorem}{2}
\begin{definition}
    An \emph{isomorphism} between two graphs is a bijection between them
    that preserves all edges. More precisely, if $\Gamma_1$ and $\Gamma_2$
    are graphs, with sets of vertices $V_1$ and $V_2$, respectively, then
    an \emph{isormorphism} from $\Gamma_1$ to $\Gamma_2$ is a bijection
    \begin{align*}
        f:V_1\to V_2
    \end{align*}
    such that $f(v_1)$ and $f(v_2)$ are joined by an edge if and only if
    $v_1$ and $v_2$ are joined by an edge.\\
    We say that $\Gamma_1$ and $\Gamma_2$ are \emph{isomorphic} if there
    exists an isomorphism $f:\Gamma_1\to \Gamma_2$.
\end{definition}

\setcounter{theorem}{8}
\begin{definition}
    A \emph{symmetry} of a graph is an isormorphism from the graph to
    itself, i.e. if the set of vertices is $V$, then a symmetry is a
    bijection $f:V\to V$ that preserves edges. That is, a symmetry is
    a bijection $f:V\to V$ such that $f(v_1)$ and $f(v_2)$ are joined
    by an edge if and only if $v_1$ and $v_2$ are joined by an edge.
\end{definition}

\subsection{Groups and examples}

\begin{definition}
    Let $S$ be any nonempty set. An operation $*$ on $S$ is a rule
    which, for every ordered pair $(a,b)$ of elements of $S$, determines
    a unique element $a* b$ of $S$. Equivalently, if we recall that
    \begin{align*}
        S\times S := \{(a,b) \sv a,b\in S\}
    \end{align*}
    then an operation is a function $S\times S\to S$.
\end{definition}

\setcounter{theorem}{2}
\begin{definition}
    We say that a nonempty set $G$ is a \emph{group under $*$} if
    \begin{enumerate}[label=G\arabic*.]
        \item (Closure) $*$ is an operation, so $g* h\in G$ for all $g,h\in G$.
        \item (Associativity) $g* (h * k)=(g* h)* k$ for all $g,h,k\in G$.
        \item (Identity) There exists an \emph{identity element} $e\in G$ such $e* g = g* e = g$
              for all $g\in G$.
        \item (Inverses) Every element $g\in G$ has an \emph{inverse} $\inv g$ 
              such that $g* \inv g= \inv g * g = e$.
    \end{enumerate}
    If $G$ is a finite group, the number of elements of $G$ is written $\abs{G}$, and is called the
    \emph{order} of $G$. If $G$ is infinite then we say that $\abs{G}=\infty$ , ther order of $G$
    is \emph{infinite}.
\end{definition}

\begin{theorem}
    The set of symmetries of a graph forms a group (under composition).
\end{theorem}

\setcounter{theorem}{5}
\begin{definition}
    Suppose that $G$ is a group and $g,h\in G$. If $g* h = h* g$ then we say
    that $g$ and $h$ \emph{commute}. If $g* h=h* g$ for all $g,h\in G$, then
    we say $G$ is an \emph{abelian} group.
\end{definition}

\subsection{Symmetries give groups}

\setcounter{theorem}{1}
\begin{definition}
    The set $S_n$ of all symmetries of $\{1,...,n\}$ is called the \emph{symmetric group}. 
\end{definition}

\begin{lemma}
    The set $S_n$ is a group under composition of order $\abs{S_n}=n!$.
\end{lemma}

\setcounter{theorem}{4}
\begin{definition}
    The set of \emph{invertible $n\times n$} matrices with entries in $\R$ is denoted
    $\GL(n,\R)$. Similarly, if $p$ is a prime, then the set of invertible $n\times n$
    matrices with entries in $Z_p$ is denoted $\GL(n,\Z_p)$.
\end{definition}

\begin{theorem}
    $\GL(n,\R)$ is a group under matrix multiplication.
\end{theorem}

\subsection{Product groups and first properties of groups}

\begin{theorem}
    Let $G,H$ be groups. The product
    \begin{align*}
        G\times H=\{(g,h) \sv g\in G, h\in H\}
    \end{align*}
    has the natrual structure of a group as follows:
    \begin{itemize}
        \item The group operation is $(g,h)*(g',h'):=(g* g', h* h')$, 
        where we write $*$ and $*$ for the group operations in $G$ and $H$, respectively.
        \item The identity $e$ in $G\times H$ is $e:=(e_G, e_H)$, where $e_G$ and 
        $e_H$ are the identity elements of $G$ and $H$, respectively.
        \item The inverse of $(g,h)$ is $(\inv g, \inv h)$, where the inverse of $g$ is taken in $G$,
        and the inverse of $h$ is taken in $H$.
    \end{itemize}
\end{theorem}

\begin{corollary*}
    Let $G,H$ be groups. Then
    \begin{align*}
        \abs{G\times H} = \abs{G}\abs{H}.
    \end{align*}
    \begin{proof}
        $$\forall g\in G,\:\abs{\{(g,h) : h\in H\}}=\abs{H} \:\Rightarrow\: \abs{G\times H} = \sum_{g\in G}\abs{H} = \abs{G}\abs{H}.$$
    \end{proof}
\end{corollary*}

\setcounter{theorem}{5}
\begin{lemma}
    Let $G$ be a group. If $g,h\in G$, then
    \begin{enumerate}
        \item There is one and only one element $k\in G$ such that $k* g=h$.
        \item There is one and only one elmenet $k\in G$ such that $g* k=h$.
    \end{enumerate}
\end{lemma}

\setcounter{theorem}{7}
\begin{corollary}
    The following statements are true:
    \begin{enumerate}
        \item In a group you can always cancel: if $g* s=g* t$ 
        then $s=t$. Similarly, if $s* g=* g$ then $s=t$.
        \item Inversese are unique.
        \item A group has only one identity.
    \end{enumerate}
\end{corollary}


\section{Subgroups and Lagrange's theorem}


\subsection{Subgroups}

\begin{definition}
    Let $G$ be a group under an operation $*$. Then we say that a nonempty
    subset $H$ of $G$ is a \emph{subgroup} of $G$ if $H$ itself is a group under $*$.
    We write $H\leq G$ if $H$ is a subgroup of $G$. If also $H\not=G$, we write
    $H<G$ and say that $H$ is a \emph{proper subgroup}.
\end{definition}

\begin{lemma}
    Suppose that $H\leq G$. Then
    \begin{enumerate}
        \item $e_H = e_G$, and
        \item if $h\in H$, the inverse of $h$ in $H$ equals the inverse of $h$ in $G$.
    \end{enumerate}
\end{lemma}

\begin{theorem}
    $H\subseteq G$ is a subgroup of $G$ if and only if
    \begin{enumerate}[label=S\arabic*.]
        \item $H$ is not empty.
        \item If $h,k\in H$ then $h*k\in H$.
        \item If $h\in H$ then $h^{-1}\in H$.
    \end{enumerate}
\end{theorem}

\begin{theorem}
    $H\subseteq G$ is a subgroup of $G$ if and only if
    \begin{enumerate}[label=$\widetilde{\text{S\arabic*}}$.]
        \item $H$ is not empty.
        \item If $h,k\in H$ then $h*k^{-1}\in H$.
    \end{enumerate}
\end{theorem}

\subsection{Orders of elements and cyclic subgroups}

\setcounter{theorem}{2}
\begin{definition}[Order of a group]
    A finite group $G$ is one with only a finite number of elements.
    The \emph{order} of a finite group, written $\abs{G}$, is the
    number of elements in $G$.
\end{definition}

\begin{definition}[Order of an element]
    Let $G$ be a group and $g\in G$. Then the \emph{order $o(g)$ of $g$}
    is the least $n\in\N$ such that
    \begin{align*}
        g^n = e.
    \end{align*}
    If no such $n$ exists, we say that $g$ has \emph{infinite order}.
\end{definition}

\setcounter{theorem}{5}
\begin{theorem}
    In a finite group, every element has finite order.
\end{theorem}

\begin{corollary}
    Let $g$ be an element of a finite group $G$. Then there exists $k\in\N$
    such that $g^k=g^{-1}$.
\end{corollary}

\begin{definition}
    Let $G$ be a group and let $g\in G$ be an element. We define the
    subset
    \begin{align*}
        \lra g := \{g^k \sv k\in\Z\}.
    \end{align*}
\end{definition}

\begin{lemma}
    If $G$ is a group and $g\in G$, then $\lra g$ is a subgroup of $G$.
\end{lemma}

\begin{definition}
    A subgroup $H\leq G$ is \emph{cyclic} if $H=\lra h$ for some
    $h\in H$. In this case, wew say that $H$ is the \emph{cyclic subgroup
    generated by $h$}. If $G=\lra g$ for some $g\in G$, then we
    say that the group $G$ is \emph{cyclic}, and that $g$ is a generator.
\end{definition}

\setcounter{theorem}{13}
\begin{lemma}
    Let $G$ be a finite group. Then
    \begin{align*}
        \text{$G$ is cyclic} \hs\Leftrightarrow\hs \text{$G$ contains an element of order $\abs{G}$}.
    \end{align*}
\end{lemma}

\begin{theorem}
    Let $G$ be a cyclic group and let $H$ be a subgroup of $G$. Then
    $H$ is cyclic.
\end{theorem}

\begin{theorem}
    Let $m,n\in\N$, let $G=\lra r$ be a cyclic group of order $m$
    and $H=\lra h$ be a cyclic group of order $n$. Then
    \begin{align*}
        \text{$G\times H$ is cyclic} \hs\Leftrightarrow\hs \text{$m$ and $n$ are coprime.}
    \end{align*}
\end{theorem}

\subsection{Cosets}

\begin{theorem}[Lagrange's Theorem]
    Let $G$ be a finite group and let $H\leq G$.
    Then $\abs{H}$ divides $\abs{G}$. 
\end{theorem}

\begin{definition}
    Let $X$ be a set, and $R\subseteq X\times X$.
    If $(s,t)\in R$ we write $s\sim t$. We call $\sim$
    a relation on $X$. A relation $\sim$ is called an
    \emph{equivalence relation} on $X$ if it satisfies
    the following three axioms
    \begin{enumerate}[label=E\arabic*.]
        \item (Reflexive) $x\sim x$ for all $x\in X$.
        \item (Symmetric) $x\sim y \Rightarrow y\sim x$ for all $x,y\in X$.
        \item (Transitive) $x\sim y$ and $y\sim z \Rightarrow x\sim z$ for all $x,y,z\in X$.
    \end{enumerate}
    The \emph{equivalence class} containing $x$ is give by
    \begin{align*}
        \cl(x) := \left\{s\in X \sv x\sim s\right\}.
    \end{align*}
\end{definition}

\setcounter{theorem}{3}
\begin{definition}
    Let $H\leq G$ and let $g\in G$. Then a \emph{left coset} of $H$
    in $G$ is a subset of $G$ of the form $gH$, for some $g\in G$.
\end{definition}

\setcounter{theorem}{5}
\begin{definition}
    We denote the set of left cosets of $H$ in $G$ by $G/H$.
\end{definition}

\setcounter{theorem}{7}
\begin{theorem}
    Let $H\leq G$.
    \begin{enumerate}
        \item For all $h\in H$, $hH=H$. In particular $eH=H$.
        \item For $g_1,g_2\in G$, the following are equivalent \begin{enumerate}
            \item $g_1H=g_2H$.
            \item $\exists h\in H,\: g_2 = g_1h$.
            \item $g_2\in g_1 H$.
        \end{enumerate}
        \item For $g_1,g_2\in G$, define $g_1\sim g_2$ if and only if $g_1H=g_2H$.
        Then $\sim$ defines an equivalence relation on $G$.
    \end{enumerate}
\end{theorem}

\begin{definition}
    The \emph{right cosets} of $H$ in $G$ are subsets of the form $Hg$.
\end{definition}

\subsection{Lagrange's Theorem}

\begin{lemma}
    Suppose that $H\leq G$ and $H$ is finite.
    \begin{enumerate}
        \item Then $\abs{gH}=\abs{H}$ for all $g\in G$.
        \item For a fixed $g\in G$, the number of $g_1\in G$ 
        such that $gH=g_1H$ is equal to $\abs{H}$.
    \end{enumerate}
\end{lemma}

\begin{theorem}
    Suppose that $G$ is a finite group.
    \begin{enumerate}
        \item If $H\leq G$, then $H$ divides $|G|$.
        \item Let $g\in G$. Then $o(g)$ divides $|G|$.
        \item For all $g\in G$, we have that $g^{\abs{G}}=e$.
    \end{enumerate}
\end{theorem}

\begin{corollary}
    $\abs{G / H}= \abs{G}/\abs{H}$.
\end{corollary}

\begin{definition}
    The \emph{index} of $H\leq G$ is the number of distinct
    left cosets of $H$ in $G$, which is $\abs{G/H}$.
\end{definition}

\begin{theorem}
    Suppose that $G$ is a group with $\abs{G}=p$, where $p$ is
    prime. Then $G$ is a cyclic group.
\end{theorem}

\begin{corollary}
    Suppose that $G$ is a group with $\abs{G}<6$. Then $G$ is abelian.
\end{corollary}

\begin{theorem}[Fermat's Little Theorem]
    If $p$ is a prime and $a\in\Z$, then
    \begin{align*}
        a^{p}\equiv a \mod p.
    \end{align*}
\end{theorem}


\section{Going between groups}


\subsection{Homomorphisms and isomorphisms}

\begin{definition}
    Let $(G, \oplus)$ and $(H, \otimes)$ be groups. A map $\phi:G\to H$ 
    is called a \emph{group homomorphism} if
    \begin{align*}
        \phi(x \oplus y) = \phi(x)\otimes \phi(y)\hs\text{for all }x,y\in G.
    \end{align*}
\end{definition}

\begin{definition}
    A group homomorphism $\phi:G\to H$ that is also a bijection is
    called an \emph{isomorphism} of groups. In this case we say that
    $G$ and $H$ are isomorphic and we write $G\cong H$. An
    isomorphism $G\to G$ is called an \emph{automorphism} of $G$.
\end{definition}

\setcounter{theorem}{4}
\begin{lemma}
    Let $\phi:G\to H$ be a group homomorphism. Then
    \begin{enumerate}
        \item $\phi(e)=e$ and further $\phi(\inv g)=\inv{(\phi(g))}$ for all $g\in G$.
        \item If $\phi$ is injective, the order of $g\in G$ equals the order of $\phi(g)\in H$.
    \end{enumerate}
\end{lemma}

\begin{definition}
    Let $\phi:G\to H$ be a group homomorphism.
    \begin{enumerate}
        \item The \emph{image} of $\phi$ is defined to be \begin{align*}
            \im\phi := \{h\in H \sv h=\phi(g)\text{ for some }g\in G\}.
        \end{align*}
        \item We define the \emph{kernel} of $\phi$ to be \begin{align*}
            \ker \phi := \{g\in G \sv \phi(g) = e_H\}.
        \end{align*}
    \end{enumerate}
\end{definition}

\begin{definition}
    A subgroup $N\leq G$ is \emph{normal} if left and right cosets
    of $N$ are equal: $gN=Ng$ for all $g\in G$. If $N$ is a normal
    subgroup of $G$ we write $N \triangleleft G$.
\end{definition}


\begin{proposition}
    Let $\phi:G\to H$ be a group homomorphism. Then $\ker\phi\triangleleft G$.
\end{proposition}

\begin{proposition}
    Let $\phi: G\to H$ be a group homomorphism. Then
    \begin{enumerate}
        \item $\phi:G\to H$ is injective if and only if $\ker\phi=\{e_G\}$.
        \item If $\phi:G\to H$ is injective, then $\phi$ gives an isomorphism $G\cong \im\phi$.
    \end{enumerate}
\end{proposition}

\subsection{Products and isomorphisms}

\begin{theorem}
    Let $H,K\leq G$ be subgroups with $H\cap K=\{e\}$.
    \begin{enumerate}
        \item The map $\phi:H\times K\to HK$ given by $\phi:(h,k)\mapsto hk$ is bijective.
        \item If further every element of $H$ commutes with every element of $K$ when
              when multiplied in $G$, then $HK$ is a subgroup of $G$, and
              furthermore it is isomorphic to $H\times K$, via $\phi$.
    \end{enumerate}
\end{theorem}

\begin{corollary}
    Let $H,K\leq G$ be infinite subgroups of a group $G$ such that
    $H\cap K=\{e\}$. Then $\abs{HK} = \abs H \times \abs K$.
\end{corollary}
\end{document}
