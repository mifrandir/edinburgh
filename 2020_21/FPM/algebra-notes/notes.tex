\documentclass{article}
\usepackage{notes-preamble}
\begin{document}
\mkfpmthms
\title{Fundamentals of Pure Mathematics - Algebra (SEM4)}
\author{Franz Miltz}
\maketitle
\tableofcontents
\pagebreak

\setcounter{section}{-1}


\section{Revision}


\subsection{Functions}

\begin{definition}
    A function $f:X\to Y$ is called
    \begin{itemize}
        \item \emph{injective} if $f(x_1)=f(x_2)$ implies that $x_1=x_2$;
        \item \emph{surjective} if for every $y\in Y$, there exists $x\in X$ such that $f(x)=y$;
        \item \emph{bijective} if it is both injective and surjective.
    \end{itemize}
\end{definition}

\begin{lemma}
    Let $f,g,h$ be functions such that the composition $h \circ (g \circ f)$ exists. Then
    \begin{align*}
        h \circ (g \circ f) = (h \circ g) \circ f = h \circ g \circ f.
    \end{align*}
\end{lemma}

\subsection{Making counting arguments rigorous}

\begin{theorem}
    Let $S$ and $T$ be two finite sets. Let $f:S\to T$ be a map.
    Then the cardinality of $S$ is equal to the sum, over the elements
    $t$ of $T$, of the cardinalities of the inverse images $f^{-1}(\{t\})\subseteq S$:
    \begin{align*}
        |S| = \sum_{t\in T}\abs{f^{-1}(\{t\})}
    \end{align*}
\end{theorem}

\begin{corollary}
    Let $S$ and $T$ be two finite sets. Then $\abs{S\times T}=\abs{S}\abs{T}$.
\end{corollary}

\begin{corollary}
    Let $S$ and $T$ be two finite sets, and let $f:S\to T$ be a map. If $f$ is
    an injection, then $\abs{S}\leq\abs{T}$. If $f$ is a surjection, then
    $\abs{S}\geq\abs{T}$. If $f$ is a bijection, then $\abs{S}=\abs{T}$.
\end{corollary}

\begin{definition}
    For any two sets $S$ and $T$, let us write $\sum(S,T)$ for the set of
    bijections $S\to T$.
\end{definition}

\begin{theorem}
    Let $S$ and $T$ be two finite sets, and assume that $\abs{S}=\abs{T}=n$.
    Then there are $n!$ bijections $S\to T$. 
\end{theorem}

\subsection{Permutations}

\begin{definition}
    A \emph{permutation} of a finite set $X$ is a bijection $X\to X$. We
    write $S_n$ for the set of permutations of $\{1,2,...,n\}$.
\end{definition}

\begin{definition}
    More generally, if $\sigma : X\to X$ is a permutation, then the \emph{order}
    of $\sigma$ is the smallest natural number $k>0$ such that $\sigma^k = e$.
\end{definition}

\begin{proposition}
    Every element of $S_n$ has a finite order.
\end{proposition}


\section{Symmetries and groups}


\subsection{Symmetries of graphs}

\begin{definition}
    A \emph{graph} is a finite set of vertices joined by edges. We will
    assume that there is at most one edge joining two given vertices and
    no edge joins a vertex to itself. The \emph{valency} of a vertex
    is the number of edges emerging from it.
\end{definition}

\setcounter{theorem}{2}
\begin{definition}
    An \emph{isomorphism} between two graphs is a bijection between them
    that preserves all edges. More precisely, if $\Gamma_1$ and $\Gamma_2$
    are graphs, with sets of vertices $V_1$ and $V_2$, respectively, then
    an \emph{isormorphism} from $\Gamma_1$ to $\Gamma_2$ is a bijection
    \begin{align*}
        f:V_1\to V_2
    \end{align*}
    such that $f(v_1)$ and $f(v_2)$ are joined by an edge if and only if
    $v_1$ and $v_2$ are joined by an edge.\\
    We say that $\Gamma_1$ and $\Gamma_2$ are \emph{isomorphic} if there
    exists an isomorphism $f:\Gamma_1\to \Gamma_2$.
\end{definition}

\setcounter{theorem}{8}
\begin{definition}
    A \emph{symmetry} of a graph is an isormorphism from the graph to
    itself, i.e. if the set of vertices is $V$, then a symmetry is a
    bijection $f:V\to V$ that preserves edges. That is, a symmetry is
    a bijection $f:V\to V$ such that $f(v_1)$ and $f(v_2)$ are joined
    by an edge if and only if $v_1$ and $v_2$ are joined by an edge.
\end{definition}

\subsection{Groups and examples}

\subsubsection{The definition of a group}

\begin{definition}
    Let $S$ be any nonempty set. An operation $\star$ on $S$ is a rule
    which, for every ordered pair $(a,b)$ of elements of $S$, determines
    a unique element $a\star b$ of $S$. Equivalently, if we recall that
    \begin{align*}
        S\times S := \{(a,b) | a,b\in S\}
    \end{align*}
    then an operation is a function $S\times S\to S$.
\end{definition}

\setcounter{theorem}{2}
\begin{definition}
    We say that a nonempty set $G$ is a \emph{group under $\star$} if
    \begin{enumerate}
        \item (Closure) $\star$ is an operation, so $g\star h\in G$ for all $g,h\in G$.
        \item (Associativity) $g\star (h \star k)=(g\star h)\star k$ for all $g,h,k\in G$.
        \item (Identity) There exists an \emph{identity element} $e\in G$ such $e\star g = g\star e = g$
              for all $g\in G$.
        \item (Inverses) Every element $g\in G$ has an \emph{inverse} $\inv g$ 
              such that $g\star \inv g= \inv g \star g = e$.
    \end{enumerate}
    If $G$ is a finite group, the number of elements of $G$ is written $\abs{G}$, and is called the
    \emph{order} of $G$. If $G$ is infinite then we say that $\abs{G}=\infty$ , ther order of $G$
    is \emph{infinite}.
\end{definition}

\begin{theorem}
    The set of symmetries of a graph forms a group (under composition).
\end{theorem}

\setcounter{theorem}{5}
\begin{definition}
    Suppose that $G$ is a group and $g,h\in G$. If $g\star h = h\star g$ then we say
    that $g$ and $h$ \emph{commute}. If $g\star h=h\star g$ for all $g,h\in G$, then
    we say $G$ is an \emph{abelian} group.
\end{definition}

\subsection{Symmetries give groups}

\setcounter{theorem}{1}
\begin{definition}
    The set $S_n$ of all symmetries of $\{1,...,n\}$ is called the \emph{symmetric group}. 
\end{definition}

\begin{lemma}
    The set $S_n$ is a group under composition of order $\abs{S_n}=n!$.
\end{lemma}

\setcounter{theorem}{4}
\begin{definition}
    The set of \emph{invertible $n\times n$} matrices with entries in $\R$ is denoted
    $\GL(n,\R)$. Similarly, if $p$ is a prime, then the set of invertible $n\times n$
    matrices with entries in $Z_p$ is denoted $\GL(n,\Z_p)$.
\end{definition}

\begin{theorem}
    $\GL(n,\R)$ is a group under matrix multiplication.
\end{theorem}

\subsection{Product groups and first properties of groups}

\begin{theorem}
    Let $G,H$ be groups. The product
    \begin{align*}
        G\times H=\{(g,h)|g\in G, h\in H\}
    \end{align*}
    has the natrual structure of a group as follows:
    \begin{itemize}
        \item The group operation is $(g,h)\star(g',h'):=(g\star_G g', h\star_H h')$, 
        where we write $\star_G$ and $\star_H$ for the group operations in $G$ and $H$, respectively.
        \item The identity $e$ in $G\times H$ is $e:=(e_G, e_H)$, where $e_G$ and 
        $e_H$ are the identity elements of $G$ and $H$, respectively.
        \item The inverse of $(g,h)$ is $(\inv g, \inv h)$, where the inverse of $g$ is taken in $G$,
        and the inverse of $h$ is taken in $H$.
    \end{itemize}
\end{theorem}

\begin{corollary*}
    Let $G,H$ be groups. Then
    \begin{align*}
        |G\times H| = |G||H|.
    \end{align*}
    \begin{proof}
        $$\forall g\in G,\:|\{(g,h) : h\in H\}|=|H| \:\Rightarrow\: |G\times H| = \sum_{g\in G}|H| = |G||H|.$$
    \end{proof}
\end{corollary*}

\setcounter{theorem}{5}
\begin{lemma}
    Let $G$ be a group. If $g,h\in G$, then
    \begin{enumerate}
        \item There is one and only one element $k\in G$ such that $k\star g=h$.
        \item There is one and only one elmenet $k\in G$ such that $g\star k=h$.
    \end{enumerate}
\end{lemma}

\setcounter{theorem}{7}
\begin{corollary}
    The following statements are true:
    \begin{enumerate}
        \item In a group you can always cancel: if $g\star s=g\star t$ 
        then $s=t$. Similarly, if $s\star g=t\star g$ then $s=t$.
        \item Inversese are unique.
        \item A group has only one identity.
    \end{enumerate}
\end{corollary}
\end{document}
