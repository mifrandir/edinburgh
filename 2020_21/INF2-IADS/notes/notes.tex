\documentclass{article}
\usepackage{notes-preamble}
\mkthms
\usepackage{pseudo}
\begin{document}
\title{Introduction to Algorithms and Data Structures (YEAR2)}
\author{Franz Miltz}
\maketitle
\tableofcontents
\pagebreak


\section{Asymptotics Analysis}



\B{Asymptotic theory} makes precise quantitive statements about efficiency of algorithms themselves.
\begin{definition}
	Let $f,g:\N\to\R_{\geq 0}$ be functions. Then
	\text{$f\in o(g)$} if, and only if, 
	\begin{align*}
		\forall c>0,\:\exists N\st \forall n \geq N, f(n)<cg(n)
	\end{align*}
	where $c\in\R$ and $N,n\in\N$.
\end{definition}
\begin{theorem}
	Let $f:\N\to\R_{\geq 0}$ and let $o(f)$ refer to some
	function within the set $o(f)$. Then
	\begin{itemize}
		\item $co(f)=o(f)$ where $c\in\R$,
		\item $o(f) + o(f) = o(f)$.
	\end{itemize}
\end{theorem}
\begin{theorem}
	Let $f,r:\N\to\R_{\geq 0}$ and let $a,b\in\R$. Then
	\begin{align*}
		f=o(g) \Leftrightarrow af=o(bg).	
	\end{align*}
\end{theorem}
\begin{definition}
	Let $f,g:\N\to\R_{\geq 0}$. Then $f=\omega(g)$ if, and only if, $g=o(f)$.
\end{definition}
\begin{definition}
	Let $f,g:\N\to\R_{\geq 0}$. Then $f\in O(g)$ if, and only if,
	\begin{align*}
		\exists C > 0\st \exists N \st \forall n \geq N,\: f(n) \geq Cg(n)
	\end{align*}
	where $C\in\R$ and $N,n\in\N$.\\
	We call $g$ an \B{asymptotic upper bound} for $f$.
\end{definition}
\begin{definition}
	Let $f,g:\N\to\R_{\geq 0}$. Then $f\in\Omega(g)$ if, and only if, $g\in O(f)$.\\
	We call $g$ an \B{asymptotic lower bound} for $f$.
\end{definition}
\begin{definition}
	Let $f,g:\N\to\R_{\geq 0}$. Then $f\in\Theta(g)$ if, and only if, $f\in O(f) \cap \Omega(f)$.\\
	We call $g$ an \B{asymptotic tight bound} for $f$.
\end{definition}
\begin{theorem}
	Let $f,g:\N\to\R_{\geq 0}$. Then $f\in\Theta(g)$ if, and only if, $g\in\Theta(f)$.
\end{theorem}
\begin{theorem}
	\begin{align*}
		\lg n! = \Theta(n \lg n).
	\end{align*}
\end{theorem}


\subsection{Recurrence relations}


\begin{theorem}[The Master Theorem]
	Assume a recurrence relation $T$ has the form
	\begin{align*}
		T(n) = \begin{cases}
			\Theta(1) &\text{if $n\leq n_0$}\\
			aT(n/b) + \Theta(n^k) &\text{if $n>n_0$}
		\end{cases}.
	\end{align*}
	Then, with $e=\log_b a$, 
	\begin{align*}
		T(n) = \begin{cases}
			\Theta(n^e) &\text{if $e>k$}\\
			\Theta(n^k\lg n) &\text{if $e=k$}\\
			\Theta(n^k) &\text{if $e<k$}
		\end{cases}
	\end{align*}
\end{theorem}



\section{Algorithms and cost models}



\begin{definition}
	The \B{cost} of an algorithm is a quantity to measure its performance.\\
	The cost may be defined in different ways depending on
	how in depth the analysis is supposed to be and what
	is of interest in a particular situation.
\end{definition}
Note that the cost model needs to be specified when comparing algorithms.
\begin{definition}
	We considering the cost of an algorithm for a specific input size there are different cases to consider:
	\begin{itemize}
		\item \B{worst-case} cost: the single worst cost out of all possible inputs
		\item \B{best-case} cost: the single best cost out of all possible inputs
		\item \B{average-case} cost: the average over all the costs for all possible inputs
	\end{itemize}
\end{definition}
\begin{theorem}
	Let $A$ be an algorithm and let $T_w$ be its worst-case runtime. Then, if $T_w=O(g)$ for some function $g$,
	we know that the runtime of $A$ \emph{in general} is $O(g)$.
\end{theorem}
\begin{theorem}
	Let $A$ be an algorithm and let $T_b$ be its best-case runtime. Then, if $T_b=\Omega(g)$ for some function $g$,
	we know that the runtime of $A$ \emph{in general} is $\Omega (g)$.
\end{theorem}


\subsection{Bubblesort}


\subsection{Insertion Sort}


\subsection{Mergesort}


\subsection{Heapsort}


\subsection{Quicksort}


\subsubsection{Algorithm}

Algorithm for input array $A$:
\begin{enumerate}
	\item If $|A|<2$, do nothing. Otherwise partition the array:
	\begin{tabular}{|c|c|c|}
		\hline
		$\leq$ pivot & pivot & $\geq$pivot\\
		\hline
	\end{tabular}
	\item Sort the resulting subarrays recursively.
\end{enumerate}

\noindent Main algorithm:
\begin{minted}{python}
def quick_sort(A, p, r):
  if p < r:
    split = partition(A, p, r)
	quick_sort(A, p, split-1)
	quick_sort(A, split+1, r)
\end{minted}

\noindent Partitioning algorithm:
\begin{minted}{python}
def partition(A, p, r):
  pivot = A[r].key
  i = p - 1
  for j in range(p,r):
    if A[j] <= pivot:
      i += 1
      exchange(A, i, j)
  exchange(A, i+1, r)
  return i+1
\end{minted}

\subsubsection{Analysis}

\begin{center}
	\begin{tabular}{| l | c |}
		\hline
		Worst Case & $\Theta(n^2)$\\
		\hline
		Average Case & $\Theta(n\lg n)$\\
		\hline
	\end{tabular}
	\end{center}


\section{Collections}



\subsection{Sets and dictionaries}


\begin{definition}
	A \B{finite set} containing values of type $X$ is a datastructure with the following interface
	\begin{align*}
		\B{contains} &: X\to \texttt{bool}\\
		\B{insert} &: X \to \texttt{void}\\
		\B{delete} &: X \to \texttt{void}\\
		\B{isEmpty} &: \texttt{void} \to \texttt{bool}
	\end{align*}
\end{definition}
\begin{definition}
	A \B{dictionary} mapping keys of type $X$ to values of type $Y$
	is a datastructure with the following interface
	\begin{align*}
		\B{lookup} &: X \to Y\\
		\B{insert} &: X \to Y \to \texttt{void}\\
		\B{delete} &: X \to \texttt{void}\\
		\B{isEmpty} &: \texttt{void} \to \texttt{bool}
	\end{align*}
\end{definition}
\begin{definition}
	A \B{hash table} is a datastructure that uses a \B{hash function} $f: X \to [0,m-1]$
	to store key value pairs $(X,Y)$ in an array of length $m$.\\
	The \B{load factor} $\alpha$ is the average number of elements associated with each cell
	of the underlying array. If $f$ is a perfect hash function, i.e. it maps all inputs uniformly
	onto $[0,m-1]$, then $\alpha$ is given by 
	\begin{align*}
		\alpha = n / m.
	\end{align*}
\end{definition}
\begin{proposition}
	Using linked lists to manage the elements in each bucket of a hash table, then
	\begin{itemize}
		\item a lookup of a key $k$  takes on average $\Theta(\alpha)$ comparisons, and
		\item a lookup of a key $k$ takes in the worst-case $\Theta(n)$ comparisons.
	\end{itemize}
\end{proposition}
\begin{definition}
	\B{Open addressing} is an algorithm to handle hash collisions within a hash table 
	by using a hash function $f: (X, [0,m-1]) \to [0,m-1]$ that generates a permutation of all possible
	hash codes for each possible input $X$. An element $x\in X$ is then stored the hash code $f(x, i)$
	if and only if all $f(x,j)$ for $j < i$ are already occupied.
\end{definition}
\begin{proposition}
	In a hash table with load factor $\alpha$ that uses open addressing requires, on average, $1/(1-\alpha)$ 
	probes for an unsuccessful lookup and less for a successful one.
\end{proposition}
\begin{definition}
	A \B{binary tree} consists of a value and at most one left child and at most one right child where
	children are other binary trees.
\end{definition}
\begin{definition}
	An \B{ordered binary tree} with a given ordering $(<) : X \to X \to \texttt{bool}$ is a binary tree 
	with a root $x$ such that
	\begin{align*}
		\forall y \in L(x), y.\texttt{key} < x.\texttt{key} 
		\text{ and } \forall y \in R(x), x.\texttt{key} < y.\texttt{key},
	\end{align*}
	and both $L(x)$ and $R(x)$ are ordered.
\end{definition}
\begin{proposition}
	For a prefectly balanced ordered binary tree, the lookup time is be $O(\log n)$.\\
	More generally, for any such tree with a maximum depth $d$ of at most $d=2 \lg n$ the
	lookup time is $O(\log n)$.
\end{proposition}
\begin{center}
	
\begin{tabular}{| l | l |}
	\hline
	Type & \\
	\hline
	Set & \begin{tabular}{ l | l | l | l }
		Implementation & \texttt{contains} & \texttt{insert} & \texttt{delete}\\
		\hline
		Linked List (simple) 
		& $\Theta(n)$, $\Theta(n)$
		& $\Theta(1)$, $\Theta(1)$
		& $\Theta(n)$, $\Theta(n)$\\
		\hline
		Ordered Linked List (simple) 
		& $\Theta(\lg n)$, $\Theta(\lg n)$
		& $\Theta(\lg n)$, $\Theta(\lg n)$
		& $\Theta(\lg n)$, $\Theta(\lg n)$\\
		\hline
		Hash Table (linked bucket list)
		& $\Theta(\alpha)$, $\Theta(n)$
		& $\Theta(1)$, $\Theta(1)$
		& $\Theta(\alpha)$, $\Theta(n)$\\
		\hline
		Hash Table (open addressing)
		& $\Theta(1/(1-\alpha))$, $\Theta(n)$
		& $\Theta(1/(1-\alpha))$, $\Theta(n)$
		& - \\
		\hline
		Ordered Binary Tree
		& $\Theta(\lg n)$, $\Theta(n)$
		& $\Theta(\lg n)$, $\Theta(n)$ 
		& $\Theta(\lg n)$, $\Theta(n)$\\
		\hline
		Red-Black Tree
		& $\Theta(\lg n)$, $\Theta(\lg n)$
		& $\Theta(\lg n)$, $\Theta(\lg n)$
		& $\Theta(\lg n)$, $\Theta(\lg n)$
	\end{tabular} \\
	\hline
\end{tabular}
\end{center}


\subsection{The Heap data structure}


\begin{definition}
	A heap is an \B{almost-complete} binary tree:
	\begin{itemize}
		\item All leaves are either at depth $h-1$ or $h$ (where $h$ is the height of the tree)
		\item The depth-$h$ leaves all appear consecutively from left-to-right
	\end{itemize}
\end{definition}
\begin{lemma}
	The height $h$ of a heap with $n$ is in the range given by
	\begin{align*}
		\lg(n)-1 < h \leq \lg(n).
	\end{align*}
\end{lemma}
\begin{theorem}
	The heap operations have the following runtimes:
	\\
	\begin{center}
	\begin{tabular}{| l | c |}
		\hline
		Operation & Asymptotic Runtime\\
		\hline
		Max & $\Theta(1)$\\
		\hline
		Heapify & $O(\lg(n))$\\
		\hline
		Extract-Max & $O(\lg(n))$\\
		\hline
		Insert & $O(\lg(n))$\\
		\hline
		Build & $O(n)$\\
		\hline
	\end{tabular}
	\end{center}
\end{theorem}


\subsection{Graphs}


\subsubsection{Directed and Undirected Graphs}

\begin{definition}
	A graph $G$ is defined as
	\begin{align*}
		G = (V,E)
	\end{align*}
	where $V$ is the set of vertices and $E\subseteq V\times V$ is the set of edges.
\end{definition}

\begin{definition}
	A graph $G$ is \B{undirected} if and only if
	\begin{align*}
		\forall v,w\in V,\: (v,w)\in E \Leftrightarrow (w,v)\in E.
	\end{align*}
	If a graph is not undirected, it is \B{directed}.
\end{definition}

\subsection{Adjacency Matrices vs Lists}

\begin{definition}
	Let $G=(V,E)$ be a graph with $n$ vertices $V=\{v_1, ..., v_n\}$.
	Then the \B{adjacency matrix} of $G$ is the $n\times n$ matrix
	$A=(a_{ij})_{i,j\in[1,n]}$ with 
	\begin{align*}
		a_{ij} = \begin{cases}
			1 &\text{if } (v_i,v_j)\in E\\
			0 &\text{otherwise}
		\end{cases}.
	\end{align*}
\end{definition}

\begin{theorem}
	Let $G=(V,E)$ with $|V|=n$ and $|E|=m$. Then the following 
	complexities apply:
	\begin{center}
	\begin{tabular}{| l | c | c |}
		\hline
		 Measure & Matrix & List\\
		\hline
		Space & $\Theta(n^2)$ & $\Theta(n+m)$ \\
		\hline
		Time to check if $w$ adjacent to $v$ & $\Theta(1)$ & $\Theta(\text{out}(v))$\\
		\hline
		Time to visit all $W$ adjacent to $v$ & $\Theta(n)$ & $\Theta(\text{out}(v))$\\
		\hline
		Time to visit all edges & $\Theta(n^2)$ & $\Theta(n+m)$\\
		\hline
	\end{tabular}
\end{center}
\begin{lemma}
	A graph $G=(V,E)$ with $|V|=n$ and $|E|=m$ is
	\begin{itemize}
		\item \B{dense} if $m$ is close to $n^2$,
		\item \B{sparse} if $m$ is much smaller than $n^2$. 
	\end{itemize}
\end{lemma}
\end{theorem}

\subsection{Graph traversal}

\begin{definition}
	A \B{traversal} is a strategy for visiting all vertices of a graph.
\end{definition} 

\section{Dynamic Programming}

\subsection{Decorators in Python}

\begin{minted}{python}
def memoise(f):
  memo = {}
  def check(s):
    if s not in memo:
      memo[s] = f(s)
    return memo[s]
  return check
\end{minted}
\begin{example}
Using the Fibonacci function:
\begin{minted}{python}
# inefficient fib function
def fib(n):
  if n == 0:
    return 0
  if n == 1:
    return 1
  return fib(n-1) + fib(n-2)

# more efficient wrapper
fib = memoise(fib)
\end{minted}
\end{example}

\subsection{The coin-changing problem}

\begin{definition}
	Given a value $v\in\N_0$, plus a sequence of coin values $c_1,c_2,...,c_k\in\N_0$,
	find a \emph{multiset} $S$ of coins whose values sum to $v$, whose cardinality
	of $S$ is the minimum possible for $v$ is this coin system.
\end{definition}

\subsection{Principles}

\begin{enumerate}
	\item Need is to see that computing the optimum solution for our original instance
		  can be achieved by finding solutions to problems of the same type, and combining
		  them.
    \item Solution needs to be expressible in terms of a \emph{recurrence}, where the RHS
		  contains one or more recursive calls for smaller instances of the same problem.
    \item We need to be able to organise storage for the results for all possible subproblems.
    \item We need an algorithm to control the order in which subproblems are solved.
\end{enumerate}

\subsection{Seam Carving}

\begin{definition}
	Given an $m\times n$ matrix $A$, a \emph{vertical seam} is sequence $(s_k)_{k\in\N}$
	of natural numbers such that
	\begin{align*}
		\forall i\leq m,\: s_i\in(1,n)\hs\text{and}\hs\forall i<m,\: \abs{s_i - s_{i+1}} \leq 1.
	\end{align*}
	A \emph{horizontal seam} of $A$ is a sequence $(s_n)_{n\in\N}$ of natural numbers
	such that
	\begin{align*}
		\forall j\leq n,\: s_j\in(1,m)\hs\text{and}\hs\forall j<n,\: \abs{s_j - s_{j+1}} \leq 1.
	\end{align*}
\end{definition}

\begin{definition}
	Let $A$ be an $m\times n$ matrix and let $e_A:[1,m]\times[1,n]\to\R$ be an
	energy function. Then the \emph{energy} of a seam $s=(s_k)_{k\in\N}$ is defined as
	\begin{align*}
		e(s) := \begin{cases}
			\sum_{i=1}^m e_A(i,s_i) &\text{if } s \text{ is a vertical seam},\\
			\sum_{j=1}^m e_A(s_j,j) &\text{if } s \text{ is a horizontal seam}.
		\end{cases}
	\end{align*}
\end{definition}

\begin{definition}
	$L_1$ gradient scoring can be written as
	\begin{align*}
		e_A(i,j) := \abs{\frac{\p}{\p x} A}_{ij} + \abs{\frac{\p}{\p y} A}_{ij}
	\end{align*}
	where $\p/\p x$ and $\p/\p y$ are gradient functions defined in the image
	processing context.
\end{definition}

\begin{definition}
	Let $A$ be an $m\times n$ matrix. 
	For every $(i,j)\in[1,m]\times[1,n]$, the minimum cost of a vertical seam 
	ending in $(i,j)$ is given by $c_A(i,j)$.
\end{definition}

\begin{theorem}
	Let $A$ be an $m\times n$ matrix and let $e_A$ be an energy function for
	$A$ such that
	\begin{align*}
		\forall i \leq m,\: e_A(i,1) = e_A(i,n) = \infty.
	\end{align*}
	Then $c_A$ is given by
	\begin{align*}
		c_A(i,j) = e_A(i,j) + \begin{cases}
			0 &\text{if }i=1,\\
			m_{i-1,j} &\text{if }i>1
		\end{cases}
	\end{align*}
	where $m_{i,j}$ is
	\begin{align*}
		m_{i,j} = \min\{c_A(i,j-1), c_A(i,j), c_A(i,j+1)\}.
	\end{align*}
\end{theorem}

\subsection{Edit distance}

\begin{definition}
	An \emph{alignment} of two sequences $s\in\Sigma^m$, $t\in\Sigma^n$ is
	any padding (with some $-$ insertions) $s'$ of $s$, and $t'$ of $t$ such that
	\begin{align*}
		\abs{s'}&=\abs{t'},\\
		(s'_i\not=-)&\text{ or }(t'_i\not=-)\hs\text{for all }i\in[1,|s'|].
	\end{align*}
\end{definition}

\begin{definition}
	The score of an alignment is the total number of \emph{insertions},
	\emph{deletions} and \emph{substitutions}.
\end{definition}

\begin{definition}
	The \emph{edit distance} $d(s,t)$ between two strings $s,t\in\Sigma^*$ is the
	minimum number of operations possible for an alignment of those strings.
\end{definition}

\begin{theorem}
	The edit distance $d$ between two strings $s_m=(s_k)_{k\leq m}$ 
	and $t_n=(t_k)_{k\leq n}$ is given by 
	\begin{align*}
		d(s_m,t_n) = \begin{cases}
			m &\text{if }n=0,\\
			n &\text{if }m=0,\\
			d(s_{m-1}, t_{n-1}) &\text{if }s_m=t_n,\\
			1 + h_{m,n} &\text{if }s_m\not=t_n
		\end{cases}
	\end{align*}
	where $h$ is given as
	\begin{align*}
		h_{m,n}= 1 + \min\{d(s_{m-1},t_{n-1}), d(s_{m-1}, t_n), d(s_m, t_{n-1})\}.	
	\end{align*}
\end{theorem}

\subsection{Parsing for context free grammars}

\begin{definition}
	A context-free grammar is in \emph{Chomsky normal form} if for
	each production the right-hand side consists of
	\begin{itemize}
		\item \emph{either} just two non-terminals (e.g $X\to YZ$)
		\item \emph{or} just one terminal (e.g. $X\to +$).
	\end{itemize}
\end{definition}

\begin{lemma}
	Every context-free grammar can be transformed into an
	equivalent one in Chomsky normal form (disregarding
	the empty string).
\end{lemma}

\begin{definition}
	The \emph{CYK algorithm} allows parsing of a context-free
	grammar in Chomsky normal form.
	\begin{pseudo}
		\textbf{CYK}$(s,G)$:\\+
			$n\leftarrow\text{length}(s)$\\
			$\text{table}\leftarrow\text{Matrix }A_{ij}\text{ for }(i,j)\in[0,n-1][1,n]$\\
			\textbf{for} $j\in\{1,..,n\}$:\\+
				\textbf{for} $(X\to t)\in G$:\\+
					\textbf{if} $t=s[j-1]$:\\+
						add $X$ to table$[j-1,j]$\\--
				\textbf{for} $i\in\{j-2,..,0\}$:\\+
					\textbf{for} $k\in\{i+1,..,,j-1\}$:\\+
						\textbf{for} $(X\to YZ)\in G$:\\+
							\textbf{if} $Y\in\text{table}[i,k]$ and $Z\in\text{table}[k,j]$:\\+
								add $X$ to table$[i,j]$\\-----
            \textbf{return} table
    \end{pseudo}
\end{definition}

\begin{theorem}
	Given a grammar $G$ in Chomsky normal form with $n$ 
	productions and an input of length $n$, the run time
	of the CYK algorithm is $O(mn^3)$.
\end{theorem}

\section{P and NP}

\subsection{Decision problems}

\begin{definition}
	A computational problem $Q$ is solvable in \emph{polynomial
	time} if there is some fixed $r\in\R$, and some deterministic
	algorithm $A$, which returns a correct solution for every
	instance $\mathcal{J}$ in time at most $O(\abs{\mathcal{J}}^r)$.
\end{definition}

\begin{definition}
	A computational problem $Q$ is a \emph{decision problem} if
	it can be described in terms of a collection of potential
	solutions $S$, where $Q(\mathcal{J})=1$ if there is a solution
	in $S$ which solves the instance $\mathcal{J}$ and $Q(\mathcal{J})=0$
	otherwise.
\end{definition}

\subsection{Complexity classes}

\begin{definition}
	The \emph{complexity class P} is the class of decision problems
	$Q$ for which there is a polynomial-time aglorithm to compute $Q$
	exactly on all input instances.
\end{definition}

\begin{definition}
	Consider a decision problem $Q$ with respect to its collection
	of potential solutions $S$. We say that a two-parameter algorithm 
	$A$ is a \emph{verifier} for $Q$ iff for all instances
	$\mathcal{J}$ of $Q$
	\begin{align*}
		\exists y\in S.\hs A(\mathcal{J},y)=1\hs\Leftrightarrow\hs Q(\mathcal{J})=1
	\end{align*}
\end{definition}

\begin{definition}
	The \emph{complexity class NP} is the class of decision problems $Q$
	for which there is a verifier $A=A(\mathcal{J}, y)$ which runs in time
	polynomial in the size $\abs{\mathcal{J}}$ of the instance.
\end{definition}

\subsection{Reductions between problems}

\begin{definition}
	A problem $R$ can be \emph{reduced} to the problem $Q$ if there
	is a polynomial-time computational function $f:\{0,1\}^*\to\{0,1\}^*$
	such that for all instances $\mathcal{J}$ of $R$
	\begin{align*}
		R(\mathcal{J})=1 \hs\Leftrightarrow\hs Q(f(\mathcal{J}))=1.
	\end{align*}
	We write $R\leq_P Q$.
\end{definition}

\begin{definition}
	A decision problem $Q$ is said to be \emph{NP-complete} if it
	belongs to the class NP, and it is also the case that for every
	problem $R$ in NP, $R\leq_P Q$.
\end{definition}

\end{document}
