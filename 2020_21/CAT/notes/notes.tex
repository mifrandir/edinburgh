\documentclass{article}
\usepackage{notes-preamble}
\usepackage{enumitem}
\begin{document}
\mkthms
\title{Category Theory (Steve Awodey)}
\author{Franz Miltz}
\maketitle
\tableofcontents
\pagebreak

\section{Categories}

\subsection{Definition}

\begin{definition}[Category]
    A \emph{category} consists of the following data:
    \begin{itemize}
        \item Objects: $A$, $B$, $C$, \dots
        \item Arrows: $f$, $g$, $h$, \dots
        \item For each arrow $f$, there are given objects \begin{align*}
            \dom(f),\hs \cod(f)
        \end{align*}
        called the \emph{domain} and \emph{codomain} of $f$. We write
        \begin{align*}
            f:A\to B
        \end{align*}
        to indicate that $A=\dom(f)$ and $B=\cod(f)$.
        \item Given arrows $f:A\to B$ and $g:B\to C$, that is, with \begin{align*}
            \cod(f) = \dom(g)
        \end{align*}
        there is given an arrow
        \begin{align*}
            g\circ f: A\to C
        \end{align*}
        called the \emph{composite} of $f$ and $g$.
        \item For each object $A$ there is given an arrow \begin{align*}
            1_A : A\to A
        \end{align*}
        called the \emph{identity arrow} of $A$.
    \end{itemize}
    These data are required to satisfy the following laws \begin{enumerate}[label=C\arabic*.]
        \item \emph{Associativity}: \begin{align*}
            f \circ (g \circ h) = (f\circ g) \circ h
        \end{align*}
        for all arrows $f:A\to B$, $g:B\to C$, $h:C\to D$.
        \item \emph{Unit}: \begin{align*}
            f \circ 1_A = f = 1_B \circ f
        \end{align*}
        for all $f:A\to B$.
    \end{enumerate}
\end{definition}

\subsection{Examples}

\begin{definition*}[Posets and monotone functions]
    A partially ordered set or \emph{poset} is a set $A$ equipped
    with a binary relation $a\leq_A b$ such that the following conditions
    hold for all $a,b,c\in A$: 
    \begin{enumerate}[label=P\arabic*.]
        \item \emph{reflexivity}: $a\leq_A a$.
        \item \emph{transitivity}: If $a\leq_A b$ and $b\leq_A c$, then $a\leq_A c$.
        \item \emph{antisymmetry}: If $a\leq_A b$ and $b\leq_A a$, then $a=b$.
    \end{enumerate}
    A function $f:A\to B$ between two posets is \emph{monotone} if
    \begin{align*}
        a \leq_A a' \text{ implies } f(a) \leq_A f(a')\hs \text{for all }a,a'\in A.
    \end{align*}
\end{definition*}

\begin{definition}[Functors]
    A \emph{functor}
    \begin{align*}
        F:\mathbf{C}\to\mathbf{D}
    \end{align*}
    between categories $\mathbf{C}$ and $\mathbf{D}$ is a mapping
    of objects to objects and arrows to arrows, in such a way that
    \begin{enumerate}[label=F\arabic*.]
        \item $F(f:A\to B)=F(f):F(A)\to F(B)$
        \item $F(1_A)=1_{F(A)}$
        \item $F(f\circ g)=F(f)\circ F(g)$
    \end{enumerate}
\end{definition}

\begin{definition*}[Monoids and homomorphisms]
    A \emph{monoid} is a set $M$ equipped with a binary operation
    $\cdot : M\times M \to M$ and a unit element $u\in M$ such that
    for all $x,y,z\in M$
    \begin{align*}
        x\cdot (y\cdot z) = (x\cdot y) \cdot z 
    \end{align*} 
    and
    \begin{align*}
        u \cdot x = x = x \cdot u.
    \end{align*}
    For monoids $M,N$ a \emph{monoid homomorphism} is a function
    $h:M\to N$ such that for all $a,b\in M$,
    \begin{align*}
        h(a\cdot_M b) = h(a)\cdot_N h(b)
    \end{align*}
    and
    \begin{align*}
        h(u_M) = u_N.
    \end{align*}
\end{definition*}

\paragraph{Examples of categories}
\begin{itemize}
    \item \textbf{Sets}: of sets and functions
    \item $\textbf{Sets}_\text{fin}$: of finite sets and functions
    \item \textbf{Pos}: of posets and monotone functions
    \item \textbf{Rel}: of sets and binary relations
    \item \textbf{Cat}: of categories and functors
    \item \textbf{Dis$(X)$}: of elements in $X$ and all identity arrows
    \item \textbf{Mon}: of monoids and monoid homomorphisms
\end{itemize}


\end{document}
