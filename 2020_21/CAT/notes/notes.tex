\documentclass{article}
\usepackage{notes-preamble}
\usepackage{enumitem}
\usepackage{extarrows}
\begin{document}
\mkawodeythms
\title{Category Theory (Steve Awodey)}
\author{Franz Miltz}
\maketitle
\tableofcontents
\pagebreak

\section{Categories}

\subsection{Definition}

\begin{definition}[Category]
    A \emph{category} consists of the following data:
    \begin{itemize}
        \item Objects: $A$, $B$, $C$, \dots
        \item Arrows: $f$, $g$, $h$, \dots
        \item For each arrow $f$, there are given objects \begin{align*}
            \dom(f),\hs \cod(f)
        \end{align*}
        called the \emph{domain} and \emph{codomain} of $f$. We write
        \begin{align*}
            f:A\to B
        \end{align*}
        to indicate that $A=\dom(f)$ and $B=\cod(f)$.
        \item Given arrows $f:A\to B$ and $g:B\to C$, that is, with \begin{align*}
            \cod(f) = \dom(g)
        \end{align*}
        there is given an arrow
        \begin{align*}
            g\circ f: A\to C
        \end{align*}
        called the \emph{composite} of $f$ and $g$.
        \item For each object $A$ there is given an arrow \begin{align*}
            1_A : A\to A
        \end{align*}
        called the \emph{identity arrow} of $A$.
    \end{itemize}
    These data are required to satisfy the following laws \begin{enumerate}[label=C\arabic*.]
        \item \emph{Associativity}: \begin{align*}
            f \circ (g \circ h) = (f\circ g) \circ h
        \end{align*}
        for all arrows $f:A\to B$, $g:B\to C$, $h:C\to D$.
        \item \emph{Unit}: \begin{align*}
            f \circ 1_A = f = 1_B \circ f
        \end{align*}
        for all $f:A\to B$.
    \end{enumerate}
\end{definition}

\subsection{Examples}

\begin{definition*}[Posets and monotone functions]
    A partially ordered set or \emph{poset} is a set $A$ equipped
    with a binary relation $a\leq_A b$ such that the following conditions
    hold for all $a,b,c\in A$: 
    \begin{enumerate}[label=P\arabic*.]
        \item \emph{reflexivity}: $a\leq_A a$.
        \item \emph{transitivity}: If $a\leq_A b$ and $b\leq_A c$, then $a\leq_A c$.
        \item \emph{antisymmetry}: If $a\leq_A b$ and $b\leq_A a$, then $a=b$.
    \end{enumerate}
    A function $f:A\to B$ between two posets is \emph{monotone} if
    \begin{align*}
        a \leq_A a' \text{ implies } f(a) \leq_A f(a')\hs \text{for all }a,a'\in A.
    \end{align*}
\end{definition*}

\begin{definition}[Functors]
    A \emph{functor}
    \begin{align*}
        F:\cat{C}\to\cat{D}
    \end{align*}
    between categories $\cat{C}$ and $\cat{D}$ is a mapping
    of objects to objects and arrows to arrows, in such a way that
    \begin{enumerate}[label=F\arabic*.]
        \item $F(f:A\to B)=F(f):F(A)\to F(B)$
        \item $F(1_A)=1_{F(A)}$
        \item $F(f\circ g)=F(f)\circ F(g)$
    \end{enumerate}
\end{definition}

\begin{definition*}[Monoids and homomorphisms]
    A \emph{monoid} is a set $M$ equipped with a binary operation
    $\cdot : M\times M \to M$ and a unit element $u\in M$ such that
    for all $x,y,z\in M$
    \begin{align*}
        x\cdot (y\cdot z) = (x\cdot y) \cdot z 
    \end{align*} 
    and
    \begin{align*}
        u \cdot x = x = x \cdot u.
    \end{align*}
    For monoids $M,N$ a \emph{monoid homomorphism} is a function
    $h:M\to N$ such that for all $a,b\in M$,
    \begin{align*}
        h(a\cdot_M b) = h(a)\cdot_N h(b)
    \end{align*}
    and
    \begin{align*}
        h(u_M) = u_N.
    \end{align*}
\end{definition*}

\paragraph{Examples of categories}
\begin{itemize}
    \item \textbf{Sets}: of sets and functions
    \item $\textbf{Sets}_\text{fin}$: of finite sets and functions
    \item \textbf{Pos}: of posets and monotone functions
    \item \textbf{Rel}: of sets and binary relations
    \item \textbf{Cat}: of categories and functors
    \item \textbf{Dis$(X)$}: of elements in $X$ and all identity arrows
    \item \textbf{Mon}: of monoids and monoid homomorphisms
\end{itemize}

\subsection{Isomorphisms}

\begin{definition}
    In any category $\cat{C}$, an arrow $f:A\to B$ is called an
    \emph{isomorphism} if there exists an arrow $g:B\to A$ in
    $\cat{C}$ such that 
    \begin{align*}
        g\circ f = 1_A\hs\text{and}\hs f\circ g = 1_B.
    \end{align*}
    We say $A$ is isomorphic to $B$ if there exists an isomorphism
    between them and write $A\cong B$.
\end{definition}

\begin{definition}
    A \emph{group} $G$ is a monoid with an inverse $\inv g$ for every element $g$.
    Thus, $G$ is a category with one object, in which every arrow is an isomorphism.
\end{definition}

\begin{theorem}[Cayley]
    Every group $G$ is isomorphic to a group of permutations.
\end{theorem}

\begin{theorem}
    Every category $\cat{C}$ with a set of arrows is isomorphic
    to one in which the objects are sets and the arrows are functions.
\end{theorem}

\subsection{Construction of categories}

\begin{definition*}
    The product of two categories $\cat{C}$ and $\cat{D}$, written
    as
    \begin{align*}
        \cat{C}\times\cat{D}
    \end{align*} 
    has objects of the form $(C,D)$ for $C\in\cat{C}$ and $D\in\cat{D}$,
    and arrows of the form
    \begin{align*}
        (f,g):(C,D)\to (C',D')
    \end{align*}
    for $f:C\to C'\in\cat{C}$ and $g:D\to D'\in\cat{D}$.
\end{definition*}

\begin{definition*}
    The \emph{opposite} category $\catop C$ of a category $\cat C$ has
    the same objects as $\cat C$ and an arrow $f:C\to D$ in $\catop C$
    is an arrow $f:D\to C$ in $\cat C$. We write
    \begin{align*}
        f^* : D^* \to C^*
    \end{align*} 
    in $\catop C$ for $f:C\to D$ in $\cat C$. We further define
    \begin{align*}
         (1_{C^*}) &= (1_C)^*,\\ 
         f^*\circ g^*&= (g\circ f)^*.
    \end{align*}
    There are two obvious projection functors
    \begin{align*}
        \cat C \xlongleftarrow{\hs\pi_1\hs} \cat C\times \cat D \xlongrightarrow{\hs\pi_2\hs} \cat D
    \end{align*}
\end{definition*}

\begin{definition*}
    The \emph{arrow category} $\catar C$ of a category $\cat C$
    has the arrows of $\cat C$ as objects and an arrow $g$ from
    $f:A\to B$ to $f':A'\to B'$ in $\catar C$ is a pair of arrows
    $g=(g_1, g_2)$ in $\cat C$ such that 
    \begin{align*}
        g_2\circ f = f'\circ g_1.
    \end{align*}
    Observe that there are two functors
    \begin{align*}
        \cat C \xlongleftarrow{\hs\textbf{dom}\hs} \catar C \xlongrightarrow{\hs\textbf{cod}\hs} \cat C.
    \end{align*}
\end{definition*}

\begin{definition*}
    The \emph{slice category} $\cat C/C$ of a category $\cat C$ over an
    object $C\in\cat C$ has as objects all arrows $f\in\cat C$ such that
    $\dom(f)=C$ and an arrow from $f:X\to C$ to $f':X'\to C$ is an arrow
    $a:X\to X'$ in $\cat C$ such that $f'\circ a = f$.\\
    Note that there is a functor $U:\cat C/C \to \cat C$ defined by
    \begin{align*}
        U(A:X\to C) &= X &\text{for all objects }A\in\cat C/C,\\
        U(f:(A:X\to C)\to(B:X'\to C))&= a:X\to X' &\text{for all arrows }f\in\cat C/C.
    \end{align*}
    Further, if $g:C\to D\in\cat C$ is any arrow, then there is a composition
    functor,
    \begin{align*}
        g_*:\cat C/C\to\cat C/D
    \end{align*}
    defined by $g_*(f)= g\circ f$.
\end{definition*}

\end{document}
