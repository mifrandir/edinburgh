\documentclass{article}
\usepackage[a4paper]{geometry}
\usepackage{babel}
\usepackage{amsmath}
\usepackage{amssymb}
\usepackage{mathtools}
\usepackage{nicefrac}
\geometry{tmargin=2cm, bmargin=3cm}
\DeclarePairedDelimiter{\floor}{\lfloor}{\rfloor}
\title{CAP: Homework 6 (Workshop 71)}
\author{Franz Miltz (UNN: S1971811)}
\DeclareMathOperator{\sech}{sech}
\DeclareMathOperator{\csch}{csch}
\DeclareMathOperator{\arccot}{\text{cot}^{-1}}
\DeclareMathOperator{\arccsc}{\text{csc}^{-1}}
\DeclareMathOperator{\arccosh}{\text{cosh}^{-1}}
\DeclareMathOperator{\arcsinh}{\text{sinh}^{-1}}
\DeclareMathOperator{\arctanh}{\text{tanh}^{-1}}
\DeclareMathOperator{\arcsech}{\text{sech}^{-1}}
\DeclareMathOperator{\arccsch}{\text{csch}^{-1}}
\DeclareMathOperator{\arccoth}{\text{coth}^{-1}} 
\DeclareMathOperator{\dom}{dom}
\DeclareMathOperator{\range}{rng}
\DeclareMathOperator{\st}{s.t.}
\begin{document}
\maketitle
\section*{Question 1}
We need to find the height $h(t)$ at $t$ seconds after liftoff:
\begin{align*}
  h(t) = \int_0^t v(t)\: dt.
\end{align*}
We can solve the indefinite integral easily:
\begin{align*}
  \int v(t)\:dt &= \int \left(-gt -v_e \ln \left(\frac{m-rt}{m}\right)\right)\:dt \\
  &= -g \int t\: dt -v_e \int \ln\left(1-\frac{r}{m}t\right)\: dt \\
  &=-\frac{1}{2}gt^2-v_e \int \ln\left(1-\frac{r}{m}t\right)\:dt.
\end{align*}
For the second part we can let $u=1-\frac{r}{m}t$  and  $du = -\frac{r}{m}dt$. This gives
\begin{align*}
  \int\ln\left(1-\frac{r}{m}t\right)\: dt = -\frac{m}{r}\int \ln u\: du = -\frac{m}{r}\left(\left(1-\frac{r}{m}t\right)\ln\left(1-\frac{r}{m}t\right)-1+\frac{r}{m}t\right)+C'.
\end{align*}
The original integral therefore is
\begin{align*}
  \int v(t)dt =-\frac{1}{2}gt^2+\frac{mv_e}{r}\left(\left(1-\frac{r}{m}t\right)\ln\left(1-\frac{r}{m}t\right)+\frac{r}{m}t\right)+C.
\end{align*}
Since I combined the constants into $C$, this is $0$ at $t=0$ with $C=0$. Thus the formula for $h(t)$ is
\begin{align*}
  h(t)=562500\left(\left(1-\frac{2}{375}t\right)\ln\left(1-\frac{2}{375}t\right)+\frac{2}{375}t\right) -4.9\cdot t^2
\end{align*}
and therfore the height above ground of the rocket one minute after liftoff in metres $h(60)$ is
\begin{align*}
  h(60)=14844.1.
\end{align*}
\section*{Question 2}
\subsection*{a)}
We will first evaluate the definite integral $I$:
\begin{align*}
  I(t)=\int \sqrt{t}\ln t\: dt.
\end{align*}
Using integration by parts with 
\begin{align*}
  \begin{aligned}
    u &= \ln t\\
    du &= \frac{1}{t}\:dt
  \end{aligned}
  \hspace{1cm}
  \begin{aligned}
    dv &= \sqrt{t}\:dt\\
    v &= \frac{2}{3}t^{\nicefrac{3}{2}}
  \end{aligned}
\end{align*}
we get
\begin{align*}
  I(t)&=\frac{2}{3}t^{\nicefrac{3}{2}}\ln t - \frac{2}{3}\int \sqrt{t}\:dt\\
  &=\frac{2}{3}t^{\nicefrac{3}{2}}(\ln t - \frac{2}{3}) + C.
\end{align*}
We can evaluate this at $t=1$ and $t=4$:
\begin{align*}
  I(1)&=\frac{2}{3}\cdot 1\cdot (0-\frac{2}{3})+C=-\frac{4}{9}+C,\\
  I(4)&=\frac{2}{3}\cdot 8\cdot(\ln 4 - \frac{2}{3})+C=\frac{16}{3}(\ln 4 - \frac{2}{3})+C.
\end{align*}
Thus we can find the value of the original integral:
\begin{align*}
  \int_1^4 \sqrt{t}\ln t\:dt &= I(4)-I(1) \\
  &=\frac{16}{3}\ln 4 - \frac{28}{4}\\
  &= \frac{4}{3}(8\ln 2 - \frac{7}{3})\approx 4.282.
\end{align*}
\subsection*{b)}
At first, we can use the quality
\begin{align*}
  \tan^2 x = \sec^2 x - 1
\end{align*}
to simplify the integral:
\begin{align*}
  I &= \int \tan^2 x \sec x\:dx 
  = \int (\sec^2 x - 1)\sec x\:dx \\
  &= \int \sec^3 x\:dx  - \int \sec x\:dx
\end{align*}
We know 
\begin{align}
  \label{eq:1}
  \int \sec x\: dx = \ln |\sec x + \tan x| + C_1.
\end{align}
Let's evaluate the other integral. We use integration by parts with
\begin{align*}
  \begin{aligned}
  u &= \sec x \\
  du &= \sec x \tan x\: dx
  \end{aligned}
  \hspace{1cm}
  \begin{aligned}
  dv &= \sec^2 x\: dx \\
  v &= \tan x
  \end{aligned}
\end{align*}
to get
\begin{align*}
  I'=\int \sec^3 x \: dx&=\sec x\tan x - \int \sec x \tan^2 x\: dx\\
  &= \sec x \tan x - \int \sec x (\sec^2 x - 1)\:dx\\
  &= \sec x \tan x - I' + \int \sec x \: dx.
\end{align*}
Using equation \ref{eq:1}, we can solve for $I'$ to get
\begin{align*}
  I'=\frac{1}{2}(\sec x \tan x + \ln |\sec x + \tan x|) + C_2.
\end{align*}
Thus we conclude that
\begin{align*}
  I=\frac{1}{2}(\sec x \tan x -\ln |\sec x + \tan x|) + C.
\end{align*}
\subsection*{c)}
To solve the integral
\begin{align*}
  I(r)&= \int \frac{r^3}{\sqrt{4+r^2}}dr,
\end{align*}
we use the subsitution
\begin{align*}
  u = 4+r^2 \hspace{1cm}  du = 2rdr.
\end{align*}
Thus we get 
\begin{align*}
  I(r) &= \frac{1}{2}\int \frac{u-4}{\sqrt{u}}\:du\\
  &=\frac{1}{2}\int\sqrt{u}\:du - 2 \int \frac{1}{\sqrt{u}}\:du.
\end{align*}
We can solve those two integrals easily using the power rules:
\begin{align*}
  \int \sqrt{u}\:du&=\frac{2}{3}u^{\nicefrac{3}{2}}+C_1,\\
  \int \frac{1}{\sqrt{u}}\:du &=2\sqrt{u}+C_2.
\end{align*}
Therefore
\begin{align*}
  I(r) = \frac{1}{3}u^{\nicefrac{3}{2}}-4\sqrt{u}+C.
\end{align*}
Substituting $u=r^2+4$ gives
\begin{align*}
  I(r) &= \frac{1}{3}(4+r^2)^{\nicefrac{3}{2}}-4\sqrt{4+r^2} + C\\
  &= \frac{1}{3}(r^2-8)\sqrt{4+r^2}+C.
\end{align*}
Now we can evaluate at $r=0$ and $r=1$:
\begin{align*}
  I(0)&=\frac{1}{3}(0-8)\sqrt{4+0}+C=-\frac{16}{3}+C\\
  I(1)&=\frac{1}{3}(1-8)\sqrt{4+1}+C=-\frac{7\sqrt{5}}{3}+C.
\end{align*}
Therefore we get
\begin{align*}
  \int_0^1\frac{r^3}{\sqrt{4+r^2}}\:dr = \frac{16-7\sqrt{5}}{3} \approx 0.116.
\end{align*}
\subsection*{d)}
To solve
\begin{align*}
  I(x)=\int e^{\sqrt{x}}\:dx
\end{align*}
we can use
\begin{align*}
  u=\sqrt{x}\hspace{1cm}du=\frac{1}{2\sqrt{x}}
\end{align*}
to get
\begin{align*}
  I(x)=2\int ue^u\:du.
\end{align*}
Now we can integrate by parts with
\begin{align*}
  \begin{aligned}
    f &= u\\
    df &= du
  \end{aligned}
  \hspace{1cm}
  \begin{aligned}
    dg &= e^u\: du\\
    g&= e^u
  \end{aligned}
\end{align*}
to get
\begin{align*}
  I(x)=2ue^u-2\int e^udu=2(u-1)e^u+C.
\end{align*}
Undoing the subsitution $u=\sqrt{x}$, we get
\begin{align*}
  \int e^{\sqrt{x}}dx = I(x)=2(\sqrt{x}-1)e^{\sqrt{x}}+C.
\end{align*}
Evaluating at $x=0$ and $x=1$ gives
\begin{align*}
  I(0)=2\cdot(0-1)\cdot 1+C=-2+C,\\
  I(1)=2\cdot(1-1)\cdot e^1+C=0+C.
\end{align*}
Thus
\begin{align*}
  \int_0^1 e^{\sqrt{x}}\: dx = 0 - (-2) = 2.
\end{align*}
\end{document}