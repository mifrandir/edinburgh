\documentclass{article}
\usepackage{homework-preamble}
\title{CAP: Homework 9 (Workshop 71)}
\author{Franz Miltz (UUN: S1971811)}
\begin{document}
\maketitle
\section*{Question 1}
\subsection*{(a)}
You can find the sketch attached to this document.
\subsection*{(b)}
\begin{align*}
	\frac{dy}{dx}=\frac{xy}{\sqrt{1-x}}
\end{align*}
Rearranging gives
\begin{align*}
	\frac{dy}{y}=\frac{x\:dx}{\sqrt{1-x}}.
\end{align*}
Now we can integrate both sides to get
\begin{align*}
	\ln|y| = \int \frac{x}{\sqrt{1-x}}\:dx.
\end{align*}
We can use $u=1-x$ and $du=-dx$ to get
\begin{align*}
	\ln|y| = \int \frac{u-1}{\sqrt{u}}\:du=\int \sqrt{u}\:du - \int \frac{1}{\sqrt{u}}\:du.
\end{align*}
Solving both integrals, we get
\begin{align*}
	\ln|y|=\frac{2}{3}u^{2/3}-2\sqrt{u}+C=2\sqrt{u}\left(\frac{u}{3}-1\right)+C.
\end{align*}
Resubstituting $u$ gives the general solution
\begin{align*}
	\ln|y|=\frac{2}{3}(x+2)\sqrt{1-x}+C.
\end{align*}
This may be rewritten as a function of $y$ in terms of $x$:
\begin{align*}
	y(x)=  ae^{\frac{2}{3}(x+2)\sqrt{1-x}}
\end{align*}
where $a=\pm e^C$.
\subsection*{(c)}
Since $1>0$, we know that we have to find a solution where $y(x)>0$.\\
To satisfy $y(0)=1$, we need the exponent to be $0$, i.e.
\begin{align*}
	0                 & =\frac{2}{3}(0+2)\sqrt{1-0}+C \\
	\Leftrightarrow C & = -\frac{4}{3}.
\end{align*}
Therefore the required particular solution is
\begin{align*}
	y(x)=e^{\frac{2}{3}(x+2)\sqrt{1-x}-\frac{4}{3}}.
\end{align*}
The domain of this function is only limited by the term $\sqrt{1-x}$, which implies \begin{align*}
	\dom y(x) = \{x\in\R\:|\: x\leq 1\}.
\end{align*}
\section*{Question 2}
Let the sequence $\{a_n\}$ be defined by
\begin{align*}
	a_n = \frac{1}{n+1}+\frac{1}{n+2}+\cdots+\frac{1}{2n} \text{ for } n\in\N.
\end{align*}
\begin{claim}
	The sequence $a_n$ is bounded.
\end{claim}
\begin{proof}
	Firstly, note that $a_n>0$ for all $n\in\N$.
	Secondly, since for all $k\in\N$
	\begin{align*}
		\frac{1}{n+k}\leq \frac{1}{n+1},
	\end{align*}
	we know that
	\begin{align*}
		a_n = \sum_{k=1}^n \frac{1}{n+k} \leq \sum_{k=1}^n \frac{1}{n+1}=\frac{n}{n+1}.
	\end{align*}
	Further, we know that for all $n\in\N$
	\begin{align*}
		\frac{n}{n+1} < 1.
	\end{align*}
	Therefore
	\begin{align*}
		a_n \leq \frac{n}{n+1} < 1.
	\end{align*}
	Thus $a_n$ is bounded below by $0$ and above by $1$.
\end{proof}
\begin{claim}
	The sequence $a_n$ is convergent.
\end{claim}
\begin{proof}
	To show this, we need to find the difference between two consecutive terms $a_{n+1}-a_n$.
	Since the first term of the sum is dropped in each step and two more are added to the end, this works out to be
	\begin{align*}
		a_{n+1}-a_n = \frac{1}{2n+1}+\frac{1}{2n+2}-\frac{1}{n+1}=\frac{1}{2n+1}-\frac{1}{2n+2}.
	\end{align*}
	Since
	\begin{align*}
		\frac{1}{2n+1}>\frac{1}{2n+2}
	\end{align*}
	for all $n\in\N$, we know that
	\begin{align*}
		a_{n+1}-a_n > 0
	\end{align*}
	and therefore $a_n$ is strictly increasing.
	Using the \emph{Monotone Convergence Theorem} we know that, since $a_n$ is bounded and strictly increasing, it has to be convergent.
\end{proof}
\noindent We can now calculate the first five terms of the sequence to get
\begin{align*}
	a_1 = \frac{1}{2}       & \approx 0.5,     \\
	a_2 = \frac{7}{12}      & \approx 0.58333, \\
	a_3 = \frac{37}{60}     & \approx 0.61667, \\
	a_4 = \frac{533}{840}   & \approx 0.63452, \\
	a_5 = \frac{1627}{2520} & \approx 0.64563.
\end{align*}
These approximations look like $a_n$ is converging to approximately $0.7$.
Using bigger values for $n$, you can find out that the acutal limit is closer to $0.69315$.
\end{document}