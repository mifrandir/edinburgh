\documentclass{article}
\usepackage{notes-preamble}
\mkthms
\begin{document}
\title{Calculus and its Applications (SEM2)}
\author{Franz Miltz}
\date{May 28, 2020}
\maketitle
\tableofcontents
\pagebreak
\section{Functions and models}
\subsection{Four ways to represent a function}
\begin{definition}
    A \B{function} $f$ is a rule that assigns to each element $x$ in a set $D$ exactly one element, called $f(x)$, in a set $E$.\\
    $D$ is called the \B{domain}.\\
    The set of all possible values of $f(x)$ as $x$ varies throughout the domain is called the \B{range}.\\
\end{definition}
\begin{definition}
    A function $f$ that satisfies $f(-x)=f(x)$ for all $x$ in its domain is called an \B{even function}.
\end{definition}
\begin{definition}
    A function $f$ that satisfies $f(-x)=-f(x)$ for all $x$ in its domain is called an \B{odd function}.
\end{definition}
\begin{definition}
    A function $f$ is called \B{increasing} on an interval $I$ if
    \begin{align*}
        \forall x_1, x_2 \in I.\:x_1 < x_2 \Rightarrow f(x_1) < f(x_2).
    \end{align*}
    It is called \B{decreasing} on $I$ if
    \begin{align*}
        \forall x_1, x_2 \in I.\: x_1 < x_2 \Rightarrow f(x_1) > f(x_2).
    \end{align*}
\end{definition}
\subsection{Mathematical models: Essential functions}
\begin{definition}
    A \B{linear function} is a function $f$ of the form
    \begin{align*}
        f(x) = mx + b
    \end{align*}
    where $m, b\in\R$.
\end{definition}
\begin{definition}
    A function $P$ is called a \B{polynomial} if
    \begin{align*}
        P(x) = a_nx^n+a_{n-1}x^{n-1}+\cdots +a_2x^2 + a_1x+a_0
    \end{align*}
    where $n\in\N$ is the \B{degree} and the $a_0, a_1, ..., a_n$ are constants called the \B{coefficients}.
\end{definition}
\begin{definition}
    A function $f$ is called \B{rational function} if
    \begin{align*}
        f(x) = \frac{P(x)}{Q(x)}
    \end{align*}
    where $P$ and $Q$ are polynomials.
\end{definition}
\begin{definition}
    A function $f$ is called an \B{algebraic function} if it can be constructed using algebraic operations starting with polynomials.
\end{definition}
\begin{definition}
    A function $f$ is called an \B{exponential function} if it has the form
    \begin{align*}
        f(x) = a^x
    \end{align*}
\end{definition}
\begin{definition}
    The \B{logarithmic functions} $f(x) = \log_ax$ where the base $a$ is a positive constant, are the inverse functions of the exponential functions.
\end{definition}
\subsection{New functions from old functions}
\begin{definition}
    Shifting the graph of a function in a certain direction is called \B{translation}.
    \begin{itemize}
        \item $y = f(x) + c$ shifts the graph a distance $c$ upward.
        \item $y= f(x - c)$ shifts the graph a distance $c$ to the right.
    \end{itemize}
\end{definition}
\begin{definition}
    The graph of a function can be \B{stretched}. 
    \begin{itemize}
        \item $y=cf(x)$ stretches the graph vertically by a factor of $c$.
        \item $y=f(x/c)$ stretches the graph horizontally by a factor of $c$.
    \end{itemize}
\end{definition}
\begin{definition}
    The graph of a function can be \B{reflected}.
    \begin{itemize}
        \item $y=-f(x)$ reflects the graph about the $x$-axis.
        \item $y=f(-x)$ reflects the graph about the $y$-axis.
    \end{itemize}
\end{definition}
\begin{definition}
    Functions may be \B{combined}.
    \begin{align*}
        (f+g)(x) &:= f(x) + g(x)\\
        (f-g)(x) &:= f(x) - g(x)\\
        (fg)(x) &:= f(x)g(x)\\
        (\frac{f}{g})(x)&:=\frac{f(x)}{g(x)}\\
        (f\circ g)(x) &:= f(g(x))
    \end{align*}
\end{definition}
\setcounter{subsection}{4}
\subsection{Exponential functions}
\begin{theorem}
    Let $a$ and $b$ be positive numbers and $x$ and $y$ be any real numbers. Then
    \begin{enumerate}
        \item $a^{x+y}=a^xa^y$,
        \item $a^{x-y}=\frac{a^x}{a^y}$,
        \item $(a^x)^y=a^{xy}$,
        \item $(ab)^x=a^xb^x$.
    \end{enumerate}
\end{theorem}
\subsection{Inverse functions and logarithms}
\begin{definition}
    A function $f$ with the domain $D$ is called a \B{one-to-one function} if it never takes on the same value twice; that is
    \begin{align*}
        \forall x_1, x_2 \in D.\: x_1\not=x_2 \Rightarrow f(x_1)\not= f(x_2)
    \end{align*}
\end{definition}
\begin{definition}
    Let $f$ be a one-to-one function with domain $A$ and range $B$. Then its \B{inverse function} $f^{-1}$ has domain $B$ and range $A$ and is defined by
    \begin{align*}
        f^{-1}(y) = x \Leftrightarrow f(x) = y
    \end{align*}
\end{definition}
\begin{theorem}
    The graph of $f^{-1}$ is obtained by reflecting the graph of $f$ about the line $y=x$.
\end{theorem}
\begin{theorem}
    If $x$ and $y$ are positive numbers, then
    \begin{enumerate}
        \item $\log_a(xy)=\log_ax=\log_ay$
        \item $\log_a(\frac{x}{y}) = \log_ax - \log_ay$
        \item $\log_a(x^r)=r\log_ax$ where $r\in\R$
    \end{enumerate}
\end{theorem}
\begin{definition}
    The \B{natural logarithm} $\ln x$ is defined as
    \begin{align*}
        \ln x := \log_e x
    \end{align*}
\end{definition}
\begin{theorem}
    For any positive number $a$ ($a\not=1$), we have
    \begin{align*}
        \log_a x = \frac{\ln x}{\ln a}
    \end{align*}
\end{theorem}
\section{Limits and derivatives}
\setcounter{subsection}{1}
\subsection{The limit of a function}
\begin{definition}
    We write
    \begin{align*}
        \lim_{x\to a} f(x) = L
    \end{align*}
    and say "the limit of $f(x)$, as $x$ approaches $a$, equals $L$" if we can make the values of $f(x)$ arbitrarily close to $L$ by taking $x$ to be sufficiently close to $a$, but not equal to $a$.
\end{definition}
\begin{definition}
    We write
    \begin{align*}
        \lim_{x\to a^-}f(x) = L
    \end{align*}
    and say the \B{left-hand limit of $f(x)$ as x approachse $a$} is equal to $L$ if we can make the values of $f(x)$ arbitrarily close to $L$ by taking $x$ to be sufficiently close to $a$ and $x$ less than $a$. 
\end{definition}
\begin{theorem}
    \begin{align*}
        \lim_{x\to a}f(x) = L \text{ iff } \lim_{x\to a^-}f(x) = L \text{ and } \lim_{x\to a^+}f(x) = L
    \end{align*}
\end{theorem}
\begin{definition}
    Let $f$ be a function defined on both sides of $a$, except possibly at $a$ itself. Then
    \begin{align*}
        \lim_{x\to a} f(x) = \infty
    \end{align*}
    means that the values of $f(x)$ can be made arbitrarily large by taking $x$ sufficiently close to $a$, but not equal to $a$.
\end{definition}
\begin{definition}
    Let $f$ be a function defined on both sides of $a$, except possibly at $a$ itself. Then
    \begin{align*}
        \lim_{x\to a} f(x) = -\infty
    \end{align*}
    means that the values of $f(x)$ can be made arbitrarily large negative by taking $x$ sufficiently close to $a$, but not equal to $a$.
\end{definition}
\begin{definition}
    The line $x=a$ is called a \B{vertical asymptote} of the curve $y=f(x)$ if at least one of the following statements is true:
    \begin{align*}
        \lim_{x\to a}f(x) = \infty\\
        \lim_{x\to a^-}f(x) = \infty\\
        \lim_{x\to a^+}f(x) = \infty\\
        \lim_{x\to a}f(x) = -\infty\\
        \lim_{x\to a^-}f(x) = -\infty\\
        \lim_{x\to a^+}f(x) = -\infty
    \end{align*}
\end{definition}
\subsection{Calculating limits and using the limit laws}
\begin{theorem}
    Suppose that $c$ is a constant and the limits $\lim_{x\to a}f(x)$ and $\lim_{x\to a}g(x)$ exists. Then
    \begin{align*}
        \lim_{x\to a}\left[f(x)+g(x)\right]
        &=\lim_{x\to a}f(x)+\lim_{x\to a}g(x)\\
        \lim_{x\to a}\left[f(x)-g(x)\right]
        &=\lim_{x\to a}f(x)-\lim_{x\to a}g(x)\\
        \lim_{x\to a}\left[cf(x)\right]
        &=c\lim_{x\to a}f(x)\\
        \lim_{x\to a}\left[f(x)g(x)\right]
        &=\lim_{x\to a}f(x)\cdot\lim_{x\to a}g(x)\\
        \lim_{x\to a}\left[\frac{f(x)}{g(x)}\right]
        &=\frac{\lim_{x\to a}f(x)}{\lim_{x\to a}g(x)}\text{ if } \lim_{x\to a}g(x)\not = 0\\
        \lim_{x\to a}\left[f(x)\right]^n
        &=\left[\lim_{x\to a}f(x)\right]^n\\
        \lim_{x\to a}c &= c\\
        \lim_{x \to a}x &= a\\
        \lim_{x \to a}x^n &= a^n \text{ where } n\in\mathbb{Z}^+
    \end{align*}
\end{theorem}
\begin{theorem}
    If $f$ is a polynomial or a rational function and $a$ is in the domain of $f$, then
    \begin{align*}
        \lim_{x\to a}f(x) = f(a)
    \end{align*}
\end{theorem}
\begin{theorem}
    If $f(x) = g(x)$ when $x\not=a$, then $\lim_{x\to a}f(x) = \lim_{x\to a}g(x)$, provided the limits exists.
\end{theorem}
\begin{theorem}
    \begin{align*}
        \lim_{x\to a} f(x) = L\text{ iff } \lim_{x\to a^-}=L=\lim_{x\to a^+} f(x)
    \end{align*}
\end{theorem}
\begin{theorem}
    If $f(x) \leq g(x)$ when $x$ is near a (except possibly at $a$) and the limits of $f$ and $g$ both exists as $x$ approaches $a$, then
    \begin{align*}
        \lim_{x\to a}f(x) \leq \lim_{x\to a} g(x)
    \end{align*}
\end{theorem}
\begin{theorem}
    \B{The squeeze theorem.}\\
    If $f(x)\leq g(x)\leq h(x)$ when $x$ is near $a$ (except possibly at $a$) and
    \begin{align*}
        \lim_{x\to a}f(x) = \lim_{x\to a}h(x) = L
    \end{align*}
    then
    \begin{align*}
        \lim_{x \to a}g(x) = L.
    \end{align*}
\end{theorem}
\subsection{The precise definition of a limit}
\begin{definition}
    Let $f$ be a function defined on some open interval that contains the number $a$, except possibly at $a$ itself. Then we say that the \B{limit of $f(x)$ as $x$ approaches $a$ is $L$}, and we write
    \begin{align*}
        \lim_{x\to a}f(x) = L
    \end{align*}
    if for every number $\varepsilon>0$ there is a number $\delta>0$ such that
    \begin{align*}
        0<|x-a|<\delta \Rightarrow |f(x)-L|<\varepsilon.
    \end{align*}
\end{definition}
\begin{definition}
    \begin{align*}
        \lim_{x\to a^-}f(x) = L
    \end{align*}
    if for every number $\varepsilon > 0$ there is a number $\delta > 0$ such that
    \begin{align*}
        a-\delta < x < a \Rightarrow |f(x)-L| < \varepsilon.
    \end{align*}
\end{definition}
\begin{align*}
    \lim_{x\to a^+}f(x) = L
\end{align*}
    if for every number $\varepsilon > 0$ there is a number $\delta > 0$ such that
\begin{align*}
    a < x < a+\delta \Rightarrow |f(x)-L| < \varepsilon.
\end{align*}
\begin{definition}
    Let $f$ be a function defined on some open interval that contains the number $a$, except possibly at $a$ itself. Then
    \begin{align*}
        \lim_{x\to a}f(x) = \infty
    \end{align*}
    means that for every positive number $M$ there is a positive number $\delta$ such that
    \begin{align*}
        0<|x-a|<\delta \Rightarrow f(x)>M.
    \end{align*}
\end{definition}
\begin{definition}
    Let $f$ be a function defined on some open interval that contains the number $a$, except possibly at $a$ itself. Then
    \begin{align*}
        \lim_{x\to a}f(x) = -\infty
    \end{align*} 
    means that for every negative number $N$ there is a positive number $\delta$ such that
    \begin{align*}
        0<|x-a|<\delta \Rightarrow f(x)<N.
    \end{align*}
\end{definition}
\subsection{Continuity}
\begin{definition}
    A function $f$ is \B{continuous at a number $a$} if
    \begin{align*}
        \lim_{x\to a}f(x)=f(a).
    \end{align*}
\end{definition}
\begin{definition}
    A function $f$ is \B{continuous from the right at a number $a$} if
    \begin{align*}
        \lim_{x\to a^+}f(x) = f(a)
    \end{align*}
    and $f$ is \B{continuous from the left at $a$} if
    \begin{align*}
        \lim_{x\to a^-}f(x) = f(a).
    \end{align*}
\end{definition}
\begin{definition}
    A function $f$ is \B{continuous on an interval $I$} if 
    \begin{align*}
        \forall a \in I, \lim_{x\to a}f(x) = f(a).
    \end{align*}
\end{definition}
\begin{theorem}
    If $f$ and $g$ are continuous at $a$ and $c$ is a constant, then the following functions are also continuous at $a$:
    \begin{enumerate}
        \item $f+g$
        \item $f-g$
        \item $cf$
        \item $fg$
        \item $\frac{f}{g}$ if $g(a)\not= 0$
    \end{enumerate}
\end{theorem}
\begin{theorem}
    \begin{enumerate}
        \item Any polynomial is continuous everywhere.
        \item Any rational function is continuous wherever it is defined.
    \end{enumerate}
\end{theorem}
\begin{theorem}
    The following types of functions are continuous at every number in their domains:
    \begin{itemize}
        \item polynomials,
        \item rational functions,
        \item root functions,
        \item trigonometric functions,
        \item inverse trigonometric functions,
        \item exponential functions,
        \item logarithmic functions.
    \end{itemize}
\end{theorem}
\begin{theorem}
    If $f$ is continuous at $b$ and $\lim_{x\to a}g(x) = b$, then $\lim_{x\to a}f(g(x))=f(b)$. In other words,
    \begin{align*}
        \lim_{x\to a}f(g(x))=f\left(\lim_{x\to a}g(x)\right).
    \end{align*}
\end{theorem}
\begin{theorem}
    If $g$ is continuous at $a$ and $f$ is continuous at $g(a)$, then the composite function $f\circ g$ is continuous at $a$.
\end{theorem}
\begin{theorem}
    \B{The intermediate value theorem}\\
    Suppose that $f$ is continuous on the closed interval $[a,b]$ and let $N$ be any number between $f(a)$ and $f(b)$, where $f(a)\not=f(b)$. Then there exists a number $c$ in $(a,b)$ such that $f(c)=N$.
\end{theorem}
\subsection{Limits at infinity; horizontal asymptotes}
\begin{definition}
    Let $f$ be a function defined on some interval $(a,\infty)$. Then
    \begin{align*}
        \lim_{x\to\infty}f(x)=L
    \end{align*}
    means that the values of $f(x)$ can be made arbitrarily close to $L$ by taking $x$ sufficiently large.
\end{definition}
\begin{definition}
    Let $f$ be a function defined on some interval $(\infty, a)$. Then
    \begin{align*}
        \lim_{x\to -\infty}f(x)=L
    \end{align*}
    means that the values of $f(x)$ can be made arbitrarily close to $L$ by taking $x$ sufficiently large negative.
\end{definition}
\begin{definition}
    The line $y=L$ is called a \B{horizontal asymptote} of the curve $y=f(x)$ if either
    \begin{align*}
        \lim_{x\to \infty}f(x)=L \text{ or } \lim_{x\to-\infty}f(x)=L
    \end{align*}
\end{definition}
\begin{theorem}
    \begin{align*}
        \lim_{x\to-\infty}tan^{-1}x=-\frac{\pi}{2}.\\
        \lim_{x\to\infty}tan^{-1}x=\frac{\pi}{2}.
    \end{align*}
\end{theorem}
\begin{theorem}
    If $r>0$ is a rational number, then
    \begin{align*}
        \lim_{x\to\infty}\frac{1}{x^r}=0
    \end{align*}
    If $r>0$ is a rational number such that $x^r$ is defined for all $x$, then
    \begin{align*}
        \lim_{x\to-\infty}\frac{1}{x^r}=0.
    \end{align*}
\end{theorem}
\begin{theorem}
    \begin{align*}
        \lim_{x\to-\infty}e^x=0.
    \end{align*}
\end{theorem}
\begin{definition}
    Let $f$ be a function defined on some interval $(a, \infty)$. Then
    \begin{align*}
        \lim_{x\to\infty}f(x)=L
    \end{align*}
    means that for every $\varepsilon>0$ there is a corresponding number $N$ such that
    \begin{align*}
        x>N\Rightarrow|f(x)-L|<\varepsilon.
    \end{align*}
\end{definition}
\begin{definition}
    Let $f$ be a function defined on some interval $(-\infty, a)$. Then
    \begin{align*}
        \lim_{x\to-\infty}f(x)=L
    \end{align*}
    means that for every $\varepsilon>0$ there is a corresponding number $N$ such that
    \begin{align*}
        x<N\Rightarrow |f(x)-L|<\varepsilon.
    \end{align*}
\end{definition}
\begin{definition}
    Let $f$ be a function defined on some interval $(a, \infty)$. Then
    \begin{align*}
        \lim_{x\to\infty}f(x)=\infty
    \end{align*}
    means that for every positive number $M$ there is a corresponding positive number $N$ such that
    \begin{align*}
        x>N\Rightarrow f(x)>M.
    \end{align*}
\end{definition}
\subsection{Derivatives and rates of change}
\begin{definition}
    The \B{tangent line} to the curve $y=f(x)$ at the point $P(a, f(a))$ is the line through $P$ with slope
    \begin{align*}
        m=\lim_{x\to a}\frac{f(x)-f(a)}{x-a}
    \end{align*}
    provided that this limit exists.
\end{definition}
\begin{theorem}
    \begin{align*}
        m=\lim_{h\to 0}\frac{f(a + h)-f(a)}{h}
    \end{align*}
\end{theorem}
\begin{theorem}
    \begin{align*}
        v(a)=\lim_{h\to 0}\frac{f(a+h)-f(a)}{h}
    \end{align*}
\end{theorem}
\begin{definition}
    The \B{derivative of a function $f$ at a number $a$}, denoted $f'(a)$, is
    \begin{align*}
        f'(a)=\lim_{h\to 0}\frac{f(a+h)-f(a)}{h}
    \end{align*}
    if this limit exists.
\end{definition}
\begin{theorem}
    \begin{align*}
        f'(a)=\lim_{x\to a}\frac{f(x)-f(a)}{x-a}
    \end{align*}
\end{theorem}
\begin{definition}
    The \B{derivative of $f$}, $f'$, is defined by
    \begin{align*}
        f'(x)=\lim_{h\to 0}\frac{f(x+h)-f(x)}{h}
    \end{align*}
\end{definition}
\begin{definition}
    A function $f$ is \B{differentiable at $a$} if $f'(a)$ exists. It is \B{differentiable on an open interval $(a,b)$} if it is differentiable at every number in the interval.
\end{definition}
\begin{theorem}
    If $f$ is differentiable at $a$, then $f$ is continuous at $a$.
\end{theorem}
\section{Differentiation rules}
\subsection{Derivatives of polynomials and exponential functions}
\begin{theorem}
    \begin{align*}
        \frac{d}{dx}(c) = 0
    \end{align*}
\end{theorem}
\begin{theorem}
    \begin{align*}
        \frac{d}{dx}(x) = 1
    \end{align*}
\end{theorem}
\begin{theorem}
    If $n$ is any real number, then 
    \begin{align*}
        \frac{d}{dx}(x^n) = nx^{n-1}.
    \end{align*}
\end{theorem}
\begin{theorem}
    If $c$ is a constant and $f$ is a differentiable function, then
    \begin{align*}
        \frac{d}{dx}(cf(x))=c\frac{d}{dx}f(x).
    \end{align*}
\end{theorem}
\begin{theorem}
    If $f$ and $g$ are both differentiable, then
    \begin{align*}
        \frac{d}{dx}(f(x)+g(x))=\frac{d}{dx}f(x)+\frac{d}{dx}g(x).
    \end{align*}
\end{theorem}
\begin{theorem}
    If $f$ and $g$ are both differentiable, then
    \begin{align*}
        \frac{d}{dx}(f(x)-g(x))=\frac{d}{dx}f(x)-\frac{d}{dx}g(x).
    \end{align*}
\end{theorem}
\begin{theorem}
    \begin{align*}
        \frac{d}{dx}(e^x)=e^x
    \end{align*}
\end{theorem}
\subsection{Product and quotient rules}
\begin{theorem}
    If $f$ and $g$ are both differentiable, then
    \begin{align*}
        \frac{d}{dx}(f(x)g(x))=f(x)\frac{d}{dx}(g(x))+g(x)\frac{d}{dx}(f(x))
    \end{align*}
\end{theorem}
\begin{theorem}
    If $f$ and $g$ are differentiable, then
    \begin{align*}
        \frac{d}{dx}\left(\frac{f(x)}{g(x)}\right)=\frac{g(x)\frac{d}{dx}(f(x))-f(x)\frac{d}{dx}(g(x))}{(g(x))^2}
    \end{align*}
\end{theorem}
\subsection{Derivatives of trigonometric functions}
\begin{theorem}
    \begin{align*}
        \frac{d}{dx}(\sin x)=\cos x
    \end{align*}
\end{theorem}
\begin{theorem}
    \begin{align*}
        \frac{d}{dx}(\cos x)=-\sin x
    \end{align*}
\end{theorem}
\begin{theorem}
    \begin{align*}
        \frac{d}{dx}(\tan x)=\frac{1}{\cos^2 x}
    \end{align*}
\end{theorem}
\subsection{The chain rule}
\begin{theorem}
    If $g$ is differentiable at $x$ and $f$ is differentiable at $g(x)$, then the composite function $F=f\circ g$ is differentiable at $x$ and $F'$ is given by the product
    \begin{align*}
        F'(x)=f'(g(x))\cdot g'(x).
    \end{align*}
    In Leibniz notation, if $y=f(u)$ and $u=g(x)$ are both differentiable functions, then
    \begin{align*}
        \frac{dy}{dx}=\frac{dy}{du}\frac{du}{dx}
    \end{align*}
\end{theorem}
\begin{theorem}
    \begin{align*}
        \frac{d}{dx}(a^x)=a^x\ln a
    \end{align*}
\end{theorem}
\subsection{Implicit differentiation}
\begin{theorem}
    \B{Derivatives of inverse trigonometric functions}\\
    \begin{align*}
        \frac{d}{dx}(\sin^{-1}x)&=\frac{1}{\sqrt{1-x^2}}\\
        \frac{d}{dx}(\cos^{-1}x)&=-\frac{1}{\sqrt{1-x^2}}\\
        \frac{d}{dx}(\tan^{-1}x)&=\frac{1}{1+x^2}
    \end{align*}
\end{theorem}
\begin{theorem}
    The only solutions to the differential equation $\nicefrac{dy}{dt}=ky$ are the exponential functions
    \begin{align*}
        y(t)=y(0)e^{kt}.
    \end{align*}
\end{theorem}
\setcounter{subsection}{9}
\subsection{Linear approximations and differentials}
\begin{theorem}
    \begin{align*}
        f(x)\approx f(a)+f'(a)(x-a)
    \end{align*}
    is called the \B{linear approximation} or \B{tangent line approximation} of $f$ at $a$.
\end{theorem}
\subsection{Hyperbolic functions}
\begin{definition}
    \begin{align*}
        \sinh x &:= \frac{e^x-e^{-x}}{2}\\
        \cosh x &:= \frac{e^x+e^{-x}}{2}\\
        \tanh x &:= \frac{\sinh x}{\cosh x}
    \end{align*}
\end{definition}
\begin{theorem}
    \B{Hyperbolic identities}
    \begin{align*}
        \sinh(-x) &= -\sinh x\\
        \cosh(-x) &= \cosh x\\
        \cosh^2x - \sinh^2x &= 1\\
        1-\tanh^2 x &= \nicefrac{1}{\cosh^2x}\\
        \sinh(x+y) &=\sinh x \cosh x + \cosh x \sinh y\\
        \cosh(x+y) &= \cosh x \cosh y + \sinh x \sinh y 
    \end{align*}
\end{theorem}
\begin{theorem}
    \B{Derivatives of hyperbolic functions and their inverses}
    \begin{align*}
        \begin{aligned}
            \frac{d}{dx}(\sinh x)&=\cosh x\\
            \frac{d}{dx}(\csch x)&=-\csch x \coth x\\
            \frac{d}{dx}(\cosh x)&=\sinh x\\
            \frac{d}{dx}(\sech x)&= -\tanh x \sech x\\
            \frac{d}{dx}(\tanh x)&=\sech^2 x\\
            \frac{d}{dx}(\coth x)&= -\csch^2 x
        \end{aligned}
        \hspace{1cm}
        \begin{aligned}
            \frac{d}{dx}(\arcsinh x)&=\frac{1}{\sqrt{1+x^2}}\\
            \frac{d}{dx}(\arccsch x)&=-\frac{1}{|x|\sqrt{x^2+1}}\\
            \frac{d}{dx}(\arccosh x)&=\frac{1}{\sqrt{x^2-1}}\\
            \frac{d}{dx}(\arcsech x)&= -\frac{1}{x\sqrt{1-x^2}}\\
            \frac{d}{dx}(\arctanh x)&=\frac{1}{1-x^2}\\
            \frac{d}{dx}(\arccoth x)&= \frac{1}{1-x^2}
        \end{aligned}
    \end{align*}
\end{theorem}
\section{Applications of Differentiation}
\subsection{Maximum and minimum values}
\begin{definition}
    A function $f$ has an \B{absolute maximum} (or \B{global maximum}) at $c$ if
    \begin{align*}
        \forall x\in \dom f, f(x)\geq f(x).
    \end{align*} 
    Similarly, $f$ has an \B{absolute minimum} at $c$ if
    \begin{align*}
        \forall x\in \dom f, f(x)\leq f(x).
    \end{align*}
\end{definition}
\begin{definition}
    A function $f$ has a \B{local maximum} at $c$ if
    \begin{align*}
        \exists a,b\in \dom f \st \forall x\in (a,b), f(c)\geq f(x). 
    \end{align*}
    Similarly, $f$ has a \B{local minimum} at $c$ if
    \begin{align*}
        \exists a,b\in \dom f\st \forall x\in(a,b), f(c)\geq f(x).
    \end{align*}
\end{definition}
\begin{theorem}
    \B{The extreme value theorem}\\
    If $f$ is continuous on a close interval $[a,b]$, then $f$ attains an absolute maximum value $f(c)$ and an absolute minimum value $f(d)$ at some numbers $c,d\in[a,b]$.
\end{theorem}
\begin{theorem}
    \B{Fermat's theorem}\\
    If $f$ has a local maximum or minimum at $c$, and if $f'(c)$ exists, then $f'(c)=0$.
\end{theorem}
\begin{definition}
    A \B{critical number} of a function $f$ is a number $c$ in the domain of $f$ such that either $f'(c)=0$ or $f'(c)$ does not exists.
\end{definition}
\subsection{The mean value theorem}
\begin{theorem}
    \B{Rolle's theorem}\\
    Let $f$ be a function that satisfies the following three hypotheses:
    \begin{enumerate}
        \item $f$ is continuous on the closed interval $[a,b]$.
        \item $f$ is differentiable on the open interval $(a,b)$.
        \item $f(a)=f(b)$.
    \end{enumerate}
    Then there is a number $c \in (a,b)$ such that $f'(c)=0$.
\end{theorem}
\begin{theorem}
    \B{The mean value theorem}\\
    Let $f$ be a function that satisfies the following hypotheses:
    \begin{enumerate}
        \item $f$ is continuous on the closed interval $[a,b]$.
        \item $f$ is differentiable on the open interval $(a,b)$.
    \end{enumerate}
    Then there is a number $c\in(a,b)$ such that
    \begin{align*}
        f'(c)=\frac{f(b)-f(a)}{b-a}
    \end{align*}
\end{theorem}
\begin{theorem}
    \begin{align*}
        \text{If }
        \forall x\in(a,b), f'(x)=0, 
        \text{ then } 
        \exists y\in\R\st\forall c\in(a,b), f(c)=y.
    \end{align*}
\end{theorem}
\begin{theorem}
    \begin{align*}
        \text{If }\forall x\in(a,b), f'(x)=g'(x)\text{, then } \exists c\in\R \st\forall x\in (a,b), f(x)=g(x)+c.
    \end{align*}
\end{theorem}
\subsection{How derivatives affect the shape of a graph}
\begin{definition}
    If the graph of $f$ lies above all of its tangents on an interval $I$, then it is called \B{concave upward} on $I$. If the graph of $f$ lies below all of its tangents, then it is called \B{concave downward} on $I$.
\end{definition}
\begin{theorem}
    \B{Concavity test}
    \begin{enumerate}
        \item If $\forall x\in I, f''(x)>0$, then the graph of $f$ is concave upward on $I$.
        \item If $\forall x\in I, f''(x)<0$, then the graph of $f$ is concave downward on $I$.
    \end{enumerate}
\end{theorem}
\begin{definition}
    A point $P$ on a curve $y=f(x)$ is called an \B{inflection point} if $f$ is continuous there and the curve changes its concavity.
\end{definition}
\begin{theorem}
    Suppose $f''$ is continuous near $c$.
    \begin{enumerate}
        \item If $f'(c)=0$ and $f''(c)>0$, then $f$ has a local minimum at $c$.
        \item If $f'(c)=0$ and $f''(c)<0$, then $f$ has a local maximum at $c$.
    \end{enumerate}
\end{theorem}
\subsection{Indeterminate forms and l'Hospital's rule}
\begin{theorem}
    \B{L'Hosptial's rule}\\
    Suppose $f$ and $g$ are differentiable and $g'(x)\not=0$ on an open interval $I$ that contains $a$ (except possibly at $a$). Suppose that
    \begin{align*}
        \lim_{x\to a}f(x)=0\text{ and } \lim_{x\to a}g(x)=0
    \end{align*}
    or that
    \begin{align*}
        \lim_{x\to a}f(x)=\pm\infty \text{ and } \lim_{x\to a}g(x) = \pm\infty.
    \end{align*}
    Then
    \begin{align*}
        \lim_{x\to a}\frac{f(x)}{g(x)}=\lim_{x\to a}\frac{f'(x)}{g'(x)}
    \end{align*}
    if the limit on the right side exists (or is $\pm\infty$).
\end{theorem}
\setcounter{subsection}{6}
\subsection{Antiderivatives}
\begin{definition}
    A function $F$ is called an \B{antiderivative} of $f$ on an interval $I$ if $F'(x)=f(x)$ for all $x$ in $I$.
\end{definition}
\begin{theorem}
    If $F$ is an antiderivative of $f$ on an interval $I$, then the most general antiderivative of $f$ on $I$ is
    \begin{align*}
        F(x) + C
    \end{align*}
    where $C$ is an arbitrary constant.
\end{theorem}
\begin{theorem}
    Particular antiderivatives of common functions:
    \begin{itemize}
        \item $\int cf(x)dx = c\int f(x)dx$
        \item $\int f(x) + g(x) dx = \int f(x)dx + \int g(x)dx$
        \item $\int x^n dx=\frac{x^{n+1}}{n+1}+C$
        \item $\int \frac{1}{x}dx=\ln |x|+C$
        \item $\int e^xdx=e^x+C$
        \item $\int \cos x dx = \sin x+C$
        \item $\int \sin x dx = -\cos x +C $
        \item $\int \sec^2 x dx = \tan x + C$
        \item $\int \sec x \tan x dx = \sec x + C$
        \item $\int \frac{1}{\sqrt{1-x^2}}dx = \sin^{-1} x + C$
        \item $\int \frac{1}{1+x^2}dx = \arctan x + C$
        \item $\int \cosh x dx = \sinh x + C$
        \item $\int \sinh x dx = \cosh x + C$
    \end{itemize}
\end{theorem}
\section{Integrals}
\subsection{Areas and Distances}
\begin{definition}
  The \B{area} $A$ of the region $S$ that lies under the graph of the continuous function $f$ is the limit of the sum of the areas of appoximating rectangles:
  \begin{align*}
      A=\lim_{n\to\infty}R_n = \lim_{n\to \infty}\left(f(x_1)\Delta x + f(x_2)\Delta x + \cdots + f(x_n)\delta x\right)
  \end{align*}
\end{definition}
\subsection{The definite integral}
\begin{definition}
    If $f$ is a function defined on $[a,b]$, the \B{definite integral of $f$ from $a$ to $b$} is the number
    \begin{align*}
        \int_a^b f(x)dx=\lim_{\max \Delta x_i\to0}\sum_{i=1}^n f(x_i^*)\Delta x_i
    \end{align*}
    provided that this limit exists. If it does exist, we say that $f$ is \B{integrable} on $[a,b]$.
\end{definition}
\begin{theorem}
    If $f$ is continuous on $[a,b]$, or if $f$ has only a finite number of jump discontinuities, then $f$ is integrable on $[a,b]$; that is, the definite integral $\int_a^b f(x)dx$ exists.
\end{theorem}
\begin{theorem}
    If $f$ is integrable on $[a,b]$, then \begin{align*}
        \int_a^b f(x)dx = \lim_{n\to\infty}\sum_{i=1}^nf(x_i)\Delta x
    \end{align*}
    where \begin{align*}
    \Delta x = \frac{b-a}{n} \text{ and } x_i=a+i\Delta x
    \end{align*}
\end{theorem}
\begin{theorem}
    \B{The midpoint rule}.
    \begin{align*}
        \int_a^bf(x)dx \approx \sum_{i=1}^nf(\bar{x_i})\Delta x = \Delta x (f(\bar{x_i})+\cdots+f(\bar{x_n}))
    \end{align*}
    where
    \begin{align*}
        \Delta x = \frac{b-a}{n},\\
        \bar{x_i} = \frac{1}{2}(x_{i-1}+x_i).
    \end{align*}
\end{theorem}
\begin{theorem}
    \B{Properties of integrals}.\\
    Suppose all the following integrals exist and let $c$ be any constant.
    \begin{enumerate}
        \item $\int_a^b cdx = c(b-a)$
        \item $\int_a^b(f(x)+g(x))dx=\int_a^b+\int_a^bg(x)dx$
        \item $\int_a^b(cf(x))dx=c\int_a^bf(x)dx$
        \item $\int_a^b(f(x)-g(x))dx=\int_a^bf(x)dx-\int_a^bg(x)dx$
        \item $\int_a^cf(x)dx+\int_c^bf(x)dx = \int_a^bf(x)dx$
        \item If $f(x)\geq 0$ for $x\in[a,b]$, then $\int_a^b f(x)dx \geq 0$.
        \item If $f(x)\geq g(x)$ for $x\in[a,b]$, then $\int_a^b f(x)dx\geq \int_a^b g(x)dx$.
        \item If $f(x)\in[m,M]$ for $x\in[a,b]$, then 
        \begin{align*}
            m(b-a)\leq \int_a^b f(x)dx \leq M(b-a)
        \end{align*}
    \end{enumerate}
\end{theorem}
\subsection{Evaluating definite integrals}
\begin{theorem}
    \B{Net change theorem}. The integral of a rate of change is the net change:
    \begin{align*}
        \int_a^bF'(x)dx=F(b)-F(a)
    \end{align*}
\end{theorem}
\subsection{The fundamental theorem of calculus}
\begin{theorem}
   Suppose $f$ is continuous on $[a,b]$.
   \begin{enumerate}
       \item If $g(x)=\int_a^x f(t)dt$, then $g'(x)=f(x)$.
       \item $\int_a^b f(x)dx=F(b)-F(a)$, where $F$ is any antiderivative of $f$, that is $F'=f$.
   \end{enumerate}
\end{theorem}
\begin{definition}
    The \B{average value of $f$} on the interval $[a,b]$ is
    \begin{align*}
        f_{ave}=\frac{1}{b-a}\int_a^b f(x)dx.
    \end{align*}
\end{definition}
\begin{theorem}
    \B{The mean value theorem for integrals.} If $f$ is continuous on $[a,b]$, then there exists a number $c$ in $[a,b]$ such that
    \begin{align*}
        f(c)=f_{ave}=\frac{1}{b-a}\int_a^b f(x)dx
    \end{align*}
    that is,
    \begin{align*}
        \int_a^b f(x)dx=f(x)(b-a).
    \end{align*}
\end{theorem}
\subsection{The substitution rule}
\begin{theorem}
    \B{The substitution rule.}
    If $u=g(x)$ is a differetiable function whose range is an interval $I$ and $f$ is continuous on $I$, then
    \begin{align*}
        \int f(g(x))g'(x)\:dx=\int f(u)\:du
    \end{align*}
\end{theorem}
\begin{theorem}
    \B{The substitution rule for definite integrals.}
    If $g'$ is continuous on $[a,b]$ and $f$ is continuous on the range of $u=g(x)$, then 
    \begin{align*}
        \int_a^b f(g(x))g'(x)\:dx=\int_{g(a)}^{g(b)}f(u)\:du.
    \end{align*}
\end{theorem}
\begin{theorem}
    \B{Integrals of symmetric functions.} Suppose $f$ is continuous on $[-a,a]$.
    \begin{enumerate}
        \item If $f$ is even, then $\int_{-a}^a f(x)\:dx= 2\int_0^a f(x)\:dx$.
        \item If $f$ is odd, then $\int_{-a}^a f(x)\:dx = 0$.
    \end{enumerate}
\end{theorem}
\section{Techniques of integration}
\begin{theorem}
    \begin{align*}
        \int u dv = uv - \int v du
    \end{align*}
\end{theorem}
\begin{definition}
    \B{of an improper integral of type 1.}
    \begin{enumerate}
        \item If $\int_a^t f(x)\:dx$ exists for every number $t\geq a$, then
        \begin{align*}
            \int_a^\infty f(x)\:dx = \lim_{t\to \infty}\int_a^t f(x)\:dx.
        \end{align*}
        provided this limit exists (as a finite number).
        \item If $\int_t^b f(x)\:dx$ exists for every $t\leq b$, then
        \begin{align*}
            \int_{-\infty}^b f(x)\: dx = \lim_{t\to-\infty}\int_t^b f(x)\:dx
        \end{align*}
        provided this limit exists (as a finite number).
    \end{enumerate}
    The improper integrals $\int_a^\infty f(x)\:dx$ and $\int_{-\infty}^b f(x)\:dx$ are called \B{convergent} if the corresponding limit exists and \B{divergent} otherwise.\\
    If both  $\int_a^\infty f(x)\:dx$ and $\int_{-\infty}^a f(x)\:dx$ are convergent, then we define
    \begin{align*}
        \int_{-\infty}^\infty f(x)\:dx = \int_a^\infty f(x)\:dx + \int_{-\infty}^a f(x)\:dx
    \end{align*}
\end{definition}
\begin{definition}
    \B{of an improper integral of type 2.}
    \begin{enumerate}
        \item If $f$ is continuous on $[a,b)$ and discontinuous at $b$, then
        \begin{align*}
            \int_a^b f(x)\:dx = \lim_{t\to b^-}\int_a^t f(x)\: dx
        \end{align*}
        if this limit exists.
        \item If $f$ is continuous on $(a,b]$ and discontinuous at $a$, then
        \begin{align*}
            \int_a^b f(x)\:dx = \lim_{t\to a^+}\int_t^b f(x)\: dx
        \end{align*}
        if this limit exists.
    \end{enumerate}
    The improer integral $\int_a^b f(x)\:dx$ is called \B{convergent} if the corresponding limit exists and \B{divergent} otherwise.\\
    If $f$ has a discontinuity at $c$, where $a<c<b$, and both $\int_a^c f(x)\:dx$ and $\int_c^b f(x)\:dx$ are convergent, then we define
    \begin{align*}
        \int_a^b f(x)\: dx = \int_a^c f(x)\:dx + \int_c^b f(x)\:dx.
    \end{align*}
\end{definition}
\begin{theorem}
    \B{The comparison theorem.} Suppose that $f$ and $g$ are continuous functions with $f(x)\geq g(x)\geq 0$ for $x\geq a$.
    \begin{enumerate}
        \item If $\int_a^\infty f(x)\:dx$ is convergent, then $\int_a^\infty g(x)\:dx$ is convergent.
        \item If $\int_a^\infty g(x)\:dx$ is divergent, then $\int_a^\infty f(x)\:dx$ is divergent.
    \end{enumerate}
\end{theorem}
\section{Applications}
\begin{definition}
    The area $A$ of the region bounded by the curves $y=f(x)$, $y=g(x)$, and the lines $x=a$, $x=b$, where $f$ and $g$ are continuous and $f(x)\geq g(x)$ for all $x\in[a,b]$, is
    \begin{align*}
        A=\int_a^b \left(f(x)-g(x)\right)\:dx.
    \end{align*}
\end{definition}
\begin{definition}
    Let $S$ be a solid that lies between $x=a$ and $x=b$. If the cross-sectional area of $S$ in the plane $P_x$, through $x$ and perpendicular to the $x$-axis, is $A(x)$, where $A$ is an integrable function, then the \B{volume} of $S$ is
    \begin{align*}
        V=\int_a^b A(x)\:dx.
    \end{align*}
\end{definition}
\begin{theorem}
    \B{The arc length formula.} If $f'$ is continuous on $[a,b]$, then the length of the curve $y=f(x)$, $a\leq x \leq b$, is
    \begin{align*}
        L = \int_a^b \sqrt{1+\left(\frac{df}{dx}\right)^2}\:dx.
    \end{align*}
\end{theorem}
\begin{theorem}
    \B{of Pappus.} Let $R$ be a plane region that lies entirely on one side of a line $l$ in the plane. 
    If $R$ is rotated about $l$, then the volume of the resulting solid is the product of the area $A$ of $R$ and the distance $d$ traveled by the centroid of $R$.
\end{theorem}
\end{document}