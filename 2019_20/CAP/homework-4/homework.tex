\documentclass{article}
\usepackage{homework-preamble}

\title{CAP: Homework 4 (Workshop 71)}
\author{Franz Miltz (UUN: S1971811)}
\begin{document}
\maketitle
\section*{Question 1}
Let
\begin{align*}
  f(x) = \arctan\left(\frac{x-1}{x+1}\right)
\end{align*}
and
\begin{align*}
  g(x) = \frac{x-1}{x+1}.
\end{align*}
Then, to find the domain and range of $f$, we need to find them for $g$ and $\arctan$:
\begin{align*}
  \begin{aligned}
    \dom \arctan(x) & = \mathbb{R}                  \\
    \dom g(x)       & = \mathbb{R} \setminus \{-1\}
  \end{aligned}
  \hspace{1cm}
  \begin{aligned}
    \range \arctan(x) & = \left(-\frac{\pi}{2}, \frac{\pi}{2}\right) \\
    \range g(x)       & = \mathbb{R}\setminus\{1\}
  \end{aligned}
\end{align*}
Since arctan is defined on $\mathbb{R}$, $\dom f(x) = \dom g(x)$. The range of $f$ can be easily determined:
\begin{align*}
  \range f(x) = \range g(x) \cap \range \arctan(x) = \left(-\frac{\pi}{2}, \frac{\pi}{2}\right) \setminus \{1\}.
\end{align*}
Now, let's find asymptotes. First we need to look at the behaviour of $g$ at infinity and its undefined point.
\begin{align*}
  \begin{aligned}
    \lim_{x\to\infty}g(x)  & = 1 \\
    \lim_{x\to-\infty}g(x) & = 1
  \end{aligned}
  \hspace{1cm}
  \begin{aligned}
    \lim_{x\to-1^-}g(x) & =\infty  \\
    \lim_{x\to-1^+}g(x) & =-\infty
  \end{aligned}
\end{align*}
Observe as well that $g(x)<1$ as $x\to\infty$ and $g(x)>1$ as $x\to-\infty$. Now we can apply this information to $f$.
\begin{align*}
  \begin{aligned}
    \lim_{x\to\infty}f(x)  & = \lim_{x\to 1}\arctan(x)=\frac{\pi}{4} \\
    \lim_{x\to-\infty}f(x) & = \lim_{x\to 1}\arctan(x)=\frac{\pi}{4}
  \end{aligned}
  \hspace{1cm}
  \begin{aligned}
    \lim_{x\to-1^-}f(x) & = \lim_{x\to\infty}\arctan(x)= \frac{\pi}{2} \\
    \lim_{x\to-1^+}f(x) & = \lim_{x\to\infty}\arctan(x)=-\frac{\pi}{2}
  \end{aligned}
\end{align*}
Now, due to our observation earlier and the fact that $\arctan(x)$ is strictly increasing, we can say that $f(x)<\frac{\pi}{4}$ as $x\to\infty$ and $g(x)>\frac{\pi}{4}$ as $x\to -\infty$.\\
The last thing we need to find before we deal with the derivative, are the intercepts. Luckily, $\arctan(x)=0$ iff $x=0$. Thus, we need to find the roots of $g(x)$ to find the roots of $f(x)$.
\begin{align*}
  g(x)=\frac{x-1}{x+1} & =0 \\
  \Leftrightarrow x-1 = 0   \\
  \Leftrightarrow x = 1.
\end{align*}
So $f(x)=0$ iff $x=1$. The $y$-intercept is $\arctan(-1)=-\frac{\pi}{4}$.\\\\
Now, let's find the first derivative:
\begin{align*}
  \frac{d}{dx}f(x) & =\frac{d}{dg}\arctan(g)\frac{d}{dx}g(x)                      \\
                   & =\frac{1}{g^2(x)+1}\left(\frac{(x+1)-(x-1)}{(x+1)^2}\right)  \\
                   & =\frac{1}{\left(\frac{x-1}{x+1}\right)^2+1}\frac{2}{(x+1)^2} \\
                   & =\frac{2}{(x-1)^2+(x+1)^2} = \frac{2}{2x^2+2}                \\
                   & =\frac{1}{x^2+1}
\end{align*}
Usually we would now do the same things as we did for $f$, just in regards to the derivative. Luckily, we are very observant and notice that
\begin{align*}
  \frac{d}{dx}f(x)=\frac{1}{x^2+1}=\frac{d}{dx}\arctan(x).
\end{align*}
Since we know the general shape of the $\arctan(x)$ function (strictly increasing, inflection point at $0$), we have all the information we need to draw $f(x)$. The sketch is on the attached sheet.
\section*{Question 2}
Let's model the distance of each boat as functions:
\begin{align*}
  s_1(t) & =20             \\
  s_2(t) & =15-15t=15(1-t)
\end{align*}
where the distance $s$ is in kilometers and the time $t$ is in hours after 2 pm. Now we can use the Pythagorean theorem to get the distance $d$ of the two boats at any time $t$:
\begin{align*}
  s(t) & =\sqrt{s_1^2(t)+s_2^2(t)}=5\sqrt{9-18t+25t^2}
\end{align*}
To minimize this function, we need to find the derivative:
\begin{align*}
  \frac{ds}{dt}=5\frac{50t-18}{2\sqrt{25t^2-18t+9}}=\frac{5(25t-9)}{\sqrt{25t^2-18t+9}}.
\end{align*}
Thus
\begin{align*}
  \frac{ds}{dt}            & =0             \\
  \Leftrightarrow 5(25t-9) & =0             \\
  \Leftrightarrow t        & =\frac{9}{25}.
\end{align*}
Now we need to find the second derivative to check that this is actually a minimum. We get
\begin{align*}
  \frac{d^2 s}{dt^2}=\frac{d}{dt}\frac{5(25t-9)}{\sqrt{25t^2-18t+9}}=\frac{720}{(25t^2-18t+9)^{\frac{3}{2}}}.
\end{align*}
Setting $t=\frac{9}{25}$ yields $\frac{625}{12}$ which is greater than $0$ and thus $s$ has a minimum at $t=\frac{9}{25}$. It follows that the boats are closest $\frac{9}{25}\cdot 3600=1296$ seconds after 2 pm (2:21:36 pm).
\end{document}