\documentclass{article}
\usepackage{homework-preamble}

\title{CAP: Homework 5 (Workshop 71)}
\author{Franz Miltz (UUN: S1971811)}

\begin{document}
\maketitle
\section*{Question 1}
We can estimate of $Q(6)$ by using Riemann sums on the left and the right.
At first, we can use the values at the left end of each intervall to get a lower estimate:
\begin{align*}
	Q(6) \approx \sum_{t=0}^5 r(t)=2 + 10 + 24 +36 + 46+54=172.
\end{align*}
Note that since $\Delta t$ is $1$ for each intervall, we don't need to multiply the sum of $r$'s.
Also observe that we can not say for sure that this is a lower \emph{bound}, even though our intuition tells us that it is likely to be since $r(t)$ might have unforeseen jumps that are not reflected in the given values.
Similarly, we can use the values on the right to get an upper estimate
\begin{align*}
	Q(6) \approx \sum_{t=1}^6 r(t)=10+24+36+46+54+60=230.
\end{align*}
Using the midpoint rule we can get an even more realistic approximation.
\begin{align*}
	Q(6)\approx \sum_{t=0}^5 \frac{r(t)+r(t+1)}{2}=6+17+30+41+50+57=201.
\end{align*}
Notice that
\begin{align*}
	\frac{172+230}{2}=201.
\end{align*}
This is due to the fact that
\begin{align*}
	\sum_{t=0}^5\frac{r(t)+r(t+1)}{2}
	 & =\frac{1}{2}\left(\sum_{t=0}^5 r(t)+\sum_{t=0}^5 r(t+1)\right) \\
	 & =\frac{1}{2}\left(\sum_{t=0}^5 r(t)+\sum_{t=1}^6 r(t)\right)   \\
	 & =\frac{172+230}{2}=201.
\end{align*}
\section*{Question 2}
\begin{align*}
	6+\int_a^x\frac{f(t)}{t^2}dt=2\sqrt{x}
\end{align*}
At first we will find the derivative of the right-hand side to find out what $f$ sholud be.
\begin{align*}
	\frac{d}{dx}2\sqrt{x}=\frac{2}{2\sqrt{x}}=\frac{1}{\sqrt{x}}.
\end{align*}
Thus, a wise choice would be to let
\begin{align*}
	\frac{f(x)}{x^2}=\frac{1}{\sqrt{x}},
\end{align*}
which is the case if $f(x)=x^{\frac{3}{2}}$. We can now solve the definite integral on the left-hand side:
\begin{align*}
	\int_a^x \frac{f(t)}{t^2}dt = \int_a^x \frac{1}{\sqrt{t}} = 2\sqrt{x}-2\sqrt{a}.
\end{align*}
Inserting this into the initial equation we get
\begin{align*}
	6+2\sqrt{x}-2\sqrt{a}     & =2\sqrt{x} \\
	\Leftrightarrow 2\sqrt{a} & =6         \\
	\Leftrightarrow a         & =9.
\end{align*}
Thus the given equation holds if $f(x)=x^\frac{3}{2}$ and $a=9$.
\end{document}