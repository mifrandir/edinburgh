\documentclass{article}
\usepackage[a4paper]{geometry}
\usepackage[english]{babel}
\usepackage{amsmath}
\usepackage{amssymb}
\renewcommand{\vec}{\underline}
\newcommand{\dv}[1]{\vec #1'}
\title{ILA: H8 (Workshop 08)}
\author{Franz Miltz (UNN: s1971811)}
\date{November 19, 2019}
\begin{document}
\maketitle
This document will use the following notation:
\begin{itemize}
    \item $a$ may represent any scalar.
    \item $\vec{a}$ may represent any vector.
    \item $\begin{bmatrix}
        a_1\\
        \vdots\\
        a_n
    \end{bmatrix}$ is another representation of $\vec a$. 
    \item $a_i$ is the $i$th component of the vector $a$.
    \item $A$ may represent any matrix.
    \item $ab$ is a product of two scalars.
    \item $a\vec{b}$ is a scalar multiplication of the vector $\vec{b}$ with the scalar $a$
    \item $A\vec b$ is the matrix multiplication of the matrix $A$ with the vector $\vec b$
\end{itemize}
\section*{Dec 13: A4}
The given statement is true. Proof:\\
Let $A$ be an $n\times n$ matrix where
\begin{align*}
    A&=\begin{bmatrix}
        \vec a_1&\vec a_2&\cdots&\vec a_n
    \end{bmatrix}
\end{align*}
and
\begin{align*}
    A\vec v &= \lambda\vec v.
\end{align*}
According to the definition of the matrix multiplication we get
\begin{align*}
    A\vec v &= \begin{bmatrix}
        v_1a_{11} + v_2a_{12} + \cdots + v_na_{1n}\\
        v_1a_{21} + v_2a_{22} + \cdots + v_na_{2n}\\
        \vdots\\
        v_1a_{n1} + v_2a_{n2} + \cdots + v_na_{nn}
    \end{bmatrix}\\
    &= v_1\vec a_1 + v_2\vec a_2 + \cdots + v_n\vec a_n = \lambda \vec v
\end{align*}
Therefore, with $\lambda\not=0$,
\begin{align*}
    \vec v = \sum_{i=1}^n \frac{v_i}{\lambda} \vec a_i
\end{align*}
which shows that the eigenvector $\vec v$ of $A$ is a linear combination of the rows $\vec a_1, \vec a_2, ..., \vec a_n$ of $A$. $\square$
\section*{Dec 14: A4}
\subsection*{(a)}
The given statement is false. Proof:\\
We know that, in general, 
\begin{align*}
    A\vec v = \lambda\vec v \Rightarrow A^2\vec v = \lambda^2\vec v
\end{align*}
for all $n\times n$ matrices $A$.\\
Therefore we can let $B=A$ where the eigenvalues of the $n\times n$ matrix $A$ are $\lambda_1, \lambda_2, ..., \lambda_n$.
Thus the eigenvalues of $AB=A^2$ are $\lambda_1^2, \lambda_2^2, ..., \lambda_n^2$. This should already make clear that the eigenvalues of $A$ are, in general, not equal to the eigenvalues of $A^2$. Let's look at an example:
Let $A$ be the $2\times2$ matrix
\begin{align*}
    A = \begin{bmatrix}
        2 &0\\ 0& 3
    \end{bmatrix}.
\end{align*}
Therefore $\lambda_1=2$ and $\lambda_2=3$ are eigenvalues of $A$. You can see that the matrix $A^2$ only has the eigenvalues $\lambda_3=4$ and $\lambda_4=9$. Therefore neither $\lambda_1$ nor $\lambda_2$ are eigenvalues of $A^2=AB$ even though they are both eigenvalues of $A$ and $B$. $\square$
\subsection*{(b)}
The given statement is true. Proof:\\
Let $A$ and $B$ be two $n\times n$ matrices and let
\begin{align*}
    A\vec v =\lambda_A \vec v \text{  and  } B\vec v =\lambda_B\vec v.
\end{align*}
Therefore
\begin{align*}
    AB\vec v &= A(B\vec v)\\
    &= A(\lambda_B\vec v)\\
    &= \lambda_B(A\vec v)\\
    &=\lambda_A\lambda_B\vec v
\end{align*}
Therefore $\vec v$ is an eigenvector of $AB$ with eigenvalue $\lambda_{AB}=\lambda_A\lambda_B$. $\square$
\end{document}