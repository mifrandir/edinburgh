\documentclass{article}
\usepackage{homework-preamble}

\title{ILA: H2 (Workshop 08)}
\author{Franz Miltz (UNN: s1971811)}
\date{October 8, 2019}
\begin{document}
\maketitle
\section*{Q17}
We're looking at lines in $\mathbb{R}^2$ of following the form:
\begin{align}
    l: y = mx + c
\end{align}
If we write these lines in vector form, we get the following equation:
\begin{align}
    l: \vec x =
    \begin{bmatrix}
        0\\
        c
    \end{bmatrix}
    + r \begin{bmatrix}
        1\\
        m
    \end{bmatrix};
    r \in \mathbb{R}
\end{align}
Let's look at two lines, $l_1$ and $l_2$, which are supposed to be perpendicular to each other:
\begin{align}
    l_1: \vec x =
    \begin{bmatrix}
        0\\
        c_1
    \end{bmatrix}
    + r \begin{bmatrix}
        1\\
        m_1
    \end{bmatrix};
    r \in \mathbb{R}\\
    l_2: \vec x =
    \begin{bmatrix}
        0\\
        c_2
    \end{bmatrix}
    + s \begin{bmatrix}
        1\\
        m_2
    \end{bmatrix};
    s \in \mathbb{R}   
\end{align}
Those lines are perpendicular iff
\begin{align}
    \begin{bmatrix}
        1\\m_1
    \end{bmatrix}
    \cdot \begin{bmatrix}
        1\\m_2
    \end{bmatrix}
    &= 0\\
    \Leftrightarrow 1+m_1m_2 &=0\\ \Leftrightarrow m_1m_2 &= -1\:\:\:\square
\end{align}
\section*{Q42}
Let $P_1$ and $P_2$ be parallel planes in $\mathbb{R}^3$:
\begin{align}
    P_1:ax+by+cz=d_1\\
    P_2:ax+by+cz=d_2
\end{align}
Additionally let $B\in P_1$. Then we know that the following is true for the distance of $P_1$ and $P_2$:
\begin{align}
    d(P_1,P_2)&=d(B,P_2)
    =\frac{\vert ax_B+dy_B+cz_B-d_2\vert}{\sqrt{a^2+b^2+c^2}}
\end{align}
With $ax_B+by_B+cz_B=d_1$ and $\vec n=\begin{bmatrix}
   a\\b\\c
\end{bmatrix}$:
\begin{align}
    d(P_1, P_2)=\frac{\vert d_1-d_2\vert}{\vert \vec n \vert}\:\:\:\square
\end{align}
\end{document}