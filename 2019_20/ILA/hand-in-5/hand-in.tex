\documentclass{article}
\usepackage[a4paper]{geometry}
\usepackage{babel}
\usepackage{amsmath}
\usepackage{amssymb}
\renewcommand{\vec}{\underline}
\newcommand{\dv}[1]{\vec #1'}
\title{ILA: H2 (Workshop 08)}
\author{Franz Miltz (UNN: s1971811)}
\date{October 8, 2019}
\begin{document}
\maketitle
This document will use the following notation:
\begin{itemize}
    \item $a$ may represent any scalar.
    \item $A$ may represent or any matrix and $A^{-1}$ is its inverse.
    \item $I_n$ is the $n\times n$ identity matrix and $I$ is the $n\times n$ identity matrix in the current context (mostly resulting from a multiplication of the form $AA^{-1}$).
    \item $\begin{bmatrix}
        a_{11} & a_{12}&\dots &a_{1n}\\
        a_{21} & a_{22}&\dots &a_{2n}\\
        \vdots & \vdots & \ddots & \vdots\\
        a_{m1} & a_{m2} &\dots &a_{mn}
    \end{bmatrix}$ is another representation of $A$
\end{itemize}
\section*{Q26}
Let $A_1$, $A_2$ and $B$ be the following $2\times 2$ matrices:
\begin{align*}
    A_1 = \begin{bmatrix}
        1 &0 \\ 0 &0
    \end{bmatrix}
    \hspace{1cm} A_2 = \begin{bmatrix}
        0 &0 \\ 0 &1
    \end{bmatrix}
    \hspace{1cm} B = \begin{bmatrix}
        a &b \\ c &d
    \end{bmatrix}
\end{align*}
Now we can calculate the following products:
\begin{align*}
    A_1B = \begin{bmatrix}
        a &b \\0 &0
    \end{bmatrix}\hspace{1cm}
    BA_1 = \begin{bmatrix}
        a &0\\
        c &0
    \end{bmatrix}\\
    A_2B = \begin{bmatrix}
        0 &0\\
        c &d
    \end{bmatrix}\hspace{1cm}
    BA_2 = \begin{bmatrix}
        0 &b\\
        0 &d
    \end{bmatrix}
\end{align*}
To satisfy $A_1B=BA_1$, we now need $b = c = 0$. As you can see, the same condition let's $A_2B=BA_2$ as well.\\
Notice that there aren't any conditions for $a$ and $d$ because either they appear in both matrices in the same way or they don't appear at all.
Therefore they do not have any effect on the equality of the two products.\\
This condition includes all the values for $B$ that satisfy the given constraint. 
Since the question only asks for conditions that imply the given relation is true, but does not explicitly say, that it has to include every single solution, a much easier condition would be the trivial one: $a=b=c=d=0$.\\
This might not be, what the author of the question intended, though.
\section*{Q22}
Every line contains equivalent equations which can derived from it's predecessor by one multiplication and/or at most one expansion or reduction on each side.
\begin{align*}
    (A^{-1}X)^{-1} &= A(B^{-2}A)^{-1}\\
    X^{-1}A &=AA^{-1}B^2 &(AA^{-1}B=IB^2=B^2)\\
    X^{-1}A &=B^2\\
    XX^{-1}A &=XB^2 &(XX^{-1}A=IA=A)\\
    A &= XB^2\\
    AB^{-2} &= XB^2B^{-2} &(XB^2B^{-2}=XBIB^{-1}=XBB^{-1}=XI=X)\\
    AB^{-2} &= X
\end{align*}
Therefore $X=AB^{-2}$.
\section*{Q44}
\subsection*{a)}
I can just write the three matrices down, as the equations to proof this would all have the form $A_iA_i=A_i$.
\begin{align*}
    A_1 = \begin{bmatrix}
        0 &0\\0&0
    \end{bmatrix}\hspace{1cm}
    A_2 = \begin{bmatrix}
        1 &0\\1&0
    \end{bmatrix}\hspace{1cm}
    A_3 = \begin{bmatrix}
        1 &0\\0&0
    \end{bmatrix}
\end{align*}
\subsection*{b)}
Let $A$ be a $n\times n$ matrix which satisfies $A^2=A$ and is invertible. Then we can solve the equation in the following way:
\begin{align*}
    A^2 &= A\\
    A^{-1}A^2&=A^{-1}A\\
    IA&=I\\
    A&=I\hspace{0.5cm}\square
\end{align*}
\end{document}