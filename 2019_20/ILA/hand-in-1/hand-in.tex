\documentclass{article}
\usepackage{homework-preamble}

\title{ILA: H1 (Workshop 08)}
\author{Franz Miltz (UNN: s1971811)}
\date{October 1, 2019}
\begin{document}
\maketitle
\section*{Q32}
To find the fourth point in the rectangle, we need to calculate the vectors representing its unique sides $\vec a, \vec b$. Since we don't want to deal with signs later, we choose to let them originate from $B$, the point opposite of $D$.
\begin{align}
   \vec a = \vec{BA} = 
   \begin{bmatrix}
        1 - 3\\
        2 - 1\\
        3 - 2
   \end{bmatrix} =
   \begin{bmatrix}
       -2\\
       -4\\
       -1
   \end{bmatrix}\\
   \vec b = \vec{BC} =
   \begin{bmatrix}
        0 - 3\\
        5 - 6\\
        -4 + 2    
   \end{bmatrix} =
   \begin{bmatrix}
       -3\\
       -1\\
       -2
   \end{bmatrix}
\end{align}
Doing both of these calculations might be redundant, but it let's us verify our result later. Now we can calculate $\vec D$ by either following $b$ from $A$ or $a$ from $C$.
\begin{align}
    \vec D = \vec A + \vec b =
    \begin{bmatrix}
        1 - 3\\
        2 - 1\\
        3 - 2
    \end{bmatrix} =
    \begin{bmatrix}
        -2\\
        1\\
        1
    \end{bmatrix}
\end{align}
Now, let's do the same with $a$ and $C$:
\begin{align}
    \vec D = \vec C + \vec a =
    \begin{bmatrix}
        0 - 2\\
        5 - 4\\
        -4 + 5
    \end{bmatrix} =
    \begin{bmatrix}
        -2\\
        1\\
        1
    \end{bmatrix}
\end{align}
As you can see, the result is the same. Therefore we can be confident that this result is correct.
\section*{Q52}
By thinking about the geometrical implications of the two equations (especially the first one), we can deduce that the only way this might end up being true is, when $\vec u$ and $\vec v$ are parallel. This is the case because of the triangle inequality:
\begin{align}
    a + b > c \text{ with } a = |u|, b = |v| \text{ and } c = |u+v|
\end{align}
The fact that those three vectors may be aligned in a triangle, should be trivial.\\
Therefore, we will use the following equation for the rest of this task:
\begin{align}
    \vec v = r \vec u
\end{align}
Furthermore, we can see the similarities on the left-hand side (LHS) of both of these equations. Let's apply $(6)$ to that.
\begin{align}
    |\vec u + \vec v| &= |\vec u + r \vec u| = |(1+r)\vec u|=|1+r||\vec u|
\end{align}
Now, let's deal with the right-hand side (RHS).
\begin{align}
    |\vec u| + |\vec v| &= |\vec u| + |r \vec u| = (1 + |r|)|\vec u|\\
    |\vec u| - |\vec v| &= |\vec u| - |r \vec u| = (1 - |r|)|\vec u|
\end{align}
When does either of those equal the LHS? Let's start with (a):
\begin{align}
    |1 + r||\vec u| &= (1 + |r|)|\vec u|\\
    \Leftrightarrow|1+r| &=1+|r|
\end{align}
This equation is obviously true iff $r \geq 0$. Now, let's look at (b):
\begin{align}
    |1 + r||\vec u| &= (1 - |r|)|\vec u|\\
    \Leftrightarrow|1+r| &=1-|r|
\end{align}
One might be tempted to think, that this is only true iff $r \leq 0$. There is another restriction, though. This is because the LHS can never be less than $0$ due to the absolute value. The RHS on the other hand may very well be negative. Therefore this equation is true iff $r \leq 0 \wedge |r| \leq 1$.\\
According to the reasoning above (a) is true iff $\vec u$ and $\vec v$ are parallel and have the same direction and (b) is true iff $\vec u$ and $\vec v$ are parallel, have the same direction and $|\vec v| < |\vec u|$.
\end{document}