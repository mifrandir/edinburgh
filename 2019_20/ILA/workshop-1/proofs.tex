\documentclass{article}
\usepackage[a4paper]{geometry}
\usepackage{babel}
\usepackage{amsmath}
\usepackage{amssymb}
\title{ILA: Workshop 1, September 24}
\author{Franz Miltz}
\begin{document}
\maketitle
\section{Prolog}
This document will use the following notation:
\begin{itemize}
    \item $a$ may represent any scalar.
    \item $\vec{a}$ may represent any vector.
    \item $a_i$ is the $i$th component of the vector $a$.
    \item $A$ may represent any point in any vector space.
    \item $O$ is the origin of any vector space.
    \item $\vec{OA}=\vec{A}$ is a vector from $O$ to $A$ in the respective vector space.
    \item $\vert\vec{a}\vert$ is the magnitude of the vector $\vec{a}$
    \item $a\cdot b$ is a normal product of two scalars.
    \item $a\cdot\vec{b}$ is a scalar multiplication of the vector $\vec{b}$
    \item $\vec{a}\cdot\vec{b}$ is the scalar product of the vectors $\vec{a}$ and $\vec{b}$
\end{itemize}
\section{Page 30, Task 61}
The task is to proof the following equation with $\vec{u}, \vec{v}\in\mathbb{R}^n$:
\begin{align}
    (\vec{u}+\vec{v})\cdot(\vec{u}-\vec{v})
    =\left\vert\vec{u}\right\vert^2-\left\vert\vec{v}\right\vert^2
\end{align}
To proof that, we're going to need this equation:
\begin{align}
    (x+y)\cdot(x-y)=x^2-y^2 \text{ with } x,y \in \mathbb{R}
\end{align}
Therefore the following is also true:
\begin{align}
    (u_k+v_k)\cdot(u_k-v_k)=u_k^2-v_k^2 \text{ with } \vec{u}, \vec{v} \in \mathbb{R}^n \text{ and } n, k \in \mathbb{N}
\end{align}
Now we can do the following transformation with $\vec{u}, \vec{v} \in \mathbb{R}^n$:
\begin{align}
    (\vec{u}+\vec{v})\cdot(\vec{u}-\vec{v}) &
    =\sum_{k=1}^n (u_k+v_k)\cdot(u_k-v_k)                                              \\
                                            & =\sum_{k=1}^n u_k^2 - v_k^2
    =\sum_{k=1}^n u_k^2 -\sum_{k=1}^n v_k^2
    =\vec{u}\cdot\vec{u}-\vec{v}\cdot\vec{v}                                           \\
                                            & =\vert\vec{u}\vert^2-\vert\vec{v}\vert^2
\end{align}
\section{Page 30, Task 62}
\subsection{Subtask (a)}
The task is to proof the following equation with $\vec{u}, \vec{v}\in\mathbb{R}^n$:
\begin{align}
    \vert\vec{u}+\vec{v}\vert^2+\vert\vec{u}-\vec{v}\vert^2
    =2\cdot\vert\vec{u}\vert^2+2\cdot\vert\vec{v}\vert^2
\end{align}
We're going to expand the left-hand side:
\begin{align}
    \vert\vec{u}+\vec{v}\vert^2 + \vert\vec{u}-\vec{v}\vert^2
     & =\sum_{k=1}^n(u_k+v_k)^2+\sum_{k=1}^n (u_k-v_k)                            \\
     & =\sum_{k=1}^n u_k^2+2 u_k v_k+v_k^2 + \sum_{k=1}^n u_k^2-2 u_k v_k + v_k^2 \\
     & =\sum_{k=1}^n 2u_k^2+2v_k^2
    =2\vec{u}\cdot\vec{u}+2\vec{v}\cdot\vec{v}                                    \\
     & =2\vert\vec{u}\vert^2+2\vert\vec{v}\vert^2
\end{align}
\subsection{Subtask (b)}
The drawing of the diagram is left as an exercise to the reader.\\
What whe can infer from the diagram is that the squares of the magnitudes of the two diagonals of a parallelogram equal the squares of the magnitudes of the four sides.
\section{Page 30, Taks 63}
The task is to proof the following equation with $\vec{u}, \vec{v}\in\mathbb{R}^n$:
\begin{align}
    \frac{1}{4}\vert \vec{u}+\vec{v}\vert^2-\frac{1}{4}\vert \vec{u}-\vec{v}\vert^2
     & =\vec{u}\cdot\vec{v}
\end{align}
To do this, we expand the left-hand side:
\begin{align}
    \frac{1}{4}\vert \vec{u}+\vec{v}\vert^2-\frac{1}{4}\vert \vec{u}-\vec{v}\vert^2
     & =\frac{1}{4}\sum_{k=1}^n(u_k+v_k)^2 - \frac{1}{4}\sum_{k=1}^n(u_k-v_k)^2                     \\
     & =\frac{1}{4}\sum_{k=1}^n(u_k^2+2u_k v_k+v_k^2)-\frac{1}{4}\sum_{k=1}^n(u_k^2-2u_k v_k+v_k^2) \\
     & =\frac{1}{4}\sum_{k=1}^n(4u_k v_k)
    = \sum_{k=1}^n u_k v_k                                                                          \\
     & =\vec{u}\cdot\vec{v}
\end{align}
\section{Page 31, Task 64}
\subsection{Subtask (a)}
The task is to prove the following equation iff the vectors $\vec{u}, \vec{v}\in\mathbb{R}^n$ are orthogonal.
\begin{align}
    \vert\vec{u}+\vec{v}\vert = \vert\vec{u}-\vec{v}\vert
\end{align}
Expanding both sides yields:
\begin{align}
    \vec{u}\cdot\vec{u}+2(\vec{u}\cdot\vec{v})+\vec{v}\cdot\vec{v}
     & =\vec{u}\cdot\vec{u}-2(\vec{u}\cdot\vec{v})+\vec{v}\cdot\vec{v}
     & \vert - (\vec{u}\cdot\vec{u}+\vec{v}\cdot\vec{v})               \\
    2(\vec{u}\cdot\vec{v})
     & =-2(\vec{u}\cdot\vec{v})
\end{align}
This equation is true iff $\vec{u}\cdot\vec{v}=0$ which is the case iff $\vec{u}$ and $\vec{v}$ are orthogonal.
\subsection{Subtask (b)}
The drawing of the diagram is left as an exercise to the reader (and might already happened in 3.2).\\
The relation in (a) shows us, that the two diagonals of a parallelogram are equally long iff the sides are orthogonal, which means the shape is not only a parallelogram, but a rectangle as well.
\section{Page 32, Task 1}
\begin{align}
    \vec{P}=\frac{2}{3}\vec{A}+\frac{1}{3}\vec{B}
\end{align}
With the ratio $r\in\mathbb{R}$ and $0\leq r\leq1$ we can generalize this to
\begin{align}
    \vec{P}_r = (1-r)\cdot\vec{A}+r\cdot\vec{B}
\end{align}
where $P_r$ is the point which is on the line segment $\bar{AB}$ with a distance of $A$ which is exactly $r\vert\bar{AB}\vert$.
\section{Page 32, Task 2}
With respect to Figure 1.43 in the book, we can write the following equations:
\begin{align}
    \vec{P}=\frac{\vec{A}+\vec{C}}{2} \\
    \vec{Q}=\frac{\vec{B}+\vec{C}}{2}
\end{align}
Thus we can represent the vector $\vec{PQ}$ in the following way:
\begin{align}
    \vec{PQ} & =\vec{Q}-\vec{P}                                     \\
             & =\frac{\vec{B}+\vec{C}}{2}-\frac{\vec{A}+\vec{C}}{2}
    =\frac{\vec{B}-\vec{A}}{2}                                      \\
             & =\frac{1}{2}\vec{AB}
\end{align}
Since $\bar{AB}$ is a side of the triangle, the line $\bar{PQ}$ has to be parallel to it. Additionally, we know that $\bar{AB}$ is exactly twice as long as $\bar{PQ}$.
\section{Page 32, Task 3}
With respect to Figure 1.44 in the book, we can write the following equation:
\begin{align}
    \vec{AD} & =\vec{AB}+\vec{BC}+\vec{CD}
\end{align}
This let's us write down the equations for one side of the parallelogram $\vec{RS}$:
\begin{align}
    \vec{RS} & =\frac{\vec{CD}-\vec{AD}}{2}                   \\
             & =\frac{\vec{CD}-\vec{AB}-\vec{BC}-\vec{CD}}{2}
    =-\frac{\vec{AB}+\vec{BC}}{2}                             \\
             & =\vec{QP}
\end{align}
Since the sides $\bar{RS}$ and $\bar{QP}$ can be represented using the same vector, they have the same length and are parallel. Thus the quadrilateral $PQRS$ is a parallelogram.\\
NOTE: This is true for every vector space $\mathbb{R}^n$. This proof did not require any special aspect of $\mathbb{R}^2$.
\end{document}