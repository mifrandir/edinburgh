\documentclass{article}
\usepackage{homework-preamble}
\mkthms

\title{PPS: Homework 8 (Workshop 33)}
\author{Franz Miltz (UUN: S1971811)}
\date{30 March 2020}

\begin{document}
\maketitle
\section*{Problem 1}
\begin{claim}
  Let $S$ be the set of functions $f:\R\to\R$ and let $\sim$ be a relation on $S$ such that $f\sim g$ if $f(x)=g(x)$ except possibly at finitely many points.
  Then $\sim$ is an equivalence relation.
\end{claim}
\begin{proof}
  The relation is reflexive since $f(x)=f(x)$ for all $x\in\R$.\\
  Let $A\subset \R$ be a finite set such that
  $
    x \in A \Leftrightarrow f(x)\not= g(x)
  $
  for some $f,g\in S$.
  Then
  $
    x\in A \Leftrightarrow g(x)\not=f(x)
  $
  and, since $A$ is still finite, $\sim$ is symmetric.\\
  Let $f,g,h \in S$ and let $A$ and $B$ be finite sets such that
  \begin{align*}
    x\in A \Leftrightarrow f(x)\not= g(x), \\
    x\in B \Leftrightarrow g(x)\not= h(x).
  \end{align*}
  This implies that $f\sim g$ and $g\sim h$. Since for all $x\not\in A\cup B$
  \begin{align*}
    f(x)=g(x)=h(x)
  \end{align*}
  we conclude that $C\subseteq A \cup B$ where $C$ is the set such that $x\in C \Leftrightarrow f(x)\not=h(x)$.
  Since $A$ and $B$ are finite, so is $C$ and thus $f\sim h$ which makes $\sim$ transitive.\\
  Since $\sim$ is reflexive, symmetric and transitive, it is an equivalence relation.
\end{proof}
\section*{Problem 2}
\begin{claim}
  Let $f:X\to Y$ be a function. Then $f^{-1}(f(A))=A$ for all sets $A\subseteq X$ ($P$) if, and only if, $f$ is injective ($Q$).
\end{claim}
\begin{proof}
  Assume $P$, i.e. $f^{-1}(f(A))$ for all $A\subseteq X$. Note that for all $x\in X$, $\{x\}\subseteq X$. Thus
  \begin{align*}
    \forall x \in X,\:f^{-1}\left(f\left(\{x\}\right)\right) = \{f(x)\}.
  \end{align*}
  Now assume for some $x,y\in X$ such that $x\not=y$, $f(x)=f(y)$. Then
  \begin{align*}
    f\left(\{x\}\right)=\{f(x)\},
  \end{align*}
  but
  \begin{align*}
    f^{-1}\left(f(\{x\})\right)=f^{-1}\left(\{f(x)\}\right)=\{x,y\}\not=\{x\}.
  \end{align*}
  Therefore we conclude that $f(x)=f(y)\Rightarrow x=y$, i.e. $f$ is injective. This proves $P\Rightarrow Q$.\\
  Assume $Q$, i.e. $f$ is injective. Let $A\subseteq X$. Then $f(A)$ is the image $A$ under $f$ and therefore
  \begin{align*}
    f(A) = \{f(x)\:|\:x \in A\}.
  \end{align*}
  Similarly, $f^{-1}(f(A))$ is the preimage of $f(A)$ under $f$ and therefore
  \begin{align*}
    f^{-1}(f(A)) = \{x\in X\:|\: f(x) \in f(A)\}.
  \end{align*}
  Since $f$ is injective, we know that $f(x)=f(y)\Rightarrow x=y$, i.e. for all $y\in Y$ there exists at most one $x\in X$ such that $f(x)=y$. Thus
  \begin{align*}
    f^{-1}(f(A)) = A.
  \end{align*}
  This proves $Q\Rightarrow P$.\\
  Since $P\Rightarrow Q$ and $Q\Rightarrow P$, $P\Leftrightarrow Q$.
\end{proof}
\section*{Problem 3}
\begin{claim}
  Let $f:X\to Y$ and $g:Y\to Z$ be functions such that $g$ is injective and $f\circ g$ is bijective. Then $f$ is bijective.
\end{claim}
\begin{adjustwidth}{2em}{0pt}
  \begin{lemma}
    \label{l1}
    Let $f:X\to Y$ and $g:Y\to Z$ be functions such that $f\circ g$ is bijective. Then $f$ is injective.
  \end{lemma}
  \begin{proof}
    By contradiction. Assume $f$ is not injective, i.e.
    \begin{align*}
      \exists x,y \in X \st x \not= y \text{ and } f(x) = f(y).
    \end{align*}
    Then
    \begin{align*}
      \exists x,y \in X \st x \not= y \text{ and } \left(f\circ g\right)(x) = \left(f\circ g\right)(y),
    \end{align*}
    i.e. $f\circ g$ is not injective and therefore not bijective.
    This contradicts the premise.
    Thus the assumption has to be unsatisfiable.
  \end{proof}
  \begin{lemma}
    \label{l2}
    Let $f:X\to Y$ and $g:Y\to Z$ be functions such that $f\circ g$ is bijective. Then $g$ is surjective.
  \end{lemma}
  \begin{proof}
    By contradiction. Assume $g$ is not surjective, i.e.
    \begin{align*}
      \exists z \in Z \st \forall y \in Y,\:g(y) \not= z.
    \end{align*}
    Then
    \begin{align*}
      \exists z \in Z \st \forall x \in X,\: \left(f\circ g\right)(x) \not= z,
    \end{align*}
    i.e. $f\circ g$ is not surjective and therefore not bijective.
    This contradicts the premise.
    Thus the assumption has to be unsatisfiable.
  \end{proof}
\end{adjustwidth}
\begin{proof}
  By contradiction.\\
  Using Lemma \ref{l1} we know that $f$ is injective and using Lemma \ref{l2} we know that $g$ is surjective.
  According to the premise $g$ is also injective, which makes $g$ bijective.\\
  Now assume $f$ is not bijective which means it is not surjective, i.e.
  \begin{align}
    \label{eq1}
    \exists y \in Y \st \forall x \in X,\: f(x)\not= y.
  \end{align}
  Then, since $g$ is bijective,
  \begin{align}
    \label{eq2}
    \exists z \in Z \st \forall x \in X,\: \left(f\circ g\right)(x)\not= z,
  \end{align}
  i.e. $f\circ g$ is not surjective and therefore not bijective.
  This contradicts the premise.
  Thus the assumption has to be unsatisfiable and $f$ is always bijective.
\end{proof}
\emph{Note: The step between the statements \ref{eq1} and \ref{eq2} might seem arbitrary but you can see why it is true if you plug all of $Y$ into $g$.
  You then get $g(y)=z$, $g(f(x))=(f\circ g)(x)$ and $g(Y)=Z$. Since $y_1=y_2\Leftrightarrow g(y_1)=g(y_2)$, both statements are still equivalent after the transformation of $Y$ with $g$.}
\end{document}