\documentclass{article}
\usepackage{homework-preamble}
\mkthms

\title{PPS: Homework 7 (Workshop 33)}
\author{Franz Miltz (UNN: S1971811)}
\date{23 March 2020}

\begin{document}
\maketitle
\section*{Problem 1}
\begin{claim}
  Let $s:\Z \to \Z$ be the function defined as the alternating sum. Then for any integer $x\in\Z$, $11|x \Leftrightarrow 11|s(x)$.
\end{claim}
\begin{proof}
  Consider the alternating sum of any number $x=d_nd_{n-1}\cdots d_0$ with the digits of $d_0,...,d_n\in\Z_{10}$.
  We assume $n$ to be even. This is possible without loss of generality as we can always add a $0$ to the beginning of a number.
  Then
  \begin{align*}
    x    & = \sum_{i=0}^n 10^id_i,   \\
    s(x) & = \sum_{i=0}^n (-1)^id_i.
  \end{align*}
  Now we will consider $x\mod 11$. To do this, we can focus on a single summand $10^id_i$ (\emph{Liebeck, Proposition 13.3}).
  Notice that, since $10\equiv -1 \mod 11$, we can write
  \begin{align*}
    10^id_i \equiv (-1)^id_i \mod 11
  \end{align*}
  and thus, using the second part of \emph{Liebeck, Proposition 13.3},
  \begin{align*}
    x \equiv \sum_{i=0}^n 10^id_i \equiv \sum_{i=0}^n (-1)^id_i\equiv s(x)\mod 11.
  \end{align*}
  Since $m|a\Leftrightarrow a\equiv 0 \mod m$ for all $a\in\Z$ and $m\in\N$, this proves our claim.
\end{proof}
\section*{Problem 2}
\begin{claim}
  Let $p$ be a prime. Then $(p-1)!\equiv p-1 \mod p$.
\end{claim}
\begin{subproof}
  \begin{lemma}
    \label{l1}
    Let $p$ be a prime. Then for any $a\in\Z$ there is a unique $x\in\Z_p$ so that $ax\equiv 1 \mod p$.
  \end{lemma}
  \begin{proof}
    Since $\hcf(a,p)=1$, we know from \emph{Liebeck, Propsition 13.6} that there exists a solution $x\in\Z$.
    Consider two solutions $x_1, x_2\in \Z$. Then
    \begin{align*}
      ax_1 \equiv ax_2 \equiv 1 \mod p.
    \end{align*}
    Using \emph{Liebeck, Proposition 13.3} and $a\equiv a \mod p$ (\emph{Liebeck, Proposition 13.2}), we know that
    \begin{align*}
      ax_1                & \equiv ax_2 \mod p \\
      \Leftrightarrow x_1 & \equiv x_2 \mod p.
    \end{align*}
    Thus there is a unique solution $x\in\Z_p$ where $x\equiv x_1\equiv x_2 \mod p$.
  \end{proof}
\end{subproof}
\begin{proof}
  According to Lemma \ref{l1}, for each of the factors $a$ in $(p-1)!$, there is another factor $b$ such that
  \begin{align*}
    ab\equiv 1 \mod p.
  \end{align*}
  Observe that $a=b$ if, and only if,
  \begin{align*}
    a^2 \equiv 1 \mod p.
  \end{align*}
  This is the case if, and only if, $a\equiv 1 \mod p$ or $a\equiv -1 \equiv p-1 \mod p$.
  For all other factors $a\in(1,p-1)$ the corresponding $b$ has to be in the product as well.
  Therefore the product of all the elements in $(1,p-1)$ has to be congruent to $1\mod p$.
  Notice that for $p=2$ the intervall $(1,p-1)$ is empty.
  This is not an issue, as it does not change the result.
  Thus we get
  \begin{align*}
    (p-1)!\equiv (p-1)\cdot 1\cdot 1 \equiv p-1 \mod p.
  \end{align*}
\end{proof}
\section*{Problem 3}
\begin{claim}
  Let $p$ be a prime and $p\not=2,5$. Then there exists an integer $k$ such that $p|\sum_{i=0}^k 10^i$.
\end{claim}
\begin{adjustwidth}{2em}{2em}
  \begin{lemma}
    \label{l2}
    Let $p>5$ be a prime number. Then
    \begin{align*}
      \forall x,y\in[1,p-1],\: x=y \Leftrightarrow 10^x\equiv 10^y\mod p.
    \end{align*}
  \end{lemma}
  \begin{proof}
    At first, note that $10^n \not\equiv 0 \mod p$ for all $n\in\N$.
    This is because $10$ is not divisible by $p$.
    Now assume that for some $x,y\in[1,p-1]$ where $x>y$
    \begin{align*}
      10^x \equiv 10^y \mod p.
    \end{align*}
    This implies that
    \begin{align*}
      10^x-10^y                        & \equiv 0\mod p   \\
      \Leftrightarrow 10^y(10^{x-y}-1) & \equiv 0 \mod p.
    \end{align*}
    Since $10^y$ is not divisible by $p$, we get
    \begin{align*}
      10^{x-y}\equiv 1 \mod p.
    \end{align*}
    We know that this is the case if $x-y=0$, which is impossible since $x>y$.
    By \emph{Fermat's Little Theorem} (\emph{Liebeck, Theorem 14.1}) we also know that, $x-y=p-1$ would be an option.
    This can also be ruled out because $y>0$ and $x<p$.\\
    Therefore there exists a $z\in(0,p-1)$ such that
    \begin{align*}
      10^z \equiv 1 \mod p.
    \end{align*}
    Using \emph{Liebeck, Proposition 13.4}, we get that
    \begin{align*}
      10^{nz}\equiv 1^n\equiv 1 \mod p
    \end{align*}
    for all $n\in\N$. \\
    Using \emph{Fermat's Little Theorem} again, we know that $10^p\equiv 10^{p-1}10^1\equiv 10\mod p$ and thus for any $m\in\N$ and $x\in\Z$
    \begin{align*}
      10^{x+mp}\equiv 10^{x}\left(10^{p}\right)^m\equiv 10^{x}\mod p.
    \end{align*}
    Therefore if $x\equiv y \mod p$, then $10^x\equiv 10^y\mod p$. If we let
    \begin{align*}
      nz \equiv k \mod p
    \end{align*}
    for some $k\in\Z_p$, then
    \begin{align*}
      10^{nz}\equiv 10^{k} \equiv 1 \mod p.
    \end{align*}
    This $k$ is different for all $n\in[1,p-1]$. The proof for this is given in \emph{Liebeck} on page 118 in the proof of \emph{Fermat's Little Theorem}, if you let $a=10$ because $p$ does not divide $10$. This would therefore imply that all $10^k$, for any $k\in\Z_p$ are congruent to $1$ modulo $p$. For any prime $p>5$ this is a contradiction as $10\equiv 3 \mod 7$ and for all $p>7$, $10\in\Z_p$.
  \end{proof}
  \emph{Note: I have not included the proof from the book because it requires quite a lot of space relative to its actual content.
    Additionally, this lemma and particularly its proof may be used to get the following relation:\\
    Let $p$ be a prime number and $a$ any positive integer such that $p\nmid a$ and $a\not\equiv 1 \mod p$. Then $x\equiv y \mod p-1 \Leftrightarrow a^x\equiv a^y \mod p$.}
\end{adjustwidth}
\begin{proof}
  The claim is true for $p=3$ as $3|111$. Let's now consider $p>5$.\\
  Let $s(k)=\sum_{i=0}^{k-1} 10^i$ be a number of length $k$ consisting only of $1$'s.
  Further, let $m:\N\to\Z_p$ be the function such that for all $k\in\N$, $10^k\equiv m(k)\mod p$. Then
  \begin{align*}
    s(k) \equiv \sum_{i=0}^{k-1} m(i) \mod p.
  \end{align*}
  Since all $m(i)$ are different for $i\in[1,p-1]$ (Lemma \ref{l2}), $10^0\equiv 1\equiv 10^{p-1}\mod p$ and $m(i)\not=0$, we can use \emph{Liebeck, Proposition 13.3} to get
  \begin{align*}
    s(p-1)
    \equiv \sum_{i=0}^{p-2}m(i)
    \equiv \sum_{i=1}^{p-1}m(i)
    \equiv \sum_{i=1}^{p-1} i
    \equiv \frac{p(p-1)}{2}
    \equiv 0 \mod p.
  \end{align*}
  Thus $p|s(p-1)$ for all primes $p>5$.
\end{proof}
\emph{Note: This does not imply that $s(p-1)$ is the smallest number to satisfy the condition, it just proves that there exists at least one.}
\end{document}