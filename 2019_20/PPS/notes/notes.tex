\documentclass{article}
\usepackage[a4paper]{geometry}
\geometry{tmargin=3cm, bmargin=3cm, lmargin=2cm, rmargin=2cm}
\usepackage{babel}
\usepackage{amsmath}
\usepackage{amssymb}
\usepackage{amsthm}
\usepackage{hyperref}
\usepackage{newpxtext}
\hypersetup{
	colorlinks,
	citecolor=black,
	filecolor=black,
	linkcolor=black,
	urlcolor=black
}
\usepackage{nicefrac}
\usepackage{siunitx}
\usepackage{mathtools}
\newtheoremstyle{sltheorem} {}                % Space above
{}                % Space below
{\upshape}        % Theorem body font % (default is "\upshape")
{}                % Indent amount
{\bfseries}       % Theorem head font % (default is \mdseries)
{.}               % Punctuation after theorem head % default: no punctuation
{ }               % Space after theorem head
{}                % Theorem head spec
\theoremstyle{sltheorem}
\newtheorem{definition}{Definition}[section]
\newtheorem{theorem}{Theorem}[section]
\newtheorem{lemma}[theorem]{Lemma}
\newtheorem{corollary}[theorem]{Corollary}
\newtheorem{proposition}[theorem]{Proposition}
\newcommand{\R}{\mathbb{R}}
\newcommand{\N}{\mathbb{N}}
\renewcommand{\C}{\mathbb{C}}
\newcommand{\Z}{\mathbb{Z}}
\newcommand{\st}{\text{ s.t }}
\DeclareMathOperator{\lub}{LUB}
\DeclareMathOperator{\glb}{GLB}
\DeclareMathOperator{\hcf}{hcf}
\DeclareMathOperator{\lcm}{lcm}
\DeclareMathOperator{\sgn}{sgn}
\DeclareMathOperator{\cl}{cl}
\newcommand*\lneg[1]{\overline{#1}}
\newcommand*\B[1]{\textbf{#1}}
\newcommand*\binco[2]{\begin{pmatrix}
    #1\\#2
\end{pmatrix}}
\begin{document}
\title{Proofs and Problem Solving Notes (SEM2)}
\author{Franz Miltz}
\date{May 28, 2020}
\maketitle
\tableofcontents
\section{Logic}
\begin{theorem}
    If $A$ and $B$ are statements, then $A\Rightarrow B\equiv \lneg{A}\vee B$.
\end{theorem}
\section{The Real Line}
\begin{definition}
    A \B{field} is a set $F$ along with operations $+$ and $\cdot$ so that the following hold:\\
    Rules of Addition:
    \begin{enumerate}
        \item $a+b\in F$.
        \item $a+b = b+a$.
        \item $a+(b+c) = (a+b)+c$.
        \item $\exists 0\in F.\: \forall a\in F.\: 0+a=a$.
        \item $\forall a\in F.\: \exists -a\in F.\: a+(-a)=0$.
    \end{enumerate}
    Rules of Mulitplication:
    \begin{enumerate}
        \item $a\cdot b \in F$.
        \item $a\cdot b = b\cdot a$.
        \item $a\cdot(b\cdot c)=(a\cdot b)\cdot c$.
        \item $\exists 1\in F.\: 1\cdot a = a$.
        \item $a\not=0\Rightarrow\exists \nicefrac{1}{a}\in F.\: a\cdot\nicefrac{1}{a}=1$.
        \item $a\cdot (b+c) = a\cdot b + a\cdot c$.
    \end{enumerate}
\end{definition}
\begin{theorem}
    The rules of addition above imply that, for $x,y,z\in \R$,
    \begin{enumerate}
        \item $x+y=x+z\Leftrightarrow y=z$.
        \item $x+y=x\Rightarrow y=0$.
        \item $x+y=0\Rightarrow y=-x$.
        \item $-(-x)=x$.
    \end{enumerate}
\end{theorem}
\begin{theorem}
    The rules of multiplication above imply that, for $x, y, z \in \R$,
    \begin{enumerate}
        \item $xy=xz\Leftrightarrow y=z$.
        \item $xy=x\Rightarrow y=1$.
        \item $xy=1\Rightarrow y=\nicefrac{1}{x}$.
        \item $\nicefrac{1}{\nicefrac{1}{x}}=x$.
    \end{enumerate}
\end{theorem}
\begin{theorem}
    For $x,y\in\R$,
    \begin{enumerate}
        \item $0\cdot x = 0$.
        \item $x\not=0\not=y\Rightarrow xy\not=0$.
        \item $(-x)y=-(xy)=x(-y)$.
        \item $(-x)(-y)=xy$.
    \end{enumerate}
\end{theorem}
\begin{theorem}
    Given $x>0$ and $n\in\N$, there is a unique $y>0$ so that $y^n=x$, and we write this nuber $y$ as $y^{\nicefrac{1}{x}}$.
\end{theorem}
\begin{theorem}
    Given any two numbers $a<b$, there is $r\in\mathbb{Q}$ with $a<r<b$.
\end{theorem}
\begin{theorem}
    The number $\sqrt{2}$ is irrational.
\end{theorem}
\begin{theorem}
    Let $a$ be rational and $b$ be irrational.
    \begin{enumerate}
        \item $a+b$ is irrational.
        \item If $a\not=0$, then $ab$ is irrational.
    \end{enumerate}
\end{theorem}
\begin{theorem}
    Given $x>0$ and $n\in\N$, there is a unique $y>0$ so that $y^n=x$, and we write this number as $y=x^{\nicefrac{1}{n}}$.
\end{theorem}
\begin{theorem}
    Let $x>0$, $y>0$ and $p,q\in\mathbb{Q}$. Then
    \begin{enumerate}
        \item $x^px^q=x^{p+q}$.
        \item $(x^p)^q=x^{pq}$.
        \item $(xy)^p=x^py^p$.
    \end{enumerate}
\end{theorem}
\section{Inequalities}
\begin{definition}
    Given $x, y \in \R$, we may write $x<y$, which we prononounce "$x$ is less than $y$". The symbol $<$ satisfies the following axioms:
    \begin{enumerate}
        \item If $x\in R$ then exactly one of the following is true: $x>0$, $x=0$ or $x<0$.
        \item If $x>y$, then $-x<-y$.
        \item If $x>y$ and $c\in\R$, then $x+c > y+c$.
        \item If $x > 0$ and $y > 0$, then $xy > 0$.
        \item If $x>y$ and $y>z$, then $x>z$.
    \end{enumerate}
\end{definition}
\begin{theorem}
    If $x>0$, then $-x<0$.
\end{theorem}
\begin{theorem}
    If $x\not=0$, then $x^2 > 0$.
\end{theorem}
\begin{theorem}
    If $x>0$, then $\nicefrac{1}{x}>0$.
\end{theorem}
\begin{theorem}
    If $x>0$, then $u>v$ iff $xu>xv$.
\end{theorem}
\begin{theorem}
    If $u,v>0$, then $u^2>v^2$ iff $u>v$.
\end{theorem}
\begin{theorem}
    For $u,v\in\R$,
    \begin{align*}
        uv\leq \frac{u^2+v^2}{2}.
    \end{align*}
\end{theorem}
\begin{theorem}
    \B{AM-GM Inequality}\\
    Let $n\in\N$ and $x_1, x_2, ..., x_n \geq 0$. Then 
    \begin{align*}
        (x_1\cdot x_2\cdots x_n)^{\frac{1}{n}}\leq\frac{x_1+\cdots+x_n}{n}.
    \end{align*}
\end{theorem}
\begin{theorem}
    \B{Cauchy-Schwartz Inequality}\\
    Let $n\in\N$ and $x_1,...,x_n,y_1,...,y_n\in\R$. Then
    \begin{align*}
        x_1y_1+\cdots x_ny_n \leq \sqrt{x_1^2+\cdots+x_n^2}\sqrt{y_1^2+\cdots+y_n^2}.
    \end{align*}
\end{theorem}
\begin{definition}
    For $x\in\R$, we define the \B{absolute value} of $x$ to be
    \begin{align*}
        \vert x \vert = \begin{cases}
            x &\text{if } x \geq 0\\
            -x &\text{if } x < 0
        \end{cases}
    \end{align*}
\end{definition}
\begin{theorem}
    For $x\in\R$, $-|x|\leq x \leq |x|$.
\end{theorem}
\begin{theorem}
    For $x,y\in\R$, $|xy|=|x|\cdot |y|$.
\end{theorem}
\begin{theorem}
    For $x,y\in\R$, $|x+y| \leq |x| + |y|$.
\end{theorem}
\begin{theorem}
    For $x,y\in\R$, $|x|-|y|\leq |x+y|$.
\end{theorem}
\section{Induction}
\begin{definition}
    \B{The Principle of Mathematical Induction}\\
    Given a list of statements $P(k)$, $P(k+1)$, ..., we know that $P(n)$ is true for every integer $n\geq k$ if
    \begin{enumerate}
        \item we know that $P(k)$ is true,
        \item and we can prove that $P(n)\Rightarrow P(n+1)$ for any integer $n\geq k$.
    \end{enumerate}
\end{definition}
\begin{definition}
    \B{The Principle of Strong Mathematical Induction}\\
    Let $k\in\N$ and let $P(k)$, $P(k+1)$, ..., be statements. Suppose that
    \begin{enumerate}
        \item $P(k)$ is true,
        \item for any integer $n\geq k$, $P(k), P(k+1), P(k+2), ..., P(n)\Rightarrow P(n+1)$.
    \end{enumerate}
    Then $P(n)$ is true for all integer $n\geq k$.
\end{definition}
\begin{theorem}
    Every integer $n\geq 2$ can be written as a product of primes $n=p_1\cdots p_k$.
\end{theorem}
\begin{theorem}
    Suppose that $x,y>0$ and $n\in\N$.
    \begin{enumerate}
        \item $x^n>y^n\Leftrightarrow x>y$.
        \item $x^{\nicefrac{1}{n}}>y^{\nicefrac{1}{n}}\Leftrightarrow x>y$.
    \end{enumerate}
\end{theorem}
\section{(Least) Upper Bounds}
\begin{definition}
    Let $A\subseteq\R$.
    \begin{itemize}
        \item $A$ is said to be \B{bounded above} if $\exists M\in\R\st\forall x\in A,x\leq M$.\\
        The number $M$ is called an \B{upper bound} of $A$.
        \item Similarly, $A$ is said to be \B{bounded below} if $\exists m\in\R\st\forall x\in A,x\geq m$.\\
        The number $m$ is called a \B{lower bound} of $A$.
        \item We say $A$ is \B{bounded} if it is bounded above and below.
    \end{itemize}
\end{definition}
\begin{definition}
    Given a set $A\subseteq\R$, a number $L$ is a \B{least upper bound} (LUB) for $A$ if
    \begin{enumerate}
        \item $L$ is an upper bound for $A$, and
        \item $\forall t<L$, $t$ is not an upper bound..
    \end{enumerate}
    Similarly, we say $L$ is a \B{greatest lower bound} (GLB) for $A$ if
    \begin{enumerate}
        \item $L$ is a lower bound for $A$, and
        \item $\forall t>L$, $t$ is not a lower bound.
    \end{enumerate}
    If the LUB exists for a set $A$, we denote it by $\lub(A)$. Similarly, we denote the GLB of $A$ by $\glb(A)$.
\end{definition}
\section{Limits}
\begin{definition}
    \B{The $\varepsilon-N$ definition of a limit.} Let $(x_n)^\infty_{n=1}$ be a sequence of real numbers $x_1,x_2,...$ and let $L\in\R$. We say that $x_n$ \B{converges} to $L$ and write $\lim_{n\to\infty}x_n=L$ if for all $\varepsilon>0$, there is a number $N$ so that for $n>N$, we have
    \begin{align*}
        |x_n-L|<\varepsilon
    \end{align*}
    or equivalently, that
    \begin{align*}
        L-\varepsilon<x_n<L+\varepsilon.
    \end{align*}
\end{definition}
\begin{definition}
    We say a sequence $(a_n)$ is \B{bounded} if the set of values $\{a_1, a_2, ...\}$ is a bounded set.
\end{definition}
\begin{theorem}
    If $(a_n)$ converges, it is bounded.
\end{theorem}
\begin{theorem}
    Let $a_n$ and $b_n$ converge to $a$ and $b$ respectively. Then
    \begin{enumerate}
        \item $a_n+b_n\to a+b$
        \item $a_nb_n\to ab$
        \item If $c\in\R$, then $ca_n\to ca$.
        \item If $b\not=0$ and $b_n\not=0$ for all $n$, then $\frac{a_n}{b_n}\to\frac{a}{b}$.
    \end{enumerate}
\end{theorem}
\begin{theorem}
    For $k\geq2$ an integer, if $x_n\to L$, then $x^k_n\to L^k$.
\end{theorem}
\begin{theorem}
    Let $(x_n)$ and $(y_n)$ be sequences.
    \begin{enumerate}
        \item If $x_n\to x$, $y_n\to y$, and $x_n\leq y_n$ for all $n\in\N$, then $x\leq y$.
        \item If $x_n\to x$ and $x_n\leq y$ for all $n$, then $x\leq y$.
    \end{enumerate}
\end{theorem}
\begin{theorem}
    Let $x>0$ and $k\in\N$. There is $y>0$ so that $y^k=x$.
\end{theorem}
\begin{definition}
    Let $(x_n)$ be a sequence of real numbers.
    \begin{enumerate}
        \item We say $x_n\to \infty$ if, for all $M>0$, there is an $N$ so that $n\geq N$ implies $x_n\geq M$.
        \item Similarly, we say $x_n\to-\infty$ if, for all $M<0$, there is an $N$ so that $n\geq N$ implies $x_n\leq M$.
    \end{enumerate}
\end{definition}
\section{Convergence Theorem}
\begin{definition}
    A sequence $(a_n)^\infty_{n=1}$ is \B{increasing} if $a_n\leq a_n+1$ for all $n$ and \B{decreasing} if $a_n\geq a_{n+1}$ for all $n$. We say $a_n$ is \B{monotone} if it is either increasing or decreasing.
\end{definition}
\begin{theorem}
    \B{The Monotone Convergence Theorem}\\
    Let $(a_n)$ be an increasing sequence of real numbers that is bounded above (i.e. there is $M$ so that $a_n\leq M$ for all $n$). Then $(a_n)$ converges. Then $(a_n)$ converges. If $(a_n)$ is decreasing and is bounded below, then $(a_n)$ converges.
\end{theorem}
\begin{definition}
    Given a sequence $(x_n)$, a \B{subsequence} is a sequence of the form $(x_{n_k})$ where $n_1<n_2<\cdots$ are positive integers.
    A real number $x$ is a \B{limit point} of a sequence $(x_n)$ if there is a subsequence $(x_{n_k})$ so that $x_{n_k}\to x$.
\end{definition}
\begin{theorem}
    Given a sequence $(x_n)$ and a number $L$, $L$ is a limit point iff for all $\varepsilon > 0$ and for all integers $N$, we can find $n>N$ such that $|x_n-L|<\varepsilon$.
\end{theorem}
\begin{theorem}
    \B{Bolzano Weierstrass Theorem}. If $(x_n)$ is a bounded sequence then it has a limit point, that is, it has a convergent subsequence.
\end{theorem}
\section{Series and Decimals}
\begin{definition}
    Given a sequence $(a_n)$, we say that the series $\sum a_n$ converges if the sequence of partial sums
    \begin{align*}
        s_N=\sum_{n=1}^N a_n
    \end{align*}
    converges as $N\to\infty$. We denote its limit by $\sum_{n=1}^\infty a_n$.
\end{definition}
\begin{theorem}
    Let $a_n$ and $b_n$ be sequences. If $\sum a_n$ and $\sum b_n$ are convergent, then so is $\sum (a_n+b_n)$, and
    \begin{align*}
        \sum_{n=1}^\infty (a_n+b_n) = \sum_{n=1}^\infty a_n+ \sum_{n=1}^\infty b_n.
    \end{align*}
    If $c\in\R$, then $\sum ca_n$ is convergent and
    \begin{align*}
        \sum_{n=1}^\infty ca_n = c\sum_{n=1}^\infty a_n.
    \end{align*}
\end{theorem}
\begin{theorem}
    \B{Comparison Test.} If $\sum b_n$ is convergent and $|a_n|\leq b_n$ for all $n$, then $\sum a_n$ is convergent.
\end{theorem}
\begin{theorem}
    Let $x\in\R$. Then there is a sequence of integers $a_0\in\mathbb{Z}$ and $a_n\in\{0,1,...,9\}$ for $n\geq 1$ so that
    \begin{align*}
        \sum_{n=0}^\infty \frac{a_n}{10_n}=a_0+\frac{a_1}{10}+\cdots = x.
    \end{align*}
\end{theorem}
\begin{theorem}
    If $x\geq 0$ is rational, then it has a periodic decimal expansion, that is,
    \begin{align*}
        x=a_0.a_1a_2\cdots a_k\lneg{b_1b_2\cdots b_l}.
    \end{align*}
\end{theorem}
\begin{definition}
    \B{Pigeonhole Principle}.\\
    Suppose $k>n$, and we have $a_1,...,a_k\in S$, with $|S|=n$. Then there exists $i\not=j$ with $a_i=a_j$.
\end{definition}
\begin{theorem}
    The period of the decimal expansion of $\frac{p}{q}$ is of length at most $q$.
\end{theorem}
\begin{theorem}
    If $x\geq 0$ has periodic decimal expansion, then $x$ is rational.
\end{theorem}
\begin{theorem}
    A number $x$ is irrational iff its decimal expansion is aperiodic.
\end{theorem}
\begin{theorem}
    Let $x\geq 0$ and suppose $x=a_0.a_1a_2...=0.b_1b_2....$ Let $l$ be the first integer for which $a_l\not=b_l$. Then $b_k=0$ and $a_k=9$ for a $k>l$ and $b_l=a_l+1$.
\end{theorem}
\begin{theorem}
    A number $x$ is irrational if, for all $c>0$, there is a rational number $\frac{p}{q}\not=x$ so that
    \begin{align*}
        \left|x-\frac{p}{q}\right|<\frac{c}{q}.
    \end{align*}
\end{theorem}
\begin{theorem}
    Suppose $x=\lim_{n\to\infty}\frac{p_n}{q_n}$ and $\left|x-\frac{p_n}{q_n}\right|q_n \to 0$, then $x$ is irrational.
\end{theorem}
\begin{theorem}
    The number $e$ is irrational.
\end{theorem}
\section{Complex Numbers}
\begin{definition}
    \B{Complex Numbers.} Define $i$ to be a number such that $i^2 = -1$. 
    The \B{complex numbers} are any number of the form
    \begin{align*}
        z = x+iy \text{ where } x,y\in\R.
    \end{align*}
    The set of all complex numbers is denoted $\mathbb{C}$. We define $\Re(z) = x$ and $\Im(z)=y$ to be the \B{real} and \B{imaginary parts} of $z$.
\end{definition}
\begin{definition}
    \B{Modulus}. Given a complex number $z=x+iy$, the \B{modulus} of $z$ is
    \begin{align*}
        |z|=|x+iy|=\sqrt{x^2+y^2}.
    \end{align*}
\end{definition}
\begin{definition}
    \B{Complex Conjugate}. Given a complex number $z=x+iy$, its \B{complex conjugate} is defined to be
    \begin{align*}
        \bar{z}=x-yi.
    \end{align*}
\end{definition}
\begin{theorem}
    For all $u,v\in\C$:
    \begin{itemize}
        \item $\lneg{u+v}=\lneg{u}+\lneg{v}$.
        \item $\lneg{uv}=\lneg{u}\cdot\lneg{v}$.
        \item $|uv|=|u|\cdot|v|$.
    \end{itemize}
\end{theorem}
\begin{definition}
    Let $z\not=0$ be a complex number. 
    Let $r=|z|$ and let the \B{argument} of $z$ be the angle $\theta\in[0,2\pi)$ between the line from $0$ to $z$ and the positive $x$-axis. We can then write
    \begin{align*}
        z = r(\cos\theta+i\sin\theta)=re^{i\theta}.
    \end{align*}
    These are the \B{polar form} and \B{exponential form} of $z$.
\end{definition}
\begin{theorem}
    \B{De Moivre's Theorem}. Let $z=re^{i\theta}$ and $w=se^{i\phi}$ then $zw=rse^{i(\theta+\phi)}$.
\end{theorem}
\begin{theorem}
    \B{A very special case of DM's Theorem}. If we let $z=re^{i\theta}$, and $n\in\N$ then
    \begin{align*}
        z^n = r^ne^{in\theta},\\
        z^{-n} = r^{-n}e^{-in\theta}.
    \end{align*}
\end{theorem}
\begin{theorem}
    \B{Roots of Unity}. The solutions to $z^n=1$ are $1,w,...,w^{n-1}$ where $w=e^{\frac{2\pi i}{n}}$. That is, they are $e^{\frac{2\pi ki}{n}}$ for $k=0,1,...,n-1$.
\end{theorem}
\section{Polynomials}
\begin{definition}
    For $n\in\N$, an \B{$n$-degree complex polynomial} is a function of the form
    \begin{align*}
        p(z)=a_nz^n+a_{n-1}z^{n-1}+\cdots+a_0
    \end{align*}
    where $a_n\not=0$ and $a_i\in\C$ for all $i$. We say that $\alpha$ is a \B{root} of $p(z)$ if $p(\alpha)=0$, in other words, if $\alpha$ is a solution to the polynomial equation $p(z)=0$.
\end{definition}
\begin{theorem}
    \B{Abel-Ruffini Theorem}. There is no formula for the roots of a polynomial of degree $\geq 5$.
\end{theorem}
\begin{theorem}
    \B{Fundamental Theorem of Algebra}. Any complex polynomial has at least one root in $\C$.
\end{theorem}
\begin{theorem}
    \B{Factorization Theorem}. If $p$ is a degree $n$ polynomial, then there are $n$ roots $r_1,...,r_n\in\C$ and a number of $a\in\C$ so that
    \begin{align*}
        p(z)=a(z-r_1)(z-r_2)\cdots(z-r_n).
    \end{align*}
    Some roots may repeat. If a root appears $m$ times in $r_1, ..., r_n$ it has a \B{multiplicty} m.
\end{theorem}
\begin{theorem}
    \B{Real Polynomials hava conjugate roots}. If $p(x)$ has real coefficients and $r$ is a root, so is $\bar{r}$.
\end{theorem}
\begin{theorem}
    \B{Root-Coefficient Theorem}. If $p(x)=x^n+a_{n-1}x^{n-1}+\cdots+a_1x+a_0$, has roots $r_1, ..., r_n$, then, if $s_j$ denotes the sum of all products of $j$-tuples of the roots,
    \begin{align*}
        s_j=(-1)^ja_{n-j}.
    \end{align*}
\end{theorem}
\section{The Integers}
\begin{theorem}
    \textbf{The Remainder Theorem.} Let $a\in\N$ and $b\in\mathbb{Z}$.
    There are unique integers $q\in\mathbb{Z}$ and $0\leq r<a$ s.t.
    \begin{align*}
        b=qa+r.
    \end{align*}
\end{theorem}
\begin{definition}
    For two integers $a$ and $b$ we say $a$ \textbf{divides} $b$, or write $a|b$, if there is an integer $c$ so that $b=ac$.
\end{definition}
\begin{lemma}
    If $a|b$ and $b|a$, then $a=\pm b$.
\end{lemma}
\begin{lemma}
    \textbf{The Easy Lemma.} Let $a$ and $b$ be integers and suppose $d|a$ and $d|b$. Then $d|ma+nb$ for all $m,n\in\N$.
\end{lemma}
\begin{definition}
    Given two integers $a,b\in\mathbb{Z}$ that are not both zero, the \textbf{highest common factor} of $a$ and $b$ is the largest positive integer $d\in\N$ that divides both $a$ and $b$.
    We denote this integer $d$ by $\hcf(a,b)$. If $\hcf(a,b)=1$, then we say that $a$ and $b$ are \textbf{coprime}.
\end{definition}
\begin{theorem}
    \textbf{Bezout's Identity.} Let $a$ and $b$ be non-zero integers. Then there are integers $s$ and $t$ so that $\hcf(a,b)=sa+tb$.
\end{theorem}
\begin{corollary}
    Let $a,b,c\in\Z$. Suppose $c\not=0$, $c|ab$ and $\hcf(a,c)=1$. Then $c|b$.
\end{corollary}
\begin{corollary}
    Let $a,c\in\Z$ and let $d\in\Z$ be so that $d|a$ and $d|c$. Then $d|\hcf(a,c)$.
\end{corollary}
\begin{corollary}
    Let $d$ be a common divisor of $a$ and $c$, which is divisible by all divisors of $a$ and $c$. Then $d=\pm\hcf(a,c)$.
\end{corollary}
\begin{corollary}
    If $a,b\in\Z$, $p$ is prime, and $p|ab$, then either $p|a$ or $p|b$ (or both).
\end{corollary}
\begin{corollary}
    If $n=p_1\cdots p_k$ is a product of primes, and if a prime $p$ divides $n$, then $p=p_i$ for some $i=1,...,k$.
\end{corollary}
\begin{theorem}
    \textbf{The Fundamental Theorem of Arithmetic (FTA).} Let $n\geq 2$ be an integer.
    \begin{itemize}
        \item (Existence) Then $n$ is equal to a product $n=p_1^{r_1}\cdots p_k^{r_k}$ of powers of prime numbers where $p_1<...<p_k$ and $r_i>0$ for all $i$.
        \item (Uniqueness) The factorization is unique: If we also have
        \begin{align*}
            p_1^{r_1}\cdots p_k^{r_k}=n=q_1^{s_1}\cdots q_l^{s_l}
        \end{align*}
        where $q_1<\cdots <q_l$ are primes and $s_i>0$, then $k=l$, $p_i=q_i$, and $r_i=s_i$ for all $i$.
    \end{itemize}
\end{theorem}
\begin{lemma}
    \textbf{Existence of prime decomposition, Part I.} Every integer $n\geq 2$ can be written as a product of primes $n=p_1\cdots p_k$.
\end{lemma}
\begin{lemma}
    \textbf{Existence of prime decomposition, Part II.} Every integer $n\geq 2$ can be written as a product of powers of primes $n=p_1^{r_1}\cdots p_k^{r_k}$ and $r_i\geq 0$ for all $i$.
\end{lemma}
\begin{lemma}
    \textbf{Uniqueness of prime decomposition.} Every integer $n$ can be written as a unique product of powers of primes $n=p_1^{r_1}\cdots p_k^{r_k}$.
\end{lemma}
\begin{theorem}
    Let $n=p_1^{a_1}\cdots p_k^{a_k}$ be a prime decomposition (i.e. $p_i$s are prime, $p_1<\cdots<p_k$, and $a_i>0$). Then $m$ divides $n$ iff 
    \begin{align*}
        m=p_1^{b_1}\cdots p_k^{b_k}, \text{ with each }0\leq b_i \leq a.
    \end{align*}
\end{theorem}
\begin{definition}
    The least common multiple $\lcm(a,b)$ of positive integers $a$ and $b$ is the smalles positive integer divisible by both $a$ and $b$.
\end{definition}
\begin{theorem}
    Let $a$ and $b$ have prime factorizations,
    \begin{align*}
        a=p_1^{r_1}\cdots p_m^{r_m}, \: b=p_1^{s_1}\cdots p_m^{s_m}.
    \end{align*}
    Here $p_i$'s are distinct, but $r_i$ and $s_i$ are allowed to be zero. Then:
    \begin{itemize}
        \item $\hcf(a,b) = p_1^{\min(r_1,s_1)}\cdots p_m^{\min(r_m,s_m)}$.
        \item $\lcm(a,b) = p_1^{\max(r_1,s_1)}\cdots p_m^{\max(r_m,s_m)}$.
        \item $\lcm(a,b) = ab / \hcf(a,b)$.
    \end{itemize}
\end{theorem}
\begin{theorem}
    Let $n$ be a positive integer. Then $\sqrt{n}\in\mathbb{Q}$ iff $n$ is a perfect square.
\end{theorem}
\begin{theorem}
      If $a,b\in\N$ are coprime, and $ab$ is an $n$th power, then so are $a$ and $b$.
\end{theorem}
\section{Modular Arithmetic}
\begin{definition}
    For $m\in\N$ and $a,b\in Z$, we write $a\equiv b \mod m$ if one of the following holds
    \begin{itemize}
        \item $m|(a-b)$,
        \item $a$ and $b$ have same remainder when divided by $m$,
        \item $b=qm+a$ for some $q\in\Z$.
    \end{itemize}
\end{definition}
\begin{theorem}
    We have the following properties for integers $a,b,c\in\Z$ and $m\in\N$:
    \begin{enumerate}
        \item $a\equiv a \mod m$.
        \item If $a\equiv b \mod m$, then $b\equiv a \mod m$.
        \item If $a\equiv b \mod m$ and $b\equiv c \mod m$, then $a\equiv c \mod m$.
    \end{enumerate}
\end{theorem}
\begin{theorem}
    Suppose $a\equiv x \mod m$ and $b\equiv y \mod m$. Then
    \begin{enumerate}
        \item $a+b\equiv x+y \mod m$,
        \item $ab \equiv xy \mod m$,
        \item $a^k \equiv x^k \mod m$ for $k\in \N$.
    \end{enumerate}
\end{theorem}
\begin{proposition}
    Let $a$ and $m$ be coprime. If $x,y\in \Z$ and $xa\equiv ya \mod m$, then $x\equiv y \mod m$. In particular, if $p$ is prime, $p\not|a$, and $xa\equiv ya \mod p$, then $x\equiv y \mod p$. 
\end{proposition}
\begin{theorem}
    The equation $ax\equiv b \mod m$ has a solution iff $hcf(a,m)|b$.
\end{theorem}
\begin{theorem}
    \textbf{Fermat's Little Theorem (FLT).} If $p$ is prime and $p\not|a\in\Z$, then 
    \begin{align*}
        a^{p-1}\equiv 1 \mod p.
    \end{align*}
\end{theorem}
\begin{theorem}
    Let $n\in\N$ be a prime number. Assume $n$ and $p-1$ are coprime and $p\not|b$. Then the equation
    \begin{align*}
        x^n\equiv b \mod p
    \end{align*}
    has exactly one solution $x\in\{0, 1, ..., p-1\}$.
\end{theorem}
\section{More set theory and Equivalence relations}
\begin{definition}
    Let $A$ and $B$ be two sets.
    \begin{enumerate}
        \item We define the \textbf{union} of $A$ and $B$ to be the set of elements that are either in $A$ or in $B$. We write $A\cup B$.
        \item We define the \textbf{intersection} of $A$ and $B$ to be the set of elements that are in $A$ and $B$. We write $A\cap B$.
        \item $A$ \textbf{minus} $B$ is the set of things in $A$ that are not in $B$. We write $A-B$ or $A\setminus B$.
        \item If $A\cap B=\emptyset$, we say $A$ and $B$ are \textbf{disjoint}.
    \end{enumerate}
\end{definition}
\begin{proposition}
    Let $A,B,C$ be sets. Then
    \begin{align*}
        A\cap(B\cup C) = (A\cap B)\cup(A\cap C).
    \end{align*}
\end{proposition}
\begin{definition}
    A \textbf{partition} of a set $S$ is a collection of nonempty subsets of $S$ so that for every $x\in S$, there is exactly one set $A$ in the collection so that $x\in A$.
\end{definition}
\begin{definition}
    The \textbf{Cartesian} product of two sets $A$ and $B$ is the set of ordered pairs
    \begin{align*}
        A\times B = \{(x,y) | x\in A,\: y\in B\}.
    \end{align*}
\end{definition}
\begin{definition}
    A \textbf{relation} $R$ on a set $S$ is just a subset of $S\times S$. Given $a,b\in S$, we write $a\sim b$ if $(a,b)\in R$.
\end{definition}
\begin{definition}
    Let $S$ be a set. Let $\sim$ be a relation on $S$. We say $\sim$ is
    \begin{itemize}
        \item \textbf{reflexive} if $a\sim a$ for all $a\in S$,
        \item \textbf{symmetric} if for all $a,b\in S$, we have $a\sim b \Leftrightarrow b\sim a$, and
        \item \textbf{transitive} if for all $a,b,c\in S$, we have
        \begin{align*}
            a\sim b \sim c \Rightarrow a \sim c.
        \end{align*}
    \end{itemize}
    If $\sim$ satisfies all three of these properties, we say $\sim$ is an \textbf{equivalence relation}.
\end{definition}
\begin{definition}
    If $\sim$ is an equivalence relation on a set $S$ and $x\in S$, the \textbf{equivalence class} of $x$ is
    \begin{align*}
        \cl(x) = \{y\in S | x\sim y\}.
    \end{align*}
    The set of \textbf{equivalence classes} is the set $\{\cl(x) | x \in S\}$.
\end{definition}
\begin{proposition}
    If $\sim$ is an equivalence relation on a set $S$, then the equivalence classes form a partition of $S$. 
\end{proposition}
\section{Functions}
\begin{definition}
    Given two sets $X$ and $Y$, a \textbf{function from $X$ to $Y$}, denoted $f:X\to Y$, is a relation pairing each $x\in X$ with an element in $Y$ which we denote $f(x)$. The set $X$ is called the \textbf{domain} of $f$ and $Y$ is the \textbf{codomain} or \textbf{range} of $Y$.
\end{definition}
\begin{definition}
    Let $f:X\to Y$ be a function. We say
    \begin{itemize}
        \item $f$ is \textbf{surjective} if
        \begin{align*}
            \forall y \in Y,\: \exists x \in X \st f(x) = y,
        \end{align*}
        \item $f$ is \textbf{injective} if
        \begin{align*}
            \forall x,y \in X,\: f(x)=f(y) \Rightarrow x = y,
        \end{align*}
        \item $f$ is \textbf{bijective} if it is injective and surjective.
    \end{itemize}
\end{definition}
\begin{theorem}
    Let $X$ and $Y$ be finite sets and let $f:X\to Y$ be a function.
    \begin{enumerate}
        \item If $f$ is injective, then $|X|\leq |Y|$.
        \item If $f$ is surjective, then $|X|\geq |Y|$.
        \item If $f$ is bijective, then $|X|=|Y|$.
    \end{enumerate}
\end{theorem}
\begin{definition}
    Let $f:X\to Y$ and $g:Y\to Z$ be functions. We define the \textbf{composition of $f$ and $g$}, denoted $g\circ f:X\to Z$, to be the function
    \begin{align*}
        g\circ f(x) = g(f(x)).
    \end{align*}
\end{definition}
\begin{theorem}
    Let $f:X\to Y$ and $g:Y\to Z$ be functions.
    \begin{enumerate}
        \item If $f$ and $g$ are injective, so is $g\circ f$.
        \item If $f$ and $g$ are surjective, so is $g\circ f$.
        \item If $f$ and $g$ are bijective, so is $g\circ f$.
    \end{enumerate}
\end{theorem}
\begin{definition}
    If $f:X\to Y$ is bijective, the \textbf{inverse} of $f$, denoted $f^{-1}:Y\to X$, is the function such that
    \begin{align*}
        f^{-1}(y) = x \Leftrightarrow f(x) = y.
    \end{align*}
\end{definition}
\begin{theorem}
    Let $f:X\to Y$ and $g:Y\to X$ be functions. Then $f$ is bijective and $g=f^{-1}$ iff $f(g(y))= y$ for all $y\in Y$ and $g(f(x))=x$ for all $x\in X$.
\end{theorem}
\begin{definition}
    If $f:X\to Y$ and $A\subseteq X$, the \textbf{image of $A$} under $f$ is
    \begin{align*}
        f(A) = \{f(x):x\in A\}\subseteq Y.
    \end{align*}
    If $B\subseteq Y$, the \textbf{preimage of $B$} under $f$ is
    \begin{align*}
        f^{-1}(B) = \{x\in X:f(x)\in B\}\subseteq X.
    \end{align*}
\end{definition}
\section*{Applications: What's bigger than $\infty$?}
\begin{definition}
    We say two sets $A$ and $B$ have the same \textbf{cardinality} if there is a bijective function $f:A\to B$, and we write $|A|=|B$ or $A\sim B$. If there is an injective map $f:A\to B$, we write $|A|\leq |B|$, and if $|A|\leq |B|$ and there is no injective map from $B$ to $A$, we write $|A|<|B|$.\\
    When $A\sim\N$, we say $A$ is \textbf{countable}.
\end{definition}
\begin{theorem}
    $\mathbb{Q}\sim\N$.
\end{theorem}
\begin{theorem}
    Let $S$ be an infinite set and let $P(S)$ denote the \textbf{power set} consisting of the subsets of $S$, i.e. $P(S) = \{A\subseteq S\}$. Then $|S|<|P(S)|$.
\end{theorem}
\section{Counting}
\begin{theorem}
    \textbf{Multiplication Principle.} Let $P$ be a process consisting of $n$ stages and at each stage $i$, there are $a_i$ many choices we can make, and no two distinct choices can result in the same outcome. 
    Then after $n$ stages, the total number of outcomes is $a_1\cdot a_2\cdots a_n$.
\end{theorem}
\begin{proposition}
    Let $S$ have $n$ elements. Then the number of subsets is $2^n$.    
\end{proposition}
\begin{definition}
    Let $S$ be a set of $n$ objects. An \textbf{ordering} of $S$ is a sequence $a_1,...,a_n$ for which each element of $S$ appears exactly once in this sequence.
\end{definition}
\begin{theorem}
    \textbf{Rearrangement Theorem.} Given a set $S$ of $n$ distinct objects, there are $n!$ ways of ordering them.
\end{theorem}
\begin{proposition}
    Let $S$ be a set of $n$ elmeents and $0\leq k \leq n$.    
    \begin{enumerate}
        \item The number of ways of picking an ordered selection of $k$ elements from $S$ (i.e. an ordered list of elements $a_1, ..., a_k\in S$) is $n^k$.
        \item The number of ways of picking an ordered selection of $k$ distinct elements from $S$ is $\frac{n!}{(n-k)!}$.
    \end{enumerate}
\end{proposition}
\begin{definition}
    Given a set $S$ of size $n$ and an integer $0\leq k \leq n$, we let
    \begin{align*}
        \begin{pmatrix}
            n\\k
        \end{pmatrix}
    \end{align*}
    denote the number of subsets of $S$ of size $k$. If $k>n$, we just set $\begin{pmatrix}
        n\\k
    \end{pmatrix}=0$.
\end{definition}
\begin{theorem}
    For integers $0\leq k \leq n$, we have
    \begin{align*}
        \binco{n}{k}=\frac{n!}{k!(n-k)!}.
    \end{align*}
\end{theorem}
\begin{corollary}
    For $n\geq 0$, we have
    \begin{align*}
        2^n = \sum_{k=0}^n \binco{n}{k}.
    \end{align*}
\end{corollary}
\begin{definition}
    Let $S$ be a nonempty set.
    \begin{itemize}
        \item A \textbf{partition} of $S$ is a set $\{A_1,...,A_k\}$ of nonempty subsets of $S$ so that every $x\in\S$ is in exactly one $A_i$.
        \item An \textbf{ordered partition} of $S$ is a sequence of sets $A_1,...,A_k$ that partition $S$.
    \end{itemize}
\end{definition}
\begin{definition}
    Given nonnegative integers $r_1, ..., r_k$ such that $r_1+\cdots + r_k=n$, we denote the number of \textbf{ordered} partitions $A_1,..., A_n$ of a set $S$ such that $|A_i=r_i$ as
    \begin{align*}
        \binco{n}{r_1,...,r_k}.
    \end{align*}
\end{definition}
\begin{theorem}
    Given nonnegative integers $r_1, ..., r_k$ such that $r_1+\cdots r_k=n$ and $k\geq 2$, 
    \begin{align*}
        \binco{n}{r_1, ..., r_k}=\frac{n!}{r_1!r_2!\cdots r_k!}.
    \end{align*}
\end{theorem}
\begin{theorem}
    \textbf{The Binomial Theorem.} Let $n\in\N$ and $a,b\in\C$. Then
    \begin{align*}
        (a+b)^n=\sum_{k=0}^n \binco{n}{k}a^kb^{n-k}.
    \end{align*}
\end{theorem}
\begin{theorem}
    \textbf{The Multinomial Theorem.} Let $n\in\N$ and $x_1, ..., x_k\in\C$. Then $(x_1+\cdots+x_k)^n$ is equal to the sum over all terms of the form
    \begin{align*}
        \binco{n}{r_1,...,r_k}x_1^{r_1}\cdots x_k^{r_k}
    \end{align*}
    where $r_i\in\{0,1,...,n\}$ and $r_1+\cdots r_k=n$.
\end{theorem}
\begin{theorem}
    \textbf{Inclusion-Exclusion Principle.} Let $n\geq 2$ and suppose $A_1, A_2, ..., A_n$ are sets. Let
    \begin{align*}
        c_1=|A_1|+|A_2|+\cdots,\:c_2=|A_1\cap A_2|+|A_1\cap A_3|+|A_2\cap A_3|+\cdots,\:\cdots
    \end{align*}
    and similarly, let $c_k$ be the sum of the size of the intersections every subcollection of $k$ different $A_i$'s so $c_n=|\bigcap_{k=1}^n A_k|$. Then
    \begin{align*}
        |A_1\cup \cdots \cup A_n| = \sum_{k=1}^n c_k (-1)^{k+1}=c_1-c_2+\cdots +c_k (-1)^{n+1}.
    \end{align*}
\end{theorem}
\section{Permutations}
\begin{definition}
    Given $n\in\N$, denote by $S_n$ the set of all bijections
    \begin{align*}
        \{1,2,3,4,...,n\}\to\{1,2,3,4,...,n\}.
    \end{align*}
    We call elements in $S_n$ \textbf{permutations} of the set $\{1,2,3,4,...,n\}$.
\end{definition}
\begin{lemma}
    The set $S-N$ consists of exactly $n!$ permutations.
\end{lemma}
\begin{lemma}
    The set $S_n$ equipped with the composition rule $\circ$ has the following properties:
    \begin{itemize}
        \item for any $f,g\in S_n$, $f\circ g\in S_n$,
        \item for any $f,g,h\in S_n$, one has
        \begin{align*}
            f\circ \left(g\circ h\right)=\left(f\circ g\right)\circ h,
        \end{align*}
        \item there is a unique permutation $\iota \in S_n$ such that for all $f\in S_n$
        \begin{align*}
            f\circ \iota = \iota \circ f = f,
        \end{align*}
        \item for any $f\in S_n$, there is a unique $f^{-1}\in S_n$ such that
        \begin{align*}
            f\circ f^{-1}=f^{-1}\circ f = \iota.
        \end{align*}
    \end{itemize}
\end{lemma}
\begin{definition}
    Let $\{a_1, a_2, ...,a_r\}$ be a non-empty subset of the set $\{1,2,...,n\}$. The permutation of $f\in S_n$ such that
    \begin{align*}
        f(a_1)=a_2,\: f(a_2)=a_3,...,\: f(a_{r-1})=a_r,\: f(a_r)=a_1
    \end{align*}
    and 
    \begin{align*}
        f(k)=k \Leftrightarrow k\not\in \{a_1,a_2,...,a_r\}
    \end{align*}
    is denoted by
    \begin{align*}
        (a_1\:a_2\:\cdots\:a_r)
    \end{align*}
    and is called a \textbf{cycle} or \textbf{cyclic permutation}.
\end{definition}
\begin{corollary}
    If $a_1, ..., a_r$ are distinct numbers in $\{1,2,...,n\}$ and $f=(a_1\: a_2\: \cdots \: a_r)$, then $r$ is the smalles positive integer such that $f^r=\iota$.
\end{corollary}
\begin{definition}
    If $f=(a_1\:a_2\:\cdots\: a_r)$ and $g=(b_1\:b_2\:\cdots\:b_2)$ where $\{a_1,...,a_r\}$ and $\{b_1,..., b_n\}$ are disjoint sets, we say that $f$ and $g$ are \textbf{disjoint} in $S_n$.
\end{definition}
\begin{proposition}
    If $f$ and $g$ are disjoint cycles in $S_n$, then $fg=gf$.    
\end{proposition}
\begin{theorem}
    Any permutation $f\in S_n$ is a composition of disjoint cycles, that is, $f=f_1\cdots f_k$ where $f_1,...,f_k$ are disjoint cycles.
\end{theorem}
\begin{definition}
    Let $f$ be a permutation in $S_n$. The smallest positive integer $m$ such that
    \begin{align*}
        f^m = \iota
    \end{align*}
    is called the \textbf{order} of the permutation $f$.
\end{definition}
\begin{lemma}
    Let $f\in S_n$ have order $m$. If $f^k=\iota$ for some integer $k\in\N$, then $m|k$.
\end{lemma}
\begin{lemma}
    Let $f\in S_n$ and let $m$ be its order. Write
    \begin{align*}
        f=\sigma_1\sigma_2\cdots \sigma_s
    \end{align*}
    where $\sigma_1,...,\sigma_s$ are disjoint cycles of lengths $r_1, ..., r_s$ respectively, Then
    \begin{align*}
        m=\lcm(r_1,r_2,...,r_s).
    \end{align*}
\end{lemma}
\begin{lemma}
    Every permutation in $S_n$ is a product of cycles of length $2$.
\end{lemma}
\begin{definition}
    We say a permutation $f\in S_n$ is \textbf{even} if $\sgn(f)=1$ and \textbf{odd} if $\sgn(f) = -1$.
\end{definition}
\begin{lemma}
    Let $n\in\N$.
    \begin{enumerate}
        \item The signature of $\iota$ is $1$ and any $2$-cycle is $-1$.
        \item For any $f,g\in S_n$, we have $\sgn(fg)=\sgn(f)\sgn(g)$.
        \item The signature of a cycle of length $r$ is $(-1)^{r-1}$.
        \item For every permutation $f\in S_n$, let $f=\sigma_1\sigma_2\cdots \sigma_s$ where $\sigma_1,...,\sigma_s$ are disjoint cycles of lengths $r_1,...,r_s$. Then
        \begin{align*}
            \sgn(f)=(-1)^{r_1-1}(-1)^{r_2-1}(-1)^{r_3-1}\cdots(-1)^{r_s-1}.
        \end{align*}
        \item If $f\in S_n$, then $\sgn(f)=\sgn(f^{-1})$.
    \end{enumerate}
\end{lemma}
\begin{corollary}
    Let $f$ be a permutation in $S_n$.
    \begin{enumerate}
        \item $f$ is even iff it is a product of even numbers of cycles of order $2$.
        \item $f$ is odd iff it is a product of odd numbers of cycles of order $2$.
    \end{enumerate}
\end{corollary}
\end{document}
