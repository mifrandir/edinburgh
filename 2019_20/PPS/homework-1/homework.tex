\documentclass{article}
\usepackage{homework-preamble}

\title{PPS: Homework 4 (Workshop 33)}
\author{Franz Miltz (UUN: S1971811)}
\date{10 February 2020}

\begin{document}
\maketitle
\section*{Question 1}
Claim: Let $P$ and $Q$ be statements. Then $P\Rightarrow Q\equiv \bar{Q}\Rightarrow \bar{P}$. (Where $\equiv$ means equivalent.)\\
Proof:
\begin{align*}
	P\Rightarrow Q\equiv\overline{P}\vee Q\equiv \overline{\overline{Q}}\vee \overline{P}\equiv\overline{Q}\Rightarrow\overline{P}.\:\square
\end{align*}
\section*{Question 2}
Claim: Let $n$ be an integer such that $n^2$ is even. Then $n$ is even.\\
Proof: We shall assume the opposite and proof by contradiction.\\
Assume $n$ is odd while $n^2$ is even. Then
\begin{align*}
	n=2k+1 \text{ for some } n\in\mathbb{N}.
\end{align*}
Thus
\begin{align*}
	n^2=(2k+1)^2=4k^2+4k+1=2(2k^2+2k)+1.
\end{align*}
With $l=2k^2+2k$, $n^2=2l+1$, which is odd and therefore contradicting the premise.\\
We conclude, that $n$ cannot be odd and therefore has to be even. $\square$
\section*{Question 3}
Claim: If $a+b\sqrt{2}=c+d\sqrt{2}$ for some $a,b,c,d\in\mathbb{Q}$, then $a=c$ and $b=d$.\\
Proof:
\begin{align*}
	a+b\sqrt{2}         & =c+d\sqrt{2}         \\
	\Leftrightarrow a-c & =d\sqrt{2}-b\sqrt{2} \\
	\Leftrightarrow a-c & =\sqrt{2}(d-b)
\end{align*}
By \emph{Liebeck, Proposition 2.4}, we know that $\sqrt{2}(d-b)$ is irrational if $d-b\not=0$. Since $a-c$ is rational, $\sqrt{2}(d-b)$ has to be as well. Thus
\begin{align*}
	d-b=0 \Rightarrow a-c=0
\end{align*}
Therefore
\begin{align*}
	a=c \text{ and } d=b.\:\square
\end{align*}
\end{document}