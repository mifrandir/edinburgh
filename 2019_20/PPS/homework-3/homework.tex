\documentclass{article}
\usepackage{homework-preamble}

\title{PPS: Homework 2 (Workshop 33)}
\author{Franz Miltz (UNN: S1971811)}
\date{24 February 2020}

\begin{document}
\maketitle
\section*{Task 1}
\subsection*{Claim}
\begin{align*}
  \lim_{n\to\infty}\frac{n^2-5}{n^2+n+1}=1
\end{align*}
\subsection*{Proof}
The claim is by definition equivalent to
\begin{align*}
  \forall \varepsilon > 0, \exists N>0 \st \forall n>N, \left|\frac{n^2-5}{n^2+n+1}-1\right|<\varepsilon.
\end{align*}
Observe that, with $n>0$,
\begin{align*}
  \varepsilon > \left|\frac{n^2-5}{n^2+n+1}-1\right|
  =  \left|\frac{-n-6}{n^2+n+1}\right|
  =  \frac{n+6}{n^2+n+1}.
\end{align*}
With $n\geq 1$, we also get
\begin{align*}
  \frac{n+6}{n^2+n+1}<\frac{n+6}{n^2+n+1}<\frac{7n}{n^2+n+1}<\frac{7}{n+1}<\frac{7}{n}.
\end{align*}
Now, let $N=\frac{7}{\varepsilon}$ and $n>N$. Then
\begin{align*}
  \frac{7}{n}<\frac{7}{N}=\varepsilon.
\end{align*}
Thus, if $N = \max\{1, \frac{7}{\varepsilon}\}$,
\begin{align*}
  \forall n > N, \left|\frac{n^2-5}{n^2+n+1}-1\right|<\varepsilon.\:\square  
\end{align*}
NOTE: Since the task does not specify anything about $n$, I assumed it to be any real number. If $n\in\N$, $N=\frac{7}{\varepsilon}$ is sufficient.
\section*{Task 2}
\subsection*{Claim}
If $x_n\to\infty$ and $x_n>0$ for all $n\in\N$, then $\frac{1}{x_n}\to 0$.
\subsection*{Proof}
By definition of $x_n\to\infty$ we know
\begin{align*}
  \forall M > 0, \exists N> 0 \st \forall n > N, x_n > M.
\end{align*}
We need to prove
\begin{align*}
  \forall \varepsilon > 0, \exists N > 0\st \forall n > N, \left|\frac{1}{x_n}\right|<\varepsilon.
\end{align*}
Let $\varepsilon > 0$. Then
\begin{align*}
  \varepsilon > \left|\frac{1}{x_n}\right| = \frac{1}{x_n} \Leftrightarrow x_n > \frac{1}{\varepsilon}.
\end{align*}
With $x_n\to \infty$ and $M=\frac{1}{\varepsilon}$ it follows that
\begin{align*}
  \exists N > 0 \st \forall n > N, x_n > \frac{1}{\varepsilon}
\end{align*}
and thus
\begin{align*}
  \exists N > 0 \st \forall n > N, \left|\frac{1}{x_n}\right|>\varepsilon,
\end{align*}
which is equivalent to $\frac{1}{x_n}\to 0$. $\square$
\section*{Task 3}
\subsection*{Claim}
Let $A\subseteq R$ be bounded and let $B=\{\frac{x+y}{2}\:|\:x,y\in A\}$. Then $\lub(A)=\lub(B)$.
\subsection*{Proof}
At first, we show that $\lub(A)$ is an upper bound for $B$:\\
By definition $\forall x \in A, x \leq \lub(A)$. Additionally
\begin{align*}
  x,y\leq \lub(A) \Rightarrow x + y \leq 2\lub(A).
\end{align*}
With $x,y\in A$, we get
\begin{align*}
  x+y&\leq 2\lub(A)\\
  \Leftrightarrow\frac{x+y}{2}&\leq \lub(A)
\end{align*}
and thus $\lub(A)$ is an upper bound for $B$.\\
Now we use the fact, that $B\subseteq \R$ is bounded above by $\lub(A)$ and apply the \emph{Completeness Axiom (Liebeck, 201)}. Thus $B$ has a least upper bound $\lub(B)$. Therefore
\begin{align*}
  \forall x \in B, x \leq \lub(B)\\
  \Leftrightarrow \forall x, y \in A, \frac{x+y}{2}\leq \lub(B).
\end{align*}
Thus
\begin{align*}
  \forall x \in A, \frac{x+x}{2}=x \leq \lub(B),
\end{align*}
making $\lub(B)$ an upper bound for $A$.\\
Since $\lub(A)$ is an upper bound of $B$ and $\lub(B)$ is an upper bound of $A$, it follows that $\lub(A)=\lub(B)$. $\square$ 
\end{document}