\documentclass{article}
\usepackage[a4paper]{geometry}
\usepackage{babel}
\usepackage{xcolor}
\usepackage[colorlinks=false]{hyperref}
\hypersetup{
    colorlinks = false,
    linkbordercolor = {white}
}
\usepackage{amsmath}
\usepackage{listings}
\usepackage{amssymb}
\usepackage{amsthm}
\usepackage[framemethod=TikZ]{mdframed}
\usepackage{proof}
\setlength{\inferLineSkip}{4pt}
\newcounter{example}[section]\setcounter{example}{0}
\newenvironment{example}{
\noindent\refstepcounter{example}
\paragraph*{Example}}{}
\newtheoremstyle{sltheorem} {}                % Space above
{}                % Space below
{\upshape}        % Theorem body font % (default is "\upshape")
{}                % Indent amount
{\bfseries}       % Theorem head font % (default is \mdseries)
{.}               % Punctuation after theorem head % default: no punctuation
{ }               % Space after theorem head
{}                % Theorem head spec
\theoremstyle{sltheorem}
\newtheorem{definition}{Definition}
\newtheorem{theorem}{Theorem}
\begin{document}
\section{Syllogisms \& Venn diagrams}
\begin{definition}
    A \textbf{predicate $p$} defines a subset $\{x\in U\:|\:p\:x\}$ of the universe $U$.
\end{definition}
\begin{definition}
    An \textbf{Euler Diagram} shows all the subsets defined by all the predicates in a universe and their intersections in such a way that no shown area is empty.
\end{definition}
\begin{definition}
    A \textbf{Venn Diagram} shows all the subsets defined by all the predicates in a universe and their intersections in such a way that all intersections are shown but they might be empty.
\end{definition}
\begin{theorem}
    The first rule of boolean algebra:
    \begin{gather*}
        \neg\neg a = a.
    \end{gather*}
\end{theorem}
\begin{theorem}
    Contrapositions:
    \begin{gather*}
        \infer={\neg b\vDash\neg a}{a\vDash b}
    \end{gather*}
\end{theorem}
\begin{theorem}
    The immediate rule:
    \begin{gather*}
        \infer{a\vDash a}{}
    \end{gather*}
\end{theorem}
\begin{theorem}
    De Morgan's Laws:
    \begin{gather*}
        \neg(a\vee b) = \neg a \wedge \neg b\\
        \neg(a\wedge b) = \neg a \vee \neg b
    \end{gather*}
\end{theorem}
\section{Sequent Calculus}
\begin{definition}
    \begin{gather*}
       \Gamma,a\vDash b,\Delta\text{ holds in }U\\
       \Leftrightarrow\\
       a\vDash b\text{ holds in }\{x\in U\:|\:\bigwedge\Gamma\:x, \neg\bigvee\Delta\:x\}
    \end{gather*}
\end{definition}
\begin{theorem}
    The following rules are sound:
    \begin{center}
        $\infer={\Gamma, a \vDash a, \Delta}{}$\\
    {\renewcommand{\arraystretch}{3}
    \begin{tabular}{c c c c}
           $\vcenter{\infer={\Gamma,\neg a\vDash \Delta}{\Gamma\vDash a, \Delta}}$ 
           &$\neg L$
           &$\vcenter{\infer={\Gamma\vDash \neg a, \Delta}{\Gamma, a\vDash \Delta}}$ 
           &$\neg R$\\
           $\vcenter{\infer={\Gamma, a\wedge b\vDash \Delta}{\Gamma, a, b\vDash \Delta}}$ 
           &$\wedge L$
           &$\vcenter{\infer={\Gamma\vDash a\wedge b, \Delta}{\Gamma\vDash a, \Delta\:\: \Gamma\vDash b}}$ 
           &$\wedge R$\\
           $\vcenter{\infer={\Gamma, a\vee b\vDash \Delta}{\Gamma,a\vDash \Delta\:\:\Gamma,b\vDash\Delta}}$ 
           &$\vee L$
           &$\vcenter{\infer={\Gamma\vDash a\vee b,\Delta}{\Gamma\vDash a,b,\Delta}}$ 
           &$\vee R$\\
           $\vcenter{\infer={\Gamma, a\oplus b\vDash\Delta}{\Gamma, a\vDash b, \Delta\:\:\Gamma, b\vDash a,\Delta}}$
           &$\oplus L$
           &$\vcenter{\infer={\Gamma\vDash a\oplus b}{\Gamma\vDash a,b,\Delta\:\:\Gamma,a,b\vDash \Delta}}$
           &$\oplus R$\\
           $\vcenter{\infer={\Gamma, a\to b\vDash \Delta}{\Gamma\vDash a,\Delta\:\:\Gamma, b\vDash \Delta}}$
           &$\to L$
           &$\vcenter{\infer={\Gamma\vDash a\to b, \Delta}{\Gamma,a\vDash b,\Delta}}$
           &$\to R$
    \end{tabular}
    }
    \end{center}
\end{theorem}
\begin{theorem}
    The following rules are complete:
    \begin{center}
        $\infer{\Gamma, a \vdash a, \Delta}{}$\\
    {\renewcommand{\arraystretch}{3}
    \begin{tabular}{c c c c}
           $\vcenter{\infer{\Gamma,\neg a\vdash \Delta}{\Gamma\vdash a, \Delta}}$ 
           &$\neg L$
           &$\vcenter{\infer{\Gamma\vdash \neg a, \Delta}{\Gamma, a\vdash \Delta}}$ 
           &$\neg R$\\
           $\vcenter{\infer{\Gamma, a\wedge b\vdash \Delta}{\Gamma, a, b\vdash \Delta}}$ 
           &$\wedge L$
           &$\vcenter{\infer{\Gamma\vdash a\wedge b, \Delta}{\Gamma\vdash a, \Delta\:\: \Gamma\vdash b}}$ 
           &$\wedge R$\\
           $\vcenter{\infer{\Gamma, a\vee b\vdash \Delta}{\Gamma,a\vdash \Delta\:\:\Gamma,b\vdash\Delta}}$ 
           &$\vee L$
           &$\vcenter{\infer{\Gamma\vdash a\vee b,\Delta}{\Gamma\vdash a,b,\Delta}}$ 
           &$\vee R$\\
           $\vcenter{\infer{\Gamma, a\oplus b\vdash\Delta}{\Gamma, a\vdash b, \Delta\:\:\Gamma, b\vdash a,\Delta}}$
           &$\oplus L$
           &$\vcenter{\infer{\Gamma\vdash a\oplus b}{\Gamma\vdash a,b,\Delta\:\:\Gamma,a,b\vdash \Delta}}$
           &$\oplus R$\\
           $\vcenter{\infer{\Gamma, a\to b\vdash \Delta}{\Gamma\vdash a,\Delta\:\:\Gamma, b\vdash \Delta}}$
           &$\to L$
           &$\vcenter{\infer{\Gamma\vdash a\to b, \Delta}{\Gamma,a\vdash b,\Delta}}$
           &$\to R$
    \end{tabular}
    }
    \end{center}
\end{theorem}
\section{Satisfiability}
\begin{definition}
    A \textbf{literal} is a predicate or a negated predicate.
\end{definition}
\begin{definition}
    A \textbf{disjunctive normal from} is a formula that is a disjunction of conjunctions of literals.\\
    It describes the evaluation that must be true to satisfy the formula.
\end{definition}
\begin{definition}
    A \textbf{conjunctive normal from} is a formula that is a conjunction of disjunctions of literals.\\
    It describes the evaluation that must not be true to satisfy the formula.
\end{definition}
\begin{theorem}
    \begin{gather*}
        \wedge \emptyset\vDash\vee\emptyset
    \end{gather*}
\end{theorem}
\subsection{DPLL (1962)}
How to find an evaluation that satisfies a given CNF.
\begin{enumerate}
    \item If there are no clauses, the solution is the empty valuation.
    \item If there is a clause without literals, there is no solution.
    \item If there is a clause with exactly one literal, we set this literal to true for the rest of the evaluation.
    \item If there are only clauses with more than one literal, choose one and continue two evaluations with this literal as false and as true.
\end{enumerate}
\subsection{Tseytin Transformation}
Convert any circuit to an equisatifiable 3-CNF.
\begin{enumerate}
    \item Get all the subformulas of the input.
    \item For each subformula $s_i$ add a clause of the form $x_i \leftrightarrow s_i$ to the output where $x_i$ is a new variable.
    \item Convert each of these clauses to CNF.
    \item Finally set the output to true and simplify the formula.
\end{enumerate}
\subsection{Using implications to find a valuation}
\begin{definition}
    Propositions are \textbf{ordered} by $x\leq y$ iff $x\to y$.
\end{definition}
\begin{theorem}
    Each binary constraint is equivalent to an implication:
    \begin{align*}
        A \vee B \equiv \neg A \to B \equiv \neg B \to A
    \end{align*}
\end{theorem}
\noindent Find a line that separates all the predicates such that
\begin{itemize}
    \item Every arrow either goes from False to True or connects elements on the same side.
    \item No atom and its negation are on the same side of the line.
    \item Each cycle of implications is entirely below or above the line.
\end{itemize}
\section{Finite State Machines and Regular Expressions}
\begin{definition}
    \textbf{Regular expressions:}
    \begin{itemize}
        \item any character is a regex that matches itself
        \item if $R$ and $S$ are regex, so is $RS$
        \item if $R$ and $S$ are regex, so is $R|S$
        \item if $R$ is a regex, so is $R^*$
    \end{itemize}
\end{definition}
\begin{definition}
    \textbf{Regular languages} over an alphabet $\Sigma$:
    \begin{itemize}
        \item $\emptyset$ and $\{\epsilon\}$ are regular languages.
        \item $\{a\}$ is a regular language for all $a\in\Sigma$.
        \item $A\cup B$, $A\bullet B$ and $A^*$ are regular for all regular languages over $\Sigma$.
        \item No other languages over $\Sigma$ are regular.
    \end{itemize}
\end{definition}
\begin{definition}
    An NFA is formally represented by a 5-tuple, consisting of
    \begin{itemize}
        \item \texttt{qs}, a finite set of states
        \item \texttt{as}, a finite set of symbols
        \item \texttt{ts}, a finite set of transitions \texttt{(q,a,q')}
        \item \texttt{ss}, a finite set of starting states
        \item \texttt{fs}, a finite set of finishing states
    \end{itemize}
\end{definition}
\begin{definition}
    A DFA is formally represented by a 5-tuple, consisting of
    \begin{itemize}
        \item \texttt{qs}, a finite set of states
        \item \texttt{as}, a finite set of symbols
        \item \texttt{ts}, a finite set of transitions \texttt{(q,a,q')}
        \item \texttt{s}, a starting state
        \item \texttt{fs}, a finite set of finishing states
    \end{itemize}
\end{definition}
\begin{theorem}
    The complement of a DFA regular language is a DFA regular language.
\end{theorem}
\begin{definition}
    Let $M=(Q,\Sigma,\Delta,B,A)$. Then a \textbf{trace} for $s=\langle x_0,...,x_{k-1}\rangle\in\Sigma^*$ is a sequence of $k+1$ states such that $\langle q_0, ...,q_k\rangle$ such that all $(q_i,x_i,q_{i+1})\in\Delta$
\end{definition}
\begin{definition}
    An FSM \textbf{accepts a word} iff there is a trace from some start state $q_0$ to some finish state $q_n$ along transitions that spell out the word.
\end{definition}
\begin{definition}
    If $R\subseteq(\Sigma\cup\{\epsilon\})^*$ is a regular language with the alphabet $\Sigma\cup\{\epsilon\}$ (where $\epsilon\not\in\Sigma$) then $R//\epsilon=\{s//\epsilon|s\in R\}$ is a regular language where $s//\epsilon$ is the result of removing every $\epsilon$ from $s$.
\end{definition}
\subsection{NFA to DFA}
\begin{enumerate}
    \item Find every reachable (super-)state of the NFA and make it a state of the DFA.
    \item For every symbol in the alphabet of the NFA find the transitions from each state of the DFA.
    \item Let the alphabet of the DFA be the alphabet of the NFA.
    \item Let the starting state of the DFA be the super-state containing all the starting states of the NFA.
    \item Let the finishing states be all those super-states that contain at least one finishing state of the NFA.
\end{enumerate}
\subsection{Combining NFA}
\begin{theorem}
    $R$ followed by $S$:\\
    Introduce an epsilon transition from every finishing state of $R$ to every starting state of $S$ and make the finishing states of $S$ the only finishing states.
\end{theorem}
\begin{theorem}
    Iteration $R^*$:\\
    Introduce an epsilon transition from every finishing state of $R$ to every starting state of $R$.
\end{theorem}
\begin{theorem}
    Alternation $R|S$:\\
    The result is the union of the two NFA.
\end{theorem}

\end{document}