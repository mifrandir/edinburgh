\documentclass{article}
\usepackage[a4paper]{geometry}
\usepackage{babel}
\usepackage{hyperref}
\begin{document}
\begin{titlepage}
    \begin{center}
        \LARGE Franz Miltz\\
        \url{franz@miltz.me}\\
        \vspace{5cm}
        \LARGE\textbf{Introduction to Computation}\\
        SEM1\\
        \vspace{1cm}
        Philip Wadler \& Micheal Fourman
    \end{center}
\end{titlepage}
\tableofcontents
\pagebreak
\section{Lecture 1 (FP), September 17}
\subsection{Prolog}
\begin{itemize}
    \item Tutorials start this week. (Thursday/Friday)
    \item Everyone should install Haskell.
    \item Get the Tutorial! (May involve extra steps. Sorry.)
    \item You may read in before or after the lecture.
\end{itemize}
\subsection{Why FP first?}
\begin{itemize}
    \item Diversity is important.
    \item Increasingly important.
    \item Equality throughout the course.
    \item Operate on data structure \emph{as a whole} rather than \emph{piecemeal}. $\Rightarrow$ higher level
    \item Good for councurrency.
    $\Leftarrow$ Everything is immutable.
    You don't need to worry about different instances changing the data.
\end{itemize}
Quote of the day:
\begin{quotation}
    Premature optimization is the root of all evil.
    (Philip Wadler)
\end{quotation}
\subsection{What is Haskell?}
\begin{itemize}
    \item A functional programming language
    \item Designed by a committee 30 years ago.
    \item Phil was part of it.
\end{itemize}
\subsection{What is Haskell for?}
\begin{itemize}
    \item Research
    \item Teaching
    \item Industry
\end{itemize}
\subsection{Families}
\begin{itemize}
    \item FP: Haskell, Erlang, Purescript, Racket, Scala
    \item OO: C++, Java, C\#, Python
\end{itemize}
Ideas from FP have found their way into OOP. (GC, Higher-order functions, generics, list comprehension, ...)
\subsection{Functions}
\subsubsection{Definitions}
\begin{itemize}
    \item Recipe / Instructions
    \item I/O pair set
    \item equation
    \item a graph
\end{itemize}
\subsubsection{Kinds of data}
\begin{itemize}
    \item integers
    \item floats
    \item characters
    \item strings
    \item booleans
    \item pictures
\end{itemize}
\subsubsection{Applying a function}
\begin{verbatim}
invert :: Picture -> Picture
knight :: Picture
invert knight
\end{verbatim}
\subsubsection{Composing functions}
\begin{verbatim}
Picture 
\end{verbatim}
\end{document}