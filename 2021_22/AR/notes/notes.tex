\documentclass{article}
\usepackage{notes-preamble}
\usepackage{enumitem}
\mkthms

\title{Automated Reasoning (SEM5)}
\author{Franz Miltz}
\begin{document}
\maketitle
\tableofcontents
\pagebreak
\section{Introduction}
\begin{definition}
    \emph{Automated Reasoning} refers to reasoning in a computer using logic.
    \begin{itemize}
        \item active area of research since the 50s
        \item part of artificial intelligence
    \end{itemize}
\end{definition}
\begin{theorem}
    Formalised mathematics is neither
    \begin{itemize}
        \item \emph{complete} (see G\"odel's Incompleteness Theorems), nor
        \item \emph{decidable} (see Church and Turing).
    \end{itemize}
\end{theorem}

\section{Logic}
\subsection{Propositional Logic}

\begin{definition}
    A syntactically correct formula is called a \emph{well-formed formula}.

    Given an alphabet of propositional symbols $\mathcal{L}$, the
    set of wffs is the smallest set such that 
    \begin{itemize}
        \item any symbol $A\in\mathcal{L}$ is a wff;
        \item if $P$ and $Q$ are wffs, so are $\neg P, P\vee Q, P\wedge Q, P\leftarrow Q$, and $P\leftrightarrow Q$;
        \item if $P$ is a wff, then $(P)$ is a wff.
    \end{itemize}
\end{definition}

\begin{definition}
    An \emph{interpretation} (or \emph{valuation}) $V$ is a truth assignment
    to the symbols in the alphabet $\mathcal{L}$, i.e. a function
    \begin{align*}
        V:\mathcal{L}\to \{\top,\bot\}.
    \end{align*}
\end{definition}

\begin{definition}
    An interpretation $V$ \emph{satisfies} a wff $P$ if $[\![ P]\!]_V=\top$.
\end{definition}

\begin{definition}
    A wff is \emph{satisfiable} if there exists an interpretation that satisfies it.
    Otherwise it's unsatisfiable.
\end{definition}

\begin{definition}
    A wff is \emph{valid} if every interpretation satisfies it.
\end{definition}

\begin{definition}
    The wffs $P_1, P_2,...,P_n$ \emph{entail} $Q$ if any interpretation
    which satisfies all of $P_1, P_2,...,P_n$ also satisfies $A$.
    We write 
    \begin{align*}
        P_1, P_2, ..., P_n \vDash Q.
    \end{align*}
\end{definition}
\end{document}