\documentclass{beamer}
\usepackage{babel}
\title{Pointwise \& Uniform Convergence}
\author{Franz Miltz}
\date{30 November 2021}
\usetheme[
    numbering=none, 
    block=fill, 
    background=light, 
    progressbar=foot,
    sectionpage=progressbar,
    subsectionpage=none
]{metropolis}
\usepackage{graphicx}

\begin{document}
    \frame{\titlepage}
    \begin{frame}
        \frametitle{Motivation}
        Consider a sequence of functions $(f_n)$. For example,
        \begin{itemize}
            \item $f_n(x) = x + 1/n$,
            \item $f_n(x) = x/n$,
            \item $f_n(x) = (-1)^nx$.
        \end{itemize}
        \begin{enumerate}
            \item \emph{Does $f_n$ get arbitrarily close to some $f$ as $n\to\infty$?}
            \item \emph{How are the properties of the limit $f$ and each $f_n$ related?}
        \end{enumerate}
    \end{frame} 
    \begin{frame}
        \frametitle{Pointwise Convergence}
        \emph{Idea: Consider $(f_n(x))$ for individual $x$.}
        \begin{definition}
            $(f_n)$ converges pointwise to $f$ if 
            for all $x$,
            \begin{align*}
                \lim_{n\to\infty} f_n(x) = f(x),
            \end{align*}
            or equivalently if for all $x$,
            \begin{align*}
                \forall \varepsilon > 0.\:
                \underline{\mathbf{\forall x\in E.\:\exists N\in\mathbb{N}.}}\:
                \forall n>N.\:\left\vert f_n(x)-f(x) \right\vert < \varepsilon.
            \end{align*}
        \end{definition}
    \end{frame} 
    \begin{frame}
        \frametitle{Uniform Convergence}
        \emph{Problem: Pointwise convergence does not preserve properties of the $f_n$.
        (e.g. continuity, integrability, ...)}
        \begin{itemize}
            \item Relation between $x_1$ and $x_2$ are lost completely.
            \item Convergence of individual sequences instead of functions.
        \end{itemize}
        Let $(f_n:E\to\mathbb{R})$ be a sequence of functions.
        \begin{definition}
            $(f_n)$ converges \emph{uniformly} to $f$ if
            \begin{align*}
                \forall \varepsilon >0.\: 
                \underline{\mathbf{\exists N\in\mathbb{N}.\:\forall x\in E.}}\:
                \forall n > N.\: \left\vert f_n(x)-f(x) \right\vert < \varepsilon.
            \end{align*}
        \end{definition}
    \end{frame}
    \begin{frame}
        \frametitle{Properties}
        Preserves the following to some degree:
        \begin{itemize}
            \item Continuity at $x_0\in E$. (\emph{Theorem 2.1})
            \item Integrability and integrals on $[a,b]\subseteq E$. (\emph{Theorem 2.2})
            \item Derivatives on bounded $(a,b)\subseteq E$. (\emph{Theorem 2.3}) 
        \end{itemize}
    \end{frame}
    \begin{frame}
        \frametitle{Intuition}
        \includegraphics[width=11cm]{skills5-pointwise.png}
    \end{frame}
    \begin{frame}
        \frametitle{Intuition}
        \includegraphics[width=11cm]{skills5-uniform.png}
    \end{frame}
\end{document}
