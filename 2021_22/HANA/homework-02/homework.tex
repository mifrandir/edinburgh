\documentclass{article}
\usepackage{homework-preamble}

\begin{document}
\title{Honours Analysis: Homework 1}
\author{Franz Miltz (UUN: S1971811)}
\date{7 October 2021}
\maketitle

\section{Workshop 2 - Question 2}

\begin{claim}
   If a sequence $(x_n)$ converges then the sequence $(x_n/n)$ is also convergent.
\end{claim}
\begin{proof}
   Note that by \emph{Wade, Theorem 2.8} $x_n$ is bounded. Choose $M\in\R$ such that
   $\abs{x_n}\leq M$ for all $n\in\N$. Then
   \begin{align*}
      -M \leq x_n \leq M
   \end{align*}
   and therefore
   \begin{align*}
      -M/n \leq x_n / n \leq M/n.
   \end{align*}
   The sequences $(-M/n)$ and $(M/n)$ clearly converge to $0$. The claim follows from the
   \emph{Squeeze Theorem}.
\end{proof}

\begin{claim}
   There exists a sequence $(x_n)$ that is not convergent such that the sequence $(x_n/n)$ converges.
\end{claim}
\begin{proof}
   Let $x_n=n$. Then $x_n\to\infty$ as $n\to\infty$. However,
   $x_n/n=1 \to 1$ as $n\to\infty$.
\end{proof}

\begin{claim}
   There exist real sequences $(x_n)$ and $(y_n)$ such that $(x_n)$ converges and $(y_n)$
   is bounded but $(x_ny_n)$ is not convergent.
\end{claim}
\begin{proof}
   Let $x_n=1$ and $y_n=(-1)^n$. Then $x_n\to 1$ and $\abs{y_n}\leq 1$ for all $n\in N$.
   However, $x_ny_n=y_n$ does not converge.
\end{proof}

\begin{claim}
   There exist sequences $(x_n)$ and $(y_n)$ such that $x_n\to 0$ as $n\to\infty$ and $y_n>0$
   for all $n\in\N$ but the sequence $(x_ny_n)$ is not convergent.
\end{claim}
\begin{proof}
   Let $x_n=1/n$ and $y_n=n^2$. Then $x_n\to\infty$ and $y_n>0$. However,
   $x_ny_n=n$ does not converge.
\end{proof}

\section{Workshop 2 - Question 5}

\begin{claim}
   Let $(x_n)$ be a sequence of real numbers and $a\in(0,1)$ such that
   \begin{align}
      \label{prop1}
      \abs{x_{n+1}-x_n} \leq a^n\hs\text{fo all }n\in\N.
   \end{align}
   Then the sequence $(x_n)$ is convergent.
\end{claim}
\begin{proof}
   Let $(x_n)$ be as described. Consider the geometric series
   \begin{align*}
      \sum_{j=0}^\infty a^j = \frac{1}{1-a}
   \end{align*}
   and the convergent sequence of partial sums $(s_n)$ with
   \begin{align*}
      s_n = \sum_{j=0}^n a^j.
   \end{align*}
   Let $\e>0$. Then choose $N\in\R$ such that for all $n>N$,
   \begin{align}
      \label{prop2}
      \abs{s_n-\frac{1}{1-a}} < \e.
   \end{align}
   Consider any integers $n,k>N+1$ and assume without loss of generality that $n>k$.
   Observe that we can use the \emph{Triangle Inequality} to find
   \begin{align*}
      \abs{x_n - x_k} = \abs{x_n - x_{n-1} + x_{n-1} - x_{n-2} \cdots + x_{k+1} - x_k}
      \leq \abs{x_n-x_{n-1}} + \abs{x_{n-1} - x_{n-2}} + \cdots + \abs{x_{k+1} - x_k}.
   \end{align*}
   We can then use (\ref{prop1}) to write
   \begin{align*}
      \abs{x_n - x_k} \leq a^{n-1} + a^{n-2} + \cdots + a^k < \sum_{j=k}^\infty a^j.
   \end{align*}
   Note that this infinite sum is precisely
   \begin{align*}
      \sum_{j=k}^\infty a^j = \frac{1}{1-a}-s_{k-1}=\abs{s_{k-1}-\frac{1}{1-a}}.
   \end{align*}
   Since $k-1>N$, it follows from (\ref{prop2}) that
   \begin{align*}
      \abs{x_n - x_k} < \e
   \end{align*}
   thereby making $(x_n)$ a Cauchy sequence. By \emph{Cauchy's Theorem}, the claim follows directly.
\end{proof}

\section{Workshop 2 - Question 6}

\begin{claim}
   Let $E\subset\R$ be a bounded infinite set.
   Then there exists a point $a\in\R$ such that for all $r>0$, the set
   $E\cap(a-r, a+r)$ contains infinitely many points.
\end{claim}
\begin{proof}
   Let $M\in\R$ such that $\abs x \leq M$ for all $x\in E$.
   Consider a function $f:\N\to E$ where for all $n\in\N$,
   \begin{align*}
      f(n) \in E \setminus \{f(k) : k < n\}.
   \end{align*}
   We define a sequence $(x_n)$ such that for all $n\in\N$,
   \begin{align*}
      x_n = f(n).
   \end{align*}
   Note that all elements of the sequence are unique due to the construction of $f$.
   Since $\abs{x_n}\leq M$ for all $n$, by the \emph{Bolzano Weierstrass Theorem} there exists a subsequence $(x_{k_n})$
   that converges to a point $a\in\R$. Let $r > 0$. Find $N\in\N$ such that for all $n>N$, $\abs{x_{k_n}-a}<r$. Equivalently,
   for all such $n$, $x_{k_n}\in(a-r, a+r)$. Since there are infinitely many such $x_{k_n}\in E$ all of which are unique,
   the claim follows.
\end{proof}

\end{document}