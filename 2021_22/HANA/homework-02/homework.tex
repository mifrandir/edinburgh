\documentclass{article}
\usepackage{homework-preamble}

\begin{document}
\title{Honours Analysis: Homework 1}
\author{Franz Miltz (UNN: S1971811)}
\date{7 October 2021}
\maketitle

\section{Workshop 2 - Question 2}

\begin{claim}
   If a sequence $(x_n)$ converges then the sequence $(x_n/n)$ is also convergent.
\end{claim}
\begin{proof}
   Note that by \emph{Wade, Theorem 2.8} $x_n$ is bounded. Choose $M\in\R$ such that
   $\abs{x_n}\leq M$ for all $n\in\N$. Then
   \begin{align*}
      -M \leq x_n \leq M
   \end{align*}
   and therefore
   \begin{align*}
      -M/n \leq x_n / n \leq M/n.
   \end{align*}
   The sequences $(-M/n)$ and $(M/n)$ clearly converge to $0$. The claim follows from the
   \emph{Squeeze Theorem}.
\end{proof}

\begin{claim}
   There exists a sequence $(x_n)$ that is not convergent such that the sequence $(x_n/n)$ converges.
\end{claim}
\begin{proof}
   Let $x_n=n$. Then $x_n\to\infty$ as $n\to\infty$. However,
   $x_n/n=1 \to 1$ as $n\to\infty$.
\end{proof}

\begin{claim}
   There exist real sequences $(x_n)$ and $(y_n)$ such that $(x_n)$ converges and $(y_n)$
   is bounded but $(x_ny_n)$ is not convergent.
\end{claim}
\begin{proof}
   Let $x_n=1$ and $y_n=(-1)^n$. Then $x_n\to 1$ and $\abs{y_n}\leq 1$ for all $n\in N$.
   However, $x_ny_n=y_n$ does not converge.
\end{proof}

\begin{claim}
   There exist sequences $(x_n)$ and $(y_n)$ such that $x_n\to 0$ as $n\to\infty$ and $y_n>0$
   for all $n\in\N$ but the sequence $(x_ny_n)$ is not convergent.
\end{claim}
\begin{proof}
   Let $x_n=1/n$ and $y_n=n^2$. Then $x_n\to\infty$ and $y_n>0$. However,
   $x_ny_n=n$ does not converge.
\end{proof}

\section{Workshop 2 - Question 5}

\begin{claim}
   Let $(x_n)$ be a sequence of real numbers and $a\in(0,1)$ such that
   \begin{align}
      \label{prop1}
      \abs{x_{n+1}-x_n} \leq a^n\hs\text{fo all }n\in\N.
   \end{align}
   Then the sequence $(x_n)$ is convergent.
\end{claim}
\begin{proof}
   Let $(x_n)$ be as described. Consider the geometric series
   \begin{align*}
      \sum_{j=0}^\infty a^j = \frac{1}{1-a}
   \end{align*}
   and the sequence of partial sums $(s_n)$ with
   \begin{align*}
      s_n = \sum_{j=0}^n a^j.
   \end{align*}
   Let $\e>0$. Then choose $N\in\R$ such that for all $n>N$,
   \begin{align*}
      \abs{s_n-\frac{1}{1-a}} < \e.
   \end{align*}
   Consider any integers $n,k>N+1$ and assume without loss of generality that $n>k$.
   Observe that we can use the \emph{Triangle Inequality} to find
   \begin{align*}
      \abs{x_n - x_k} = \abs{x_n - x_{n-1} + x_{n-1} - x_{n-2} \cdots + x_{k+1} - x_k}
      \leq \abs{x_n-x_{n-1}} + \abs{x_{n-1} - x_{n-2}} + \cdots + \abs{x_{k+1} - x_k}.
   \end{align*}
   We can use (\ref{prop1}) to write
   \begin{align*}
      \abs{x_n - x_k} \leq a^{n-1} + a^{n-2} + \cdots + a^k < \sum_{j=k}^\infty a^j.
   \end{align*}
   Note that this infinite sum is precisely
   \begin{align*}
      \sum_{j=k}^\infty a^j = \frac{1}{1-a}-s_{k-1}=\abs{s_{k-1}-\frac{1}{1-a}}.
   \end{align*}
   Since $k-1>N$, it follows that
   \begin{align*}
      \abs{x_n - x_k} < \e
   \end{align*}
   thereby making $(x_n)$ a Cauchy sequence. By \emph{Cauchy's Theorem}, the claim follows directly.
\end{proof}

\section{Workshop 2 - Question 6}

\begin{claim}
   Let $E\subset\R$ be a bounded infinite set.
   Then there exists a point $a\in\R$ such that for all $r>0$, the set
   $E\cap(a-r, a+r)$ contains infinitely many points.
\end{claim}
\begin{proof}
   Note that $E$ is bounded, so both $\inf E$ and $\sup E$ exist as real numbers.
   Consider the sets
   \begin{align*}
      E' &= \left\lbrace x \in E : x \leq \frac{\inf E + \sup E}{2}\right\rbrace,\\
      E'' &= \left\lbrace x \in E : x > \frac{\inf E + \sup E}{2}\right\rbrace.
   \end{align*}
   Note that $E=E'\cup E''$. Since $E$ is infinite, one of $E'$ and $E''$ must be as well.
   We can thus define either $E_2 = E'$ or $E_2=E''$ such that $E_2$ is a bounded infinite
   subset of $\R$.
   In general, define $E_1 := E_n$,
   \begin{align*}
      E'_n &:= \left\lbrace x \in E_n : x \leq \frac{\inf E_n + \sup E_n}{2}\right\rbrace,\\
      E''_n &:= \left\lbrace x \in E_n : x > \frac{\inf E_n + \sup E_n}{2}\right\rbrace,
   \end{align*}
   and either $E_n:=E'_{n-1}$ or $E_n:=E''_{n-1}$ such that all $E_n$ are bounded and infinite.
   Observe that
   \begin{align*}
      E_1 \supseteq E_2 \supseteq E_3 \supseteq \cdots
   \end{align*}
   Consider now the sequence of nonempty closed bounded intervals $(I_n)$ where
   \begin{align*}
      I_n = [\inf E_n, \sup E_n].
   \end{align*}
   Note that for all $n\geq 2$,
   \begin{align}
      \label{eq1}
      \inf E_n \leq \begin{cases}
         \inf E_{n-1} &\text{if }E_n = E'_{n-1},\\
         \frac{\inf E_{n-1} + \sup E_{n-1}}{2} &\text{if }E_n = E''_{n-1}.\\
      \end{cases}
   \end{align}
   Similarly,
   \begin{align}
      \label{eq2}
      \sup E_n \leq \begin{cases}
         \frac{\inf E_{n-1} + \sup E_{n-1}}{2} &\text{if }E_n = E'_{n-1},\\
         \sup E_{n-1} &\text{if }E_n = E''_{n-1}.\\
      \end{cases}
   \end{align}
   Let $\lambda(I_n)$ be the length of the interval $I_n$. 
   Then by (\ref{eq1}) and (\ref{eq2})
   \begin{align*}
      \lambda(I_n) \leq \frac{\lambda (I_{n-1})}{2}
   \end{align*}
   for all $n\geq 2$.
   By definition,
   \begin{align*}
      \lambda(I_1) = \sup E - \inf E.
   \end{align*}
   Thus 
   \begin{align*}
      0 \leq \lambda(I_n) \leq \frac{\sup E - \inf E}{2^{n-1}}.
   \end{align*}
   It follows from the \emph{Sqeeze Theorem} that $\lambda(I_n)\to 0$ as $n\to\infty$.
   Further,
   \begin{align*}
      I_1 \supset I_2 \supset I_3 \supset \cdots 
   \end{align*}
   By the \emph{Nested Interval Property} that the intersection
   \begin{align*}
      \bigcap_{n\in\N} I_n
   \end{align*}
   contains exactly one element $a\in\R$. Let $r>0$. Then choose $n\in\N$ large enough
   such that
   \begin{align*}
      \lambda(I_n)\leq \frac{\sup E - \inf E}{2^{n-1}} < r.
   \end{align*}
   Note that
   \begin{align*}
      a \in I_n = [\inf E_n, \sup E_n].
   \end{align*}
   Since $E_n\subseteq I_n$, we know that for all $x\in E_n$, $\abs{x-a}<r$.
   Because $E_n\subseteq E$ is also infinite, this shows that there are infinitely many elements in
   the set $E\cap(a-r, a+r)$.
\end{proof}

\end{document}