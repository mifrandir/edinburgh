\documentclass{article}
\usepackage{homework-preamble}

\begin{document}
\title{Honours Analysis: Homework 1}
\author{Franz Miltz (UNN: S1971811)}
\date{30 September 2021}
\maketitle

\section{Workshop 1 - Question 5}

\begin{claim}
   The set of all algebraic numbers is countable.
   \begin{proof}
      Consider the set of all degree $n$ polynomials with integer coefficients $P_n$.
      Note that each $p\in P_n$ is represented by a list of coefficients
      $a_0, ..., a_n\in\Z$ where $a_n\not=0$. The solutions are then all $x\in\R$
      such that
      \begin{align*}
         a_0 + a_1 x + a_2 x^2 + \cdots + a_n x^n = 0.
      \end{align*}
      We therefore have an injective function $f:\Z^{n+1}\to P_n$.
      Note that, by \emph{Wade, Theorem 1.42}, $\Z^{n+1}\cong \Z \times \cdots \times \Z$ is countable and therefore isomorphic to $\N$
      through a bijection $\psi:\Z^{n+1}\to\N$. Thus we have an injective
      function $f\circ\inv\psi:\N\to P_n$.
      By \emph{Wade, Lemma 1.40} this means that $P_n$ is at most countable.\\
      Now consider the set of solutions to all degree $n$ polynomials with 
      integer coefficients $S_n$. Observe that every $p\in P_n$ has at most
      $n$ real solutions, so the set $S_p$ of all real solutions for $p$ is
      finite and therefore at most countable.
      We now have 
      \begin{align*}
         S_n = \bigcup_{p\in P_n} S_p.
      \end{align*}
      Thus, by \emph{Wade, Theorem 1.42}, $S_n$ is at most countable.
      Now consider the set of all algebraic numbers
      \begin{align*}
         S = \bigcup_{n\in\Z_{\geq 0}}S_n.
      \end{align*}
      Note that $\Z_{\geq 0}\subset \Z$ is at most countable
      (\emph{Wade, Remark 1.43} and \emph{Wade, Theorem 1.41}). Since
      every $S_n$ is at most countable as well, we use \emph{Wade, Theorem 1.42}
      again to show that $S$ is at most countable.
      Since $S$ is trivially infinite, this means that the set $S$ of
      all algebraic numbers is in fact countable.
   \end{proof}
\end{claim}

\section{Workshop 1 - Question 6}

\begin{claim}
   Let $(a_n)$ be a sequence of real numbers and $a\in\R$. Suppose
   $a_n\to a$. Then
   \begin{align*}
      \frac{a_1 + a_2 + \cdots + a_n}{n}\to a.
   \end{align*}
   \begin{proof}
      The proof is taken from \emph{Notes, Exercise 1.2}.
      By \emph{Wade, Theorem 2.8}, the sequence $(a_n)$ must be bounded.
      Thus there exists an $M$ such that $\abs{a_n}\leq M$ for all $n$.
      Find such an $M$.
      Given $\e > 0$ there exsits an $N$ such that $\forall n \geq N$:
      \begin{align*}
         \abs{a_n - a} < \e.
      \end{align*}
      Fix $N$ accordingly. It follows that for all $n\geq N$ we have
      \begin{align*}
         \abs{\frac{a_1 + a_2 + \cdots + a_n}{n}-a} \leq
         \frac{1}{n}\sum_{k=1}^n \abs{a_k - a}.
      \end{align*}
      We can split the sum to find 
      \begin{align*}
         \abs{\frac{a_1 + a_2 + \cdots + a_n}{n}-a} \leq 
         \frac{1}{n}\sum_{k=1}^{N-1} \abs{a_k - a} + \frac{1}{n}\sum_{k=N}^n \abs{a_k-a}.
      \end{align*}
      Since $\abs{a_n}\leq M$ and $a\leq M$, we have $\abs{a_n-a}\leq 2M$ in general.
      We now find the upper bound
      \begin{align*}
         \abs{\frac{a_1 + a_2 + \cdots + a_n}{n}-a} \leq 
         \frac{2M(N-1)}{n} + \frac{\e(n-N+1)}{n}.
      \end{align*}
      Here $M,N$ are fixed. We let $n\to\infty$. The first term goes to zero, while
      the second is less than $\e$. Hence for all $n>N$ we have
      \begin{align*}
         \abs{\frac{a_1 + a_2 + \cdots + a_n}{n}-a} < \e.
      \end{align*} 
   \end{proof}
\end{claim}
\begin{claim}
   Let $(a_n)$ be the sequence of real numbers
   where
   \begin{align*}
      a_n = (-1)^n.
   \end{align*}
   Then $a_n$ does not converge, but
   \begin{align*}
      \frac{a_1+\cdots+a_n}{n} \to 0.
   \end{align*}
   \begin{proof}
      By \emph{Wade, Example 2.3} the sequence $(a_n)$ does not converge.
      Further, note that
      \begin{align*}
         -1 \leq a_1 + \cdots + a_n \leq 1
      \end{align*}
      for all $n\in\N$. Therefore, we have 
      \begin{align*}
         -\frac{1}{n}\leq \frac{a_1+\cdots+a_n}{n} \leq \frac{1}{n}.
      \end{align*}
      Since $-1/n\to 0$ and $1/n\to 0$, by the \emph{Sqeeze Theorem} the claim follows.
   \end{proof}
\end{claim}
\end{document}