\documentclass{article}
\usepackage{notes-preamble}
\usepackage{enumitem}
\begin{document}
\mkfpmthms
\title{Honours Analysis (SEM5)}
\author{Franz Miltz}
\maketitle
\noindent Textbook: W. Wade, \emph{An Introduction to Analysis}
\tableofcontents
\pagebreak

\section{Revision}

\subsection{Real numbers}

\subsubsection{Nested intervals}

\begin{definition}[Notes 1.1]
    A sequence $(I_n)_{n\in\N}$ of sets is said to be nested if 
    \begin{align*}
        I_1 \supset I_2 \supset I_3 \supset \dots
    \end{align*}
\end{definition}

\begin{theorem}[Nested Interval Property; Notes 1.1]
    If $(I_n)_{n\in\N}$ is a nested sequence of nonempty closed bounded intervals then 
    \begin{align*}
        E=\bigcap_{n\in\N} I_n = \{x\in\R : x \in I_n \text{ for all } n\in\N\}
    \end{align*} 
    is nonempty.
    Moreover if $\lambda(I_n)\to 0$, where $\lambda(I_n)$ denotes the length of the
    interval $I_n$, then $E$ contains exactly one number.
\end{theorem}

\subsubsection{Compactness of closed bounded intervals}

\begin{definition}
    Let $E=[a,b]$ for some real numbers $a\leq b$. Suppose that $(I_\alpha)_{\alpha\in\mathcal{A}}$
    is a collection of intervals. Then we say that $(I_\alpha)_{\alpha\in\mathcal{A}}$ \textbf{covers}
    $E$ if 
    \begin{align*}
        E \subset \bigcup_{\alpha \in \mathcal{A}} I_\alpha.
    \end{align*}
\end{definition}

\begin{theorem}[Notes 1.2]
    Let $E=[a,b]$ for some real numbers $a\leq b$. Suppose that $(I_\alpha)_{\alpha\in\mathcal{A}}$
    is an arbitrary collection of open intervals that cover $E$. Then there exists a finite
    set of indices $\{\alpha_1, \alpha_2, ..., \alpha_n\}$ such that 
    \begin{align*}
        E \subset I_{\alpha_1} \cup I_{\alpha_2}\cup \cdots \cup I_{\alpha_n}.
    \end{align*}
    We say that $(I_{\alpha_i})_{i=1,2,...,n}$ is a finite subcover of $E$.
\end{theorem}

\subsection{Sequences in $\R$}

\begin{definition}[Notes 1.2]
    A sequence of real numbers $(x_n)$ is said to \emph{converge} to a real number
    $a\in\R$ if and only if for every $\e>0$ there is an $N\in\N$ such that
    \begin{align*}
        n\geq N \text{ implies } \abs{x_n-a} < \e.
    \end{align*} 
    The number $a$ is called the limit of the sequence $(x_n)$. A sequence that
    does not converge to a real number is said to \emph{diverge}.
\end{definition}

\begin{definition}[Notes 1.3]
    A sequence $(x_n)$ of numbers $x_n\in\R$ is said to be \emph{Cauchy} if for every $\e>0$ there
    exists an $N\in\N$ such that
    \begin{align*}
        \abs{x_n-x_m}<\e \hs \text{for all }n,m\geq N. 
    \end{align*}
\end{definition}

\begin{theorem}[Notes 1.3]
    If $(x_n)$ is a convergent sequence of real numbers, then $(x_n)$ is a Cauchy sequence.
\end{theorem}

\begin{theorem}[Cauchy]
    Let $(x_n)$ be a sequence of real numbers. Then $(x_n)$ is a Cauchy sequence if and only
    if $(x_n)$ is a convergent sequence. 
\end{theorem}

\begin{definition}[Notes 1.4]
    Suppose $(x_n)_{n\in\N}$ is a sequence. A \emph{subsequence} of $(x_n)$ is a sequence of the form
    $(x_{n_k})_k\in\N$ where for each $k$ there is a positive integer $n_k$ such that
    \begin{align*}
        n_1 < n_2 < \cdots n_k < n_{k+1} < ...
    \end{align*}
\end{definition}

\begin{theorem}[Bolzano-Weierstrass]
    Every bounded sequence of real numbers has a convergent subsequence. 
\end{theorem}

\begin{definition}[Notes 1.5]
    If $(x_n)$ is a bounded sequence of real numbers we denote by
    \begin{align*}
        \limsup_{n\to\infty} x_n = \lim_{n\to\infty}\left(\sup_{k\geq n}x_k\right),\hs
        \liminf_{n\to\infty} x_n = \lim_{n\to\infty}\left(\inf_{k\geq n}x_k\right).
    \end{align*} 
\end{definition}

\begin{theorem}[Notes 1.6]
    A sequence $(x_n)$ of real numbers is convergent if and only if $\limsup_{n\to\infty}x_n$
    and $\liminf_{n\to\infty}x_n$ are real numbers and
    \begin{align*}
        \limsup_{n\to\infty}x_n =\liminf_{n\to\infty} x_n.
    \end{align*}
\end{theorem}

\subsection{Infinite series of real numbers}

\begin{definition}[Notes 1.6]
    Let $S=\sum_{k=1}^\infty a_k$ be an infinite series. For each $n\in\N$, the partial
    sum of $S$ of order $n$ is defined by
    \begin{align*}
        s_n = \sum_{k=1}^n a_k. 
    \end{align*} 
    $S$ is said to \emph{converge} if and only if its sequence of partial sums $(s_n)$
    converges to some $s\in\R$ as $n\to\infty$.
\end{definition}

\begin{theorem}[Cauchy criterion for series; Notes 1.7]
    Let $S=\sum_{k=1}^\infty a_k$ be a series. Then the series $S$ is convergent
    if and only if for any $\e>0$ there exists $N$ such that for all $m\geq n\geq N$
    we have that
    \begin{align*}
        \abs{\sum_{k=n+1}^m}<\e.
    \end{align*} 
\end{theorem}

\begin{theorem}[Notes 1.8]
    Let $S=\sum_{k=1}^\infty a_k$ be an absolutely convergent series. Then
    \begin{enumerate}
        \item The series $S$ is convergent.
        \item Let $z:\N\to\N$ be a bijection. Then the series $\sum_{k=1}^\infty a_{z(k)}$
            is convergent and \begin{align*}
                \sum_{k=1}^\infty a_k = \sum_{k=1}^\infty a_{z(k)}.
            \end{align*}
    \end{enumerate} 
\end{theorem}

\begin{theorem}[Notes 1.9]
    Let $S=\sum_{k=1}^\infty a_k$ be any conditionally convergent series. Then there
    exist rearangements $z:\N\to\N$ such that
    \begin{enumerate}
        \item For any $r\in\R$ the series $\sum_{k=1}^\infty a_{z(k)}$ is conditionally
            convergent and its sum is $r$.
        \item The series $\sum_{k=1}^\infty a_{z(k)}$ diverges to $+\infty$.
        \item The series $\sum_{k=1}^\infty a_{z(k)}$ diverges to $-\infty$.
        \item The partial sums of the series $\sum_{k=1}^\infty a_{z(k)}$ oscillate between any two real numbers.
    \end{enumerate} 
\end{theorem}

\subsection{Continuity of real functions}

\begin{definition}[Notes 1.7]
    Let $f$ be a function $f:\dom(f)\to\R$ where $\dom(f)\subset\R$. We say that $f$ is
    continuous at some $a\in\dom(f)$ if for any sequence $(x_n)$ whose terms lie in
    $\dom(f)$ and which converges to $a$, we have $\lim_{n\to\infty}f(x_n)=f(a)$. If 
    $f$ is continuous at each $a\in\S\subset\dom(f)$ then we say $f$ is continuos on $S$.
    If $f$ is continuous on $\dom(f)$ then we say that $f$ is continuous.
\end{definition}

\begin{theorem}[Notes 1.10]
    Let $f,g:D\to\R$ be continuous on $D$, and let $\alpha\in\R$ then the following functions
    are continuous on $D$. 
    \begin{enumerate}
        \item $\alpha f$;
        \item $f+g$;
        \item $fg$.
    \end{enumerate} 
\end{theorem}

\begin{definition}[Notes 1.8]
    Let $A,B\subseteq\R$ be nonempty, let $f:A\to\R$, $g:B\to\R$ and $f(A)\subseteq B$. The
    composition of $g$ with $f$ is the function $g\circ f:A\to\R$ defined by
    \begin{align*}
        (g\circ f)(x) = g(f(x)),\hs \text{for all }x\in A.
    \end{align*} 
\end{definition}

\begin{theorem}[Notes 1.11]
    If $f$ is continuous at $a\in\R$ and $g$ is continuous at $f(a)$ then the composition 
    $g\circ f$ is continuous at $a$.
\end{theorem}

\begin{theorem}[Notes 1.12]
    Let $f$ be a function $f:\dom(f)\to\R$ where $\dom(f)\subset\R$. Then $f$ is continuous
    at $a\in\dom(f)$ if and only if for any $\e>0$ there exists $\delta>0$ such that
    whenever $x\in\dom(f)$ and $\abs{x-a}<\delta$ we have $\abs{f(x)-f(a)}<\e$. 
\end{theorem}

\begin{theorem}[Intermediate Value]
    Let $a<b$ be real numbers and $f:[a,b]\to\R$ be a continuous on $[a,b]$.
    If $f(a)f(b)>0$ then there eixists at least one $c\in(a,b)$ such that $f(c)=0$. 
\end{theorem}

\begin{theorem}[Extreme Value]
    Let $a<b$ be real numbers and $f:[a,b]\to\R$ be continuous on $[a,b]$. Then
    there exists points $c,d\in[a,b]$ such that
    \begin{align*}
        f(c) = \inf\{f(x):x\in[a,b]\},\hs f(d)=\sup\{f(x):x\in[a,b]\}.
    \end{align*} 
    That is the function $f$ on the interval $[a,b]$ is bounded and attains its minimal
    and maximal values at some points $c,d\in[a,b]$ respectively.
\end{theorem}

\section{Uniform convergence}

\subsection{Uniform convergence of sequences of functions}

\begin{definition}[Notes 2.1]
    Let $E$ be a nonempty subset of $\R$. A sequence of functions $f_n:E\to\R$ is said to
    \emph{converge poinwise} on $E$ if and only if $f(x)=\lim_{n\to\infty}f_n(x)$ exists
    for each $x\in E$. 
\end{definition}

\begin{theorem}
    The pointwise limit does not preserve
    \begin{itemize}
        \item continuouity, i.e. all $f_n$ are continuous but $f$ is not,
        \item differentiability, i.e. all $f_n$ are differentiable but $f$ is not,
        \item itegrability, i.e. all $f_n$ are integrable but $f$ is not,
        \item derivatives, i.e. all $f_n$ and $f$ are differentiable, but $(f_n)'\not\to f'$,
        \item integrals, i.e. all $f_n$ and $f$ are integrable, but $\int_a^b f_n(x)dx\not\to \int_a^b f(x)dx$.
    \end{itemize}
\end{theorem}

\begin{definition}
    Let $E$ be a nonempty subset of $\R$. A sequence of functions $f_n:E\to\R$ is said to
    \emph{converge uniformly} on $E$ to a function $f$ if and only if for every $\e>0$ there
    is an $N\in\N$ such that
    \begin{align*}
        n\geq N\hs\text{implies}\hs \abs{f_n(x)-f(x)}<\e
    \end{align*}
    for all $x\in E$.
\end{definition}

\begin{proposition}
    The following are equivalent concerning a sequence of functions $f_n:E\to\R$ and $f:E\to\R$:
    \begin{enumerate}
        \item $f_n\to f$ uniformly on $E$
        \item $\sup_{x\in E}\abs{f_n(x)-f(x)}\to 0$ as $n\to\infty$
        \item there exists a sequence $a_n\to 0$ such that $\abs{f_n(x)-f(x)}\leq a_n$ for all $x\in E$.
    \end{enumerate}
\end{proposition}

\begin{definition}
    A sequence of functions $f_n$ is said to be \emph{uniformly bounded} on a set $E$ if there
    is an $M>0$ such that $\abs{f_n(x)}\leq M$ for all $x\in E$ and all $n\in\N$.
\end{definition}

\begin{lemma}
    Let $f_n:E\to\R$ be a sequence of real bounded functions. If $f_n\to f$ uniformly on $E$
    then $f$ is bounded and $f_n$ is uniformly bounded on $E$.
\end{lemma}

\begin{theorem}[Notes 2.2]
    Suppose that $f_n\to f$ uniformly on a closed interval $[a,b]$. If each $f_n$ is integrable
    on $[a,b]$, then so is $f$ and
    \begin{align*}
        \lim_{n\to\infty} \int_a^b f_n(x)\:dx = \int_a^b\left(\lim_{n\to\infty}f_n(x)\right)\:dx.
    \end{align*}
    In fact, $\lim_{n\to\infty} \int_a^x f_n(t)\:dt = \int_a^x f(t)\:dt$ uniformly for $x\in[a,b]$.
\end{theorem}

\begin{theorem}[Notes 2.3]
    Let $(a,b)$ be a bounded interval and suppose that $f_n$ is a sequence of functions which converges
    at some $x_0\in(a,b)$. If each $f_n$ is differentiable on $(a,b)$, and $f'_n$ converges uniformly
    on $(a,b)$ as $n\to\infty$, then $f_n$ converges uniformly on $(a,b)$ and
    \begin{align*}
        \lim_{n\to\infty}f'_n(x)= \left(\lim_{n\to\infty}f_n(x)\right)'
    \end{align*}
    for each $x\in(a,b)$.
\end{theorem}

\subsection{Uniform convergence of series of functions}

\begin{definition}[Notes 2.3]
    Let $f_k$ be a sequence of real functions defined on some set $E$ and set 
    \begin{align*}
        s_n(x) = \sum_{k=1}^n f_k(x),\hs x\in E,n\in\N.
    \end{align*}
    \begin{enumerate}
        \item The series $\sum_{k=1}^\infty f_k$ is said to \emph{converge pointwise} on $E$ iff
            the sequence $s_n(x)$ converges pointwise on $E$ as $n\to\infty$.
        \item The series $\sum_{k=1}^\infty f_k$ is said to \emph{converge uniformly} on $E$ iff
            the sequence $s_n(x)$ converges uniformly on $E$ as $n\to\infty$.
        \item The series $\sum_{k=1}^\infty f_k$ is said to \emph{converge absolutely (pointwise)} on $E$ iff
            $\sum_{k=1}^\infty \abs{f_k(x)}$ converges for each $x\in E$.
    \end{enumerate}
\end{definition}

\begin{theorem}[Notes 2.4]
    Let $E$ be a nonempty subset of $\R$ and let $(f_k)$ be a sequence of real functions defined on $E$.
    \begin{enumerate}
        \item Suppose that $x_0\in E$and that each $f_k$ is continuous at $x_0\in E$. 
            If $f=\sum_{k=1}^\infty f_k$ converges uniformly on $E$, then $f$ is continuos at $x_0\in E$.
        \item Term-by-term integration: Suppose that $E=[a,b]$ and that each $f_k$ is integrable on $[a,b]$.
            If $f=\sum_{k=1}^\infty f_k$ converges uniformly on $[a,b]$, then $f$ is integrable on $[a,b]$
            and \begin{align*}
                \int_a^b \sum_{k=1}^\infty f_k(x)\:dx = \sum_{k=1}^\infty \int_a^b f_k(x)\:dx.
            \end{align*}
        \item Term-by-term differentiation: Suppse that $E$ is a bounded, open interval and that each 
            $f_k$ is differentiable on $E$. If $\sum_{k=1}^\infty f_k(x_0)$ converges at some $x_0\in E$,
            and $g=\sum_{k=1}^\infty f'_k$ converges uniformly on $E$, then $f=\sum_{k=1}^\infty f_k$ 
            converges uniformly on $E$, is differentiable on $E$, and \begin{align*}
                f'(x) = \left(\sum_{k=1}^\infty f_k(x)\right)' = \sum_{k=1}^\infty f'_k(x)= g(x) 
            \end{align*}
            for $x\in E$.
    \end{enumerate}
\end{theorem}

\begin{theorem}[Weierstrass M-test]
    Let $E$ be a nonempty subset of $\R$, let $f_k:E\to\R$, $k\in\N$, and suppose that $M_k\geq 0$
    satisfies $\sum_{k=1}^\infty M_k<\infty$. If $\abs{f_k(x)}\leq M_k$ for $k\in\N$ and $x\in E$,
    $f=\sum_{k=1}^\infty f_k$ converges absolutely and uniformly on $E$.
\end{theorem}

\section{Power series}

\subsection{Introduction}

\begin{definition}[Notes 3.1]
    The \emph{radius of convergence $R$} of the power series 
    \begin{align}
        \label{ps}
        \sum_{n=0}^\infty a_n(x-c)^n     
    \end{align}
    is defined by 
    \begin{align*}
        R = \sup\{r\geq 0:(a_nr^n) \text{ is bounded}\},
    \end{align*}
    unless $(a_nr^n)$ is bounded for all $r\geq 0$, in which case we declare $R=\infty$.
\end{definition}

\begin{theorem}[Notes 3.1]
    Suppose the radius of convergence $E$ of (\ref{ps}) satisfies $0<R<\infty$. 
    If $\abs{x-c}<R$, the power series (\ref{ps}) converges absolutely. If 
    $\abs{x-c}>R$, the power series (\ref{ps}) diverges.
\end{theorem}

\subsection{Continuity and differentiability of power series}

\begin{theorem}[Notes 3.2]
    Assume that $R>0$. Suppose that $0<r<R$. Then the series (\ref{ps}) converges
    uniformly and absolutely on $\abs{x-c}\leq r$ to a continuous function $f$.
    Hence 
    \begin{align*}
        f(x) = \sum_{n=0}^\infty a_n(x-c)^n     
    \end{align*}
    defines a continuous function $f:(c-R, c+R)\to\R$.
\end{theorem}

\begin{lemma}[Notes 3.1]
    The two poer series $\sum_{n=1}^\infty a_n(x-c)^n$ and $\sum_{n=1}^\infty (x-c)^{n-1}$
    have the same radius of convergence.
\end{lemma}

\begin{theorem}[Notes 3.3]
    Suppose the radius of convergence of the power series (\ref{ps}) is $R$. Then 
    the function 
    \begin{align*}
        f(x) = \sum_{n=0}^\infty a_n(x-c)^n
    \end{align*}
    is infinitely differentiable on $\abs{x-c} < R$, and for such $x$,
    \begin{align*}
        f'(x) = \sum_{n=0}^\infty na_n(x-c)^{n-1}
    \end{align*}
    and the series converges absolutely, and also uniformly on $[c-r, c+r]$ for any 
    $r<R$. Moreover, 
    \begin{align*}
        a_n = \frac{f^{(n)}(c)}{n!}.
    \end{align*}
\end{theorem}

\subsection{Analytic functions}

\begin{definition}
    A function $f$ is \emph{analytic} on $\{x: \abs{x-c}<r\}$ if there is a power
    series (\ref{ps}) which converges to $f$ on $\{x: \abs{x-c}<r\}$.
\end{definition}

\end{document}
