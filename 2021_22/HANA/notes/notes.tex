\documentclass{article}
\usepackage{notes-preamble}
\usepackage{enumitem}
\begin{document}
\mkfpmthms
\title{Honours Analysis (SEM5)}
\author{Franz Miltz}
\maketitle
\noindent Textbook: W. Wade, \emph{An Introduction to Analysis}
\tableofcontents
\pagebreak

\section{Revision}

\subsection{Real numbers}

\subsubsection{Nested intervals}

\begin{definition}[Notes 1.1]
    A sequence $(I_n)_{n\in\N}$ of sets is said to be nested if 
    \begin{align*}
        I_1 \supset I_2 \supset I_3 \supset \dots
    \end{align*}
\end{definition}

\begin{theorem}[Nested Interval Property; Notes 1.1]
    If $(I_n)_{n\in\N}$ is a nested sequence of nonempty closed bounded intervals then 
    \begin{align*}
        E=\bigcap_{n\in\N} I_n = \{x\in\R : x \in I_n \text{ for all } n\in\N\}
    \end{align*} 
    is nonempty.
    Moreover if $\lambda(I_n)\to 0$, where $\lambda(I_n)$ denotes the length of the
    interval $I_n$, then $E$ contains exactly one number.
\end{theorem}

\subsubsection{Compactness of closed bounded intervals}

\begin{definition}
    Let $E=[a,b]$ for some real numbers $a\leq b$. Suppose that $(I_\alpha)_{\alpha\in\mathcal{A}}$
    is a collection of intervals. Then we say that $(I_\alpha)_{\alpha\in\mathcal{A}}$ \textbf{covers}
    $E$ if 
    \begin{align*}
        E \subset \bigcup_{\alpha \in \mathcal{A}} I_\alpha.
    \end{align*}
\end{definition}

\begin{theorem}[Notes 1.2]
    Let $E=[a,b]$ for some real numbers $a\leq b$. Suppose that $(I_\alpha)_{\alpha\in\mathcal{A}}$
    is an arbitrary collection of open intervals that cover $E$. Then there exists a finite
    set of indices $\{\alpha_1, \alpha_2, ..., \alpha_n\}$ such that 
    \begin{align*}
        E \subset I_{\alpha_1} \cup I_{\alpha_2}\cup \cdots \cup I_{\alpha_n}.
    \end{align*}
    We say that $(I_{\alpha_i})_{i=1,2,...,n}$ is a finite subcover of $E$.
\end{theorem}

\subsection{Sequences in $\R$}

\begin{definition}[Notes 1.2]
    A sequence of real numbers $(x_n)$ is said to \emph{converge} to a real number
    $a\in\R$ if and only if for every $\e>0$ there is an $N\in\N$ such that
    \begin{align*}
        n\geq N \text{ implies } \abs{x_n-a} < \e.
    \end{align*} 
    The number $a$ is called the limit of the sequence $(x_n)$. A sequence that
    does not converge to a real number is said to \emph{diverge}.
\end{definition}

\begin{definition}[Notes 1.3]
    A sequence $(x_n)$ of numbers $x_n\in\R$ is said to be \emph{Cauchy} if for every $\e>0$ there
    exists an $N\in\N$ such that
    \begin{align*}
        \abs{x_n-x_m}<\e \hs \text{for all }n,m\geq N. 
    \end{align*}
\end{definition}

\begin{theorem}[Notes 1.3]
    If $(x_n)$ is a convergent sequence of real numbers, then $(x_n)$ is a Cauchy sequence.
\end{theorem}

\begin{theorem}[Cauchy's Theorem; Notes 1.4]
    Let $(x_n)$ be a sequence of real numbers. Then $(x_n)$ is a Cauchy sequence if and only
    if $(x_n)$ is a convergent sequence. 
\end{theorem}

\begin{definition}[Notes 1.4]
    Suppose $(x_n)_{n\in\N}$ is a sequence. A \emph{subsequence} of $(x_n)$ is a sequence of the form
    $(x_{n_k})_k\in\N$ where for each $k$ there is a positive integer $n_k$ such that
    \begin{align*}
        n_1 < n_2 < \cdots n_k < n_{k+1} < ...
    \end{align*}
\end{definition}

\begin{theorem}[Bolzano-Weierstrass; Notes 1.5]
    Every bounded sequence of real numbers has a convergent subsequence. 
\end{theorem}

\begin{definition}[Notes 1.5]
    If $(x_n)$ is a bounded sequence of real numbers we denote by
    \begin{align*}
        \limsup_{n\to\infty} x_n = \lim_{n\to\infty}\left(\sup_{k\geq n}x_k\right),\hs
        \liminf_{n\to\infty} x_n = \lim_{n\to\infty}\left(\inf_{k\geq n}x_k\right).
    \end{align*} 
\end{definition}

\begin{theorem}[Notes 1.6]
    A sequence $(x_n)$ of real numbers is convergent if and only if $\limsup_{n\to\infty}x_n$
    and $\liminf_{n\to\infty}x_n$ are real numbers and
    \begin{align*}
        \limsup_{n\to\infty}x_n =\liminf_{n\to\infty} x_n.
    \end{align*}
\end{theorem}

\subsection{Infinite series of real numbers}

\begin{definition}[Notes 1.6]
    Let $S=\sum_{k=1}^\infty a_k$ be an infinite series. For each $n\in\N$, the partial
    sum of $S$ of order $n$ is defined by
    \begin{align*}
        s_n = \sum_{k=1}^n a_k. 
    \end{align*} 
    $S$ is said to \emph{converge} if and only if its sequence of partial sums $(s_n)$
    converges to some $s\in\R$ as $n\to\infty$.
\end{definition}

\begin{theorem}[Cauchy criterion for series; Notes 1.7]
    Let $S=\sum_{k=1}^\infty a_k$ be a series. Then the series $S$ is convergent
    if and only if for any $\e>0$ there exists $N$ such that for all $m\geq n\geq N$
    we have that
    \begin{align*}
        \abs{\sum_{k=n+1}^m}<\e.
    \end{align*} 
\end{theorem}

\begin{theorem}[Notes 1.8]
    Let $S=\sum_{k=1}^\infty a_k$ be an absolutely convergent series. Then
    \begin{enumerate}
        \item The series $S$ is convergent.
        \item Let $z:\N\to\N$ be a bijection. Then the series $\sum_{k=1}^\infty a_{z(k)}$
            is convergent and \begin{align*}
                \sum_{k=1}^\infty a_k = \sum_{k=1}^\infty a_{z(k)}.
            \end{align*}
    \end{enumerate} 
\end{theorem}

\begin{theorem}[Notes 1.9]
    Let $S=\sum_{k=1}^\infty a_k$ be any conditionally convergent series. Then there
    exist rearangements $z:\N\to\N$ such that
    \begin{enumerate}
        \item For any $r\in\R$ the series $\sum_{k=1}^\infty a_{z(k)}$ is conditionally
            convergent and its sum is $r$.
        \item The series $\sum_{k=1}^\infty a_{z(k)}$ diverges to $+\infty$.
        \item The series $\sum_{k=1}^\infty a_{z(k)}$ diverges to $-\infty$.
        \item The partial sums of the series $\sum_{k=1}^\infty a_{z(k)}$ oscillate between any two real numbers.
    \end{enumerate} 
\end{theorem}

\end{document}
