\documentclass{article}
\usepackage{homework-preamble}

\begin{document}
\title{Honours Analysis: Homework 5}
\author{Franz Miltz (UNN: S1971811)}
\date{21 October 2021}
\maketitle

\section*{Workshop 5 - Question 2}

\begin{claim*}
   Let 
   \begin{align*}
      S(x)=\sum_{k=0}^\infty \frac{(-1)^k x^{2k+1}}{(2k+1)!}
   \end{align*}
   for all $x\in\R$. Then $S$ is differentiable on $\R$ and 
   \begin{align*}
      S'(x) = \sum_{k=0}^\infty \frac{(-1)^kx^{2k}}{(2k)!}.
   \end{align*}
\end{claim*}

\begin{proof}
   We shall not prove that the interval of convergence is $(-\infty,\infty)$
   as this is part of another exercise, but it can be shown easily by 
   applying the \emph{Alternating Series Test}. Let us now focus on the claim above.
   
   Firstly, observe that we can write 
   \begin{align*}
      S(x) = \sum_{k=0}^\infty a_k(x-x_0)^k
   \end{align*}
   with $x_0 = 0$ and 
   \begin{align*}
      a_k = \begin{cases}
         \frac{(-1)^{(k-1)/2}}{k!} &\text{if $k$ is odd},\\
         0 &\text{if $k$ is even}.
      \end{cases}
   \end{align*}
   By \emph{Wade, Theorem 7.30}, it follows that $S$ is differentiable on $\R$
   and 
   \begin{align*}
      S'(x) = \sum_{k=1}^\infty ka_k(x-x_0)^{k-1}
      = \sum_{k=0}^\infty (k+1)a_{k+1}x^k.
   \end{align*}
   However, observe that for all odd $k$ we have $a_{k+1}=0$
   and thus we may write 
   \begin{align*}
      S'(x) = \sum_{k=0}^\infty (2k+1)a_{2k+1}x^{2k}
      = \sum_{k=0}^\infty (2k+1)\frac{(-1)^k x^{2k}}{(2k+1)!}
      = \sum_{k=0}^\infty \frac{(-1)^k x^{2k}}{(2k)!}.
   \end{align*}
\end{proof}

\subsection*{Workshop 5 - Question 3}

Let
\begin{align*}
   S(x) =\sum_{k=0}^\infty \frac{(-1)^k x^{2k+1}}{(2k+1)!},\hs \text{and}\hs
   C(x) = \sum_{k=0}^\infty \frac{(-1)^k x^{2k}}{(2k)!}.
\end{align*}

\begin{claim*}
   $C$ is differentiable on $\R$ and $C'=S$.
\end{claim*}
\begin{proof}
   Fix $x\not=0$. Let $(a_n)$ be the sequence such that
   \begin{align*}
      a_n = \frac{(-1)^{k}x^{2k}}{(2k)!}.
   \end{align*}
   Then note that
   \begin{align*}
      \frac{\abs{a_{n+1}}}{\abs{a_n}} = \frac{x^2}{(2k+2)(2k+1)} \leq 
   \end{align*}
\end{proof}

\begin{claim*}
   Then $(C(x))^2 + (S(x))^2 = 1$ for all $x$ and $\abs{S(x)},\abs{C(x)}\leq 1$ 
   for all $x$.
\end{claim*}

\begin{proof}
   
\end{proof}

\end{document}