\documentclass{article}
\usepackage{homework-preamble}

\begin{document}
\title{Honours Analysis: Homework 8}
\author{Franz Miltz (UNN: S1971811)}
\date{18 November 2021}
\maketitle

\section*{Workshop 8 - Question 6}

\begin{claim*}
   Let $f(x)=n$ for all $x\in((n+1)^{-1}, n^{-1}]$, $n\in\N$. Then $f$ is not integrable on 
   $(0,1]$.
\end{claim*}
\begin{proof}
   Consider the function $g:(0,1]\to\R$ given by 
   \begin{align*}
      g(x) = \sum_{j=1}^\infty \chi_{J_j}(x)
   \end{align*}
   where $J_j = (0, j^{-1}]$. Note that for $x\in((n+1)^{-1}, n^{-1}]$ where $n\in\N$,
   we have $\chi_{J_j}(x)=1$ if and only if $j\leq n$. Thus 
   \begin{align*}
      g(x) = \sum_{j=1}^n 1 = n = f(x)
   \end{align*}
   for all $x\in(0,1]$. Therefore we can write 
   \begin{align*}
      f(x) = \sum_{j=1}^\infty f_j(x)
   \end{align*}
   where $f_j=\chi_{J_j}$. Note that the series 
   \begin{align*}
      \sum_{j=1}^\infty \int_{(0,1]} f_j = \sum_{j=1}^\infty \lambda(J_j)  
      = \sum_{j=1}^\infty \frac{1}{j}
   \end{align*}
   diverges. It follows from \emph{Notes, Theorem 4.3} that $f$ is not integrable on $(0,1]$. 
\end{proof}

\section*{Workshop 8 - Question 7}

We define the function $L:\R_{>0}\to\R$ by
\begin{align*}
   L(x) = \int_1^x \frac{dt}{t}.
\end{align*}

\begin{claim*}
   $L$ is differentiable on $\R_{>0}$ and
   \begin{align}
      \label{deriv}
      L'(x) = \frac{1}{x}.
   \end{align}
\end{claim*}

\begin{proof}
   Let $g(x) = 1/x$. Then 
   \begin{align*}
      L(x) = \int_{x_0}^x g(t) dt
   \end{align*}
   where $x_0=1$. Observe that $1/x$ is a rational function defined on $\R_{\geq 0}$
   and therefore continuous. The claim follows directly from \emph{Notes, Theorem 4.10}.
\end{proof}

\begin{claim*}
   For all $x,y\in\R_{>0}$,
   \begin{align*}
      L(xy) = L(x) + L(y).
   \end{align*}
\end{claim*}

\begin{proof}
   Note that the claim holds for $x=y=1$ since 
   \begin{align*}
      L(1) = L(1) + L(1) = 0.
   \end{align*}
   Consider the following derivative where $y\in\R_{>0}$:
   \begin{align*}
      \frac{d}{dt}\left(L(ty)\right) = yL'(ty) = \frac{1}{t}.
   \end{align*}
   where we used the \emph{Chain Rule} and (\ref{deriv}). We can write
   \begin{align*}
      \int_1^x \left(\frac{d}{dt}L(ty)\right)dt = \int_1^x \frac{dt}{t}.
   \end{align*}
   By definition of $L$ and \emph{Notes, Theorem 4.11}, we then get
   \begin{align*}
      L(xy) - L(y) = L(x).
   \end{align*}
\end{proof}

\begin{claim*}
   Let $E:\R\to\R_{>0}$ be the exponential function defined by 
   \begin{align*}
      E(x)=\sum_{k=0}^\infty \frac{x^k}{k!}. 
   \end{align*}
   Then $L$ is the inverse of $E$. That is, $L(E(x))=x$ for all $x\in\R$.  
\end{claim*}
\begin{proof}
   Consider the derivative 
   \begin{align*}
      (L\circ E)'(x) = L'(E(x))E'(x)
   \end{align*}
   which we obtained by the \emph{Chain Rule}. We know from previous results that $L'(x) = 1/x$ 
   and $E'(x) = E(x)$. Thus we obtain 
   \begin{align*}
      (L\circ E)'(x) = L(E(x))E(x) = \frac{E(x)}{E(x)} = 1
   \end{align*}
   as $E(x)>0$ for all $x\in\R$. We may then conclude that 
   \begin{align*}
      \int_0^x (L\circ E)'(t)dt = \int_0^x dt.
   \end{align*}
   By \emph{Notes, Theorem 4.11}, it follows that 
   \begin{align*}
      (L\circ E)(x) - (L\circ E)(0) = x.
   \end{align*}
   However, since $L(E(0)) = L(1) = 0$, the claim follows directly.
\end{proof}

\begin{claim*}
   Let $g:[a,b]\to\R$ be continuous and nonnegative. If $\int_a^b g = 0$ then 
   $g=0$ on $[a,b]$.
\end{claim*}
\begin{proof}
   We define $G:[a,b]\to\R$ by 
   \begin{align}
      G(x) = \int_a^x g(t)dt.
   \end{align}
   Observe that, by \emph{Notes, Theorem 4.10},
   \begin{align*}
      G'(x) = g(x) \geq 0. 
   \end{align*}
   Therefore $G$ is nondecreasing. Further, clearly $G(a)=G(b)=0$.
   Due to the \emph{Mean Value Theorem}, we conclude that for all $x\in[a,b]$, $G(x) = 0$.
   We differentiate on $[a,b]$ to find 
   \begin{align*}
      G'(x) = g(x) = 0.
   \end{align*}
\end{proof}

\end{document}