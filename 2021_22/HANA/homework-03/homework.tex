\documentclass{article}
\usepackage{homework-preamble}

\begin{document}
\title{Honours Analysis: Homework 3}
\author{Franz Miltz (UUN: S1971811)}
\date{14 October 2021}
\maketitle

\section{Workshop 3 - Question 1}

\begin{claim}
	There exists a series $\sum_{n=1}^\infty a_n$ that diverges even though the
	series $(a_n)$ converges.
\end{claim}
\begin{proof}
	Let $a_n=1/n$. Then $a_n\to 0$ as $n\to\infty$ but the harmonic series
	$\sum_{n=1}^\infty a_n$ is known to diverge.
\end{proof}

\begin{claim}
	There exists a series $\sum_{n=1}^\infty a_n$ that diverges even though
	$a_n\to 0$ as $n\to\infty$.
\end{claim}
\begin{proof}
	Let $a_n=1/n$. Then $a_n\to 0$ as $n\to\infty$ but the harmonic series
	$\sum_{n=1}^\infty a_n$ is known to diverge.
\end{proof}

\begin{claim}
	There exists a series $\sum_{n=1}^\infty a_n$ that converges even though
	the series $\sum_{n=1}^\infty \abs{a_n}$ diverges.
\end{claim}
\begin{proof}
	Let $a_n=(-1)^n/n$. By the \emph{Alternating Series Test}, the series $\sum_{n=1}^\infty a_n$
	converges. However,
	\begin{align}
		\sum_{n=1}^\infty \abs{a_n} = \sum_{n=1}^\infty \frac{1}{n}
	\end{align}
	which is the harmonic series we know to diverge.
\end{proof}

\section{Workshop 3 - Question 5}

\begin{claim}
	Suppose that a series $\sum_{n=1}^\infty a_n$ converges conditionally. Prove
	that $\sum_{n=1}^\infty n^pa_n$ diverges for all $p> 1$.
\end{claim}
\begin{proof}
	By contradiction. Assume that $\sum_{n=1}^\infty n^pa_n$ converges. Then the sequence $(n^pa_n)$
	must converge to zero and therefore it must be bounded. Choose $M\in\R$ such
	that
	\begin{align*}
		\abs{n^pa_n}\leq M.
	\end{align*}
	Observe that $n^p > 0$ for all $n\in\N$. Therefore, the above implies
	\begin{align*}
		n^p\abs{a_n} \leq M
	\end{align*}
	and thereby
	\begin{align}
		\label{ineq1}
		\abs{a_n} \leq \frac{M}{n^p} .
	\end{align}
	Note that the infinite sum
	\begin{align*}
		\sum_{n=1}^\infty \frac{M}{n^p} = M\sum_{n=1}^\infty \frac{1}{n^p}
	\end{align*}
	is a constant multiple of a $p$-series with $p>1$. Therefore it converges.
	Now note that, due to (\ref{ineq1}), we can apply the \emph{Comparison Test}
	to find that the series $\sum_{n=1}^\infty a_n$ converges absolutely.
	This contradicts the premise and therefore our assumption must be false.
\end{proof}

\section{Workshop 3 - Question 6}

\begin{claim}
	Let $f(x)=x^2$ when $x$ is rational and $f(x)=0$ when $x$ is irrational.
	Then $f$ is continuous and differentiable only at $x=0$.
\end{claim}
\begin{proof}
	Consider a sequence $(x_n)$ such that $x_n\to 0$ as $n\to \infty$. Let $\e>0$.
	Pick $N$ such that $\abs{x_n}<\sqrt{\e}$ for all $n>N$. Then, for all $n>N$ we have
	\begin{align*}
		\abs{f(x_n) - f(0)} = \begin{cases}
			\abs{x_n^2} & \text{if $x_n$ is rational},   \\
			0           & \text{if $x_n$ is irrational}.
		\end{cases}
	\end{align*}
	Note that either way, $\abs{f(x_n)-f(0)}<\e$ and thus $f(x_n)\to 0$ as $n\to\infty$.
	This shows that $f$ is continuous at $0$. Now consider
	\begin{align*}
		\lim_{h\to 0} \frac{f(h) - f(0)}{h} = \lim_{h\to 0} \frac{f(h)}{h}.
	\end{align*}
	Let $\e>0$. Let $\delta = \sqrt{h}$. Then for all $0 < \abs{h} < \delta$,
	\begin{align}
		\abs{\frac{f(h)}{h}} = \begin{cases}
			h^2/h = h, & \text{if $h$ is rational},   \\
			0        , & \text{if $h$ is irrational}.
		\end{cases}
	\end{align}
	In both cases, $\abs{f(h)/h}<\e$. Thus $f$ is differentiable at $0$ and $f'(0)=0$.

	Consider a sequence of rational numbers $(x_n)$ such that $x_n\to L$ where $L\not=0$.
	Then clearly
	\begin{align}
		\lim_{n\to\infty} f(x_n) = L^2.
	\end{align}
	However, for a sequence of irrational numbers $(y_n)$ where $y_n\to L$,
	\begin{align}
		\lim_{n\to\infty} f(y_n) = 0.
	\end{align}
	Since these limits disagree for all $L\not=0$, $f$ is only continuous at $0$.
	Note that such sequences exist due to the completeness of the rationals and
	irrationals.
\end{proof}

\end{document}