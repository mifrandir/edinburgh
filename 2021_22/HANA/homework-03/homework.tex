\documentclass{article}
\usepackage{homework-preamble}

\begin{document}
\title{Honours Analysis: Homework 1}
\author{Franz Miltz (UNN: S1971811)}
\date{14 October 2021}
\maketitle

\section{Workshop 3 - Question 1}

\begin{claim}
   There exists a series $\sum_{n=1}^\infty a_n$ that diverges even though the 
   series $(a_n)$ converges.
\end{claim}
\begin{proof}
   Let $a_n=1/n$. Then $a_n\to 0$ as $n\to\infty$ but the harmonic series 
   $\sum_{n=1}^\infty a_n$ is known to diverge.
\end{proof}

\begin{claim}
   There exists a series $\sum_{n=1}^\infty a_n$ that diverges even though
   $a_n\to 0$ as $n\to\infty$.
\end{claim}
\begin{proof}
   Let $a_n=1/n$. Then $a_n\to 0$ as $n\to\infty$ but the harmonic series 
   $\sum_{n=1}^\infty a_n$ is known to diverge.
\end{proof}

\begin{claim}
   There exists a series $\sum_{n=1}^\infty a_n$ that converges even though
   the series $\sum_{n=1}^\infty \abs{a_n}$ diverges.
\end{claim}
\begin{proof}
   Let $a_n=(-1)^{n+1}/n$. 
\end{proof}

\section{Workshop 3 - Question 5}

\begin{claim}
   Suppose that a series $\sum_{n=1}^\infty a_n$ converges conditionally. Prove
   that $\sum_{n=1}^\infty n^pa_n$ diverges for all $p> 1$.
\end{claim}
\begin{proof}
   By contradiction. Assume that $\sum_{n=1}^\infty n^pa_n$ converges. Then the sequence $(n^pa_n)$
   must converge to zero and therefore it must be bounded. Choose $M\in\R$ such 
   that
   \begin{align*}
      \abs{n^pa_n}\leq M.
   \end{align*}
   Observe that $n^p > 0$ for all $n\in\N$. Therefore, the above implies 
   \begin{align*}
      n^p\abs{a_n} \leq M
   \end{align*}
   and thereby
   \begin{align}
      \label{ineq1}
      \abs{a_n} \leq \frac{M}{n^p} .
   \end{align}
   Note that the infinite sum
   \begin{align*}
      \sum_{n=1}^\infty \frac{M}{n^p} = M\sum_{n=1}^\infty \frac{1}{n^p}
   \end{align*}
   is a constant multiple of a $p$-series with $p>1$. Therefore it converges.
   Now note that, due to (\ref{ineq1}), we can apply the \emph{Comparison Test}
   to find that the series $\sum_{n=1}^\infty a_n$ converges absolutely.
   This contradicts the premise and therefore our assumption must be false.
\end{proof}


\end{document}