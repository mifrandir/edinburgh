\documentclass{article}
\usepackage{homework-preamble}

\begin{document}
\title{Honours Analysis: Homework 7}
\author{Franz Miltz (UNN: S1971811)}
\date{11 November 2021}
\maketitle

\section*{Workshop 7 - Question 4}

\begin{claim*}
   Let $I$ be a bounded interval and let $f:I\to\R$ be such that 
   \begin{align*}
      f(x) = \begin{cases}
         1, &\text{for }x\not\in\Q,\\
         0, &\text{for }x\in\Q.
      \end{cases}
   \end{align*}
   Then $f$ is integrable and $\int_I f = \lambda(I)$.
\end{claim*}
\begin{proof}
   We can write $f$ as a difference of characteristic functions:
   \begin{align*}
      f(x) = \chi_I(x) - \chi_\Q(x)\hs\text{for all }x\in I.
   \end{align*}
   Note that $\chi_I$ is integrable by definition and $\chi_\Q$ is integrable due to
   \emph{Notes, Exercise 4.4}. 
   
   Thus, by \emph{Notes, Theorem 4.2}, $f$ is integrable and 
   \begin{align*}
      \int_I f = \int_I \chi_I - \int_I \chi_\Q = \lambda(I)-0 = \lambda(I).
   \end{align*}
\end{proof}

\section*{Workshop 7 - Question 5}

Let $f:[a,b]\to\R$ be continuous and let $M=\sup_{x\in[a,b]}\abs{f(x)}$. Suppose 
$M>0$. Let $p>0$.

\begin{claim*}
   For every $\e$ such that $0<\e<M/2$ there is a non-empty open interval $I\subseteq[a,b]$
   such that 
   \begin{align}
      \label{fullineq}
      (M-\e)^p\lambda(I)\leq \int_a^b \abs{f(x)}^p dx \leq M^p(b-a).
   \end{align}
\end{claim*}
\begin{proof}
   Observe that the absolute value function is continuous. Therefore $\abs{f(x)}$
   is a composition of continuous functions and thereby continuous itself.
   By the \emph{Extreme Value Theorem}, there exists $x_0\in[a,b]$ such that 
   \begin{align*}
      \abs{f(x_0)} = M.
   \end{align*}
   Further, by continuity of $\abs{f}$, there exists $\delta>0$ such that for all 
   $x\in I=(x_0-\delta, x_0+\delta)\cap[a,b]$,
   \begin{align*}
      M-\e < \abs{f(x)}.
   \end{align*}
   Note that $\epsilon > M/2 > 0$ and $p>0$. Therefore 
   \begin{align*}
      (M-\e)^p < \abs{f(x)}^p.
   \end{align*}
   By \emph{Notes, Theorem 4.2} we find 
   \begin{align*}
      \int_I (M-\e)^p \leq \int_I \abs{f(x)}^pdx.
   \end{align*}
   Since $(M-\e)^p$ is constant and $\abs{f(x)}^p$ is non-negative, we can use 
   \emph{Notes, Theorem 4.8} and $I\subset [a,b]$ to write
   \begin{align}
      \label{leftineq}
      (M-\e)^p \lambda(I)=
      \int_I (M-\e)^p \leq \int_I \abs{f(x)}^pdx
      \leq \int_a^b \abs{f(x)}^p dx.
   \end{align}
   Further, we have
   \begin{align*}
      \abs{f(x)}^p \leq M^p
   \end{align*}
   and thus 
   \begin{align}
      \label{rightineq}
      \int_a^b \abs{f(x)}^p \leq \int_a^b M^p = M^p(b-a).
   \end{align}
   Combining (\ref{leftineq}) and (\ref{rightineq}), the claim follows.
\end{proof}

\begin{claim*}
   \begin{align*}
      \lim_{p\to\infty} \left(\int_a^b \abs{f(x)}^p dx\right)^{1/p} = M.
   \end{align*}
\end{claim*}
\begin{proof}
   Consider a sequence $(p_n)$ where $p_n>0$ for all $n\in\N$ and 
   $p_n\to\infty$ as $n\to\infty$. Then define the sequence $(s_n)$ by
   \begin{align*}
      s_n = \int_a^b \abs{f(x)}^pdx. 
   \end{align*}
   Fix $0<\e<M/2$. Then, by (\ref{fullineq}), we have 
   \begin{align*}
      (M-\e)^p\lambda(I) \leq s_n \leq M^p(b-a).
   \end{align*}
   where $I\subseteq [a,b]$. Since all the terms are positive, we have
   \begin{align*}
      (M-\e)\left(\lambda(I)\right)^{1/p} \leq 
      s_n^{1/p} \leq 
      M(b-a)^{1/p}.
   \end{align*}
   Consider the limits
   \begin{align*}
      \lim_{p\to\infty} (M-\e)(\lambda(I))^{1/p} &= M-\e,\\
      \lim_{p\to\infty} M(b-a)^{1/p} &= M,\\
   \end{align*}
   Therefore, for any $0<\e<M/2$, 
   \begin{align*}
      M-\e \leq \lim_{p\to\infty} s_n^{1/p} \leq M.
   \end{align*}
   Note that $s_n$ does not depend on $\e$ so we must have 

   \begin{align*}
      \lim_{p\to\infty} s_n^{1/p} = \lim_{p\to\infty} \left(\int_a^b \abs{f(x)}dx\right)^{1/p} = M.
   \end{align*}
\end{proof}

\end{document}