\documentclass{article}
\usepackage{homework-preamble}

\begin{document}
\title{Honours Analysis: Homework 7}
\author{Franz Miltz (UNN: S1971811)}
\date{11 November 2021}
\maketitle

\section*{Workshop 7 - Question 4}

\begin{claim*}
   Let $I$ be a bounded interval and let $f:I\to\R$ be such that 
   \begin{align*}
      f(x) = \begin{cases}
         1, &\text{for }x\not\in\Q,\\
         0, &\text{for }x\in\Q.
      \end{cases}
   \end{align*}
   Then $f$ is integrable and $\int_I f = \lambda(I)$.
\end{claim*}
\begin{proof}
   We can write $f$ as a difference of characteristic functions:
   \begin{align*}
      f(x) = \chi_I(x) - \chi_\Q(x)\hs\text{for all }x\in I.
   \end{align*}
   Note that $\chi_I$ is integrable by definition and $\chi_\Q$ is integrable due to
   \emph{Notes, Exercise 4.4}. 
   
   Thus, by \emph{Notes, Theorem 4.2}, $f$ is integrable and 
   \begin{align*}
      \int_I f = \int_I \chi_I - \int_I \chi_\Q = \lambda(I)-0 = \lambda(I).
   \end{align*}
\end{proof}

\end{document}