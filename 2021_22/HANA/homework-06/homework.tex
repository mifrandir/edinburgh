\documentclass{article}
\usepackage{homework-preamble}

\begin{document}
\title{Honours Analysis: Homework 6}
\author{Franz Miltz (UNN: S1971811)}
\date{4 November 2021}
\maketitle

\section*{Workshop 6 - Question 5}
\begin{claim*}
   Let $I$ be an open interval in $\R$. Suppose $f:I\to\R$ is differentiable
   and its derivative $f'$ is bounded on $I$. Then $f$ is uniformly convergent.
\end{claim*}
\begin{proof}
   Find $M$ such that $\abs{f'(x)} \leq M$ for all $x\in I$. By the
   \emph{Mean Value Theorem} we know that for any closed interval $[a,b]\subset I$,
   \begin{align*}
      -M(b-a) \leq f(b) - f(a) \leq M(b-a)
   \end{align*}
   or equivalently
   \begin{align*}
      \abs{f(b)-f(a)} \leq M(b-a).
   \end{align*}
   Fix $\e>0$. We choose $\delta = \e/M$. Let $x,y\in I$ be such that $\abs{x-y}<\delta$
   and assume without loss of generality that $x>y$. Then
   \begin{align*}
      \abs{f(x) - f(y)} \leq M(x-y) = M\abs{x-y} < M\delta = \e.
   \end{align*}
\end{proof}

\section*{Workshop 6 - Question 7}

\begin{claim*}
   Let $I$ be any interval in $\R$. A function $f:I\to\R$ is uniformly continuous
   on $I$ if and only if whenever $s_n,t_n\in I$ are sequences such that
   $\abs{s_n-t_n}\to 0$, then $\abs{f(s_n)-f(t_n)}\to 0$.
\end{claim*}
\begin{proof}
   ($\Rightarrow$) Assume $f:I\to\R$ is uniformly continuous on $I$. Let $s_n,t_n$
   be sequences such that $s_n,t_n\in I$ for all $n\in\N$ and $\abs{s_n-t_n}\to 0$
   as $n\to\infty$. Let $\e>0$ and choose $\delta > 0$
   such that for all $x,y\in I$ where $\abs{x-y}<\delta$,
   \begin{align*}
      \abs{f(x)-f(y)}<\e.
   \end{align*}
   Then choose $N$ such that for all $n>N$,
   \begin{align*}
      \abs{s_n-t_n} < \delta.
   \end{align*}
   Then for all $n>N$ we have $\abs{s_n-t_n}<\delta$ which implies $\abs{f(s_n)-f(t_n)}<\e$
   by definition.

   ($\Leftarrow$) By contradiction.
   Let $f:I\to\R$ be such that whenever $\abs{s_n-t_n}\to 0$, we have
   $\abs{f(s_n)-f(t_n)}\to 0$.  Assume $f:I\to\R$ is not uniformly continuous.
   Then there exists $\e > 0$ such that for all $\delta>0$ there are $x,y\in I$ for
   which $\abs{x-y}<\delta$ but $\abs{f(x)-f(y)}\geq \e$. Find such $\e>0$. Then let
   $(\delta_n)$ be a sequence where $\delta_n>0$ for all $n\in\N$ and $\delta_n\to 0$ as $n\to\infty$.
   Then for each $n\in\N$ find $s_n,t_n$ such that
   \begin{align*}
      \abs{s_n-t_n} < \delta_n \hs\text{but}\hs \abs{f(s_n)-f(t_n)} \geq \e.
   \end{align*}
   Observe that by the \emph{Squeeze Theorem} $\abs{s_n-t_n}\to 0$, but clearly
   $\abs{f(s_n)-f(t_n)}\not\to 0$. This contradicts the premise and thereby proves that $f$
   is uniformly continuous on $I$.
\end{proof}

\section*{Workshop 6 - Question 8}

\begin{claim*}
   Suppose $f:[a,b]\to\R$ is continuous. Then it is uniformly continuous.
\end{claim*}
\begin{proof}
   By contradiction. Assume $f$ is not uniformly continuous. Then there exists an $\e>0$
   such that for all $\delta > 0$ there are $x,y\in[a,b]$ such that $\abs{x-y}<\delta$
   but $\abs{f(x)-f(y)}\geq\e$. Fix $\e>0$. Let $(\delta_n)$ be such that $\delta_n\to 0$
   as $n\to\infty$ and $\delta_n > 0$ for all $n\in\N$. Find $x_n,y_n\in[a,b]$ such that
   $\abs{x_n-y_n}<\delta_n$ but
   \begin{align}
      \label{geqe}
      \abs{f(x_n)-f(y_n)}\geq\e.
   \end{align}

   By the \emph{Bolzano-Weierstrass theorem} there must be a subsequence $x_{n_k}$
   such that $x_{n_k}\to L\in\R$ as $k\to\infty$. In particular, fix $\e'>0$. Then there
   exists $K_1\in\N$ such that for all $k>K_1$,
   \begin{align*}
      \abs{x_{n_k}-L}<\frac{\e'}{2},
   \end{align*}
   and $K_2\in\N$ such that for all $k>K_2$,
   \begin{align*}
      \abs{x_{n_k}-y_{n_k}} < \frac{\e'}{2}.
   \end{align*}
   Choose $K=\max\{K_1,K_2\}$. Then for all $k>K$, $\abs{y_{n_k} - L} < \e'$. Thus
   $y_{n_k} \to L$ as $k\to\infty$.
   Therefore
   \begin{align*}
      \lim_{k\to\infty} \abs{f(x_{n_k})-f(y_{n_k})}
      = \abs{\lim_{k\to\infty} f(x_{n_k})-\lim_{k\to\infty} f(y_{n_k})}
      = \abs{f(L)-f(L)} = 0.
   \end{align*}
   This contradicts (\ref{geqe}) thus $f$ must be uniformly continuous.
\end{proof}

\end{document}