\documentclass{article}
\usepackage{homework-preamble}

\begin{document}
\title{Honours Analysis: Homework 4}
\author{Franz Miltz (UNN: S1971811)}
\date{21 October 2021}
\maketitle

\section*{Workshop 4 - Question 2}

\begin{claim*}
   Let $f_n:\left[0,1\right)\to\R$ be defined by $f_n(x)=nx^n$.
   Then $f_n\to 0$ pointwise but $\int_0^1 f_n(x)\:dx\to 1$. 
\end{claim*}

\begin{proof}
   We show $f_n(x)\to 0$.
   Note if $x=0$, then $f_n(x)=0$ for all $n$. Consider $x\in(0,1)$.
   Consider the limit 
   \begin{align*}
      r = \lim_{n\to\infty} \frac{\abs{f_{n+1}(x)}}{\abs{f_n(x)}}
      = \lim_{n\to\infty}\frac{\abs{(n+1)x^{n+1}}}{\abs{nx^n}}
      = \lim_{n\to\infty}\frac{(n+1)x}{n} = x < 1.
   \end{align*}
   By the \emph{Ratio Test} this implies the series $\sum_{n=1}^\infty 
   f_n(x)$ converges absolutely for all $x$ which implies that
   $f_n\to 0$ pointwise. However,
   \begin{align*}
      \lim_{n\to\infty} \int_0^1 f_n(x)\:dx 
      = \lim_{n\to\infty}\int_0^1 nx^n\:dx
      = \lim_{n\to\infty}\frac{n}{n+1} = 1.
   \end{align*}
   This demonstrates that uniform convergence is required as a premise for 
   \emph{Theorem 2.2} in the notes since
   \begin{align*}
      \lim_{n\to\infty} \int_0^1 f_n(x)\:dx  
      \not = \int_0^1\left(\lim_{n\to\infty} f_n(x)\right)\:dx.
   \end{align*}
\end{proof}

\section*{Workshop 4 - Question 5}

\begin{claim*}
   Let $f_n:[0,\infty)\to\R$ be defined by 
   \begin{align*}
      f_n(x) = \frac{x^n}{1+x^n}.
   \end{align*}
   Then $f_n\to f$ pointwise where 
   \begin{align*}
      f(x) = \begin{cases}
         1, &\text{if }x > 1,\\
         1/2, &\text{if }x = 1, \\
         0, &\text{if }x < 1.
      \end{cases}
   \end{align*}
   This convergence is not uniform on any interval $[0, a)$ where
   $a\geq 1/2$ is an extended real number but it 
\end{claim*}
\begin{proof}
   Consider $x>1$. Let $\e>0$. Then pick $N\in\R$ as follows
   \begin{align*}
      N = \begin{cases}
         0, &\text{if }1/\e \leq 1,\\
         \log_x(1/\e-1), &\text{otherwise.}
      \end{cases} 
   \end{align*} 
   Observe that now, for all $n>N$, 
   \begin{align*}
      \frac{1}{\e} - 1 < x^n,
      \hs\text{or equivalently}\hs
      \frac{1}{1+x^n}< \e.
   \end{align*}
   Since $x^n\geq 0$ for all $x,n$, we have
   \begin{align*}
      \abs{\frac{1}{1+x^n}}=\abs{\frac{x^n}{1+x^n}-\frac{1+x^n}{1+x^n}}
      =\abs{\frac{x^n}{1+x^n}-1}<\e.
   \end{align*}
   Now observe that for $x=1$, $f_n(x)=1/2$ for all $n$.
   This shows that $f_n\to f$ pointwise on $[1,\infty)$. Consider $x<1$.
   We observe
   \begin{align*}
      0 \leq f_n(x)=\frac{x^n}{1 + x^n} \leq x^n\hs\text{for all }n.
   \end{align*}
   Since $x^n\to 0$, we can use the \emph{Squeeze Theorem} to show that
   $f_n\to 0$ pointwise as $n\to\infty$. Combining all the cases, we have shown that
   $f_n\to f$ pointwise on $[0, \infty)$.
   
\end{proof}

\end{document}