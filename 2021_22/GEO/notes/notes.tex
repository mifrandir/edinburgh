\documentclass{article}
\usepackage{notes-preamble}
\mkthms
\begin{document}
\title{Geometry (SEM5)}
\author{Franz Miltz}
\maketitle
\noindent Textbook: K. Tapp, \emph{Differential Geometry of Curves and Surfaces}
\tableofcontents
\pagebreak

\section{Revision}

\subsection{Abstract vector spaces}

\begin{definition}[Notes 2.1]
    A \emph{real vector space} $V$ is a set together with two operations, which obey the following
    axioms; for all $u,v,w\in V$ and $a,b\in\R$
    \begin{align*}
        \vec u + (\vec v + \vec w) &= (\vec u + \vec v) + \vec w\\
        \vec v + \vec w &= \vec w + \vec v \\
        \vec v + \vec 0 &= \vec v \\
        \vec v + (-\vec v) &= \vec 0 \\
        a(\vec v + \vec w) &= a\vec v + a\vec w \\
        (a+b)\vec v &= a\vec v + b\vec v \\
        a(b\vec v) &= (ab)\vec v\\
        1\vec v &= \vec v
    \end{align*}
\end{definition}

\begin{definition}[Notes 2.3]
    A \emph{basis} of a vector space $V$ is a set $\{\vec e_i \in V : i = 1, ..., n\}$
    with the following properties:
    \begin{itemize}
        \item Spans $V$: for all $\vec v\in V, \vec v = \sum_{i=1}^n v_i \vec e_i$, for some $v_i\in\R$.
        \item Linearly independent: $\sum_{i=1}^n a_i\vec e_i = 0$ implies $a_i = 0$ for all $i=1,...,n$.
    \end{itemize}
    The number of elements in a basis $n$ is called the dimension of $V$.
\end{definition}

\subsection{Dual vector spaces}

\begin{definition}[Dual vector space and basis]
    Let $V$ be a vector space over $K$. Then we can associate a new vector space $V^*$,
    also over $K$, which consists just of the linear functions on $V$:
    \begin{align*}
        V^* = \{\phi : V\to K : \phi \text{ $K$-linear}\}.
    \end{align*}
    Given a basis $(\vec e_1,..., \vec e_n)$ for $V$, we have a basis $(\alpha_1, ..., \alpha_n)$,
    given by 
    \begin{align*}
        \alpha_i(\vec e_j) = \delta_j^i
    \end{align*}
    where $\delta_j^i$ is the Kronecker delta:
    \begin{align*}
        \delta_j^i=\begin{cases}
            1 &\text{if }i=j,\\
            0 &\text{if }i\not=j.
        \end{cases}
    \end{align*}
\end{definition}

\begin{definition}
    A \emph{bilinear form} of a vector space $V$ over $K$ is a map
    \begin{align*}
        \lra{.,.} : V\times V \to K : (\vec v, \vec w) \mapsto \lra{\vec v, \vec w}.
    \end{align*}
\end{definition}

\subsection{The inverse and implicit function theorems}

\begin{theorem}[Inverse Function Theorem (one dimension); Notes 2.12]
    Let $f:(a,b)\subset \R \to \R$ be a smooth real-valued function defined on an 
    interval. If, for some $p\in(a,b)$, we have that $f'(p)\not=0$, then there exists
    a subinterval $(c,d)\subset(a,b)$ containing $p$ such that $f$ restricted to $(c,d)$
    is invertible, and the inverse function is also smooth. Moreover, if
    $g:f(c,d)\to(c,d)$ is the inverse function, then
    \begin{align*}
        g'(f(p))=\frac{1}{f'(p)}.
    \end{align*} 
\end{theorem}

\begin{theorem}[Inverse Function Theorem; Notes 2.14]
    Let 
    \begin{align*}
        F = (f^1,...,f^n):D\subset \R^n\to\R^n\\
        (x^1,...,x^n)\mapsto (f^1(x^1,...,x^n), ..., f^n(x^1,...,x^n))
    \end{align*} 
    be a smooth function defined on an open set $D\subset\R^n$, with Jacobian matrix 
    \begin{align*}
        J_F(p)=\left[\frac{\p f^i}{\p x^j}(p)\right]
    \end{align*}
    If, at any point $p\in D$, $J_F(p)$ is invertible as a matrix, then there exists 
    an open set $\tilde D\subset D\subset \R^n$ containing $p$ such that $F|_{\tilde D}$
    is invertible.
    Moreover, if $G:E\subset\R^n\to \R^n$ is the inverse:
    \begin{align*}
        G\circ F|_{\tilde D}=\text{Id}_{\tilde D} \hs\text{and}\hs F|_{\tilde D}\circ G=\text{Id}_E
    \end{align*}
    with $E$ an open subset containing $F(p)$, then
    \begin{align*}
        J_G(F(p))=\inv{(J_F(p))}.
    \end{align*}
\end{theorem}

\begin{theorem}[Implicit Function Theorem; Notes 2.15]
    Let  
    \begin{align*}
        F = (f^1,...,f^m):D\subset \R^{n+m}\to\R^m\\
    \end{align*} 
    be a smooth function defined on an open set $D\subset\R^{n+m}$ such that
    \begin{align*}
        (x^1,...,x^n,y^1,...,y^n)\mapsto (f^1(x^1,...,x^n,y^1,...,y^m), ..., f^n(x^1,...,x^n,y^1,...,y^m))
    \end{align*} 
    with Jacobian matrix
    \begin{align*}
        J_F = \begin{bmatrix}
            \frac{\p f^1}{\p x^1} & \cdots & \frac{\p f^1}{\p x^n}&\frac{\p f^1}{\p y^1} & \cdots &\frac{\p f^1}{\p y^m}\\
            \vdots & \ddots & \vdots & \vdots & \ddots & \vdots \\
            \frac{\p f^m}{\p x^1} & \cdots & \frac{\p f^m}{\p x^n}&\frac{\p f^m}{\p y^1} & \cdots &\frac{\p f^m}{\p y^m}
        \end{bmatrix}.
    \end{align*}
    If, at some point fixed $p\in D$, we have that the matrix $J_F(p)$ is invertible,
    then there exists a smooth function
    \begin{align*}
        \Phi : E \subset \R^n \to \R^m
    \end{align*}
    (the \emph{implicit function}) defined on an open set $E\subset \R^n$, such that 
    the following hold:
    \begin{enumerate}
        \item There exists an open set $\tilde D$ with $p\in \tilde D\subset D\subset\R^{n+m}$ of $p$, such that  \begin{align*}
            \tilde D \cap (\R^n\subset\R^{n+m})=E,
        \end{align*}
        and such that for all $\tilde p\in\tilde D$ we have \begin{align*}
            F(\tilde p) = F(p) \hs\Leftrightarrow\hs \tilde p =(x^1,...,x^n,\Phi(x^1,...,x^n)) \hs\text{for some}\hs (x^1,...,x^n)\in E;
        \end{align*}
        \item for all $q\in E$, we have that the Jacobian matrix is given by \begin{align*}
            J_\Phi(q)=-\inv{\begin{bmatrix}
                \frac{\p f^1}{\p y^1} & \cdots & \frac{\p f^1}{\p y^m} \\
                \vdots                & \ddots & \vdots \\
                \frac{\p f^m}{\p y^1} & \cdots & \frac{\p f^m}{\p y^m} 
            \end{bmatrix}}
            \begin{bmatrix}
                \frac{\p f^1}{\p x^1} & \cdots & \frac{\p f^1}{\p x^n} \\
                \vdots                & \ddots & \vdots \\
                \frac{\p f^m}{\p x^1} & \cdots & \frac{\p f^m}{\p x^n} 
            \end{bmatrix}.
        \end{align*}
    \end{enumerate}
\end{theorem}

\end{document}
