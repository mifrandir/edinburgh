\documentclass{article}
\usepackage{notes-preamble}
\mkthms
\begin{document}
\title{Geometry (SEM5)}
\author{Franz Miltz}
\maketitle
\noindent Textbook: K. Tapp, \emph{Differential Geometry of Curves and Surfaces}
\tableofcontents
\pagebreak

\section{Prerequisites}

\subsection{Abstract vector spaces}

\begin{definition}[Notes 2.1]
    A \emph{real vector space} $V$ is a set together with two operations, which obey the following
    axioms; for all $u,v,w\in V$ and $a,b\in\R$
    \begin{align*}
        \vec u + (\vec v + \vec w) &= (\vec u + \vec v) + \vec w\\
        \vec v + \vec w &= \vec w + \vec v \\
        \vec v + \vec 0 &= \vec v \\
        \vec v + (-\vec v) &= \vec 0 \\
        a(\vec v + \vec w) &= a\vec v + a\vec w \\
        (a+b)\vec v &= a\vec v + b\vec v \\
        a(b\vec v) &= (ab)\vec v\\
        1\vec v &= \vec v
    \end{align*}
\end{definition}

\begin{definition}[Notes 2.3]
    A \emph{basis} of a vector space $V$ is a set $\{\vec e_i \in V : i = 1, ..., n\}$
    with the following properties:
    \begin{itemize}
        \item Spans $V$: for all $\vec v\in V, \vec v = \sum_{i=1}^n v_i \vec e_i$, for some $v_i\in\R$.
        \item Linearly independent: $\sum_{i=1}^n a_i\vec e_i = 0$ implies $a_i = 0$ for all $i=1,...,n$.
    \end{itemize}
    The number of elements in a basis $n$ is called the dimension of $V$.
\end{definition}

\subsection{Dual vector spaces}

\begin{definition}[Dual vector space and basis]
    Let $V$ be a vector space over $K$. Then we can associate a new vector space $V^*$,
    also over $K$, which consists just of the linear functions on $V$:
    \begin{align*}
        V^* = \{\phi : V\to K : \phi \text{ $K$-linear}\}.
    \end{align*}
    Given a basis $(\vec e_1,..., \vec e_n)$ for $V$, we have a basis $(\alpha_1, ..., \alpha_n)$,
    given by 
    \begin{align*}
        \alpha_i(\vec e_j) = \delta_j^i
    \end{align*}
    where $\delta_j^i$ is the Kronecker delta:
    \begin{align*}
        \delta_j^i=\begin{cases}
            1 &\text{if }i=j,\\
            0 &\text{if }i\not=j.
        \end{cases}
    \end{align*}
\end{definition}

\begin{definition}[Dual maps]
    Let $V,W$ be a vector spaces over $K$ and $f:V\to W$ a linear map between them.
    Then there exists a canonical linear map 
    \begin{align*}
        f^*:W^*\to V^*
    \end{align*}
    defined by
    \begin{align*}
        \left(f^*(\beta)\right)(\vec v) = \beta(f(\vec v)), \hs\text{for all } \beta\in W^*,\vec v\in V.
    \end{align*}
\end{definition}

\begin{definition}
    A \emph{bilinear form} of a vector space $V$ over $K$ is a map
    \begin{align*}
        \lra{.,.} : V\times V \to K : (\vec v, \vec w) \mapsto \lra{\vec v, \vec w}.
    \end{align*}
\end{definition}

\subsection{The inverse and implicit function theorems}

\begin{theorem}[Inverse Function Theorem (one dimension); Notes 2.12]
    Let $f:(a,b)\subset \R \to \R$ be a smooth real-valued function defined on an 
    interval. If, for some $p\in(a,b)$, we have that $f'(p)\not=0$, then there exists
    a subinterval $(c,d)\subset(a,b)$ containing $p$ such that $f$ restricted to $(c,d)$
    is invertible, and the inverse function is also smooth. Moreover, if
    $g:f(c,d)\to(c,d)$ is the inverse function, then
    \begin{align*}
        g'(f(p))=\frac{1}{f'(p)}.
    \end{align*} 
\end{theorem}

\begin{theorem}[Inverse Function Theorem; Notes 2.14]
    Let 
    \begin{align*}
        F = (f^1,...,f^n):D\subset \R^n\to\R^n\\
        (x^1,...,x^n)\mapsto (f^1(x^1,...,x^n), ..., f^n(x^1,...,x^n))
    \end{align*} 
    be a smooth function defined on an open set $D\subset\R^n$, with Jacobian matrix 
    \begin{align*}
        J_F(p)=\left[\eval{\frac{\p f^i}{\p x^j}}{p}\right]
    \end{align*}
    If, at any point $p\in D$, $J_F(p)$ is invertible as a matrix, then there exists 
    an open set $\tilde D\subset D\subset \R^n$ containing $p$ such that $F|_{\tilde D}$
    is invertible.
    Moreover, if $G:E\subset\R^n\to \R^n$ is the inverse:
    \begin{align*}
        G\circ \eval{F}{\tilde D}=\text{Id}_{\tilde D} \hs\text{and}\hs \eval{F}{\tilde D}\circ G=\text{Id}_E
    \end{align*}
    with $E$ an open subset containing $F(p)$, then
    \begin{align*}
        J_G(F(p))=\inv{(J_F(p))}.
    \end{align*}
\end{theorem}

\begin{theorem}[Implicit Function Theorem; Notes 2.15]
    Let  
    \begin{align*}
        F = (f^1,...,f^m):D\subset \R^{n+m}\to\R^m\\
    \end{align*} 
    be a smooth function defined on an open set $D\subset\R^{n+m}$ such that
    \begin{align*}
        (x^1,...,x^n,y^1,...,y^n)\mapsto (f^1(x^1,...,x^n,y^1,...,y^m), ..., f^n(x^1,...,x^n,y^1,...,y^m))
    \end{align*} 
    with Jacobian matrix
    \begin{align*}
        J_F = \begin{bmatrix}
            \frac{\p f^1}{\p x^1} & \cdots & \frac{\p f^1}{\p x^n}&\frac{\p f^1}{\p y^1} & \cdots &\frac{\p f^1}{\p y^m}\\
            \vdots & \ddots & \vdots & \vdots & \ddots & \vdots \\
            \frac{\p f^m}{\p x^1} & \cdots & \frac{\p f^m}{\p x^n}&\frac{\p f^m}{\p y^1} & \cdots &\frac{\p f^m}{\p y^m}
        \end{bmatrix}.
    \end{align*}
    If, at some point fixed $p\in D$, we have that the matrix $J_F(p)$ is invertible,
    then there exists a smooth function
    \begin{align*}
        \Phi : E \subset \R^n \to \R^m
    \end{align*}
    (the \emph{implicit function}) defined on an open set $E\subset \R^n$, such that 
    the following hold:
    \begin{enumerate}
        \item There exists an open set $\tilde D$ with $p\in \tilde D\subset D\subset\R^{n+m}$ of $p$, such that  \begin{align*}
            \tilde D \cap (\R^n\subset\R^{n+m})=E,
        \end{align*}
        and such that for all $\tilde p\in\tilde D$ we have \begin{align*}
            F(\tilde p) = F(p) \hs\Leftrightarrow\hs \tilde p =(x^1,...,x^n,\Phi(x^1,...,x^n)) \hs\text{for some}\hs (x^1,...,x^n)\in E;
        \end{align*}
        \item for all $q\in E$, we have that the Jacobian matrix is given by \begin{align*}
            J_\Phi(q)=-\inv{\begin{bmatrix}
                \frac{\p f^1}{\p y^1} & \cdots & \frac{\p f^1}{\p y^m} \\
                \vdots                & \ddots & \vdots \\
                \frac{\p f^m}{\p y^1} & \cdots & \frac{\p f^m}{\p y^m} 
            \end{bmatrix}}
            \begin{bmatrix}
                \frac{\p f^1}{\p x^1} & \cdots & \frac{\p f^1}{\p x^n} \\
                \vdots                & \ddots & \vdots \\
                \frac{\p f^m}{\p x^1} & \cdots & \frac{\p f^m}{\p x^n} 
            \end{bmatrix}.
        \end{align*}
    \end{enumerate}
\end{theorem}

\section{Curves in Euclidean space}

\subsection{Regular curves}

\begin{definition}
    A \emph{parametrised curve in Euclidean space} is a smooth map
    $\vec x : (a,b) \to \E^n$,
    \begin{align*}
        t \mapsto\vec x(t) = \begin{pmatrix}
            x^1(t)\\
            x^2(t)\\
            \vdots\\
            x^n(t)
        \end{pmatrix}.
    \end{align*}
\end{definition}

\begin{definition}
    The \emph{velocity vector} of the curve $\vec x(t)$ is given by
    \begin{align*}
        \vec x'(t) = \begin{pmatrix}
            (x^1)'(t)\\
            (x^2)'(t)\\
            \vdots\\
            (x^n)'(t)
        \end{pmatrix}.
    \end{align*}
    A curve $\vec x(t)$ is \emph{regular} if its velocity $\vec x'(t)\not=0$
    for all $t\in(a,b)$. The \emph{tangent line} to a regular curve $\vec x(t)$
    at $\vec x(t^0)$ is the line 
    \begin{align*}
        \{\vec x(t^0)+\lambda\vec x'(t^0) : \lambda \in\R\}.
    \end{align*}
\end{definition}

\begin{definition}
    The norm of the velocity
    \begin{align*}
        v(t)=\abs{\vec x'(t)}
    \end{align*}
    is the \emph{speed} of the curve at $\vec x(t)$. Note $v(t)>0$ for all $t$
    if and only if the curve is regular. A parametrisation of a regular curve $\vec x(t)$
    such that $v(t)=1$ everywhere is called a \emph{unit-speed parametrisation}.
\end{definition}

\begin{definition}
    The \emph{arc-length} of a regular curve $\vec x:(a,b)\to \E^n$ from $\vec x(t^0)$
    to $\vec x(t)$ is
    \begin{align*}
        s(t) = \int_{t^0}^t v(t)\:dt.
    \end{align*}
    For a unit-speed parametrisation $s=t-t^0$, hence it is also called an
    \emph{arc-length parametrisation}.
\end{definition}

\begin{definition}
    Let $g:(k,l)\to(a,b), \tau\to t$ be a smooth map with $g'(\tau)\not=0$ everywhere.
    The curve $\vec{\tilde x}(\tau)=\vec x(g(\tau))$ is a \emph{reparametrisation}
    of $\vec x(t)$. If $g'>0$ it preserves orientation.
\end{definition}

\begin{theorem}
    For any regular curve $\vec x:(a,b)\to\E^n$, there exists a reparametrisation
    which is unit-speed.
\end{theorem}

\begin{definition}
    A \emph{vector field along the curve $\vec x(t)$} in Euclidean space is a vector
    function $\vec v:(a,b)\to\E^n$.
\end{definition}

\begin{definition}
    The \emph{unit tangent} vector field along a regular curve $\vec x(t)$ is 
    \begin{align*}
        \vec T(t) = \frac{\vec x'(t)}{v(t)}.
    \end{align*}
    Thus, for a unit-speed curve $\vec x(s)$ it is simply $\vec T(s)=\vec x'(s)$.
\end{definition}

\begin{definition}
    For a unit-speed curve $\vec x(s)$ the curvature $\kappa(s)$ is defined by
    \begin{align*}
        \kappa(s) = \abs{\vec T'(s)}.
    \end{align*}
    For a general parametrisation thsi becomes $\kappa(t)=\abs{\vec T(t)}/v(t)$.
\end{definition}

\subsection{The Frenet-Serret frame and the structure equations}

\begin{definition}
    A unit-speed curve $\vec x(s)$ is \emph{biregular} if $\kappa(s)\not=0$ for
    all values of $s$.
\end{definition}

\begin{definition}
    The \emph{principal normal} along a unit-speed biregular curve $\vec x(s)$ is
    \begin{align*}
        \vec N(s) = \frac{\vec T'(s)}{\abs{\vec T'(s)}}=\frac{\vec T'(s)}{\kappa}.
    \end{align*}
    The \emph{binormal} vector field along $\vec x(s)$ is
    \begin{align*}
        \vec B = \vec T \times \vec N.
    \end{align*}
\end{definition}

\begin{proposition}
    The vector fields $\{\vec T(s), \vec N(s), \vec B(s)\}$ along a biregular curve
    $\vec x(s)$ are an orthonormal basis for $\E^3$ for each $s$. This is called the
    \emph{Frenet-Serret frame} of $\vec x(t)$.
\end{proposition}

\begin{definition}
    The \emph{torsion} of a biregular unit-speed curve $\vec x(t)$ is defined by
    \begin{align*}
        \tau = - \vec B' \cdot \vec N.
    \end{align*}
\end{definition}

\begin{theorem}
    Let $\vec x$ be a unit-speed biregular curve in $\E^3$. The Frenet-Serret
    frame $\vec T(s), \vec N(s), \vec B(s)$ along $\vec x(s)$ satisfies
    \begin{align*}
        \begin{pmatrix}
            \vec T\\
            \vec N\\
            \vec B
        \end{pmatrix}'
        = \begin{pmatrix}
            0       & \kappa & 0 \\
            -\kappa & 0      & \tau \\
            0       & -\tau  & 0
        \end{pmatrix}
        \begin{pmatrix}
            \vec T\\
            \vec N\\
            \vec B
        \end{pmatrix}.
    \end{align*}
    This is called the \emph{structure equation for a unit-speed space curve},
    or sometimes the "Frenet-Serret equation".
\end{theorem}

\begin{theorem}
    For a general parametrisation $\vec x(t)$ of a biregular space curve the structure
    equation becomes 
    \begin{align*}
        \begin{pmatrix}
            \vec T\\
            \vec N\\
            \vec B
        \end{pmatrix}'
        = v\begin{pmatrix}
            0       & \kappa & 0 \\
            -\kappa & 0      & \tau \\
            0       & -\tau  & 0
        \end{pmatrix}
        \begin{pmatrix}
            \vec T\\
            \vec N\\
            \vec B
        \end{pmatrix}.
    \end{align*}
    where $v$ is the speed of the curve.
\end{theorem}

\begin{theorem}
    A biregular curve $\vec x(t)$ is a plane curve if and only if $\tau=0$ everywhere.
\end{theorem}

\begin{theorem}
    The curvature and torsion of a biregular space curve in any parametrisation can be
    computed by
    \begin{align*}
        v^3\kappa = \abs{\vec x' \times \vec x''},\hs
        v^6\kappa^2\tau = (\vec x' \times \vec x'') \cdot \vec x'''.
    \end{align*}
\end{theorem}

\subsection{The equivalence problem}

\begin{definition}
    An \emph{isometry} of $\E^3$ is a map $\E^3\to\E^3$ given by
    \begin{align*}
        \vec x \mapsto A\vec x + \vec b
    \end{align*}
    where $A$ is an orthogonal matrix and $\vec b$ is a fixed vector. If $\det A = 1$,
    so that $A$ is a rotation matrix, then the isometry is said to be an \emph{Euclidean motion}
    or a \emph{rigid motion}. If $\det A = -1$ the isometry is orientation-reversing.
\end{definition}

\begin{proposition}
    The curvature and torsion of a biregular curve are unchanged by a Euclidean motion.
    In other words, taking a unit speed parametrisation, if
    \begin{align*}
        \vec{\hat x}(s) = A\vec x(s) + \vec b,
    \end{align*}
    where $A$ is a fixed rotation matrix and $\vec b$ is a fixed vector, then
    \begin{align*}
        \hat\kappa(s) = \kappa(s), \hs \hat\tau(s)=\tau(s).
    \end{align*}
    If instead $\det A=-1$ then
    \begin{align*}
        \hat\kappa(s)=\kappa(s), \hs \hat\tau(s) = -\tau(s).
    \end{align*}
\end{proposition}

\begin{theorem}[The fundamental theorem of curves]
    If two biregular space curves have the same curvature and torsion then they differ at most
    by a Euclidean motion.
\end{theorem}

\begin{corollary}
    Every biregular curve with $\kappa,\tau$ both constant is a helix, unless $\tau=0$ in
    which case it is a circle.
\end{corollary}

\section{Vector fields and one-froms}

\subsection{Tangent space and vector fields}

\begin{definition}
    If $\vec v$ is a vector at $p\in\E^3$ and $f:\E^3\in\R$ is a differentiable function,
    the directional derivative of $f$ along $\vec v$ at $p$ is
    \begin{align*}
        (D_{\vec v}f)_p = (\vec v \cdot \grad f)(p) = \sum_{k=1}^n v^k \eval{\frac{\p f}{\p x^k}}{p}
    \end{align*}
\end{definition}

\begin{definition}
    For each $p\in D$, we define the \emph{tangent space $T_pD$ to $D$ at $p$} as the
    set of all derivative operators at $p$, called \emph{tangent vectors} at $p\in D$,
    \begin{align*}
        \sum_{i=1}^n v^i \eval{\frac{\p}{\p x^i}}{p},\hs v^i\in\R.
    \end{align*}
    A smooth \emph{vector field} on $D$ is a smooth map $v$ which assigns to each $p\in D$
    an element in $T_p D$; in other words, it is a differential operator
    \begin{align*}
        v = \sum_{i=1}^n v^i \frac{\p}{\p x^i},
    \end{align*}
    where now each $v^i:D\to\R$ is a smooth function.
\end{definition}

\begin{definition}
    If $v,w$ are vector fields
    \begin{align*}
        v = \sum_i v^i \frac{\p}{\p x^i}\hs\text{and}\hs w = \sum_i w^i \frac{\p }{\p x^i}
    \end{align*}
    and if $\lambda\in\R$, then $\lambda v$ and $v+w$ are vector fields defined by
    \begin{align*}
        \lambda v = \sum_i \lambda v^i \frac{\p }{\p x^i}\hs\text{and}\hs
        v+w = \sum_i (v^i+w^i)\frac{\p }{\p x^i}.
    \end{align*}
    Furthermore, if $f$ is a function then we can define a vector field $fv$ by
    \begin{align*}
        fv = \sum_i fv^i\frac{\p }{\p x^i}.
    \end{align*}
\end{definition}

\begin{theorem}[Notes 4.4]
    For any $D\subseteq \R^n$ and $p\in D$ the tangent space $T_pD$ is a real vector space
    of dimension $n$. The set of tangent vectors 
    $\left\lbrace\eval{\frac{\p }{\p x^1}}{p}, ...,\eval{\frac{\p }{\p x^n}}{p}\right\rbrace$
    is a basis of $T_pD$.
\end{theorem}

\begin{definition}
    Given a vector field $v$ and a smooth function $f$, we can define a function 
    \begin{align*}
        v(f) = \sum_i v^i\frac{\p f}{\p x^i}.
    \end{align*}
\end{definition}

\begin{proposition}
    For all vector fields $v,w$ and function $f,g$, the following identities hold:
    \begin{align*}
        (v+w)(f) &= v(f) + w(f)\\=ja\\
        (fv)(g)  &= fv(g) \\
        v(f+g)   &= v(f) + v(g) \\
        v(fg)    &= gv(f) + fv(g)
    \end{align*}
\end{proposition}

\begin{definition}
    A \emph{curve} in $D$ given by
    \begin{align*}
        c(t) = (x^1(t), x^2(t), ..., x^n(t))
    \end{align*}
    where each $x^i:(a,b)\to\R$ is a smooth function, such that its velocity
    \begin{align*}
        c'(t)=(x^1(t), x^2(t), ..., x^n(t))
    \end{align*}
    where each $x^i:(a,b)\to\R$ is a smooth function, such that its velocity
    is non-vanishing for all $t\in(a,b)$. 
    We say a curve $c$ \emph{passes through $p\in D$} if $c(0)=p$.
\end{definition}

\begin{proposition}
    Let $c:(a,b)\to D$ be a curve that passes through $p$. There exists a unique
    $c'_p\in T_p D$ such that for any smooth function $f:D\to\R$
    \begin{align*}
        c'_p(f) = \eval{\frac{d}{dt}}{t=0} f(c(t)).
    \end{align*}
\end{proposition}

\begin{corollary}
    There is a one-to-one correspondence between velocities of curves that pass through 
    $p\in D$ and tangent vectors $T_pD$. By (standard) abuse of notation sometimes 
    we denote $c'_p$ by the corresponding velocity $c'(0)$.
\end{corollary}

\subsection{One-forms}

\begin{definition}
    A \emph{1-form at $p\in D$} is a linear map $\alpha:T_pD \to \R$. This means, for all
    $\vec v, \vec w\in T_pD$ and $a\in\R$,
    \begin{align*}
        \alpha(\vec v + \vec w) = \alpha(v) + \alpha(w),\hs \alpha(a\vec v) = a\alpha(\vec v    la.)
    \end{align*}
    The set of 1-forms at $p\in D$, denoted by $T^*_p D$ is called the \emph{dual vector space} of $T_p D$.
\end{definition}

\begin{definition}
    We define 1-forms $\eval{dx^j}p$ at each $p\in D$ by their action on the basis
    $\left\lbrace\eval{\frac{\p }{\p x^i}}p\right\rbrace$:
    \begin{align*}
        \eval{dx^j}p\left(\eval{\frac{\p }{\p x^k}}p\right)= \begin{cases}
            1, &\text{if } j = k\\
            0, &\text{otherwise}.
        \end{cases}
    \end{align*}
    Equivalently, $\eval{dx^i}{p}$ are defined by their action on an arbitrary tangent vector
    $\vec v = \sum_{i=1}^n v^i\eval{\frac{\p }{\p x^i}}p$:
    \begin{align*}
        \eval{dx^j}{p}(\vec v) = v^j.
    \end{align*}
\end{definition}

\begin{definition}
    A \emph{differential 1-form} on $D$ is a smooth map $\alpha$ which assigns to each $p\in D$
    a 1-form in $T^*_pD$; it can be written as
    \begin{align*}
        \alpha = \sum_{i=1}^n\alpha_i dx^i
    \end{align*}
    where $\alpha_i:D\to\R$ are smooth functions. Given 1-forms $\alpha=\sum_i\alpha_idx^i$ and 
    $\beta=\sum_i\beta_idx^i$ and a function $f$ on $D$, addition and scalar multiplication of 
    1-forms are given by
    \begin{align*}
        \alpha + \beta = \sum_i (\alpha_i + \beta_i)dx^i,\hs f\alpha = \sum_i (f\alpha_i)dx^i.
    \end{align*}
\end{definition}

\begin{lemma}
    If $\alpha=\sum_i\alpha_i dx^i$ and $\vec v = \sum_i v^i\frac{\p }{\p x^i}$, then $\alpha(\vec v)$
    is the smooth function
    \begin{align*}
        \alpha(\vec v) = \sum_{i=1}^n \alpha_i v^i.    
    \end{align*}
\end{lemma}

\begin{definition}
    Given a smooth function $f$ on $D$, its \emph{exterior derivative} is the 1-form $df$ defined by
    \begin{align*}
        (df)(v) = v(f)
    \end{align*}
    for any vector field $v$. Equivalently,
    \begin{align*}
        df=\sum_{i=1}^n \frac{\p f}{\p x^i}dx^i.
    \end{align*}
\end{definition}

\begin{proposition}
    For any functions $f,g$ on $D$, we have 
    \begin{itemize}
        \item $d(f+g)=df + dg$,
        \item $d(fg) =fdg + gdf$,
        \item $df=0$ if and only if $f$ is constant.
    \end{itemize}
\end{proposition}

\subsection{Line integrals}

\begin{definition}
    Let $c:[a,b]\to D$ be a curve and $\alpha$ a 1-form on $D$. The \emph{integral of $\alpha$ over
    the curve $c$} is 
    \begin{align*}
        \int_c \alpha = \int_a^b \alpha(c'(t))\:dt
    \end{align*}
    where $c'(t)$ is the tangent vector field to the curve.
\end{definition}

\begin{proposition}
    Let $c:[a,b]\to D$ be a curve in $D$ and let $f$ a function on $D$. Then
    \begin{align*}
        \int_c df = f(c(b)) - f(c(a)).
    \end{align*}
\end{proposition}

\begin{corollary}
    If $c$ is a closed curve, i.e. $c(a)=c(b)$, then $\int_c df = 0$ for any function $f$.
\end{corollary}

\begin{proposition}
    If $\tilde c$ is a reparametrisation of $c$ and $\alpha$ is a 1-form then 
    \begin{align*}
        \int_c \alpha = \pm \int_{\tilde c} \alpha,
    \end{align*}
    where the plus sign is taken when the reparametrisation is orientation-preserving,
    and the minus sign if it is orientation-reversing.
\end{proposition}

\begin{definition}
    The \emph{pull-back} of $\alpha$ by the map $c:[a,b]\to D$ is the 1-form on the
    interval $[a,b]$ defined by
    \begin{align*}
        c^*\alpha :=  \alpha(c'(t))dt.
    \end{align*}
    The integral of $\alpha$ over $c$ may now be thought of as the integral of the pull-back 
    $c^*\alpha$ over the interval $[a,b]$.
\end{definition}

\section{Differential forms}

\subsection{The exterior algebra of differential forms}

\begin{definition}
    A 2-form at $p\in D$ is a map $\alpha:T_pD\times T_p D \to \R$ which is linear in each argument
    and alternating $\alpha(v,w)=-\alpha(w,v)$ for all $v,w\in T_pD$. More generally, a $k$-form at
    $p\in D$ is a map of $k$ vectors in $T_pD$ to $\R$ which is \emph{multilinear} and \emph{alternating}.
\end{definition}

\begin{definition}
    The \emph{wedge product} $\alpha \wedge \beta$ of $1$-forms $\alpha$ and $\beta$ is a $2$-form defined
    by the following bilinear and alternating map,
    \begin{align*}
        (\alpha\wedge\beta)(v,w) = \alpha(v)\beta(w) - \alpha(w)\beta(v) = \begin{vmatrix}
            \alpha(v) &\alpha(w) \\
            \beta(v) &\beta(w)
        \end{vmatrix}.
    \end{align*}
    More generally, the wedge product of $k$ $1$-forms $\alpha^1,\alpha^2,...,\alpha^k$ can be defined as a
    map acting on $k$ vectors $v_1,v_2,...,v_k$
    \begin{align*}
        (\alpha^1\wedge\cdots\wedge\alpha^k)(v_1,...,v_k) = \begin{vmatrix}
            \alpha^1(v_1) &\cdots &\alpha^1(v_k)\\
            \vdots &\ddots &\vdots \\ 
            \alpha^k(v_1) &\cdots &\aleph^k(v_k)
        \end{vmatrix}.
    \end{align*}
    The resulting map is linear in each vector separately and changes sign if any pair of vectors is exchanged.
    Hence it defines a $k$-form.
\end{definition}

\begin{definition}
    By a \emph{multi-index $I$ of length $\abs{I}=k$} we shall mean an increasing sequence
    $I=(i_1,...,i_k)$ of integers $1\leq i_1<i_2<\cdots<i_k\leq n$. We will write
    \begin{align*}
        dx^I = dx^{i_1} \wedge \cdots \wedge dx^{i_k}.
    \end{align*}
\end{definition}

\begin{definition}
    A \emph{differential $k$-form} or a \emph{differential form of degree $k$} on $D$ is a
    smooth map $\alpha$ which assigns to each $p\in D$ a $k$-form at $p$; it can be written as
    \begin{align*}
        \alpha=\alpha_Idx^I
    \end{align*}
    where $\alpha_I:D\to\R$ are smooth functions, and the sum happens over all multi-indices
    $I$ with $\abs{I}=k$. Given two differential $k$ forms $\alpha,\beta$ and a function $f$ 
    the differential $k$-forms $\alpha+\beta$ and $f\alpha$ are 
    \begin{align*}
        \alpha + \beta = (\alpha_I + \beta_I)dx^I, \hs f\alpha = f\alpha_I dx^I.
    \end{align*}
    The set of $k$-forms on $D$ is denoted $\Omega^k(D)$. By convention, a \emph{zero-form}
    is a function. If $k>n$ then $\Omega^k(D)=\emptyset$.
\end{definition}

\begin{definition}
    We extend $\wedge$ linearly in order to define the \emph{wedge product} of a $k$-form $\alpha$
    and an $l$-form $\beta$. Explicitly,
    \begin{align*}
        (\alpha_Idx^I) \wedge (\beta_Jdx^J) = \alpha_I\beta_Jdx^I\wedge dx^J.
    \end{align*}
    Then the wedge product defines a bilinear map 
    \begin{align*}
        \wedge : \Omega^k(D) \times \Omega^l(D) \to \Omega^{k+l} (D).
    \end{align*}
\end{definition}

\begin{proposition}
    For all $\alpha\in\Omega^k(D)$ and $\beta\in\Omega^l(D)$,
    \begin{align*}
        \alpha\wedge\beta = (-1)^{kl} \beta \wedge \alpha.
    \end{align*}
    Note that if $f\in\Omega^0(D)$ is a function, then we define 
    \begin{align*}
        f\wedge\beta = f\beta
    \end{align*}
    and the formula still applies.
\end{proposition}

\subsection{The exterior derivative}

\begin{definition}
    If $\alpha=\alpha_Idx^I\in\Omega^k(D)$, then its \emph{exterior derivative} $d\alpha\in\Omega^{k+1}(D)$
    is 
    \begin{align*}
        d\alpha = d\alpha_I \wedge dx^I
    \end{align*}
    where $d\alpha_I$ denotes the exterior derivative of the function $\alpha_I$. Note that 
    a consequence of this definition is $d(dx^I) = d(1)\wedge dx^I=0$.
\end{definition}

\begin{theorem}
    The exterior derivative $d:\Omega^k(D)\to \Omega^{k+1}(D)$ is a linear map satisfying
    the following properties 
    \begin{enumerate}
        \item $d$ obeys the graded derivation property, for any $\alpha\in\Omega^k(D)$ \begin{align*}
            d(\alpha \wedge \beta) = d\alpha\wedge\beta +(-1)^k\alpha\wedge d\beta.
        \end{align*}
        \item $d(d\alpha)=0$ for any $\alpha\in\Omega^k(D)$, or more compactly $d^2=0$.
    \end{enumerate}
\end{theorem}

\begin{definition}
    A form $\alpha\in\Omega^k(D)$ is said to be \emph{closed} if $d\alpha = 0$ and it is said
    to be \emph{exact} if $\alpha=d\beta$ for some $\beta\in\Omega^{k-1} (D)$.
\end{definition}

\begin{lemma}[Poincar\'e]
    Every closed differential form on $\R^n$ is exact.
\end{lemma}

\subsection{Integration in $\R^n$}

\begin{definition}
    The \emph{standard orientation} is defined by
    \begin{align*}
        dx^1\wedge dx^2 \wedge \cdots \wedge dx^n.
    \end{align*}
    Coordinates $(y^1, ..., y^n)$ are said to be \emph{oriented} on $D$ iff
    $dy^1\wedge\cdots\wedge dy^n$ is a positive multiple of $dx^1\wedge\cdots\wedge dx^n$
    for all $x\in D\subseteq \R^n$.
\end{definition}

\begin{proposition}
    Let $(x^1,...,x^n)$ be oriented coordinates for $\R^n$. Let $(y^1, ..., y^n)$ be 
    smooth functions on $\R^n$. Then
    \begin{align*}
        dy^1\wedge\cdots\wedge dy^n = \frac{\p(y^1,...,y^n)}{\p(x^1,...,x^n)}dx^1\wedge\cdots\wedge dx^n,
    \end{align*}
    where the factor on the RHS is the Jacobian matrix of the coordinate transformation.
\end{proposition}

\begin{definition}
    Let $(x^1,...,x^n)$ be oriented coordinates on $D\subseteq \R^n$ and write 
    \begin{align*}
        \omega = f(x^1, ..., x^n)dx^1\wedge\cdots\wedge dx^n\in\Omega^n(D).
    \end{align*}
    Then the \emph{integral of $\omega$ over $D$} is defined by
    \begin{align*}
        \int_D \omega = \int_D f(x^1,...,x^n)dx^1\cdots dx^n,
    \end{align*}
    where the RHS is now the usual multi-integral of several variable calculus.
\end{definition}

\end{document}
