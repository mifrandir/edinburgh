\documentclass{article}
\usepackage{notes-preamble}
\mkthms
\begin{document}
\title{Geometry (SEM5)}
\author{Franz Miltz}
\maketitle
\noindent Textbook: K. Tapp, \emph{Differential Geometry of Curves and Surfaces}
\tableofcontents
\pagebreak

\section{Prerequisites}

\subsection{Abstract vector spaces}

\begin{definition}[Notes 2.1]
    A \emph{real vector space} $V$ is a set together with two operations, which obey the following
    axioms; for all $u,v,w\in V$ and $a,b\in\R$
    \begin{align*}
        \vec u + (\vec v + \vec w) &= (\vec u + \vec v) + \vec w\\
        \vec v + \vec w &= \vec w + \vec v \\
        \vec v + \vec 0 &= \vec v \\
        \vec v + (-\vec v) &= \vec 0 \\
        a(\vec v + \vec w) &= a\vec v + a\vec w \\
        (a+b)\vec v &= a\vec v + b\vec v \\
        a(b\vec v) &= (ab)\vec v\\
        1\vec v &= \vec v
    \end{align*}
\end{definition}

\begin{definition}[Notes 2.3]
    A \emph{basis} of a vector space $V$ is a set $\{\vec e_i \in V : i = 1, ..., n\}$
    with the following properties:
    \begin{itemize}
        \item Spans $V$: for all $\vec v\in V, \vec v = \sum_{i=1}^n v_i \vec e_i$, for some $v_i\in\R$.
        \item Linearly independent: $\sum_{i=1}^n a_i\vec e_i = 0$ implies $a_i = 0$ for all $i=1,...,n$.
    \end{itemize}
    The number of elements in a basis $n$ is called the dimension of $V$.
\end{definition}

\subsection{Dual vector spaces}

\begin{definition}[Dual vector space and basis]
    Let $V$ be a vector space over $K$. Then we can associate a new vector space $V^*$,
    also over $K$, which consists just of the linear functions on $V$:
    \begin{align*}
        V^* = \{\phi : V\to K : \phi \text{ $K$-linear}\}.
    \end{align*}
    Given a basis $(\vec e_1,..., \vec e_n)$ for $V$, we have a basis $(\alpha_1, ..., \alpha_n)$,
    given by 
    \begin{align*}
        \alpha_i(\vec e_j) = \delta_j^i.
    \end{align*}
\end{definition}

\begin{definition}[Dual maps]
    Let $V,W$ be a vector spaces over $K$ and $f:V\to W$ a linear map between them.
    Then there exists a canonical linear map 
    \begin{align*}
        f^*:W^*\to V^*
    \end{align*}
    defined by
    \begin{align*}
        \left(f^*(\beta)\right)(\vec v) = \beta(f(\vec v)), \hs\text{for all } \beta\in W^*,\vec v\in V.
    \end{align*}
\end{definition}

\begin{definition}
    A \emph{bilinear form} of a vector space $V$ over $K$ is a map
    \begin{align*}
        \lra{.,.} : V\times V \to K : (\vec v, \vec w) \mapsto \lra{\vec v, \vec w}.
    \end{align*}
\end{definition}

\begin{lemma}
    Let $V$ be a vector space and choose a basis $\vec e_1,...,\vec e_n$. Further, let $f:V\to V$
    be a linear map, and choose a basis $\vec v_1, ...,\vec v_n$ so that it can be represented 
    by a square matrix $A$, i.e. 
    \begin{align*}
        f(\vec v) = f\left(\sum_{i=1}^n v^i \vec e_i\right) \longleftrightarrow A\begin{pmatrix}
            v^1\\\vdots\\v^n
        \end{pmatrix}.
    \end{align*}
    If we use the dual basis $b_1,...,b_n$ for $V^*$ then the dual linear map $f^*:V^*\to V^*$ gets represented
    by the same matrix $A$, which now acts on the row-vectors by multiplication from the right, i.e.
    \begin{align*}
        f^*(\alpha) = f^*\left(\sum_{j=1}^n b_j\alpha^j\right)\longleftrightarrow \begin{pmatrix}
            b_1 & \cdots & b_n
        \end{pmatrix}A.
    \end{align*}
\end{lemma}

\subsection{The inverse and implicit function theorems}

\begin{theorem}[Inverse Function Theorem (one dimension); Notes 2.12]
    Let $f:(a,b)\subset \R \to \R$ be a smooth real-valued function defined on an 
    interval. If, for some $p\in(a,b)$, we have that $f'(p)\not=0$, then there exists
    a subinterval $(c,d)\subset(a,b)$ containing $p$ such that $f$ restricted to $(c,d)$
    is invertible, and the inverse function is also smooth. Moreover, if
    $g:f(c,d)\to(c,d)$ is the inverse function, then
    \begin{align*}
        g'(f(p))=\frac{1}{f'(p)}.
    \end{align*} 
\end{theorem}

\begin{theorem}[Inverse Function Theorem; Notes 2.14]
    Let 
    \begin{align*}
        F = (f^1,...,f^n):D\subset \R^n\to\R^n\\
        (x^1,...,x^n)\mapsto (f^1(x^1,...,x^n), ..., f^n(x^1,...,x^n))
    \end{align*} 
    be a smooth function defined on an open set $D\subset\R^n$, with Jacobian matrix 
    \begin{align*}
        J_F(p)=\left[\eval{\frac{\p f^i}{\p x^j}}{p}\right]
    \end{align*}
    If, at any point $p\in D$, $J_F(p)$ is invertible as a matrix, then there exists 
    an open set $\tilde D\subset D\subset \R^n$ containing $p$ such that $F|_{\tilde D}$
    is invertible.
    Moreover, if $G:E\subset\R^n\to \R^n$ is the inverse:
    \begin{align*}
        G\circ \eval{F}{\tilde D}=\text{Id}_{\tilde D} \hs\text{and}\hs \eval{F}{\tilde D}\circ G=\text{Id}_E
    \end{align*}
    with $E$ an open subset containing $F(p)$, then
    \begin{align*}
        J_G(F(p))=\inv{(J_F(p))}.
    \end{align*}
\end{theorem}

\begin{theorem}[Implicit Function Theorem; Notes 2.15]
    Let  
    \begin{align*}
        F = (f^1,...,f^m):D\subset \R^{n+m}\to\R^m\\
    \end{align*} 
    be a smooth function defined on an open set $D\subset\R^{n+m}$ such that
    \begin{align*}
        (x^1,...,x^n,y^1,...,y^n)\mapsto (f^1(x^1,...,x^n,y^1,...,y^m), ..., f^n(x^1,...,x^n,y^1,...,y^m))
    \end{align*} 
    with Jacobian matrix
    \begin{align*}
        J_F = \begin{bmatrix}
            \frac{\p f^1}{\p x^1} & \cdots & \frac{\p f^1}{\p x^n}&\frac{\p f^1}{\p y^1} & \cdots &\frac{\p f^1}{\p y^m}\\
            \vdots & \ddots & \vdots & \vdots & \ddots & \vdots \\
            \frac{\p f^m}{\p x^1} & \cdots & \frac{\p f^m}{\p x^n}&\frac{\p f^m}{\p y^1} & \cdots &\frac{\p f^m}{\p y^m}
        \end{bmatrix}.
    \end{align*}
    If, at some point fixed $p\in D$, we have that the matrix 
    \begin{align*}
        \begin{bmatrix}
             \frac{\p f^1}{\p y^1} & \cdots & \frac{\p f^1}{\p y^m} \\
             \vdots & \ddots & \vdots \\
             \frac{\p f^m}{\p y^1} & \cdots & \frac{\p f^m}{\p y^m}
        \end{bmatrix}
    \end{align*}
    is invertible,
    then there exists a smooth function
    \begin{align*}
        \Phi : E \subset \R^n \to \R^m
    \end{align*}
    (the \emph{implicit function}) defined on an open set $E\subset \R^n$, such that 
    the following hold:
    \begin{enumerate}
        \item There exists an open set $\tilde D$ with $p\in \tilde D\subset D\subset\R^{n+m}$ of $p$, such that  \begin{align*}
            \tilde D \cap (\R^n\subset\R^{n+m})=E,
        \end{align*}
        and such that for all $\tilde p\in\tilde D$ we have \begin{align*}
            F(\tilde p) = F(p) \hs\Leftrightarrow\hs \tilde p =(x^1,...,x^n,\Phi(x^1,...,x^n)) \hs\text{for some}\hs (x^1,...,x^n)\in E;
        \end{align*}
        \item for all $q\in E$, we have that the Jacobian matrix is given by \begin{align*}
            J_\Phi(q)=-\inv{\begin{bmatrix}
                \frac{\p f^1}{\p y^1} & \cdots & \frac{\p f^1}{\p y^m} \\
                \vdots                & \ddots & \vdots \\
                \frac{\p f^m}{\p y^1} & \cdots & \frac{\p f^m}{\p y^m} 
            \end{bmatrix}}
            \begin{bmatrix}
                \frac{\p f^1}{\p x^1} & \cdots & \frac{\p f^1}{\p x^n} \\
                \vdots                & \ddots & \vdots \\
                \frac{\p f^m}{\p x^1} & \cdots & \frac{\p f^m}{\p x^n} 
            \end{bmatrix}.
        \end{align*}
    \end{enumerate}
\end{theorem}

\section{Curves in Euclidean space}

\subsection{Regular curves}

\begin{definition}
    A \emph{parametrised curve in Euclidean space} is a smooth map
    $\vec x : (a,b) \to \E^n$,
    \begin{align*}
        t \mapsto\vec x(t) = \begin{pmatrix}
            x^1(t)\\
            x^2(t)\\
            \vdots\\
            x^n(t)
        \end{pmatrix}.
    \end{align*}
\end{definition}

\begin{definition}
    The \emph{velocity vector} of the curve $\vec x(t)$ is given by
    \begin{align*}
        \vec x'(t) = \begin{pmatrix}
            (x^1)'(t)\\
            (x^2)'(t)\\
            \vdots\\
            (x^n)'(t)
        \end{pmatrix}.
    \end{align*}
    A curve $\vec x(t)$ is \emph{regular} if its velocity $\vec x'(t)\not=0$
    for all $t\in(a,b)$. The \emph{tangent line} to a regular curve $\vec x(t)$
    at $\vec x(t^0)$ is the line 
    \begin{align*}
        \{\vec x(t^0)+\lambda\vec x'(t^0) : \lambda \in\R\}.
    \end{align*}
\end{definition}

\begin{definition}
    The norm of the velocity
    \begin{align*}
        v(t)=\abs{\vec x'(t)}
    \end{align*}
    is the \emph{speed} of the curve at $\vec x(t)$. Note $v(t)>0$ for all $t$
    if and only if the curve is regular. A parametrisation of a regular curve $\vec x(t)$
    such that $v(t)=1$ everywhere is called a \emph{unit-speed parametrisation}.
\end{definition}

\begin{definition}
    The \emph{arc-length} of a regular curve $\vec x:(a,b)\to \E^n$ from $\vec x(t^0)$
    to $\vec x(t)$ is
    \begin{align*}
        s(t) = \int_{t^0}^t v(t)\:dt.
    \end{align*}
    For a unit-speed parametrisation $s=t-t^0$, hence it is also called an
    \emph{arc-length parametrisation}.
\end{definition}

\begin{definition}
    Let $g:(k,l)\to(a,b), \tau\to t$ be a smooth map with $g'(\tau)\not=0$ everywhere.
    The curve $\vec{\tilde x}(\tau)=\vec x(g(\tau))$ is a \emph{reparametrisation}
    of $\vec x(t)$. If $g'>0$ it preserves orientation.
\end{definition}

\begin{theorem}
    For any regular curve $\vec x:(a,b)\to\E^n$, there exists a reparametrisation
    which is unit-speed.
\end{theorem}

\begin{definition}
    A \emph{vector field along the curve $\vec x(t)$} in Euclidean space is a vector
    function $\vec v:(a,b)\to\E^n$.
\end{definition}

\begin{definition}
    The \emph{unit tangent} vector field along a regular curve $\vec x(t)$ is 
    \begin{align*}
        \vec T(t) = \frac{\vec x'(t)}{v(t)}.
    \end{align*}
    Thus, for a unit-speed curve $\vec x(s)$ it is simply $\vec T(s)=\vec x'(s)$.
\end{definition}

\begin{definition}
    For a unit-speed curve $\vec x(s)$ the curvature $\kappa(s)$ is defined by
    \begin{align*}
        \kappa(s) = \abs{\vec T'(s)}.
    \end{align*}
    For a general parametrisation thsi becomes $\kappa(t)=\abs{\vec T(t)}/v(t)$.
\end{definition}

\subsection{The Frenet-Serret frame and the structure equations}

\begin{definition}
    A unit-speed curve $\vec x(s)$ is \emph{biregular} if $\kappa(s)\not=0$ for
    all values of $s$.
\end{definition}

\begin{definition}
    The \emph{principal normal} along a unit-speed biregular curve $\vec x(s)$ is
    \begin{align*}
        \vec N(s) = \frac{\vec T'(s)}{\abs{\vec T'(s)}}=\frac{\vec T'(s)}{\kappa}.
    \end{align*}
    The \emph{binormal} vector field along $\vec x(s)$ is
    \begin{align*}
        \vec B = \vec T \times \vec N.
    \end{align*}
\end{definition}

\begin{proposition}
    The vector fields $\{\vec T(s), \vec N(s), \vec B(s)\}$ along a biregular curve
    $\vec x(s)$ are an orthonormal basis for $\E^3$ for each $s$. This is called the
    \emph{Frenet-Serret frame} of $\vec x(t)$.
\end{proposition}

\begin{definition}
    The \emph{torsion} of a biregular unit-speed curve $\vec x(t)$ is defined by
    \begin{align*}
        \tau = - \vec B' \cdot \vec N.
    \end{align*}
\end{definition}

\begin{theorem}
    Let $\vec x$ be a unit-speed biregular curve in $\E^3$. The Frenet-Serret
    frame $\vec T(s), \vec N(s), \vec B(s)$ along $\vec x(s)$ satisfies
    \begin{align*}
        \begin{pmatrix}
            \vec T\\
            \vec N\\
            \vec B
        \end{pmatrix}'
        = \begin{pmatrix}
            0       & \kappa & 0 \\
            -\kappa & 0      & \tau \\
            0       & -\tau  & 0
        \end{pmatrix}
        \begin{pmatrix}
            \vec T\\
            \vec N\\
            \vec B
        \end{pmatrix}.
    \end{align*}
    This is called the \emph{structure equation for a unit-speed space curve},
    or sometimes the "Frenet-Serret equation".
\end{theorem}

\begin{theorem}
    For a general parametrisation $\vec x(t)$ of a biregular space curve the structure
    equation becomes 
    \begin{align*}
        \begin{pmatrix}
            \vec T\\
            \vec N\\
            \vec B
        \end{pmatrix}'
        = v\begin{pmatrix}
            0       & \kappa & 0 \\
            -\kappa & 0      & \tau \\
            0       & -\tau  & 0
        \end{pmatrix}
        \begin{pmatrix}
            \vec T\\
            \vec N\\
            \vec B
        \end{pmatrix}.
    \end{align*}
    where $v$ is the speed of the curve.
\end{theorem}

\begin{theorem}
    A biregular curve $\vec x(t)$ is a plane curve if and only if $\tau=0$ everywhere.
\end{theorem}

\begin{theorem}
    The curvature and torsion of a biregular space curve in any parametrisation can be
    computed by
    \begin{align*}
        v^3\kappa = \abs{\vec x' \times \vec x''},\hs
        v^6\kappa^2\tau = (\vec x' \times \vec x'') \cdot \vec x'''.
    \end{align*}
\end{theorem}

\subsection{The equivalence problem}

\begin{definition}
    An \emph{isometry} of $\E^3$ is a map $\E^3\to\E^3$ given by
    \begin{align*}
        \vec x \mapsto A\vec x + \vec b
    \end{align*}
    where $A$ is an orthogonal matrix and $\vec b$ is a fixed vector. If $\det A = 1$,
    so that $A$ is a rotation matrix, then the isometry is said to be an \emph{Euclidean motion}
    or a \emph{rigid motion}. If $\det A = -1$ the isometry is orientation-reversing.
\end{definition}

\begin{proposition}
    The curvature and torsion of a biregular curve are unchanged by a Euclidean motion.
    In other words, taking a unit speed parametrisation, if
    \begin{align*}
        \vec{\hat x}(s) = A\vec x(s) + \vec b,
    \end{align*}
    where $A$ is a fixed rotation matrix and $\vec b$ is a fixed vector, then
    \begin{align*}
        \hat\kappa(s) = \kappa(s), \hs \hat\tau(s)=\tau(s).
    \end{align*}
    If instead $\det A=-1$ then
    \begin{align*}
        \hat\kappa(s)=\kappa(s), \hs \hat\tau(s) = -\tau(s).
    \end{align*}
\end{proposition}

\begin{theorem}[The fundamental theorem of curves]
    If two biregular space curves have the same curvature and torsion then they differ at most
    by a Euclidean motion.
\end{theorem}

\begin{corollary}
    Every biregular curve with $\kappa,\tau$ both constant is a helix, unless $\tau=0$ in
    which case it is a circle.
\end{corollary}

\section{Vector fields and one-froms}

\subsection{Tangent space and vector fields}

\begin{definition}
    If $\vec v$ is a vector at $p\in\E^3$ and $f:\E^3\in\R$ is a differentiable function,
    the directional derivative of $f$ along $\vec v$ at $p$ is
    \begin{align*}
        (D_{\vec v}f)_p = (\vec v \cdot \grad f)(p) = \sum_{k=1}^n v^k \eval{\frac{\p f}{\p x^k}}{p}
    \end{align*}
\end{definition}

\begin{definition}
    For each $p\in D$, we define the \emph{tangent space $T_pD$ to $D$ at $p$} as the
    set of all derivative operators at $p$, called \emph{tangent vectors} at $p\in D$,
    \begin{align*}
        \sum_{i=1}^n v^i \eval{\frac{\p}{\p x^i}}{p},\hs v^i\in\R.
    \end{align*}
    A smooth \emph{vector field} on $D$ is a smooth map $v$ which assigns to each $p\in D$
    an element in $T_p D$; in other words, it is a differential operator
    \begin{align*}
        v = \sum_{i=1}^n v^i \frac{\p}{\p x^i},
    \end{align*}
    where now each $v^i:D\to\R$ is a smooth function.
\end{definition}

\begin{definition}
    If $v,w$ are vector fields
    \begin{align*}
        v = \sum_i v^i \frac{\p}{\p x^i}\hs\text{and}\hs w = \sum_i w^i \frac{\p }{\p x^i}
    \end{align*}
    and if $\lambda\in\R$, then $\lambda v$ and $v+w$ are vector fields defined by
    \begin{align*}
        \lambda v = \sum_i \lambda v^i \frac{\p }{\p x^i}\hs\text{and}\hs
        v+w = \sum_i (v^i+w^i)\frac{\p }{\p x^i}.
    \end{align*}
    Furthermore, if $f$ is a function then we can define a vector field $fv$ by
    \begin{align*}
        fv = \sum_i fv^i\frac{\p }{\p x^i}.
    \end{align*}
\end{definition}

\begin{theorem}[Notes 4.4]
    For any $D\subseteq \R^n$ and $p\in D$ the tangent space $T_pD$ is a real vector space
    of dimension $n$. The set of tangent vectors 
    $\left\lbrace\eval{\frac{\p }{\p x^1}}{p}, ...,\eval{\frac{\p }{\p x^n}}{p}\right\rbrace$
    is a basis of $T_pD$.
\end{theorem}

\begin{definition}
    Given a vector field $v$ and a smooth function $f$, we can define a function 
    \begin{align*}
        v(f) = \sum_i v^i\frac{\p f}{\p x^i}.
    \end{align*}
\end{definition}

\begin{proposition}
    For all vector fields $v,w$ and function $f,g$, the following identities hold:
    \begin{align*}
        (v+w)(f) &= v(f) + w(f)\\=ja\\
        (fv)(g)  &= fv(g) \\
        v(f+g)   &= v(f) + v(g) \\
        v(fg)    &= gv(f) + fv(g)
    \end{align*}
\end{proposition}

\begin{definition}
    A \emph{curve} in $D$ given by
    \begin{align*}
        c(t) = (x^1(t), x^2(t), ..., x^n(t))
    \end{align*}
    where each $x^i:(a,b)\to\R$ is a smooth function, such that its velocity
    \begin{align*}
        c'(t)=(x^1(t), x^2(t), ..., x^n(t))
    \end{align*}
    where each $x^i:(a,b)\to\R$ is a smooth function, such that its velocity
    is non-vanishing for all $t\in(a,b)$. 
    We say a curve $c$ \emph{passes through $p\in D$} if $c(0)=p$.
\end{definition}

\begin{proposition}
    Let $c:(a,b)\to D$ be a curve that passes through $p$. There exists a unique
    $c'_p\in T_p D$ such that for any smooth function $f:D\to\R$
    \begin{align*}
        c'_p(f) = \eval{\frac{d}{dt}}{t=0} f(c(t)).
    \end{align*}
\end{proposition}

\begin{corollary}
    There is a one-to-one correspondence between velocities of curves that pass through 
    $p\in D$ and tangent vectors $T_pD$. By (standard) abuse of notation sometimes 
    we denote $c'_p$ by the corresponding velocity $c'(0)$.
\end{corollary}

\subsection{One-forms}

\begin{definition}
    A \emph{1-form at $p\in D$} is a linear map $\alpha:T_pD \to \R$. This means, for all
    $\vec v, \vec w\in T_pD$ and $a\in\R$,
    \begin{align*}
        \alpha(\vec v + \vec w) = \alpha(v) + \alpha(w),\hs \alpha(a\vec v) = a\alpha(\vec v    la.)
    \end{align*}
    The set of 1-forms at $p\in D$, denoted by $T^*_p D$ is called the \emph{dual vector space} of $T_p D$.
\end{definition}

\begin{definition}
    We define 1-forms $\eval{dx^j}p$ at each $p\in D$ by their action on the basis
    $\left\lbrace\eval{\frac{\p }{\p x^i}}p\right\rbrace$:
    \begin{align*}
        \eval{dx^j}p\left(\eval{\frac{\p }{\p x^k}}p\right)= \begin{cases}
            1, &\text{if } j = k\\
            0, &\text{otherwise}.
        \end{cases}
    \end{align*}
    Equivalently, $\eval{dx^i}{p}$ are defined by their action on an arbitrary tangent vector
    $\vec v = \sum_{i=1}^n v^i\eval{\frac{\p }{\p x^i}}p$:
    \begin{align*}
        \eval{dx^j}{p}(\vec v) = v^j.
    \end{align*}
\end{definition}

\begin{definition}
    A \emph{differential 1-form} on $D$ is a smooth map $\alpha$ which assigns to each $p\in D$
    a 1-form in $T^*_pD$; it can be written as
    \begin{align*}
        \alpha = \sum_{i=1}^n\alpha_i dx^i
    \end{align*}
    where $\alpha_i:D\to\R$ are smooth functions. Given 1-forms $\alpha=\sum_i\alpha_idx^i$ and 
    $\beta=\sum_i\beta_idx^i$ and a function $f$ on $D$, addition and scalar multiplication of 
    1-forms are given by
    \begin{align*}
        \alpha + \beta = \sum_i (\alpha_i + \beta_i)dx^i,\hs f\alpha = \sum_i (f\alpha_i)dx^i.
    \end{align*}
\end{definition}

\begin{lemma}
    If $\alpha=\sum_i\alpha_i dx^i$ and $\vec v = \sum_i v^i\frac{\p }{\p x^i}$, then $\alpha(\vec v)$
    is the smooth function
    \begin{align*}
        \alpha(\vec v) = \sum_{i=1}^n \alpha_i v^i.    
    \end{align*}
\end{lemma}

\begin{definition}
    Given a smooth function $f$ on $D$, its \emph{exterior derivative} is the 1-form $df$ defined by
    \begin{align*}
        (df)(v) = v(f)
    \end{align*}
    for any vector field $v$. Equivalently,
    \begin{align*}
        df=\sum_{i=1}^n \frac{\p f}{\p x^i}dx^i.
    \end{align*}
\end{definition}

\begin{proposition}
    For any functions $f,g$ on $D$, we have 
    \begin{itemize}
        \item $d(f+g)=df + dg$,
        \item $d(fg) =fdg + gdf$,
        \item $df=0$ if and only if $f$ is constant.
    \end{itemize}
\end{proposition}

\subsection{Line integrals}

\begin{definition}
    Let $c:[a,b]\to D$ be a curve and $\alpha$ a 1-form on $D$. The \emph{integral of $\alpha$ over
    the curve $c$} is 
    \begin{align*}
        \int_c \alpha = \int_a^b \alpha(c'(t))\:dt
    \end{align*}
    where $c'(t)$ is the tangent vector field to the curve.
\end{definition}

\begin{proposition}
    Let $c:[a,b]\to D$ be a curve in $D$ and let $f$ a function on $D$. Then
    \begin{align*}
        \int_c df = f(c(b)) - f(c(a)).
    \end{align*}
\end{proposition}

\begin{corollary}
    If $c$ is a closed curve, i.e. $c(a)=c(b)$, then $\int_c df = 0$ for any function $f$.
\end{corollary}

\begin{proposition}
    If $\tilde c$ is a reparametrisation of $c$ and $\alpha$ is a 1-form then 
    \begin{align*}
        \int_c \alpha = \pm \int_{\tilde c} \alpha,
    \end{align*}
    where the plus sign is taken when the reparametrisation is orientation-preserving,
    and the minus sign if it is orientation-reversing.
\end{proposition}

\begin{definition}
    The \emph{pull-back} of $\alpha$ by the map $c:[a,b]\to D$ is the 1-form on the
    interval $[a,b]$ defined by
    \begin{align*}
        c^*\alpha :=  \alpha(c'(t))dt.
    \end{align*}
    The integral of $\alpha$ over $c$ may now be thought of as the integral of the pull-back 
    $c^*\alpha$ over the interval $[a,b]$.
\end{definition}

\section{Differential forms}

\subsection{The exterior algebra of differential forms}

\begin{definition}
    A 2-form at $p\in D$ is a map $\alpha:T_pD\times T_p D \to \R$ which is linear in each argument
    and alternating $\alpha(v,w)=-\alpha(w,v)$ for all $v,w\in T_pD$. More generally, a $k$-form at
    $p\in D$ is a map of $k$ vectors in $T_pD$ to $\R$ which is \emph{multilinear} and \emph{alternating}.
\end{definition}

\begin{definition}
    The \emph{wedge product} $\alpha \wedge \beta$ of $1$-forms $\alpha$ and $\beta$ is a $2$-form defined
    by the following bilinear and alternating map,
    \begin{align*}
        (\alpha\wedge\beta)(v,w) = \alpha(v)\beta(w) - \alpha(w)\beta(v) = \begin{vmatrix}
            \alpha(v) &\alpha(w) \\
            \beta(v) &\beta(w)
        \end{vmatrix}.
    \end{align*}
    More generally, the wedge product of $k$ $1$-forms $\alpha^1,\alpha^2,...,\alpha^k$ can be defined as a
    map acting on $k$ vectors $v_1,v_2,...,v_k$
    \begin{align*}
        (\alpha^1\wedge\cdots\wedge\alpha^k)(v_1,...,v_k) = \begin{vmatrix}
            \alpha^1(v_1) &\cdots &\alpha^1(v_k)\\
            \vdots &\ddots &\vdots \\ 
            \alpha^k(v_1) &\cdots &\aleph^k(v_k)
        \end{vmatrix}.
    \end{align*}
    The resulting map is linear in each vector separately and changes sign if any pair of vectors is exchanged.
    Hence it defines a $k$-form.
\end{definition}

\begin{definition}
    By a \emph{multi-index $I$ of length $\abs{I}=k$} we shall mean an increasing sequence
    $I=(i_1,...,i_k)$ of integers $1\leq i_1<i_2<\cdots<i_k\leq n$. We will write
    \begin{align*}
        dx^I = dx^{i_1} \wedge \cdots \wedge dx^{i_k}.
    \end{align*}
\end{definition}

\begin{definition}
    A \emph{differential $k$-form} or a \emph{differential form of degree $k$} on $D$ is a
    smooth map $\alpha$ which assigns to each $p\in D$ a $k$-form at $p$; it can be written as
    \begin{align*}
        \alpha=\alpha_Idx^I
    \end{align*}
    where $\alpha_I:D\to\R$ are smooth functions, and the sum happens over all multi-indices
    $I$ with $\abs{I}=k$. Given two differential $k$ forms $\alpha,\beta$ and a function $f$ 
    the differential $k$-forms $\alpha+\beta$ and $f\alpha$ are 
    \begin{align*}
        \alpha + \beta = (\alpha_I + \beta_I)dx^I, \hs f\alpha = f\alpha_I dx^I.
    \end{align*}
    The set of $k$-forms on $D$ is denoted $\Omega^k(D)$. By convention, a \emph{zero-form}
    is a function. If $k>n$ then $\Omega^k(D)=\emptyset$.
\end{definition}

\begin{definition}
    We extend $\wedge$ linearly in order to define the \emph{wedge product} of a $k$-form $\alpha$
    and an $l$-form $\beta$. Explicitly,
    \begin{align*}
        (\alpha_Idx^I) \wedge (\beta_Jdx^J) = \alpha_I\beta_Jdx^I\wedge dx^J.
    \end{align*}
    Then the wedge product defines a bilinear map 
    \begin{align*}
        \wedge : \Omega^k(D) \times \Omega^l(D) \to \Omega^{k+l} (D).
    \end{align*}
\end{definition}

\begin{proposition}
    For all $\alpha\in\Omega^k(D)$ and $\beta\in\Omega^l(D)$,
    \begin{align*}
        \alpha\wedge\beta = (-1)^{kl} \beta \wedge \alpha.
    \end{align*}
    Note that if $f\in\Omega^0(D)$ is a function, then we define 
    \begin{align*}
        f\wedge\beta = f\beta
    \end{align*}
    and the formula still applies.
\end{proposition}

\subsection{The exterior derivative}

\begin{definition}
    If $\alpha=\alpha_Idx^I\in\Omega^k(D)$, then its \emph{exterior derivative} $d\alpha\in\Omega^{k+1}(D)$
    is 
    \begin{align*}
        d\alpha = d\alpha_I \wedge dx^I
    \end{align*}
    where $d\alpha_I$ denotes the exterior derivative of the function $\alpha_I$. Note that 
    a consequence of this definition is $d(dx^I) = d(1)\wedge dx^I=0$.
\end{definition}

\begin{theorem}
    The exterior derivative $d:\Omega^k(D)\to \Omega^{k+1}(D)$ is a linear map satisfying
    the following properties 
    \begin{enumerate}
        \item $d$ obeys the graded derivation property, for any $\alpha\in\Omega^k(D)$ \begin{align*}
            d(\alpha \wedge \beta) = d\alpha\wedge\beta +(-1)^k\alpha\wedge d\beta.
        \end{align*}
        \item $d(d\alpha)=0$ for any $\alpha\in\Omega^k(D)$, or more compactly $d^2=0$.
    \end{enumerate}
\end{theorem}

\begin{definition}
    A form $\alpha\in\Omega^k(D)$ is said to be \emph{closed} if $d\alpha = 0$ and it is said
    to be \emph{exact} if $\alpha=d\beta$ for some $\beta\in\Omega^{k-1} (D)$.
\end{definition}

\begin{lemma}[Poincar\'e]
    Every closed differential form on $\R^n$ is exact.
\end{lemma}

\subsection{Integration in $\R^n$}

\begin{definition}
    The \emph{standard orientation} is defined by
    \begin{align*}
        dx^1\wedge dx^2 \wedge \cdots \wedge dx^n.
    \end{align*}
    Coordinates $(y^1, ..., y^n)$ are said to be \emph{oriented} on $D$ iff
    $dy^1\wedge\cdots\wedge dy^n$ is a positive multiple of $dx^1\wedge\cdots\wedge dx^n$
    for all $x\in D\subseteq \R^n$.
\end{definition}

\begin{proposition}
    Let $(x^1,...,x^n)$ be oriented coordinates for $\R^n$. Let $(y^1, ..., y^n)$ be 
    smooth functions on $\R^n$. Then
    \begin{align*}
        dy^1\wedge\cdots\wedge dy^n = \frac{\p(y^1,...,y^n)}{\p(x^1,...,x^n)}dx^1\wedge\cdots\wedge dx^n,
    \end{align*}
    where the factor on the RHS is the Jacobian matrix of the coordinate transformation.
\end{proposition}

\begin{definition}
    Let $(x^1,...,x^n)$ be oriented coordinates on $D\subseteq \R^n$ and write 
    \begin{align*}
        \omega = f(x^1, ..., x^n)dx^1\wedge\cdots\wedge dx^n\in\Omega^n(D).
    \end{align*}
    Then the \emph{integral of $\omega$ over $D$} is defined by
    \begin{align*}
        \int_D \omega = \int_D f(x^1,...,x^n)dx^1\cdots dx^n,
    \end{align*}
    where the RHS is now the usual multi-integral of several variable calculus.
\end{definition}

\section{Surfaces}

\subsection{Regular surfaces}

\begin{definition}
    A \emph{local surface} in $\E^3$ is a smooth, injective, map $\vec x: D\to\E^3$
    with a continuous inverse $\inv{\vec x} : \vec x(D) \to D$. We may denote the image
    $\vec x (D)$ by $S$.
\end{definition}

\begin{definition}
    Given a local surface $\vec x$ we define
    \begin{align*}
        \vec x_{u^1} = \frac{\p x^i}{\p u^1}\frac{\p}{\p x^i}, \hs 
        \vec x_{u^2} = \frac{\p x^i}{\p u^2}\frac{\p}{\p x^i}.
    \end{align*}
    For every point $p\in D$, these are vectors in $T_{\vec x()}\E^3$, which we will 
    identify with $\E^3$ itself. We say that a local surface $\vec x$ is \emph{regular
    at $p\in D$} if $\vec x_{u^1}(p)$ and $\vec x_{u^2}(p)$ are linearly independent.
    A local surface is \emph{regular} if it is regular at $p$ for all $p\in D$.
\end{definition}

\begin{definition}
    $\Sigma\subset\E^3$ is a \emph{regular surface} if for each $p\in\Sigma$ there
    exists are regular local surface $\vec x:D\to\E^3$ such that $\vec p\in\vec x(D)$
    and $\vec x(D)=U\cap\Sigma$ for some open set $U\subset\R^3$.
\end{definition}

\begin{definition}
    At a regular point on a local surface, the plane spanned by $\vec x_{u^1}(p)$ 
    and $\vec x_{u^2}(p)$ is the \emph{tangent plane to the surface at $\vec x(p)$},
    which we denote by $T_{\vec x(p)}S$. At a regular point, the \emph{unit normal}
    to the surface is 
    \begin{align*}
        \vec N(p)=\frac{\vec x_{u^1}(p)\times\vec x_{u^2}(p)}{\abs{\vec x_{u^1}(p)\times\vec x_{u^2}(p)}}.
    \end{align*}
    Clearly, $\vec N(p)$ is orthogonal to the tangent plane $T_{\vec x(p)}S$.
\end{definition}

\begin{proposition}
    Given a regular surface the map $\vec N:D\to\E^3$ is a smooth function whose
    image lies in a unit sphere $S^2\subset\E^3$. The map $\vec N$ is called the
    local \emph{Gauss map}.
\end{proposition}

\subsection{Standard surfaces}

\begin{definition}
    Let $g:D\to\R$ be a smooth function. The \emph{graph of} $g$ is the local surface 
    defined by 
    \begin{align*}
        \vec x(u^1, u^2) = (u^1, u^2, g(u^1, u^2)).
    \end{align*}
\end{definition}

\begin{proposition}
    Graphs are always regular, and the unit normal is given by
    \begin{align*}
        \vec N = \frac{1}{\sqrt{1 + \left(\frac{\p g}{\p u^1}\right) + \left(\frac{\p y}{\p u^2}\right)}}\begin{pmatrix}
            -\frac{\p g}{\p u^1} \\ \frac{-\p g}{\p u^2} \\ 1
        \end{pmatrix}.
    \end{align*}
\end{proposition}

\begin{definition}
    An \emph{implicitly defined surface $\Sigma$} is the zero set of a function
    $f:\E^3\to\R$, i.e., $\Sigma=\inv f(0)$.
\end{definition}

\begin{proposition}
    An implicitly defined surface $\Sigma = \inv f(0)$, such that $df\not=0$
    everywhere on $\Sigma$, is a regular surface.
\end{proposition}

\begin{definition}
    A \emph{surface of revolution} with profile curve $f(u)>0$ is a local 
    surface of the form 
    \begin{align*}
        \vec x(u, \phi) = (f(u)\cos\phi,f(u)\sin\phi,u).
    \end{align*}
    A surface of revolution can be constructed by rotating a curve $x^1=f(x^3)$
    around the $x^3$-axis in $\R^3$. It thus has cylindrical symmetry.
\end{definition}

\begin{definition}
    A \emph{ruled surface} is a surface of the form 
    \begin{align*}
        \vec x(u,v)= \vec z(u) + v\vec p(u).
    \end{align*}
    Notice that curves of constant $u$ are straight lines in $\E^3$ through 
    $\vec z(u)$ in the direction $\vec p(u)$.
\end{definition}

\section{The fundamental forms}

\subsection{Symmetric tensors}

\begin{definition}
    A symmetric bilinear form on $T_pD$ is a bilinear map $B:T_pD\times T_pD 
    \to\R$ such that $B(v,w)=B(w,v)$ for all $v,w\in T_pD$. Given two $1$-forms
    $\alpha$, $\beta$ at $p\in D$ define a symmetric bilinear form $\alpha\beta$
    on $T_pD$ by 
    \begin{align*}
        (\alpha\beta)(v, w) = \frac{1}{2}(\alpha(v)\beta(w) \alpha(w)\beta(v)) 
    \end{align*}
    where $v,w\in T_pD$. Note that $\alpha\beta=\beta\alpha$ and we write $\alpha^2$
    for $\alpha\alpha$.
\end{definition}

\begin{definition}
    A symmetric tensor on $D$ is a map which assigns to each $p\in D$ a symmetric
    bilinear form on $T_pD$; it can be written as
    \begin{align*}
        B=B_{ij}dx^idx^j
    \end{align*}
    where $B_{ij}$ are smooth functions on $D$.
\end{definition}

\begin{definition}
    A (Riemannian) \emph{metric} on $D$ is a symmetric tensor $g=g_{ij}dx^idx^j$ 
    which is positive definite at each point; $g(\vec v,\vec v)\geq 0$ for all
    $\vec v \in T_pD$, with equality if and only if $v=0$. Equivalently, it is a 
    choice for each $p\in D$ of an inner product on $T_pD$.
\end{definition}

\subsection{The first fundamental form}

\begin{proposition}
    Consider a regular local surface defined by $\vec x: D\to \E^3$. The linear map
    $\eval{d\vec x}{p}:T_pD\to T_{\vec x(p)}S$ is a bijection.
\end{proposition}

\begin{definition}
    Given a regular local surface $\vec x: D\to\E^3$, the \emph{first fundamental form}
    is defined by 
    \begin{align*}
        I = d\vec x \cdot d\vec x.,
    \end{align*}
    where we have introduced the notation $d\vec x\cdot d\vec x=(dx^1)^2+(dx^2)^2+(dx^3)^2$.
\end{definition}

\begin{proposition}
    The first fundamental form is a metric on $D$.
\end{proposition}

\begin{proposition}
    The first fundamental form of a regular local surface $\vec x:D\to\E^3$ is
    \begin{align*}
        I = E(du^1)^2 + 2Fdu^1du^2+ G(du^2)^2
    \end{align*}
    where $E,F,G$ are functions on $D$ given by
    \begin{align*}
        E=\vec x_{u^1} \cdot \vec x_{u^1},\hs 
        F=\vec x_{u^1} \cdot \vec x_{u^2},\hs 
        G=\vec x_{u^2} \cdot \vec x_{u^2}.
    \end{align*}
\end{proposition}

\subsection{The second fundamental form}


\begin{definition}
    Given a regular local surface $\vec x: D\to\E^3$, the \emph{second fundamental form}
    is defined by 
    \begin{align*}
        \vec\I = -d\vec x\cdot d\vec N,
    \end{align*}
    with the dot product interpreted as above.
\end{definition}

\begin{proposition}
    The second fundamental form is given by
    \begin{align*}
        \vec\I = l(du^1)^2 + 2mdu^1du^2+n(du^2)^2
    \end{align*}
    where $l,m,n$ are functions on $D$ given by
    \begin{align*}
        l=-\vec x_{u^1}\cdot\vec N_{u^1}, \hs 
        m=-\vec x_{u^1}\cdot\vec N_{u^2}, \hs 
        n=-\vec x_{u^2}\cdot\vec N_{u^2}.
    \end{align*}
\end{proposition}

\section{Curvature of surfaces}

\subsection{Bilinear algebra}

\begin{definition}
    Let $A,B$ be two symmetric bilinear forms. Then the \emph{eigenvalues of $B$ with respect to $A$} 
    are roots of the polynomial 
    \begin{align*}
        \det(B-\lambda A) = 0,
    \end{align*}
    where by abuse of notation $A,B$ are the matrices of the symmetric forms $A,B$ relative to the 
    basis.
\end{definition}

\begin{lemma}
    Let $A,B$ be two symmetric bilinear forms. Then the eigenvalues and eigenvectors of $B$ with respect 
    to $A$ are independent of the basis chosen.
\end{lemma}

\begin{proposition}
    If $A:D\times D\to\R$ is positive definite there exists a basis $\{e_1, ..., e_n\}$ of $V$ such that
    \begin{enumerate}
        \item $\{e_1,...,e_n\}$ is orthonormal with respect to $A$,
        \item each $e_k$ is an eigenvector of $B$ with respect to $A$ with a real eigenvalue.
    \end{enumerate}
\end{proposition}

\subsection{Gauss and mean curvatures}

\begin{definition}
    The eigenvalues $k_1,k_2$ of $\I$ with respect to $I$ are \emph{principal curvatures} of the surface.
    The corresponding eigenvectors are the \emph{principal directions} of the surface. Hence the
    principal curvatures are the roots of the polynomial $\det(\I-\lambda I)=0$. Their mean is the
    \emph{mean curvature} of the surface:
    \begin{align*}
        H=\frac{k_1+k_2}{2},
    \end{align*}
    and their product is the \emph{Gauss curvature}:
    \begin{align*}
        K = k_1k_2.
    \end{align*}
    The principal curvatures may vary with position and are thereby (smooth) functions on $D$.
\end{definition}

\section{The meaning of curvature}

\subsection{Curves on surfaces}

Let $c:[a,b]\to D$ be a curve with $c(t)=(u(t),v(t))$ and let $\vec x: D\to\E^3$ be a 
local surface.

\begin{lemma}
    We have 
    \begin{align*}
        \vec x' := \frac{d}{dt}\vec x(c(t)) = d\vec x(c')\hs\text{and}\hs 
        \vec N' := \frac{d}{dt}\vec N(c(t)) = d\vec N(c').
    \end{align*}
    where $c'$ is the tangent vector to the curve $c$ in $D$, so that 
    $c'=u'\frac{\p }{\p u}+v'\frac{\p }{\p v}$.
\end{lemma}

\begin{proposition}
    The arclength of the curve $\vec x(c(t))$, $t\in[a,b]$, lying on a surface is 
    \begin{align*}
        s = \int_a^b \sqrt{I(c',c')}\:dt
    \end{align*}
    where $I$ is the first fundamental form of the surface.
\end{proposition}

\begin{proposition}
    For a curve lying on a surface,
    \begin{align*}
        \vec N\cdot \frac{d^2}{dt^2}\vec x(c(t)) = \I(c',c').
    \end{align*}
\end{proposition}

\subsection{Invariance under Euclidean motions}

\begin{theorem}
    Let $\vec x: D\to\E^3$ and $\vec{\hat x}:D\to\E^3$ be two surfaces related by a 
    Euclidean motion, so 
    \begin{align*}
        \vec{\hat x} (u,v) = A\vec x(u,v)+\vec a
    \end{align*}
    forall $(u,v)\in D$ where $A$ is an orthogonal matrix with $\det A=1$ and $\vec a\in\E^3$
    Then we have 
    \begin{align*}
        \hat I = I \hs \text{and} \hs \hat\I = \I
    \end{align*}
    and hence in particular
    \begin{align*}
        \hat H = H \hs \text{and} \hs \hat K = K.
    \end{align*}
\end{theorem}

\begin{theorem}[Fundamental theorem of surfaces]
    The converse of the above is also true: two surfaces with the same first and second 
    fundamental forms must be related by a Euclidean motion.
\end{theorem}

\subsection{Taylor series}

\begin{proposition}
    Let $\vec p \in \vec x(D)$ be apoint on a regular local surface. By a Euclidean motion,
    choose $\vec p$ to be at the origin and the unit normal at that point to be along the positive
    $x^3$-axis so $T_{\vec p}S$ is the $(x^1,x^2)$ plane. Near $\vec p$ we can parametrise 
    the surface as the graph 
    \begin{align*}
        \vec x(u,v)=\begin{pmatrix}
            u \\ v \\ f(u,v)
        \end{pmatrix}
    \end{align*}
    where at the origin
    \begin{align*}
        I_{(0,0)}&=(du)^2+(dv)^2,\\
        \I_{(0,0)}&=f_{uu}(0,0)(du)^2+2f_{uv}(0,0)dudv+f_{vv}(0,0)(dv)^2.
    \end{align*}
\end{proposition}

\begin{proposition}
    In setup of the previous Proposition, suppose the $x^1$, $x^2$ axes correspond 
    to the principal directions. Then the Taylor series of the surface near the origin
    is 
    \begin{align*}
        x^3 = f(x^1, x^2) = \frac{k_1}{2}(x^1)^2 + \frac{k_2}{2}(x^2)^2 + \text{higher order terms}
    \end{align*}
    where $k_1$ and $k_2$ are the principal curvatures at $\vec p$.
\end{proposition}

\begin{corollary}
    If $K>0$ at a point then the surface is bowl-shaped around that point and if $K<0$ the it is
    saddle-shaped. If $K=0$ at a point then there are two possibilities: if both principal
    curvatures are zero, then the surface is planar around that point; otherwise it's like 
    a bent plane.
\end{corollary}

\begin{definition}
    A surface with $K=0$ everywhere is called \emph{flat}.
\end{definition}

\begin{definition}
    A \emph{minimal surface} is one with mean curvature $H=0$ everywhere.
\end{definition}

\begin{lemma}
    A minimal surface has $k_1=-k_2$ and thus $K \leq 0$ everywhere.
\end{lemma}

\subsection{Umbilical points and surfaces}

\begin{theorem}
    A regular surface has $\I=0$ iff it is (a piece of) a plane.
\end{theorem}

\begin{theorem}
    A regular local surface has $\I = \lambda I$ where $\lambda \not=0$ is a constant iff 
    the surface is (a piece of) a sphere with radius $1/\abs\lambda$.
\end{theorem}

\begin{definition}
    An \emph{umbilical point} on a surface is a point where $k_1=k_2$. Consequently all 
    directions are principal at such a point.
\end{definition}

\begin{lemma}
    For any surface $K\leq H^2$, with equality only at umbilical points.
\end{lemma}

\begin{theorem}
    A regular local surface for which every point is umbilical, is either part of a sphere 
    or a plane.
\end{theorem}

\section{Moving frames in Euclidean space}

\subsection{Moving frames in $\E^3$}

\begin{definition}
    A \emph{moving frame} for $\E^3$ on $D$ is a collection of maps $\vec e_i:D\to\E^3$
    for $i=1,2,3$ such that for all $u\in D$ the $\vec e_i(u)$ form an oriented orthonormal
    basis of $\E^3$.
\end{definition}

\begin{definition}
    If $\vec v:D\to\E^3$, given by 
    \begin{align*}
        \vec v(u)=\vec v(u^1,...,u^n)=\begin{pmatrix}
            v^1(u^1,...,u^n)\\ 
            v^2(u^1,...,u^n)\\
            v^3(u^1,...,u^n)
        \end{pmatrix}
    \end{align*}
    is an $\E^3$-valued function on $D$, then we write $d\vec v$ for its entry
    by entry exterior derivative 
    \begin{align*}
        d\vec v = \begin{pmatrix}
            dv^1 \\ dv^2 \\ dv^3
        \end{pmatrix}.
    \end{align*}
    We refer to an object like $d\vec v$ as a \emph{matrix of $1$-forms on $D$} or as a 
    \emph{$\E^3$-valued $1$-form on $D$}. If $\vec w$ is a vector field on $D$, then we write
    \begin{align*}
        d\vec v(w) = \begin{pmatrix}
            dv^1(\vec w) \\ dv^2(\vec w) \\ dv^3(\vec w)
        \end{pmatrix} : D\to\E^3.
    \end{align*}
\end{definition}

\subsection{Connection forms and the structure equations}

\begin{definition}
    The $1$-forms $\omega_i^j=\vec e_j\cdot d\vec e_i\in\Omega^1(D)$ are called the 
    \emph{connection $1$-forms} and by definition satisfy $d\vec e_i = \vec e_j\omega_i^j$.
\end{definition}

\begin{proposition}
    The connection $1$-forms $\omega_j^i$ are related by the antisymmetry property: 
    \begin{align*}
        \omega_i^j = -\omega_j^i
    \end{align*}
    for all $i,j$. In particular $\omega_i^i = 0$ for all $i$.
\end{proposition}

\begin{lemma}
    We denote 
    \begin{align*}
        \vec w = \begin{pmatrix}
            0           & \omega_2^1  & \omega_3^1 \\
            -\omega_2^1 & 0           & \omega_3^2 \\
            -\omega_3^1 & -\omega_3^2 & 0
        \end{pmatrix}
    \end{align*}
    and 
    \begin{align*}
        d\vec E = \begin{pmatrix}
            d\vec e_1 & d\vec e_2 & d\vec e_3
        \end{pmatrix}.
    \end{align*}
    Then 
    \begin{align*}
        d\vec E = \vec E\vec\omega.
    \end{align*}
\end{lemma}

\begin{theorem}
    The \emph{first sturcture equations} are 
    \begin{align*}
        d\vec\theta + \vec w \wedge \vec \theta = 0,
    \end{align*}
    or equivalently 
    \begin{align*}
        d\theta^i + \omega_j^i \wedge \theta^j = 0,\hs i=1,2,3.
    \end{align*}
\end{theorem}

\begin{theorem}
    The \emph{second structure equations} are 
    \begin{align*}
        d\vec\omega + \vec \omega \wedge\vec \omega = 0,
    \end{align*}
    or equivalently 
    \begin{align*}
        d\omega_j^i + \omega_k^i\wedge\omega_j^k=0.
    \end{align*}
\end{theorem}

\section{The structure equations for a surface}

\subsection{Adapted frames and the structure equations}

Let $\vec x: D \to \E^3$ be an oriented local surface.

\begin{definition}[Notes 11.1]
    A moving frame $(\vec e_1, \vec e_2, \vec e_3)$ for $\E^3$ on $D$
    is said to be adapted to the surface if $\vec e_3 = \vec N$.
\end{definition}

\begin{proposition}[Notes 11.2]
    The first and second structure equations for a regular local surface 
    with respect to an adapted frame, give the \emph{structure equations
    of a surface}:
    \begin{align*}
        d\theta^1 + \omega_2^1\wedge\theta^2 = 0\\
        d\theta^2 + \omega_1^2\wedge\theta^1 = 0\\
        \omega_1^3\wedge\theta^1 + \omega_2^3\wedge\theta^2 = 0\\
        d\omega_2^1+\omega_3^1\wedge\omega_2^3 = 0\\
        d\omega_3^1 + \omega_2^1 \wedge \omega_3^2 = 0\\
        d\omega_3^2 + \omega_1^2 \wedge \omega_3^1 = 0
    \end{align*}
\end{proposition}

\begin{lemma}[Notes 11.3]
    $\{\theta^1, \theta^2\}$ is a basis for $\Omega^1(D)$. 
\end{lemma}

\begin{lemma}
    There are unique functions $a,b,c$ on $D$ such that 
    \begin{align*}
        \omega_3^1 = a\theta^1+b\theta^2,\\
        \omega_3^2 = b\theta^1+c\theta^2.
    \end{align*}
\end{lemma}

\begin{proposition}
    The first and second fundamental forms are 
    \begin{align*}
        I = (\theta^1)^2 + (\theta^2)^2,\hs 
        \I = -a(\theta^1)^2-2b\theta^1\theta^2-c(\theta^2)^2.
    \end{align*}
    The Gauss curvature $K$ and the mean curvature $H$ are given by 
    \begin{align*}
        K=ac-b^2, \hs H=-\frac{1}{2}(a+c).
    \end{align*}
\end{proposition}

\begin{proposition}
    The Gauss equation is equivalent to 
    \begin{align*}
        d\omega_2^1 = K\theta^1\wedge\theta^2.
    \end{align*}
\end{proposition}

\subsection{Gauss's Theorema Egregium}

\begin{proposition}
    Let $\vec x:D\to\E^3$ be a regular local surface with first fundamental form 
    $I$ and $\theta^1$, $\theta^2$ be $1$-forms on $D$ such that 
    \begin{align*}
        I = (\theta^1)^2+(\theta^2)^2.
    \end{align*}
    Then there exists a unique adapted frame such that $\theta^1=\vec e_1\cdot d\vec x$
    and $\theta^2 = \vec e_2 \cdot d\vec x$.
\end{proposition}

\begin{definition}
    Two regular surfaces $\vec x,\tilde{\vec x}:D\to\E^3$ are \emph{isometric}
    if $I=\tilde I$.
\end{definition}

\begin{theorem}[Gauss's Theorema Egregium]
    Isometric surfaces have the same Gauss curvature.
\end{theorem}

\section{Geodesics}

\subsection{Geodesics in Euclidean space}

\begin{definition}
    Let $\vec x:[a,b]\to\E^n$ be a unit speed curve in Euclidean space joining two points 
    $\vec x(a),\vec x(b)\in\E^n$. Consider a $1$-parameter family of \emph{nearby} curves 
    \begin{align*}
        \vec x_\e (t) = \vec x(t) + \e\vec y(t)
    \end{align*}
    where $\vec x'(t) \cdot \vec y(t)  = 0$ and $\vec y(a)=\vec y(b)=\vec 0$.
    We call $\vec y(t)$ a \emph{connecting vector}.
\end{definition}

\begin{definition}
    We say a unit speed curve $\vec x(t)$ as above has a \emph{stationary length} if 
    the length of nearby curves $s_\e = \int_a^b \abs{\vec x'_c} dt$ satisfies 
    \begin{align*}
        \eval{\frac{ds_\e}{d\e }}{\e=0}=0
    \end{align*}
    for all connecting vectors $\vec y(t)$.
\end{definition}

\begin{proposition}
    A unit speed curve $\vec x(t)$ in Euclidean space has stationary length if and only if 
    it is the straight line joining the two end points.
\end{proposition}

\subsection{Geodesics on surfaces}

\begin{definition}
    A unit speed curve $\vec x(c(t))$ lying in a surface is \emph{geodesic} if its acceleration
    is everywhere normal to the surface, that is, 
    \begin{align*}
        \frac{d^2}{dt^2}\vec x(c(t)) = A(t)\vec N(c(t)).
    \end{align*}
    where $\vec N$ is the unit normal to the surface and $A$ is some function along the curve.
\end{definition}

\begin{proposition}
    A curve lying in a surface has stationary length if and only if it is a geodesic.
\end{proposition}

\begin{proposition}
    A curve $\vec x(c(t))$ lying in a surface is a geodesic if and only if, in an adapted frame 
    it obeys the \emph{geodesic equations}
    \begin{align*}
        \frac{d}{dt}(\theta^1 (c')) + \omega_2^1(c')\theta^2(c') = 0,\\
        \frac{d}{dt}(\theta^2 (c')) + \omega_1^2(c')\theta^1(c') = 0,
    \end{align*}
    and the \emph{energy equation}
    \begin{align*}
        (\theta^1(c'))^2 + (\theta^2(c'))^2 = 1.
    \end{align*}
\end{proposition}

\begin{proposition}
    The fact that the energy
    \begin{align*}
        E = (\theta^1(c'))^2 + (\theta^2(c'))^2
    \end{align*}
    is constant is a consequence of the geodesic equations. Setting it to one just means we get 
    curves of unit speed, rather than some constant speed.
\end{proposition}

\begin{proposition}
    Given a point $\vec p$ on a surface and a unit tangent vector $\vec v$ to the surface at $\vec p$,
    there exists a unique geodesic on the surface $t\mapsto \vec x(c(t))$ for $\abs t < \e$,
    such that $\vec x(0) =\vec p$ and $\vec x'(0) \vec v$.
\end{proposition}

\subsection{The hyperbolic plane}

\begin{definition}
    \emph{Two-dimensional hyperbolic space} is the upper half plane 
    \begin{align*}
        H = \{(x,y) : x > 0\}\subset\R^2
    \end{align*}
    equipped with the first fundamental form given by 
    \begin{align*}
        I = \frac{(dx)^2 + (dy)^2}{y^2}.
    \end{align*}
\end{definition}

\begin{theorem}
    Hyperbolic space has $K=-1$.    
\end{theorem}

\begin{theorem}[Hilbert]
    No complete regular surface of constant negative Gauss curvature exists in $\E^3$. 
\end{theorem}

\begin{theorem}[Nash]
    Every Riemannian manifold can be isometrically immersed in some $\E^n$. 
\end{theorem}

\section{Integration over surfaces}

\subsection{Integration of 2-forms over surfaces}

\begin{definition}
    Let $\vec x:D\to\R^3$ define a local surface 
    \begin{align*}
        \vec x(u,v)=(x^1(u,v), x^2(u,v), x^3(u,v)).
    \end{align*}
    Let 
    \begin{align*}
        \alpha = \alpha_{12}dx^1\wedge dx^2 + \alpha_{13}dx^1\wedge dx^3 + \alpha_{23}dx^2\wedge dx^3
    \end{align*}
    be a $2$-form on $\R^3$. We define the \emph{pull-back} $x^*\alpha$ of $\alpha$ by the map $x$
    to be the $2$-form on $D$ given by 
    \begin{align*}
        x^*\alpha = \alpha_{12}(\vec x(u,v))dx^1\wedge dx^2+ \alpha_{13}(\vec x(u,v))dx^1\wedge dx^3 + \alpha_{23}(\vec x(u,v))dx^2\wedge dx^3
    \end{align*}
    where here $dx^k$ is the exterior derivative of $x^k:D\to\R$.
\end{definition}

\begin{definition}
    Let $\vec x:D\to\R^3$ be a local surface and let $\alpha$ be a $2$-form on $\R^3$. We define 
    the \emph{integral} of $\alpha$ over the local surface to be 
    \begin{align*}
        \int_{\vec x(D)} \alpha = \int_D \vec x^*\alpha.
    \end{align*}
\end{definition}

\begin{definition}
    If $\Sigma$ is an oriented regular surface, the integral over $\Sigma$ of a $2$-form $\alpha$
    is defined by writing $\Sigma$ as a union of images of regular oriented local surfaces, and 
    adding the integrals of $\alpha$ over these local surfaces. 
\end{definition}

\begin{definition}
    A $1$-dimensional submanifold of $\R^3$ is a regular curve, a $2$-dimensional submanifold of 
    $\R^3$ is a regular surface, a $3$-dimensional submanifold of $\R^3$ is a domain.
\end{definition}

\begin{definition}
    Let $\Sigma$ be an oriented surface in $\R^3$ with boundary an oriented curve $C$.
    The \emph{induced orientation} on $C$ is such that if $(u_1,u_2)$ is an oriented coordinate
    chart on $\Sigma$ such that $u_1\leq 0$ and $u_1=0$ is the boundary, then $u_2$ is an
    oriented coordinate on $C$. 
    
    Let $M$ be a $3$-dimensional submanifold of $\R^3$ bounded by 
    an oriented surface $\Sigma$. The induced orientation on $\Sigma$ is such that $(u_1,u_2,u_3)$
    is an oriented coordinate chart on $M$ such that $u_1\leq 0$ and $u_1=0$ is the boundary,
    then $(u_2,u_3)$ are oriented coordinates on $\Sigma$.
\end{definition}

\begin{theorem}[Stokes]
    Let $\Sigma$ be a $k$-dimensional oriented closed and bounded submanifold in $\R^3$ with 
    boundary $\sigma\Sigma$ given the induced orientation and $\alpha\in\Omega^{k-1}(\Sigma)$.
    Then
    \begin{align*}
        \int_\Sigma d\alpha = \int_{\sigma\Sigma}\alpha.
    \end{align*}
\end{theorem}

\begin{corollary}[Green]
    Let $D\subset\R^2$ with boundary curve $C$. The induced orientation on $C$ from 
    the standard orientation on $D$ is anti-clockwise. Then Stokes' theorem for the $1$-form 
    $\alpha=P(x,y)dx + Q(x,y)dy$ is 
    \begin{align*}
        \int_C(Pdx + Qdy) = \int_D\left(\frac{\p Q}{\p x}-\frac{\p P}{\p y}\right)dx\wedge dy. 
    \end{align*} 
\end{corollary}

\subsection{Integration of functions over surfaces}

\begin{definition}
    Given a local surface $\vec x:D\to\E^3$, the integral of a 
    function $f:\E^3\to\R$ over $\vec x$ is 
    \begin{align*}
        \int_{\vec x(D)} = \int_D f(\vec x(u,v))\abs{\vec x_u \times \vec x_v}dudv.
    \end{align*}
\end{definition}

\begin{lemma}
    For a local surface $\abs{\vec x_u \times \vec x_v}=\sqrt{\det I}$. Hence the area 
    of $\vec x$ is 
    \begin{align*}
        A = \int_D \sqrt{\det I}dudv.
    \end{align*}
    Thus the area is an intrinsic property of the surface.
\end{lemma}

\begin{lemma}
    For a local surface in $\E^3$ with adapted frame,
    \begin{align*}
        \theta^1\wedge\theta^2 = \abs{\vec x_u \times \vec x_v} du \wedge dx.
    \end{align*}
\end{lemma}

\begin{corollary}
    Let $\vec x: D\to\E^3$ be a local surface and $f:\E^3\to\R$ be a function. Then 
    the integral of $f$ over the surface is given by 
    \begin{align*}
        \int_{\vec x(D)} f = \int_D f(\vec x(u,v))\theta^1\wedge\theta^2.
    \end{align*}
    In particular, 
    \begin{align*}
        A = \int_D \theta^1\wedge\theta^2
    \end{align*}
    gives the area of the local surface. 
\end{corollary}

\section{Gauss-Bonnet theorem}

\subsection{Geodesic triangles}

\begin{definition}
    A \emph{geodesic triangle} $\Delta$ in a surface is three points each joined to the other 
    two by geodesics. We assume our triangles are sufficiently small that they fit in the image 
    of a single local surface on which we have an adapted frame.
\end{definition}

\begin{proposition}
    Let $t\mapsto \vec x(c(t)), t\in[a,b]$, be a geodesic in a local surface and $(\vec e_1, \vec e_2)$
    be an adapted moving frame for the surface. Let $\theta(t)$ be such that 
    \begin{align*}
        \vec x' = \cos(\theta(t)) \vec e_1 + \sin(\theta(t)) \vec e_2
    \end{align*}
    along the geodesic. Then 
    \begin{align*}
        \theta'(t) = \omega_2^1 (c'(t)).
    \end{align*}
\end{proposition}

\begin{theorem}
    Let $\Delta$ be a geodesic triangle on a local surface with interior angles $\alpha$, $\beta$,
    $\gamma$. Assume $\Delta$ is contractible to a point. Then 
    \begin{align*}
        \int_\Delta K\theta^1\wedge\theta^2 = \alpha + \beta + \gamma - \pi.
    \end{align*}
\end{theorem}

\begin{corollary}
    Let $\Delta$ be a geodesic triangle with angles $\alpha,\beta,\gamma$ and area $A(\Delta)$.
    We have 
    \begin{enumerate}
        \item on the plane, $\alpha+\beta+\gamma = \pi$;
        \item on the unit sphere, $\alpha + \beta + \gamma = \pi + A(\Delta) > \pi$;
        \item in hyperbolic space, $\alpha + \beta + \gamma = \pi - A(\Delta) < \pi$.
    \end{enumerate}
\end{corollary}

\subsection{Gauss-Bonnet}

\begin{definition}
    Dissect $\Sigma$ into polygons. Let $V$ be the number of vertices , $E$ the number of edges
    and $F$ the number of faces of the dissection. The \emph{Euler characteristic} of the 
    dissected surface is 
    \begin{align*}
        \chi(\Sigma) = V - E + F.
    \end{align*}
\end{definition}

\begin{proposition}
    The Euler characteristic of a surface is independent of the particular dissection chosen.
    Furthermore, it is unchanged by reasonable deformations of the surface.
\end{proposition}

\begin{definition}
    The \emph{torus} is the surface of revolution whose profile is a circle 
    \begin{align*}
        (f(u)-a)^2 + u^2 = b
    \end{align*}
    with $b<a$. The Euler characteristic of the torus is $0$.
\end{definition}

\begin{theorem}[Gauss-Bonnet]
    Let $\Sigma$ be an oriented closed and bounded surface with no boundary. Then 
    \begin{align*}
        \chi(\Sigma) = \frac{1}{2\pi}\int_\Sigma K.
    \end{align*} 
\end{theorem}

\end{document}
