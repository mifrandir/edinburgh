\documentclass{article}
\usepackage{homework-preamble}

\begin{document}
\title{Geometry: Hand-in 3}
\author{Franz Miltz (UNN: S1971811)}
\date{8 November 2021}
\maketitle

\begin{claim*}
   Let $\Sigma\subset\E^3$ be a regular surface, that meets a plane $P\subset\E^3$ in a 
   single point $\vec p$. Then $P$ coincides with the tangent plane of $\Sigma$ at $\vec p$,
   $T_{\vec p}\Sigma$.
\end{claim*}
\begin{proof}
   Let $P$ be given by the equation
   \begin{align*}
      \vec n \cdot (\vec x - \vec p) = 0 \hs\text{for all }\vec x\in P
   \end{align*}
   where we assume without loss of generality that $\vec n$ is a unit vector. Then the 
   distance of any point $\vec x\in\E^3$ the distance to $P$ is
   \begin{align*}
      D(\vec x) = \vec n \cdot (\vec x - \vec p).
   \end{align*}
   Further, since $\Sigma$ is regular, there exists a local regular surface
   $\vec y: D\subset\R^3\to \E^3$ such that $\vec p \in\vec y(D)$ and $\vec y(D)=U\cap\Sigma$
   for some open set $U\subset\R^3$.
   The distance of a point on this surface given by $(u,v)\in D$ then is 
   \begin{align*}
      D(\vec y(u,v)) = \vec n \cdot (\vec y(u,v) - \vec p) =: g(u,v).
   \end{align*}
   Consider the Jacobian matrix
   \begin{align*}
      J_g =
      \begin{bmatrix}
         \frac{\p g}{\p u} & \frac{\p g}{\p v}
      \end{bmatrix}.
   \end{align*}
   Note that the $1\times1$ matrix $\left[\eval{\frac{\p g}{\p v}}{q}\right]$ is invertible 
   at $q\in D$ where $\vec y(q) = \vec p$ if and only if 
   \begin{align}
      \label{non_tangent}
      \eval{\frac{\p g}{\p v}}{q} = \vec n \cdot \vec y_v(q)\not= 0.
   \end{align}
   Assume, for contradiction, that $P$ does not coincide with the tangent space 
   $T_{\vec p}\Sigma$. Then we know that at $q$ either 
   \begin{align*}
      \vec n \cdot \vec y_u(q) \not=0, \hs\text{or}\hs \vec n \cdot \vec y_v(q) \not=0.
   \end{align*}
   This is because $\vec y$ is regular so $\vec y_u$ and $\vec y_v$ must be linearly independent
   and span $T_{\vec p} \Sigma$. We can assume without loss of generality assume that the latter holds.
   We therefore have (\ref{non_tangent}).
   
   By the \emph{Implicit Function Theorem} there must exist a function 
   \begin{align*}
      \Phi : E \subset \R \to \R
   \end{align*}
   defined on an open set $E\subset \R$ such that there exists an open set $\tilde D$
   with $q\in \tilde D\subset D$ such that 
   \begin{align*}
      \tilde D \cap (\R \subset \R^2) = E,
   \end{align*}
   and such that for all $\tilde q \in \tilde D$ we have 
   \begin{align*}
      g(\tilde q) = g(q) \hs\Leftrightarrow\hs \tilde q = (r, \Phi(r)) \hs \text{for some $r\in E$}.
   \end{align*}
   Since $E$ is open and nonempty, there exists more than one such $r$. This implies that there 
   is more than one point on $\Sigma$ where the distance to $P$ is zero which contradicts the premise.
   Therefore $P$ and $T_{\vec p}\Sigma$ must coincide.
\end{proof}


\end{document}