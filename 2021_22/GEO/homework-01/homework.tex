\documentclass{article}
\usepackage{homework-preamble}

\begin{document}
\title{Geometry: Hand-in 1}
\author{Franz Miltz (UNN: S1971811)}
\date{11 October 2021}
\maketitle

\begin{claim*}[Question 1]
   Let $\vec x: I\to \E^3$ be the space curve given by
   \begin{align*}
      \vec x(s) = \begin{pmatrix}
         \frac{\sqrt{3}}{2}\sin s\\
         \cos s\\
         \frac{1}{2}\sin s
      \end{pmatrix}.
   \end{align*}
   Then
   \begin{align*}
      v(s) = 1, \kappa(s) = 1, \tau(s) = 0,\\
   \end{align*}
   and
   \begin{align*}
      \{\vec T(s), \vec N(s), \vec B(s)\} = 
      \left\lbrace  \begin{pmatrix}
         \frac{\sqrt{3}}{2}\cos s\\
         - \sin s\\
         \frac{1}{2}\cos s
      \end{pmatrix}, \begin{pmatrix}
         -\frac{\sqrt{3}}{2}\sin s\\
         - \cos s\\
         -\frac{1}{2}\sin s
      \end{pmatrix}, \begin{pmatrix}
         1/2\\ 0 \\ \sqrt{3}/2
      \end{pmatrix}
      \right\rbrace
   \end{align*}
   for all $s\in I$.
\end{claim*}
\begin{proof}
   We have
   \begin{align*}
      \vec x'(s) = \begin{pmatrix}
         \frac{\sqrt{3}}{2}\cos s\\
         - \sin s\\
         \frac{1}{2}\cos s
      \end{pmatrix},\hs
      \vec x''(s) = \begin{pmatrix}
         -\frac{\sqrt{3}}{2}\sin s\\
         - \cos s\\
         -\frac{1}{2}\sin s
      \end{pmatrix}.
   \end{align*}
   Therefore 
   \begin{align*}
      v(s) = \abs{x'(s)} &= \frac{3}{4}\cos^2 s + \sin^2 s + \frac{1}{4}\cos^2 s = 1,\\
      \kappa(s) = \abs{x''(s)} &= \frac{3}{4}\sin^2 s + \cos^2 s + \frac{1}{4}\cos^2 s = 1.
   \end{align*}
   Since $v(s)=1$ we have $\vec T(s)=\vec x'(x)$. Further, since 
   $\vec T'(s)=\vec x''(s)$,
   \begin{align*}
      \vec N(s) = \frac{\vec T'(s)}{\abs{\vec T'(s)}} = \vec x''(s).
   \end{align*}
   Finally,
   \begin{align*}
      \vec B (s) = \vec T(s) \times \vec N(s) = \begin{pmatrix}
         1/2\\ 0 \\ \sqrt{3}/2
      \end{pmatrix}.
   \end{align*}
   This allows us to compute the torsion $\tau$,
   \begin{align*}
      \tau(s) = -\vec B'(s) \cdot \vec N(s) = \vec 0 \cdot \vec N(s) = 0.
   \end{align*}
   The Frenet-Serret frame is given by the vectors
   \begin{align*}
      \vec T(s) = \begin{pmatrix}
         \frac{\sqrt{3}}{2}\cos s\\
         - \sin s\\
         \frac{1}{2}\cos s
      \end{pmatrix},\: 
      \vec N(s)=\begin{pmatrix}
         -\frac{\sqrt{3}}{2}\sin s\\
         - \cos s\\
         -\frac{1}{2}\sin s
      \end{pmatrix}, \:
      \vec B(s) = \begin{pmatrix}
         1/2\\ 0 \\ \sqrt{3}/2
      \end{pmatrix}.
   \end{align*}
\end{proof}

\begin{claim*}[Question 2.1]
   Let $\vec x : I \to \E^n$ be a unit-speed parameterised curve such that
   \begin{align}
      \label{premise}
      \abs{\vec x(s) - \vec p}^2 = R^2 \hs\text{for all } s\in I
   \end{align}
   where $R\in\R$ and $\vec p\in\E^n$. Then $\kappa(s)\geq 1/R$ for all $s\in I$.
\end{claim*}
\begin{proof}
   We differentiate (\ref{premise}) with respect to $s$ to obtain
   \begin{align*}
      \vec x'(s) \cdot (\vec x(s) - \vec p) = 0. 
   \end{align*}
   Doing this again yields
   \begin{align}
      \label{d2}
      \vec x''(s)\cdot (\vec x(s) - \vec p) + \vec x'(s)\cdot\vec x'(s) = 0.
   \end{align}
   Note that $\vec x(s)$ is unit-speed parameterised and therefore
   \begin{align*}
      \vec x'(s) = \vec T(s), \hs \abs{\vec x'(s)} = 1.
   \end{align*}
   We use this to simplify (\ref{d2}) to 
   \begin{align}
      \label{t}
      \vec T'(s) \cdot (\vec x(s) - \vec p) + 1 = 0. 
   \end{align}
   Therefore, $\vec T'(s)\not= \vec 0$, making $\vec x$ biregular. This allows 
   us to use the structure equations to obtain
   \begin{align*}
      \kappa(s)\vec N(s)\cdot(\vec x(s)-\vec p) + 1 = 0
   \end{align*}
   which, since $\vec N(s)\cdot(\vec x(s)-\vec p)$ is clearly nonzero, we
   can rearrange as follows
   \begin{align*}
      \kappa(s)=-\frac{1}{\vec N(s)\cdot(\vec x(s) - \vec p)}.
   \end{align*}
   However, we know that $\kappa(s)>0$ for all $s$. Thus 
   \begin{align}
      \label{curvature}
      \kappa(s) = \frac{1}{\abs{\vec N(s)\cdot(\vec x(s)-\vec p)}}.
   \end{align}
   By the \emph{Cauchy-Schwarz inequality} we have
   \begin{align*}
      0 < \abs{\vec N(s)\cdot(\vec x(s) - \vec p)} \leq \abs{\vec N(s)}\abs{\vec x(s)-\vec p}.
   \end{align*} 
   Observe that $\abs{\vec N(s)} = 1$, $\abs{\vec x(s) - \vec p} = R$ and thereby
   \begin{align*}
      0 <\abs{\vec N(s) \cdot (\vec x(s) - \vec p)} \leq R
   \end{align*} 
   which shows
   \begin{align*}
      \kappa(s) \geq \frac{1}{R}.
   \end{align*}
\end{proof}

\begin{claim*}[Question 2.2]
   Let $\vec x : I \to \E^n$ be a unit-speed parameterised curve such that
   \begin{align*}
      \abs{\vec x(s) - \vec p}^2 = R^2,\hs\kappa(s)=c \hs\text{for all } s\in I
   \end{align*}
   where $c, R\in\R$ and $\vec p\in\E^n$.
   Then $\vec x(s)$ is part of a circle.
\end{claim*}
\begin{proof}
   In the proof above we have established that 
   \begin{align*}
      \kappa(s)\vec N(s)\cdot (\vec x(s)-\vec p) + 1 = 0.
   \end{align*}
   Since $\kappa(s)=c$, this is equivalent to
   \begin{align}
      \label{eq1}
      \vec N(s)\cdot (\vec x(s) - \vec p) + \frac{1}{c} = 0.
   \end{align}
   We can differentiate to obtain 
   \begin{align*}
      \vec N'(s)\cdot (\vec x(s) - \vec p) + \vec N(s)\cdot \vec x'(s)=0.
   \end{align*}
   Since $\vec T(s) = \vec x'(s)$ and $\vec T(s) \cdot \vec N(s)=0$ by
   \emph{Notes, Proposition 3.14}, this simplifies to
   \begin{align*}
      \vec N'(s)\cdot (\vec x(s)-\vec p) = 0.
   \end{align*}
   We can use the structure equations again to get 
   \begin{align*}
      (-c \vec T'(s) + \tau\vec B'(s))\cdot (\vec x(s)-\vec p) = 0.
   \end{align*}
   By differentiation this shows 
   \begin{align*}
      (-c \vec T'(s) + \tau \vec B'(s)) \cdot (\vec x(s) - \vec p) 
      + (-c \vec T(s) + \tau \vec B(s)) \cdot \vec T = 0,
   \end{align*}
   which we can simplify by using $\vec T(s)\cdot \vec T(s)=1$ and $\vec B(s)\cdot \vec T(s)=0$:
   \begin{align*}
      (-c \vec T'(s) + \tau \vec B'(s)) \cdot (\vec x(s) - \vec p) = c.
   \end{align*}
   However, due to (\ref{t}), we find
   \begin{align*}
      \tau \vec B'(s)\cdot (\vec x(s) - \vec p) = 0.
   \end{align*}
   By applying the structure equations one last time, we find
   \begin{align*}
      -\tau^2 \vec N(s)\cdot (\vec x(s)-\vec p) = 0.
   \end{align*}
   Due to (\ref{eq1}), $\vec N(s)\cdot (\vec x(s)-\vec p)\not=0$, which shows $\tau=0$.
   By \emph{Notes, Corollary 3.26}, this makes $\vec x(s)$ part of a circle.
\end{proof}

\end{document}