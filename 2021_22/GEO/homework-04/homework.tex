\documentclass{article}
\usepackage{homework-preamble}

\begin{document}
\title{Geometry: Hand-in 4}
\author{Franz Miltz (UNN: S1971811)}
\date{22 November 2021}
\maketitle
\noindent Let $D=\{u,v,z\in\R : u,v\not=0\}$ and let $\vec x:D\to\E^3$ be such that 
\begin{align*}
   \vec x(u,v,z) = \begin{pmatrix}
      uv \\ \frac{1}{2}(v^2 - u^2) \\ z
   \end{pmatrix}
\end{align*}

\section*{Question 1}

\begin{claim*}
   \begin{align}
      \label{dx}
      d\vec x = \sqrt{u^2 + v^2}du\vec e_1
      + \sqrt{u^2 + v^2}dv \vec e_2
      + dz\vec e_3
   \end{align}
   where
   \begin{align*}
      \vec e_1 = \frac{1}{\sqrt{u^2 + v^2}}\begin{pmatrix}
         v \\ -u \\ 0
      \end{pmatrix} , \hs
      \vec e_2 = \frac{1}{\sqrt{u^2 + v^2}}\begin{pmatrix}
         u \\ v \\ 0
      \end{pmatrix}, \hs 
      \vec e_3 = \begin{pmatrix}
         0 \\ 0 \\ 1
      \end{pmatrix}
   \end{align*}
   are a moving frame on $D$.
\end{claim*}
\begin{proof}
   Firstly, we calculate 
   \begin{align*}
      d\vec x = \begin{pmatrix}
         v du + u dv \\
         -u du + v dv \\
         dz
      \end{pmatrix}.
   \end{align*} 
   It's straightforward to verify algebraically that (\ref{dx}) holds. Secondly,
   we note that $u,v\not=0$ and therefore that the $\vec e_i$ are well-defined on $D$
   and smooth. Lastly, we observe that 
   \begin{align}
      \vec e_i \cdot \vec e_j = \delta_i^j,
   \end{align}
   making $\{\vec e_1, \vec e_2, \vec e_3\}$ a moving frame.
\end{proof}

\section*{Question 2}

\end{document}