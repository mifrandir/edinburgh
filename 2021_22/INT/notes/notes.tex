\documentclass{article}
\usepackage{notes-preamble}
\usepackage{enumitem}
\begin{document}
\mkthmstwounified
\title{Honours Algebra (SEM6)}
\author{Franz Miltz}
\maketitle
\tableofcontents
\pagebreak

\section{Pythagorean triples}

\begin{definition}
    Three natural numbers $(a,b,c)$ are called a \emph{pythagorean triple} if 
    \begin{align*}
        a^2 + b^2 = c^2.
    \end{align*}
\end{definition}

\begin{lemma}
    If $(a,b,c)$ is a pythagorean triple and $d$ is any positive integer then 
    so is $(da,db,dc)$.
\end{lemma}

\begin{definition}
    A pythagorean triple $(a,b,c)$ is called \emph{primitive} if they have no common factor.
\end{definition}

\begin{theorem}
    Every primitive pythagorean triple $(a,b,c)$ with $a$ odd satsifies 
    \begin{align*}
        a=s,\hs b=\frac{s^2-t^2}{2},\hs c=\frac{s^2+t^2}{2},
    \end{align*}
    where $s>t\geq 1$ are chosen to be odd integers with no common factors.
\end{theorem}

\end{document}
