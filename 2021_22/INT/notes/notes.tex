\documentclass{article}
\usepackage{notes-preamble}
\usepackage{enumitem}
\begin{document}
\mkthmstwounified
\title{Introduction to Number Theory (SEM6)}
\author{Franz Miltz}
\maketitle
\tableofcontents
\pagebreak

\section{Rings}

\begin{definition}[Ring]
	A \emph{ring} $(R,+,\cdot)$ is a set $R$ equipped with two binary operations
	$+,\cdot:R\times R\to R$ such that $(R,+)$ is an abelian group, $(R,\cdot)$
	is a monoid and the following equalities hold for all $a,b,c\in R$:
	\begin{align*}
		a       \cdot (b + c)   = a \cdot b + a \cdot c, \\
		(a+b)   \cdot c         = a \cdot c + b \cdot c.
	\end{align*}
\end{definition}

\begin{definition}[Commutative ring]
	A ring $(R,+,\cdot)$ where for all $a,b,c\in R$, $a\cdot(b\cdot c) = (a\cdot b)\cdot c$
	is called a \emph{commutative ring}.
\end{definition}

\begin{definition}[Unital ring]
	A ring $(R,+,\cdot)$ where there exists an element $1\in R$ such that
	for all $a\in R$, $1\cdot a = a \cdot 1 = a$ is called a \emph{unital ring}.
\end{definition}

\begin{definition}[Commutative ring with identity]
	A ring which is commutative and unital is called a \emph{commutative ring
		with identity}.
\end{definition}

\begin{definition}
	Let $R$ be a ring, and let $x$ be any symbol not found in $R$. A polynomial in $x$ over $R$
	is an expression of the form
	\begin{align*}
		\alpha(x)=\sum_{i=0}^\infty a_ix^i
	\end{align*}
	where $a_i\in R$ and $a_i=0$ almost everywhere.
\end{definition}

\begin{definition}
	A \emph{field} is a non-zero commutative ring $(F, +, \cdot)$ such that for every non-zero
	$x\in F$ there exists $\inv x\in F$ such that $x\inv x = 1$.
\end{definition}

\begin{definition}
	We define the following:
	\begin{enumerate}
		\item $\N$ the set of \emph{natural numbers},
		\item $\Z$ the ring of \emph{integers},
		\item $\R$ the field of \emph{real numbers}, and
		\item $\C$ the field of \emph{complex numbers}.
	\end{enumerate}
\end{definition}

\section{Integers}

\subsection{Pythagorean triples}

\begin{definition}
	Three natural numbers $(a,b,c)$ are called a \emph{Pythagorean triple} if
	\begin{align*}
		a^2 + b^2 = c^2.
	\end{align*}
\end{definition}

\begin{lemma}[Lecture 1]
	If $(a,b,c)$ is a Pythagorean triple and $d$ is any positive integer then
	so is $(da,db,dc)$.
\end{lemma}

\begin{definition}
	A Pythagorean triple $(a,b,c)$ is called \emph{primitive} if $a,b,c$ have no common factor.
\end{definition}

\begin{theorem}[Lecture 1]
	Every primitive Pythagorean triple $(a,b,c)$ with $a$ odd satsifies
	\begin{align*}
		a=s,\hs b=\frac{s^2-t^2}{2},\hs c=\frac{s^2+t^2}{2},
	\end{align*}
	where $s>t\geq 1$ are chosen to be odd integers with no common factors.
\end{theorem}

\subsection{Division and remainders}

\begin{theorem}[Division algorithm; Lecture 2]
	Let $a,b\in\Z$ with $b>0$. Then there exists a unique pair $q,r\in\Z$ such that
	\begin{align*}
		a = qb + r
	\end{align*}
	and $0\leq r<b$. $q$ is called the quotient and $r$ is called the remainder.
\end{theorem}

\begin{definition}
	Let $a,b\in\Z$. We say that \emph{$a$ divides $b$}, written $a|b$, if there exists $c\in\Z$
	such that $b=ac$.
\end{definition}

\begin{definition}[gcd, lcm]
	Let $a,b\in\Z$, not both zero. The \emph{greatest common divisor} of $a$ and $b$ is the largest integer
	that divides both of them. Note $\gcd(a,b)$ is well-defined and $1\leq\gcd(a,b)\leq\min(a,b)$.

	Similarly, the \emph{least common multiple} of $a$ and $b$ is the smallest
	integer that is divided by both of them. Note $\lcm(a,b)$ is well-defined
	and $\max(a,b)\leq\lcm(a,b)\leq ab$.
\end{definition}

\begin{proposition}[Boocher 3.3]
	Let $\gcd(a,b)=d$. Then $\gcd(a/d,b/d)=1$.
\end{proposition}

\begin{proposition}[Boocher 3.4]
	For all integers $a,b,c$, $\gcd(a+cb,b)=\gcd(a,b)$.
\end{proposition}

\begin{theorem}[Bezout's Lemma]
	Let $a,b\in\Z$, not both zero. Then there exist $s,t\in\Z$ such that
	$as + bt = \gcd(a,b)$. Moreover, any common divisor of $a$ and $b$ divides $\gcd(a,b)$.
\end{theorem}

\begin{corollary}[Boocher 3.7]
	The set of linear combinations of two integers $a,b$ is equal to the set
	of multiples of $\gcd(a,b)$:
	\begin{align*}
		\{ax+by : x,y\in\Z \} = \{k\gcd(a,b):k\in\Z\}=\gcd(a,b)\Z.
	\end{align*}
\end{corollary}

\begin{theorem}[Extended Euclidean algorithm; Lecture 2]
	Let $a,b\in Z$ such that $a\geq b>0$.
	The Euclidean algorithm described below always gives the greatest common divisor of
	$a$ and $b$.
	\begin{enumerate}
		\item There exist unique integers $q_1,r_1$ such that $a=q_1b+r_1$, $0\leq r_1<b$.
		\item Assuming that $r_1>0$, then we can apply the division algorithm to the pair
		      $b,r_1$ and get $b=q_2r_1 + r_2$, $0\leq r_2<r_1$. If $r_2>0$ we may divide
		      $r_1$ by $r_2$, and so on.
		\item Continuing in this way, we eventually get \begin{align*}
			      r_k=q_{k+2}r_{k+1}+r_{k+2}
		      \end{align*}
		      with $r_{k+2}=0$ for some $k$.
		\item We have $r_{k+1}=\gcd(a,b)$.
	\end{enumerate}
\end{theorem}

\subsection{Primes}

\begin{theorem}[Fundamental Theorem of Arithmetic]
	Let $n>1$. Then there exist primes $p_1,...,p_k\in\Z$ such that
	\begin{align*}
		n=p_1\cdots p_k.
	\end{align*}
	Moreover, if $n=p_1\cdots p_k=q_1\cdots q_m$ where all $p_i,q_j\in\Z$ are prime
	then $k=m$ and there exists $\pi\in S_k$ such that, for all $i$, $p_i=q_{\pi(i)}$.
\end{theorem}

\begin{proposition}[Boocher 2.3]
	Suppose $p,n,m\in\N$ with $p$ prime and $p|nm$. Then either $p|m$ or $p|n$ or both.
\end{proposition}

\begin{proposition}[Boocher 3.2]
	Given $m,n\in\N$ let $p_1,...,p_k$ be the primes dividing $mn$, and write
	\begin{align*}
		m = p_1^{e_1}\cdots p_k^{e_k}, \hs n = p_1^{f_1}\cdots p_k^{f_k}
	\end{align*}
	where $e_i,f_i\geq 0$ for all $0<i\leq k$. Then
	\begin{enumerate}
		\item $\gcd(m,n)=p_1^{\min(e_i,f_i)}\cdots p_k^{\min(e_k,f_k)}$;
		\item $\lcm(m,n)=p_1^{\max(e_i,f_i)}\cdots p_k^{\max(e_k,f_k)}$;
		\item $\gcd(m,n)\cdot\lcm(m,n)=mn$;
		\item \begin{align*}
			      \gcd\left(\frac{n}{\gcd(n,m)},\frac{\lcm(n,m)}{n}\right)=1.
		      \end{align*}
	\end{enumerate}
\end{proposition}

\begin{theorem}[Boocher 2.5]
	If $n$ is composite then it must be divisible by some prime $p\leq\sqrt{n}$.
\end{theorem}

\begin{theorem}[Euclid; Lecture 3]
	There are infinitely many prime numbers.
\end{theorem}

\begin{theorem}[Lecture 3]
	There are infinitely many primes of the form $4k-1$.
\end{theorem}

\begin{theorem}[Lecture 3]
	There are infinitely many primes of the form $4k+1$.
\end{theorem}

\begin{theorem}[Lecture 3]
	There are infinitely many primes of the form $3k-1$.
\end{theorem}

\begin{theorem}[Dirichlet; Lecture 3]
	If $a$ and $b$ are positive integers not divisible by the same prime then there are
	infinitely many primes of the form $ak+b$.
\end{theorem}

\begin{theorem}[Boocher 2.8]
	The sum of the reciprocals of the primes diverges
	\begin{align*}
		\sum_{i=1}^\infty \frac{1}{p_i} \to \infty
	\end{align*}
	where $p_1,p_2,...,p_n,...$ are all prime numbers.
\end{theorem}

\begin{theorem}[Boocher 2.9]
	When $n$ is large, the number of primes less than $x$ is approximately equal to
	$x/\ln x$. I.e.
	\begin{align*}
		\lim_{x\to\infty} \frac{\pi(x)}{x/\ln(x)}=1.
	\end{align*}
\end{theorem}

\begin{conjecture}[Twin prime; Boocher 2.10]
	There are infinitely many prime numbers $p$ such that
	$p+2$ is also prime.
\end{conjecture}

\begin{conjecture}[Goldbach; Boocher 2.11]
	Every even integer larger than $2$ can be written as the
	sum of two primes
\end{conjecture}

\subsection{Congruences}

\begin{definition}[Congruence]
	The congruence $a\equiv b\mod n$ means that the difference $(a-b)$ is
	divisible by $n$.
\end{definition}

\begin{definition}
	A \emph{complete system of residues modulo $n$} is a set $S\subset\Z$ of integers
	such that every integer is congruent modulo $n$ to exactly one integer
	in the set. I.e., for all $a\in Z$, there exists $b\in S$ such that $a\equiv b\mod n$.
	A \emph{least positive residue} for an integer $a$ is the
	positive integer $b$ such that $a\equiv b \mod n$.
\end{definition}

\begin{theorem}[Lecture 4]
	If $a,b,c,d\in\Z$ and $n\in\N$ with $a\equiv b\mod n$ and $c\equiv d\mod n$
	then
	\begin{align*}
		a+c & \equiv b+d \mod n, \\
		a-c & \equiv b-d \mod n, \\
		ac  & \equiv bd  \mod n.
	\end{align*}
\end{theorem}

\begin{definition}[Ring of integers modulo $n$]
	Let $n\in\N$ and $a,b,c\in\Z$. Let $\Z_n$ denote the set of equivalence
	classes due to congruence and denote the equivalence class of $a\in\Z$ by
	$\left[a\right]_n\in\Z_n=\Z/n\Z$. We define the operations $+,.:\Z_n\times\Z_n\to\Z_n$
	by
	\begin{align*}
		\left[a\right]_n +     \left[b\right]_n & = \left[a+b\right]_n, \\
		\left[a\right]_n \cdot \left[b\right]   & = \left[ab\right]_n.
	\end{align*}
	Then \emph{the ring of integers modulo $n$} is $\Z_n=(\Z_n,+,\cdot)$.
\end{definition}

\begin{lemma}[Lecture 4]
	Let $p$ be prime. Then the ring $\F_p:=\Z_p$ is a field.
\end{lemma}

\begin{theorem}[Wilson; Lecture 4]
	If $p$ is prime then $(p-1)!\equiv -1 \mod p$.
\end{theorem}

\begin{lemma}[Lecture 5]
	Let $m,n\in\Z$ be such that $\gcd(m,n)=1$. Then there exists $q\in\Z$ such that
	\begin{align*}
		mq\equiv 1\mod n.
	\end{align*}
\end{lemma}

\subsection{Chinese remainder theorem}

\begin{theorem}[Chinese remainder theorem; Lecture 5]
	Let $m_1,...,m_k\in\N$ be pairwise coprime and let $a_1,...,a_k\in\Z$. Then there exists a
	unique $0\leq x<m_1\cdots m_k$ such that
	\begin{align*}
		x\equiv a_1 & \mod m_1 \\
		            & \vdots   \\
		x\equiv a_k & \mod m_k
	\end{align*}
	\begin{proof}
		\begin{enumerate}
			\item Let $N=m_1\cdots m_k$ and $N_i=N/m_i$, for every $i$.
			\item Fix $i$. There exists a solution to the congruence \begin{align*}
				      N_ix\equiv 1 \mod m_i
			      \end{align*}
			      because to find a solution to this congruence is equivalent to find a solution to
			      the equation
			      \begin{align*}
				      N_ix +m_iy=1
			      \end{align*}
			      and $\gcd(N_i,m_i)=1$ so a solution exist by \emph{Bezout's Lemma}.
			\item Let $x_i\in\Z$ be such that \begin{align*}
				      N_ix_i\equiv 1 \mod m_i.
			      \end{align*}
			\item Denote $x=\sum_{i=1}^k a_iN_ix_i$. Then $x\equiv a_i \mod m_i$ for all $i$.
			\item Observe that $x'=x+jN$ also satisfies $x'\equiv a_i \mod m_i$ for all $i$.
			\item Notice on the other hand that if $x,x'$ are solutions to the system above, then
			      $x-x'\equiv 0 \mod m_i$ for all $i$.
		\end{enumerate}
	\end{proof}
\end{theorem}

\begin{theorem}[Lecture 5]
	Let $m,n\in\Z$ be coprime. Let $a\in\Z$ then the congruence
	\begin{align*}
		x \equiv a \mod mn
	\end{align*}
	has the same solutions in integers as the system of congruences
	\begin{align*}
		x \equiv a \mod m \\
		x \equiv a \mod n
	\end{align*}
\end{theorem}

\subsection{Eulers theorem}

\begin{definition}
	The \emph{Euler function $\phi(n)$} is defined as the number of natural numbers not exceeding
	$n$ which are coprime with $n$ and $\phi(1)=1$.
\end{definition}

\begin{theorem}[Euler]
	Let $n>1$ be a natural number and let $a\in\Z$ such that $n$ and $a$ are coprime. Then
	\begin{align*}
		a^{\phi(n)}-1 \equiv 0 \mod n.
	\end{align*}
\end{theorem}

\begin{theorem}[Lecture 6, Theorem 1]
	If $m,n$ are coprime then $\phi(m\cdot n)=\phi(m)\cdot\phi(n)$.
\end{theorem}

\begin{theorem}[Lecture 6, Theorem 2]
	Let $m>1$. Suppose that $p_1,...,p_k\in\Z$ are all distinct primes that divide $m$. Then
	\begin{align*}
		\phi(m) = m\prod_{i=1}^k \left(1-\frac{1}{p_i}\right).
	\end{align*}
\end{theorem}

\begin{theorem}[Lecture 7, Theorem 1]
	Let $n\in\N$. Then
	\begin{align*}
		\sum_{d|n}\phi(d) = n.
	\end{align*}
\end{theorem}

\subsection{Primitive roots}

\begin{definition}
	Let $p$ be prime. We say that $i\in\Z$ is a \emph{primitive root modulo $p$}
	iff all elements $i^0,...,i^{p-1}$ are pairwise distinct modulo $p$. I.e.
	$i^m\not\equiv 1\mod p$ for $0<m<p-1$.
\end{definition}



\begin{theorem}[Lecture 7, Theorem 2]
	A non-zero polynomial $f\in\F_p[x]$ of degree $n$ has at most $n$ roots $x$
	in $\F_p$.
\end{theorem}

\begin{theorem}[Lecture 7, Theorem 3]
	Let $p>0$ be a prime number then there exists $k\in\N$ which is a
	primitive root modulo $p$.
\end{theorem}

\begin{definition}
	Let $G$ be a group. An element $g\in G$ generates $G$ if
	\begin{align*}
		G = \{g^k : k \in \N\}.
	\end{align*}
	If such a $g$ exists, we say $G$ is cyclic and write $G=\lra{g}$.
\end{definition}

\begin{definition}
	Let $G$ be a group and $g\in G$. Then the order of $g$ is the smallest positive
	integer $n$ such that $g^n = 1$.
\end{definition}

\begin{corollary}[Lecture 8, Corollary 1]
	Let $G$ be a finite cyclig group. Then $g\in G$ is a generator iff the order of
	$g$ equals $\abs{G}$, the cardinality of $G$.
\end{corollary}

\begin{lemma}
	Let $G$ be a group and $g\in G$. If for integers $m,n$ we have $g^m=g^n=1$
	then $g^{\gcd(m,n)}=1$.
\end{lemma}

\begin{theorem}[Lagrange; Lecture 8, Theorem 1]
	Let $G$ be a finite cyclic group of cardinality $N$. If $g\in G$ then
	the order of $g$ divides $N$.
\end{theorem}

\begin{theorem}[Lecture 8, Theorem 2]
	Let $G=\lra{g}$ and let $\alpha\in\Z$. Then $G=\lra{g^\alpha}$ iff
	$\gcd(\alpha,\abs{G})=1$.
\end{theorem}

\begin{corollary}[Lecture 8, Corollary 2]
	Let $G$ a finite cyclic group. Then $G$ has $\phi(\abs G)$ generators.
\end{corollary}

\begin{definition}
	The group of units in $\Z_p$ is denoted as $\Z_p^*$.
\end{definition}

\begin{theorem}[Lecture 8, Theorem 3]
	Let $p$ prime and $i\in\Z$. Then $i$ is a primitive root modulo $p$ iff
	$\Z_p^*=\lra{\overline i}$ where $\overline i$ denotes the coset of $i$ in
	$\Z_p$.
\end{theorem}

\subsection{The Legendre symbol}

\begin{definition}
	Let $p>0$ prime and let $a,b\in\Z$ coprime to $p$.
	Then the \emph{Legendre symbol} of $a$ modulo $p$, denoted $(a/p)$, is given by
	\begin{align*}
		\left(\frac{a}{p}\right)=\begin{cases}
			1  & \text{if }a\equiv r^2\mod p\text{ for some $r\in\Z$},    \\
			-1 & \text{if }a\not\equiv r^2\mod p\text{ for all $r\in\Z$}.
		\end{cases}
	\end{align*}
\end{definition}

\begin{theorem}[Lecture 11]
	Let $p>0$ prime and $a\in\Z$ coprime to $p$. Then the following hold
	\begin{enumerate}
		\item If $p>2$ then \begin{align*}
			      \left(\frac{a}{p}\right)\equiv a^{\frac{p-1}{2}} \mod p.
		      \end{align*}
		\item Multiplication rule: \begin{align*}
			      \left(\frac{a}{p}\right)\left(\frac{b}{p}\right)=\left(\frac{ab}{p}\right).
		      \end{align*}
		\item If $a\equiv b$ then \begin{align*}
			      \left(\frac{a}{p}\right)\equiv\left(\frac{b}{p}\right).
		      \end{align*}
		\item If $p,q>2$, $p\not=q$, and either $p\equiv 1\mod 4$ or $q\equiv 1\mod 4$ then \begin{align*}
			      \left(\frac{p}{q}\right)=\left(\frac{q}{p}\right).
		      \end{align*}
		\item If $p,q>2$, $p\not=q$, and $p,q\equiv 3\mod 4$ then \begin{align*}
			      \left(\frac{p}{q}\right)=-\left(\frac{q}{p}\right).
		      \end{align*}
	\end{enumerate}
\end{theorem}

\begin{lemma}[Gauss]
	Let $p>2$ prime and $a\in\Z$ coprime to $p$. Let $p'=(p-1)/2$. Consider the set
	\begin{align*}
		S=\left\lbrace k\cdot a : 1\leq k\leq p'\right\rbrace
	\end{align*}
	and let $n$ be the number of elements in $S$ that, when reduced modulo $p$ to lie between
	$-p'$ and $p$, are less than zero. Then
	\begin{align*}
		\left(\frac{a}{p}\right)=(-1)^n.
	\end{align*}
\end{lemma}

\section{Gaussian integers}

\begin{definition}
	Let $\alpha(x)$ and $\gamma(x)$ be polynomials in $x$ over $R$ such that
	\begin{align*}
		\alpha(x)=\sum_{i=0}^\infty a_ix^i,\hs
		\gamma(x)=\sum_{i=0}^\infty c_ix^i.
	\end{align*}
	Then \begin{enumerate}
		\item $\alpha(x)=\gamma(x)$ iff $a_i=c_i$ for all $i\geq 0$,
		\item $\alpha(x)+\gamma(x)=\sum_{i=0}^\infty (a_i+c_i)x^i$,
		\item $\alpha(x)\cdot\gamma(x)=\sum_{i=0}^\infty \sum_{j=0}^i a_jc_{i-j}x^i$.
	\end{enumerate}
\end{definition}

\begin{definition}
	The \emph{Gaussian integers} are the subring $\Z[i]\subseteq\C$ given by
	\begin{align*}
		\Z[i]=\{a+bi:a,b\in\Z\}.
	\end{align*}
\end{definition}

\subsection{Units, primes and irreducibles}

\begin{definition}
	A Gaussian integer $z=\Z[i]$ is a unit if there exists $z'\in\Z[i]$ such that
	$zz'=1$.
\end{definition}

\begin{lemma}[Lecture 13]
	Let $z=a+bi$ be a unit in $\Z[i]$. Then $\abs z=1$ and $z\in\{\pm 1, \pm i\}$.
\end{lemma}

\begin{definition}
	Let $z,z'$ be Gaussian integers with $z\not=0$. We say $z$ divides $z'$ if $z'=zw$ for
	some $w\in\Z[i]$.
\end{definition}

\begin{definition}[Irreducibles]
	Let $z=a+bi\in\Z[i]$ such that $z$ is neither zero nor a unit. We say that $z$ is \emph{irreducible}
	if whenever $z=uv$ then either $u$ or $v$ is a unit.
\end{definition}

\begin{definition}[Gaussian primes]
	Let $z=a+bi\in\Z[i]$ such that $z$ is neither zero nor a unit. We say that $z$ is \emph{prime} if
	whenever $z|uv$ for $u,v\in\Z[i]$ it follows that either $z|u$ or $z|v$.
\end{definition}

\begin{theorem}[Division algorithm for Gaussian integers]
	Let $z,w\in\Z[i]$ with $w\not=0$. There exist $q,r\in\Z[i]$ with $z=qw+r$ and $\abs r<\abs w$.
\end{theorem}

\begin{theorem}[Bezout Lemma]
	Let $a,b\in\Z[i]$ with $a\not=0$ or $b\not=0$ and let $d=\gcd(a,b)$. Then there exist $s,t\in\Z$
	such that $as+bt=d$. Moreover, any common divisor of $a$ and $b$ divides $d$.
\end{theorem}

\begin{theorem}[Lectures 15-16]
	Let $z\in\Z[i]$. Then $z$ is prime iff $z$ is irreducible.
\end{theorem}

\begin{theorem}[Fundamental Theorem of Arithmetic for Gaussian integers]
	Let $a$ be a non-zero Gaussian integer. Then either $a$ is a unit or $a$ may be expressed
	as a product of finitely many primes.

	Moreover, if $a=p_1\cdots p_n=q_1\cdots q_m$ where all $p_i,q_j\in\Z[i]$ are prime then
	$n=m$ and there exists $\pi\in S_n$ such that, for all $i$, $p_i$ and $q_{\pi(i)}$ are associates.
\end{theorem}

\subsection{Sums of two squares}

\begin{theorem}[Lectures 15-16]
	Let $p=4k+1$ be a prime in $\Z$ for some $k\in\N$. Then $p=a^2+b^2$ for some
	$a,b\in\Z$.
\end{theorem}

\begin{theorem}[Lectures 15-16]
	Let $n\in\N$ and suppose there exist primes $p_1,...,p_s\in\N$ of the form
	$p_i=4k_i+1$ and primes $q_1,...,q_s\in\N$ of the form $q_j=4k'_j+3$ such that
	\begin{align*}
		n=2^c p_1^{a_1}\cdots p_s^{a_s} q_1^{d_1}\cdots q_t^{d_t}
	\end{align*}
	for some $a_1,...,a_s,d_1,...,d_t\in\N$. Then there exit $a,b\in\Z$ with
	$n=a^2+b^2$ iff, for each $i$, $d_i$ is even.
\end{theorem}

\begin{lemma}[Lectures 15-16]
	Let $p\in\Z$ be prime. Then $p=a^2+b^2$ for some $a,b\in\Z$ iff $p$ is a reducible
	as a Gaussian integer.
\end{lemma}

\section{Coding theory}

\begin{definition}[Lecture 9]
	A \emph{(block) linear code} $C$ of length $n$ over the finite field $F$ is a subspace of the
	$F$-vector space $F^n$.
\end{definition}

\begin{definition}[Cyclic code; Lecture 9]
	A linear code $C$ is called cyclic if $c_1\cdots c_n\in C$ implies $c_nc_1\cdots c_{n-1}\in C$.
\end{definition}

\begin{definition}[Hamming metric; Lecture 10]
	Let $C$ be an $F$-code. The \emph{Hamming metric $d$} on $C$ is the metric
	$d:C\times C\to\R$ given by
	\begin{align*}
		d(x,y) = \abs{\{i\in{1,...,n}:x_i\not=y_i\}}
	\end{align*}
	for all $x=x_1\cdots x_n$ and $y=y_1\cdots y_n$.
	For $x\in C$ we define the \emph{weight} of $x$, $w(x)$ by setting
	\begin{align*}
		w(x)=d(x,0)=\abs\{i\in {1,...,n} : x_i\not=0\}.
	\end{align*}
\end{definition}

\begin{lemma}[Lecture 10]
	Let $C$ be a linear code. Then the minimal distance is equal to the minimal possible weight
	of a non-zero codeword in this code.
\end{lemma}

\begin{definition}[Lecture 10]
	The triple $(n,k,d)$ comprises the three parameters of a linear code $C$: $n$ is the length of the
	codewords, $k$ is the dimension of the code $\dim_F C=k$ an $d$ is the minimal distance of the code.
\end{definition}

\begin{definition}[Lecture 10]
	Let $F$ be a field. For any word $x\in F^n$ and any $r\geq 0$, the neighbourhood of $x$ with radius
	$r$ is the closed ball
	\begin{align*}
		N_r(x)=\{y\in F^n : d(x,y)\leq r\}.
	\end{align*}
\end{definition}

\begin{lemma}[Lecture 10]
	If $C$ is a code with the minimal distance $d>2r$, then for any codewords $c,c'\in C$ the neighbourhoods
	$N_r(c)$ and $N_r(c')$ are disjoint.
\end{lemma}

\begin{definition}[Lecture 10]
	Let $F$ be a field and $C\subseteq F^n$ be a code. A decision rule for $C$ is a function
	$\sigma:F^n\to C$ which assigns to each $z\in F^n$ a codeword $c\in C$.
\end{definition}

\begin{definition}[Minimum distance rule; Lecture 10]
	The \emph{minimum distance rule} is a decision rule $\sigma:F^n\to C$ such that
	whenever $\sigma(x)=c$, for all $c'\in C$, $d(x,c) \leq d(x,c')$.
\end{definition}

\end{document}
