\documentclass{article}
\usepackage{notes-preamble}
\usepackage{enumitem}
\begin{document}
\mkthmstwounified
\title{Honours Algebra (SEM6)}
\author{Franz Miltz}
\maketitle
\tableofcontents
\pagebreak

\section{Pythagorean triples}

\begin{definition}
    Three natural numbers $(a,b,c)$ are called a \emph{pythagorean triple} if 
    \begin{align*}
        a^2 + b^2 = c^2.
    \end{align*}
\end{definition}

\begin{lemma}
    If $(a,b,c)$ is a pythagorean triple and $d$ is any positive integer then 
    so is $(da,db,dc)$.
\end{lemma}

\begin{definition}
    A pythagorean triple $(a,b,c)$ is called \emph{primitive} if they have no common factor.
\end{definition}

\begin{theorem}
    Every primitive pythagorean triple $(a,b,c)$ with $a$ odd satsifies 
    \begin{align*}
        a=s,\hs b=\frac{s^2-t^2}{2},\hs c=\frac{s^2+t^2}{2},
    \end{align*}
    where $s>t\geq 1$ are chosen to be odd integers with no common factors.
\end{theorem}

\section{Integers}

\begin{theorem}[Well-ordering principle]
    Every non-empty set of positive integers contains a least element.
\end{theorem}

\begin{theorem}[Division algorithm]
    Let $a,b\in\Z$ with $b>0$. Then there exists a unique pair $q,r\in\Z$ such that 
    \begin{align*}
        a = qb + r
    \end{align*} 
    and $0\leq r<b$. $q$ is called the quotient and $r$ is called the remainder.
\end{theorem}

\begin{definition}
    Let $a,b\in\Z$. We say that $a$ divides $b$, written $a|b$, if there exists $c\in\Z$
    such that $b=ac$.
\end{definition}

\begin{definition}
    Let $a,b\in\Z$, not both zero. The greatest common divisor of $a$ and $b$ is the largest integer 
    that divides both of them. Note $\gcd(a,b)$ is well-defined and $1\leq\gcd(a,b)\leq\min(a,b)$.
\end{definition}

\begin{lemma}[Bezout]
    Let $a,b\in\Z$, not both zero. Then there exist $s,t\in\Z$ such that 
    $as + bt = \gcd(a,b)$. Moreover, any common divisor of $a$ and $b$ divides $\gcd(a,b)$.  
\end{lemma}

\begin{theorem}
    Let $a,b\in Z$ such that $a\geq b>0$.
    The Euclidean algorithm described below always gives the greatest common divisor of 
    $a$ and $b$.
    \begin{enumerate}
        \item There exist unique integers $q_1,r_1$ such that $a=q_1b+r_1$, $0\leq r_1<b$.
        \item Assuming that $r_1>0$, then we can apply the division algorithm to the pair
            $b,r_1$ and get $b=q_2r_1 + r_2$, $0\leq r_2<r_1$. If $r_2>0$ we may divide 
            $r_1$ by $r_2$, and so on.
        \item Continuing in this way, we eventually get \begin{align*}
            r_k=q_{k+2}r_{k+1}+r_{k+2}
        \end{align*}
        with $r_{k+2}=0$ for some $k$.
        \item We have $r_{k+1}=\gcd(a,b)$.
    \end{enumerate}
\end{theorem}

\end{document}
