\documentclass{article}
\usepackage{homework-preamble}
\usepackage{enumitem}
\DeclareMathOperator{\diam}{diam}
\begin{document}
\title{Introduction to Number Theory: Homework 2}
\author{Franz Miltz (UUN: S1971811)}
\date{11 March 2022}
\maketitle

\begin{claim*}[1]
   Let $p$ be prime such that $p=4k+3$ for some $k\in\N$. Let $0<j<p$ be an integer. Then 
   there exists an integer $r$ such that either $p|j-r^2$ or $p|j+r^2$.
   \begin{proof}
      By contradiction. Assume, for all $r\in\Z$, $p$ divides neither $j-r^2$ nor $j+r^2$. 
      Equivalently, 
      \begin{align}
         \label{assumption1}
         \left(\frac{j}{p}\right)=\left(\frac{-j}{p}\right)=-1,
      \end{align}
      i.e. neither $j$ nor $-j$ are quadratic residues modulo $p$. Note that, since $p=4k+3$
      for some $k$, 
      \begin{align}
         \label{minus_one}
         \left(\frac{-1}{p}\right)=(-1)^{2k+1}=-1
      \end{align}
      by \emph{Theorem 7} in the notes for lecture 11. However, using the multiplication rule,
      (\ref{assumption1}) leads us to
      \begin{align*}
         \left(\frac{j}{p}\right) = \left(\frac{-j}{p}\right) 
                                  = \left(\frac{-1}{p}\right)\left(\frac{j}{p}\right)
                                  = -\left(\frac{j}{p}\right),
      \end{align*}
      a contradiction. The claim follows.
   \end{proof}
\end{claim*}

\begin{claim*}[2]
   Let $p$ be a prime such that $p=4k+3$ for some $k\in\N$. Then there exist 
   integers $0<x,y<p$ such that $x^2+y^2+1\equiv 0\mod p$.
   \begin{proof}
      Let $0<y<p$ and $j=1+y^2$. By the previous claim, there exists an $r$ such that 
      $p|j-r^2$ or $p|j+r^2$. Observe, in either case, $r^2\not\equiv 0\mod p$. We distinguish the following cases:
      \begin{enumerate}[label=C\arabic*.]
         \item If for some $y$ there exists an $r$ such that $p|j+r^2$, we immediately find the 
            solution $1+x^2+y^2\equiv 0\mod p$ where $0<x<p$ such that $x\equiv r\mod p$.
         \item If for all $y$ there exists an $r$ such that $p|j-r^2$, we have $1+y^2\equiv r^2\mod p$.
            Note that this implies there exists an $s$ such that $1+r^2\equiv s^2\mod p$. I.e.
            \begin{align}
               \label{induction}
               \left(\frac{i}{p}\right)=1\hs\Longrightarrow\hs \left(\frac{i+1}{p}\right)=1.
            \end{align}
            Note $(1/p)=1$.  By applying (\ref{induction}) $p-2$ times starting with $i=1$, we find that 
            all $1\leq x<p$ are quadratic residues modulo $p$. In particular, we find $p-1\equiv -1\mod p$
            is a quadratic residue, directly contradicting (\ref{minus_one}).
      \end{enumerate}
      Since C2 leads to a contradiction, C1 must hold, thus proving the claim.
   \end{proof}
\end{claim*}

\end{document}