\documentclass{article}
\usepackage{homework-preamble}
\DeclareMathOperator{\diam}{diam}
\begin{document}
\title{Introduction to Number Theory: Homework 2}
\author{Franz Miltz (UUN: S1971811)}
\date{11 March 2022}
\maketitle

\begin{claim*}[1]
   Let $p$ be prime such that $p=4k+3$ for some $k\in\N$. Let $0<j<p$ be an integer. Then 
   there exists an integer $r$ such that either $p|j-r^2$ or $p|j+r^2$.
   \begin{proof}
      By contradiction. Assume, for all $r\in\Z$, $p$ divides neither $j-r^2$ nor $j+r^2$. 
      Equivalently, 
      \begin{align}
         \label{assumption1}
         \left(\frac{j}{p}\right)=\left(\frac{-j}{p}\right)=-1,
      \end{align}
      i.e. neither $j$ nor $-j$ are quadratic residues modulo $p$. Note that, since $p=4k+3$
      for some $k$, 
      \begin{align*}
         \left(\frac{-1}{p}\right)=(-1)^{2k+1}=-1
      \end{align*}
      by \emph{Theorem 7} in the notes for lecture 11. However, using the multiplication rule,
      (\ref{assumption1}) leads us to
      \begin{align*}
         \left(\frac{j}{p}\right) = \left(\frac{-j}{p}\right) 
                                  = \left(\frac{-1}{p}\right)\left(\frac{j}{p}\right)
                                  = -\left(\frac{j}{p}\right),
      \end{align*}
      a contradiction. The claim follows.
   \end{proof}
\end{claim*}

\begin{claim*}[2]
   Let $p$ be a prime such that $p=4k+3$ for some $k\in\N$. Then there exist 
   integers $0<x,y<p$ such that $x^2+y^2+1\equiv 0\mod p$.
   \begin{proof}
   \end{proof}
\end{claim*}

\end{document}