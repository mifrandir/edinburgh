\documentclass{article}
\usepackage{homework-preamble}
\DeclareMathOperator{\diam}{diam}
\begin{document}
\title{Introduction to Number Theory: Homework 1}
\author{Franz Miltz (UNN: S1971811)}
\date{7 February 2022}
\maketitle

\begin{claim*}[1]
   Let $p>0$ be prime and let $q\in\N$ such that $\phi\left(p^2\right) = \phi\left(q^2\right)$.
   Then $p=q$.
   \begin{proof}
      Note that $p$ is prime, thus by \emph{Boocher, Proposition 11.9} we
      have $\phi\left(p^2\right)=\phi\left(q^2\right)=p(p-1)$. Assume $p\not= q$.
      Then $q$ cannot be prime, i.e. there exist distinct primes $q_1,...,q_k$ and natural
      numbers $e_1,...,e_k$ such that
      \begin{align*}
         q = q_1^{e_1}\cdots q_k^{e_k}.
      \end{align*}
      However, note that $\phi$ is multiplicative by \emph{Boocher, Theorem 11.10}
      so
      \begin{align}
         \label{factorisation}
         \phi(p^2)
         =\phi(q^2)
         =\phi\left(\left(q_1^{e_1}\cdots q_k^{e_k}\right)^2\right)
         =\phi\left(q_1^{2e_1}\cdots q_k^{2e_k}\right)
         =\phi\left(q_1^{2e_1}\right)\cdots\phi\left(q_k^{2e_k}\right).
      \end{align}
      We observe that $p|\phi\left(q_i^{2e_i}\right)$ for some $i$.
      Further, $\phi\left(q_i^k\right)=q_i^{k-1}(q_i-1)$. Now there are two possibilities:
      \begin{enumerate}
         \item $p|q_i^{2e_i-1}$. Since $p$ and $q_i$ are prime, this implies
               $p=q_i$, i.e. $p|q$. If $q=p^m$ for some $m>1$, then \begin{align*}
                  \phi\left(q^2\right)=p^{2k-1}(p-1)=p^{2k-3}\phi\left(p^2\right)
                  >\phi\left(p^2\right).
               \end{align*}
               This is a contradiction. Thus there must exist another prime number $r$
               such that $r|q$ and $r\not=p$. However, by multiplicativity of $\phi$,
               then
               \begin{align*}
                  \phi\left(q^2\right)
                  \geq \phi\left((pr)^2\right)
                  = \phi\left(p^2\right)\phi\left(r^2\right)
                  >    \phi\left(p^2\right) ,
               \end{align*}
               contradicting the premise again.
         \item $p|q_i-1$. Then we note $p\leq q_i-1$ so $p<q_i$.  Clearly,
               \begin{align*}
                  \phi\left(p^2\right)=p(p-1)<q_i(q_i-1)=\phi\left(q_i^2\right).
               \end{align*}
               However, $\phi\left(q_i^2\right)\leq \phi\left(q^2\right)$ using
               (\ref{factorisation}) and $\phi(n)\geq 1$ for $n\in\N$. Thus
               $\phi\left(p^2\right)<\phi\left(q^2\right)$; contradiction.
      \end{enumerate}
      Since the assumption leads to a contradiction in all cases, we conclude
      that $p=q$.
   \end{proof}
\end{claim*}


\end{document}