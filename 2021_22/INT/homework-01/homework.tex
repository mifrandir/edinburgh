\documentclass{article}
\usepackage{homework-preamble}
\DeclareMathOperator{\diam}{diam}
\begin{document}
\title{Introduction to Number Theory: Homework 1}
\author{Franz Miltz (UUN: S1971811)}
\date{7 February 2022}
\maketitle

\begin{claim*}[1]
	Let $p>0$ be prime and let $q\in\N$ such that $\phi\left(p^2\right) = \phi\left(q^2\right)$.
	Then $p=q$.
	\begin{proof}
		Note that $p$ is prime, thus by \emph{Boocher, Proposition 11.9} we
		have $\phi\left(p^2\right)=\phi\left(q^2\right)=p(p-1)$. Assume $p\not= q$.
		Then $q$ cannot be prime, i.e. there exist distinct primes $q_1,...,q_k$ and natural
		numbers $e_1,...,e_k$ such that
		\begin{align*}
			q = q_1^{e_1}\cdots q_k^{e_k}.
		\end{align*}
		However, note that $\phi$ is multiplicative by \emph{Boocher, Theorem 11.10}
		so
		\begin{align}
			\label{factorisation}
			\phi(p^2)
			=\phi(q^2)
			=\phi\left(\left(q_1^{e_1}\cdots q_k^{e_k}\right)^2\right)
			=\phi\left(q_1^{2e_1}\cdots q_k^{2e_k}\right)
			=\phi\left(q_1^{2e_1}\right)\cdots\phi\left(q_k^{2e_k}\right).
		\end{align}
		We observe that $p|\phi\left(q_i^{2e_i}\right)$ for some $i$.
		Further, $\phi\left(q_i^k\right)=q_i^{k-1}(q_i-1)$. Now there are two possibilities:
		\begin{enumerate}
			\item $p|q_i^{2e_i-1}$. Since $p$ and $q_i$ are prime, this implies
			      $p=q_i$, i.e. $p|q$. If $q=p^m$ for some $m>1$, then \begin{align*}
				      \phi\left(q^2\right)=p^{2k-1}(p-1)=p^{2k-3}\phi\left(p^2\right)
				      >\phi\left(p^2\right).
			      \end{align*}
			      This is a contradiction. Thus there must exist another prime $r$ and some
			      positive integer $s$ such that $q=prs$. However, by multiplicativity of $\phi$,
			      then
			      \begin{align*}
				      \phi\left(q^2\right)
				      =    \phi\left((prs)^2\right)
				      =    \phi\left(p^2\right)\phi\left(r^2\right)\phi(s^2)
				      \geq \phi\left(p^2\right)\phi\left(r^2\right)
				      >    \phi\left(p^2\right) ,
			      \end{align*}
			      contradicting the premise again.
			\item $p|q_i-1$. Then we note $p\leq q_i-1$ so $p<q_i$.  Clearly,
			      \begin{align*}
				      \phi\left(p^2\right)=p(p-1)<q_i(q_i-1)=\phi\left(q_i^2\right).
			      \end{align*}
			      However, $\phi\left(q_i^2\right)\leq \phi\left(q^2\right)$ using
			      (\ref{factorisation}) and $\phi(n)\geq 1$ for $n\in\N$. Thus
			      $\phi\left(p^2\right)<\phi\left(q^2\right)$; contradiction.
		\end{enumerate}
		Since the assumption leads to a contradiction in all cases, we conclude
		that $p=q$.
	\end{proof}
\end{claim*}

\begin{claim*}[2]
	$\phi(2025)=1593$
	\begin{proof}
		We note $2025 = 3^4 5^2$.
		By multiplicativity of $\phi$ and \emph{Boocher's Notes, Corollary 11.11}
		we obtain
		\begin{align*}
			\phi(2025) = \phi(3^4)\phi(5^2) = (81-27)(25-5) = 1593.
		\end{align*}
	\end{proof}
\end{claim*}

\renewcommand{\hs}{\hspace{0.2cm}}
\begin{claim*}[3]
	Let $p=5$ and $q=7$. Then the number of primitive roots modulo $p$ and $q$ are equal.
	\begin{proof}
		Consider $p=5$. We check the candidates $1,...,p-1=4$:
		\begin{align*}
			1^1 \equiv 1,\hs \underline{1^2 \equiv 1},\hs 1^3 \equiv 1,\hs 1^4 \equiv 1 \mod p \\
			2^1 \equiv 2,\hs 2^2 \equiv 4,            \hs 2^3 \equiv 3,\hs 2^4 \equiv 1 \mod p \\
			3^1 \equiv 3,\hs 3^2 \equiv 4,            \hs 3^3 \equiv 2,\hs 3^4 \equiv 1 \mod p \\
			4^1 \equiv 4,\hs \underline{4^2 \equiv 1},\hs 4^3 \equiv 4,\hs 2^4 \equiv 1 \mod p
		\end{align*}
		Thus the set of primitive roots modulo $5$ is $S=\{2,3\}$.

		Now consider $q=7$. We check the candidates $1,...,p-1=6$:
		\begin{align*}
			1^1 \equiv 1,\hs \underline{1^2 \equiv 1},\hs 1^3 \equiv 1,             \hs 1^4 \equiv 1,\hs 1^5 \equiv 1,\hs 1^6 \equiv 1 \mod q \\
			2^1 \equiv 2,\hs 2^2 \equiv 4,            \hs \underline{2^3 \equiv 1}, \hs 2^4 \equiv 2,\hs 2^5 \equiv 4,\hs 2^6 \equiv 1 \mod q \\
			3^1 \equiv 3,\hs 3^2 \equiv 2,            \hs 3^3 \equiv 6,             \hs 3^4 \equiv 4,\hs 3^5 \equiv 5,\hs 3^6 \equiv 1 \mod q \\
			4^1 \equiv 4,\hs 4^2 \equiv 2,            \hs \underline{4^3 \equiv 1}, \hs 4^4 \equiv 4,\hs 4^5 \equiv 2,\hs 4^6 \equiv 1 \mod q \\
			5^1 \equiv 5,\hs 5^2 \equiv 4,            \hs 5^3 \equiv 6,             \hs 5^4 \equiv 2,\hs 5^5 \equiv 3,\hs 5^6 \equiv 1 \mod q \\
			6^1 \equiv 6,\hs \underline{6^2 \equiv 1},\hs 6^3 \equiv 6,             \hs 6^4 \equiv 1,\hs 6^5 \equiv 6,\hs 3^6 \equiv 1 \mod q
		\end{align*}
		Thus the set of primitive roots modulo $7$ is $T=\{3, 5\}$.
		Note $\abs S = \abs T = 2$ as claimed.
	\end{proof}
\end{claim*}

\end{document}