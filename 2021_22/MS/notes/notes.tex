\documentclass{article}
\usepackage{notes-preamble}
\usepackage{enumitem}
\usepackage{mathrsfs}
\begin{document}
\mkthmstwounified
\title{Metric Spaces (SEM6)}
\author{Franz Miltz}
\maketitle
\noindent Textbook: V. Bryant, \emph{Metric Spaces: Iteration and Application}
\tableofcontents
\pagebreak

\section{Metrics, norms and inner products}

\subsection{Metrics and metric spaces}

\begin{definition}[Metric space]
	A \emph{metric} on a non-empty set $X$ is a function $d:X\times X\to\R$
	such that, for all $x,y,z\in X$,
	\begin{enumerate}
		\item $d(x,y)=0$ iff $x=y$,
		\item $d(x,y)=d(y,x)$,
		\item $d(x,y)\leq d(x,z) + d(z,y)$.
	\end{enumerate}
	A \emph{metric space} is an ordered pair $(X,d)$ where $X$ is a non-empty set and
	$d$ is a metric on $X$.
\end{definition}

\begin{lemma}
	The following are metric spaces:
	\begin{enumerate}
		\item $(\R^n, d_1)$ where $d_1:\R^n\times\R^n\to\R$ is defined by \begin{align*}
			      d_1(x,y)=\abs{x_1-y_1}+\cdots+\abs{x_n-y_n}.
		      \end{align*}
		\item $(\R^n, d_\infty)$ where $d_\infty:\R^n\times\R^n\to\R$ is defined by \begin{align*} d_\infty(x,y)=\max\{\abs{x_1-y_1},...,\abs{x_n-y_n}\}.
		      \end{align*}
	\end{enumerate}
\end{lemma}

\begin{lemma}
	Let $a,b\in\R$ with $a<b$.
	Then the set of all real valued continuous functions defined of real valued
	continuous functions defined on $[a,b]$, denoted $C([a,b])$, together with
	the usual operations of addition and scalar multiplication is a vector space.
	Further let $d:C([a,b])\times C([a,b])\to \R$ be given by
	\begin{align}
		\label{metric-cab}
		d(f,g)=\sup\{\abs{f(x)-g(x)}:x\in[a,b]\}.
	\end{align}
	Then $(C([a,b]),d)$ is a metric space.
\end{lemma}

\begin{definition}[Metric subspace]
	Let $(X,d)$ be a metric space. For any non-empty subset $Y\subseteq X$ we define
	a function $d_Y:Y\times Y\to\R$ by
	\begin{align*}
		d_Y(y_1,y_2)=d(y_1,y_2).
	\end{align*}
	Then $(Y,d_Y)$ is a metric subspace of $(X,d)$.
\end{definition}

\begin{definition}[Ultrametric]
	Let $(X,d)$ be a metric space. If for all $x,y,z\in X$,
	\begin{align*}
		d(x,y)\leq \max\{d(x,z),d(z,y)\}
	\end{align*}
	then $d$ is called an \emph{ultrametric}.
\end{definition}

\begin{proposition}[Reverse triangle inequality]
	Let $(X,d)$ be a metric space and let $x,y,z\in X$. Then
	\begin{align*}
		\abs{d(x,y)-d(x,z)}\leq d(y,z).
	\end{align*}
\end{proposition}

\subsection{Norms and normed linear spaces}

\begin{definition}[Normed linear space]
	Let $X$ be a real vector space. A \emph{norm} is a function $\vabs{-}:X\to\R$
	where for all $x,y\in X$ and all $\lambda\in\R$,
	\begin{enumerate}
		\item $\vabs x \geq 0$,
		\item $\vabs x = 0$ iff $x = 0$,
		\item $\vabs{\lambda x}=\abs \lambda \vabs x$,
		\item $\vabs{x+y}\leq \vabs x +\vabs y$.
	\end{enumerate}
	A \emph{normed linear space} is a pair $(X,\vabs{-})$ where $X$ is a real vector
	space and $\vabs{-}$ is a norm on $X$.
\end{definition}

\begin{lemma}
	Let $p\geq 1$ be a real number.
	Let the set of all real sequences $(x_n)_{n\in\N}$ such that the series
	$\sum_{n=1}^\infty \abs{x_n}^p$ converges be denoted by $L^p$.
	We define the following operations:
	\begin{align}
		\label{sequence_operations}
		+:L^1\times L^1\to L^1;(x,y)\mapsto x+y               & =(x_n+y_n)_{n\in\N}=(x_1+y_1,x_2+y_2,...,x_n+y_n,...),                 \\
		\cdot:\R\times L^1\to L^1;(\lambda,x)\mapsto\lambda x & =(\lambda x_n)_{n\in\N}=(\lambda x_1,\lambda x_2,...,\lambda x_n,...).
	\end{align}
	Then $(L^p,+,\cdot)$ is a vector space.
	We define $\vabs{-}_p:L^p\to\R$ by
	\begin{align*}
		\vabs x_p = \left(\sum_{n=1}^\infty \abs{x_n}^p\right)^{1/p}.
	\end{align*}
	Then $(L^p, \vabs{-}_p)$ is a normed linear space.
	Further, we define $d_p:L^p\times L^p\to\R$ by
	\begin{align*}
		d_p(x,y)=\vabs{x-y}_p=\left(\sum_{n=1}^\infty\abs{x_n-y_n}^p\right)^{1/p}.
	\end{align*}
	Then $(L^p,d_p)$ is a metric space.
\end{lemma}

\begin{lemma}
	Let the set of all bounded real sequences be denoted by $L^\infty$. Then, together with
	the usual operations defined in (\ref{sequence_operations}), $L^\infty$ is a vector space.
	We define $\vabs{-}_\infty:L^\infty\to\R$ by
	\begin{align*}
		\vabs x_\infty = \sup\{\abs{x_n}:n\in\N\}.
	\end{align*}
	Then $(L^\infty,\vabs{-}_\infty)$ is a normed linear space.
	Further, let $d_\infty:L^\infty\times L^\infty\to\R$ is given by
	\begin{align*}
		d_\infty(x,y)=\vabs{x-y}_\infty = \sup\{\abs{x_1-y_1,x_2-y_2,...}\}.
	\end{align*}
	Then $(L^\infty, d_\infty)$ is a metric space.
\end{lemma}

\begin{lemma}[Minkowski's inequality]
	Let $p\geq 1$ be a real number. Let $x,y\in L^p$. Then
	\begin{align*}
		\left(\sum_{i=1}^\infty \abs{x_i+y_i}^p\right)^{1/p}
		\leq \left(\sum_{i=1}^\infty \abs{x_i}^p\right)^{1/p}
		+ \left(\sum_{i=1}^\infty \abs{y_i}^p\right)^{1/p}.
	\end{align*}
\end{lemma}

\begin{lemma}[H\"older's inequality]
	Let $p,q\geq 1$ be real numbers with $1/p+1/q=1$. Let $x\in L^p$ and $y\in L^q$. Then
	\begin{align*}
		\sum_{n=1}^\infty \abs{x_ny_n}\leq
		\left(\sum_{n=1}^\infty\abs{x_n}^p\right)^{1/p}
		\left(\sum_{n=1}^\infty\abs{y_n}^q\right)^{1/q}.
	\end{align*}
\end{lemma}

\subsection{Inner products and inner product spaces}

\begin{definition}[Inner product space]
	Let $X$ be a real vector space over $\R$. An \emph{inner product}
	is a map $\lra{.,.}:X\times X\to\R$ where for all $x,y,z\in X$ and
	$\lambda,\mu\in\R$,
	\begin{enumerate}
		\item $\lra{x,x}\geq 0$ with $\lra{x,x}=0$ iff $x=0$,
		\item $\lra{x,y}=\lra{y,x}$,
		\item $\lra{\lambda x+\mu y,z}=\lambda\lra{x,z}+\mu{y,z}$.
	\end{enumerate}
	A \emph{inner product space} is a pair $(X, \lra{.,.})$ where $X$ is a real
	vector space and $\lra{.,.}$ is a norm on $X$.
\end{definition}

\begin{lemma}
	Let $(X,\lra{.,.})$ be an inner product space. Let $\vabs{-}:X\to\R$ be given
	by
	\begin{align*}
		\vabs x = \sqrt{\lra{x,x}}.
	\end{align*}
	Then $(X,\vabs{-})$ is a normed linear space. Further, let $d:X\times X\to\R$
	be given by
	\begin{align*}
		d(x,y)=\vabs{x-y}.
	\end{align*}
	Then $(X,d)$ is a metric space.
\end{lemma}

\section{The real line}

\subsection{Open and closed sets}

\begin{definition}
	A subset $F\subseteq\R$ is said to be \emph{closed} if it contains the limit of all
	its convergent sequences, i.e. if $(x_n)_{n\in\N}$ is any sequence with $x_n\in F$
	for all $n$, and $x_n\to x$ for some $x\in\R$, then $x\in F$.
\end{definition}

\begin{proposition}[2.2]
	The union of finitely many closed sets is a closed set.
\end{proposition}

\begin{proposition}[2.4]
	The intersection of any family of closed sets is a closed set.
\end{proposition}

\begin{definition}
	A subset $G\subseteq\R$ is said to be \emph{open} if it its complement $G^c$ is closed.
\end{definition}

\begin{proposition}[2.6]
	Let $G\subseteq\R$. The following are equivalent:
	\begin{enumerate}
		\item $G$ is open.
		\item For every point $x\in G$, there exists a positive raidus $r$, such that
		      $(x-r,x+r)\subseteq G$.
	\end{enumerate}
\end{proposition}

\begin{proposition}[2.8]
	The union of any family of open sets is an open set.
\end{proposition}

\begin{proposition}[2.9]
	The intersection of finitely many open sets is an open set.
\end{proposition}

\begin{theorem}[2.10]
	A subset of the real line is open iff it is the union of open intervals.
\end{theorem}

\begin{theorem}[2.11]
	Every open subset of the real line can be written as an at most countable union
	of pairwise disjoint open intervals.
\end{theorem}

\subsection{Another look at continuity and uniform continuity}

\begin{proposition}[2.23]
	If $K\subseteq\R$ is compact and $f:K\to\R$ is continuous, then $f$ is
	uniformly continuous.
\end{proposition}

\begin{proposition}[2.24]
	A function $f:\R\to\R$ is continuous iff the inverse image of every
	open set is an open set.
\end{proposition}

\begin{definition}
	Let $A\subseteq\R$. A subset $G\subseteq A$ is said to be \emph{relatively open}
	if there exists an open subset $O\subseteq\R$ such that $G=A\cap O$.
\end{definition}

\begin{theorem}
	Let $A\subseteq\R$. A function $f:A\to\R$ is continuous iff the inverse
	image $\inv f(G)$ of every open set $G$ is a relatively open subset of $A$.
\end{theorem}

\subsection{Compact sets and the Heine-Borel theorem}

\begin{definition}
	A subset $K\subseteq\R$ is said to be\emph{compact} if every sequence of elements
	of $K$ hasa subsequence that converges to an element of $K$.
\end{definition}

\begin{theorem}[2.15]
	On the real line, a set is compact if it is closed and bounded.
\end{theorem}

\begin{theorem}[2.17]
	Let $K$ be a compact set on the real line. If $f:K\to\R$ is continuous
	then it is bounded.
\end{theorem}

\begin{theorem}[Extreme Value Theorem]
	Let $K$ be a compact set on the real line. If $f:K\to\R$ is continuous
	then $f$ has a maximum and a minimum.
\end{theorem}

\begin{definition}
	Let $E$ be a subset of $\R$. An \emph{open cover} of $E$ is a family
	$(G_i)_{i\in I}$ of open sets such that
	\begin{align*}
		E\subseteq \bigcup_{i\in I}G_i.
	\end{align*}
	A \emph{subcover} of $(G_i)_{i\in I}$ is a smaller family $(G_i)_{i\in I'}$,
	where $I'\subseteq I$, that still covers $E$.
\end{definition}

\begin{theorem}[Heine-Borel]
	Let $a,b\in\R$ with $a<b$. Every open cover of $[a,b]$ has a finite subcover.
\end{theorem}

\begin{theorem}[2.21]
	Let $K\subseteq\R$. The following are equivalent:
	\begin{enumerate}
		\item $K$ is compact.
		\item Every open cover of $K$ has a finite subcover.
	\end{enumerate}
\end{theorem}

\section{Open balls}

\begin{definition}
	Let $(X,d)$ be a metric space, $c\in X$ and $r>0$. The \emph{open ball} with center $c$
	and radius $r$, denoted by $B(c,r)$, is the set of all elements of $X$ whose distance
	from $c$ is less than $r$, i.e.
	\begin{align*}
		B(c,r) = \{x\in X : d(x,c)<r\}.
	\end{align*}
	Thet set
	\begin{align*}
		\{x\in X: d(x,c)\leq r\}
	\end{align*}
	is called the \emph{closed ball} with center $c$ and radius $r$.
\end{definition}

\begin{proposition}[3.7]
	Let $(X,d)$ be a metric space, $c\in X$ and $r>0$. For every point $x$ in the open
	ball $B(c,r)$ there exists a positive number $r'$ such that $B(x,r')\subseteq B(c,r)$.
\end{proposition}

\section{Open sets and closed sets}

Let $(X,d)$ be a metric space.

\subsection{Open sets}

\begin{definition}
	Let $A\subseteq X$. A point $a$ in $A$ is
	an \emph{interior point of $A$} if there exists $r>0$ such that $B(a,r)\subseteq A$.
\end{definition}

\begin{definition}
	Let $G\subseteq X$. Then $G$ is \emph{open} if
	every point in $G$ is an interior point of $G$, i.e. for every $x\in G$ there exists
	$r>0$ such that $B(x,r)\subseteq G$.
\end{definition}

\begin{theorem}[4.6]
	Let $(G_i)_{i\in I}$ be any family of open subsets of $X$. Then the union
	$\bigcup_{i\in I} G_i$ is open.
\end{theorem}

\begin{theorem}[4.7]
	Let $G_1,...,G_n\subset X$ be open. Then the intersection $G_1\cap\cdots\cap G_n$
	is open.
\end{theorem}

\begin{proposition}[4.9]
	A set in a metric space is open iff it can be written as a union of open balls.
\end{proposition}

\begin{definition}
	Let $(Y,d_Y)$ be a subspace of $(X,d)$ and let $G\subseteq Y$. Then $G$ is called
	\emph{relatively open} if it is open in the metric space $(Y,d_Y)$.
\end{definition}

\begin{proposition}[4.10]
	Let $(Y,d_Y)$ be a subspace of $(X,d)$ and let $G\subseteq Y$. The following
	are equivalent
	\begin{itemize}
		\item $G$ is open in the metric space $(Y,d_Y)$;
		\item there exists $O\subseteq X$ that is open in the metric space $(X,d)$ such that $G=O\cap Y$.
	\end{itemize}
\end{proposition}

\begin{proposition}[Exercise 4.11]
	Let $(Y,d_Y)$ be a subspace of $(X,d)$ such that $Y$ is open and non-empty. Then a
	subset $G\subseteq Y$ is open in the metric space $(Y,d_Y)$ iff it is open in the
	metric space $(X,d)$.
\end{proposition}

\begin{definition}
	A metric space $(X,d)$ is said to be \emph{discrete} if every subset of $X$ is open.
\end{definition}

\subsection{Closed sets}

\begin{definition}
	Let $F\subseteq X$. Then $F$ is \emph{closed} if $F^c$ is open.
\end{definition}

\begin{theorem}[4.20]
	Let $(F_i)_{i\in I}$ be a family of closed subsets of $X$. Then $\bigcap_{i\in I}F_i$
	is closed.
\end{theorem}

\begin{theorem}[4.21]
	Let $F_1,...,F_n\subseteq X$ be closed. Then $F_1\cup\cdots\cup F_n$ is closed.
\end{theorem}

\begin{definition}
	Let $(Y,d_Y)$ be a subspace of $(X,d)$ and let $F\subseteq Y$. Then $F$ is called
	\emph{relatively closed} if it is closed in the metric space $(Y,d_Y)$.
\end{definition}

\begin{proposition}
	Let $(Y,d_Y)$ be a subspace of $(X,d)$ and let $F\subseteq Y$. The following are
	equivalent
	\begin{itemize}
		\item $F$ is closed in the metric space $(Y,d_Y)$;
		\item there exists $C\subset X$ that is closed in the metric space $(X,d)$ such that
		      $F=C\cap Y$.
		\item $F=\overline F$. (\emph{Proposition 4.20})
		\item $\p F\subseteq F$. (\emph{Problem 4.43})
		\item For all accumulation points $x$ of $F$ in $Y$, $x\in F$. (\emph{Problem 4.47})
		\item For all sequences $(x_n)_{n\in\N}$ of elements in $F$, $x_n\to x$ implies $x\in F$. (\emph{Proposition 5.9})
	\end{itemize}
\end{proposition}

\begin{definition}
	A set $A\subseteq X$ is called \emph{clopen} if it is both closed and open
	in the metric space $(X,d)$.
\end{definition}

\subsection{Interior}

\begin{definition}
	Let $(X,d)$ be a metric space and $A\subseteq X$. The largest open set contained in $A$
	is called the interior of $A$ and is denoted $\interior A$.
\end{definition}

\begin{definition}
	Let $A\subseteq X$. A point $x\in X$ is called an \emph{interior point} of $A$ if
	there exists an open ball $B(x,r)\subseteq A$.
\end{definition}

\begin{proposition}[4.24]
	Let $(X,d)$ be a metric space and $A,B$ be subsets of $X$. Then
	\begin{enumerate}
		\item $\interior\emptyset=\emptyset$ and $\interior X=X$,
		\item $\interior(\interior{A})=\interior A$,
		\item $A$ is open iff $A=\interior A$,
		\item if $A\subseteq B$, then $\interior A\subseteq\interior B$.
	\end{enumerate}
\end{proposition}

\begin{proposition}[4.25]
	Let $(X,d)$ be a metric space, $A\subseteq X$ and $x\in X$. Then $x\in\interior A$ iff
	$x$ is an interior point of $A$.
\end{proposition}

\subsection{Exterior and boundary}

\begin{definition}
	Let $A\subseteq X$. A point $x\in X$ is called an \emph{exterior point} of $A$ if
	there exists an open ball $B(x,r)\subseteq A^c$. The set of all exterior points is denoted
	$\exterior A$ and called \emph{exterior} of $A$.
\end{definition}

\begin{definition}
	Let $A\subseteq X$. A point $x\in X$ is called a \emph{boundary point} of $A$ if, for all
	$r>0$,
	\begin{align*}
		B(x,r)\cap A\not=\emptyset\hs\text{and}\hs B(x,r)\cap A^c\not=\emptyset.
	\end{align*}
	The set of all boundary points of $A$ is denoted $\p A$ and is called the boundary of $A$.
\end{definition}

\subsection{Closure}

\begin{definition}
	Let $A\subseteq X$. The smallest closed set containing $A$, denoted $\overline A$, is called
	the \emph{closure} of $A$.
\end{definition}

\begin{proposition}[4.26]
	Let $A,B\subseteq X$. Then
	\begin{itemize}
		\item $\overline\emptyset = \emptyset$ and $\overline X = X$,
		\item $\overline{\overline A} = \overline A$,
		\item $A$ is closed iff $A=\overline A$,
		\item $A\subseteq B$ implies $\overline A\subseteq\overline B$,
		\item $\overline{A\cup B} = \overline A \cup \overline B$.
	\end{itemize}
\end{proposition}

\begin{theorem}[Lectures Week 5, Theorem 2]
	Let $x\in X$, $A\subseteq X$. Then $x\in\overline A$ iff $\dist(x,A)=0$.
\end{theorem}

\subsection{Adherent points}

\begin{definition}[4.27]
	Let $A\subseteq X$ and $x\in X$. We say that $x$ is an \emph{adherent point of $A$}
	iff, for every $r>0$, we have $B(x,r)\cap A\not=\emptyset$.
\end{definition}

\begin{proposition}[4.29]
	Let $A\subseteq X$. A point $x\in X$ is an adherent point of $A$ iff
	$x\in\overline A$.
\end{proposition}

\begin{definition}
	Let $A\subset X$ and $x\in X$.
	\begin{enumerate}
		\item We say that $x$ is an \emph{accumulation point} of $A$ if, for every $r>0$,
		      \begin{align*}
			      B(x,r)\cap A\not=\{x\}.
		      \end{align*}
		\item We say that $x$ is an \emph{isolated point} of $A$ if there exists $r_0>0$
		      such that \begin{align*}
			      B(x,r_0)\cap A = \{x\}.
		      \end{align*}
	\end{enumerate}
\end{definition}

\subsection{Dense sets}

\begin{definition}
	A subset $D\subseteq X$ is said to be \emph{dense} if, for every $x\in X$
	and every $r>0$, there exists a point $y\in D$ such that $d(x,y)<r$.
\end{definition}

\begin{proposition}[4.33]
	A subset $D\subseteq X$ is dense iff it intersects every non-empty open
	subset of $X$.
\end{proposition}

\section{Convergent sequences}

\subsection{Sequences and their limits}

\begin{definition}
	Let $(X,d)$ be a metric space. A \emph{sequence} of elements of $X$ is a function defined
	on $\N$ with values in $X$. If $x:\N\to\N$ is such a sequence and $n$ is a positive integer
	we write $x_n:=x(n)$ and we denote the whole sequence by $(x_n)_{n\in\N}$.
\end{definition}

\begin{definition}
	Let $(X,d)$ be a metric space. Let $(x_n)_{n\in\N}$ be a sequence of elements of $X$ and let
	$x\in X$. We say that $(x_n)_{n\in\N}$ converges to $x$ if the real sequence $(d(x_n,x))_{n\in\N}$
	converges to zero.
\end{definition}

\begin{proposition}[5.3]
	In a metric space, $x_n\to x$ as $n\to\infty$ iff for every open ball $B$ centered at $x$, there
	exists an index $N$, such that, for all $n\geq N$, we have $x_n\in B$.
\end{proposition}

\begin{proposition}[5.4]
	In a metric space, $x_n\to x$ as $n\to\infty$ iff, for every open set $G$ containing $x$, there
	exists an index $N$, such that, for all indices $n$ with $n\geq N$, we have $x_n\in G$.
\end{proposition}

\begin{proposition}[5.5]
	Let $(X,d)$ be a metric space, let $(x_n)_{n\in\N}$ be a sequence in $X$ and let $x,x'\in X$.
	If $x_n\to x$ and $x_n\to x'$ as $n\to\infty$ then $x=x'$.
\end{proposition}

\begin{proposition}[5.6]
	Let $(X,d)$ be a metric space and let $(x_n)_{n\in\N},(y_n)_{n\in\N}$ be two sequences in $X$
	that converge to limts $x$ and $y$ respectively. Then $d(x_n,y_n)\to d(x,y)$.
\end{proposition}

\begin{corollary}[5.7]
	Let $(X,d)$ be a metric space, $(x_n)_{n\in\N}$ be a convergent sequence in $X$ with limit $x$,
	and $y$ be any point in $X$. Then $d(x_n,y)\to d(x,y)$.
\end{corollary}

\subsection{Bounded sequences}

\begin{definition}
	A sequence $(x_n)_{n\in\N}$ in a metric space is said to be \emph{bounded} if there exists
	a point $c\in X$ and a positive real number $r$ such that, for all $n$, $x_n\in B(c,r)$.
\end{definition}

\begin{proposition}[5.10]
	Let $(X,d)$ be a metric space and $F\subseteq X$. The set $F$ is closed iff the limit of
	every convergent sequence of elements of $F$ is an element of $F$.
\end{proposition}

\subsection{Subsequences}

\begin{definition}
	Let $(x_n)_{n\in\N}$ be a sequence in a metric space and let $(n_k)_{k\in\N}$ be a strictly
	increasing sequence of indices. Then the sequence $(x_{n_k})_{k\in\N}$ is called
	a subsequence of $(x_n)_{n\in\N}$.
\end{definition}

\begin{proposition}
	If a sequence $(x_n)_{n\in\N}$ in a metric space converges to a limit $x$ then every subsequence
	$(x_{n_k})_{k\in\N}$ converges to $x$.
\end{proposition}

\subsection{The metric space $C([a,b])$}

\begin{theorem}
	Let $a<b$ be real numbers. Then define $X=C([a,b])$ as the set of all continuous functions
	$f:[a,b]\to\R$ and the map $d_\infty:X\to X\to\R$ by
	\begin{align*}
		d_\infty(f,g) = \max\{\abs{f(t)-g(t)}: t\in[a,b]\}.
	\end{align*}
	Then $(X,d_\infty)$ is a metric space.
\end{theorem}

\begin{lemma}
	A sequence $(f_n\in C[a,b])_{n\in N}$ converges to $f\in C[a,b]$ iff $(f_n)_{n\in N}$
	converges uniformly to $f$.
\end{lemma}

\section{Continuity}

\begin{definition}
	Let $(X,d_X),(Y,d_Y)$ be metric spaces and $f:X\to Y$ a function. We say $f$ is \emph{continuous
		at a point $x_0$} in $X$ if, for every positive $\e$, there exists a positive $\delta$, such that,
	for all $x\in X$ with $d_X(x,x_0)<\delta$ we have $d_Y(f(x),f(x_0))<\e$.
	We say $f$ is \emph{continuous} if it is continuous at every point in $X$.
\end{definition}

\begin{proposition}[6.2]
	Let $(X,d_X),(Y,d_Y)$ be metric spaces, $f:X\to Y$ be a function and $x_0\in X$. Then $f$ is continuous
	at $x_0$ iff, for every open neighbourhood $G$ of $f(x_0)$, there exists an open neighbourhood $O$ of
	$x_0$, such that, for all $x\in O$, we have $f(x)\in G$.
\end{proposition}

\begin{theorem}[6.5]
	Let $(X,d_X),(Y,d_Y)$ be metric spaces and $f:X\to Y$ a function. Then $f$ is continuous iff
	the inverse image $\inv f(G)$ of any open subset $G\subset Y$ is an open subset of $X$.
\end{theorem}

\section{Completeness}

\subsection{Cauchy sequences}

\begin{definition}
	A sequence $(x_n)_{n\in\N}$ in a metric space $(X,d)$ is said to be a \emph{Cauchy sequence}
	if, for every positive $\e$ there exists an index $N$, such that, for all indices $n,m$ with
	$n,m\geq N$,
	\begin{align*}
		d(x_n,x_m)<\e.
	\end{align*}
\end{definition}

\begin{proposition}[7.2]
	In a metric space, every convergent sequence is a Cauchy sequence.
\end{proposition}

\subsection{Complete metric spaces}

\begin{definition}
	A metric space $(X,d)$ is said to be complete if every Cauchy sequence in $X$ converges to some
	$x\in X$.
\end{definition}

\begin{proposition}[7.5]
	$\R^n$ with the Euclidean metric $d_2$ is a complete metric space.
\end{proposition}

\begin{proposition}[7.6]
	$C([a,b])$ with the supremum metric (\vref{metric-cab}) is complete.
\end{proposition}

\begin{proposition}[7.7]
	Let $(X,d_X)$ and $(Y,d_Y)$ be metric spaces and let $C(X,Y)$ denote the set of all
	bounded continuous functions from $X$ to $X$. We define $D:C(X,Y)\times C(X,Y)\to\R$
	by
	\begin{align*}
		D(f,g)=\sup\{d_Y(f(x),g(x)):x\in X\}.
	\end{align*}
	Then $D$ is a metric. If $(Y,d_Y)$ is complete then $(C(X,Y),D)$ is complete.
\end{proposition}

\begin{proposition}[7.8]
	The metric space $L^2$ is complete.
\end{proposition}

\begin{proposition}[7.10]
	If $(X,d)$ is a complete metric space and $F$ is a closed subset of $X$, then
	the subspace $(F,d_F)$ is complete.
\end{proposition}

\begin{proposition}[7.11]
	Let $(X,d)$ be a metric space and $F$ be a subset of $X$ such that the subspace $(F,d_F)$
	is complete. Then $F$ is a closed subset of $X$.
\end{proposition}

\subsection{Completion of a metric space}

\begin{definition}
	A metric space $(X,d)$ is said to be bounded iff there exist  $c\in X$
	and $r>0$ such that $X\subseteq B(c,r)$.
\end{definition}

\begin{proposition}[7.14]
	Let $(X,d)$ be a bounded metric space. For each $x\in X$ we define the function
	$f_x:X\to\R$ by $f_x(t)=d(x,t)$.
	\begin{enumerate}
		\item Each $f_x$ is a bounded continuous function.
		\item For any $x,x'\in X$, \begin{align*}
			      \sup\{\abs{f_X(t)-f_{X'}(t)}:t\in X\}=d(x,x').
		      \end{align*}
	\end{enumerate}
\end{proposition}

\begin{definition}
	Let $(X,d_X)$ and $(Y,d_Y)$ be metric spaces. A function $f:X\to Y$ is said to be
	an \emph{isometry} if, for all $x,x'\in X$,
	\begin{align*}
		d_Y(f(x),f(x')) = d_X(x,x').
	\end{align*}
	The metric spaces $(X,d_X)$ and $(Y,d_Y)$ are said to be \emph{isometric}
	if there exists an isometry from $X$ onto $Y$.
\end{definition}

\begin{corollary}
	Every isometry is injective.
\end{corollary}

\begin{proposition}[7.16]
	For every metric space $(X,d)$, there exists a complete metric space that has a dense subspace
	isometric to $X$.
\end{proposition}

\section{Separability}

\begin{definition}
	A metric space is called separable iff it has an at most countable dense subset.
\end{definition}
\begin{itemize}
	\item $\R$ with the standard metric is separable due to $\Q$;
	\item $\R^n$ with the Euclidean metric is separable due to $\Q^n$;
	\item $L^2$ is separable due to the set of all rational sequences that are eventually zero;
	\item $L^\infty$ is not separable.
\end{itemize}

\begin{proposition}[8.3]
	In a separable metric space, every familty of pairwise disjoint non-empty open
	sets is at most countable.
\end{proposition}

\begin{definition}
	An open base $B$ in a metric space $(X,d)$ is a collection $B$ of open subsets
	of $X$ such that any open subset of $X$ can be written as a union of elements of $B$.
\end{definition}

\begin{proposition}[8.5]
	Let $(X,d)$ be a metric space and let $D$ be a countable dense subset. Let $B$
	be the collection of open balls centered at elements of $D$ whose radii are positive
	rationals. Then $B$ is an open base.
\end{proposition}

\begin{proposition}[8.6]
	Let $(X,d)$ a metric space, let $\emptyset\subset Y\subseteq X$, and let
	$d_Y$ be the induced metric. If $(X,d)$ is separable then so is $(Y,d_Y)$.
\end{proposition}

\section{Compactness}

\begin{definition}
	A metric space is compact iff every sequence has a convergent subsequence.
\end{definition}

\begin{definition}
	Let $(X,d)$ a metric space and $K\subseteq X$. Then $K$ is a compact subset iff
	every sequence of elements of $K$ has a convergent subsequence whose limit is in
	$K$.
\end{definition}
\begin{itemize}
	\item Every closed and bounded interval on the real line is compact;
	\item not every closed and bounded subset of a metric space is compact in general;
	\item the closed unit ball in $(R^n,d_2)$ is a compact subset;
	\item the closed unit ball in $L^2$ is not a compact subset.
\end{itemize}

\begin{proposition}[9.7, 9.8]
	Every compact subset of a metric space is closed and bounded.
\end{proposition}

\begin{definition}
	A metric space $(X,d)$ is totally bounded iff for every $\e>0$ there exists a
	finite family of open balls of radius $\e$ that covers $X$.
\end{definition}

\begin{proposition}[9.13]
	Every compact metric space is totally bounded.
\end{proposition}

\begin{proposition}[9.14]
	Every compact metric space is separable.
\end{proposition}

\begin{definition}
	Let $(X,d)$ a metric space. An open cover of $X$ is a family $(G_i)_{i\in I}$ such
	that each $G_i$ is open and $X=\bigcup_{i\in I}G_i$. A subcover of $(G_i)_{i\in I}$
	is a subfamily $(G_i)_{i\in I'}$ where $I'\subseteq I$ that still covers $X$.
\end{definition}

\begin{proposition}[Lebesgue number of an open cover]
	Let $(X,d)$ a metric space and $(G_i)_{i\in I}$ an open cover. There exists
	$\delta>0$ such that for any $\emptyset\subset A\subseteq X$ with $\diam A < \delta$
	there exists an $i\in I$ such that $A\subseteq G_i$.
\end{proposition}

\begin{definition}[Finite intersection property]
	Let $(X,d)$ a metric space. A family $(A_i)_{i\in I}$ has the FIP iff
	the intersection of any finite subfamily is non-empty.
\end{definition}

\begin{theorem}[9.20]
	Let $(X,d)$ a metric space. The following are equivalent:
	\begin{enumerate}
		\item $X$ is compact.
		\item Every open cover of $X$ has a finite subcover.
		\item Every family of closed subsets of $X$ with the FIP has a non-empty intersection.
	\end{enumerate}
\end{theorem}

\begin{proposition}[9.22]
	Let $(X,d)$ a metric space and $f:X\to\R$ a function.
	If $X$ is compact and $f$ is continuous then $f$ is bounded.
\end{proposition}

\begin{proposition}[9.23]
	Let $(X,d)$ a metric space and $f:X\to\R$ a function. If $X$ is compact
	and $f$ is continuous then $f$ attains a maximum and a minimum on $X$.
\end{proposition}

\begin{definition}
	Let $(X,d_X),(Y,d_Y)$ metric spaces. A function $f:X\to Y$ is
	uniformly continuous iff, for every $\e>0$, there exists a $\delta>0$ such that,
	for all $x,x'\in X$ with $d_X(x,x')<\delta$, we have $d_Y(f(x),f(x'))<\e$.
\end{definition}

\begin{proposition}[9.26]
	Let $(X,d)$ a metric space and $K\subseteq X$ compact. If a function $f:X\to Y$
	is continuous on $K$ then it is uniformly continuous on $K$.
\end{proposition}

\section{Connectedness}

\subsection{Connected metric spaces}

\begin{definition}
	A metric space $(X,d)$ is disconnected iff there exist non-empty disjoint open sets
	$G_1,G_2$ such that $X=G_1\cup G_2$. Otherwise $X$ is connected.
\end{definition}

\begin{lemma}[10.2]
	Let $(X,d)$ a metric space. Then the following are equivalent:
	\begin{enumerate}
		\item $(X,d)$ is disconnected.
		\item There exists a clopen set $\emptyset\subset S\subset X$.
	\end{enumerate}
\end{lemma}

\begin{proposition}[10.4]
	Let $I\subseteq\R$ an interval. Define $d:I\times I\to\R$ by $\lra{x,y}\mapsto\abs{x-y}$.
	Then $(I,d)$ is a connected metric space.
\end{proposition}

\begin{proposition}[10.5]
	\label{connected_real_interval}
	Let $\emptyset\subset Y\subseteq\R$ and let $d:Y\times Y\to\R$ the standard metric.
	If $(Y,d)$ is a connected metric space then $Y$ is an interval.
\end{proposition}

\subsection{Intermediate value theorem}

\begin{proposition}[10.6]
	\label{connected_range}
	Let $(X,d_X),(Y,d_Y)$ metric spaces and $f:X\to Y$ be a function. If $X$ is connected
	and $f$ is continuous surjection then $Y$ is connected.
\end{proposition}

\begin{theorem}[Intermediate Value Theorem]
	Let $(X,d)$ be a connected metric space and $f:X\to\R$ be a continuous function. If
	$x_1,x_2\in X$ with $f(x_1)\neq f(x_2)$ and $y\in\R$ with $f(x_1)\leq y\leq f(x_2)$ then there exist
	$x\in X$ such that $f(x)=y$.
	\begin{proof}
		$f(X)$ is a connected metric space by \ref{connected_range}.
		$f(X)$ is an interval by \ref{connected_real_interval}.
		Since $f(x_1)$ and $f(x_2)$ belong to the interval, so does $y$.
	\end{proof}
\end{theorem}

\subsection{Connected components}

\begin{definition}
	Let $(X,d)$ a metric space and $Y\subseteq X$. We say that $Y$ is a \emph{disconnected
	subset of $X$} iff there exist open sets $O_1,O_2$ such that 
	\begin{align*}
		Y\subseteq O_1\cup O_2,\: Y\cap O_1\cap O_2 = \emptyset,\: Y\cap O_1\neq\emptyset,
		\:Y\cap O_2\not=\emptyset.
	\end{align*}
	Otherwise we say that $Y$ is a \emph{connected subset of $X$}.
\end{definition}

\begin{proposition}[10.11]
	Let $(X,d)$ a metric space and $A,B\subseteq X$ connected with $A\cap B\neq\emptyset$.
	Then $A\cup B$ is connected.	
\end{proposition}

\begin{lemma}
	The intersection of two connected subsets of a metric space may not be connected.
\end{lemma}

\subsection{Path-connectedness}

\begin{definition}
	Let $(X,d)$ a metric space and $x_0,x_1\in X$. A \emph{path} in $X$ from $x_0$ to $x_1$ is a 
	continuous function $\gamma:[0,1]\to X$ such that $\gamma(0)=x_0$, $\gamma(1)=x_1$.	

	$(X,d)$ is said to be \emph{path-connected} iff for any $x_0,x_1\in X$ there exists a path 
	from $x_0$ to $x_1$.
\end{definition}

\begin{proposition}
	Every path-connected metric space is connected.
\end{proposition}

\end{document}
