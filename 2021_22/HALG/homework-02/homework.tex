\documentclass{article}
\usepackage{homework-preamble}
\mkanonthms
\begin{document}
\title{Honours Algebra: Practicing Proof 2}
\author{Franz Miltz (UUN: S1971811)}
\date{11 March 2022}
\maketitle
\newcommand{\gaussint}{\Z[i]}
We write $i=\sqrt{-1}$ and $\overline x = \overline{r+si} = r - si$.
\begin{claim*}[1]
   $\Z[i]$ is an integral domain.
   \begin{proof}
      Clearly, $\gaussint$ is a non-zero commutative ring. Let $x,y\in\gaussint$ such that 
      $xy=0$. Assume $x\not=0$ and $y\not=0$. We then have $\overline{xy}=0$ and thus 
      \begin{align*}
         (xy)(\overline{xy})=(xy)(\overline x\,\overline y)=(x\overline x)(y\overline y) = 0.
      \end{align*}
      However, note that for any $x=r+si\in\Z$, 
      \begin{align*}
         x\overline x = r^2-s^2 \in\Z.
      \end{align*}
      Thus $x\overline x$ is a zero divisor in $\Z$ contradicting the fact that $\Z$ is an 
      integral domain. Thus $x=y=0$, proving the claim.
   \end{proof}
\end{claim*}

\begin{claim*}[2]
   The group of units of $\gaussint$ has order four. In particular, $\gaussint^\times = \lra{i}$.
   \begin{proof}
      Let $u=a+bi\in\C^\times$. Then there exists $x=c+di\in\C$ such that $(a+bi)(c+di) = 1$.
      Equivalently,
      \begin{align*}
         a+bi = \frac{c-di}{c^2+d^2}.
      \end{align*}
      Therefore 
      \begin{align*}
         a = \frac{c}{c^2+d^2},\hs b=-\frac{d}{c^2+d^2}.
      \end{align*}
      In particular, $a^2+b^2=1$. Note that any unit in $\gaussint$ is also a unit in $\C$. 
      The integer solutions to $a^2+b^2=1$ are $(a,b)\in\{(0,1),(0,-1),(1,0),(-1,0)\}$.
      These correspond to $x\in\{i,-i,1,-1\}\supseteq\gaussint^\times$.

      We now note that 
      \begin{align*}
         (-1)(-1) = 1,\hs i(-i) = 1
      \end{align*}
      and further
      \begin{align*}
         i^2 = -1,\hs i^3 = -i, \hs i^4 = 1.
      \end{align*}
      Therefore all integer solutions correspond to a unit in $\gaussint$, showing 
      $\gaussint^\times=\{i,-i,1,-1\}$, and further $i$ is a generator, i.e. $\gaussint^\times=\lra{i}$.
   \end{proof}
\end{claim*}

\begin{claim*}[3]
   The map $\phi:\gaussint\to\Z_{q}$ given by 
   \begin{align*}
      a + bi \mapsto [a]+[n][b]
   \end{align*}
   is a surjective ring homorphism.
   \begin{proof}
      We note that $(\Z_q,+)$ is an abelian group. Since $n$ and $q$ are coprime, it follows from 
      \emph{Lagrange's theorem} that the order of $[n]$ in $(\Z_q,+)$ is $q$. Thus 
      \begin{align*}
         \Z_q = \lra{[n]} = \{k[n] : k\in\Z\}.
      \end{align*}
      Noting $k[n]=[kn]$ by definition of $[n]$, we find 
      \begin{align*}
         \Z_q = \{[kn] : k\in\Z\} = \{[k][n] : k\in\Z\} = \{\phi(0+ki):k\in\Z\}.
      \end{align*}
      This shows that $\phi$ is surjective.

      Let $(a+bi),(c+di)\in\C$. Then
      \begin{align*}
         \phi(a+bi) + \phi(c+di) = [a] + [n][b] + [c] + [n][c] = [a+c] + [n][b+d] = \phi((a+bi) + (c+di))
      \end{align*}
      where me made extensive use of the definition of the ring $(\Z_q,+,\times)$.
      Similarly,
      \begin{align*}
         \phi(a+bi)\phi(c+di)=([a]+[n][b])([c]+[n][d])=[ac]+[n][ad+bc]+[n][n][bd].
      \end{align*}
      We now note that $[n][n]=[n^2]=[-1]$. Using this, the operations on $\Z_q$ and the above,
      \begin{align*}
         \phi(a+bi)\phi(c+di)=[ac-bd]+[n][ad+bc]=\phi((a+bi)(c+di)).
      \end{align*}
      This shows that $\phi$ is a ring homorphism.
   \end{proof}
\end{claim*}

\begin{claim*}[4]
   \begin{align*}
      \ker\phi = I_n.
   \end{align*} 
   \begin{proof}
      Assume $x\in I_n$. Then $x=(r+si)(n-i)$ for some $r,s\in\Z$. We find 
      \begin{align*}
         \phi(x)=\phi(rn + s + (sn - r)i) = [rn + s] + [n][sn -r] = [rn + s + sn^2 -rn] = [s(n^2+1)] = [0].
      \end{align*}
      Thus $x\in\ker\phi$.

      Assume $x\in\ker\phi$. Then $x=r+si$ for some $r,s\in\Z$. We have
      \begin{align*}
         \phi(x) = [r] + [n][s] = [0].
      \end{align*}
      This clearly implies 
      \begin{align*}
         r + sn \equiv 0 \mod q
      \end{align*}
      and equivalently 
      \begin{align*}
         r &\equiv -sn \mod q\\
         \Leftrightarrow rn &\equiv -sn^2 \mod q\\
         \Leftrightarrow rn - s &\equiv -s(n^2+1) \equiv 0 \mod q
      \end{align*}
      These imply $q|r+sn$ and $q|rn-s$. We define $a,b\in\Z$ by
      \begin{align*}
         a=\frac{rn-s}{n^2+1},\hs 
         b=\frac{r+sn}{n^2+1}.
      \end{align*}
      Now we note that 
      \begin{align*}
         a+bi=\frac{rn-s+(r+sn)i}{n^2+1}=\frac{(r+si)(n+i)}{n^2+1}=\frac{r+si}{n-i}.
      \end{align*}
      Thus $x=(a+bi)(n-i)\in I_n$ as desired.
   \end{proof}
\end{claim*}

\begin{claim*}[5]
   $\gaussint/I_n$ is a field if and only if $q$ is prime.   
   \begin{proof}
      Firstly, note that $\phi$ is surjective so $\im\phi=\Z_q$. Secondly, by the \emph{First Isomorphism
      Theorem for Rings}, there exists a ring isomorphism
      \begin{align*}
         \frac{\gaussint}{\ker\phi}\cong\Z_q. 
      \end{align*}
      Therefore $\gaussint/I_n$ is a field if and only if $\Z_q$ is.
      Finally, by \emph{Proposition 3.2.16} in the notes, we know that $\Z_q$ is a field if and only if $q$ is prime.
   \end{proof}
\end{claim*}

\end{document}