\documentclass{article}
\usepackage{notes-preamble}
\usepackage{enumitem}
\begin{document}
\mkthmstwounified
\title{Honours Algebra (SEM6)}
\author{Franz Miltz}
\maketitle
\tableofcontents
\pagebreak

\section{Vector spaces}

\subsection{Solutions of simultaneous linear equations}

\begin{definition}
    Let $F$ be a field. A \emph{system of linear equations} with $n$ equations and 
    $m$ unknowns are equations 
    \begin{align*}
        a_{11}x_1 + a_{12}x_2 + \cdots + a_{1m}x_m &= b_1\\
        a_{21}x_1 + a_{22}x_2 + \cdots + a_{2m}x_m &= b_2\\
         \vdots \\
        a_{n1}x_1 + a_{n2}x_2 + \cdots + a_{nm}x_m &= b_n\\
    \end{align*}
    where $a_{ij},b_i\in F$ are fixed.

    If all $b_i$ are zero, then we call our system \emph{homogeneous}. Every system 
    has an associated \emph{homogenised} system of equations.
    The set $L\subseteq F^m$ consisting of all $m$-tuples that satisfy the 
    $n$ equations is called the \emph{solution set} of our system.
\end{definition}

\begin{theorem}[Notes 1.1.4]
    If the solution set of a linear system of equations is non-empty, then we obtain
    all solutions by adding componentwise an arbitrary solution of the associated 
    homogenised system to a fixed soltuion of the system itself.
\end{theorem}

\subsection{Fields and vector spaces}

\begin{definition}
    A \emph{field} $F$ is a set with functions 
    \begin{align*}
        + &: F\times F \to F ; (\lambda,\mu) \mapsto \lambda + \mu,\\
        \cdot &: F\times F \to F ; (\lambda,\mu) \mapsto \lambda \mu
    \end{align*}
    such that $(F,+)$ and $(F\setminus{0},\cdot)$ are abelian groups, with 
    \begin{align*}
        \lambda(\mu + \nu) = \lambda \mu + \lambda \nu \in F
    \end{align*}
    for any $\lambda,\mu,\nu\in F$. The neutral elements are called $0_F,1_F$.
\end{definition}

\begin{definition}
    A \emph{vector space $V$ over a field $F$} is a pair consisting of an abelian 
    group $V=(V,+)$ and a mapping 
    \begin{align*}
        F\times V\to V:(\lambda,\vec v) \mapsto \lambda \vec v
    \end{align*}
    such that for all $\lambda,\mu\in F$ and $\vec v,\vec w\in V$ the following 
    identities hold:
    \begin{align*}
        \lambda(\vec v + \vec w) &= (\lambda\vec v) + (\lambda \vec w)\\
        (\lambda +\mu)\vec v &= (\lambda\vec v) + (\mu \vec v)\\
        \lambda(\mu\vec v) &= (\lambda\mu)\vec v\\
        1_F \vec v &= \vec v
    \end{align*}
\end{definition}

\begin{lemma}[Notes 1.2.2]
    If $V$ is a vector space and $\vec v\in V$, then $0\vec v =\vec 0$.
\end{lemma}

\begin{lemma}[Notes 1.2.3]
    If $V$ is a vector space and $\vec v\in V$, then $(-1)\vec v = -\vec v$.
\end{lemma}

\begin{lemma}[Notes 1.2.4]
    If $V$ is a vector space over a field $F$, then $\lambda\vec 0=\vec 0$
    for all $\lambda F$. Further, if $\lambda\vec v=\vec 0$ then either 
    $\lambda =0$ or $\vec v =\vec 0$.
\end{lemma}

\begin{definition}
    The one-element abelian group $V=0$ equipped with the 
    obvious operation is a vector space over every field, and called the 
    \emph{trivial vector space} or the \emph{zero vector space}.
\end{definition}

\subsection{Vector subspaces}

\begin{definition}
    A subset $U$ of a vector space $V$ is called a \emph{vector subspace} or 
    \emph{subspace} if $U$ contains the zero vector and whenever $\vec u,\vec v\in U$
    and $\lambda\in F$ we have $\vec u + \vec v\in U$ and $\lambda\vec v\in U$.
\end{definition}

\begin{proposition}[Notes 1.4.5]
    Let $T$ be a subset of a vector space $V$ over a field $F$. Then amongst all
    vector subspaces of $V$ that include $T$ there is a smallest vector subspace 
    \begin{align*}
        \lra{T} = \lra{T}_F\subseteq V
    \end{align*}
    called the \emph{span of $T$}.
    It can be described as the set of all vectors $\alpha_1\vec v_1+\cdots+\alpha_r\vec v_r$
    with $\alpha_1,...,\alpha_r\in F$ and $\vec v_1,...,\vec v_r\in T$, together 
    with the zero vector in the case $T=\emptyset$.
\end{proposition}

\begin{definition}
    A subset $U$ of a vector space $V$ is called a \emph{generating set of $V$}
    if its span is all of $V$. A vector space that has a finite generating set 
    is said to be \emph{finitely generated}.
\end{definition}

\subsection{Linear independence and bases}

\begin{definition}
    A subset $L$ of a vector space $V$ is called \emph{linearly independent}
    if for all pairwise different vectors $\vec v_1,...,\vec v_r\in L$ and 
    arbitrary scalars $\alpha_1,...,\alpha_r\in F$,
    \begin{align*}
        \alpha_1\vec v_1+\cdots+\alpha_r\vec v_r = \vec 0
        \hs\text{implies}\hs
        \alpha_1=\cdots=\alpha_r=0.
    \end{align*}
\end{definition}

\begin{definition}
    A \emph{basis of a vector space} $V$ is a linearly independent generating 
    set in $V$.
\end{definition}

\begin{theorem}[Notes 1.5.11]
    Let $F$ be a field, $V$ a vector space over $F$ and $\vec v_1,...,\vec v_r\in V$
    vectors. The family $\left(\vec v_i\right)_{1\leq i\leq r}$ is a basis of $V$
    if and only if the following mapping 
    \begin{align*}
        \Phi:F^r&\to V\\
        (\alpha_1,...,\alpha_r)&\mapsto \alpha_1\vec v_1 + \cdots + \alpha_r\vec v_r
    \end{align*}
    is a bijection.
\end{theorem}

\begin{theorem}[Theorem 1.5.12]
    The following are equivalent for a subset $E$ of a vectors space $V$:
    \begin{enumerate}
        \item $E$ is a basis of $V$,
        \item $E$ is minimal among all generating sets, 
        \item $E$ is maximal among all linearly independent subsets.
    \end{enumerate} 
\end{theorem}

\begin{corollary}[Notes 1.5.13]
    Let $V$ be a finitely generated vector space over a field $F$. Then $V$
    has a basis. 
\end{corollary}

\begin{theorem}[Notes 1.5.14]
    Let $V$ be a vector space.
    \begin{enumerate}
        \item Let $L\subseteq V$ be a linearly independent subset and $E$ is minimal 
            amongst all generating sets of $V$ with the property that $L\subseteq E$,
            then $E$ is a basis. 
        \item If $E\subseteq V$ is a generating set and if $L$ is maximal amongst all
            linearly independent subsets of $V$ with the property $L\subseteq E$, then 
            $L$ is a basis.
    \end{enumerate} 
\end{theorem}

\begin{definition}
    Let $X$ be a set and $F$ a field. The set $\text{Maps}(X,F)$ of all mappings
    $f:X\to F$ becomes an $F$-vector space with the operations of pointwise 
    addition and multiplication by a scalar. The subset of all mappings which send 
    almost all elements of $X$ to zero is a vector subspace 
    \begin{align*}
        F\lra{X}\subseteq \text{Maps}(X,F).
    \end{align*}
    This vector subspace is called the \emph{free vector space on the set $X$}.
\end{definition}

\begin{theorem}[Notes 1.5.16]
    Let $F$ be a field, $V$ and $F$-vector space and $(\vec v_i)_{i\in I}$ a 
    family of vectors from $V$. The following are equivalent:
    \begin{enumerate}
        \item The family $(\vec v_i)_{i\in I}$ is a basis for $V$,
        \item for each vector $\vec v\in V$ there is precisely one family $(a_i)_{i\in I}$
            of elements in $F$, almost all of which are zero and such that \begin{align*}
                \vec v = \sum_{i\in I}a_i\vec v_i.
            \end{align*}
    \end{enumerate} 
\end{theorem}

\subsection{Dimension of a vector space}

\begin{theorem}[Fundamental Estimate of Linear Algebra; Notes 1.6.1]
    No linearly independent subset of a given vector space has more elements than
    a generating set. Thus if $V$ is a vector space, $L\subset V$ a linearly independent
    subset and $E\subseteq V$ a generating set, then \begin{align*}
        \abs L \leq \abs E.
    \end{align*} 
\end{theorem}

\begin{theorem}[Steinitz Exchange Theorem; Notes 1.6.2]
    Let $V$ be a vector space, $L\subset V$ a finite linearly independent subset 
    and $E\subseteq V$ a generating set . Then there is an injection $\phi:L\to E$
    such that $(E\setminus\phi(L))\cup L$ is also a generating set for $V$.
\end{theorem}

\begin{lemma}[Exchange Lemma; Notes 1.6.3]
    Let $V$ be a vector space, $M\subseteq V$ a linearly independent subset, and 
    $E\subseteq V$ a generating subset, such that $M\subseteq E$. If $\vec w\in V\setminus M$
    such that $M\cap\{\vec w\}$ is linearly independent, then there exists 
    $\vec e\in E\setminus M$ such that $\{E\setminus\{\vec e\}\}\cup \{\vec w\}$ is a 
    generating set for $V$.
\end{lemma}

\begin{corollary}[Notes 1.6.4]
    Let $V$ be a finitely generated vector space.
    \begin{enumerate}
        \item $V$ has a finite basis.
        \item $V$ cannot have an infinite basis.
        \item Any two bases of $V$ have the same number of elements.
    \end{enumerate} 
\end{corollary}

\begin{definition}
    The cardinality of a basis of a finitely generated vector space $V$
    is called the \emph{dimension} of $V$ and will be denoted $\dim V$. 
    If $\dim V=\infty$ we say $V$ is \emph{infinite dimensional}.
\end{definition}

\begin{corollary}[Notes 1.6.7]
    Let $V$ be a finitely generated vector space.
    \begin{enumerate}
        \item Each linearly independent subset $L\subseteq V$ has at most
            $\dim V$ elements, and if $\abs L=\dim V$ then $L$ a basis,
        \item each generating set $E\subset V$ has at least $\dim V$ elements, 
            and if $\abs E=\dim V$ then $E$ is actually a basis.
    \end{enumerate} 
\end{corollary}

\begin{corollary}[Notes 1.6.8]
    A proper vector subspace of a finite dimensional vector space has itself a 
    strictly smaller dimension. 
\end{corollary}

\subsection{Linear Mappings}

\begin{definition}
    Let $V,W$ be vector spaces over a field $F$. A mapping $f:V\to W$ is called
    \emph{$F$-linear} or a \emph{homomorphism} if for all $\vec v_1,\vec v_1\in V$ and $\lambda \in F$ we 
    have 
    \begin{align*}
        f(\vec v_1+\vec v_2) &= f(\vec v_1) + f(\vec v_2),\\
        f(\lambda \vec v_1) &= \lambda f(\vec v_1).
    \end{align*}
    A bijective linear mapping is called an \emph{isomorphism} of vector spaces. 
    A homorphism from one vector space to itself is called an \emph{endomorphism}.
 An isomorphism of a vector space to itself is called an \emph{automorphism}.
\end{definition}

\begin{definition}
    A point that is sent to itself by a mapping is called a \emph{fixed point} of 
    the mapping. Given a mapping $f:X\to X$, we denote the set of fixed points 
    \begin{align*}
        X^f = \{x\in X : f(x)=x\}.
    \end{align*}
\end{definition}

\begin{definition}
    Two vector subspaces $V_1,V_2$ of a vector space $V$ are called \emph{complementary}
    if addition defines a bijection 
    \begin{align*}
        V_1\times V_2 \to V.
    \end{align*}
\end{definition}

\begin{definition}
    Let $V$ be a vector space with vector subspaces $V_1,...,V_n$. Then the vector 
    subspace of $V$ they generated is called the sum and denoted by $V_1+\cdots+V_n$.
    If the natural homomorphism given by addition $V_1+\cdots +V_n\to V$ is an 
    injection then we say the \emph{sum of the vector subspaces $V_i$ is direct} and
    we write it as $V_1\oplus\cdots\oplus V_n$.
\end{definition}

\begin{theorem}[Notes 1.7.7]
    Let $n$ be a natural number. Then a vector space over a field is isomorphic 
    to $F^n$ if and only if it has dimension $n$.
\end{theorem}

\begin{lemma}[Notes 1.7.8]
    Let $V,W$ be vector spaces over $F$ and let $B\subset V$ be a basis. Then 
    restriction of a mapping gives a bijection 
    \begin{align*}
        \hom_F(V,W)&\to \text{Maps}(B,W)\\
        f&\mapsto \eval{f}{B}
    \end{align*} 
\end{lemma}

\begin{proposition}[Notes 1.7.9]
   Let $V,W$ be vector spaces.
   \begin{enumerate}
       \item Every injective mapping $f:V\to W$ has a \emph{left inverse}.
       \item Every surjective mapping $f:V\to W$ has a \emph{right inverse}.
   \end{enumerate} 
\end{proposition}

\subsection{Rank-nullity theorem}

\begin{definition}
    The \emph{image} of a linear mapping $f:V\to W$ is the subset $\gim(f)=f(V)\subset W$.
    The dimension of the image $\dim(\gim V)$ is called the rank of $f$.
    The preimage of the zero vector of a linear mapping $f:V\to W$ is denoted by
    \begin{align*}
        \gke(f):=\inv f(0)=\{v\in V : f(v)=0\}
    \end{align*} 
    and is called the \emph{kernel} of the linear mapping $f$. The dimension of the 
    kernel $\dim(\gke V)$ is called the nullity of $f$.
\end{definition}

\begin{lemma}[Notes 1.8.2]
    A linear mapping is injective if and only if its kernel is zero.
\end{lemma}

\begin{theorem}[Rank-Nullity Theorem; Notes 1.8.4]
    Let $f:V\to W$ be a linear mapping between two vector spaces. Then
    \begin{align*}
        \dim V = \dim(\gke f) + \dim(\gim f).
    \end{align*} 
\end{theorem}

\begin{corollary}
    Let $V$ be a vector space, and $U,W\subseteq V$ vector subspaces. Then
    \begin{align*}
        \dim(U+W)+\dim(U\cap W) = \dim U + \dim W.
    \end{align*}
\end{corollary}

\end{document}
