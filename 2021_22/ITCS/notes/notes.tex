\documentclass{article}
\usepackage{notes-preamble}
\usepackage{enumitem}
\mkthmstwounified

\title{Introduction to Theoretical Computer Science (SEM6)}
\author{Franz Miltz}
\begin{document}
\maketitle
\noindent Textbook: M. Sipser, \emph{Introduction to the Theory of Computation}
\tableofcontents
\pagebreak

\section{Regular Languages}

\subsection{Finite automata}

\begin{definition}[DFA; Sipser, p. 35]
    A \emph{deterministic finite automaton} is a $5$-tuple $(Q,\Sigma,\delta,q_0,F)$ where 
    \begin{enumerate}
        \item $Q$ is a finite set called the \emph{states},
        \item $\Sigma$ is a finite set called the \emph{alphabet},
        \item $\delta:Q\times\Sigma\to Q$ is the \emph{transition function},
        \item $q_0\in Q$ is the \emph{start state}, and 
        \item $F\subseteq Q$ is the \emph{set of accept states}.
    \end{enumerate} 
\end{definition}

\begin{definition}
    A \emph{language} $A$ over an alphabet $\Sigma$ is a set such that for all 
    $w\in A$ there exist $w_1,w_2,...,w_n\in\Sigma$ such that $w=w_1w_2\cdots w_n$.
    
    A finite automaton $M=(Q,\Sigma,\delta,q_0,F)$ accepts a word $w=w_1\cdots w_n\in A$ 
    if there exists a sequence of states $r_0,...,r_n\in Q$ such that 
    \begin{enumerate}
        \item $r_0=q_0$,
        \item $\delta(r_i,w_{i+1})=r_{i+1}$, for $i<n$, and 
        \item $r_n\in F$.
    \end{enumerate}
    We say $M$ recognises $A$ if and only if $A=\{w : M\text{ accepts }w\}$.
\end{definition}

\begin{definition}[Sipser p. 40]
    A language is called a \emph{regular language} if some finite automaton recognises it.
\end{definition}

\begin{definition}[Sipser p. 44]
    Let $A$ and $B$ be languages. We define the following \emph{regular operations}:
    \begin{enumerate}
        \item \emph{union}: $A\cup B=\{x : x \in A \text{ or } x \in B\}$.
        \item \emph{concatenation}: $A\circ B=\{xy : x \in A \text{ or } y \in B\}$.
        \item \emph{star}: $A^* = \{x_1x_2\dots x_k:k\geq 0 \text{ and each }x_i\in A\}$.
    \end{enumerate}
\end{definition}

\begin{theorem}[Sipser p. 45, 60, 62]
    The class of regular languages is closed under regular operations.
\end{theorem}

\begin{theorem}[Pumping lemma; Sipser p. 78]
    If $A$ is a regular language, then there is a number $p$ where if $s$
    is any string in $A$ of length at least $p$, then $s$ may be divided 
    into three pieces, $s=xyz$, satisfying the following conditions:
    \begin{enumerate}
        \item for each $i\geq 0$, $xy^iz\in A$,
        \item $\abs y > 0$, and 
        \item $\abs{xy} \leq p$.
    \end{enumerate} 
\end{theorem}

\subsection{Nondeterminism}

\begin{definition}[NFA; Sipser p. 53]
    A \emph{nondeterministic finite automaton} is a $5$-tuple $(Q,\Sigma,\delta,q_0,F)$ where 
    \begin{enumerate}
        \item $Q$ is a finite set called the \emph{states},
        \item $\Sigma$ is a finite set called the \emph{alphabet},
        \item $\delta:Q\times\Sigma\to\mathcal{P}(Q)$ is the \emph{transition function},
        \item $q_0\in Q$ is the \emph{start state}, and 
        \item $F\subseteq Q$ is the \emph{set of accept states}.
    \end{enumerate} 
\end{definition}

\begin{definition}[$\e$-NFA; Sipser p. 53]
    An \emph{$\e$-NFA} is a $5$-tuple $(Q,\Sigma,\delta,q_0,F)$ where 
    \begin{enumerate}
        \item $Q$ is a finite set called the \emph{states},
        \item $\Sigma$ is a finite set called the \emph{alphabet},
        \item $\delta:Q\times\Sigma_\e\to \mathcal{P}(Q)$ is the \emph{transition function},
        \item $q_0\in Q$ is the \emph{start state}, and 
        \item $F\subseteq Q$ is the \emph{set of accept states}.
    \end{enumerate} 
\end{definition}

\begin{theorem}[Sipser p. 55]
    Let $L$ be a language. Then the following statements are equivalent:
    \begin{enumerate}
        \item $L$ is regular.
        \item There exists a DFA $M$ that recognises $L$.
        \item There exists an NFA $M$ that recognises $L$.
        \item There exists an $\e$-NFA $M$ that recognises $L$.
    \end{enumerate}
\end{theorem}

\subsection{Regular expressions}

\begin{definition}[Sipser p. 64]
    $R$ is a \emph{regular expression} over an alphabet $\Sigma$ if one of the following holds:
    \begin{enumerate}
        \item $R\in\Sigma$,
        \item $R=\e$,
        \item $R=\emptyset$,
        \item $R=(R_1\cup R_2)$, where $R_1,R_2$ are regular expressions,
        \item $R=(R_1\circ R_2)$, where $R_1,R_2$ are regular expressions, or
        \item $R=(R_1^*)$, where $R_1$ is a regular expression.
    \end{enumerate} 
\end{definition}

\begin{theorem}[Sipser p. 66]
    A language is regular if and only if some regular expression describes it. 
\end{theorem}

\begin{definition}[GNFA; Sipser p. 53]
    A \emph{generalised NFA} is a $5$-tuple $(Q,\Sigma,\delta,q_0,q_1)$ where 
    \begin{enumerate}
        \item $Q$ is a finite set called the \emph{states},
        \item $\Sigma$ is a finite set called the \emph{alphabet},
        \item $\delta:(Q\setminus\{q_0\})\times(Q\setminus\{q_1\})\to R$ is the \emph{transition function},
        \item $q_0\in Q$ is the \emph{start state}, and 
        \item $q_1\in Q$ is the \emph{accept state},
    \end{enumerate} 
    where $R$ is the collection of all regular expressions over the alphabet $\Sigma$.
\end{definition}

\section{Context-free languages}

\subsection{Context-free grammars}

\begin{definition}[Sipser p. 104]
    A \emph{context-free grammar} (CFG) is a 4-tuple $(V,\Sigma,R,S)$ where
    \begin{enumerate}
        \item $V$ is a finite set called the \emph{variables},
        \item $\Sigma$ is a finite set called the \emph{terminals},
        \item $R\subseteq V\times(V\cup\Sigma)^*$ is a finite set of \emph{rules},
        \item $S\in V$ is the \emph{start variable}.
    \end{enumerate} 
\end{definition}

\begin{definition}
    Let $G=(V,\Sigma,R,S)$ be a CFG and let $u,v\in(V\cup\Sigma)^*$.
    We say \emph{$u$ derives $v$}, written $u\Rightarrow^* v$, if $u=v$ or if a sequence
    $u_1,...,u_k$ exists for $k\geq 0$ such that 
    \begin{align*}
        u\Rightarrow u_1 \Rightarrow ... \Rightarrow u_k \Rightarrow v.
    \end{align*}
    The \emph{language of the grammar} $G$ is $L=\{w\in\Sigma^*:S\Rightarrow^* w\}$,
    called a \emph{context-free language} (CFL).
\end{definition}

\begin{definition}[Sipser p. 108]
    A string $w$ is derived \emph{ambiguously} in CFG $G$ if it has 
    two or more different leftmost derivations. Grammar $G$ is \emph{ambiguous} if it 
    generates some string ambiguously. 
    A language which is only generated by ambiguous grammars is called \emph{inherently
    ambiguous}.
\end{definition}

\begin{definition}[Sipser p. 109]
    A CFG $G=(V,\Sigma,R,S)$ is in \emph{Chomsky normal form} if every rule is of the 
    form 
    \begin{align*}
        A\to BC, \hs
        A\to a, \hs
        S\to\e
    \end{align*}
    where $a\in\Sigma$ and $B,C\in V$
\end{definition}

\begin{theorem}[Sipser p. 109]
    Any CFL is generated by a CFG in Chomsky normal form. 
\end{theorem}

\subsection{Pushdown automata}

\begin{definition}[Sipser p. 113]
    A \emph{pushdown automaton} (PDA) is a 6-tuple $(Q,\Sigma,\Gamma,\delta,q_0,F)$,
    where 
    \begin{enumerate}
        \item $Q$ is the finite set of states,
        \item $\Sigma$ is the finite input alphabet,
        \item $\Gamma$ is a finite set called the \emph{stack alphabet},
        \item $\delta:Q\times \Sigma_\e\times\Gamma_\e\to\mathcal{P}(Q\times\Gamma_\e)$ is the transition function,
        \item $q_0\in Q$ is the start state, and 
        \item $F\subseteq Q$ is the set of accept states.
    \end{enumerate}
    Let $M=(Q,\Sigma,\Gamma,\delta,q_0,F)$ and $w=w_1\cdots w_m$, where
    $w_1,...,w_m\in\Sigma_\e$. Then $M$ accepts $w$ if there exists a sequence of 
    states $r_0,...,r_m\in Q$ and strings $s_0,...,s_m\in\Gamma^*$ such that 
    \begin{enumerate}
        \item $r_0=q_0$ and $s_0=\e$,
        \item for $i<m$, we have $(r_{i+1},b)\in\delta(r_i,w_{i+1},a)$, where $s_i=at$
            and $s_{i+1}=bt$ for some $a,b\in\Gamma_\e$ and $t\in\Gamma^*$,
        \item $r_m\in F$.
    \end{enumerate}
\end{definition}

\begin{theorem}[Sipser p. 117]
    A language is context-free if and only if a pushdown automaton recognises it. 
\end{theorem}

\begin{corollary}[Sipser p. 124]
    Every regular language is context-free.
\end{corollary}

\subsection{Non-context-free languages}

\begin{theorem}[Pumping lemma for context-free languages; Sipser p. 125]
    If $A$ is a context-free language, then there is a number $p$ where,
    if $s$ is any string in $A$ of length at least $p$, then $s$ may be 
    divided into five pieces $s=uvxyz$ such that 
    \begin{enumerate}
        \item for each $i\geq 0$ $uv^ixy^iz\in A$,
        \item $\abs{vy}>0$, and 
        \item $\abs{vxy}\leq p$.
    \end{enumerate} 
\end{theorem}

\subsection{Deterministic context-free languages}

Let $G=(V,\Sigma,R,S)$ be a CFG. 

\begin{definition}[Sipser p. 130]
    A \emph{deterministic PDA} is a 6-tuple $(Q,\Sigma,\Gamma,\delta,q_0,F)$,
    where 
    \begin{enumerate}
        \item $Q$ is the finite set of states,
        \item $\Sigma$ is the finite input alphabet,
        \item $\Gamma$ is the finite stack alphabet,
        \item $\delta:Q\times\Sigma_\e\times\Gamma_\e\to(Q\times\Gamma_\e)\cup\{\emptyset\}$ is the transition function 
        \item $q_0\in Q$ is the start state, and 
        \item $F\subseteq Q$ is the set of accept states.
    \end{enumerate}
    The transition function $\delta$ must satisfy the following condition. For every 
    $q\in Q$, $a\in\Sigma$, and $x\in\Gamma$, exactly one of the values
    \begin{align*}
        \delta(q,a,x),\hs \delta(q,a,\e),\hs \delta(q,\e,x),\hs \delta(q,\e,\e)
    \end{align*}
    is not $\emptyset$. The language accepted by a DPDA is called a \emph{deterministic
    context-free language} (DCFL).
\end{definition}

\begin{lemma}[Sipser p. 132]
    Every DPDA has an equivalent DPDA that always reads the entire input string.
\end{lemma}

\begin{theorem}[Sipser p. 133]
    The class of DCFLs is closed under complementation.
\end{theorem}

\begin{definition}[Sipser p. 134]
    Let $A$ be a language. Then the \emph{endmarked language} $A\!\dashv$ is given by 
    \begin{align*}
        A\!\dashv\: = \{w\!\dashv\:: w \in A\}.
    \end{align*}
\end{definition}

\begin{theorem}[Sipser p. 134]
    Let $A$ be a language. Then $A$ is a DCFL if and only if $A\!\dashv$ is a DCFL.
\end{theorem}

\begin{definition}[Sipser p. 136]
    Let $u,v\in(V\cup\Sigma)^*$. We say $u$
    \emph{reduces} $v$, written $u\Leftarrow^*v$ if there exists a sequence 
    $u_1,...,u_k$ for $k\geq 0$ such that 
    \begin{align*}
        u = u_1 \Leftarrow ... \Leftarrow u_k \Leftarrow v.
    \end{align*}
    The sequence $u_1,...,u_k$ is called a \emph{reduction from u to v}. A \emph{reduction from $u$}
    is a reduction from $u$ to $S$.
\end{definition}

\begin{definition}[Sipser p. 136]
    In a \emph{leftmost reduction}, each reducing string is reduced only after all
    other reducing strings that lie entirely to its left.

    Let $w\in L(G)$ and let $u_1,...,u_k$ be a leftmost reduction of $w$. 
    Let $T\to h$ be a rule in $R$ such that $u_i=xhy$ and $u_{i+1}=xTy$ for some $x,y$ and some 
    $i<k$. Then $h$ together with the rule $T\to h$ is called a \emph{handle} of $u_i$.
\end{definition}

\begin{definition}[Sipser p. 136]
    A string $u\in(V\cup\Sigma)^*$ is called \emph{valid} if 
    there exists a word $w\in L(G)$ and a leftmost reduction $u_1,...,u_k$ such that $u=u_i$ for
    some $i\leq k$.
\end{definition}

\begin{lemma}[Sipser p. 137]
    In an unambiguous CFG, any valid string has precisely one handle.
\end{lemma}

\begin{definition}[Forced handles; Sipser p. 138]
    A handle $h$ of a valid string $v=xhy$ with respect to $G$ is \emph{forced} if $h$ is 
    the unique handle in every valid string $xh\hat y$ where $\hat y\in\Sigma^*$.
\end{definition}

\begin{definition}[DCFG; Sipser p. 139]
    A \emph{deterministic CFG} (DCFG) is a CFG such that every valid string has a forced handle.
\end{definition}

\begin{lemma}[Sipser p. 140]
    \label{klemma}
    Let $G=(V,\Sigma,R,S)$ be a DCFG and let $K$ be an NFA such that the states correspond to 
    dotted rules in $G$, i.e. for every rule $B\to u_1\cdots u_k$ with $u_1,...,u_k\in V\cup\Sigma$
    there are $k+1$ dotted rules
    \begin{align*}
        B&\to .u_1u_2\cdots u_k\\
        B&\to u_1.u_2\cdots u_k\\
        B&\to u_1u_2.\cdots u_k\\
        &\hs\vdots\\
        B&\to u_1u_2\cdots .u_k\\
        B&\to u_1u_2\cdots u_k,
    \end{align*}
    the transitions correspond to shift-moves, where for some rule $B\to uav$ we transition from 
    $B\to u.av$ to $B\to ua.v$ upon reading $a$, and $\e$-moves, where for some rules 
    $B\to uCv$ and $C\to r$ we $\e$-transition from $B\to u.Cv$ to $C\to r.$ and the accept states 
    are all $B\to u.$.

    Then $K$ may enter state $T\to u.v$ on reading input $z$ if and only if $z=xu$ and 
    $xuvy$ is a valid string with handle $uv$ and reducing rule $T\to uv$, for some $y\in\Sigma^*$.
    Further, $K$ may enter accept state $T\to h.$ on input $z$ if and only if $z=xh$ and $h$ is a 
    handle of some valid string $xhy$ with reducing rule $T\to h$.
\end{lemma}

\begin{theorem}[DK-test; Sipser p. 143]
    Let $DK$ be the DFA corresponding to the NFA $K$ from (\ref{klemma}). Then $G$ is a 
    DCFG if and only if every one of $DK$'s accept states contains
    \begin{enumerate}
        \item exactly one completed rule, and 
        \item no dotted rule of the form $B\to u.av$ for $a\in\Sigma$.
    \end{enumerate}
\end{theorem}

\begin{theorem}
    An endmarked language is generated by a DCFG if and only if it is deterministic context-free.
\end{theorem}

\begin{definition}[LR(k) grammar; Sipser p. 152]
    A handle $h$ of a valid string $v=xhy$ with respect to CFG $G$ is \emph{forced by lookahead k} 
    if $h$ is the unique handle in every valid string $xh\hat y$ where $\hat y\in\Sigma^*$ and 
    $y$ and $\hat y$ do not differ in their first $k$ symbols.

    An $LR(k)$ grammar is a CFG grammar such that the handle of every valid string is forced by 
    lookahead $k$.
\end{definition}

\begin{theorem}[Sipser p. 154]
    An endmarked language is generated by an $LR(1)$ grammar if and only if it is a DCFL. 
\end{theorem}

\end{document}