\documentclass{article}
\usepackage{notes-preamble}
\usepackage{enumitem}
\mkthmstwounified

\title{Introduction to Theoretical Computer Science (SEM6)}
\author{Franz Miltz}
\begin{document}
\maketitle
\noindent Textbook: M. Sipser, \emph{Introduction to the Theory of Computation}
\tableofcontents
\pagebreak

\section{Regular Languages}

\subsection{Finite automata}

\begin{definition}[DFA; Sipser, p. 35]
    A \emph{deterministic finite automaton} is a $5$-tuple $(Q,\Sigma,\delta,q_0,F)$ where 
    \begin{enumerate}
        \item $Q$ is a finite set called the \emph{states},
        \item $\Sigma$ is a finite set called the \emph{alphabet},
        \item $\delta:Q\times\Sigma\to Q$ is the \emph{transition function},
        \item $q_0\in Q$ is the \emph{start state}, and 
        \item $F\subseteq Q$ is the \emph{set of accept states}.
    \end{enumerate} 
\end{definition}

\begin{definition}
    A \emph{language} $A$ over an alphabet $\Sigma$ is a set such that for all 
    $w\in A$ there exist $w_1,w_2,...,w_n\in\Sigma$ such that $w=w_1w_2\cdots w_n$.
    
    A finite automaton $M=(Q,\Sigma,\delta,q_0,F)$ accepts a word $w=w_1\cdots w_n\in A$ 
    if there exists a sequence of states $r_0,...,r_n\in Q$ such that 
    \begin{enumerate}
        \item $r_0=q_0$,
        \item $\delta(r_i,w_{i+1})=r_{i+1}$, for $i<n$, and 
        \item $r_n\in F$.
    \end{enumerate}
    We say $M$ recognises $A$ if and only if $A=\{w : M\text{ accepts }w\}$.
\end{definition}

\begin{definition}[Sipser p. 40]
    A language is called a \emph{regular language} if some finite automaton recognises it.
\end{definition}

\begin{definition}[Sipser p. 44]
    Let $A$ and $B$ be languages. We define the following \emph{regular operations}:
    \begin{enumerate}
        \item \emph{union}: $A\cup B=\{x : x \in A \text{ or } x \in B\}$.
        \item \emph{concatenation}: $A\circ B=\{xy : x \in A \text{ or } y \in B\}$.
        \item \emph{star}: $A^* = \{x_1x_2\dots x_k:k\geq 0 \text{ and each }x_i\in A\}$.
    \end{enumerate}
\end{definition}

\begin{theorem}[Sipser p. 45, 60, 62]
    The class of regular languages is closed under regular operations.
\end{theorem}

\begin{theorem}[Pumping lemma; Sipser p. 78]
    If $A$ is a regular language, then there is a number $p$ where if $s$
    is any string in $A$ of length at least $p$, then $s$ may be divided 
    into three pieces, $s=xyz$, satisfying the following conditions:
    \begin{enumerate}
        \item for each $i\geq 0$, $xy^iz\in A$,
        \item $\abs y > 0$, and 
        \item $\abs{xy} \leq p$.
    \end{enumerate} 
\end{theorem}

\subsection{Nondeterminism}

\begin{definition}[NFA; Sipser p. 53]
    A \emph{nondeterministic finite automaton} is a $5$-tuple $(Q,\Sigma,\delta,q_0,F)$ where 
    \begin{enumerate}
        \item $Q$ is a finite set called the \emph{states},
        \item $\Sigma$ is a finite set called the \emph{alphabet},
        \item $\delta:Q\times\Sigma\to\mathcal{P}(Q)$ is the \emph{transition function},
        \item $q_0\in Q$ is the \emph{start state}, and 
        \item $F\subseteq Q$ is the \emph{set of accept states}.
    \end{enumerate} 
\end{definition}

\begin{definition}[$\e$-NFA; Sipser p. 53]
    An \emph{$\e$-NFA} is a $5$-tuple $(Q,\Sigma,\delta,q_0,F)$ where 
    \begin{enumerate}
        \item $Q$ is a finite set called the \emph{states},
        \item $\Sigma$ is a finite set called the \emph{alphabet},
        \item $\delta:Q\times\Sigma_\e\to \mathcal{P}(Q)$ is the \emph{transition function},
        \item $q_0\in Q$ is the \emph{start state}, and 
        \item $F\subseteq Q$ is the \emph{set of accept states}.
    \end{enumerate} 
\end{definition}

\begin{theorem}[Sipser p. 55]
    Let $L$ be a language. Then the following statements are equivalent:
    \begin{enumerate}
        \item $L$ is regular.
        \item There exists a DFA $M$ that recognises $L$.
        \item There exists an NFA $M$ that recognises $L$.
        \item There exists an $\e$-NFA $M$ that recognises $L$.
    \end{enumerate}
\end{theorem}

\subsection{Regular expressions}

\begin{definition}[Sipser p. 64]
    $R$ is a \emph{regular expression} over an alphabet $\Sigma$ if one of the following holds:
    \begin{enumerate}
        \item $R\in\Sigma$,
        \item $R=\e$,
        \item $R=\emptyset$,
        \item $R=(R_1\cup R_2)$, where $R_1,R_2$ are regular expressions,
        \item $R=(R_1\circ R_2)$, where $R_1,R_2$ are regular expressions, or
        \item $R=(R_1^*)$, where $R_1$ is a regular expression.
    \end{enumerate} 
\end{definition}

\begin{theorem}[Sipser p. 66]
    A language is regular if and only if some regular expression describes it. 
\end{theorem}

\begin{definition}[GNFA; Sipser p. 53]
    A \emph{generalised NFA} is a $5$-tuple $(Q,\Sigma,\delta,q_0,q_1)$ where 
    \begin{enumerate}
        \item $Q$ is a finite set called the \emph{states},
        \item $\Sigma$ is a finite set called the \emph{alphabet},
        \item $\delta:(Q\setminus\{q_0\})\times(Q\setminus\{q_1\})\to R$ is the \emph{transition function},
        \item $q_0\in Q$ is the \emph{start state}, and 
        \item $q_1\in Q$ is the \emph{accept state},
    \end{enumerate} 
    where $R$ is the collection of all regular expressions over the alphabet $\Sigma$.
\end{definition}


\end{document}