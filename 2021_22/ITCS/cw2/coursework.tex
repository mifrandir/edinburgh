\documentclass{article}
\usepackage{homework-preamble}
\usepackage{enumitem}
\mkthms

\title{Introduction to Theoretical Computer Science: Coursework 2}
\author{}
\begin{document}
\maketitle
\noindent The encoding of the register machine program $P$ as described in the lectures shall be denoted $\enc P$.
We shall also say that an RM $M$ halts with value $y$ on input $x$ if, given $x$ in $R_0$, $M$ halts 
with $y$ in $R_0$.
\renewcommand{\N}{\mathbb{Z}_{\geq 0}}
\section{Partial functions}
\begin{claim*}[a, 1]
    Let $\hat H:\N\times\N\rightharpoonup\N$ be the partial function given by 
    \begin{align*}
        \hat H(m,n)=\begin{cases}
            0 &\text{if $m$ is not the code of any RM program $P$;}\\
            1 &\text{if $m=\enc P$ for some $P$, and $P$ halts on input $n$;}\\
            \bot & \text{otherwise}.
        \end{cases}
    \end{align*}
    Then $\hat H$ is computable.
    \begin{proof}
        As mentioned in (b), it is easily decidable whether a number $n$ is the code $\enc P$ of a 
        machine. Further, as used in the proof of \emph{Theorem I.26} of the notes, it is possible 
        to use the encoding $\enc M$ of an RM $M$ to simulate a single step of the computation and decide whether 
        $M$ has halted. Combining these, we construct an RM using the following algorithm to compute 
        $\hat H(m,n)$:
        \begin{enumerate}
            \item Decide wether $m$ is an encoding of some $P$. If not, return $0$.
            \item Encode $\enc M=\lra{m, \lra{n, \lra{}}, 0}$ and store the result in $R_0$.
            \item While the encoding in $R_0$ is not in a halting state, simulate another step and store the result in $R_0$.
            \item Return $1$.
        \end{enumerate}
        Observe that, if $P$ halts on input $n$, the algorithm above will indeed return $1$. Trivially,
        if there is no $P$ with encoding $m$, then the algorithm returns $0$.
        Appealing to the \emph{Church-Turing thesis} and the computabilitiy of the functions involved, 
        we conclude that there exists a machine that computes the partial function $\hat H$.
    \end{proof}
\end{claim*}

\begin{claim*}[a, 2]
    Let $f:\N\rightharpoonup\N$ be a computable partial function. Then it is undecidable whether $f$ 
    is total. 
    \begin{proof}
        Let $\enc M$ be a register machine state encoding.  Note that the partial function $H_M:\N\to\N$
        given by $\hat H_M(n)=H(\enc M, n)$ is computable.

        Assume that there exists an RM $I$ that decides whether, given a suitable encoding, a partial
        function $f:\N\to\N$ is total. Then we may construct an RM $H$ which takes as input an encoding 
        of a register machine state $\enc M$, encodes the function $\hat H_M$, and runs $I$ on the result.
        Clearly, $H$ halts in all cases and returns `1' iff $M$ halts on all inputs. In other words,
        $UH$ decides the uniform halting problem; contradiction.
    \end{proof}
\end{claim*}

\begin{claim*}[b]
    The partial function $d:\N\rightharpoonup\N$ given by 
    \begin{align*}
        d(n)=\begin{cases}
            P_n(n)+1 &\text{if $P_n$ returns a result on input $n$;}\\
            \bot &\text{otherwise,}
        \end{cases}
    \end{align*} 
    where $(P_k)_{k\in\N}$ is the sequence of all register machines, is 
    computable.
    \begin{proof}
        We construct the following machine $M$:
        \begin{enumerate}
            \item Given input $n$, compute $\hat H(P_n,n)$.
            \item Simulate $P_n$ until termination and store the result in $R_0$.
            \item Increment $R_0$ and halt.
        \end{enumerate}
        Clearly, all steps of this computation are possible. In particular, $M$ doesn't halt 
        if $\hat H(P_n,n)=\bot$ which implies $d(n)=\bot$. This is consistent. Thus $M$ 
        computes the partial function $d$.
        
        \emph{It is worth mentioning that an implicit consequence of the above is $d(k)=\bot$
        whenever $P_k$ computes $d$.}
    \end{proof}
\end{claim*}

\begin{claim*}[c]
    Let $f:\N\to\N$ be a total function such that $f(n)=d(n)$ for all $n\in\N$ where 
    $d(n)\not=\bot$. Then $f$ is not computable.
    \begin{proof}
        Assume there exists an RM $M$ that computes $f$. Then there exists $k\in\N$ such that 
        $P_k$ computes $f$. Since $f$ is total, $P_k$ returns $f(k)$ on input $k$. However,
        then $d(k)=f(k)+1$. This contradicts the premise that $f(k)=d(k)$ whenever $P_k$ terminates 
        on input $k$.
    \end{proof}
\end{claim*}

\section{P, NP}

\begin{claim*}[a]
    \begin{align}
        \label{on2}
        n^2&\not=O(n\lg n)\\
        \label{onlgn}
        n\lg n &=O(n^2)\\
        \label{o3n}
        3^n&\not=O(2^n)
    \end{align}
    The statement $3^n=2^{O(n)}$ is nonsensical and may therefore be true or false, depending on 
    definition.
\end{claim*}

\begin{claim*}[b]
    Let $(D,X)$ and $(D,Y)$ be decision problems in \ptime. Then the decision problems
    \begin{align*}
        (D,X\cup Y), (D,X\cap Y), (D,D\setminus X)
    \end{align*}
    are in \ptime.
    \begin{proof}
        Let $M_X,M_Y$ be the RMs that decide $(D,X),(D,Y)$ in time $T_X(n)=O(n^c),T_Y(n)=O(n^d)$ for some $c,d$, 
        respectively. Then we may construct $U$ that decides $(D,X\cup Y)$ as follows:
        \begin{enumerate}[label=U\arabic*]
            \item Run $M_X$ on input $d$ and store the output in $R_1$.
            \item Run $M_Y$ on input $d$ and store the output in $R_2$.
            \item If $R_1$ or $R_2$ contains a non-zero value then return $1$, else return $0$.
        \end{enumerate}
        Note that the runtime is $T_U(n)=T_X(n)+T_Y(n)+\Theta(1)$ since there are only four possible scenarios 
        for U3. Thus we obtain the very generous bound $T_U(n)=O(n^{c+d})$, showing that $X\cup Y$ is in \ptime.

        Similarly, we construct the RM $I$ that decides $X\cap Y$:
        \begin{enumerate}[label=I\arabic*]
            \item Run $M_X$ on input $d$ and store the output in $R_1$.
            \item Run $M_Y$ on input $d$ and store the output in $R_2$.
            \item If $R_1$ or $R_2$ contain zero then return $0$, else return $1$.
        \end{enumerate}
        Completely analogously to $U$, we obtain the runtime $T_I(n)=O(n^{c+d})$, again proving that 
        $X\cap Y$ is in \ptime.

        Even simpler is the construction of $N$ that decides $(D, D\setminus X)$, i.e. `$\neg X$':
        \begin{enumerate}[label=N\arabic*]
            \item Run $M_X$ on input $d$ and store the output in $R_0$.
            \item If $R_0$ is $0$ then return $1$, otherwise return $0$.
        \end{enumerate}
        The runtime we obtain is $T_N(n)=O(n^c)$. Thus $\neg X$ is in \ptime.
    \end{proof}
\end{claim*}

\begin{claim*}[c]
    Let $L_1,L_2$ be languages of strings over some finite alphabet $\Sigma$ whose decision problems are in 
    \nptime. Then the decision problem of $L=L_1L_2$ is in \nptime.
    \begin{proof}
        Let $M_1,M_2$ be the NRMs bounded by the polynomials $f_1(n),f_2(n)$ that decide $L_1,L_2$, respectively.
        We construct an NRM $M$ that decides $L$ as follows:
        \begin{enumerate}[label=M\arabic*]
            \item \label{split} Non-deterministically split the input into two parts. This could be done in the following way:
                \begin{enumerate}
                    \item Write the empty sequence into $R_1$.
                    \item \label{emptycheck} If $R_0$ is the empty sequence, go to \ref{deterministic}.
                    \item Maybe go to \ref{deterministic}.
                    \item Take one character from the end of the sequence in $R_0$ and append it to $R_1$.
                    \item Go to \ref{emptycheck}.
                \end{enumerate}
            \item \label{deterministic} Reverse $R_1$.
            \item \label{runm1} Run $M_1$ on $R_0$.
            \item \label{runm2} Run $M_2$ on $R_1$.
            \item \label{trivial} If both $R_0$ and $R_1$ returned `1', accept the string.
        \end{enumerate}
        Firstly, we shall show that $M$ behaves as intended. If there exist words $x_1\in L_1$ and $x_2\in L_2$ such that the input 
        is the concatenation $w=x_1x_2$ then one of the runs will result in the correct 
        split and $M_1$ and $M_2$ will accept each string individually.  Conversely, 
        if $M$ accepts a word $w$ it means that there exists a split such that 
        $w=x_1x_2$ and $x_1\in L_1$ and $x_2\in L_2$, i.e. $w\in L_1L_2$. Thus $M$ 
        decides $L$.

        Secondly, we show that $M$ is polynomially bounded. Consider \ref{split}. Since $\Sigma$ is finite, we may 
        represent a single character using 
        $d$ bits. Thus a word $w=w_1\cdots w_k \in\Sigma^*$ may be encoded using $n=kd$ bits in total.
        Multiplication and division (by constants $p$ and $q$) are polynomial time operations, so we know
        they must be bounded by some polynomials $g_1(n),g_2(n)$ respectively. In the worst case, $M$ may move
        every character out of $R_0$ into $R_1$ individually, which takes $k$ steps. Each one of these steps 
        may be solved by dividing by $2^d$, multiplying by $2^d$ twice (once $R_1$ for appending and once a
        temporary register for the modulo operation) and then a constant number of additions and multiplications.
        Thus we end up with a runtime bound for each step of $2g_1(n)+g_2(n)+O(n)$ so clearly all of \ref{split}
        is polynomially bounded.

        The runtime of \ref{deterministic} is similar to the worst case of \ref{split}. One can reverse a list 
        by removing them and appending them to a temporary register. This has been shown above to be possible in 
        polynomial time.

        Finally, it is clear that \ref{runm1} and \ref{runm2} are bounded by $f_1(n)$, $f_2(n)$ up to some 
        constant overhead. Similarly, \ref{trivial} only takes a constant number of instructions, thus it is polynomially bounded, too. 
        We conclude that $M$ as a whole is polynomially bounded. Since $M$ decides $L$, the decision problem 
        $(\Sigma^*, L)$ is in \nptime.
    \end{proof}
\end{claim*}

\end{document}