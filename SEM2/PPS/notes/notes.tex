\documentclass{article}
\usepackage[a4paper]{geometry}
\geometry{tmargin=3cm, bmargin=3cm, lmargin=2cm, rmargin=2cm}
\usepackage{babel}
\usepackage{amsmath}
\usepackage{amssymb}
\usepackage{amsthm}
\usepackage{nicefrac}
\usepackage{siunitx}
\usepackage{mathtools}
\newtheoremstyle{sltheorem} {}                % Space above
{}                % Space below
{\upshape}        % Theorem body font % (default is "\upshape")
{}                % Indent amount
{\bfseries}       % Theorem head font % (default is \mdseries)
{.}               % Punctuation after theorem head % default: no punctuation
{ }               % Space after theorem head
{}                % Theorem head spec
\theoremstyle{sltheorem}
\newtheorem{definition}{Definition}[section]
\newtheorem{theorem}{Theorem}[section]
\newcommand{\R}{\mathbb{R}}
\newcommand{\N}{\mathbb{N}}
\newcommand{\st}{\text{ s.t }}
\DeclareMathOperator{\lub}{LUB}
\DeclareMathOperator{\glb}{GLB}
\newcommand*\lneg[1]{\overline{#1}}
\begin{document}
\section{Logic and the reals}
\begin{theorem}
    If $A$ and $B$ are statements, then $A\Rightarrow B\equiv \lneg{A}\vee B$.
\end{theorem}
\section{The reals}
\begin{definition}
    A \textbf{field} is a set $F$ along with operations $+$ and $\cdot$ so that the following hold:\\
    Rules of Addition:
    \begin{enumerate}
        \item $a+b\in F$.
        \item $a+b = b+a$.
        \item $a+(b+c) = (a+b)+c$.
        \item $\exists 0\in F.\: \forall a\in F.\: 0+a=a$.
        \item $\forall a\in F.\: \exists -a\in F.\: a+(-a)=0$.
    \end{enumerate}
    Rules of Mulitplication:
    \begin{enumerate}
        \item $a\cdot b \in F$.
        \item $a\cdot b = b\cdot a$.
        \item $a\cdot(b\cdot c)=(a\cdot b)\cdot c$.
        \item $\exists 1\in F.\: 1\cdot a = a$.
        \item $a\not=0\Rightarrow\exists \nicefrac{1}{a}\in F.\: a\cdot\nicefrac{1}{a}=1$.
        \item $a\cdot (b+c) = a\cdot b + a\cdot c$.
    \end{enumerate}
\end{definition}
\begin{theorem}
    The rules of addition above imply that, for $x,y,z\in \R$,
    \begin{enumerate}
        \item $x+y=x+z\Leftrightarrow y=z$.
        \item $x+y=x\Rightarrow y=0$.
        \item $x+y=0\Rightarrow y=-x$.
        \item $-(-x)=x$.
    \end{enumerate}
\end{theorem}
\begin{theorem}
    The rules of multiplication above imply that, for $x, y, z \in \R$,
    \begin{enumerate}
        \item $xy=xz\Leftrightarrow y=z$.
        \item $xy=x\Rightarrow y=1$.
        \item $xy=1\Rightarrow y=\nicefrac{1}{x}$.
        \item $\nicefrac{1}{\nicefrac{1}{x}}=x$.
    \end{enumerate}
\end{theorem}
\begin{theorem}
    For $x,y\in\R$,
    \begin{enumerate}
        \item $0\cdot x = 0$.
        \item $x\not=0\not=y\Rightarrow xy\not=0$.
        \item $(-x)y=-(xy)=x(-y)$.
        \item $(-x)(-y)=xy$.
    \end{enumerate}
\end{theorem}
\begin{theorem}
    Given $x>0$ and $n\in\N$, there is a unique $y>0$ so that $y^n=x$, and we write this nuber $y$ as $y^{\nicefrac{1}{x}}$.
\end{theorem}
\begin{theorem}
    Given any two numbers $a<b$, there is $r\in\mathbb{Q}$ with $a<r<b$.
\end{theorem}
\begin{theorem}
    The number $\sqrt{2}$ is irrational.
\end{theorem}
\begin{theorem}
    Let $a$ be rational and $b$ be irrational.
    \begin{enumerate}
        \item $a+b$ is irrational.
        \item If $a\not=0$, then $ab$ is irrational.
    \end{enumerate}
\end{theorem}
\begin{theorem}
    Given $x>0$ and $n\in\N$, there is a unique $y>0$ so that $y^n=x$, and we write this number as $y=x^{\nicefrac{1}{n}}$.
\end{theorem}
\begin{theorem}
    Let $x>0$, $y>0$ and $p,q\in\mathbb{Q}$. Then
    \begin{enumerate}
        \item $x^px^q=x^{p+q}$.
        \item $(x^p)^q=x^{pq}$.
        \item $(xy)^p=x^py^p$.
    \end{enumerate}
\end{theorem}
\section{Inequalities}
\begin{definition}
    Given $x, y \in \R$, we may write $x<y$, which we prononounce "$x$ is less than $y$". The symbol $<$ satisfies the following axioms:
    \begin{enumerate}
        \item If $x\in R$ then exactly one of the following is true: $x>0$, $x=0$ or $x<0$.
        \item If $x>y$, then $-x<-y$.
        \item If $x>y$ and $c\in\R$, then $x+c > y+c$.
        \item If $x > 0$ and $y > 0$, then $xy > 0$.
        \item If $x>y$ and $y>z$, then $x>z$.
    \end{enumerate}
\end{definition}
\begin{theorem}
    If $x>0$, then $-x<0$.
\end{theorem}
\begin{theorem}
    If $x\not=0$, then $x^2 > 0$.
\end{theorem}
\begin{theorem}
    If $x>0$, then $\nicefrac{1}{x}>0$.
\end{theorem}
\begin{theorem}
    If $x>0$, then $u>v$ iff $xu>xv$.
\end{theorem}
\begin{theorem}
    If $u,v>0$, then $u^2>v^2$ iff $u>v$.
\end{theorem}
\begin{theorem}
    For $u,v\in\R$,
    \begin{align*}
        uv\leq \frac{u^2+v^2}{2}.
    \end{align*}
\end{theorem}
\begin{theorem}
    \textbf{AM-GM Inequality}\\
    Let $n\in\N$ and $x_1, x_2, ..., x_n \geq 0$. Then 
    \begin{align*}
        (x_1\cdot x_2\cdots x_n)^{\frac{1}{n}}\leq\frac{x_1+\cdots+x_n}{n}.
    \end{align*}
\end{theorem}
\begin{theorem}
    \textbf{Cauchy-Schwartz Inequality}\\
    Let $n\in\N$ and $x_1,...,x_n,y_1,...,y_n\in\R$. Then
    \begin{align*}
        x_1y_1+\cdots x_ny_n \leq \sqrt{x_1^2+\cdots+x_n^2}\sqrt{y_1^2+\cdots+y_n^2}.
    \end{align*}
\end{theorem}
\begin{definition}
    For $x\in\R$, we define the \textbf{absolute value} of $x$ to be
    \begin{align*}
        \vert x \vert = \begin{cases}
            x &\text{if } x \geq 0\\
            -x &\text{if } x < 0
        \end{cases}
    \end{align*}
\end{definition}
\begin{theorem}
    For $x\in\R$, $-|x|\leq x \leq |x|$.
\end{theorem}
\begin{theorem}
    For $x,y\in\R$, $|xy|=|x|\cdot |y|$.
\end{theorem}
\begin{theorem}
    For $x,y\in\R$, $|x+y| \leq |x| + |y|$.
\end{theorem}
\begin{theorem}
    For $x,y\in\R$, $|x|-|y|\leq |x+y|$.
\end{theorem}
\section{Induction}
\begin{definition}
    \textbf{The Principle of Mathematical Induction}\\
    Given a list of statements $P(k)$, $P(k+1)$, ..., we know that $P(n)$ is true for every integer $n\geq k$ if
    \begin{enumerate}
        \item we know that $P(k)$ is true,
        \item and we can prove that $P(n)\Rightarrow P(n+1)$ for any integer $n\geq k$.
    \end{enumerate}
\end{definition}
\begin{definition}
    \textbf{The Principle of Strong Mathematical Induction}\\
    Let $k\in\N$ and let $P(k)$, $P(k+1)$, ..., be statements. Suppose that
    \begin{enumerate}
        \item $P(k)$ is true,
        \item for any integer $n\geq k$, $P(k), P(k+1), P(k+2), ..., P(n)\Rightarrow P(n+1)$.
    \end{enumerate}
    Then $P(n)$ is true for all integer $n\geq k$.
\end{definition}
\begin{theorem}
    Every integer $n\geq 2$ can be written as a product of primes $n=p_1\cdots p_k$.
\end{theorem}
\begin{theorem}
    Suppose that $x,y>0$ and $n\in\N$.
    \begin{enumerate}
        \item $x^n>y^n\Leftrightarrow x>y$.
        \item $x^{\nicefrac{1}{n}}>y^{\nicefrac{1}{n}}\Leftrightarrow x>y$.
    \end{enumerate}
\end{theorem}
\section{(Least) Upper Bounds}
\begin{definition}
    Let $A\subseteq\R$.
    \begin{itemize}
        \item $A$ is said to be \textbf{bounded above} if $\exists M\in\R\st\forall x\in A,x\leq M$.\\
        The number $M$ is called an \textbf{upper bound} of $A$.
        \item Similarly, $A$ is said to be \textbf{bounded below} if $\exists m\in\R\st\forall x\in A,x\geq m$.\\
        The number $m$ is called a \textbf{lower bound} of $A$.
        \item We say $A$ is \textbf{bounded} if it is bounded above and below.
    \end{itemize}
\end{definition}
\begin{definition}
    Given a set $A\subseteq\R$, a number $L$ is a \textbf{least upper bound} (LUB) for $A$ if
    \begin{enumerate}
        \item $L$ is an upper bound for $A$, and
        \item $\forall t<L$, $t$ is not an upper bound..
    \end{enumerate}
    Similarly, we say $L$ is a \textbf{greatest lower bound} (GLB) for $A$ if
    \begin{enumerate}
        \item $L$ is a lower bound for $A$, and
        \item $\forall t>L$, $t$ is not a lower bound.
    \end{enumerate}
    If the LUB exists for a set $A$, we denote it by $\lub(A)$. Similarly, we denote the GLB of $A$ by $\glb(A)$.
\end{definition}
\section{Limits}
\begin{definition}
    \textbf{The $\varepsilon-N$ definition of a limit.} Let $(x_n)^\infty_{n=1}$ be a sequence of real numbers $x_1,x_2,...$ and let $L\in\R$. We say that $x_n$ \textbf{converges} to $L$ and write $\lim_{n\to\infty}x_n=L$ if for all $\varepsilon>0$, there is a number $N$ so that for $n>N$, we have
    \begin{align*}
        |x_n-L|<\varepsilon
    \end{align*}
    or equivalently, that
    \begin{align*}
        L-\varepsilon<x_n<L+\varepsilon.
    \end{align*}
\end{definition}
\begin{definition}
    We say a sequence $(a_n)$ is \textbf{bounded} if the set of values $\{a_1, a_2, ...\}$ is a bounded set.
\end{definition}
\begin{theorem}
    If $(a_n)$ converges, it is bounded.
\end{theorem}
\begin{theorem}
    Let $a_n$ and $b_n$ converge to $a$ and $b$ respectively. Then
    \begin{enumerate}
        \item $a_n+b_n\to a+b$
        \item $a_nb_n\to ab$
        \item If $c\in\R$, then $ca_n\to ca$.
        \item If $b\not=0$ and $b_n\not=0$ for all $n$, then $\frac{a_n}{b_n}\to\frac{a}{b}$.
    \end{enumerate}
\end{theorem}
\begin{theorem}
    For $k\geq2$ an integer, if $x_n\to L$, then $x^k_n\to L^k$.
\end{theorem}
\begin{theorem}
    Let $(x_n)$ and $(y_n)$ be sequences.
    \begin{enumerate}
        \item If $x_n\to x$, $y_n\to y$, and $x_n\leq y_n$ for all $n\in\N$, then $x\leq y$.
        \item If $x_n\to x$ and $x_n\leq y$ for all $n$, then $x\leq y$.
    \end{enumerate}
\end{theorem}
\begin{theorem}
    Let $x>0$ and $k\in\N$. There is $y>0$ so that $y^k=x$.
\end{theorem}
\begin{definition}
    Let $(x_n)$ be a sequence of real numbers.
    \begin{enumerate}
        \item We say $x_n\to \infty$ if, for all $M>0$, there is an $N$ so that $n\geq N$ implies $x_n\geq M$.
        \item Similarly, we say $x_n\to-\infty$ if, for all $M<0$, there is an $N$ so that $n\geq N$ implies $x_n\leq M$.
    \end{enumerate}
\end{definition}
\section{Convergence Theorem}
\begin{definition}
    A sequence $(a_n)^\infty_{n=1}$ is \textbf{increasing} if $a_n\leq a_n+1$ for all $n$ and \textbf{decreasing} if $a_n\geq a_{n+1}$ for all $n$. We say $a_n$ is \textbf{monotone} if it is either increasing or decreasing.
\end{definition}
\begin{theorem}
    \textbf{The Monotone Convergence Theorem}\\
    Let $(a_n)$ be an increasing sequence of real numbers that is bounded above (i.e. there is $M$ so that $a_n\leq M$ for all $n$). Then $(a_n)$ converges. Then $(a_n)$ converges. If $(a_n)$ is decreasing and is bounded below, then $(a_n)$ converges.
\end{theorem}
\begin{definition}
    Given a sequence $(x_n)$, a \textbf{subsequence} is a sequence of the form $(x_{n_k})$ where $n_1<n_2<\cdots$ are positive integers.
    A real number $x$ is a \textbf{limit point} of a sequence $(x_n)$ if there is a subsequence $(x_{n_k})$ so that $x_{n_k}\to x$.
\end{definition}
\begin{theorem}
    Given a sequence $(x_n)$ and a number $L$, $L$ is a limit point iff for all $\varepsilon > 0$ and for all integers $N$, we can find $n>N$ such that $|x_n-L|<\varepsilon$.
\end{theorem}
\begin{theorem}
    \textbf{Bolzano Weierstrass Theorem}. If $(x_n)$ is a bounded sequence then it has a limit point, that is, it has a convergent subsequence.
\end{theorem}
\section{Series and Decimals}
\begin{definition}
    Given a sequence $(a_n)$, we say that the series $\sum a_n$ converges if the sequence of partial sums
    \begin{align*}
        s_N=\sum_{n=1}^N a_n
    \end{align*}
    converges as $N\to\infty$. We denote its limit by $\sum_{n=1}^\infty a_n$.
\end{definition}
\begin{theorem}
    Let $a_n$ and $b_n$ be sequences. If $\sum a_n$ and $\sum b_n$ are convergent, then so is $\sum (a_n+b_n)$, and
    \begin{align*}
        \sum_{n=1}^\infty (a_n+b_n) = \sum_{n=1}^\infty a_n+ \sum_{n=1}^\infty b_n.
    \end{align*}
    If $c\in\R$, then $\sum ca_n$ is convergent and
    \begin{align*}
        \sum_{n=1}^\infty ca_n = c\sum_{n=1}^\infty a_n.
    \end{align*}
\end{theorem}
\begin{theorem}
    \textbf{Comparison Test.} If $\sum b_n$ is convergent and $|a_n|\leq b_n$ for all $n$, then $\sum a_n$ is convergent.
\end{theorem}
\begin{theorem}
    Let $x\in\R$. Then there is a sequence of integers $a_0\in\mathbb{Z}$ and $a_n\in\{0,1,...,9\}$ for $n\geq 1$ so that
    \begin{align*}
        \sum_{n=0}^\infty \frac{a_n}{10_n}=a_0+\frac{a_1}{10}+\cdots = x.
    \end{align*}
\end{theorem}
\begin{theorem}
    If $x\geq 0$ is rational, then it has a periodic decimal expansion, that is,
    \begin{align*}
        x=a_0.a_1a_2\cdots a_k\lneg{b_1b_2\cdots b_l}.
    \end{align*}
\end{theorem}
\begin{definition}
    \textbf{Pigeonhole Principle}.\\
    Suppose $k>n$, and we have $a_1,...,a_k\in S$, with $|S|=n$. Then there exists $i\not=j$ with $a_i=a_j$.
\end{definition}
\begin{theorem}
    The period of the decimal expansion of $\frac{p}{q}$ is of length at most $q$.
\end{theorem}
\begin{theorem}
    If $x\geq 0$ has periodic decimal expansion, then $x$ is rational.
\end{theorem}
\begin{theorem}
    A number $x$ is irrational iff its decimal expansion is aperiodic.
\end{theorem}
\begin{theorem}
    Let $x\geq 0$ and suppose $x=a_0.a_1a_2...=0.b_1b_2....$ Let $l$ be the first integer for which $a_l\not=b_l$. Then $b_k=0$ and $a_k=9$ for a $k>l$ and $b_l=a_l+1$.
\end{theorem}
\begin{theorem}
    A number $x$ is irrational if, for all $c>0$, there is a rational number $\frac{p}{q}\not=x$ so that
    \begin{align*}
        \left|x-\frac{p}{q}\right|<\frac{c}{q}.
    \end{align*}
\end{theorem}
\begin{theorem}
    Suppose $x=\lim_{n\to\infty}\frac{p_n}{q_n}$ and $\left|x-\frac{p_n}{q_n}\right|q_n \to 0$, then $x$ is irrational.
\end{theorem}
\begin{theorem}
    The number $e$ is irrational.
\end{theorem}
\end{document}