\documentclass{article}
\usepackage[a4paper]{geometry}
\geometry{tmargin=3cm, bmargin=3cm, lmargin=2cm, rmargin=2cm}
\usepackage{babel}
\usepackage{amsmath}
\usepackage{amssymb}
\usepackage{amsthm}
\usepackage{nicefrac}
\usepackage{siunitx}
\newtheoremstyle{sltheorem} {}                % Space above
{}                % Space below
{\upshape}        % Theorem body font % (default is "\upshape")
{}                % Indent amount
{\bfseries}       % Theorem head font % (default is \mdseries)
{.}               % Punctuation after theorem head % default: no punctuation
{ }               % Space after theorem head
{}                % Theorem head spec
\theoremstyle{sltheorem}
\newtheorem{definition}{Definition}[section]
\newtheorem{theorem}{Theorem}[section]
\newcommand{\R}{\mathbb{R}}
\newcommand{\N}{\mathbb{N}}
\begin{document}
\section{Logic and the reals}
\begin{theorem}
    If $A$ and $B$ are statements, then $A\Rightarrow B\equiv \bar{A}\vee B$.
\end{theorem}
\section{The reals}
\begin{definition}
    A \textbf{field} is a set $F$ along with operations $+$ and $\cdot$ so that the following hold:\\
    Rules of Addition:
    \begin{enumerate}
        \item $a+b\in F$.
        \item $a+b = b+a$.
        \item $a+(b+c) = (a+b)+c$.
        \item $\exists 0\in F.\: \forall a\in F.\: 0+a=a$.
        \item $\forall a\in F.\: \exists -a\in F.\: a+(-a)=0$.
    \end{enumerate}
    Rules of Mulitplication:
    \begin{enumerate}
        \item $a\cdot b \in F$.
        \item $a\cdot b = b\cdot a$.
        \item $a\cdot(b\cdot c)=(a\cdot b)\cdot c$.
        \item $\exists 1\in F.\: 1\cdot a = a$.
        \item $a\not=0\Rightarrow\exists \nicefrac{1}{a}\in F.\: a\cdot\nicefrac{1}{a}=1$.
        \item $a\cdot (b+c) = a\cdot b + a\cdot c$.
    \end{enumerate}
\end{definition}
\begin{theorem}
    The rules of addition above imply that, for $x,y,z\in \R$,
    \begin{enumerate}
        \item $x+y=x+z\Leftrightarrow y=z$.
        \item $x+y=x\Rightarrow y=0$.
        \item $x+y=0\Rightarrow y=-x$.
        \item $-(-x)=x$.
    \end{enumerate}
\end{theorem}
\begin{theorem}
    The rules of multiplication above imply that, for $x, y, z \in \R$,
    \begin{enumerate}
        \item $xy=xz\Leftrightarrow y=z$.
        \item $xy=x\Rightarrow y=1$.
        \item $xy=1\Rightarrow y=\nicefrac{1}{x}$.
        \item $\nicefrac{1}{\nicefrac{1}{x}}=x$.
    \end{enumerate}
\end{theorem}
\begin{theorem}
    For $x,y\in\R$,
    \begin{enumerate}
        \item $0\cdot x = 0$.
        \item $x\not=0\not=y\Rightarrow xy\not=0$.
        \item $(-x)y=-(xy)=x(-y)$.
        \item $(-x)(-y)=xy$.
    \end{enumerate}
\end{theorem}
\begin{theorem}
    Given $x>0$ and $n\in\N$, there is a unique $y>0$ so that $y^n=x$, and we write this nuber $y$ as $y^{\nicefrac{1}{x}}$.
\end{theorem}
\begin{theorem}
    Given any two numbers $a<b$, there is $r\in\mathbb{Q}$ with $a<r<b$.
\end{theorem}
\begin{theorem}
    The number $\sqrt{2}$ is irrational.
\end{theorem}
\begin{theorem}
    Let $a$ be rational and $b$ be irrational.
    \begin{enumerate}
        \item $a+b$ is irrational.
        \item If $a\not=0$, then $ab$ is irrational.
    \end{enumerate}
\end{theorem}
\begin{theorem}
    Given $x>0$ and $n\in\N$, there is a unique $y>0$ so that $y^n=x$, and we write this number as $y=x^{\nicefrac{1}{n}}$.
\end{theorem}
\begin{theorem}
    Let $x>0$, $y>0$ and $p,q\in\mathbb{Q}$. Then
    \begin{enumerate}
        \item $x^px^q=x^{p+q}$.
        \item $(x^p)^q=x^{pq}$.
        \item $(xy)^p=x^py^p$.
    \end{enumerate}
\end{theorem}
\section{Inequalities}
\begin{definition}
    Given $x, y \in \R$, we may write $x<y$, which we prononounce "$x$ is less than $y$". The symbol $<$ satisfies the following axioms:
    \begin{enumerate}
        \item If $x\in R$ then exactly one of the following is true: $x>0$, $x=0$ or $x<0$.
        \item If $x>y$, then $-x<-y$.
        \item If $x>y$ and $c\in\R$, then $x+c > y+c$.
        \item If $x > 0$ and $y > 0$, then $xy > 0$.
        \item If $x>y$ and $y>z$, then $x>z$.
    \end{enumerate}
\end{definition}
\begin{theorem}
    If $x>0$, then $-x<0$.
\end{theorem}
\begin{theorem}
    If $x\not=0$, then $x^2 > 0$.
\end{theorem}
\begin{theorem}
    If $x>0$, then $\nicefrac{1}{x}>0$.
\end{theorem}
\begin{theorem}
    If $x>0$, then $u>v$ iff $xu>xv$.
\end{theorem}
\begin{theorem}
    If $u,v>0$, then $u^2>v^2$ iff $u>v$.
\end{theorem}
\begin{theorem}
    For $u,v\in\R$,
    \begin{align*}
        uv\leq \frac{u^2+v^2}{2}.
    \end{align*}
\end{theorem}
\begin{theorem}
    \textbf{AM-GM Inequality}\\
    Let $n\in\N$ and $x_1, x_2, ..., x_n \geq 0$. Then 
    \begin{align*}
        (x_1\cdot x_2\cdots x_n)^{\frac{1}{n}}\leq\frac{x_1+\cdots+x_n}{n}.
    \end{align*}
\end{theorem}
\begin{theorem}
    \textbf{Cauchy-Schwartz Inequality}\\
    Let $n\in\N$ and $x_1,...,x_n,y_1,...,y_n\in\R$. Then
    \begin{align*}
        x_1y_1+\cdots x_ny_n \leq \sqrt{x_1^2+\cdots+x_n^2}\sqrt{y_1^2+\cdots+y_n^2}.
    \end{align*}
\end{theorem}
\begin{definition}
    For $x\in\R$, we define the \textbf{absolute value} of $x$ to be
    \begin{align*}
        \vert x \vert = \begin{cases}
            x &\text{if } x \geq 0\\
            -x &\text{if } x < 0
        \end{cases}
    \end{align*}
\end{definition}
\begin{theorem}
    For $x\in\R$, $-|x|\leq x \leq |x|$.
\end{theorem}
\begin{theorem}
    For $x,y\in\R$, $|xy|=|x|\cdot |y|$.
\end{theorem}
\begin{theorem}
    For $x,y\in\R$, $|x+y| \leq |x| + |y|$.
\end{theorem}
\begin{theorem}
    For $x,y\in\R$, $|x|-|y|\leq |x+y|$.
\end{theorem}
\section{Induction}
\begin{definition}
    \textbf{The Principle of Mathematical Induction}\\
    Given a list of statements $P(k)$, $P(k+1)$, ..., we know that $P(n)$ is true for every integer $n\geq k$ if
    \begin{enumerate}
        \item we know that $P(k)$ is true,
        \item and we can prove that $P(n)\Rightarrow P(n+1)$ for any integer $n\geq k$.
    \end{enumerate}
\end{definition}
\begin{definition}
    \textbf{The Principle of Strong Mathematical Induction}\\
    Let $k\in\N$ and let $P(k)$, $P(k+1)$, ..., be statements. Suppose that
    \begin{enumerate}
        \item $P(k)$ is true,
        \item for any integer $n\geq k$, $P(k), P(k+1), P(k+2), ..., P(n)\Rightarrow P(n+1)$.
    \end{enumerate}
    Then $P(n)$ is true for all integer $n\geq k$.
\end{definition}
\begin{theorem}
    Every integer $n\geq 2$ can be written as a product of primes $n=p_1\cdots p_k$.
\end{theorem}
\begin{theorem}
    Suppose that $x,y>0$ and $n\in\N$.
    \begin{enumerate}
        \item $x^n>y^n\Leftrightarrow x>y$.
        \item $x^{\nicefrac{1}{n}}>y^{\nicefrac{1}{n}}\Leftrightarrow x>y$.
    \end{enumerate}
\end{theorem}
\end{document}