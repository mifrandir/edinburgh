\documentclass{article}
\usepackage[a4paper]{geometry}
\usepackage{babel}
\usepackage{amsmath}
\usepackage{amssymb}
\usepackage{mathtools}
\usepackage{nicefrac}
\usepackage{amsthm}
\usepackage{changepage}
\geometry{tmargin=2cm, bmargin=3cm}
\DeclarePairedDelimiter{\floor}{\lfloor}{\rfloor}
\title{PPS: Homework 6 (Workshop 33)}
\author{Franz Miltz (UNN: S1971811)}
\newcommand*\lneg[1]{\overline{#1}}
\newcommand{\R}{\mathbb{R}}
\newcommand{\N}{\mathbb{N}}
\newcommand{\Z}{\mathbb{Z}}
\newcommand{\C}{\mathbb{C}}
\newcommand{\st}{\text{ s.t. }}
%newenvironment{claim}[1]{\noindent\emph{Claim.}\space#1}{}
\newenvironment{claimproof}[1]{\par\noindent\emph{Proof.}\space#1}{\hfill $\blacksquare$}
\newtheorem{claim}[section]{Claim}
\newtheorem{lemma}{Lemma}[section]
\DeclareMathOperator{\lub}{\text{LUB}}
\DeclareMathOperator{\hcf}{hcf}
\DeclareMathOperator{\lcm}{lcm}
\begin{document}
\maketitle
\section*{Problem 1}
\begin{claim}
  Let $a,b\in\N$ and $d=\hcf(a,b)$. Then there exist $s,t\in\N$ such that $d=sa-tb$.
\end{claim}
\begin{proof}
  We can rewrite the equation as
  \begin{align}
    \label{eq:1}
    sa = tb + d
  \end{align}
  which, by \emph{Definition 12.1} in the course notes, is equivalent to
  \begin{align*}
    sa \equiv d \mod b.
  \end{align*}
  By \emph{Liebeck, Proposition 13.6}, this has a solution $s\in\Z$ if and only if $\hcf(a,b)$ divides $d$, which is true since $d=\hcf(a,b)$.\\
  Further, if $s\leq 0$ is a solution to the congruence equation, then, by \emph{Liebeck, Proposition 13.3}, so are all $s'$ such that $s' \equiv s \mod b$.
  This means, that there are infinitely many solutions $s\in\N$.
  Picking some $s>1$ and using equation $1$, we find that $sa-d>0$ and thus $tb>0$ and therefore $t\in\N$.
\end{proof}
\section*{Problem 2}
\begin{claim}
  The equation $x^2-2x=4y^2+4y+1$ does not have any solutions such that $x,y\in\Z$.
\end{claim}
\begin{proof}
  By contradiction. Assume some $x,y\in\Z$ solve the equation
  \begin{align*}
    x^2-x=4y^2+4y+1.
  \end{align*}
  Observe that we can factorize to get
  \begin{align*}
    x(x-1)=4(y^2+y)+1.
  \end{align*}
  This lets us deduce that
  \begin{align*}
    x(x-1) \equiv 1 \mod 4.
  \end{align*}
  By \emph{Liebeck, Proposition 13.3} this allows us to cycle through the four possible values of $x\mod 4$:
  \begin{align*}
    x\equiv 0 \mod 4 &\Rightarrow x(x-1) \equiv 0\cdot 4 \equiv 0 \mod 4,\\
    x\equiv 1 \mod 4 &\Rightarrow x(x-1) \equiv 1\cdot 0 \equiv 0 \mod 4,\\
    x\equiv 2 \mod 4 &\Rightarrow x(x-1) \equiv 2\cdot 1 \equiv 2 \mod 4,\\
    x\equiv 3 \mod 4 &\Rightarrow x(x-1) \equiv 3\cdot 2 \equiv 2 \mod 4.
  \end{align*}
  Therefore there is no value $x$ such that $x(x-1)\equiv 1 \mod 4$.
  This contradicts our assumption.
  Thus there are no $x,y\in\Z$ that solve the equation.
\end{proof}
\section*{Problem 3}
\begin{claim}
  Let $x,y\in\Z$ and $x^2=y^4-77$. Then $x=\pm 2$ and $y=\pm 3$.
\end{claim}
\begin{proof}
  Firstly, observe that, because of the even powers, if $(x,y)$ is a solution, so are all $(\pm x, \pm y)$.
  This lets us set $x,y\geq 0$ without loss of generality.\\
  We can factorize as follows:
  \begin{align*}
    x^2&=y^4-77\\
    \Leftrightarrow 77&=y^4-x^2\\
    \Leftrightarrow 77&=(y^2+x)(y^2-x).
  \end{align*}
  Since $y^2+x\geq 0$ and $77>0$, $y^2-x$ has to be greater than zero, too.
  Since $77=7\cdot 11$, there are two possible values for the two factors.
  Firstly,
  \begin{align*}
    y^2+x = 77\\
    y^2-x = 1.
  \end{align*}
  By solving we get $x=38$ and $y^2=39$. Since $\sqrt{39}\not\in\Z$, this does not result in a valid solution.
  Secondly,
  \begin{align*}
    y^2+x&=11\\
    y^2-x&=7.
  \end{align*}
  This gives $y^2=9\Rightarrow y=3$ and $x=2$, which is the only solution where $x,y\geq 0$.
  Note that, since $x\geq 0$, $y^2+x\geq y^2-x$ and thus we do not switch the factors and solve again.\\
  We conclude that all solutions $(x,y)$ are of the form $(\pm 2, \pm 3)$.
\end{proof}
\end{document}