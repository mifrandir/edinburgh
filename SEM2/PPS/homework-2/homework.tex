\documentclass{article}
\usepackage[a4paper]{geometry}
\usepackage{babel}
\usepackage{amsmath}
\usepackage{amssymb}
\usepackage{mathtools}
\usepackage{nicefrac}
\geometry{tmargin=2cm, bmargin=3cm}
\DeclarePairedDelimiter{\floor}{\lfloor}{\rfloor}
\title{PPS: Homework 2 (Workshop 33)}
\author{Franz Miltz (UNN: s1971811)}
\newcommand*\lneg[1]{\overline{#1}}
\begin{document}
\maketitle
\section*{Task 1}
\subsection*{Claim}
Let $x,y\in\mathbb{R}$. Then
\begin{align*}
  |xy|=|x||y|
\end{align*}
\subsection*{Proof}
Note that $|x|=\sqrt{x^2}$. Thus we can write
\begin{align*}
  |xy|=\sqrt{x^2y^2}=\sqrt{x^2}\sqrt{y^2}=|x||y|.\:\square
\end{align*}
\section*{Task 2}
\subsection*{Claim} 
Let $x,y,z>0$. Then
\begin{align*}
  \frac{x^2}{y+z}+\frac{y^2}{x+z}+\frac{z^2}{x+y}\geq\frac{1}{2}(x+y+z).
\end{align*}
\subsection*{Proof}
Let $s=x+y+z$. Then we can write equivalent to the claim:
\begin{align*}
  \frac{x^2}{s-x}+\frac{y^2}{x-y}+\frac{z^2}{s-z}\geq\frac{1}{2}s.
\end{align*}
Now we can subsitute $a=\nicefrac{x}{s}$, $b=\nicefrac{y}{s}$ and $c=\nicefrac{z}{s}$ to get
\begin{align*}
  \frac{a^2}{1-a}+\frac{b^2}{1-b}+\frac{c^2}{1-c}\geq \frac{1}{2}.
\end{align*}
Subsituting in the geometric series $\frac{r^2}{1-r}=\sum_{n=2}^\infty r^n$ we get
\begin{align*}
  (a^2+a^3+a^4+\cdots)+(b^2+b^3+b^4+\cdots)+(c^2+c^3+c^4+\cdots) &\geq \frac{1}{2}\\
  \Leftrightarrow (a^2+b^2+c^2)+(a^3+b^3+c^3)+(a^4+b^4+c^4)+\cdots &\geq \frac{1}{2}
\end{align*}
As proven below, the summands $(a^n+b^n+c^n)$ are minimal, if $a=b=c=\frac{1}{3}$. Since
\begin{align*}
  3\cdot\frac{1}{9}\cdot\frac{3}{2}=\frac{1}{2}\geq \frac{1}{2}
\end{align*}
we know that our inequality is true for all possible $a$, $b$ and $c$. $\square$\\
\subsection*{Claim} Let $a,b,c>0$ such that $a+b+c=1$. Then the sum $a^n+b^n+c^n$ for any $n\in\mathbb{N}$ is minimal if $a=b=c$.
\subsection*{Proof}
By induction. Let $P(n)$ be the statement that for some $a,b,c>0$ such that $a+b+c=1$ the sum $a^n+b^n+c^n$ is minimal if $a=b=c=\frac{1}{3}$.\\
$P(1)$ is trivially satisfied because all $a^1+b^1+c^1=1$:
\begin{align*}
  \frac{1}{3}+\frac{1}{3}+\frac{1}{3}=1.
\end{align*}
Now we assume $P(n)$ and write 
\begin{align*}
  a^{n+1}+b^{n+1}+c^{n+1}=a\cdot a^n+b\cdot b^n + c\cdot c^n.
\end{align*}
Without losing generality we can also assume $a\leq b\leq c$.
This lets us use the rearrangement inequality to get the two following inequalities.
\begin{align*}
  a^{n+1}+b^{n+1}+c^{n+1}\geq ba^n+cb^n+ac^n,\\
  a^{n+1}+b^{n+1}+c^{n+1}\geq ca^n+ab^n+bc^n.
\end{align*}
Adding those to the trivial inequality
\begin{align*}
  a^{n+1}+b^{n+1}+c^{n+1}\geq a^{n+1}+b^{n+1}+c^{n+1},
\end{align*}
we get
\begin{align*}
  3(a^{n+1}+b^{n+1}+c^{n+1})\geq (a+b+c)(a^n+b^n+c^n) = a^n+b^n+c^n.
\end{align*}
By $P(n)$ we know that the right side is minimal. Setting $a=b=c=\frac{1}{3}$ we can see that the equality holds and can therefore conclude $P(n)\Rightarrow P(n+1)$.
\begin{align*}
  3\cdot\left(\frac{1}{3}\right)^{n+1}+\left(\frac{1}{3}\right)^{n+1}+\left(\frac{1}{3}\right)^{n+1}=3\cdot3\cdot\frac{1}{3^{n+1}}=3\cdot\frac{1}{3^n}=\left(\frac{1}{3}\right)^n+\left(\frac{1}{3}\right)^n+\left(\frac{1}{3}\right)^n
\end{align*}
Since $P(1)$ and $P(n)\Rightarrow P(n+1)$ we know that $P(n)$ for all $n\in\mathbb{N}$. $\square$
\section*{Task 3}
\subsection*{Claim}
For any $n\geq 1$, it is possible to tile a $2^n\times 2^n$  grid missing any one square with 'L'-shaped tiles composed of three squares.
\subsection*{Proof}
By induction. Let $P(n)$ be the statement that it is possible to tile a $2^n\times 2^n$ grid missing any one square with 'L'-shaped tiles composed of three squares.\\
It is trivial to tile a $2\times 2$ square with a single such tile. Thus $P(1)$ is true.\\
Let's now prove $P(n)\Rightarrow P(n+1)$. 
Assume $P(n)$.\\
We can then split any $2^{n+1}\times 2^{n+1}$ grid into four $2^n\times 2^n$ grids. One of them is missing exactly one square. 
Since $P(n)$, this subsquare can be completely tiled. 
Now we are left with three $2^n\times 2^n$ subsquares that are arranged in an 'L'-shape.
It is possible to tile exactly one square of each of those subsquares by placing a tile around the corner of the already tiled subsquare.
This creates three $2^{n}\times 2^{n}$ squares that are each missing exactly one square. Those can be tiled by $P(n)$.\\
Thus $P(1)$ and $P(n)\Rightarrow P(n+1)$ and therefore $P(n)$ for all $n\in\mathbb{N}$. $\square$
\end{document}