\documentclass{article}
\usepackage[a4paper]{geometry}
\geometry{tmargin=3cm, bmargin=3cm, lmargin=2cm, rmargin=2cm}
\usepackage{babel}
\usepackage{amsmath}
\usepackage{amssymb}
\usepackage{amsthm}
\usepackage{siunitx}
\newtheoremstyle{sltheorem} {}                % Space above
{}                % Space below
{\upshape}        % Theorem body font % (default is "\upshape")
{}                % Indent amount
{\bfseries}       % Theorem head font % (default is \mdseries)
{.}               % Punctuation after theorem head % default: no punctuation
{ }               % Space after theorem head
{}                % Theorem head spec
\theoremstyle{sltheorem}
\newtheorem{definition}{Definition}[section]
\newtheorem{theorem}{Theorem}[section]
\newcommand{\R}{\mathbb{R}}
\newcommand{\N}{\mathbb{N}}
\begin{document}
\section{Functions and models}
\subsection{Four ways to represent a function}
\begin{definition}
    A \textbf{function} $f$ is a rule that assigns to each element $x$ in a set $D$ exactly one element, called $f(x)$, in a set $E$.\\
    $D$ is called the \textbf{domain}.\\
    The set of all possible values of $f(x)$ as $x$ varies throughout the domain is called the \textbf{range}.\\
\end{definition}
\begin{definition}
    A function $f$ that satisfies $f(-x)=f(x)$ for all $x$ in its domain is called an \textbf{even function}.
\end{definition}
\begin{definition}
    A function $f$ that satisfies $f(-x)=-f(x)$ for all $x$ in its domain is called an \textbf{odd function}.
\end{definition}
\begin{definition}
    A function $f$ is called \textbf{increasing} on an interval $I$ if
    \begin{align*}
        \forall x_1, x_2 \in I.\:x_1 < x_2 \Rightarrow f(x_1) < f(x_2).
    \end{align*}
    It is called \textbf{decreasing} on $I$ if
    \begin{align*}
        \forall x_1, x_2 \in I.\: x_1 < x_2 \Rightarrow f(x_1) > f(x_2).
    \end{align*}
\end{definition}
\subsection{Mathematical models: Essential functions}
\begin{definition}
    A \textbf{linear function} is a function $f$ of the form
    \begin{align*}
        f(x) = mx + b
    \end{align*}
    where $m, b\in\R$.
\end{definition}
\begin{definition}
    A function $P$ is called a \textbf{polynomial} if
    \begin{align*}
        P(x) = a_nx^n+a_{n-1}x^{n-1}+\cdots +a_2x^2 + a_1x+a_0
    \end{align*}
    where $n\in\N$ is the \textbf{degree} and the $a_0, a_1, ..., a_n$ are constants called the \textbf{coefficients}.
\end{definition}
\begin{definition}
    A function $f$ is called \textbf{rational function} if
    \begin{align*}
        f(x) = \frac{P(x)}{Q(x)}
    \end{align*}
    where $P$ and $Q$ are polynomials.
\end{definition}
\begin{definition}
    A function $f$ is called an \textbf{algebraic function} if it can be constructed using algebraic operations starting with polynomials.
\end{definition}
\begin{definition}
    A function $f$ is called an \textbf{exponential function} if it has the form
    \begin{align*}
        f(x) = a^x
    \end{align*}
\end{definition}
\begin{definition}
    The \textbf{logarithmic functions} $f(x) = \log_ax$ where the base $a$ is a positive constant, are the inverse functions of the exponential functions.
\end{definition}
\subsection{New functions from old functions}
\begin{definition}
    Shifting the graph of a function in a certain direction is called \textbf{translation}.
    \begin{itemize}
        \item $y = f(x) + c$ shifts the graph a distance $c$ upward.
        \item $y= f(x - c)$ shifts the graph a distance $c$ to the right.
    \end{itemize}
\end{definition}
\begin{definition}
    The graph of a function can be \textbf{stretched}. 
    \begin{itemize}
        \item $y=cf(x)$ stretches the graph vertically by a factor of $c$.
        \item $y=f(x/c)$ stretches the graph horizontally by a factor of $c$.
    \end{itemize}
\end{definition}
\begin{definition}
    The graph of a function can be \textbf{reflected}.
    \begin{itemize}
        \item $y=-f(x)$ reflects the graph about the $x$-axis.
        \item $y=f(-x)$ reflects the graph about the $y$-axis.
    \end{itemize}
\end{definition}
\begin{definition}
    Functions may be \textbf{combined}.
    \begin{align*}
        (f+g)(x) &:= f(x) + g(x)\\
        (f-g)(x) &:= f(x) - g(x)\\
        (fg)(x) &:= f(x)g(x)\\
        (\frac{f}{g})(x)&:=\frac{f(x)}{g(x)}\\
        (f\circ g)(x) &:= f(g(x))
    \end{align*}
\end{definition}
\setcounter{subsection}{4}
\subsection{Exponential functions}
\begin{theorem}
    Let $a$ and $b$ be positive numbers and $x$ and $y$ be any real numbers. Then
    \begin{enumerate}
        \item $a^{x+y}=a^xa^y$,
        \item $a^{x-y}=\frac{a^x}{a^y}$,
        \item $(a^x)^y=a^{xy}$,
        \item $(ab)^x=a^xb^x$.
    \end{enumerate}
\end{theorem}
\subsection{Inverse functions and logarithms}
\begin{definition}
    A function $f$ with the domain $D$ is called a \textbf{one-to-one function} if it never takes on the same value twice; that is
    \begin{align*}
        \forall x_1, x_2 \in D.\: x_1\not=x_2 \Rightarrow f(x_1)\not= f(x_2)
    \end{align*}
\end{definition}
\begin{definition}
    Let $f$ be a one-to-one function with domain $A$ and range $B$. Then its \textbf{inverse function} $f^{-1}$ has domain $B$ and range $A$ and is defined by
    \begin{align*}
        f^{-1}(y) = x \Leftrightarrow f(x) = y
    \end{align*}
\end{definition}
\begin{theorem}
    The graph of $f^{-1}$ is obtained by reflecting the graph of $f$ about the line $y=x$.
\end{theorem}
\begin{theorem}
    If $x$ and $y$ are positive numbers, then
    \begin{enumerate}
        \item $\log_a(xy)=\log_ax=\log_ay$
        \item $\log_a(\frac{x}{y}) = \log_ax - \log_ay$
        \item $\log_a(x^r)=r\log_ax$ where $r\in\R$
    \end{enumerate}
\end{theorem}
\begin{definition}
    The \textbf{natural logarithm} $\ln x$ is defined as
    \begin{align*}
        \ln x := \log_e x
    \end{align*}
\end{definition}
\begin{theorem}
    For any positive number $a$ ($a\not=1$), we have
    \begin{align*}
        \log_a x = \frac{\ln x}{\ln a}
    \end{align*}
\end{theorem}
\section{Limits and derivatives}
\setcounter{subsection}{1}
\subsection{The limit of a function}
\begin{definition}
    We write
    \begin{align*}
        \lim_{x\to a} f(x) = L
    \end{align*}
    and say "the limit of $f(x)$, as $x$ approaches $a$, equals $L$" if we can make the values of $f(x)$ arbitrarily close to $L$ by taking $x$ to be sufficiently close to $a$, but not equal to $a$.
\end{definition}
\begin{definition}
    We write
    \begin{align*}
        \lim_{x\to a^-}f(x) = L
    \end{align*}
    and say the \textbf{left-hand limit of $f(x)$ as x approachse $a$} is equal to $L$ if we can make the values of $f(x)$ arbitrarily close to $L$ by taking $x$ to be sufficiently close to $a$ and $x$ less than $a$. 
\end{definition}
\begin{theorem}
    \begin{align*}
        \lim_{x\to a}f(x) = L \text{ iff } \lim_{x\to a^-}f(x) = L \text{ and } \lim_{x\to a^+}f(x) = L
    \end{align*}
\end{theorem}
\begin{definition}
    Let $f$ be a function defined on both sides of $a$, except possibly at $a$ itself. Then
    \begin{align*}
        \lim_{x\to a} f(x) = \infty
    \end{align*}
    means that the values of $f(x)$ can be made arbitrarily large by taking $x$ sufficiently close to $a$, but not equal to $a$.
\end{definition}
\begin{definition}
    Let $f$ be a function defined on both sides of $a$, except possibly at $a$ itself. Then
    \begin{align*}
        \lim_{x\to a} f(x) = -\infty
    \end{align*}
    means that the values of $f(x)$ can be made arbitrarily large negative by taking $x$ sufficiently close to $a$, but not equal to $a$.
\end{definition}
\begin{definition}
    The line $x=a$ is called a \textbf{vertical asymptote} of the curve $y=f(x)$ if at least one of the following statements is true:
    \begin{align*}
        \lim_{x\to a}f(x) = \infty\\
        \lim_{x\to a^-}f(x) = \infty\\
        \lim_{x\to a^+}f(x) = \infty\\
        \lim_{x\to a}f(x) = -\infty\\
        \lim_{x\to a^-}f(x) = -\infty\\
        \lim_{x\to a^+}f(x) = -\infty
    \end{align*}
\end{definition}
\subsection{Calculating limits and using the limit laws}
\begin{theorem}
    Suppose that $c$ is a constant and the limits $\lim_{x\to a}f(x)$ and $\lim_{x\to a}g(x)$ exists. Then
    \begin{align*}
        \lim_{x\to a}\left[f(x)+g(x)\right]
        &=\lim_{x\to a}f(x)+\lim_{x\to a}g(x)\\
        \lim_{x\to a}\left[f(x)-g(x)\right]
        &=\lim_{x\to a}f(x)-\lim_{x\to a}g(x)\\
        \lim_{x\to a}\left[cf(x)\right]
        &=c\lim_{x\to a}f(x)\\
        \lim_{x\to a}\left[f(x)g(x)\right]
        &=\lim_{x\to a}f(x)\cdot\lim_{x\to a}g(x)\\
        \lim_{x\to a}\left[\frac{f(x)}{g(x)}\right]
        &=\frac{\lim_{x\to a}f(x)}{\lim_{x\to a}g(x)}\text{ if } \lim_{x\to a}g(x)\not = 0\\
        \lim_{x\to a}\left[f(x)\right]^n
        &=\left[\lim_{x\to a}f(x)\right]^n\\
        \lim_{x\to a}c &= c\\
        \lim_{x \to a}x &= a\\
        \lim_{x \to a}x^n &= a^n &\text{where } n\in\mathbb{Z}^+
    \end{align*}
\end{theorem}
\begin{theorem}
    If $f$ is a polynomial or a rational function and $a$ is in the domain of $f$, then
    \begin{align*}
        \lim_{x\to a}f(x) = f(a)
    \end{align*}
\end{theorem}
\begin{theorem}
    If $f(x) = g(x)$ when $x\not=a$, then $\lim_{x\to a}f(x) = \lim_{x\to a}g(x)$, provided the limits exists.
\end{theorem}
\begin{theorem}
    \begin{align*}
        \lim_{x\to a} f(x) = L\text{ iff } \lim_{x\to a^-}=L=\lim_{x\to a^+} f(x)
    \end{align*}
\end{theorem}
\begin{theorem}
    If $f(x) \leq g(x)$ when $x$ is near a (except possibly at $a$) and the limits of $f$ and $g$ both exists as $x$ approaches $a$, then
    \begin{align*}
        \lim_{x\to a}f(x) \leq \lim_{x\to a} g(x)
    \end{align*}
\end{theorem}
\begin{theorem}
    If $f(x)\leq g(x)\leq h(x)$ when $x$ is near $a$ (except possibly at $a$) and
    \begin{align*}
        \lim_{x\to a}f(x) = \lim_{x\to a}h(x) = L
    \end{align*}
    then
    \begin{align*}
        \lim_{x \to a}g(x) = L.
    \end{align*}
\end{theorem}
\end{document}