\documentclass{article}
\usepackage[a4paper]{geometry}
\usepackage{babel}
\usepackage{amsmath}
\usepackage{amssymb}
\usepackage{mathtools}
\usepackage{nicefrac}
\geometry{tmargin=2cm, bmargin=3cm}
\DeclarePairedDelimiter{\floor}{\lfloor}{\rfloor}
\title{CAP: Homework 6 (Workshop 71)}
\author{Franz Miltz (UNN: S1971811)}
\DeclareMathOperator{\sech}{sech}
\DeclareMathOperator{\csch}{csch}
\DeclareMathOperator{\arccot}{\text{cot}^{-1}}
\DeclareMathOperator{\arccsc}{\text{csc}^{-1}}
\DeclareMathOperator{\arccosh}{\text{cosh}^{-1}}
\DeclareMathOperator{\arcsinh}{\text{sinh}^{-1}}
\DeclareMathOperator{\arctanh}{\text{tanh}^{-1}}
\DeclareMathOperator{\arcsech}{\text{sech}^{-1}}
\DeclareMathOperator{\arccsch}{\text{csch}^{-1}}
\DeclareMathOperator{\arccoth}{\text{coth}^{-1}} 
\DeclareMathOperator{\dom}{dom}
\DeclareMathOperator{\range}{rng}
\DeclareMathOperator{\st}{s.t.}
\begin{document}
\maketitle
\section*{Question 1}
\section*{Question 2}
\subsection*{a)}
We will first evaluate the definite integral $I$:
\begin{align*}
  I(t)=\int \sqrt{t}\ln t\: dt.
\end{align*}
Using integration by parts with $u=\ln t$ and $dv=\sqrt{t}dt$, we get
\begin{align*}
  I(t)&=\frac{2}{3}t^{\nicefrac{3}{2}}\ln t - \frac{2}{3}\int \sqrt{t}dt\\
  &=\frac{2}{3}t^{\nicefrac{3}{2}}(\ln t - \frac{2}{3}) + C.
\end{align*}
We can evaluate this at $t=1$ and $t=4$:
\begin{align*}
  I(1)&=\frac{2}{3}\cdot 1\cdot (0-\frac{2}{3})+C=-\frac{4}{9}+C,\\
  I(4)&=\frac{2}{3}\cdot 8\cdot(\ln 4 - \frac{2}{3})+C=\frac{16}{3}(\ln 4 - \frac{2}{3})+C.
\end{align*}
Thus we can find the value of the original integral:
\begin{align*}
  \int_1^4 \sqrt{2}\ln t\:dt = I(4)-I(1) =\frac{16}{3}\ln 4 - \frac{28}{4}= \frac{4}{3}(8\ln 2 - \frac{7}{3})
\end{align*}
\subsection*{b)}
At first, we can use the quality
\begin{align*}
  \tan^2 x = \sec^2 x - 1
\end{align*}
simplify the integral:
\begin{align*}
  I(x) &= \int \tan^2 x \sec x\:dx 
  = \int (\sec^2 x - 1)\sec x\:dx \\
  &= \int \sec^3 x\:dx  - \int \sec x\:dx
\end{align*}
We know 
\begin{align}
  \label{eq:1}
  \int \sec x dx = \ln |\sec x + \tan x| + C_1.
\end{align}
Let's evaluate the other integral. We use integration by parts with
\begin{align*}
  u &= \sec x, \\
  du &= \sec x \tan x\: dx,\\
  dv &= \sec^2 x\: dx, \\
  v &= \tan x
\end{align*}
to get
\begin{align*}
  I'(x)=\int \sec^3 x \: dx&=\sec x\tan x - \int \sec x \tan^2 x\: dx\\
  &= \sec x \tan x - \int \sec x (\sec^2 x - 1)dx\\
  &= \sec x \tan x - I'(x) + \int \sec x \: dx.
\end{align*}
Using equation \ref{eq:1}, we can solve for $I'$ to get
\begin{align*}
  I'(x)=\frac{1}{2}(\sec x \tan x + \ln |\sec x + \tan x|) + C_2.
\end{align*}
Thus we conclude that
\begin{align*}
  I(x)=\frac{1}{2}(\sec x \tan x -\ln |\sec x + \tan x|) + C.
\end{align*}
\subsection*{c)}
The integral
\begin{align*}
  I(x) &= \int \frac{x^3}{\sqrt{4+x^2}}dx 
\end{align*}
may be solved using the trigonometric substitution $x = 4 \tan \theta$. Using this, we get
\begin{align*}
  I(x) 
\end{align*}
\end{document}