\documentclass{article}
\usepackage[a4paper]{geometry}
\usepackage{babel}
\usepackage{amsmath}
\usepackage{amssymb}
\usepackage{mathtools}
\usepackage{nicefrac}
\geometry{tmargin=2cm, bmargin=3cm}
\DeclarePairedDelimiter{\floor}{\lfloor}{\rfloor}
\title{CAP: Homework 1 (Workshop 71)}
\author{Franz Miltz (UNN: s1971811)}
\begin{document}
\maketitle
\section*{Question 1}
\begin{align*}
  f(x) = x - \floor{x}
\end{align*}
\subsection*{a)}
You can find it on the attached sheet.
\subsection*{b)}
\begin{align*}
  \lim_{x\to n^-}f(x)=1\\
  \lim_{x\to n^+}f(x)=0
\end{align*}
Why is this the case? First off, you can see that $f$ is continuous in every interval $(n,n+1)$ for every integer $n\in Z$, because $\floor{x}$ and $x$ are both continuous in these intervals.\\
Within the interval $f(x)$ is directly proportional to $x$, because $\floor{x}$ does not change. 
This means that to the right-hand side, $f(x)$ approaches the biggest possible difference between a real number and the maximum integer less than or equal to it. 
This difference approaches $1$. Thus the limit from the left is solved.\\
The limit from the right follows the same rules. 
Since $f$ is periodic, it approaches the lowest possible difference $x-\floor{x}$.
Since $x\geq\floor{x}$ for all $x\in\mathbb{R}$ by definition, $f$ is at its minimum wherever $x=\floor{x}$.
This is the case when $x\in\mathbb{Z}$.
\subsection*{c)}
We know that $f$ is continuous everywhere except at integers. At those integers the right-hand side and the left-hand side limits differ and thus the limit itself is not defined. Therefore we know that the limit $\lim_{x\to a}f(x)$ is only defined for real numbers $a$ which are not integers:
\begin{align*}
  a\in\{x\:|\: x\in\mathbb{R}\setminus\mathbb{Z}\}
\end{align*}
were $\mathbb{Z}$ is the set of all integers and $\mathbb{R}$ is the set of all real numbers.
\section*{Question 2}
\begin{align*}
  f(x)=\begin{cases}
    \frac{x^2-4}{x-2} &\text{if }x<2\\
    ax^2-bx+3 &\text{if }2\leq x<3\\
    2x-a+b &\text{if }3\leq x
  \end{cases}
\end{align*}
The first part does not contain any $a$ or $b$. Therefore we can start by calculating
\begin{align*}
  \lim_{x\to 2^-}f(x)=4.
\end{align*}
For $f$ to be continuous at $2$ we thus need $f(2)=4$. Thus
\begin{align*}
  a\cdot 4 - b\cdot 2 + 3 = 4 \Leftrightarrow 4a-2b=1
\end{align*}
This is our first equation. Further we need to let $\lim_{x\to3^-}f(x)=f(3)$. This gives
\begin{align*}
  \lim_{x\to 3^-}f(x)&=f(3)\\ 
  \Leftrightarrow 9a-3b+3&=6-a+b\\
  \Leftrightarrow 10a-4b&=3
\end{align*}
Solving these two equations for $a$ and $b$ gives $a=b=\nicefrac{1}{2}$.\\
Let's verify our solution:
\begin{align*}
  f(x) &= \begin{cases}
    \frac{x^2-4}{x-2} &\text{if } x < 2\\
    \frac{1}{2}x^2-\frac{1}{2}x+3 &\text{if } 2\leq x < 3\\
    2x &\text{if } 3 \leq x
  \end{cases}\\
  \lim_{x\to 2^-}f(x)&=4\\
  \lim_{x\to 2^+}f(x)&=f(2)=2-1+3=4\\
  \Rightarrow \lim_{x\to 2}f(x) &= f(2) = 4\\
  \lim_{x\to 3^-}f(x)&=\frac{9}{2}-\frac{3}{2}+3=3+3=6\\
  \lim_{x\to 3^+}f(x)&=f(3)=2\cdot 3 =6\\
  \Rightarrow \lim_{x\to 3}f(x) &= f(3) = 6
\end{align*}
Thus $f$ is now continuous everywhere.
\end{document}