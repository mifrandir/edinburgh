\documentclass{article}
\usepackage[a4paper]{geometry}
\geometry{tmargin=3cm, bmargin=3cm, lmargin=2cm, rmargin=2cm}
\usepackage{babel}
\usepackage{amsmath}
\usepackage{amssymb}
\DeclareMathOperator{\sech}{sech}
\DeclareMathOperator{\csch}{csch}
\DeclareMathOperator{\arccot}{\text{cot}^{-1}}
\DeclareMathOperator{\arccsc}{\text{csc}^{-1}}
\DeclareMathOperator{\arccosh}{\text{cosh}^{-1}}
\DeclareMathOperator{\arcsinh}{\text{sinh}^{-1}}
\DeclareMathOperator{\arctanh}{\text{tanh}^{-1}}
\DeclareMathOperator{\arcsech}{\text{sech}^{-1}}
\DeclareMathOperator{\arccsch}{\text{csch}^{-1}}
\DeclareMathOperator{\arccoth}{\text{coth}^{-1}} 
\DeclareMathOperator{\dom}{dom}
\DeclareMathOperator{\st}{s.t.}
\usepackage{amsthm}
\usepackage{siunitx}
\usepackage{nicefrac}
\newtheoremstyle{sltheorem} {}                % Space above
{}                % Space below
{\upshape}        % Theorem body font % (default is "\upshape")
{}                % Indent amount
{\bfseries}       % Theorem head font % (default is \mdseries)
{.}               % Punctuation after theorem head % default: no punctuation
{ }               % Space after theorem head
{}                % Theorem head spec
\theoremstyle{sltheorem}
\newtheorem{definition}{Definition}[section]
\newtheorem{theorem}{Theorem}[section]
\newcommand{\R}{\mathbb{R}}
\newcommand{\N}{\mathbb{N}}
\begin{document}
\tableofcontents
\setcounter{section}{3}
\section{Applications of differentiation}
\setcounter{subsection}{6}
\subsection{Antiderivatives}
\begin{definition}
    A function $F$ is called an \textbf{antiderivative} of $f$ on an interval $I$ if $F'(x)=f(x)$ for all $x$ in $I$.
\end{definition}
\begin{theorem}
    If $F$ is an antiderivative of $f$ on an interval $I$, then the most general antiderivative of $f$ on $I$ is
    \begin{align*}
        F(x) + C
    \end{align*}
    where $C$ is an arbitrary constant.
\end{theorem}
\begin{theorem}
    Particular antiderivatives of common functions:
    \begin{itemize}
        \item $\int cf(x)dx = c\int f(x)dx$
        \item $\int f(x) + g(x) dx = \int f(x)dx + \int g(x)dx$
        \item $\int x^n dx=\frac{x^{n+1}}{n+1}+C$
        \item $\int \frac{1}{x}dx=\ln |x|+C$
        \item $\int e^xdx=e^x+C$
        \item $\int \cos x dx = \sin x+C$
        \item $\int \sin x dx = -\cos x +C $
        \item $\int \sec^2 x dx = \tan x + C$
        \item $\int \sec x \tan x dx = \sec x + C$
        \item $\int \frac{1}{\sqrt{1-x^2}}dx = \sin^{-1} x + C$
        \item $\int \frac{1}{1+x^2}dx = \arctan x + C$
        \item $\int \cosh x dx = \sinh x + C$
        \item $\int \sinh x dx = \cosh x + C$
    \end{itemize}
\end{theorem}
\section{Integrals}
\subsection{Areas and Distances}
\begin{definition}
  The \textbf{area} $A$ of the region $S$ that lies under the graph of the continuous function $f$ is the limit of the sum of the areas of appoximating rectangles:
  \begin{align*}
      A=\lim_{n\to\infty}R_n = \lim_{n\to \infty}\left(f(x_1)\Delta x + f(x_2)\Delta x + \cdots + f(x_n)\delta x\right)
  \end{align*}
\end{definition}
\subsection{The definite integral}
\begin{definition}
    If $f$ is a function defined on $[a,b]$, the \textbf{definite integral of $f$ from $a$ to $b$} is the number
    \begin{align*}
        \int_a^b f(x)dx=\lim_{\max \Delta x_i\to0}\sum_{i=1}^n f(x_i^*)\Delta x_i
    \end{align*}
    provided that this limit exists. If it does exist, we say that $f$ is \textbf{integrable} on $[a,b]$.
\end{definition}
\begin{theorem}
    If $f$ is continuous on $[a,b]$, or if $f$ has only a finite number of jump discontinuities, then $f$ is integrable on $[a,b]$; that is, the definite integral $\int_a^b f(x)dx$ exists.
\end{theorem}
\begin{theorem}
    If $f$ is integrable on $[a,b]$, then \begin{align*}
        \int_a^b f(x)dx = \lim_{n\to\infty}\sum_{i=1}^nf(x_i)\Delta x
    \end{align*}
    where \begin{align*}
    \Delta x = \frac{b-a}{n} \text{ and } x_i=a+i\Delta x
    \end{align*}
\end{theorem}
\begin{theorem}
    \textbf{The midpoint rule}.
    \begin{align*}
        \int_a^bf(x)dx \approx \sum_{i=1}^nf(\bar{x_i})\Delta x = \Delta x (f(\bar{x_i})+\cdots+f(\bar{x_n}))
    \end{align*}
    where
    \begin{align*}
        \Delta x = \frac{b-a}{n},\\
        \bar{x_i} = \frac{1}{2}(x_{i-1}+x_i).
    \end{align*}
\end{theorem}
\begin{theorem}
    \textbf{Properties of integrals}.\\
    Suppose all the following integrals exist and let $c$ be any constant.
    \begin{enumerate}
        \item $\int_a^b cdx = c(b-a)$
        \item $\int_a^b(f(x)+g(x))dx=\int_a^b+\int_a^bg(x)dx$
        \item $\int_a^b(cf(x))dx=c\int_a^bf(x)dx$
        \item $\int_a^b(f(x)-g(x))dx=\int_a^bf(x)dx-\int_a^bg(x)dx$
        \item $\int_a^cf(x)dx+\int_c^bf(x)dx = \int_a^bf(x)dx$
        \item If $f(x)\geq 0$ for $x\in[a,b]$, then $\int_a^b f(x)dx \geq 0$.
        \item If $f(x)\geq g(x)$ for $x\in[a,b]$, then $\int_a^b f(x)dx\geq \int_a^b g(x)dx$.
        \item If $f(x)\in[m,M]$ for $x\in[a,b]$, then 
        \begin{align*}
            m(b-a)\leq \int_a^b f(x)dx \leq M(b-a)
        \end{align*}
    \end{enumerate}
\end{theorem}
\end{document}