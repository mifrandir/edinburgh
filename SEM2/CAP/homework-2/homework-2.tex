\documentclass{article}
\usepackage[a4paper]{geometry}
\usepackage{babel}
\usepackage{amsmath}
\usepackage{amssymb}
\usepackage{mathtools}
\usepackage{nicefrac}
\geometry{tmargin=2cm, bmargin=3cm}
\DeclarePairedDelimiter{\floor}{\lfloor}{\rfloor}
\title{CAP: Homework 2 (Workshop 71)}
\author{Franz Miltz (UNN: s1971811)}
\begin{document}
\maketitle
\section*{Question 1}
\subsection*{a)}
Let $f(x)=\sqrt{3x+1}$. Then
\begin{align*}
  \frac{d}{dx}f(x)=\lim_{h\to0}\frac{f(x+h)-f(x)}{f(h)}
  =\lim_{h\to0}\frac{\sqrt{3x+3h+1}-\sqrt{3x+1}}{h}.
\end{align*}
We can get rid of the roots in the numerator by multiplying by $1$.
\begin{align*}
  \frac{d}{dx}f(x)&=\lim_{h\to0}\left(\frac{\sqrt{3x+3h+1}-\sqrt{3x+1}}{h}\cdot\frac{\sqrt{3x+3h+1}+\sqrt{3x+1}}{\sqrt{3x+3h+1}+\sqrt{3x+1}}\right)\\
  &=\lim_{h\to0}\left(\frac{3x+3h+1-3x-1}{h(\sqrt{3x+3h+1}+\sqrt{3x+1})}\right)\\
  &=\lim_{h\to0}\left(\frac{3}{\sqrt{3x+3h+1}+\sqrt{3x+1}}\right).
\end{align*}
This may now be easily simplified:
\begin{align*}
  \frac{d}{dx}f(x)&= 3\lim_{h\to0}\left(\frac{1}{\sqrt{3x+3h+1}+\sqrt{3x+1}}\right)\\
  &=3\frac{1}{\sqrt{3x+1}+\sqrt{3x+1}}=\frac{3}{2\sqrt{3x+1}}.
\end{align*}
Thus
\begin{align*}
  \frac{d}{dx}\sqrt{3x+1}=\frac{3}{2\sqrt{3x+1}}.
\end{align*}
\subsection*{b)}
Let $f(x)=\frac{1}{x^2}$. Then
\begin{align*}
  \frac{d}{dx}f(x)=\lim_{h\to0}\frac{f(x+h)-f(x)}{h}=\lim_{h\to0}\frac{\nicefrac{1}{(x+h)^2}-\nicefrac{1}{x^2}}{h}.
\end{align*}
Adjusting to a common denominator yields
\begin{align*}
  \frac{d}{dx}f(x)&=\lim_{h\to0}\frac{x^2-(x+h)^2}{hx^2(x+h)^2}
  =\lim_{h\to0}\frac{h^2-2xh}{hx^2(x+h)^2}\\
  &=\lim_{h\to0}\frac{h-2x}{x^2(x+h)^2}=-\frac{2x}{x^4}.
\end{align*}
Thus
\begin{align*}
  \frac{d}{dx}\frac{1}{x^2}=-\frac{2}{x^3}.
\end{align*}
\section*{Question 2}
\subsection*{a)}
Let $f(x)=x^{\nicefrac{1}{3}}$. Then
\begin{align*}
  f(1001)&\approx L(x)= f(1000)+f'(1000)(1001-1000)=10+f'(10^3).
\end{align*}
Using rules of differentiation, we get
\begin{align*}
  \frac{d}{dx}x^{\nicefrac{1}{3}}&=\frac{1}{3}x^{\nicefrac{-2}{3}}.
\end{align*}
Therefore
\begin{align*}
  f'(10^3)=\frac{1}{3}(10^3)^{\nicefrac{-2}{3}}=\frac{1}{3\cdot10^2}=\frac{1}{300}.
\end{align*}
Thus
\begin{align*}
  L(x)=10+\frac{1}{300}=\frac{3001}{300}\approx10.0033.
\end{align*}
According to my scientific calculator, this is within a $10^{-5}$ margin.
\subsection*{b)}
Let $f(x)=\frac{1}{x}$. Then
\begin{align*}
  f(4.002)\approx L(x)=f(4)+f'(4)(4.002-4)=\frac{1}{4}+f'(4)\cdot2\cdot10^{-3}.
\end{align*}
Using rules of differentiation, we get
\begin{align*}
  \frac{d}{dx}\frac{1}{x}=-\frac{1}{x^2}.
\end{align*}
Therefore
\begin{align*}
  f'(4)=-\frac{1}{16}.
\end{align*}
Thus
\begin{align*}
  L(x)&=\frac{1}{4}-\frac{1}{16}\cdot 2\cdot 10^{-3}=\frac{2\cdot 10^3-1}{8\cdot10^3}\\
  &=\frac{1999}{8000}\approx0.2499.
\end{align*}
According to my scientific calculator, this is within a $10^{-7}$ margin.
\end{document}