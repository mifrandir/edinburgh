\documentclass{article}
\usepackage[a4paper]{geometry}
\geometry{tmargin=3cm, bmargin=3cm, lmargin=2cm, rmargin=2cm}
\usepackage{babel}
\usepackage{amsmath}
\usepackage{amssymb}
\usepackage{amsthm}
\usepackage{siunitx}
\usepackage{hyperref}
\usepackage{units}
\usepackage{mhchem}
\newtheoremstyle{sltheorem} {}                % Space above
{}                % Space below
{\upshape}        % Theorem body font % (default is "\upshape")
{}                % Indent amount
{\bfseries}       % Theorem head font % (default is \mdseries)
{.}               % Punctuation after theorem head % default: no punctuation
{ }               % Space after theorem head
{}                % Theorem head spec
\theoremstyle{sltheorem}
\newtheorem{definition}{Definition}[section]
\newtheorem{theorem}{Theorem}[section]
\title{Astrobiology Exam Notes}
\date{\today}
\author{Franz Miltz}
\begin{document}
\maketitle
\tableofcontents
\pagebreak
\section{The Structure of matter and life}
\subsection{Types of Bonds}
\begin{itemize}
    \item \textbf{Ionic Bonding}
    \begin{itemize}
        \item \textbf{strong}, $e^-$ transfer
        \item between positively and negatively charged ions
        \item ionic compounds mostly crystaline solid
        \item lowest energy state without spare electrons
        \item e.g. in \textbf{proteins}
    \end{itemize}
    \item \textbf{Covalent Bonding}
    \begin{itemize}
        \item \textbf{strong}, $e^-$ sharing
        \item between similar or identical atoms (e.g. hydrogen, carbon)
        \item formed by sharing outer electron
        \item prominent in \textbf{extreme temperatures}
    \end{itemize}
    \item Metallic Bonding
    \begin{itemize}
        \item strong, $e^-$ sharing
        \item found in metals
        \item not relevant for life
    \end{itemize}
    \item \textbf{Van der Waals forces}
    \begin{itemize}
        \item \textbf{very weak}, dipole interaction
        \item \textbf{Dipole-dipole forces}
        \begin{itemize}
            \item charge difference across molecules length
            \item two dipoles attract each other
        \end{itemize}
        \item \textbf{Induced dipole}
        \begin{itemize}
            \item no natural dipole
            \item through induction by dipole/ion
        \end{itemize}
        \item \textbf{Dispersion Forces}
        \begin{itemize}
            \item between all atoms and molecules
            \item due to uneven distribution of electrons in atoms
        \end{itemize}
    \end{itemize}
    \item \textbf{Hydrogen Bonding}
    \begin{itemize}
        \item \textbf{weak}, dipole interaction
        \item in molecules containing OH groups
        \item attraction of electrons of other molecules
    \end{itemize}
\end{itemize}
\subsection{Types of biological molecules}
\begin{itemize}
    \item \textbf{Proteins}
    \begin{itemize}
        \item long chains of amino acids
        \item primary, secondary and teriary structures
        \item structural roles
    \end{itemize}
    \item \textbf{Sugars}
    \begin{itemize}
        \item energy storage
        \item structural support
    \end{itemize}
    \item \textbf{Lipids}
    \begin{itemize}
        \item long chains
        \item energy storage
        \item in membranes
    \end{itemize}
    \item Nucleic acids (DNA/RNA)
    \begin{itemize}
        \item information storage
        \item ribose sugar, phosphate group and base pairs 
    \end{itemize}
\end{itemize}
\subsection{Chirality}
\begin{itemize}
    \item different effects on light
    \item proteins and amino acids are left handed
    \item sugars are right handed
    \item origin unknown
\end{itemize}
\subsection{Carbon in Life}
\textbf{Characteristics}
\begin{itemize}
    \item similar bond energies to HNOPS
    \item stable links
    \item wide range of double and triple bond configurations
\end{itemize}
\textbf{Silicon as an alternative}
\begin{itemize}
    \item complex compounds possible
    \item organo-silicon compounds possible
    \item Problem: 
    \begin{itemize}
        \item \textbf{very stable, rock-like}
        \item unlikely around oxygen, oxidates easily
    \end{itemize}
\end{itemize}
\subsection{Water in Life}
Roles
\begin{itemize}
    \item medium for chemical reactions
    \item carries nutrients and energy
    \item helps maintain thermal balance
\end{itemize}
Characteristics
\begin{itemize}
    \item solves a lot of ions
    \item large heat capacity
    \item cosmically abundant
    \item wide temperature range
\end{itemize}
\textbf{Amonia as an alterantive}

\begin{itemize}
    \item solves a lot of chemicals
    \item large heat capacity
    \item less viscous 
    \item \textbf{BUT: narrower temperature range}
\end{itemize}
\textbf{Hydrogen fluoride as an alternative}
\begin{itemize}
    \item wide temperature range
    \item dissolves wide range of substances
    \item \textbf{BUT: rare in the universe; tends to destroy organic compounds}
\end{itemize}
\subsection{Cell structure}
Three basic components
\begin{itemize}
    \item membrane (increased pressure makes chemistry possible)
    \item information system
    \item energy system
\end{itemize}
\subsection{DNA transcription}
\begin{enumerate}
    \item RNA polymerase attaches to DNA sequence
    \item RNA polymerase splits double stranded DNA
    \item single DNA strand gets copied
    \item RNA polymerase dettaches upon \textbf{terminator sequence}
\end{enumerate}
\subsection{DNA translation}
\begin{itemize}
    \item tRNA brings amino acids to the ribosome
    \item three base pairs correspond to one amino acid
    \item $64$ possible codons for $20$ amino acids
    \item allows for tolerance of mutation and errors
\end{itemize}
\subsection{Chemiosmosis}
\begin{itemize}
    \item Protons are moved out of the cell by electron from electron donor.
    \item Protons flow back through enzyme to make $ATP$ from $ADP$ + $P_i$
\end{itemize}
In aerobic organisms:
\begin{itemize}
    \item dependent on the maintenance of a proton reservoir
    \item dependent on the continued movement of electrons
    \item dependent on availability of oxygen
    \item oxygen to keep electrons flowing and glucose to provide electrons
\end{itemize}
\section{Habitability}
\subsection{The environment}
\textbf{Keplers's Laws}
\begin{enumerate}
    \item The orbit of each planet is an ellipse with the Sun (or another star) at one focus.
    \item The radius vector to a planet sweeps out equal areas in equal time.
    \item The squares of the sidereal periods of the planets orbits are proportional to the cubes of the semimajor axes of their orbits. ($T^2=ka^3$)
\end{enumerate}
\textbf{Newton's Laws}
\begin{itemize}
    \item An object moves at a constant velocity if there is no net force acting upon it.
    \item $\vec F = m \vec a$
    \item For every force, there is an equal but opposite reaction force.
\end{itemize}
\subsection{Low mass stars}
General information:
\begin{itemize}
    \item $M < 2M_{Sun}$
    \item lifetime of around 10 billion years
    \item hydrogen burning around the core
    \item helium burning when $T > 100 \times 10^6$
    \item helium flash: core initially supported by electron degeneracy pressure so $T$ increases rapidly
    \item double shell burning around carbon core
    \item outer layers ejected into space
    \item planetary nebula and white dwarf
\end{itemize}
\subsection{High mass stars}
General information:
\begin{itemize}
    \item $M > 8M_{Sun}$
    \item reach the main-sequence in less than 150000 years
    \item lifetime of around 10 million years
    \item $L\approx M^4$ ($\Rightarrow$ much brighter)
    \item higher temperatures $\Rightarrow$ CNO cycle
\end{itemize}
Becoming a super giant:
\begin{itemize}
    \item hydrogen burning in shell around helium core
    \item no helium flash, thermal pressure still high
    \item helium burning for few hundred thousand years
    \item at $600 \times 10^6K$ carbon fusion starts
\end{itemize}
Advanced nuclear burning:
\begin{itemize}
    \item burns element
    \item core shrinks, temperature rises
    \item burning starts again
    \item H, He, C, O, Ne, Mg, Si, Fe
\end{itemize}
Death of a high mass star:
\begin{itemize}
    \item no fusion therefore no heat to push outwards
    \item starts collapsing
    \item electron degeneracy pressure is overcome by gravity
    \item protons and electrons combine to neutrons
    \item iron core collapses into ball of neutrons
    \item neutron degeneracy pressure
    \item collapse releases energy $\Rightarrow$ supernova
    \item core either remains as neutrons star or collapses into black hole
\end{itemize}
\subsection{Supernova Nucleosynthesis}
\begin{itemize}
    \item can create elements heavier than iron
    \item released neutrons form neutron rich nuclei
    \item unstable nuclei decay via beta decay to form heavier elements
\end{itemize}
Elements in the universe:
\begin{enumerate}
    \item big bang: H, He, Li
    \item high mass stars: $\leq$ Fe
    \item nucleosynthesis: $\geq$ C
\end{enumerate}
\subsection{Solar system formation}
\begin{itemize}
    \item interstellar nebula forms disc (due to gravity and angular momentum)
    \item center forms into sun
    \item planets form in outer regions
    \item smaller planets cannot keep atmosphere
\end{itemize}
\subsection{Snow line}
\begin{itemize}
    \item distance from planetary discs center where volatile compounds freeze
    \item frozen particles lead to more mass and therefore bigger planets
    \item \textbf{separates terrestrial planets from gas giants}
\end{itemize}
\subsection{Exoplanets}
\textbf{Doppler Technique}
\begin{itemize}
    \item messure starlight and difference in blue/red shift over time
    \item $\frac{\Delta\lambda}{\lambda}=\frac{v_r}{c}$
    \item Finding orbital distance of a planet: $a^3\approx \frac{GM_{Star}}{4\pi^2}P_{Planet}^2$ (Kepler's third law)
    \item \textbf{Problem: unknown inclination}
\end{itemize}
\textbf{Transit Method}
\begin{itemize}
    \item messure blocked starlight due to transit of planet
    \item amount blocked tells us radius of planet
    \item with period we can determine semi-major axis
\end{itemize}
\textbf{Hot Jupiters}
\begin{itemize}
    \item Jupiter mass planet
    \item have to form far away from star (snow line!)
    \item migrate inwards after formation
\end{itemize}
\textbf{Habitable Zone}
\begin{itemize}
    \item zone within which planets could support life
    \item generally zone where liquid water is possible
    \item zone moves because of change in stars properties
\end{itemize}
\textbf{Internal Heat}
\begin{itemize}
    \item origin
    \begin{itemize}
        \item heat of accretion
        \item heat from differentiation
        \item heat from radioactivity
    \end{itemize}
    \item loss
    \begin{itemize}
        \item convection (hotter material moves outwards)
        \item conduction (hot material transfers heat)
        \item radiation (radiating light into space)
        \item smaller planets have bigger surface and therefore cool faster
    \end{itemize}
    \item importance
    \begin{itemize}
        \item volcanic outgassing (for atmosphere)
        \item tectonic motion
        \item magnetic field (protects from harmful rays)
    \end{itemize}
\end{itemize}
\textbf{Atmosphere}
\begin{itemize}
    \item origin
    \begin{itemize}
        \item volcanic outgassing
        \item evaporation or sublimation of surface liquids and ices
        \item micrometeorites and high energy particles
    \end{itemize}
    \item loss
    \begin{itemize}
        \item condensation
        \item chemical reactions
        \item impacts
        \item solar wind
        \item thermal escape
    \end{itemize}
    \item importance
    \begin{itemize}
        \item increased pressure allows for liquid water
        \item breathing
        \item absorbs and scatter light
        \item winds and weather
        \item \textbf{greenhouse effect}
    \end{itemize}
\end{itemize}
\textbf{Habitability around non-solar stars}
\begin{itemize}
    \item wavelengths
    \item different habitable zone
    \item tidally locked planets
    \item no Jupiter?
    \item no Moon?
    \item circular orbits?
\end{itemize}
\section{Life}
\subsection{Definition}
Life is a self-sustained chemical system capable of undergoing Darwinian evolution.
\textbf{Characteristics}
\begin{itemize}
    \item It's complex and exhibits complex behaviours.
    \item It grows.
    \item It replicates.
    \item It metabolises.
    \item It evolves.
\end{itemize}
\subsection{Origin}
\textbf{Interstellar grains}
\begin{itemize}
    \item siliceous and carbonaceous material
    \item covered in ice
    \item size of about $1\mu m$
    \item molecules can accumulate on surface thus overcoming the proximity problem
    \item surface can be heated by solar energy thus overcoming the temperature problem
\end{itemize}
\textbf{Giant Molecular Clouds}
\begin{itemize}
    \item $T\approx (30\pm20)K$
    \item $n\approx 10^{6\pm1}cm^{-3}$
    \item Typically form thousands for low-mass stars and several high mass stars.
    \item About 100 molecules detected.
    \item Mostly $H_2$.
\end{itemize}
\textbf{Endogenous production of amino acids}
\begin{itemize}
    \item can be made from water, methane, ammonia and hydrogen
    \item hydrothermal vents create steam
    \item lightnings allow for reactions
    \item condensation and repetition
\end{itemize}
\textbf{Exogenous arrival of amino acids}
\begin{itemize}
    \item complex amino acids may be found on meteorites
    \item don't need to be formed on Earth
\end{itemize}
\textbf{RNA world}
\begin{itemize}
    \item RNA can be enzymes, called ribozymes
    \item ribozymes can catalyze RNA synthesis
    \item therefore RNA might have been the first molecule to reproduce
    \item diversity due to mutations
\end{itemize}
\textbf{Hydrothermal vents as the location for the origin of life}
\begin{itemize}
    \item today we find many microbes, worms and crabs there
    \item Ferrodoxin: ancient iron-sulfur protein
    \item iron-sulfur world hypothesis
    \begin{itemize}
        \item assumes formation of simple organic molecules from simple inorganic cases
        \item reactions on iron-nickel-sulfur mineral (as in black smoker)
        \item surface gives environment for reactions of $HCN$ to yield precursors to the nucleobases
    \end{itemize}
\end{itemize}
\subsection{Classification}
\textbf{Hierarchy}
\begin{itemize}
    \item Species
    \item Genus
    \item Family
    \item Order
    \item Class
    \item Phylum
    \item Kingdom
    \item Domain
\end{itemize}
\textbf{Binomial nomenclature}: generic name and specific name (genus and species)\\
\textbf{Phylogenetic trees}
\begin{itemize}
    \item nodes: branching point
    \item outgroup: species that's very different to cluster of other species
    \item root: last common ancestor of all the taxa
    \item sister taxa: taxa that are direct result of a node
    \item problem: analogies (different species in same circumstances evolve similarly)
\end{itemize}
\textbf{Molecular phylogentics}\\
Comptutation to distinguish homology from analogy.\\
\textbf{Testing hypotheses}\\
Inferences:
\begin{itemize}
    \item Type 1: Both ancestors have trait
    \item Type 2: One ancestor has trait
    \item Type 3: Neither ancestor has trait
\end{itemize}
Allows for statements about extinct species.\\
\textbf{Complications}
\begin{itemize}
    \item Microorganisms have encorporated others.
    \item Bacteria can take up DNA from the environment.
    \item DNA can be injected into bacteria.
    \item Microorganisms can transfer DNA.
\end{itemize}
\subsection{In Extremes}
e.g. hot environments\\
Challenges:
\begin{itemize}
    \item breakdown of molecules
    \item increased membrane fluidity
\end{itemize}
Adaptations:
\begin{itemize}
    \item thermostable proteins and enzymes
    \item modify cell compositions
\end{itemize}
\subsection{Gibbs free energy}
$\Delta G^0 =\sum G^0(\text{products}) - \sum G^0(\text{reactants})$
\section{Earth}
\subsection{Moon}
\begin{itemize}
    \item very large relative to Earth
    \item very small iron core
    \item bulk density similar to Earths mantle
    \item identical oxygen isotope composition
\end{itemize}
Giant Impact Hypothesis:
\begin{itemize}
    \item collision with Mars size object
    \item Earth material would have gone into orbit and formed Moon
    \item Moon is leaving Earth.
    \item Moon appeared a lot bigger in the beginning.
\end{itemize}
\subsection{Oceans}
\begin{itemize}
    \item vapor from inside the Earth
    \item water rich asteroids (isotopes match)
\end{itemize}
\subsection{Evidence for Late Heavy Bombardment}
\begin{itemize}
    \item Mare Orientale (giant crater on back of the Moon)
    \item basins on Mercury, Moon and Mars 
    \item If Mercury, Mars and Moon were hit ... Earth was probably hit too.
\end{itemize}
\subsection{Differentiation}
\begin{itemize}
    \item heavier elements move towards the center
    \item leads to \textbf{inner and outer core} as well as \textbf{mantle} and \textbf{crust}
    \item took about 50 million years
\end{itemize}
\subsection{Fractionation}
\begin{itemize}
    \item difference in bond strength
    \item open systems let different isotopes escape more easily
    \item lower temperatures lead to big impact of mass
\end{itemize}
\begin{align*}
    \delta\ce{^{13}C}=\left(\frac{\nicefrac{\ce{^{13}C}}{\ce{^{12}C_{\text{sample}}}}}{\nicefrac{\ce{^{13}C}}{\ce{^{12}C_{\text{std}}}}}-1\right)\times 1000\text{ ppt}
\end{align*}
\end{document}