\documentclass{article}
\usepackage[a4paper]{geometry}
\geometry{tmargin=3cm, bmargin=3cm}
\usepackage{babel}
\usepackage{amsmath}
\usepackage{amssymb}
\title{ILA: H3 (Workshop 08)}
\author{Franz Miltz (UNN: S1971811)}
\date{October 15, 2019}
\begin{document}
\maketitle
This document will use the following notation:
\begin{itemize}
    \item $a$ may represent any scalar.
    \item $\vec{a}=\begin{bmatrix}
        a_1\\
        \vdots\\
        a_n
    \end{bmatrix}$ may represent any vector.
    \item $A=\begin{bmatrix}
        a_{11} &\dots &a_{1n}\\
        \vdots &\ddots &\vdots\\
        a_{m1} &\dots &a_{mn}
    \end{bmatrix}$ may represent any matrix.
    \item $ab$ is a normal product of two scalars.
    \item $a\vec{b}$ is a scalar multiplication of the vector $\vec{b}$ with the scalar $a$
    \item $A\vec b$ is the matrix multiplication of the matrix $A$ with the vector $\vec b$
\end{itemize}
\section*{Q40}
\subsection*{a)}
We can just use $t=x_1$ to subsitute $t$ for $x_1$ in the equations for $x_2$ and $x_3$. Thus we get the following system:
\begin{align*}
    -x_1+x_2=1\\
    x_1+x_3=2 
\end{align*}
\subsection*{b)}
In the given equations we can subsitute $x_3$ for $s$ and $t$ for $x_1$.
\begin{align*}
    x_2 = 1 + x_1\\
    s = 2 - x_1
\end{align*}
Using those and appending $x_3=s$ we get the following solution:
\begin{align*}
    x_1 &= 2-s\\ 
    x_2 &= 3-s\\ 
    x_3 &= s
\end{align*}
\section*{Q20}
To see the effect of those row operations, let's define a matrix $A$ with two rows:
\begin{align}
    A = \begin{bmatrix}
        \vec a\\
        \vec b
    \end{bmatrix}
\end{align}
Here $\vec a$ and $\vec b$ are row vectors of arbitrary size with arbitrary values. 
(They might even be in one-dimensional, i.e. scalars.) Note as well, that the given operations do not influence any other row. Therefore we don't need to consider bigger matrices.\\
If we write this transformation matrix into an equation, we can apply the given row operations:
\begin{align}
    \begin{bmatrix}
        \vec a\\
        \vec b
    \end{bmatrix} \vec x &= \vec d\\
    \Leftrightarrow\begin{bmatrix}
        \vec a\\
        \vec a + \vec b
    \end{bmatrix} \vec x &= \vec d
    & (R_2 + R_1)\\
    \Leftrightarrow\begin{bmatrix}
        -\vec b\\
        \vec a + \vec b
    \end{bmatrix} \vec x &= \vec d
    & (R_1 - R_2)\\
    \Leftrightarrow\begin{bmatrix}
        -\vec b\\
        \vec a
    \end{bmatrix} \vec x &= \vec d
    & (R_2 + R_1)\\
    \Leftrightarrow\begin{bmatrix}
        \vec b\\
        \vec a
    \end{bmatrix} \vec x &= \vec d
    & (-R_1)
\end{align}
The net effect is obvious: the first two rows of the matrix are swapped.
This is therefore an important sequence of operations because it shows, that moving rows is possible without any side effects, which is necessary for finding the REF and RREF.
\section*{Q24}
There are infinetly many RREFs of $3\times3$ matrices because there are infently many possible values. Yet, we can sort them into four categories, based on how many zero rows they contain: zero, one, two or three. \\
The possiblilities for zero or three zero rows are very limited, the only possible matrices in these categories are
\begin{align*}
    \begin{bmatrix}
        0 &0 &0\\
        0 &0 &0\\
        0 &0 &0
    \end{bmatrix}\text{ and }
    \begin{bmatrix}
        1 &0 &0\\
        0 &1 &0\\
        0 &0 &1
    \end{bmatrix}.
\end{align*}
For two zero rows, we have three possiblilities for where the first entry is in the row. With $s_i$ being arbitrary scalar entries, we get the following forms:
\begin{align*}
    \begin{bmatrix}
        1 &s_1 &s_2\\
        0 &0 &0\\
        0 &0 &0
    \end{bmatrix}\:
    \begin{bmatrix}
        0 &1 &s_1\\
        0 &0 &0\\
        0 &0 &0
    \end{bmatrix}\:
    \begin{bmatrix}
        0 &0 &1\\
        0 &0 &0\\
        0 &0 &0
    \end{bmatrix}
\end{align*}
From the first two matrices we can derive three general forms that contain two non-zero rows:
\begin{align*}
    \begin{bmatrix}
        1 &s_1 &s_2\\
        0 &1 &s_3\\
        0 &0 &0
    \end{bmatrix}\:
    \begin{bmatrix}
        1 &s_1 &s_2\\
        0 &0 &1\\
        0 &0 &0
    \end{bmatrix}\:
    \begin{bmatrix}
        0 &1 &s_1\\
        0 &0 &1\\
        0 &0 &0
    \end{bmatrix}
\end{align*}
Since there is neither another way of inserting a non-zero row into the $0$ matrix nor a way of adding a non-zero row to the second line of matrices while following the definition of the RREF, we can be sure that all of the RREFs of $3\times3$ matrices fit one of those eight patterns. 
\end{document}