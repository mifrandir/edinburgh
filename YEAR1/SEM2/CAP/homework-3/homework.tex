\documentclass{article}
\usepackage[a4paper]{geometry}
\usepackage{babel}
\usepackage{amsmath}
\usepackage{amssymb}
\usepackage{mathtools}
\usepackage{nicefrac}
\geometry{tmargin=2cm, bmargin=3cm}
\DeclarePairedDelimiter{\floor}{\lfloor}{\rfloor}
\title{CAP: Homework 2 (Workshop 71)}
\author{Franz Miltz (UNN: s1971811)}
\begin{document}
\maketitle
\section*{Question 1}
\subsection*{a)}
\begin{align*}
  \lim_{t\to\infty} p(t) = \lim_{t\to\infty} \frac{1}{1+ae^{-kt}}.
\end{align*}
With
\begin{align*}
  \lim_{t\to\infty} ae^{-kt} = 0
\end{align*}
we get
\begin{align*}
  \lim_{t\to\infty}\frac{1}{1+ae^{-kt}}=\frac{1}{1+0}=1.
\end{align*}
Thus
\begin{align*}
  \lim_{t\to\infty}p(t)=1.
\end{align*}
This means that the rumour will not stop spreading but it will never reach everybody.
For any finite population this model is wrong, though.
Since people are discrete entities and therefore the function $p(t)$ would have to rise in discrete steps.
\subsection*{b)}
The rate of spread $p'(t)$ at any time $t$ is given by
\begin{align*}
  p'(t)=\frac{d}{dt}\left(\frac{1}{1+ae^{-kt}}\right)=-\frac{1}{(1+ae^{-kt})^2}\frac{d}{dt}(1+ae^{-kt})
  =\frac{kae^{-kt}}{(1+ae^{-kt})^2}.
\end{align*}
\subsection*{c)}
\begin{align*}
  p&=\frac{1}{1+ae^{-kt}}\\
  \Leftrightarrow \frac{1}{p}-1&=ae^{-kt}\\
  \Leftrightarrow \ln \frac{1-p}{p} &= \ln(a) -kt\\
  \Leftrightarrow t &= \frac{\ln(a)-\ln(\frac{1-p}{p})}{k}=\frac{1}{k}\ln\frac{ap}{1-p}
\end{align*}
Thus $t(p)=\frac{1}{k}\ln \frac{ap}{1-p}$. This function returns the time $t$ the rumour needs to spread to the proportion $p$ of the population.
\section*{Question 2}
Since $\forall x\in\mathbb{R}, f(x) > 0$ and $\lim_{x\to a}f(x)=0$, we know that there exists an open intervall $I$ such that $a\in I$ and $\forall x\in I, f(x) < 1$. This follows from the definition of a limit with $\varepsilon = 1$.\\
With
\begin{align*}
  \forall 0<a<1, \lim_{x\to\infty}a^x=0,
\end{align*}
and
\begin{align*}
  \forall a>0, \lim_{x\to 0^+}x^a=0,
\end{align*}
we can conclude that since $f(x)<1$ in some intervall $I$ around $a$ and $g(x)\to\infty$ as $x\to a$ (and thus $g(x)>0$ in some intervall around $a$),
\begin{align*}
  \lim_{x\to a}\left(f(x)\right)^{g(x)} = 0
\end{align*}

\end{document}