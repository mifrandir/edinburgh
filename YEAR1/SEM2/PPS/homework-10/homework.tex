\documentclass{article}
\usepackage[a4paper]{geometry}
\usepackage{babel}
\usepackage{amsmath}
\usepackage{amssymb}
\usepackage{mathtools}
\usepackage{nicefrac}
\usepackage{amsthm}
\usepackage{changepage}
\geometry{tmargin=2cm, bmargin=3cm}
\DeclarePairedDelimiter{\floor}{\lfloor}{\rfloor}
\title{PPS: Homework 10 (Workshop 33)}
\author{Franz Miltz (UNN: S1971811)}
\newcommand*\lneg[1]{\overline{#1}}
\newcommand{\R}{\mathbb{R}}
\newcommand{\N}{\mathbb{N}}
\newcommand{\Z}{\mathbb{Z}}
\newcommand{\C}{\mathbb{C}}
\newcommand{\st}{\text{ s.t. }}
%newenvironment{claim}[1]{\noindent\emph{Claim.}\space#1}{}
\newenvironment{claimproof}[1]{\par\noindent\emph{Proof.}\space#1}{\hfill $\blacksquare$}
\newtheorem{claim}[section]{Claim}
\newtheorem{lemma}{Lemma}[section]
\DeclareMathOperator{\lub}{\text{LUB}}
\DeclareMathOperator{\hcf}{hcf}
\DeclareMathOperator{\lcm}{lcm}
\setcounter{MaxMatrixCols}{20}
\newcommand*\binco[2]{\begin{pmatrix}
  #1\\#2
\end{pmatrix}}
\begin{document}
\maketitle
\section*{Problem 1}
\begin{claim}
  Let $f$ and $g$ be permutations such that
  \begin{align*}
    \begin{aligned}
    f = \begin{pmatrix}
      1&2&3&4&5&6&7\\
      3&1&5&7&2&6&4
    \end{pmatrix},
    \end{aligned}
    \hspace{1cm}
    \begin{aligned}
    g = \begin{pmatrix}
      1&2&3&4&5&6&7\\
      3&1&7&6&4&5&2
    \end{pmatrix}.
    \end{aligned}
  \end{align*}
  Then $f^{-100}g^{146}f^{301}=g^2f$ where
  \begin{align*}
    g^2f=\begin{pmatrix}
      1&2&3&4&5&6&7\\
      2&5&1&7&3&6&4
    \end{pmatrix}.
  \end{align*}
\end{claim}
\begin{proof}
  By using \emph{Lemma 7.6} in the notes we can apply modular arithmetic to simplify the problem. 
  We see that, since the order of $f$ and thus $f^{-1}$ is $4$, $f^{-100} = f^{0} = \iota$.
  Similarly we get that $g^{146}=g^2$ and $f^{301}=f$ since $g$ has order $6$. Therefore
  \begin{align*}
    f^{-100}g^{146}f^{301}=g^2f.
  \end{align*}
  With $g=(1\:3\:7\:2)(4\:6\:5)$, we can find $g^2=(1\:7)(2\:3)(4\:5\:6)$. Now we can work out $g^2f$ in matrix notation to get
  \begin{align*}
    g^2f=\begin{pmatrix}
      1&2&3&4&5&6&7\\
      2&5&1&7&3&6&4
    \end{pmatrix}.
  \end{align*}
\end{proof}
\section*{Problem 2}
\begin{claim}
  Let $f$ be a permutation such that
  \begin{align*}
    f=\begin{pmatrix}
      1&2&3&4&5&6&7&8&9&10&11&12&13&14&15&16\\
      1&9&2&10&3&11&4&12&5&13&6&14&7&15&8&16
    \end{pmatrix}.
  \end{align*}
  Then there exists no $k\in\N$ such that $f^k$ only swaps two consecutive numbers.
\end{claim}
\begin{proof}
  By exhaustive search. Since
  \begin{align*}
    f = (2\:9\:5\:3)(4\:10\:13\:7)(6\:11)(12\:14\:15\:8)
  \end{align*}
  we can use \emph{Lemma 7.7} in the notes to find that $f$ has order $4$.
  It is therefore enough to list all the permutations $f^2$ to $f^4$ to prove that the desired criterion is never met.
  \begin{align*}
    f^2&=(2\:5)(3\:9)(4\:13)(7\:10)(12\:15)(8\:14),\\
    f^3&=(2\:3\:5\:9)(4\:7\:13\:10)(6\:11)(12\:8\:15\:14),\\
    f^4&=\iota.
  \end{align*}
  Since none of these permutations is a single cycle with two consecutive numbers, it is impossible to achieve this with any number of repetitions.
\end{proof}
\section*{Problem 3}
\begin{claim}
  Let $f$ and $g$ be permutations. Then $h=fgf^{-1}$ and $g$ have the same order.
\end{claim}
\begin{proof}
  Let $m\in\N$. Assume $h^m=\iota$. Then
  \begin{align*}
    h^m = (fgf^{-1})^m= f g^m f^{-1}&=\iota\\
    \Leftrightarrow f^{-1}f g^m f^{-1}f&=f^{-1}f\\
    \Leftrightarrow g^m&=\iota
  \end{align*}
  Now assume $g^m=\iota$. Then
  \begin{align*}
    h^m=f g^m f^{-1}=ff^{-1}=\iota.
  \end{align*}
  Therefore $h^m=\iota$ if, and only if, $g^m=\iota$ for all $m\in\N$.
  Thus the smallest $m$ to satisfy this is the order of $g$ and $h$.
\end{proof}
\end{document}